% Copyright 2007 by Till Tantau and Mark Wibrow
%
% This file may be distributed and/or modified
%
% 1. under the LaTeX Project Public License and/or
% 2. under the GNU Free Documentation License.
%
% See the file doc/generic/pgf/licenses/LICENSE for more details.

\section{Shape Library}
\label{section-libs-shapes}

In addition to the standard shapes |rectangle|, |circle| and
|coordinate|, there exist a number of additional shapes defined in
different shape libraries. 	Many of these shapes have been 
contributed by Mark Wibrow. In the present section, these shapes are
described.


\begin{pgflibrary}{shapes}
  This library packages just includes all of the libraries defined in
  the following. Note that it includes only those libraries starting
  with |shapes.|, more special-purpose libraries are described in
  dedicated sections.
\end{pgflibrary}

\subsection{Rotating shape borders} \label{section-rotating-shape-borders}

	Some shapes (but not all), support a special kind of rotation. This 
	rotation affects only the border of a shape and is independent of the 
	node contents, but \emph{in addition} to any other transformations.
	
\begin{codeexample}[]
\tikzstyle{every node}=[dart, shape border uses incircle, 
  inner sep=1pt, draw]
\begin{tikzpicture}
  \foreach \a/\b/\c in {A/0/0, B/45/0, C/0/45, D/45/45}
    \node [shape border rotate=\b, rotate=\c] at (\b/36,-\c/36) {\a};
\end{tikzpicture}
\end{codeexample}

	There are two types of rotation: restricted and unrestricted. Which 
	type of rotation is applied is determined by on how the shape border 
	is	constructed. If the shape border is contructed using an incircle, 
	that is, a circle that tightly fits the node contents (including 
	the |inner sep|), then the rotation can be unrestricted. If, however,
	the border is constructed using the natural dimensions of the node
	contents, the rotation is restricted to integer multiples of 90 
	degrees.
	
	Why should there be two kinds of rotation and border construction?
	Borders constructed using the natural dimensions of the node contents
	provide a much tighter fit to the node contents, but to maintain 
	this tight fit, the border rotation must be restricted to integer 
	multiples of 90 degrees. 
	By using an incircle, unrestricted rotation is possible, but the 
	border will not make a very tight fit to the node contents.
	
\begin{codeexample}[]
\tikzstyle{every node}=[isosceles triangle, draw]
\begin{tikzpicture}
  \node {abc};
  \node [shape border uses incircle] at (2,0) {abc};
\end{tikzpicture}
\end{codeexample}

	There are \pgfname{} keys determine how a shape border is 
	contructed, and to specify its rotation.
	It should be noted that not all shapes support these keys, so 
	reference should be made to the documentation for individual 
	shapes. 
	
	The \pgfname{} keys are shown below. To use these keys in \tikzname,
	simply remove the \declare{|/pgf/|} path.

\begin{key}{/pgf/shape border uses incircle=\meta{boolean} (initially false)}
   Determines if the border of a shape is constructed using the 
   incircle. If no value is given \meta{boolean} will take the default
   value |true|.
\end{key}


\begin{key}{/pgf/shape border rotate=\meta{angle} (initially 0)}
   Rotates the border of a shape independently of the node contents,
   but in addition to any other transformations. If the shape 
   border is not constructed using the incircle, the rotation will be
   rounded to the nearest integer multiple of 90 degrees when the
   shape is drawn. 
\end{key}

	It should also be noted that if the border of the shape is rotated, 
	the compass point anchors, and `text box' anchors (including 
	|mid east|, |base west|, and so on), \emph{do not rotate}, but the 
	other anchors do:
	
\begin{codeexample}[]
\tikzstyle{every node}=[shape=trapezium, draw, shape border uses incircle]
\begin{tikzpicture}
  \node at (0,0)  (A) {A};
  \node [shape border rotate=30] at (1.5,0) (B) {B};
  \foreach \s/\t in 
    {left side/base east, bottom side/north, bottom left corner/base}{
       \fill[red]  (A.\s) circle(1.5pt) (B.\s) circle(1.5pt);
       \fill[blue] (A.\t) circle(1.5pt) (B.\t) circle(1.5pt);
  }
\end{tikzpicture}
\end{codeexample}


\subsection{Predefined Shapes}
\label{section-predefined-shapes}

The three shapes |rectangle|, |circle|, and |coordiante| are always
defined and no library needs to be loaded for them. while the
|coordinate| shape defines only the |center| anchor, the other two
shapes define a standard set of anchors.

\begin{shape}{circle}
  This shape draws a tightly fitting circle around the text. The
  following figure shows the anchors this shape defines; the anchors
  |10| and |130| are example of border anchors. 
\begin{codeexample}[]
\Huge
\begin{tikzpicture}
  \node[name=s,shape=circle,shape example] {Circle\vrule width 1pt height 2cm};
  \foreach \anchor/\placement in
    {north west/above left, north/above, north east/above right, 
     west/left, center/above, east/right, 
     mid west/right, mid/above, mid east/left, 
     base west/left, base/below, base east/right, 
     south west/below left, south/below, south east/below right, 
     text/left, 10/right, 130/above}
     \draw[shift=(s.\anchor)] plot[mark=x] coordinates{(0,0)}
       node[\placement] {\scriptsize\texttt{(s.\anchor)}};
\end{tikzpicture}
\end{codeexample}
\end{shape}

\begin{shape}{rectangle}
  This shape, which is the standard, is a rectangle around the
  text. The inner   and outer separations (see
  Section~\ref{section-shape-seps}) influence the white space around
  the text. The following figure shows the anchors this
  shape defines; the anchors |10| and |130| are example of border anchors.
\begin{codeexample}[]
\Huge
\begin{tikzpicture}
  \node[name=s,shape=rectangle,shape example] {Rectangle\vrule width 1pt height 2cm};
  \foreach \anchor/\placement in
    {north west/above left, north/above, north east/above right, 
     west/left, center/above, east/right, 
     mid west/right, mid/above, mid east/left, 
     base west/left, base/below, base east/right, 
     south west/below left, south/below, south east/below right, 
     text/left, 10/right, 130/above}
     \draw[shift=(s.\anchor)] plot[mark=x] coordinates{(0,0)}
       node[\placement] {\scriptsize\texttt{(s.\anchor)}};
\end{tikzpicture}
\end{codeexample}
\end{shape}


\subsection{Geometric Shapes}

\begin{pgflibrary}{shapes.geometric}
  This library defines different shapes that correspond to basic
  geometric objects like ellipses or polygons.
\end{pgflibrary}



\begin{shape}{diamond}
  This shape is a diamond tightly fitting the text box. The ratio
  between width and height is 1 by default, but can be changed by
  setting the shape aspect ratio using the following \pgfname{}
  keys (to use these keys in \tikzname{} simply remove the 
  \declare{|/pgf/|} path). 

 	\begin{key}{/pgf/shape aspect=\meta{value} (initially 1.0)}
 		The aspect is a recommendation for the quotient of the width and 
 		the height of a shape. This key will simultaneously set the 
 		key \declare{|/pgf/shape aspect inverse|} to the reciprocal of 
 		\meta{value}. 
 	\end{key}
 	
 	\begin{key}{/pgf/shape aspect inverse=\meta{value} (initially 1.0)}
 		The apsect is a recommendation for the quotient of the height and 
 		the width of a shape. This key will simultaneously set the 
 		key \declare{|/pgf/shape aspect|} to the reciprocal of 
 		\meta{value}. 
 	\end{key}

	The following figure shows the anchors this
  shape defines; the anchors |10| and |130| are example of border
  anchors.
  
\begin{codeexample}[]
\Huge
\begin{tikzpicture}
  \node[name=s,shape=diamond,shape example] {Diamond\vrule width 1pt height 2cm};
  \foreach \anchor/\placement in
    {north west/above left, north/above, north east/above right, 
     west/left, center/above, east/right, 
     mid/above, 
     base/below,  
     south west/below left, south/below, south east/below right, 
     text/left, 10/right, 130/above}
     \draw[shift=(s.\anchor)] plot[mark=x] coordinates{(0,0)}
       node[\placement] {\scriptsize\texttt{(s.\anchor)}};
\end{tikzpicture}
\end{codeexample}
\end{shape}

\begin{shape}{ellipse}
  This shape is an ellipse tightly fitting the text box, if no inner
  separation is given. The following figure shows the anchors this
  shape defines; the anchors |10| and |130| are example of border anchors.
\begin{codeexample}[]
\Huge
\begin{tikzpicture}
  \node[name=s,shape=ellipse,shape example] {Ellipse\vrule width 1pt height 2cm};
  \foreach \anchor/\placement in
    {north west/above left, north/above, north east/above right, 
     west/left, center/above, east/right, 
     mid west/right, mid/above, mid east/left, 
     base west/left, base/below, base east/right, 
     south west/below left, south/below, south east/below right, 
     text/left, 10/right, 130/above}
     \draw[shift=(s.\anchor)] plot[mark=x] coordinates{(0,0)}
       node[\placement] {\scriptsize\texttt{(s.\anchor)}};
\end{tikzpicture}
\end{codeexample}
\end{shape}





\begin{shape}{trapezium}
	This shape is a trapezium, that is, a quadrilateral with a single
	pair of parallel lines (this can sometimes be known as a trapezoid).
	The trapezium shape supports the rotation of the shape border, as 
	described in Section~\ref{section-rotating-shape-borders}. 
   
  The lower internal angles at the lower corners of the trapezium can 
  be specified independently, and the resulting extensions are in 
  addition to the natural dimensions of the node contents.
  When specifying any minimum size requirements, it should be 
  remembered that, in order to maintain the internal angles,
  increasing the height will increase the width and vice versa. 

	
\begin{codeexample}[]
\begin{tikzpicture}
   \tikzstyle{every node}=[trapezium, draw]
   \node at (0,2) {A};
   \node[trapezium left angle=75, trapezium right angle=45pt]
         at (0,1) {B};
   \node[trapezium left angle=120, trapezium right angle=60pt]
         at (0,0) {C};
\end{tikzpicture}
\end{codeexample}

         
	The \pgfname{} keys to set the lower internal angles of the trapezium 
	are shown	below. 
	To use these keys in \tikzname, simply remove the \declare{|/pgf/|} path.
	
	\begin{key}{/pgf/trapezium left angle=\meta{angle} (initially 60)}
    Set the internal angle of the left side. 
   \end{key}
   
   \begin{key}{/pgf/trapezium right angle=\meta{angle} (initially 60)}
    Set the internal angle of the right side. 
   \end{key}
   
   \begin{key}{/pgf/trapezium angle=\meta{angle}}
    This key stores no value itself, but sets the value of the
    previous two keys to \meta{angle}. 
   \end{key}
   
   The anchors for the trapezium are shown below. The anchor |160| is an
	example of a border anchor.

\begin{codeexample}[]
\Huge
\begin{tikzpicture}
  \node[name=s, shape=trapezium, shape example, inner sep=1cm] 
    {Trapezium\vrule width 1pt height 2cm};
  \foreach \anchor/\placement in
    {bottom left corner/below, top right corner/right, 
     top left corner/left,     bottom right corner/below,
     bottom side/below,        left side/left, 
     right side/right,         top side/above,
     center/above,   text/below,      mid/right,       base/below, 
     mid west/right, base west/below, mid east/left,   base east/below, 
     west/above,     east/above,      north/below,     south/above,
     north west/above, north east/above, 
     south west/below, south east/below, 160/above}    
  \draw[shift=(s.\anchor)] plot[mark=x] coordinates{(0,0)}
    node[\placement] {\scriptsize\texttt{(s.\anchor)}};
\end{tikzpicture}
\end{codeexample}  
   
\end{shape}





\begin{shape}{semicircle}
	
	This shape is a semicircle, which tightly fits the node contents.
	This shape supports the rotation of the shape border, as described in 
	Section~\ref{section-rotating-shape-borders}.
	The anchors for the |semicircle| shape are shown below. 
	Anchor |30| is an example of a border anchor.
	
\begin{codeexample}[]
\Huge
\begin{tikzpicture}
  \node[name=s,shape=semicircle,shape border rotate=0,shape example, inner sep=1cm] 
  	{Semicircle\vrule width 1pt height 2cm};
  \foreach \anchor/\placement in
    {apex/above,      arc start/below, arc end/below,  chord center/below,
     center/above,    base/below,      mid/right,      text/left,
     base west/below, base east/below, mid west/left, mid east/right, 
     north/below,     south/above,     east/above,     west/above,
     north west/above left, north east/above right,
     south west/below,      south east/below, 30/right}
     \draw[shift=(s.\anchor)] plot[mark=x] coordinates{(0,0)}
       node[\placement] {\scriptsize\texttt{(s.\anchor)}};
\end{tikzpicture}
\end{codeexample}
\end{shape}





\begin{shape}{regular polygon}
  This shape is a regular polygon, which, by default, is drawn so that 
  a side (rather than a corner) is always at the bottom. 
  This shape supports the rotation as described in 
  Section~\ref{section-rotating-shape-borders}, but the border of the 
  polygon is \emph{always} constructed using the incircle, whose
  radius is calculated to tightly fit the node contents (including
  any |inner sep|).
  
\begin{codeexample}[]
\begin{tikzpicture}
  \foreach \a in {3,...,7}{
    \draw[red, dashed] (\a*2,0)  circle(0.5cm);
    \node[regular polygon, regular polygon sides=\a, draw,
     inner sep=0.3535cm] at (\a*2,0) {};
   }  
\end{tikzpicture}
\end{codeexample}	
	
  If the node is enlarged to any specified minimum size, 
  this is interpreted as the diameter of the the 
  circumcircle, that is, the circle that passes through all the 
  corners of the polygon border.

\begin{codeexample}[]
\begin{tikzpicture}
  \foreach \a in {3,...,7}{
    \draw[blue, dashed] (\a*2,0)  circle(0.5cm);
    \node[regular polygon, regular polygon sides=\a, minimum size=1cm, draw] at (\a*2,0) {};
   }  
\end{tikzpicture}
\end{codeexample}	

  There is a \pgfname{} key to set the number of sides for the regular
  polygon.
  To use this key in \tikzname, simply remove the \declare{|/pgf/|} path.
	
  \begin{key}{/pgf/regular polygon sides=\meta{integer} (initially 5)}
  \end{key}
  
  The anchors for a regular polygon shape are shown below.  
  The anchor |75| is an example of a border anchor.
  
\begin{codeexample}[]
\Huge
\begin{tikzpicture}
  \node[name=s, shape=regular polygon, shape example, inner sep=.5cm] 
    {Regular Polygon\vrule width 1pt height 2cm};
  \foreach \anchor/\placement in
    {corner 1/above, corner 2/above, corner 3/left, corner 4/right, corner 5/above, 
     side 1/above,   side 2/left,    side 3/below,  side 4/right,   side 5/above,  
     center/above, text/left,  mid/right,   base/below, 75/above,
     west/above,   east/above, north/below, south/above,
     north east/below, south east/above, north west/below, south west/above}
  \draw[shift=(s.\anchor)] plot[mark=x] coordinates{(0,0)}
    node[\placement] {\scriptsize\texttt{(s.\anchor)}};
\end{tikzpicture}
\end{codeexample}

\end{shape}

\begin{shape}{star}
  This shape is a star, which by default (minus any transformations) is
  drawn with the first point pointing upwards.  
  This shape supports the rotation as described in 
  Section~\ref{section-rotating-shape-borders}, but the border of the 
  star is \emph{always} constructed using the incircle.
  
  A star should be thought of as having an set of ``inner points'' and
  and ``outer points''. 
  The inner points of the border are based on the radius of the circle
  which tightly fits the node contents, and the outer points are based
  on the circumcircle, the circle that passes through every outer
  point.
  Any specified minimum size, width or height, is interpreted as the 
  diameter of the circumcircle.
 
\begin{codeexample}[]
\begin{tikzpicture}
   \draw [help lines]   (0,0) grid (2,2);
   \draw [blue, dashed]  (1,1) circle(1cm);
   \draw [red, dashed] (1,1) circle(.5cm);
   \node [star, star point height=.5cm, minimum size=2cm, draw] 
       at (1,1) {S};
\end{tikzpicture}
\end{codeexample} 
  
  The \pgfname{} keys to set the number of star points, and the height
  of the star points, are shown below. To use these keys in \tikzname,
  simply remove the \declare{|/pgf/|} path.
  
  \begin{key}{/pgf/star points=\meta{integer} (initially 5)}
    Sets the number of points for the star.
  \end{key}
  
  \begin{key}{/pgf/star point height=\meta{distance} (initially .5cm)}
    Sets the height of the star points. This is the distance between the
    inner point and outer point radii. If the star is enlarged to some
    specified minimum size, the inner radius is increased to maintain
    the point height.	
  \end{key}
  
  \begin{key}{/pgf/star point ratio=\meta{number} (initially 1.5)}
    Sets the ratio between the inner point and outer point radii.		
    If the star is enlarged to some specified minimum size, the
    inner radius is increased to maintain the ratio.	
  \end{key}

	The inner and outer points form the principle anchors for the star,
   as shown below (anchor |75| is an example of a border anchor).
  
  \begin{codeexample}[]
\Huge
\begin{tikzpicture}
  \node[name=s, shape=star, star points=5, star point ratio=1.65, shape example, inner sep=1.5cm] 
    {Star\vrule width 1pt height 2cm};
  \foreach \anchor/\placement in
     {inner point 1/above, inner point 2/above, inner point 3/below, inner point 4/right, 
      inner point 5/above, outer point 1/above, outer point 2/above, outer point 3/left,  
      outer point 4/right, outer point 5/above,
      center/above, text/left,  mid/right,   base/below, 75/above,
     	west/above,   east/above, north/below, south/above,
     	north east/below, south east/above, north west/below, south west/above}
  \draw[shift=(s.\anchor)] plot[mark=x] coordinates{(0,0)}
    node[\placement] {\scriptsize\texttt{(s.\anchor)}};
\end{tikzpicture}
\end{codeexample}
\end{shape}





\begin{shape}{isosceles triangle}
	This shape is an isosceles triangle, which supports the rotation of 
	the shape border, as described in 
	Section~\ref{section-rotating-shape-borders}. The angle of rotation
	determines the direction in which the apex of the triangle points
	(provided no other transformations are applied).
	
	Minimum size and minimum width requirements ensure the literal width 
	and height of the isosceles triangle, but are applied as if the 
	triangle is rotated to 90 degrees (i.e., with the apex pointing up). 
	In order to keep the apex angle the same, increasing the height will 
	increase the width and vice versa. 
   
\begin{codeexample}[]
\begin{tikzpicture}
   \tikzstyle{every node}=[isosceles triangle, draw, inner sep=0pt, 
      anchor=left corner, shape border rotate=90]
   \draw[help lines] grid(4,2);
   \foreach \a/\c in {1.5/blue, 1/green, 0.5/red}{
      \color{\c}
      \node[minimum height=\a cm] at (0,0) {};
      \node[minimum width=\a cm] at (2,0) {};
   }
\end{tikzpicture}
\end{codeexample}	

	There is a \pgfname{} key to set the apex angle of the
	triangle. 
	To use this key in \tikzname, simply remove the \declare{|/pgf/|} 
	path.
    
  \begin{key}{/pgf/isosceles triangle apex angle=\meta{angle} (initially 45)}
    Sets the angle of the apex of the isosceles triangle. The height
    and width of the triangle may be adjusted to maintain this
    angle.
  \end{key}
  
   The anchors for the |isosceles triangle| are shown below (the border 
	has been rotated 90 degrees anticlockwise). Anchor |150| is an
	example of a border anchor. Note that the |center| anchor does not
	correspond to any kind of geometric center.
	
\begin{codeexample}[]
\Huge
\begin{tikzpicture}
  \node[name=s, shape=isosceles triangle, shape example, inner xsep=1cm]
    {Isosceles Triangle\vrule width 1pt height 2cm};
  \foreach \anchor/\placement in
    {apex/above,      left corner/right, right corner/right,
     left side/above, right side/below,  lower side/right,    
     center/above,    text/right,        150/above,
     mid/right,       mid west/above,    mid east/right,
     base/below,      base west/below,   base east/below,
     west/above, east/below, north/below, south/above,
     north west/below, north east/below, 
     south west/above, south east/above}  
  \draw[shift=(s.\anchor)] plot[mark=x] coordinates{(0,0)}
    node[\placement] {\scriptsize\texttt{(s.\anchor)}};
\end{tikzpicture}
\end{codeexample} 
\end{shape}


\par\leavevmode
\begin{shape}{kite}

	This shape is a kite, which supports the rotation of the shape border, 
	as described in Section~\ref{section-rotating-shape-borders}. 
	There are \pgfname{} keys to specify the upper and lower vertex angles
	of the kite. 
	To use these keys in \tikzname, simply remove the \declare{|/pgf/|} 
	path.
	
	\begin{key}{/pgf/kite upper vertex angle=\meta{angle} (initially 120)}
	Set the upper internal angle of the kite.
	\end{key}
	
	\begin{key}{/pgf/kite lower vertex angle=\meta{angle} (initially 60)}
	Set the lower internal angle of the kite.
	\end{key}
	
	\begin{key}{/pgf/kite vertex angles=\meta{angle specification}}
		This key sets the keys for both the upper and lower vertex angles
		(it stores no value itself).
	   \meta{angle specification} can be pair of angles in the form
	   \meta{upper angle} |and| \meta{lower angle}, or a single angle.
	   In this latter case, both the upper and lower vertex angles will 
	   be the same.
	\end{key}%
    
\begin{codeexample}[]
\begin{tikzpicture}
  \tikzstyle{every node}=[kite, draw]
  \node[kite upper vertex angle=135, kite lower vertex angle=70] at (0,0) {A};
  \node[kite vertex angles=90 and 45] at (1,0) {B};
  \node[kite vertex angles=60]        at (2,0) {C};
\end{tikzpicture}
\end{codeexample}


	The anchors for the |kite| are shown below. Anchor |110| is an 
	example of a border anchor.
	
\begin{codeexample}[]
\Huge
\begin{tikzpicture}
  \node[name=s, shape=kite, shape example, inner sep=1.5cm] 
    {Kite\vrule width 1pt height 2cm};
  \foreach \anchor/\placement in
    {upper vertex/above, left vertex/above,    lower vertex/below, 
     right vertex/above, upper left side/above, upper right side/above,
     lower left side/below, lower right side/below,
     center/above,   text/left,       mid/right,        base/below, 
     mid west/left,  base west/below, mid east/right,   base east/below,
     west/above,     east/above,      north/below,     south/above,
     north west/left, north east/right, 
     south west/above, south east/above, 110/above}  
  \draw[shift=(s.\anchor)] plot[mark=x] coordinates{(0,0)}
    node[\placement] {\scriptsize\texttt{(s.\anchor)}};
\end{tikzpicture}
\end{codeexample}
\end{shape}


\begin{shape}{dart}


	This shape is a dart (which can also be known as an arrowhead or
	concave kite). This shape supports the rotation of the shape border, 
	as described in Section~\ref{section-rotating-shape-borders}. 
	The angle of the border rotation determines the direction in which 
	the dart points (unless other transformations have been applied).
	
	There are \pgfname{} keys to set the 
	angle for the `tip' of the dart and the angle between the `tails'
	of the dart. 
	To use these keys in \tikzname, simply remove the \declare{|/pgf/|} 
	path.

\begin{codeexample}[]
\begin{tikzpicture}
   \node[dart, draw, gray, shape border uses incircle, shape border rotate=45] 
       (d) {dart};
   \draw [<->] (d.tip)++(202.5:.5cm) arc(202.5:247.5:.5cm);
   \node [left of=.5cm] at (d.tip) {tip angle};
   \draw [<->] (d.tail center)++(157.5:.5cm) arc(157.5:292.5:.5cm);
   \node [right] at (d.tail center) {tail angle};
\end{tikzpicture}
\end{codeexample}

	\begin{key}{/pgf/dart tip angle=\meta{angle} (initially 45)}
		Set the angle at the tip of the dart.
	\end{key}
	
	\begin{key}{/pgf/dart tail angle=\meta{angle} (initially 135)}
		Set the angle between the tails of the dart.
	\end{key}
		
	The anchors for the |dart| shape are shown below (note that the 
	shape is rotated 90 degrees anti-clockwise). Anchor |110| is an 
	example of a border anchor.
\begin{codeexample}[]
\Huge
\begin{tikzpicture}
  \node[name=s, shape=dart, shape border rotate=90, shape example, inner sep=1.25cm] 
    {Dart\vrule width 1pt height 2cm};
  \foreach \anchor/\placement in
    {tip/above,       tail center/below, right tail/below, 
     left tail/below, right tail/below,  left side/left,   right side/right,
     center/above,    text/left,         mid/right,        base/below, 
     mid west/left,   base west/below,   mid east/right,   base east/below,
     west/above,      east/above,        north/below,      south/above,
     north west/left, north east/right,  south west/above, south east/above,
     110/above}    
  \draw[shift=(s.\anchor)] plot[mark=x] coordinates{(0,0)}
    node[\placement] {\scriptsize\texttt{(s.\anchor)}};
\end{tikzpicture}
\end{codeexample}
\end{shape}




\begin{shape}{circular sector}

	This shape is a circular sector (which can also be known as a
	wedge).
	This shape supports the rotation of the shape border, 
	as described in Section~\ref{section-rotating-shape-borders}. 
	The angle of the border rotation determines the direction in which 
	the `apex' of the sector points (unless other transformations have 
	been applied).
	
\begin{codeexample}[]
\begin{tikzpicture}
	\tikzstyle{every node}=[circular sector, shape border uses incircle, draw];
   \node at (0,0) {A};
   \node [shape border rotate=30] at (1.5,0) {A};
\end{tikzpicture}
\end{codeexample}

	There is a \pgfname{} key to set the central angle of the sector, 
	which is expected to be less than 180 degrees. 
	To use this key in \tikzname,	simply remove the \declare{|/pgf/|} 
	path.
	
	\begin{key}{/pgf/circular sector angle=\meta{angle} (initially 60)}
		Set the central angle of the sector. 
	\end{key}
	
	The anchors for the circular sector shape are shown below.
	Anchor |30| is an example of a border anchor.
	
\begin{codeexample}[]
\Huge
\begin{tikzpicture}
  \node[name=s,shape=circular sector,  style=shape example, inner sep=1cm] 
  	{Circular Sector\vrule width 1pt height 2cm};
  \foreach \anchor/\placement in
   {sector center/above, arc start/below, arc end/below, arc center/below,
    center/above,        base/below,      mid/right,     text/below,
    north/below,         south/above,     east/below,    west/above,
    north west/above left, north east/above right,
    south west/below,      south east/below, 30/right}
     \draw[shift=(s.\anchor)] plot[mark=x] coordinates{(0,0)}
       node[\placement] {\scriptsize\texttt{(s.\anchor)}};
\end{tikzpicture}
\end{codeexample}
\end{shape}




\begin{shape}{cylinder}
	This shape is a 2-dimensional representation of a cylinder, which 
	supports the rotation of the shape border as described in
	Section~\ref{section-rotating-shape-borders}.

\begin{codeexample}[]
\begin{tikzpicture}
  \node[cylinder, draw, shape aspect=.5] {ABC};
\end{tikzpicture}
\end{codeexample}
		
	Regardless the rotation of the shape border, the height is always the
	distance between the curved ends, and the width is always the	
	distance between the straight sides. 

\begin{codeexample}[]
\begin{tikzpicture}[>=stealth]
  \node [cylinder, gray!50, rotate=30, draw, 
    minimum height=2cm, minimum width=1cm] (c) {Cylinder};
  \draw[red, <->] (c.top)   -- (c.bottom) 
    node [at end, below, black]   {height};
  \draw[red, <->] (c.north) -- (c.south) 
    node [at start, above, black] {width};
\end{tikzpicture}
\end{codeexample}

	Enlarging the shape to some minimum height will stretch only the body
	of the cylinder. By contrast, enlarging the shape to some minimum 
	width will stretch the curved ends.
	
\begin{codeexample}[]
\begin{tikzpicture}[>=stealth, shape aspect=.5]
  \tikzset{every node/.style={cylinder, shape border rotate=90, draw}}
  \node [minimum height=1.5cm]            {A};  
  \node [minimum width=1.5cm]  at (1.5,0) {B};  
\end{tikzpicture}
\end{codeexample}

  There are various keys to customize this shape (to use \pgfname{}
  keys in \tikzname{}, simply remove the \declare{|/pgf/|} path).
  
\begin{key}{/pgf/shape aspect=\meta{value} (initially 1.0)}
  The aspect is a recommendation for the quotient of the radii of
  the cylinder end. This may be ignored if the shape is enlarged
  to some minimum width.

\begin{codeexample}[]
\begin{tikzpicture}[>=stealth]
  \tikzset{every node/.style={cylinder, shape border rotate=90, draw}}
  \node [shape aspect=1.0]           {A};  
  \node [shape aspect=0.5]  at (1,0) {B};  
  \node [shape aspect=0.25] at (2,0) {C};  
\end{tikzpicture}
\end{codeexample}

\end{key}

\begin{key}{/pgf/cylinder uses custom fill=\meta{boolean} (default true)}
	This enables the use of a custom fill for the body and the end of 
	the cylinder. The background path for the shape should not be 
	filled (e.g., in \tikzname{}, the |fill| option for the node must 
	be implicity or explicitly set to |none|).
  Internally, this key sets the \TeX-if 
  |\ifpgfcylinderusescustomfill| appropriately.
\end{key}

\begin{codeexample}[]
\begin{tikzpicture}[>=stealth, shape aspect=0.5]
  \node [cylinder, cylinder uses custom fill, cylinder end fill=red!50,
         cylinder body fill=red!25] {Cylinder};  
\end{tikzpicture}
\end{codeexample}

\begin{key}{/pgf/cylinder end fill=\meta{color} (initially white)}
	Set the color for the end of the cylinder.
\end{key}
\begin{key}{/pgf/cylinder body fill=\meta{color} (initially white)}
	Set the color for the body of the cylinder.
\end{key}


  The anchors this shape are shown below (anchor |160| is an
	example of a border anchor). Note the the cylinder shape does not 
	distinguish between |outer xsep| and |outer ysep|. Only the larger 
	of the two values is used for the shape. Note also the difference 
	between the |center| and |shape center| anchors: |center| is the
	center of the cylinder body and also the center of rotation. 
	The |shape center| is the center of the shape which includes the 
	2-dimensional representation of the cylinder top.	
	 

\begin{codeexample}[]
\Huge
\begin{tikzpicture}
  \node[name=s, shape=cylinder, shape example, shape aspect=.5, inner xsep=3cm,
        inner ysep=1cm] {Cylinder\vrule width 1pt height 2cm};
  \foreach \anchor/\placement in
    {before top/above,    top/above,       after top/below,
     before bottom/below, bottom/above,    after bottom/above,
     mid/right,           mid west/right,  mid east/left,  
     base/below,          base west/below, base east/below,
     center/above,        text/above,      shape center/right, 
     west/right, east/left, north/above, south/below,
     north west/below, north east/above, 
     south west/above, south east/below, 160/above}    
  \draw[shift=(s.\anchor)] plot[mark=x] coordinates{(0,0)}
    node[\placement] {\scriptsize\texttt{(s.\anchor)}};
\end{tikzpicture}
\end{codeexample}  


\end{shape}






\subsection{Symbol Shapes}

\begin{pgflibrary}{shapes.symbols}
  This library defines shapes that can be used for drawing symbols
  like a forbidden sign or a cloud.
\end{pgflibrary}



\begin{shape}{forbidden sign}
  This shape places the node inside a circle with a diagonal from the
  lower left to the upper right added. The circle is part of the
  background, the diagonal line part of the foreground path; thus, the
  diagonal line is on top of the text.
  
\begin{codeexample}[]
\begin{tikzpicture}
  \node [forbidden sign,line width=1ex,draw=red,fill=white] {Smoking};
\end{tikzpicture}
\end{codeexample}

  The shape inherits all anchors from the |circle| shape, see also the
  following figure:
\begin{codeexample}[]
\Huge
\begin{tikzpicture}
  \node[name=s,shape=forbidden sign,shape example] {Forbidden\vrule width 1pt height 2cm};
  \foreach \anchor/\placement in
    {north west/above left, north/above, north east/above right, 
     west/left, center/above, east/right, 
     mid west/right, mid/above, mid east/left, 
     base west/left, base/below, base east/right, 
     south west/below left, south/below, south east/below right, 
     text/left, 10/right, 130/above}
     \draw[shift=(s.\anchor)] plot[mark=x] coordinates{(0,0)}
       node[\placement] {\scriptsize\texttt{(s.\anchor)}};
\end{tikzpicture}
\end{codeexample}
\end{shape}


\begin{shape}{cloud}

	This shape is a cloud, drawn to tightly fit the node contents 
	(strictly speaking, using an ellipse which tightly fits the node
	contents -- including any |inner sep|). 
	
\begin{codeexample}[]
\begin{tikzpicture}
  \node[cloud, draw, fill=gray!20] {ABC};
  \node[cloud, draw, fill=gray!20] at (1.5,0) {D};
\end{tikzpicture}
\end{codeexample}

	A cloud should be thought of as having a number of ``puffs'', which
	are the individual arcs drawn around the border. There are \pgfname{}
	keys to specify how the cloud is drawn (to use these keys in 
	\tikzname{}, simply remove the \declare{|/pgf/|} path).
	
	\begin{key}{/pgf/cloud puffs=\meta{integer} (initially 10)}
	  Set the number of puffs for the cloud.
	\end{key}
	
	\begin{key}{/pgf/cloud puff arc=\meta{angle} (initially 135)}
	  Set the length of the puff arc (in degrees). A shorter arc can 
	  produce better looking joins between puffs for larger line widths.
	\end{key}
	
	Due to the complexity of the cloud shape, it is 
	(unfortunately) not possible to guarantee that any minimum size 
	requirements will be met exactly, so please remember that 
	\pgfname{} can only treat them as recommended sizes.
	
  The anchors for the cloud shape are shown below for a cloud with
  eleven puffs. Anchor 70 is an example of a border anchor. 
  
\begin{codeexample}[]
\Huge
\begin{tikzpicture}
  \node[name=s, shape=cloud, style=shape example, cloud puffs=11, 
       cloud puff arc=120, cloud aspect=2] {Cloud\vrule width 1pt height 2cm};
  \foreach \anchor/\placement in
   {puff 1/above, puff 2/above,  puff 3/above,  puff 4/below, 
    puff 5/left,  puff 6/below,  puff 7/below,  puff 8/right,
    puff 9/below, puff 10/above, puff 11/above, 70/right,
    center/above, base/below,    mid/right,     text/left, 
    north/below,  south/below,   east/above,    west/above,
    north west/left,             north east/right, 
    south west/below,            south east/below}
     \draw[shift=(s.\anchor)] plot[mark=x] coordinates{(0,0)}
       node[\placement] {\scriptsize\texttt{(s.\anchor)}};
\end{tikzpicture}
\end{codeexample}
\end{shape} 






\begin{shape}{starburst}

	This shape is a randomly generated eliptical star,
	which supports the rotating of the shape border as described in 
	Section~\ref{section-rotating-shape-borders}. 
\begin{codeexample}[]
\begin{tikzpicture}
  \node[starburst, fill=yellow, draw=red, line width=2pt] {\bf BANG!};
\end{tikzpicture}
\end{codeexample}	
	Like the |star| shape, the starburst should be thought of as having a set
	of inner points and outer points. The inner points lie on the ellipse
	which tightly fits the node contents (including any |inner sep|).
	
	Using a specified `starburst point height' value, the outer points
	are generated randomly between this value and one quarter of this 
	value. For a given starburst shape the angle between each point is 
	fixed, and is determined by the number of points specified for
	the starburst.
	
	It is important to note that, whilst the maximum possible point 
	height is used to calculate minimum width or height requirements, 
	the outer points are randomly generated, so there is (unfortunately) 
	no guarantee that any such requirements will be fully met. 
	
\begin{codeexample}[]
\begin{tikzpicture}
  \draw[help lines] grid(3,2);
  \node[starburst, draw, minimum width=3cm, minimum height=2cm] 
    at (1.5, 1) {\bf BOOM!};
\end{tikzpicture}
\end{codeexample}

	There are \pgfname{} keys to control the drawing of the starburst
	shape. To use these keys in \tikzname,	simply remove the 
	\declare{|/pgf/|}	path.

	\begin{key}{/pgf/starburst points=\meta{integer} (initially 17)}
		Set the number of points for the starburst.
	\end{key}
	\begin{key}{/pgf/starburst point height=\meta{length} (initially .5cm)}
      Set the \emph{maximum} distance between the inner point radius  
      and the outer point radius.
	\end{key}
	
	\begin{key}{/pgf/random starburst=\meta{integer} (initially 100)}
      Set the seed for the random number generator for creating the
      starburst.  The maximum value for \meta{integer} is |16383|.
      If \meta{integer}|=0|, the random number generator will not be 
      used, and the maximum point height will be used for all outer 
      points. If \meta{integer} is omitted, a seed will be randomly
      chosen.
	\end{key}
	
	The basic anchors for a nine point |starburst| shape are shown below. 
	Anchor |80| is an example of a border anchor.
\begin{codeexample}[]
\Huge
\begin{tikzpicture}
  \node[name=s, shape=starburst, starburst points=9, starburst point height=3.5cm, 
        style=shape example,inner sep=1cm] 
    {Starburst\vrule width 1pt height 2cm};
  \foreach \anchor/\placement in
    {outer point 1/above, outer point 2/above, outer point 3/right,
     outer point 4/above, outer point 5/below, outer point 6/above,
     outer point 7/left,  outer point 8/above, outer point 9/above,
     inner point 1/below, inner point 2/above, inner point 3/left,
     inner point 4/above, inner point 5/above, inner point 6/above,
     inner point 7/below, inner point 8/above, inner point 9/below,
     center/above, text/left,   mid/right, base/below, 80/above,
     north/below,  south/below, east/left, west/right,
     north east/below, south west/below, south east/below, north west/below}
  \draw[shift=(s.\anchor)] plot[mark=x] coordinates{(0,0)}
    node[\placement] {\scriptsize\texttt{(s.\anchor)}};
\end{tikzpicture}
\end{codeexample}
\end{shape}

\begin{shape}{signal}

	This shape is a ``signal'' or sign shape, that is, a rectangle, with
	optionally pointed sides. A signal can point ``to'' somewhere, with 
	outward points in that direction. It can also be ``from'' 
	somewhere, with inward points from that direction. The resulting 
	points extend the node contents (which include the |inner sep|).
	
\begin{codeexample}[]
\begin{tikzpicture}[every node/.style={signal, draw,  text=white}]
  \node[fill=green!65!black, signal to=east] at (0,1) {To East};
  \node[fill=red!65!black, signal from=east] at (0,0) {From East};
\end{tikzpicture}
\end{codeexample}

	There are \pgfname{} keys for drawing the signal shape (to use these
	keys in \tikzname{}, simply remove the \declare{|/pgf/|} path):
	
	\begin{key}{/pgf/signal pointer angle=\meta{angle} (initially 90)} 
		Set the angle for the pointed sides of the shape. This angle is
		maintained when enforcing any minimum size requirements, so
		any adjustment to the width will affect the height, and vice versa.
	\end{key}
	
	\begin{key}{/pgf/signal from=\meta{direction}\space\opt{and \meta{opposite direction}} (initially nowhere)} 
		Set which sides take an inward pointer (i.e., point towards the
		center of the shape). The possible values for \meta{direction} and 
		\meta{opposide direction} are as follows:
		
		\begin{itemize}
			\item \declare{|north|} or \declare{|above|}
				adds a pointer to the upper side of the signal.
			\item \declare{|south|} or \declare{|below|}
				adds a pointer to the lower side of the signal.
			\item \declare{|east|} or \declare{|right|}
				adds a pointer to the right side of the signal.
			\item \declare{|west|} or \declare{|left|}
				adds a pointer to the left side of the signal.
			\item \declare{|nowhere|} 
				resets the sides so they have no pointers. When used with 
				|signal from| key, this only resets inward pointers;
				used with the |signal to| key, it only resets outward
				pointers. 
		\end{itemize}
	\end{key}
	
	\begin{key}{/pgf/signal to=\meta{direction}\space\opt{and \meta{opposite direction}} (initially east)} 
		Set which sides take an outward pointer (i.e., point away from the
		the shape). The possible values for \meta{direction} and 
		\meta{opposide direction} are as for the |signal from| key.
	\end{key}
	
	Note that \pgfname{} will ignore any instruction to use directions
	that are not opposites (so using the value |east and north|, will
	result in only |north| being assigned a pointer). This is also 
	the case if non-opposite values are used in the |signal to| and
	|signal from| keys at the same time. So, for example, it is not 
	possible for a signal to have an outward point to the left, and also
	have an inward point from below.
	
	The anchors for the signal shape are shown below. Anchor |70| is an
	example of a border anchor.
	
\begin{codeexample}[]
\Huge
\begin{tikzpicture}
  \node[name=s, shape=signal, signal from=west, shape example, inner sep=2cm] 
    {Signal\vrule width1pt height2cm};
  \foreach \anchor/\placement in
    {text/left,   center/above,    70/above,
     base/below,  base east/below, base west/below,
     mid/right,   mid east/above left,  mid west/above left, 
     north/above,      south/below, 
     east/above,       west/above,        
     north west/above, north east/above, 
     south west/below, south east/below}
     \draw[shift=(s.\anchor)] plot[mark=x] coordinates{(0,0)}
       node[\placement] {\scriptsize\texttt{(s.\anchor)}};
\end{tikzpicture}
\end{codeexample}

\end{shape}





\begin{shape}{tape}
	This shape is a rectangle with optional, ``bendy'' top and bottom
	sides, which tightly fits the node contents (including the 
	|inner sep|).
	
\begin{codeexample}[]
\begin{tikzpicture}
  \node[tape, draw]{ABCD};
  \node[tape, draw, tape bend top=none] at (1.5, 0) {EFGH};
\end{tikzpicture}
\end{codeexample}

  There are \pgfname{} keys to specify which sides bend and how high
  the bends are (to use these keys in \tikzname{}, simply remove the
  \declare{|/pgf/|} path):
  
  \begin{key}{/pgf/tape bend top=\meta{bend style} (initially in and out)}
  	Specify how the top side bends. The \meta{bend style} is either
  	|in and out|, |out and in| or |none| (i.e., a straight line). 
  	The bending sides are drawn in a 
  	clockwise direction, and using the bend style |in and out| will mean 
  	the side will first	bend inwards and then bend outwards. 
  	The opposite holds true for	|out and in|. 
  	
\begin{codeexample}[]
\begin{tikzpicture}[-stealth]
  \node[tape, draw, gray, minimum width=2cm](t){Tape};
  \draw [blue]([yshift=5pt] t.north west) -- ([yshift=5pt]t.north east) 
         node[midway, above, black]{in and out};
  \draw [blue]([yshift=-5pt]t.south east) -- ([yshift=-5pt]t.south west) 
         node[sloped, allow upside down, midway, above, black]{in and out};
\end{tikzpicture}
\end{codeexample}  

    This might take a bit of getting used to, but just remember that 
    when you want the bendy sides to be parallel, the sides take the 
    same bend style. It is possible for the top and bottom sides to 
    take opposite bend styles, but the author of this shape cannot 
    think of a single use for such a combination.
    
\begin{codeexample}[]
\begin{tikzpicture}
  \tikzstyle{every node}=[tape, draw]
  \node [tape bend top=out and in, tape bend bottom=out and in] {Parallel};
  \node at (2,0) [tape bend bottom=out and in]                  {Why?};
\end{tikzpicture}
\end{codeexample} 

	\end{key}
	
	\begin{key}{/pgf/tape bend bottom=\meta{bend style} (initially in and out)}
		Specify how the bottom side bends.
	\end{key}%
	
	\begin{key}{/pgf/tape bend height=\meta{length} (initially 5pt)}
		Set the total height for a side with a bend.
		
\begin{codeexample}[]
\begin{tikzpicture}[>=stealth]
  \draw [help lines] grid(3,2);
  \node [tape, fill, minimum size=2cm, red!50, tape bend top=none,
         tape bend height=1cm] at (1.5,1.5) (t) {};
  \draw [|<->|, blue] (1.5,0) -- (1.5,1) 
         node [at end, above, black]{tape bend height};
\end{tikzpicture}
\end{codeexample} 
 
	\end{key}
	
	The anchors for the tape shape are shown below. Anchor |60| is an
	example of a border anchor.
	
\begin{codeexample}[]
\Huge
\begin{tikzpicture}
  \node[name=s, shape=tape, tape bend height=1cm, shape example, inner xsep=3cm] 
    {Tape\vrule width1pt height2cm};
   \foreach \anchor/\placement in
    {text/left,  center/above,    60/above, 
     base/below, base east/below, base west/below,
     mid/right,  mid east/left,   mid west/right,  
     north/above, south/below,  east/above, west/above,        
     north west/above, north east/above, 
     south west/below, south east/below}
     \draw[shift=(s.\anchor)] plot[mark=x] coordinates{(0,0)}
       node[\placement] {\scriptsize\texttt{(s.\anchor)}};
\end{tikzpicture}
\end{codeexample}

\end{shape}%



\subsection{Arrow Shapes}

\begin{pgflibrary}{shapes.arrows}
  This library defines shapes such as arrows and callouts, that can 
  be used to point at things.
  
\end{pgflibrary}

\begin{shape}{single arrow}
	This shape is an arrow, which tightly fits the note contents 
	(including any |inner sep|). 
	This shape supports the rotation of the shape border, as 
	described in Section~\ref{section-rotating-shape-borders}. 
	The angle of rotation determines which direction the arrow
	points (provided no other rotational transformations are applied).
	
\begin{codeexample}[]
\begin{tikzpicture}[every node/.style={single arrow, draw},
    rotate border/.style={shape border uses incircle, shape border rotate=#1}]
  \node {right};
  \node at (2,0) [shape border rotate=90]{up};
  \node at (1,1) [rotate border=37, inner sep=0pt]{$37^\circ$};
\end{tikzpicture}
\end{codeexample}

	Regardless of the rotation of the arrow border, the width is 
  measured between the back ends of the arrow head, and the 
  height is measured from the arrow tip to the end of the arrow 
  tail.

\begin{codeexample}[]
\begin{tikzpicture}[>=stealth, 
    rotate border/.style={shape border uses incircle, shape border rotate=#1}]
  \node[rotate border=-30, fill=gray!25, minimum height=3cm, single arrow, 
    single arrow head extend=.5cm, single arrow head indent=.25cm] (arrow) {};
  \draw[red, <->] (arrow.before tip) -- (arrow.after tip)
    node [near end, left, black] {width};
  \draw[red, <->] (arrow.tip) -- (arrow.tail)
    node [near end, below left, black] {height};
\end{tikzpicture}
\end{codeexample}

	There are \pgfname{} keys that can be used to customize this shape (to
	use these keys in \tikzname{}, simply remove the \declare{|/pgf/|}
	path).
	
\begin{key}{/pgf/single arrow tip angle=\meta{angle} (initially 45)}
  Set the angle for the arrow tip. Enlarging the arrow to some
  minimum width may increase the the height of the shape to maintain
  this angle.
\end{key}

\begin{key}{/pgf/single arrow head extend=\meta{length} (initially 1.5ex)}
  This sets the distance between the tail of the arrow and the outer
  end of the arrow head. This may change if the shape is enlarged to
  some minimum width.
  
\begin{codeexample}[]
\begin{tikzpicture}
  \node[single arrow, draw, single arrow head extend=.5cm, gray!50, rotate=60] 
     (a) {Arrow};
  \draw[red, |<->|] (a.before tip) -- (a.before head) 
    node [midway, below, sloped, black] {head extend};
\end{tikzpicture}
\end{codeexample}
\end{key}

\begin{key}{/pgf/single arrow head indent=\meta{length} (initially 0pt)}
  This moves the point where the arrow head joins the shaft of the
  arrow \emph{towards} the arrow tip, by \meta{length}.
  
\begin{codeexample}[]
\begin{tikzpicture}[every node/.style={single arrow, draw=none, rotate=60}]
  \node [fill=red!50]                                           {arrow 1};
  \node [fill=blue!50, single arrow head indent=1ex] at (1.5,0) {arrow 2};
\end{tikzpicture}
\end{codeexample}
\end{key}

  The anchors for this shape are shown below (anchor |20| is an 
  example of a border anchor).
  
\begin{codeexample}[]
\Huge
\begin{tikzpicture}
  \node[name=s,shape=single arrow, shape example, single arrow head extend=1.5cm] 
    {Single Arrow\vrule width1pt height2cm};
  \foreach \anchor/\placement in
    {text/above,      center/above, 20/above,
     mid west/left,   mid/above,    mid east/above left,
     base west/below, base/below,   base east/below,
     tip/above, before tip/above, after tip/below, before head/above, 
     after head/below, after tail/above, before tail/below, tail/right,   
     north/above, south/below, east/below, west/above,
     north west/above, north east/below, south west/below, south east/above}    
     \draw[shift=(s.\anchor)] plot[mark=x] coordinates{(0,0)}
       node[\placement] {\scriptsize\texttt{(s.\anchor)}};
\end{tikzpicture}
\end{codeexample}

\end{shape}





\begin{shape}{double arrow}
  This shape is a double arrow, which tightly fits the note contents 
	(including any |inner sep|), and supports the rotation of the shape
	 border, as described in Section~\ref{section-rotating-shape-borders}. 
	 
	
\begin{codeexample}[]
\begin{tikzpicture}[every node/.style={double arrow, draw}]
  \node [double arrow, draw] {Left or Right};
\end{tikzpicture}
\end{codeexample}

  The double arrow behaves exactly like the single arrow, so you
  need to remember that the width is \emph{always} the distance
  between the back ends of the arrow heads, and the height
  is \emph{always} the the tip-to-tip distance.
  
\begin{codeexample}[]
\begin{tikzpicture}[>=stealth, 
    rotate border/.style={shape border uses incircle, shape border rotate=#1}]
  \node[rotate border=210, fill=gray!25, minimum height=3cm, double arrow, 
    double arrow head extend=.5cm, double arrow head indent=.25cm] (arrow) {};
  \draw[red, <->] (arrow.before tip 1) -- (arrow.after tip 1)
    node [near start, right, black] {width};
  \draw[red, <->] (arrow.tip 1) -- (arrow.tip 2)
    node [near end, above left, black] {height};
\end{tikzpicture}
\end{codeexample}

  The \pgfname{} keys that can be used to customize the double arrow 
  behave similarly to the keys for the single arrow (to
	use these keys in \tikzname{}, simply remove the \declare{|/pgf/|}
	path).
  
\begin{key}{/pgf/double arrow tip angle=\meta{angle} (initially 45)}
  Set the angle for the arrow tip. Enlarging the arrow to some
  minimum width may increase the the height of the shape to maintain
  this angle.
\end{key}

\begin{key}{/pgf/double arrow head extend=\meta{length} (initially 1.5ex)}
  This sets the distance between the shaft of the arrow and the outer
  end of the arrow heads. This may change if the shape is enlarged to
  some minimum width.
\end{key}

\begin{key}{/pgf/double arrow head indent=\meta{length} (initially 0pt)}
  This moves the point where the arrow heads join the shaft of the
  arrow \emph{towards} the arrow tips, by \meta{length}.
  \begin{codeexample}[]
\begin{tikzpicture}[every node/.style={double arrow, draw=none, rotate=-60}]
  \node [fill=red!50]                                           {arrow 1};
  \node [fill=blue!50, double arrow head indent=1ex] at (1.5,0) {arrow 2};
\end{tikzpicture}
\end{codeexample}
\end{key}


  The anchors for this shape are shown below (anchor |20| is an 
  example of a border anchor).
  
  
\begin{codeexample}[]
\Huge
\begin{tikzpicture}
  \node[name=s,shape=double arrow, double arrow head extend=1.5cm, shape example, inner xsep=2cm] 
    {Double Arrow\vrule width1pt height2cm};
  \foreach \anchor/\placement in
    {text/above, center/above, 20/above,
     mid west/above right, mid/above, mid east/above left,
     base west/below, base/below, base east/below,
     before head 1/above, before tip 1/above, tip 1/above, after tip 1/below, after head 1/below,
     before head 2/above, before tip 2/below, tip 2/above, after tip 2/above, after head 2/below,
     north/above, south/below, east/below, west/below,
     north west/below, north east/below, south west/above, south east/above}    
     \draw[shift=(s.\anchor)] plot[mark=x] coordinates{(0,0)}
       node[\placement] {\scriptsize\texttt{(s.\anchor)}};
\end{tikzpicture}
\end{codeexample}
\end{shape}





\subsection{Shapes with Multiple Text Parts}

\begin{pgflibrary}{shapes.multipart}
  This library defines general-purpose shapes that are composed of
  multiple (text) parts. 
\end{pgflibrary}


\begin{shape}{circle split}
  This shape is a multi-part shape consisting of a circle with a line
  in the middle. The upper part is the main part (the |text| part),
  the lower part is the |lower| part.
  
\begin{codeexample}[]
\begin{tikzpicture}
  \node [circle split,draw,double,fill=red!20]
  {
    $q_1$
    \nodepart{lower}
    $00$
  };
\end{tikzpicture}
\end{codeexample}

  The shape inherits all anchors from the |circle| shape and defines
  the |lower| anchor in addition. See also the
  following figure:
\begin{codeexample}[]
\Huge
\begin{tikzpicture}
  \node[name=s,shape=circle split,shape example] {text\nodepart{lower}lower};
  \foreach \anchor/\placement in
    {north west/above left, north/above, north east/above right, 
     west/left, center/below, east/right, 
     mid west/right, mid/above, mid east/left, 
     base west/left, base/below, base east/right, 
     south west/below left, south/below, south east/below right, 
     text/left, lower/left, 130/above}
     \draw[shift=(s.\anchor)] plot[mark=x] coordinates{(0,0)}
       node[\placement] {\scriptsize\texttt{(s.\anchor)}};
\end{tikzpicture}
\end{codeexample}
\end{shape}


\begin{shape}{rectangle split}
  This shape is a rectangle which can optionally be split into 
  two, three or four node parts, or even used with a single node 
  part. 

\begin{codeexample}[]
\begin{tikzpicture}[every text node part/.style={text centered}]
  \node[rectangle split, rectangle split parts=3, draw, text width=2.75cm] 
    {Student
     \nodepart{second}
       age:int \\
       name:String
     \nodepart{third}
       getAge():int \\
       getName():String};
\end{tikzpicture}
\end{codeexample} 

  
  The contents of node parts which are not used are ignored. 
  Which node parts are used in each case is shown below:

\begin{codeexample}[]
\begin{tikzpicture}
  \foreach \a/\x/\y in {1/0/0, 2/1.5/0, 3/0/-1.5, 4/1.5/-1.5}
  \node[rectangle split, rectangle split parts=\a, draw, anchor=north] 
    at (\x,\y){
      text
    \nodepart{second}
      second
    \nodepart{third}
      third
    \nodepart{fourth}
      fourth};
\end{tikzpicture}
\end{codeexample} 

  There are several \pgfname{} keys to specify how the shape is
  drawn. To use these keys in \tikzname, simply remove the 
  \declare{|/pgf/|} path:
  
  \begin{key}{/pgf/rectangle split parts=\meta{number} (initially 4)}
    Split the rectangle into \meta{number} parts, 
    which should be in the range |1| to |4|.
  \end{key}
  
  \begin{key}{/pgf/rectangle split empty part height=\meta{length} (initially 1ex)}
    Set the default height for a node part box if it is empty.
  \end{key}
  
  \begin{key}{/pgf/rectangle split part align={\ttfamily\char`\{}\meta{list}{\ttfamily\char`\}} (initially center)}
  	Set the alignment of the boxes inside the node parts.
  	There should be a maximum of four entries in \meta{list}, 
  	separated by commas (so if there is more than one entry in 
  	\meta{list} it must be surrounded by braces).
  	Each entry is one of |left|,
    |right|, or |center|. If \meta{list} has less entries than 
    node parts then the remaining node parts are aligned according to 
    the last entry in the list.    
    Note that this only aligns the boxes in each part and \emph{does not} 
    affect the alignment of the contents of the boxes.
    
\begin{codeexample}[]
\def\v#1{\vrule width#1ex height2ex}
\def\x{\v2 \nodepart{second} \v5 \nodepart{third} \v2 \nodepart{fourth} \v2}
\begin{tikzpicture}[every node/.style={rectangle split, draw, text=blue!40}]
  \node[rectangle split part align={center, left, right}] at (0,0)    {\x};
  \node[rectangle split part align={center, left}]        at (1.25,0) {\x};
  \node[rectangle split part align={center}]              at (2.5,0)  {\x};
\end{tikzpicture}
\end{codeexample}
  \end{key}
   
  \begin{key}{/pgf/rectangle split draw splits=\meta{boolean} (initially true)}
  	Set whether the line or lines between node parts will be drawn.
  	Internally, this sets the \TeX-if |\ifpgfrectanglesplitdrawsplits| 
  	appropriately.
  \end{key}
  
  \begin{key}{/pgf/rectangle split use custom fill=\meta{boolean} (initially false)}
    This enables the use of a custom fill for each of the node
    parts (including the area covered by the |inner sep|). The 
    background path for the shape should not be filled (e.g., in
    \tikzname{}, the |fill|
    option for the node must be implicity or explicitly set to |none|).
    Internally, this key sets the \TeX-if 
    |\ifpgfrectanglesplitusecustomfill| appropriately.
  \end{key}
  
  \begin{key}{/pgf/rectangle split part fill={\ttfamily\char`\{}\meta{list}{\ttfamily\char`\}} (initially white)}
  	Set the custom fill color for each node part shape. 
  	There should be a maximum of four entries in \meta{list} (one 
  	for each node part), separated by commas (so if there is more than 
  	one entry in \meta{list} it must be surrounded by braces).
  	If \meta{list}  has less entries than node
    parts then the remaining node parts use the color from
    the last entry in the list. This key will automatically set
    |/pgf/rectangle split use custom fill|.
    
\begin{codeexample}[]
\begin{tikzpicture}
  \tikzset{every node/.style={rectangle split, draw, minimum width=.5cm}}
  \node[rectangle split part fill={red!50, green!50, blue!50, yellow!50}]  {};
  \node[rectangle split part fill={red!50, green!50, blue!50}] at (0.75,0) {};
  \node[rectangle split part fill={red!50, green!50}]          at (1.5,0)  {};
  \node[rectangle split part fill={red!50}]                    at (2.25,0) {};
\end{tikzpicture}
\end{codeexample}

	\end{key}
	
  The anchors for the |rectangle split| shape, are
  shown below (anchor |70| is an example of a border angle). When a 
  node part is missing (i.e., when the number of parts is less than 
  |4|), the anchors prefixed with name of that node part should be 
  considered unavailable. They are (unavoidably) defined, but default 
  to other anchor positions.
  
\begin{codeexample}[]
\Huge
\begin{tikzpicture}
  \node[name=s,shape=rectangle split, rectangle split parts=4, shape example] 
    {\nodepart{text}Text\nodepart{second}Second
		\nodepart{third}Third\nodepart{fourth}fourth};
  \foreach \anchor/\placement in
    {text/left,   text east/above,   text west/above, 
     second/left, second east/above, second west/above,    
     third/left,  third east/below,  third west/below,
     fourth/left, fourth east/below, fourth west/below,
     text split/left,   text split east/above,   text split west/above,
     second split/left, second split east/above, second split west/above,    
     third split/left,  third split east/below,  third split west/below, 
     north/above,  south/below,  east/below,  west/below,  
     center/above, 70/above,     mid/above,   base/below,
     north west/above, north east/above, south west/below, south east/below}
     \draw[shift=(s.\anchor)] plot[mark=x] coordinates{(0,0)}
       node[\placement] {\scriptsize\texttt{(s.\anchor)}};
\end{tikzpicture}
\end{codeexample}

\end{shape}






\subsection{Miscellaneous Shapes}

\begin{pgflibrary}{shapes.misc}
  This library defines general-purpose shapes that do not fit in the
  previous categories.
\end{pgflibrary}



\begin{shape}{cross out}
  This shape ``crosses out'' the node. Its foreground path are simply
  two diagonal lines that between the corners of the node's bounding
  box. Here is an example:

\begin{codeexample}[]
\begin{tikzpicture}
  \draw[help lines] (0,0) grid (3,2);
  \node [cross out,draw=red] at (1.5,1) {cross out};
\end{tikzpicture}
\end{codeexample}

  A useful application is inside text as in the following example:
\begin{codeexample}[]
Cross \tikz[baseline] \node [cross out,draw,anchor=text] {me}; out!  
\end{codeexample}

  This shape inherits all anchors from the |rectangle| shape, see also
  the following figure:
\begin{codeexample}[]
\Huge
\begin{tikzpicture}
  \node[name=s,shape=cross out,shape example] {cross out\vrule width 1pt height 2cm};
  \foreach \anchor/\placement in
    {north west/above left, north/above, north east/above right, 
     west/left, center/above, east/right, 
     mid west/right, mid/above, mid east/left, 
     base west/left, base/below, base east/right, 
     south west/below left, south/below, south east/below right, 
     text/left, 10/right, 130/above}
     \draw[shift=(s.\anchor)] plot[mark=x] coordinates{(0,0)}
       node[\placement] {\scriptsize\texttt{(s.\anchor)}};
\end{tikzpicture}
\end{codeexample}
\end{shape}

\begin{shape}{strike out}
  This shape is idential to the |cross out| shape, only its foreground
  path consists of a single line from the lower left to the upper
  right.
  
\begin{codeexample}[]
Strike \tikz[baseline] \node [strike out,draw,anchor=text] {me}; out!  
\end{codeexample}

  See the |cross out| shape for the anchors.
\end{shape}




\begin{shape}{rounded rectangle}
	This shape is a rectangle with semicircular arcs at each end. In 
	order	to maintain the semicircular arcs, increasing the height to
	fulfil any |minimum height| requirements will also increase the 
	width.

\begin{codeexample}[]
\begin{tikzpicture}
  \node[rounded rectangle, draw, fill=red!20]{Hallo};
\end{tikzpicture}
\end{codeexample}

	There is a key to specify which of the two ends are rounded (to use
	this key in \tikzname, simply remove the \declare{|/pgf/|} path).
	
\begin{key}{/pgf/rounded rectangle round=\meta{sides} (initially left and right)}
	Set the sides to be rounded. Currently these can only be
	\begin{itemize}
	\item
	  |left| or |west|,
	\item
		|right| or |east|.
	\end{itemize}
	However, two sides can be specified at the same time (e.g., |east and west|).
	If no value is given, no sides are rounded.
	
\begin{codeexample}[]
\begin{tikzpicture}
  \node[rounded rectangle, draw, rounded rectangle round=east]{123};
\end{tikzpicture}
\end{codeexample}

\end{key}
	
	The anchors for this shape are shown below (anchor |10| is an example
	of a border angle). Note that if only one side is rounded, the 
	|center| anchor will not be the precise center of the shape.
	
\begin{codeexample}[]
\Huge
\begin{tikzpicture}
  \node[name=s,shape=rounded rectangle, shape example, inner xsep=1.5cm] 
  	{Rounded Rectangle\vrule width 1pt height 2cm};
  \foreach \anchor/\placement in
    {center/above, text/below,      10/above,
     mid/above,    mid west/right,  mid east/left, 
     base/below,   base west/below, base east/below, 
     north/above,  south/below, east/above, west/above, 
     north west/above left,    north east/above right,     
     south west/below left,    south east/below right}
     \draw[shift=(s.\anchor)] plot[mark=x] coordinates{(0,0)}
       node[\placement] {\scriptsize\texttt{(s.\anchor)}};
\end{tikzpicture}
\end{codeexample}

\end{shape}



\begin{shape}{chamfered rectangle}

	This shape is a rectangle with optionally chamfered corners.
	
\begin{codeexample}[]
\begin{tikzpicture}
  \node[chamfered rectangle, white, fill=red, double=red, draw, very thick]
    {\bf STOP!};
\end{tikzpicture}
\end{codeexample}

	There are \pgfname{} keys to specify how this shape is drawn (to use
	these keys in \tikzname{} simply remove the \declare{|/pgf/|} path).

\begin{key}{/pgf/chamfered rectangle angle=\meta{angle} (initially 45)}
	Set the angle \emph{from the vertical} for the chamfer.
	
\begin{codeexample}[]
\begin{tikzpicture}
  \tikzset{every node/.style={chamfered rectangle, draw}}
  \node[chamfered rectangle angle=30] {abc};
  \node[chamfered rectangle angle=60] at (1.5,0) {123};
\end{tikzpicture}
\end{codeexample}
\end{key}

\begin{key}{/pgf/chamfered rectangle xsep=\meta{length} (initially .666ex)}
	Set the distance that the chamfer extends horizontally beyond the node 
	contents (which includes the |inner sep|). 
	If \meta{length} is large, such
	that the top and bottom chamfered edges would cross, then 
	\meta{length} is ignored and the chamfered edges are drawn so that
	they meet in the middle.

\begin{codeexample}[]
\begin{tikzpicture}
  \tikzset{every node/.style={chamfered rectangle, draw}}
  \node[chamfered rectangle xsep=2pt] {def};
  \node[chamfered rectangle xsep=2cm] at (1.5,0) {456};
\end{tikzpicture}
\end{codeexample}
	
\end{key}

\begin{key}{/pgf/chamfered rectangle ysep=\meta{length} (initially .666ex)}
	Set the distance that the chamfer extends vertically beyond the node 
	contents. 
	If \meta{length} is large, such that the left and right chamfered 
	edges would cross, then \meta{length} is ignored and the chamfered 
	edges are drawn so that	they meet in the middle.
\end{key}

\begin{key}{/pgf/chamfered rectangle sep=\meta{length} (initially .666ex)}
	Set both the |xsep| and |ysep| simultaneously.
\end{key}

\begin{key}{/pgf/chamfered rectangle corners=\meta{list} (initially chamfer all)}
	Specify which corners are chamfered. The corners are identified by 
	their ``compass point'' directions (i.e. |north east|, |north west|,
	|south west|, and |south east|), and must be separated by commas (so
	if there is more than one corner in the list, it must be surrounded
	by braces). Any corners not mentioned in 
	\meta{list} are automatically not chamfered. Two additional values
	|chamfer all| and |chamfer none|, are also permitted.

\begin{codeexample}[]
\begin{tikzpicture}
  \tikzset{every node/.style={chamfered rectangle, draw}}
  \node[chamfered rectangle corners=north west] {ghi};
  \node[chamfered rectangle corners={north east, south east}] at (1.5,0) {789};
\end{tikzpicture}
\end{codeexample}
\end{key}


	The anchors for this shape are shown below (anchor |60| is an example
	of a border angle.
	
\begin{codeexample}[]
\Huge
\begin{tikzpicture}
  \node[name=s,shape=chamfered rectangle, chamfered rectangle sep=1cm,
        shape example, inner ysep=1cm, inner xsep=.75cm] 
    {Chamfered Rectangle\vrule width1pt height2cm};
  \foreach \anchor/\placement in
    {text/right, center/above,    70/above, 
     base/below, base east/left, base west/right,
     mid/right,  mid east/above,   mid west/above,  
     north/above, south/below, east/above, west/above,
     before north east/above, north east/above, after north east/above,
     before north west/above, north west/above, after north west/above,
     before south west/below, south west/below, after south west/below,
     before south east/below, south east/below, after south east/below}     
     \draw[shift=(s.\anchor)] plot[mark=x] coordinates{(0,0)}
       node[\placement] {\scriptsize\texttt{(s.\anchor)}};
\end{tikzpicture}
\end{codeexample}

\end{shape}


%%% Local Variables: 
%%% mode: latex
%%% TeX-master: "pgfmanual-pdftex-version"
%%% End: 
