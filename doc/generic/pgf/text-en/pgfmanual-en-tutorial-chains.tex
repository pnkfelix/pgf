% Copyright 2006 by Till Tantau
%
% This file may be distributed and/or modified
%
% 1. under the LaTeX Project Public License and/or
% 2. under the GNU Free Documentation License.
%
% See the file doc/generic/pgf/licenses/LICENSE for more details.


\section{Tutorial: Putting a Diagram in Chains}

In this tutorial we have a look at how chains and matrices can be used
to typeset a diagram.

Ilka, who has just got tenured for her professorship on Really Old and
Obscure Programming Languages, has recently dug up an interesting
diagram in the dusty cellar of the library of her university. Having
been created in the good old times using a pen and a rules, it looks
like this:

\bigskip
\begin{tikzpicture}[
  >=stealth,thick,
  /pgf/every decoration/.style={/tikz/sharp corners},
  fuzzy/.style={decorate,decoration={post length=2pt,random steps,segment length=1mm,amplitude=0.2pt}},
  fuzzy round/.style={decorate,decoration={post length=2pt,random steps,segment length=0.3mm,amplitude=0.2pt}},
  minimum size=6mm,line join=round,
  terminal/.style={rectangle,draw,fill=white,ultra thick,fuzzy round,rounded corners=3mm},
  nonterminal/.style={rectangle,draw,fill=white,ultra thick,fuzzy},
  node distance=3mm]

  \ttfamily
  \begin{scope}[start chain,
                every node/.style={on chain},
                terminal/.append style={join=by {->,shorten >=1pt,fuzzy}},
                nonterminal/.append style={join=by {->,shorten >=1pt,fuzzy}},
                support/.style={coordinate,join=by fuzzy}]
    \node [support]             (start)        {};
    \node [nonterminal]                        {unsigned integer};
    \node [support]             (after ui)     {};
    \node [terminal]                           {.};
    \node [support]             (after dot)    {};
    \node [terminal]                           {digit};
    \node [support]             (after digit)  {};
    \node [support]             (skip)         {};    
    \node [support]             (before E)     {};
    \node [terminal]                           {E};
    \node [support]             (after E)      {};
    \node [support,xshift=5mm]  (between)      {};
    \node [support,xshift=5mm]  (before last)  {};
    \node [nonterminal]                        {unsigned integer};
    \node [support]             (after last)   {};
    \node [join=by ->]          (end)          {};
  \end{scope}
  \node (plus)  [terminal,above=of between] {$+$};
  \node (minus) [terminal,below=of between] {$-$};

  \begin{scope}[->,shorten >=1pt,rounded corners=7pt,every path/.style=fuzzy]
    \draw (after ui)    -- +(0,.7)  -| (skip);
    \draw (after digit) -- +(0,-.7) -| (after dot);
    \draw (before E)    -- +(0,-1.2) -| (after last);
    \draw (after E)     |- (plus);
    \draw (plus)        -| (before last);
    \draw (after E)     |- (minus);
    \draw (minus)       -| (before last);
  \end{scope}
\end{tikzpicture}
\bigskip

Ilka decides to redo this diagram in ``clearer'' fashion.

