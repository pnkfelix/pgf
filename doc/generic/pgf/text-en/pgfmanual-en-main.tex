% Copyright 2006 by Till Tantau
%
% This file may be distributed and/or modified
%
% 1. under the LaTeX Project Public License and/or
% 2. under the GNU Free Documentation License.
%
% See the file doc/generic/pgf/licenses/LICENSE for more details.


% pgf version is defined in \pgfversion in file
% generic/pgf/utilities/pgfrcs.code.tex 

\def\xcolorversion{2.00}

\usepackage[version=latest]{pgf}

\usepackage{xkeyval,calc,listings,tikz}

% We need lots of libraries...
\usetikzlibrary{%
  arrows,%
  calc,%
  fit,%
  patterns,%
  plotmarks,%
  shapes.geometric,%
  shapes.misc,%
  shapes.symbols,%
  shapes.arrows,%
  shapes.callouts,%
  shapes.multipart,%
  shapes.gates.logic.US,%
  shapes.gates.logic.IEC,%
  er,%
  automata,%
  backgrounds,%
  chains,%
  topaths,%
  trees,%
  petri,%
  mindmap,%
  matrix,%
  calendar,%
  folding,%
  fadings,%
  through,%
  positioning,%
  scopes,%
  decorations.fractals,%
  decorations.shapes,%
  decorations.text,%
  decorations.pathmorphing,%
  decorations.pathreplacing,%
  decorations.footprints,%
  decorations.markings,%
  shadows}

\usepackage[a4paper,left=2.25cm,right=2.25cm,top=2.5cm,bottom=2.5cm,nohead]{geometry}
\usepackage{amsmath,amssymb}
\usepackage{xxcolor}
\usepackage{pifont}
\usepackage{makeidx}
\usepackage[latin1]{inputenc}
\usepackage{amsmath}

% Copyright 2006 by Till Tantau
%
% This file may be distributed and/or modified
%
% 1. under the LaTeX Project Public License and/or
% 2. under the GNU Free Documentation License.
%
% See the file doc/generic/pgf/licenses/LICENSE for more details.

% $Header: /cvsroot/pgf/pgf/doc/generic/pgf/macros/pgfmanual-en-macros.tex,v 1.25 2008/01/15 18:05:09 tantau Exp $


\providecommand\href[2]{\texttt{#1}}


\colorlet{examplefill}{yellow!80!black}
\definecolor{graphicbackground}{rgb}{0.96,0.96,0.8}
\definecolor{codebackground}{rgb}{0.8,0.8,1}

\newenvironment{pgfmanualentry}{\list{}{\leftmargin=2em\itemindent-\leftmargin\def\makelabel##1{\hss##1}}}{\endlist}
\newcommand\pgfmanualentryheadline[1]{\itemsep=0pt\parskip=0pt\item\strut#1\par\topsep=0pt}
\newcommand\pgfmanualbody{\parskip3pt}



\newenvironment{pgflayout}[1]{
  \begin{pgfmanualentry}
    \pgfmanualentryheadline{\texttt{\string\pgfpagesuselayout\char`\{\declare{#1}\char`\}}\oarg{options}}
    \index{#1@\protect\texttt{#1} layout}%
    \index{Page layouts!#1@\protect\texttt{#1}}%
    \pgfmanualbody
}
{
  \end{pgfmanualentry}
}


\newenvironment{command}[1]{
  \begin{pgfmanualentry}
    \extractcommand#1\@@
    \pgfmanualbody
}
{
  \end{pgfmanualentry}
}

%% MW: START MATH MACROS
\def\mvar#1{{\rmfamily\textit{#1}}}

\makeatletter

\def\extractmathfunctionname#1{\extractmathfunctionname@#1(,)\tmpa\tmpb}
\def\extractmathfunctionname@#1(#2)#3\tmpb{\def\mathname{#1}}

\def\extractmathoperatorname{\begingroup\def\mvar##1{}\def\ {}\extractmathoperatorname@}
\def\extractmathoperatorname@#1{\xdef\mathname{#1}\endgroup}

\makeatother
	
\def\vskipspecial#1{\vskip#1\vskip0em}

\newenvironment{math-function}[1]{
	\begin{pgfmanualentry}
		\extractmathfunctionname{#1}
		\pgfmanualentryheadline{\texttt{#1}}%
		\index{\mathname @\protect\texttt{\mathname} math function}%
		\index{Math functions!\mathname @\protect\texttt{\mathname}}
		\pgfmanualbody
}
{
	\end{pgfmanualentry}\vskipspecial{-3em}
}

\newenvironment{math-operator}[1]{	
	\begin{pgfmanualentry}
		\extractmathoperatorname{#1}
		\pgfmanualentryheadline{\texttt{#1}}%
		\index{\mathname @\protect\texttt{\mathname} math operator}%
		\index{Math operators!\mathname @\protect\texttt{\mathname}}
    	\pgfmanualbody
}
{%
	\end{pgfmanualentry}\vskipspecial{-3em}
}

\newenvironment{math-constant}[1]{
	\begin{pgfmanualentry}
		\pgfmanualentryheadline{\texttt{#1}}%
		\index{#1@\protect\texttt{#1} math constant}%
		\index{Math constants!#1@\protect\texttt{#1}}
		\pgfmanualbody
}
{
	\end{pgfmanualentry}\vskipspecial{-3em}
}
\def\calcname{\textsc{calc}}
%% MW: END MATH MACROS


\def\extractcommand#1#2\@@{%
  \pgfmanualentryheadline{\declare{\texttt{\string#1}}#2}%
  \removeats{#1}%
  \index{\strippedat @\protect\myprintocmmand{\strippedat}}}


\renewenvironment{environment}[1]{
  \begin{pgfmanualentry}
    \extractenvironement#1\@@
    \pgfmanualbody
}
{
  \end{pgfmanualentry}
}

\def\extractenvironement#1#2\@@{%
  \pgfmanualentryheadline{{\ttfamily\char`\\begin\char`\{\declare{#1}\char`\}}#2}%
  \pgfmanualentryheadline{{\ttfamily\ \ }\meta{environment contents}}%
  \pgfmanualentryheadline{{\ttfamily\char`\\end\char`\{\declare{#1}\char`\}}}%
  \index{#1@\protect\texttt{#1} environment}%
  \index{Environments!#1@\protect\texttt{#1}}}


\newenvironment{plainenvironment}[1]{
  \begin{pgfmanualentry}
    \extractplainenvironement#1\@@
    \pgfmanualbody
}
{
  \end{pgfmanualentry}
}

\def\extractplainenvironement#1#2\@@{%
  \pgfmanualentryheadline{{\ttfamily\declare{\char`\\#1}}#2}%
  \pgfmanualentryheadline{{\ttfamily\ \ }\meta{environment contents}}%
  \pgfmanualentryheadline{{\ttfamily\declare{\char`\\end#1}}}%
  \index{#1@\protect\texttt{#1} environment}%
  \index{Environments!#1@\protect\texttt{#1}}}


\newenvironment{contextenvironment}[1]{
  \begin{pgfmanualentry}
    \extractcontextenvironement#1\@@
    \pgfmanualbody
}
{
  \end{pgfmanualentry}
}

\def\extractcontextenvironement#1#2\@@{%
  \pgfmanualentryheadline{{\ttfamily\declare{\char`\\start#1}}#2}%
  \pgfmanualentryheadline{{\ttfamily\ \ }\meta{environment contents}}%
  \pgfmanualentryheadline{{\ttfamily\declare{\char`\\stop#1}}}%
  \index{#1@\protect\texttt{#1} environment}%
  \index{Environments!#1@\protect\texttt{#1}}}


\newenvironment{shape}[1]{
  \begin{pgfmanualentry}
  	\pgfmanualentryheadline{Shape {\ttfamily\declare{#1}}}%
    \index{#1@\protect\texttt{#1} shape}%
    \index{Shapes!#1@\protect\texttt{#1}}
    \pgfmanualbody
}
{
  \end{pgfmanualentry}
}


\newenvironment{handler}[1]{
  \begin{pgfmanualentry}
    \extracthandler#1\@nil%
    \pgfmanualbody
}
{
  \end{pgfmanualentry}
}

\def\gobble#1{}
\def\extracthandler#1#2\@nil{%
  \pgfmanualentryheadline{Key handler \meta{key}{\ttfamily/\declare{#1}}#2}%
  \index{\gobble#1@\protect\texttt{#1} handler}%
  \index{Key handlers!#1@\protect\texttt{#1}}
}


\makeatletter


\newenvironment{stylekey}[1]{
  \begin{pgfmanualentry}
    \def\extrakeytext{style, }
    \extractkey#1\@nil%
    \pgfmanualbody
}
{
  \end{pgfmanualentry}
}


\newenvironment{key}[1]{
  \begin{pgfmanualentry}
    \def\extrakeytext{}
    %\def\altpath{\emph{\color{gray}or}}%
    \extractkey#1\@nil%
    \pgfmanualbody
}
{
  \end{pgfmanualentry}
}

\def\extractkey#1\@nil{%
  \pgfutil@in@={#1}%
  \ifpgfutil@in@%
    \extractkeyequal#1\@nil
  \else%
    \pgfutil@in@{(initial}{#1}%
    \ifpgfutil@in@%
      \extractequalinitial#1\@nil%
    \else
      \pgfmanualentryheadline{{\ttfamily\declare{#1}}\hfill(\extrakeytext no value)}%
      \def\mykey{#1}%
      \def\mypath{}%
      \def\myname{}%
      \firsttimetrue%
      \decompose#1/\nil%
    \fi
  \fi%
}

\def\extractkeyequal#1=#2\@nil{%
  \pgfutil@in@{(default}{#2}%
  \ifpgfutil@in@%
    \extractdefault{#1}#2\@nil%
  \else%
    \pgfutil@in@{(initial}{#2}%
    \ifpgfutil@in@%
      \extractinitial{#1}#2\@nil%
    \else
      \pgfmanualentryheadline{{\ttfamily\declare{#1}=}#2\hfill(\extrakeytext no default)}%
    \fi%
  \fi%
  \def\mykey{#1}%
  \def\mypath{}%
  \def\myname{}%
  \firsttimetrue%
  \decompose#1/\nil%
}

\def\extractdefault#1#2(default #3)\@nil{%
  \pgfmanualentryheadline{{\ttfamily\declare{#1}\opt{=}}\opt{#2}\hfill (\extrakeytext default {\ttfamily#3})}%
}

\def\extractinitial#1#2(initially #3)\@nil{%
  \pgfmanualentryheadline{{\ttfamily\declare{#1}=}#2\hfill (\extrakeytext no default, initially {\ttfamily#3})}%
}

\def\extractequalinitial#1 (initially #2)\@nil{%
  \pgfmanualentryheadline{{\ttfamily\declare{#1}}\hfill (\extrakeytext initially {\ttfamily#2})}%
  \def\mykey{#1}%
  \def\mypath{}%
  \def\myname{}%
  \firsttimetrue%
  \decompose#1/\nil%
}

\def\keyalias#1{\vspace{-3pt}\item{\small alias {\ttfamily/#1/\myname}}\vspace{-2pt}\par}

\newif\iffirsttime

\makeatother

\def\decompose/#1/#2\nil{%
  \def\test{#2}%
  \ifx\test\empty%
    % aha.
    \index{#1@\protect\texttt{#1} key}%
    \index{\mypath#1@\protect\texttt{#1}}%
    \def\myname{#1}%
  \else%
    \iffirsttime
      \def\mypath{#1@\protect\texttt{/#1/}!}%
      \firsttimefalse
    \else
      \expandafter\def\expandafter\mypath\expandafter{\mypath#1@\protect\texttt{#1/}!}%
    \fi
    \def\firsttime{}
    \decompose/#2\nil%
  \fi%
}


\newenvironment{predefinednode}[1]{
  \begin{pgfmanualentry}
    \pgfmanualentryheadline{Predefined node {\ttfamily\declare{#1}}}%
    \index{#1@\protect\texttt{#1} node}%
    \index{Predefined node!#1@\protect\texttt{#1}}
    \pgfmanualbody
}
{
  \end{pgfmanualentry}
}

\newenvironment{coordinatesystem}[1]{
  \begin{pgfmanualentry}
    \pgfmanualentryheadline{Coordinate system {\ttfamily\declare{#1}}}%
    \index{#1@\protect\texttt{#1} coordinate system}%
    \index{Coordinate systems!#1@\protect\texttt{#1}}
    \pgfmanualbody
}
{
  \end{pgfmanualentry}
}

\newenvironment{snake}[1]{
  \begin{pgfmanualentry}
    \pgfmanualentryheadline{Snake {\ttfamily\declare{#1}}}%
    \index{#1@\protect\texttt{#1} snake}%
    \index{Snakes!#1@\protect\texttt{#1}}
    \pgfmanualbody
}
{
  \end{pgfmanualentry}
}

\newenvironment{decoration}[1]{
  \begin{pgfmanualentry}
    \pgfmanualentryheadline{Decoration {\ttfamily\declare{#1}}}%
    \index{#1@\protect\texttt{#1} decoration}%
    \index{Decorations!#1@\protect\texttt{#1}}
    \pgfmanualbody
}
{
  \end{pgfmanualentry}
}

\newenvironment{meta-decoration}[1]{
  \begin{pgfmanualentry}
    \pgfmanualentryheadline{Meta Decoration {\ttfamily\declare{#1}}}%
    \index{#1@\protect\texttt{#1} decoration}%
    \index{Decorations!#1@\protect\texttt{#1}}
    \pgfmanualbody
}
{
  \end{pgfmanualentry}
}

\def\pgfmanualbar{\char`\|}
\makeatletter
\newenvironment{pathoperation}[3][]{
  \begin{pgfmanualentry}
    \pgfmanualentryheadline{\textcolor{gray}{{\ttfamily\char`\\path}\
        \ \dots}
      \declare{\texttt{#2}}#3\ \textcolor{gray}{\dots\texttt{;}}}%
    \def\pgfmanualtest{#1}%
    \ifx\pgfmanualtest\@empty%
      \index{#2@\protect\texttt{#2} path operation}%
      \index{Path operations!#2@\protect\texttt{#2}}%
    \fi%
    \pgfmanualbody
}
{
  \end{pgfmanualentry}
}
\makeatother

\def\extractcommand#1#2\@@{%
  \pgfmanualentryheadline{\declare{\texttt{\string#1}}#2}%
  \removeats{#1}%
  \index{\strippedat @\protect\myprintocmmand{\strippedat}}}

\def\doublebs{\texttt{\char`\\\char`\\}}


\newenvironment{package}[1]{
  \begin{pgfmanualentry}
    \pgfmanualentryheadline{{\ttfamily\char`\\usepackage\char`\{\declare{#1}\char`\}\space\space \char`\%\space\space  \LaTeX}}
    \index{#1@\protect\texttt{#1} package}%
    \index{Packages and files!#1@\protect\texttt{#1}}%
    \pgfmanualentryheadline{{\ttfamily\char`\\input \declare{#1}.tex\space\space\space \char`\%\space\space  plain \TeX}}
    \pgfmanualentryheadline{{\ttfamily\char`\\usemodule[\declare{#1}]\space\space \char`\%\space\space  Con\TeX t}}
    \pgfmanualbody
}
{
  \end{pgfmanualentry}
}


\newenvironment{pgfmodule}[1]{
  \begin{pgfmanualentry}
    \pgfmanualentryheadline{{\ttfamily\char`\\usepgfmodule\char`\{\declare{#1}\char`\}\space\space\space
        \char`\%\space\space  \LaTeX\space and plain \TeX\space and pure pgf}}
    \index{#1@\protect\texttt{#1} module}%
    \index{Modules!#1@\protect\texttt{#1}}%
    \pgfmanualentryheadline{{\ttfamily\char`\\usepgfmodule[\declare{#1}]\space\space \char`\%\space\space  Con\TeX t\space and pure pgf}}
    \pgfmanualbody
}
{
  \end{pgfmanualentry}
}

\newenvironment{pgflibrary}[1]{
  \begin{pgfmanualentry}
    \pgfmanualentryheadline{{\ttfamily\char`\\usepgflibrary\char`\{\declare{#1}\char`\}\space\space\space
        \char`\%\space\space  \LaTeX\space and plain \TeX\space and pure pgf}}
    \index{#1@\protect\texttt{#1} library}%
    \index{Libraries!#1@\protect\texttt{#1}}%
    \pgfmanualentryheadline{{\ttfamily\char`\\usepgflibrary[\declare{#1}]\space\space \char`\%\space\space  Con\TeX t\space and pure pgf}}
    \pgfmanualentryheadline{{\ttfamily\char`\\usetikzlibrary\char`\{\declare{#1}\char`\}\space\space
        \char`\%\space\space  \LaTeX\space and plain \TeX\space when using \tikzname}}
    \pgfmanualentryheadline{{\ttfamily\char`\\usetikzlibrary[\declare{#1}]\space
        \char`\%\space\space  Con\TeX t\space when using \tikzname}}
    \pgfmanualbody
}
{
  \end{pgfmanualentry}
}

\newenvironment{tikzlibrary}[1]{
  \begin{pgfmanualentry}
    \pgfmanualentryheadline{{\ttfamily\char`\\usetikzlibrary\char`\{\declare{#1}\char`\}\space\space \char`\%\space\space  \LaTeX\space and plain \TeX}}
    \index{#1@\protect\texttt{#1} library}%
    \index{Libraries!#1@\protect\texttt{#1}}%
    \pgfmanualentryheadline{{\ttfamily\char`\\usetikzlibrary[\declare{#1}]\space \char`\%\space\space Con\TeX t}}
    \pgfmanualbody
}
{
  \end{pgfmanualentry}
}



\newenvironment{filedescription}[1]{
  \begin{pgfmanualentry}
    \pgfmanualentryheadline{File {\ttfamily\declare{#1}}}%
    \index{#1@\protect\texttt{#1} file}%
    \index{Packages and files!#1@\protect\texttt{#1}}%
    \pgfmanualbody
}
{
  \end{pgfmanualentry}
}


\newenvironment{packageoption}[1]{
  \begin{pgfmanualentry}
    \pgfmanualentryheadline{{\ttfamily\char`\\usepackage[\declare{#1}]\char`\{pgf\char`\}}}
    \index{#1@\protect\texttt{#1} package option}%
    \index{Package options for \textsc{pgf}!#1@\protect\texttt{#1}}%
    \pgfmanualbody
}
{
  \end{pgfmanualentry}
}



\newcommand\opt[1]{{\color{black!50!green}#1}}
\newcommand\ooarg[1]{{\ttfamily[}\meta{#1}{\ttfamily]}}

\def\opt{\afterassignment\pgfmanualopt\let\next=}
\def\pgfmanualopt{\ifx\next\bgroup\bgroup\color{black!50!green}\else{\color{black!50!green}\next}\fi}



\def\beamer{\textsc{beamer}}
\def\pdf{\textsc{pdf}}
\def\pgfname{\textsc{pgf}}
\def\tikzname{Ti\emph{k}Z}
\def\pstricks{\textsc{pstricks}}
\def\prosper{\textsc{prosper}}
\def\seminar{\textsc{seminar}}
\def\texpower{\textsc{texpower}}
\def\foils{\textsc{foils}}

{
  \makeatletter
  \global\let\myempty=\@empty
  \global\let\mygobble=\@gobble
  \catcode`\@=12
  \gdef\getridofats#1@#2\relax{%
    \def\getridtest{#2}%
    \ifx\getridtest\myempty%
      \expandafter\def\expandafter\strippedat\expandafter{\strippedat#1}
    \else%
      \expandafter\def\expandafter\strippedat\expandafter{\strippedat#1\protect\printanat}
      \getridofats#2\relax%
    \fi%
  }

  \gdef\removeats#1{%
    \let\strippedat\myempty%
    \edef\strippedtext{\stripcommand#1}%
    \expandafter\getridofats\strippedtext @\relax%
  }
  
  \gdef\stripcommand#1{\expandafter\mygobble\string#1}
}

\def\printanat{\char`\@}

\def\declare{\afterassignment\pgfmanualdeclare\let\next=}
\def\pgfmanualdeclare{\ifx\next\bgroup\bgroup\color{red!75!black}\else{\color{red!75!black}\next}\fi}


\let\textoken=\command
\let\endtextoken=\endcommand

\def\myprintocmmand#1{\texttt{\char`\\#1}}

\def\example{\par\smallskip\noindent\textit{Example: }}
\def\themeauthor{\par\smallskip\noindent\textit{Theme author: }}


\def\indexoption#1{%
  \index{#1@\protect\texttt{#1} option}%
  \index{Graphic options and styles!#1@\protect\texttt{#1}}%
}

\def\itemcalendaroption#1{\item \declare{\texttt{#1}}%
  \index{#1@\protect\texttt{#1} date test}%
  \index{Date tests!#1@\protect\texttt{#1}}%
}



\def\class#1{\list{}{\leftmargin=2em\itemindent-\leftmargin\def\makelabel##1{\hss##1}}%
\extractclass#1@\par\topsep=0pt}
\def\endclass{\endlist}
\def\extractclass#1#2@{%
\item{{{\ttfamily\char`\\documentclass}#2{\ttfamily\char`\{\declare{#1}\char`\}}}}%
  \index{#1@\protect\texttt{#1} class}%
  \index{Classes!#1@\protect\texttt{#1}}}

\def\partname{Part}

\makeatletter
\def\index@prologue{\section*{Index}\addcontentsline{toc}{section}{Index}
  This index only contains automatically generated entries. A good
  index should also contain carefully selected keywords. This index is
  not a good index.
  \bigskip
}
\c@IndexColumns=2
  \def\theindex{\@restonecoltrue
    \columnseprule \z@  \columnsep 35\p@
    \twocolumn[\index@prologue]%
       \parindent -30pt
       \columnsep 15pt
       \parskip 0pt plus 1pt
       \leftskip 30pt
       \rightskip 0pt plus 2cm
       \small
       \def\@idxitem{\par}%
    \let\item\@idxitem \ignorespaces}
  \def\endtheindex{\onecolumn}
\def\noindexing{\let\index=\@gobble}



\newcommand\symarrow[1]{
  \index{#1@\protect\texttt{#1} arrow tip}%
  \index{Arrow tips!#1@\protect\texttt{#1}}
  \texttt{#1}& yields thick  
  \begin{tikzpicture}[arrows={#1-#1},thick,baseline]
    \useasboundingbox (0pt,-0.5ex) rectangle (1cm,2ex);
    \draw (0pt,.5ex) -- (1cm,.5ex);
  \end{tikzpicture} and thin
  \begin{tikzpicture}[arrows={#1-#1},thin,baseline]
    \useasboundingbox (0pt,-0.5ex) rectangle (1cm,2ex);
    \draw (0pt,.5ex) -- (1cm,.5ex);
  \end{tikzpicture}
}

\newcommand\sarrow[2]{
  \index{#1@\protect\texttt{#1} arrow tip}%
  \index{Arrow tips!#1@\protect\texttt{#1}}
  \index{#2@\protect\texttt{#2} arrow tip}%
  \index{Arrow tips!#2@\protect\texttt{#2}}
  \texttt{#1-#2}& yields thick  
  \begin{tikzpicture}[arrows={#1-#2},thick,baseline]
    \useasboundingbox (0pt,-0.5ex) rectangle (1cm,2ex);
    \draw (0pt,.5ex) -- (1cm,.5ex);
  \end{tikzpicture} and thin
  \begin{tikzpicture}[arrows={#1-#2},thin,baseline]
    \useasboundingbox (0pt,-0.5ex) rectangle (1cm,2ex);
    \draw (0pt,.5ex) -- (1cm,.5ex);
  \end{tikzpicture}
}

\newcommand\carrow[1]{
  \index{#1@\protect\texttt{#1} arrow tip}%
  \index{Arrow tips!#1@\protect\texttt{#1}}
  \texttt{#1}& yields for line width 1ex
  \begin{tikzpicture}[arrows={#1-#1},line width=1ex,baseline]
    \useasboundingbox (0pt,-0.5ex) rectangle (1.5cm,2ex);
    \draw (0pt,.5ex) -- (1.5cm,.5ex);
  \end{tikzpicture}
}
\def\myvbar{\char`\|}
\newcommand\plotmarkentry[1]{%
  \index{#1@\protect\texttt{#1} plot mark}%
  \index{Plot marks!#1@\protect\texttt{#1}}
  \texttt{\char`\\pgfuseplotmark\char`\{\declare{#1}\char`\}} &
  \tikz\draw[color=black!25] plot[mark=#1,mark options={fill=examplefill,draw=black}] coordinates{(0,0) (.5,0.2) (1,0) (1.5,0.2)};\\
}
\newcommand\plotmarkentrytikz[1]{%
  \index{#1@\protect\texttt{#1} plot mark}%
  \index{Plot marks!#1@\protect\texttt{#1}}
  \texttt{mark=\declare{#1}} & \tikz\draw[color=black!25]
  plot[mark=#1,mark options={fill=examplefill,draw=black}] 
    coordinates {(0,0) (.5,0.2) (1,0) (1.5,0.2)};\\
}



\ifx\scantokens\@undefined
  \PackageError{pgfmanual-macros}{You need to use extended latex
    (elatex) or (pdfelatex) to process this document}{}
\fi

\begingroup
\catcode`|=0
\catcode`[= 1
\catcode`]=2
\catcode`\{=12
\catcode `\}=12
\catcode`\\=12 |gdef|find@example#1\end{codeexample}[|endofcodeexample[#1]]
|endgroup

\begingroup
\catcode`\^=7
\catcode`\^^M=13
\catcode`\ =13%
\gdef\returntospace{\catcode`\ =13\def {\space}\catcode`\^^M=13\def^^M{}}%
\endgroup

\begingroup
\catcode`\%=13
\catcode`\^^M=13
\gdef\commenthandler{\catcode`\%=13\def%{\@gobble@till@return}}
\gdef\@gobble@till@return#1^^M{}
\gdef\@gobble@till@return@ignore#1^^M{\ignorespaces}
\gdef\typesetcomment{\catcode`\%=13\def%{\@typeset@till@return}}
\gdef\@typeset@till@return#1^^M{{\def%{\char`\%}\textsl{\char`\%#1}}\par}
\endgroup

\define@key{codeexample}{width}{\setlength\codeexamplewidth{#1}}
\define@key{codeexample}{graphic}{\colorlet{graphicbackground}{#1}}
\define@key{codeexample}{code}{\colorlet{codebackground}{#1}}
\define@key{codeexample}{execute code}{\csname code@execute#1\endcsname}
\define@key{codeexample}{code only}[]{\code@executefalse}
\define@key{codeexample}{pre}{\def\code@pre{#1}}
\define@key{codeexample}{post}{\def\code@post{#1}}
\define@key{codeexample}{vbox}[]{\def\code@pre{\vbox\bgroup\setlength{\hsize}{\linewidth-6pt}}\def\code@post{\egroup}}
\define@key{codeexample}{ignorespaces}[]{\let\@gobble@till@return=\@gobble@till@return@ignore}
\define@key{codeexample}{leave comments}[]{\def\code@catcode@hook{\catcode`\%=12}\let\commenthandler=\relax\let\typesetcomment=\relax}

\def\code@pre{}
\def\code@post{}
\def\code@catcode@hook{}

\newdimen\codeexamplewidth
\newif\ifcode@execute
\newbox\codeexamplebox
\def\codeexample[#1]{%
  \begingroup%
  \code@executetrue
  \setlength\codeexamplewidth{4cm+7pt}
  \setkeys{codeexample}{#1}%
  \parindent0pt
  \begingroup%
  \par%
  \medskip%
  \let\do\@makeother%
  \dospecials%
  \obeylines%
  \@vobeyspaces%
  \catcode`\%=13%
  \catcode`\^^M=13%
  \code@catcode@hook%
  \relax%
  \find@example}
\def\endofcodeexample#1{%
  \endgroup%
  \ifcode@execute%
    \setbox\codeexamplebox=\hbox{%
      {%
        {%
          \returntospace%
          \commenthandler%
          \xdef\code@temp{#1}% removes returns and comments
        }%
        \colorbox{graphicbackground}{\color{black}\ignorespaces%
          \code@pre\expandafter\scantokens\expandafter{\code@temp\ignorespaces}\code@post\ignorespaces}%
      }%
    }%
    \ifdim\wd\codeexamplebox>\codeexamplewidth%
      \def\code@start{\par}%
      \def\code@flushstart{}\def\code@flushend{}%
      \def\code@mid{\parskip2pt\par\noindent}%
      \def\code@width{\linewidth-6pt}%
      \def\code@end{}%
    \else%
      \def\code@start{%
        \linewidth=\textwidth%
        \parshape \@ne 0pt \linewidth
        \leavevmode%
        \hbox\bgroup}%
      \def\code@flushstart{\hfill}%
      \def\code@flushend{\hbox{}}%
      \def\code@mid{\hskip6pt}%
      \def\code@width{\linewidth-12pt-\codeexamplewidth}%
      \def\code@end{\egroup}%
    \fi%
    \code@start%
    \noindent%
    \begin{minipage}[t]{\codeexamplewidth}\raggedright
      \hrule width0pt%
      \footnotesize\vskip-1em%
      \code@flushstart\box\codeexamplebox\code@flushend%
      \vskip-1ex
      \leavevmode%
    \end{minipage}%
  \else%
    \def\code@mid{\par}
    \def\code@width{\linewidth-6pt}
    \def\code@end{}
  \fi%
  \code@mid%  
  \colorbox{codebackground}{%
    \begin{minipage}[t]{\code@width}%
      {%
        \let\do\@makeother
        \dospecials
        \frenchspacing\@vobeyspaces
        \normalfont\ttfamily\footnotesize
        \typesetcomment%
        \@tempswafalse
        \def\par{%
          \if@tempswa
          \leavevmode \null \@@par\penalty\interlinepenalty
          \else
          \@tempswatrue
          \ifhmode\@@par\penalty\interlinepenalty\fi
          \fi}%
        \obeylines
        \everypar \expandafter{\the\everypar \unpenalty}%
        #1}
    \end{minipage}}%
  \code@end%
  \par%
  \medskip
  \end{codeexample}
}

\def\endcodeexample{\endgroup}


\makeatother


%%% Local Variables: 
%%% mode: latex
%%% TeX-master: "beameruserguide"
%%% End: 


\makeindex

\makeatletter
\renewcommand*\l@subsection{\@dottedtocline{2}{1.5em}{2.8em}}
\renewcommand*\l@subsubsection{\@dottedtocline{3}{4.3em}{3.2em}}
\makeatother

%\includeonly{}

% Global styles:
\tikzset{
  every plot/.style={prefix=plots/pgf-},
  shape example/.style={
    color=black!30,
    draw,
    fill=yellow!30,
    line width=.5cm,
    inner xsep=2.5cm,
    inner ysep=0.5cm}
}

\index{Options for graphics|see{Graphic options and styles}}
\index{Styles for graphics|see{Graphic options and styles}}
\index{Options for packages|see{Package options}}
\index{Handlers for keys|see{Key handlers}}
\index{File|see{Packages and files}}
\index{Layout|see{Page layout}}
\index{Node|see{Predefined node}}

\begin{document}

%% Copyright 2006 by Till Tantau
%
% This file may be distributed and/or modified
%
% 1. under the LaTeX Project Public License and/or
% 2. under the GNU Free Documentation License.
%
% See the file doc/generic/pgf/licenses/LICENSE for more details.


\section{Tutorial: Putting a Diagram in Chains}

In this tutorial we have a look at how chains and matrices can be used
to typeset a diagram.

Ilka, who has just got tenured for her professorship of Old and
Lovable Programming Languages, has recently dug up a technical report entitled
\emph{The Programming Language Pascal} in the dusty cellar of the
library of her university. Having been created in the good old times
using a pen and a rules, it looks like this:

% \bigskip
% \begin{tikzpicture}[
%   >=latex,thick,
%   /pgf/every decoration/.style={/tikz/sharp corners},
%   fuzzy/.style={decorate,decoration={random steps,segment length=0.5mm,amplitude=0.15pt}},
%   minimum size=6mm,line join=round,line cap=round,
%   terminal/.style={rectangle,draw,fill=white,fuzzy,rounded corners=3mm},
%   nonterminal/.style={rectangle,draw,fill=white,fuzzy},
%   node distance=3mm]

%   \ttfamily
%   \begin{scope}[start chain,
%                 every node/.style={on chain},
%                 terminal/.append style={join=by {->,shorten >=-1pt,fuzzy,decoration={post length=4pt}}},
%                 nonterminal/.append style={join=by {->,shorten >=-1pt,fuzzy,decoration={post length=4pt}}},
%                 support/.style={coordinate,join=by fuzzy}]
%     \node [support]             (start)        {};
%     \node [nonterminal]                        {unsigned integer};
%     \node [support]             (after ui)     {};
%     \node [terminal]                           {.};
%     \node [support]             (after dot)    {};
%     \node [terminal]                           {digit};
%     \node [support]             (after digit)  {};
%     \node [support]             (skip)         {};    
%     \node [support]             (before E)     {};
%     \node [terminal]                           {E};
%     \node [support]             (after E)      {};
%     \node [support,xshift=5mm]  (between)      {};
%     \node [support,xshift=5mm]  (before last)  {};
%     \node [nonterminal]                        {unsigned integer};
%     \node [support]             (after last)   {};
%     \node [join=by ->]          (end)          {};
%   \end{scope}
%   \node (plus)  [terminal,above=of between] {$+$};
%   \node (minus) [terminal,below=of between] {$-$};

%   \begin{scope}[->,decoration={post length=4pt},rounded corners=2mm,every path/.style=fuzzy]
%     \draw (after ui)    -- +(0,.7)  -| (skip);
%     \draw (after digit) -- +(0,-.7) -| (after dot);
%     \draw (before E)    -- +(0,-1.2) -| (after last);
%     \draw (after E)     |- (plus);
%     \draw (plus)        -| (before last);
%     \draw (after E)     |- (minus);
%     \draw (minus)       -| (before last);
%   \end{scope}
% \end{tikzpicture}
% \bigskip

For her next lecture, Ilka decides to redo this diagram, but this time
perhaps a bit ``cleaner'' and perhaps also bit ``cooler.''


\bigskip
\begin{tikzpicture}[
  >=stealth',thick,draw=black!70,fill=black!70,
  minimum size=6mm,line join=round,line cap=round,
  terminal/.style={font=\ttfamily,rectangle,very thick,draw=black!50,bottom
    color=black!20,top color=white,rounded corners=3mm},
  nonterminal/.style={font=\itshape,rectangle,very thick,draw=red!50!black!50,bottom
    color=red!50!black!20,top color=white},
  node distance=3mm,
  text height=8pt,text depth=2pt]

  \begin{scope}[start chain,
                every node/.style={on chain},
                terminal/.append style={join=by {->,shorten >=1pt}},
                nonterminal/.append style={join=by {->,shorten >=1pt}},
                support/.style={coordinate,join}]
    \node [support]             (start)        {};
    \node [nonterminal]                        {unsigned integer};
    \node [support]             (after ui)     {};
    \node [terminal]                           {.};
    \node [support]             (after dot)    {};
    \node [terminal]                           {digit};
    \node [support]             (after digit)  {};
    \node [support]             (skip)         {};    
    \node [support]             (before E)     {};
    \node [terminal]                           {E};
    \node [support]             (after E)      {};
    \node [support,xshift=5mm]  (between)      {};
    \node [support,xshift=5mm]  (before last)  {};
    \node [nonterminal]                        {unsigned integer};
    \node [support]             (after last)   {};
    \node [join=by ->]          (end)          {};
  \end{scope}
  \node (plus)  [terminal,above=of between] {+};
  \node (minus) [terminal,below=of between] {-};

  \begin{scope}[->,shorten >=1pt,decoration={post length=4pt},rounded corners]
    \draw (after ui)    -- +(0,.7)  -| (skip);
    \draw (after digit) -- +(0,-.7) -| (after dot);
    \draw (before E)    -- +(0,-1.2) -| (after last);
    \draw (after E)     |- (plus);
    \draw (plus)        -| (before last);
    \draw (after E)     |- (minus);
    \draw (minus)       -| (before last);
  \end{scope}
\end{tikzpicture}
\bigskip


Having read the previous tutorials, Ilka knows already how to setup
the environment for her graphic, namely using a |tikzpicture|
environment. She wonders which libraries she will need. She decides
that she will postpone the decision and add the necessary libraries as
needed as she constructs the picture.


\subsection{Styling the Nodes}

The bulk of this tutorial will be about arranging the nodes and
connecting them using chains, but let us start with setting up styles
for the nodes.

There are two kinds of nodes, namely what theoreticians like to call
\emph{terminals} and \emph{nonterminals}. For the terminals, Ilka
decides to use a black color, which visually shows that ``nothing
needs to be done about them.'' The nonterminals, which still need to
be ``processed'' further, get a bit of red mixed in.

Ilka starts with the simpler nonterminals, as there are no rounded
corners involved. Naturally, she sets up a style:

\tikzset{
  nonterminal/.style={
      % The shape:
      rectangle,
      % The size:
      minimum size=6mm,
      % The border:
      very thick,           
      draw=red!50!black!50,         % 50% red and 50% black,
                                    % and that mixed with 50% white
      % The filling:
      top color=white,              % a shading that is white at the top...
      bottom color=red!50!black!20, % and something else at the bottom
      % Font
      font=\itshape                 
}}
\begin{codeexample}[]
\begin{tikzpicture}[
    nonterminal/.style={
      % The shape:
      rectangle,
      % The size:
      minimum size=6mm,
      % The border:
      very thick,           
      draw=red!50!black!50,         % 50% red and 50% black,
                                    % and that mixed with 50% white
      % The filling:
      top color=white,              % a shading that is white at the top...
      bottom color=red!50!black!20, % and something else at the bottom
      % Font
      font=\itshape                 
    }]
  \node [nonterminal] {unsigned integer};
\end{tikzpicture}
\end{codeexample}
Ilka is pretty proud of the use of the |minimum size| option: As the
name suggests, this option ensures that the node is at least 6mm by
6mm, but it will expand in size as necessary to accommodate longer
text. By giving this option to all nodes, they will all have the same
height of 6mm.

Styling the terminals is a bit more difficult because of the round
corners. Ilka has several options how she can achieve them. Once way
is to use the |rounded corners| option. It gets a dimension as
parameter and causes all corners to be replaced by little arcs with
the given dimension as radius. By setting the radius to 3mm, she will
get exactly what she needs: circles, when the shapes are, indeed,
exactly 6mm by 6mm and otherwise half circles on the sides:

\begin{codeexample}[]
\begin{tikzpicture}[node distance=5mm,
                    terminal/.style={
                      % The shape:
                      rectangle,minimum size=6mm,rounded corners=3mm,
                      % The rest
                      very thick,draw=black!50,
                      top color=white,bottom color=black!20,
                      font=\ttfamily}]
  \node (dot)   [terminal]                {.};
  \node (digit) [terminal,right=of dot]   {digit};
  \node (E)     [terminal,right=of digit] {E};
\end{tikzpicture}
\end{codeexample}

Another option is to use a shape that is specially made for
typesetting rectangles with arc on the sides (she has to use the
|shapes.misc| library to use it). This shape gives Ilka
much more control over the appearance. For instance, she could have an
arc only on the left side, but she will not need this.
\begin{codeexample}[]
\begin{tikzpicture}[node distance=5mm,
                    terminal/.style={
                      % The shape:
                      rounded rectangle,
                      minimum size=6mm,
                      % The rest
                      very thick,draw=black!50,
                      top color=white,bottom color=black!20,
                      font=\ttfamily}]
  \node (dot)   [terminal]                {.};
  \node (digit) [terminal,right=of dot]   {digit};
  \node (E)     [terminal,right=of digit] {E};
\end{tikzpicture}
\end{codeexample}
Either method seems fine to Ilka.

At this point, she notices a problem. The baseline of the text in the
nodes is not aligned:
\tikzset{terminal/.style={
                      % The shape:
                      rounded rectangle,
                      minimum size=6mm,
                      % The rest
                      very thick,draw=black!50,
                      top color=white,bottom color=black!20,
                      font=\ttfamily}}
\begin{codeexample}[]
\begin{tikzpicture}[node distance=5mm]
  \node (dot)   [terminal]                {.};
  \node (digit) [terminal,right=of dot]   {digit};
  \node (E)     [terminal,right=of digit] {E};

  \draw [help lines] let \p1 = (dot.base),
                         \p2 = (digit.base),
                         \p3 = (E.base)
                     in (-.5,\y1) -- (3.5,\y1)
                        (-.5,\y2) -- (3.5,\y2)
                        (-.5,\y3) -- (3.5,\y3);                     
\end{tikzpicture}
\end{codeexample}
(Ilka has moved the style definition to the preamble by
saying |\tikzset{terminal/.style=...}|, so that she can use it in all
pictures.)

For the |digit| and the |E| the difference in the baselines is almost
imperceptible, but for the dot the problem is quite severe: It looks
more like a multiplication dot than a period.

Ilka toys with the idea of using the |base right=of...| option rather than
the |right=of...| to align the nodes in such a way that the baselines
are all on the same line (the |base right| option places a node
right of something so that the baseline is right of the baseline of
the other object). However, this does not have the desired effect:
\begin{codeexample}[]
\begin{tikzpicture}[node distance=5mm]
  \node (dot)   [terminal]                {.};
  \node (digit) [terminal,base right=of dot]   {digit};
  \node (E)     [terminal,base right=of digit] {E};
\end{tikzpicture}
\end{codeexample}
The nodes suddenly ``dance around''! There is no hope of changing the
position of text inside a node using anchors. Instead, Ilka must use a
trick: The problem of mismatching baselines is caused by the fact that
|.| and |digit| and |E| all have different heights and depth. If they
all had the same, they would all be positioned vertically in the same
manner. So, all Ilka needs to do is to use the |text height| and
|text depth| options to explicitly specify a height and depth for the
nodes.
\begin{codeexample}[]
\begin{tikzpicture}[node distance=5mm,
                    text height=1.5ex,text depth=.25ex]
  \node (dot)   [terminal]                {.};
  \node (digit) [terminal,right=of dot]   {digit};
  \node (E)     [terminal,right=of digit] {E};
\end{tikzpicture}
\end{codeexample}



\subsection{Aligning  the Nodes Using Positioning Options}

Ilka now has the ``styling'' of the nodes ready. The next problem is
to place them in the right places. There are several ways to do
this. The most straightforward is to simply explicitly place the nodes
at certain coordinates ``calculated by hand.'' For very simple
graphics this is perfectly alright, but it has several disadvantages:
\begin{enumerate}
\item For more difficult graphics, the calculation may become
  complicated.
\item Changing the next of the nodes may make it necessary to
  recalculate the coordinates.
\item The source code of the graphic is not very clear since the
  relationships between the positions of the nodes are not made
  explicit. 
\end{enumerate}

For these reasons, Ilka decides to try out different ways of arranging
the nodes on the page.

The first method is the use of \emph{positioning options}. To use
them, you need to load the |positioning| library. This gives you
access to advanced implementations of options like |above| or |left|,
since you can now say |above=of some node| in order to place a node
above of |some node|, with the borders separated by |node distance|.

Ilka can use this to draw the place the nodes in a long row:
\tikzset{terminal/.append style={text height=1.5ex,text depth=.25ex}}
\tikzset{nonterminal/.append style={text height=1.5ex,text
    depth=.25ex}}
\begin{codeexample}[]
\begin{tikzpicture}[node distance=5mm and 5mm]
  \node (ui1)   [nonterminal]                     {unsigned integer};
  \node (dot)   [terminal,right=of ui1]           {.};
  \node (digit) [terminal,right=of dot]           {digit};
  \node (E)     [terminal,right=of digit]         {E};
  \node (plus)  [terminal,above right=of E]       {+};
  \node (minus) [terminal,below right=of E]       {-};
  \node (ui2)   [nonterminal,below right=of plus] {unsigned integer};
\end{tikzpicture}
\end{codeexample}

For the plus and minus nodes, Ilka is a bit startled by their
placements. Shouldn't they be more to the right? The reason they are
placed in that manner is the following: The |north east| anchor of the
|E| node lies at the ``upper start of the right arc,'' which, a bit
unfortunately in this case, happens to be the top of the
node. Likewise, the |south west| anchor of the |+| node is actually at
its bottom and, indeed, the horizontal and vertical distances between
the top of the |E| node and the bottom of the |+| node are both 5mm.

There are several ways of fixing this problem (and with matrices,
described in a moment, this problem will completely disappear). The
easiest way is to simply add a little bit of horizontal shift by hand:
\begin{codeexample}[]
\begin{tikzpicture}[node distance=5mm and 5mm]
  \node (E)     [terminal]                                   {E};
  \node (plus)  [terminal,above right=of E,xshift=5mm]       {+};
  \node (minus) [terminal,below right=of E,xshift=5mm]       {-};
  \node (ui2)   [nonterminal,below right=of plus,xshift=5mm] {unsigned integer};
\end{tikzpicture}
\end{codeexample}

Now that the nodes have been placed, Ilka needs to add
connections. Here, some connections are more difficult than
other. Consider for instance the ``repeat'' line around the
|digit|. One way of describing this line is to say ``it starts a
little to the right of |digit| than goes down and then goes to the
left and finally ends at a point a little to the right of |digit|.''
Ilka can put this into code as follows:
\begin{codeexample}[]
\begin{tikzpicture}[node distance=5mm and 5mm]
  \node (dot)   [terminal]                        {.};
  \node (digit) [terminal,right=of dot]           {digit};
  \node (E)     [terminal,right=of digit]         {E};

  \path (dot)   edge[->] (digit)  % simple edges
        (digit) edge[->] (E);

  \draw [->]
     % start right of digit.east, that is, at the point that is the
     % linear combination of digit.east and the vector (2mm,0pt). We
     % use the ($ ... $) notation for computing linear combinations
     ($ (digit.east) + (2mm,0) $)  
     % Now go down
     -- ++(0,-.5)
     % And back to the left of digit.west
     -| ($ (digit.west) - (2mm,0) $);
\end{tikzpicture}
\end{codeexample}

Since Ilka needs this ``go up/down then horizontally and than up/down
to a target'' several times, it seems sensible to define a special
\emph{to-path} for this. Whenever the |edge| command is used, it
simply adds the current value of |to path| to the path. So, Ilka can
setup a style that contains the correct path:
\begin{codeexample}[]
\begin{tikzpicture}[node distance=5mm and 5mm,
    skip loop/.style={to path={-- ++(0,-.5) -| (\tikztotarget)}}]
  \node (dot)   [terminal]                        {.};
  \node (digit) [terminal,right=of dot]           {digit};
  \node (E)     [terminal,right=of digit]         {E};

  \path (dot)   edge[->]           (digit)  % simple edges
        (digit) edge[->]           (E)
        ($ (digit.east) + (2mm,0) $)
                edge[->,skip loop] ($ (digit.west) - (2mm,0) $);
\end{tikzpicture}
\end{codeexample}

Ilka can even go a step further and make her |skip look| style
parametrized. For this, the skip loop's vertical offset is passed as
parameter |#1|. Also, in the following code Ilka specifies the start
and targets differently, namely as the positions that are ``in the
middle between the nodes.''
\begin{codeexample}[]
\begin{tikzpicture}[node distance=5mm and 5mm,
    skip loop/.style={to path={-- ++(0,#1) -| (\tikztotarget)}}]
  \node (dot)   [terminal]                        {.};
  \node (digit) [terminal,right=of dot]           {digit};
  \node (E)     [terminal,right=of digit]         {E};

  \path (dot)   edge[->]                (digit)  % simple edges
        (digit) edge[->]                (E)
        ($ (digit.east)!.5!(E.west) $)
                edge[->,skip loop=-5mm] ($ (digit.west)!.5!(dot.east) $);
\end{tikzpicture}
\end{codeexample}


\subsection{Aligning  the Nodes Using Matrices}

Ilka is still bothered a bit by the placement of the plus and minus
nodes. Somehow, having to add an explicit |xshift| seems too much like
cheating.

A perhaps better way of positioning the nodes is to use a
\emph{matrix}. In \tikzname\ matrices can be used to align quite
arbitrary graphical objects in rows and columns. The syntax is very
similar to the use of arrays and tables in \TeX\ (indeed, internally
\TeX\ tables are used, but a lot of stuff is going on additionally).

In Ilka's graphic, there will be three rows: One row containing only
the plus node, one row containing the main nodes and one row
containing only the minus node.
\begin{codeexample}[]
\begin{tikzpicture}
  \matrix[row sep=1mm,column sep=5mm] {
    % First row:
      & & & & \node [terminal] {+}; & \\
    % Second row:
    \node [nonterminal] {unsigned integer}; &
    \node [terminal]    {.};                &
    \node [terminal]    {digit};            &
    \node [terminal]    {E};                &
                                            &
    \node [nonterminal] {unsigned integer}; \\
    % Third row:
      & & & & \node [terminal] {-}; & \\    
  };
\end{tikzpicture}
\end{codeexample}
That was easy! By toying around with the row and columns separations,
Ilka can achieve all sorts of pleasing arrangements of the nodes.

Ilka now faces the same connecting problem as before. This time, she
has an idea: She adds small nodes (they will be turned into
coordinates later on and be invisible) at all the places
where she would like connections to start and end.
\begin{codeexample}[]
\begin{tikzpicture}[point/.style={circle,inner sep=0pt,minimum size=2pt,fill=red},
                   skip loop/.style={to path={-- ++(0,#1) -| (\tikztotarget)}}]
  \matrix[row sep=1mm,column sep=2mm] {
    % First row:
    & & & & & & &  & & & & \node [terminal] {+};\\
    % Second row:
    \node (p1) [point]  {};                 &
    \node [nonterminal] {unsigned integer}; &
    \node (p2) [point]  {};                 &
    \node [terminal]    {.};                &
    \node (p3) [point]  {};                 &
    \node [terminal]    {digit};            &
    \node (p4) [point]  {};                 &
    \node (p5) [point]  {};                 &
    \node (p6) [point]  {};                 &
    \node [terminal]    {E};                &
    \node (p7) [point]  {};                 &
                                            &
    \node (p8) [point]  {};                 &
    \node [nonterminal] {unsigned integer}; &
    \node (p9) [point]  {};                 \\
    % Third row:
    & & & & & & &  & & & & \node [terminal] {-};\\
  };

  \path (p4) edge [->,skip loop=-5mm] (p3)
        (p2) edge [->,skip loop=5mm]  (p6);
\end{tikzpicture}
\end{codeexample}
Now, its only a small step to add all the missing edges.



\subsection{Connecting the Nodes using Chains}

Matrices allow Ilka to align the nodes nicely, but the connections are
not quite perfect. The problem is that the code does not really
reflect the paths that underlie the diagram.

For this reason, Ilka decides to try out \emph{chains} by including
the |chain| library. Basically, a chain is just a sequence of
(usually) connected nodes. The nodes can already have been constructed
or they can be constructed as the chain is constructed (or these
processes can be mixed).

Ilka starts with creating a chain from scratch. For this, she starts a
chain using the |start chain| option in a scope. Then, inside the
scope, she uses the |on chain| option on nodes to add them to the
chain.
\begin{codeexample}[]
\begin{tikzpicture}[start chain,node distance=5mm]
  \node [on chain,nonterminal]  {unsigned integer};
  \node [on chain,terminal]     {.};
  \node [on chain,terminal]     {digit};
  \node [on chain,terminal]     {E};
  \node [on chain,nonterminal]  {unsigned integer};
\end{tikzpicture}
\end{codeexample}
(Ilka will add the plus and minus nodes later.)

As can be seen, the nodes of a chain are placed in a row. This can be
changed, for instance by saying |start chain=going below| we get a
chain where each node is below the previous one.

The next step is to \emph{join} the nodes of the chain. For this, we
add the |join| option to each node. This joins the node with the
previous node (for the first node nothing happens).
\begin{codeexample}[]
\begin{tikzpicture}[start chain,node distance=5mm]
  \node [on chain,join,nonterminal]  {unsigned integer};
  \node [on chain,join,terminal]     {.};
  \node [on chain,join,terminal]     {digit};
  \node [on chain,join,terminal]     {E};
  \node [on chain,join,nonterminal]  {unsigned integer};
\end{tikzpicture}
\end{codeexample}
In order to get a arrow tip, we redefine the |every join| style. Also,
we move the |join| option to the |every on chain|
style so that we do not have to repeat them so often.
\begin{codeexample}[]
\begin{tikzpicture}[start chain,node distance=5mm, every on chain/.style={join}, every join/.style={->}]
  \node [on chain,nonterminal]  {unsigned integer};
  \node [on chain,terminal]     {.};
  \node [on chain,terminal]     {digit};
  \node [on chain,terminal]     {E};
  \node [on chain,nonterminal]  {unsigned integer};
\end{tikzpicture}
\end{codeexample}

It is now time to add the plus and minus signs. They obviously
\emph{branch off} the main chain. For this reason, we start a branch
for them using the |start branch| option.
\begin{codeexample}[]
\begin{tikzpicture}[start chain,node distance=5mm, every on chain/.style={join}, every join/.style={->}]
  \node [on chain,nonterminal]  {unsigned integer};
  \node [on chain,terminal]     {.};
  \node [on chain,terminal]     {digit};
  \node [on chain,terminal]     {E};
  \begin{scope}[start branch=plus]
    \node (plus)  [terminal,on chain=going above right] {+};
  \end{scope}
  \begin{scope}[start branch=minus]
    \node (minus) [terminal,on chain=going below right] {-};
  \end{scope}
  \node [nonterminal,on chain,join=with plus,join=with minus]  {unsigned integer};
\end{tikzpicture}
\end{codeexample}

Let us see, what is going on here. First, the |start branch| begins a
branch, starting at the |E| node. This is implicitly also the first
node on this branch. A branch is nothing different from a chain, which
is why the plus node is put on this branch using the |on chain|
option. However, this time we specify the placement of the node
explicitly using |going |\meta{direction}. This causes the plus sign
to be placed above and right of the |E| node. It is automatically
joined to its predecessor on the branch by the implicit |join|
option.

When the first branch ends, only the plus node has been added and the
current chain is the original chain once more and we are back to the
|E| node. Now we start a new branch for the minus node. After this
branch, the current branch is the |E| node once more.

Finally, the rightmost unsigned integer is added to the (main) chain,
which is why it is joined correctly with the |E| node. The two
additional |join| options get a special |with| parameter. This allows
you to join a node with a node other than the predecessor on the
chain. The  |with| should be followed by the name of a node.

Since Ilka will need scopes more often in the following, she includes
the |scopes| library. This allows her to replace |\begin{scope}|
  simply by an opening brace and  |\end{scope}| by the corresponding
closing brace. Also, in the following example we reference
the nodes |plus| and |minus| using
their automatic name: The $i$th node on a chain is called
|chain-|\meta{i}. For a branch \meta{branch}, the $i$th node is called
|chain/|meta{branch}|-|\meta{i}. The \meta{i} can be replaced by
|begin| and |end| to reference the first and (currently) last node on
the chain.

\begin{codeexample}[]
\begin{tikzpicture}[start chain,node distance=5mm, every on chain/.style={join}, every join/.style={->}]
  \node [on chain,nonterminal]  {unsigned integer};
  \node [on chain,terminal]     {.};
  \node [on chain,terminal]     {digit};
  \node [on chain,terminal]     {E};
  { [start branch=plus]
    \node (plus)  [terminal,on chain=going above right] {+};
  }
  { [start branch=minus]
    \node (minus) [terminal,on chain=going below right] {-};
  }
  \node [nonterminal,on chain,join=with chain/plus-end,join=with chain/minus-end]  {unsigned integer};
\end{tikzpicture}
\end{codeexample}

The next step is to add intermediate coordinate nodes in the same
manner as Ilka did for the matrix. For them, we change the |join|
style slightly, namely for these nodes we do not want an arrow
tip. This can be achieved either by (locally) changing the
|every join| style or, which is what is done in the below example, by
giving the desired style using |join=by ...|, where |...| is the style
to be used for the join.

\begin{codeexample}[]
\begin{tikzpicture}[start chain,node distance=5mm and 2mm,
                    every node/.style={on chain},    
                    nonterminal/.append style={join=by ->},    
                    terminal/.append style={join=by ->},    
                    point/.style={join=by -,circle,fill=red,minimum size=2pt,inner sep=0pt}]
  \node [point]            {};
  \node [nonterminal]      {unsigned integer};
  \node [point]            {};
  \node [terminal]         {.};
  \node [point]            {};
  \node [terminal]         {digit};
  \node [point]            {};
  \node [point]            {};
  \node [point]            {};
  \node [terminal]         {E};
  \node [point]            {};
  { [start branch=plus]
    \node (plus)  [terminal,yshift=7mm] {+};
  }
  { [start branch=minus]
    \node (minus) [terminal,yshift=-7mm] {-};
  }
  \node [point]            {};
  \node [point,join=with chain/plus-end by ->,join=with chain/minus-end by ->] {};
  \node [nonterminal]      {unsigned integer};
\end{tikzpicture}
\end{codeexample}

Still missing...

\subsection{Chains and Matrices}

Still missing...

%\end{document}



% The titlepage

{
  \parindent0pt
  \null
  \colorlet{mintgreen}{green!50!black!50}
  
  \thispagestyle{empty}
  \vskip3cm
  \vfill
  \hfil
  \begin{tikzpicture}[overlay]
    \coordinate (front) at (0,0);
    \coordinate (horizon) at (0,.31\paperheight);
    \coordinate (bottom) at (0,-.6\paperheight);
    \coordinate (sky) at (0,.57\paperheight);
    \coordinate (left) at (-.51\paperwidth,0);
    \coordinate (right) at (.51\paperwidth,0);
    
    \shade [bottom color=blue!30!black!10,top color=blue!30!black!50]
      ([yshift=-5mm]horizon -|  left) rectangle (sky -| right);
    \shade [bottom color=black!70!green!25,top color=black!70!green!10]
      (front -| left) -- (horizon -| left)
      decorate [decoration=random steps] { -- (horizon -| right) }
      -- (front -| right) -- cycle;
    \shade [top color=black!70!green!25,bottom color=black!25]
      ([yshift=-5mm-1pt]front -| left) rectangle ([yshift=1pt]front -| right);
    \fill [black!25] (bottom -| left) rectangle ([yshift=-5mm]front -| right);

    \def\nodeshadowed[#1]#2;{\node[scale=2,above,#1]{#2};\node[scale=2,above,#1,yscale=-1,scope fading=south,opacity=0.4]{#2};}

    \nodeshadowed [at={(-5,5  )},yslant=0.05] {\Huge Ti\textcolor{orange}{\emph{k}}Z};
    \nodeshadowed [at={( 0,5.3)}] {\huge \textcolor{mintgreen}{\&}};
    \nodeshadowed [at={( 5,5  )},yslant=-0.05] {\Huge \textsc{PGF}};
    \nodeshadowed [at={( 0,2  )}] {Manual for Version \pgftypesetversion};

    \foreach \i in {0.5,0.6,...,2}
      \fill [white,decoration=Koch snowflake,opacity=.9]
            [shift=(horizon),shift={(rand*11,rnd*7)},scale=\i]
            [double copy shadow={opacity=0.2,shadow xshift=0pt,shadow
              yshift=3*\i pt,fill=white,draw=none}]
        decorate { 
          decorate { 
            decorate {
              (0,0) -- ++(60:1) -- ++(-60:1) -- cycle
            }
          }
        };

  \node (left text) [text width=.5\paperwidth-2cm,below right,at={(-.5\paperwidth+1cm,-1.5cm)}]
  {
    \fontfamily{pcr}
    \def\textbraceleft{\char`\{}
    \def\textbraceright{\char`\}}
    \def\textbackslash{\char`\\}
    \begin{lstlisting}[basicstyle=\scriptsize\color{black},
                       keywordstyle=\bfseries\color{white},
                       identifierstyle=\bfseries\color{black},
                       keywords={tikzpicture,shade,fill,draw,path,node},
                       literate={-}{{-}}1]
\begin{tikzpicture}
  \coordinate (front) at (0,0);
  \coordinate (horizon) at (0,.31\paperheight);
  \coordinate (bottom) at (0,-.6\paperheight);
  \coordinate (sky) at (0,.57\paperheight);
  \coordinate (left) at (-.51\paperwidth,0);
  \coordinate (right) at (.51\paperwidth,0);
    
  \shade [bottom color=white,
          top color=blue!30!black!50]
              ([yshift=-5mm]horizon -|  left)
    rectangle (sky -| right);
  
  \shade [bottom color=black!70!green!25,
          top color=black!70!green!10]
    (front -| left) -- (horizon -| left)
    decorate [decoration={name={random steps}] {
      -- (horizon -| right)  }
    -- (front -| right) -- cycle;
    
  \shade [top color=black!70!green!25,
         bottom color=black!25]
              ([yshift=-5mm-1pt]front -| left)
    rectangle ([yshift=1pt]front -| right);

  \fill [black!25]
              (bottom -| left)
    rectangle ([yshift=-5mm]front -| right);
 
  \def\nodeshadowed[#1]#2;{
    \node[scale=2,above,#1]{#2};
    \node[scale=2,above,#1,yscale=-1,
          scope fading=south,opacity=0.4]{#2};
  }
\end{lstlisting}
};

  \node (right text) [text width=.5\paperwidth-2cm,below right,at={(1cm,-1.5cm)}]
  {
    \fontfamily{pcr}
    \def\textbraceleft{\char`\{}
    \def\textbraceright{\char`\}}
    \def\textbackslash{\char`\\}
    \begin{lstlisting}[basicstyle=\scriptsize\color{black},
                       keywordstyle=\bfseries\color{white},
                       identifierstyle=\bfseries\color{black},
                       keywords={tikzpicture,shade,fill,draw,path,node},
                       literate={-}{{-}}1]
  \nodeshadowed [at={(-5,8  )},yslant=0.05]
    {\Huge Ti\textcolor{orange}{\emph{k}}Z};
  \nodeshadowed [at={( 0,8.3)}]
    {\huge \textcolor{green!50!black!50}{\&}};
  \nodeshadowed [at={( 5,8  )},yslant=-0.05]
    {\Huge \textsc{PGF}};
  \nodeshadowed [at={( 0,5  )}]
    {Manual for Version \pgftypesetversion};

  \foreach \i in {0.5,0.6,...,2}
    \fill
      [white,opacity=\i/2,
       decoration={name=Koch snowflake},
       shift=(horizon),shift={(rand*11,rnd*7)},
       scale=\i,double copy shadow={
         opacity=0.2,shadow xshift=0pt,
         shadow yshift=3*\i pt,
         fill=white,draw=none}]
      decorate {
        decorate { 
          decorate { 
            (0,0)- ++(60:1) -- ++(-60:1) -- cycle
          } } };

   \node (left text) ...
   \node (right text) ...

   \fill [decorate,
          decoration={footprints,foot of=gnome},
          opacity=.5,brown] (left text.south)
     to [out=-45,in=135]    (right text.north);
   \fill [decorate,
     decoration={footprints,foot of=felis silvestris,
       foot length=5pt,stride length=15pt,foot angle=0},
     opacity=.5,green!50!black] (left text.south)
    to [out=20,in=180] (right text.north west);
\end{tikzpicture}
  \end{lstlisting}
  };

  \fill [decorate,decoration=footprints,
         decoration={footprints,foot of=gnome},
         opacity=.5,brown] (left text.south)
    to [out=-45,in=135]    (right text.north);
  \fill [decorate,decoration={footprints,foot length=5pt,foot of=felis
    silvestris,stride  length=15pt,foot angle=0},
     opacity=.5,green!50!black] (left text.south)
    to [out=20,in=180] (right text.north west);
\end{tikzpicture}
\vfill
\vbox{}
\clearpage
}

{
  \vbox{}
  \vskip0pt plus 1fill
  F�r meinen Vater, damit er noch viele sch�ne \TeX-Graphiken
  erschaffen kann.
  \vskip1em
  \hfill\emph{Till}
  \vskip0pt plus 3fill

  \parindent=0pt
  Copyright 2007 by Till Tantau

  \medskip  
  Permission is granted to copy, distribute and/or modify \emph{the documentation}
  under the terms of the \textsc{gnu} Free Documentation License, Version 1.2
  or any later version published by the Free Software Foundation;
  with no Invariant Sections, no Front-Cover Texts, and no Back-Cover Texts.
  A copy of the license is included in the section entitled \textsc{gnu}
  Free Documentation License.

  \medskip  
  Permission is granted to copy, distribute and/or modify \emph{the
    code of the package} under the terms of the \textsc{gnu} Public License, Version 2
  or any later version published by the Free Software Foundation.
  A copy of the license is included in the section entitled \textsc{gnu}
  Public License.

  \medskip  
  Permission is also granted to distribute and/or modify \emph{both
    the documentation and the code} under the conditions of the LaTeX
  Project Public License, either version 1.3 of this license or (at
  your option) any later version. A copy of the license is included in
  the section entitled \LaTeX\ Project Public License. 

  \vbox{}
  \clearpage
}


\title{\bfseries The \tikzname\ and {\Large PGF} Packages\\
  \large Manual for version \pgfversion\\[1mm]
\large\href{http://sourceforge.net/projects/pgf}{\texttt{http://sourceforge.net/projects/pgf}}}
\author{Till Tantau\footnote{Editor of this documentation. Parts of
    this documentation have been written by other authors as indicated
    in these parts or chapters and in Section~\ref{section-authors}.}\\
  \normalsize Institut f\"ur Theoretische Informatik\\[-1mm]
  \normalsize Universit\"at zu L\"ubeck}

\maketitle

\tableofcontents

\clearpage


% Copyright 2006 by Till Tantau
%
% This file may be distributed and/or modified
%
% 1. under the LaTeX Project Public License and/or
% 2. under the GNU Free Documentation License.
%
% See the file doc/generic/pgf/licenses/LICENSE for more details.


\section{Introduction}

The \pgfname\ package, where ``\pgfname'' is supposed to mean ``portable
graphics format'' (or ``pretty, good, functional'' if you
prefer\dots), is a package for creating graphics in an ``inline''  
manner. It defines a number of \TeX\ commands that draw
graphics. For example, the code |\tikz \draw (0pt,0pt) -- (20pt,6pt);|
yields the line \tikz \draw (0pt,0pt) -- (20pt,6pt); and the code
|\tikz \fill[orange] (1ex,1ex) circle (1ex);| yields \tikz
\fill[orange] (1ex,1ex) circle (1ex);.

In a sense, when you use \pgfname\ you ``program'' your graphics, just
as you ``program'' your document when you use \TeX.  You get all  
the advantages of the ``\TeX-approach to typesetting'' for your 
graphics: quick creation of simple graphics, precise positioning, the
use of macros, often superior typography. You also inherit all the
disadvantages: steep learning curve, no \textsc{wysiwyg}, small
changes require a long recompilation time, and the code does not
really ``show'' how things will look like. 



\subsection{Structure of the System}

The \pgfname\ system consists of different layers:

\begin{description}
\item[System layer:] This layer provides a complete abstraction of what is
  going on ``in the driver.'' The driver is a program like |dvips| or
  |dvipdfm| that takes a |.dvi| file as input and generates a |.ps| or
  a |.pdf| file. (The |pdftex| program also counts as a driver, even
  though it does not take a |.dvi| file as input. Never mind.) Each
  driver has its own syntax for the generation of graphics, causing
  headaches to everyone who wants to create graphics in a portable
  way. \pgfname's system layer ``abstracts away'' these
  differences. For example, the system command
  |\pgfsys@lineto{10pt}{10pt}| extends the current path  to the coordinate
  $(10\mathrm{pt},10\mathrm{pt})$ of the current
  |{pgfpicture}|. Depending on whether |dvips|, 
  |dvipdfm|, or |pdftex| is used to process the document, the system
  command will be converted to different |\special| commands.
  The system layer is as ``minimalistic'' as possible since each
  additional command makes it more work to port \pgfname\ to a new
  driver.

  As a user, you will not use the system layer directly.
\item[Basic layer:]
  The basic layer provides a set of basic commands that allow
  you to produce complex graphics in a much easier manner than by using
  the system layer directly. For example,  the system layer provides
  no commands for creating circles since circles can be composed from
  the more basic B�zier curves (well, almost). However, as a user you
  will want to have a simple command to create circles
  (at least I do) instead of having to write down half a page of
  B�zier  curve  support coordinates. Thus, the basic layer provides a
  command |\pgfpathcircle| that generates the necessary curve
  coordinates for you.

  The basic layer is consists of a \emph{core}, which consists of
  several interdependent packages that can only be loaded \emph{en
    bloc,} and additional packages that extend the core by more
  special-purpose commands like node management or a plotting
  interface. For instance, the \textsc{beamer} package uses the core, 
  but not all of the additional packages of the basic layer.
\item[Frontend layer:]
  A frontend (of which there can be several) is a set of commands
  or a special syntax that makes using the basic layer easier. A
  problem with directly using the basic layer is that code written for
  this layer is often too ``verbose.'' For example, to draw a simple
  triangle, you may need as many as five commands when using the basic
  layer: One for beginning a path at the first corner of the triangle,
  one for extending the path to the second corner, one for going to
  the third, one for closing the path, and one for actually painting
  the triangle (as opposed to filling it). With the |tikz| frontend
  all this boils down to a single simple \textsc{metafont}-like
  command: 
\begin{verbatim}
\draw (0,0) -- (1,0) -- (1,1) -- cycle;
\end{verbatim}

  There are different frontends:
  \begin{itemize}
  \item
    The \tikzname\ frontend is the ``natural'' frontend for \pgfname. It gives
    you access to all features of \pgfname, but it is intended to be
    easy to use. The syntax is a mixture of \textsc{metafont} and
    \textsc{pstricks} and some ideas of myself. This frontend is
    \emph{neither} a complete \textsc{metafont} compatibility layer nor
    a \textsc{pstricks} compatibility layer and it is not intended to
    become either. 
  \item
    The |pgfpict2e| frontend reimplements the standard \LaTeX\
    |{picture}|  environment and commands like |\line| or |\vector|
    using the \pgfname\ basic layer. This layer is not really ``necessary''
    since the |pict2e.sty| package does at least as good a job at
    reimplementing the |{picture}| environment. Rather, the idea
    behind this package is to have a simple demonstration of how a
    frontend can be implemented.
  \end{itemize}

  It would be possible to implement a |pgftricks| frontend that maps
  \textsc{pstricks} commands to \pgfname\ commands. However, I have not
  done this and even if fully implemented, many things that work in
  \pstricks\ will not work, namely whenever some \pstricks\ command
  relies too heavily on PostScript trickery. Nevertheless, such a
  package might be useful in some situations.
\end{description}

As a user of \pgfname\ you will use the commands of a
frontend plus perhaps some commands of the basic layer. For this
reason, this manual explains the frontends first, then the basic
layer, and finally the system layer.



\subsection{Comparison with Other Graphics Packages}

\pgfname\ is not the only graphics package for \TeX. In the following,
I try to give a reasonably fair comparison of the \pgfname-system and
other packages.
\begin{enumerate}
\item
  The standard \LaTeX\ |{picture}| environment allows you to create
  simple graphics, but little more. This is certainly not
  due to a lack of knowledge or imagination on the part of
  \LaTeX's designer(s). Rather, this is the price paid for the
  |{picture}| environment's portability: It works together with all
  backend drivers.
\item
  The |pstricks| package is certainly powerful enough to create
  any conceivable kind of graphic, but it is not portable at all. Most
  importantly, it does not work with |pdftex| nor with any other
  driver that produces anything but PostScript code.

  Compared to \pgfname, |pstricks| has a broader support base. There
  are many nice extra packages for special purpose sitations that have
  been contributed by users over the last decade.

  The \tikzname\ syntax is more consistent than the |pstricks| syntax
  as \tikzname\ was developed ``in a more centralized manner'' and
  also ``with the shortcomings on |pstricks| in mind.''

  Note that a number of neat tricks that |pstricks| can do are
  impossible in \pgfname. In particular, |pstricks| has access to the
  powerful PostScript programming language, which allows trickery
  such as inline function plotting.
\item
  The |xypic| package is an older package for creating
  graphics. However, it is more difficult to use and to learn because
  the syntax and the documentation are a bit cryptic. 
\item
  The |dratex| package is a small graphic package for creating a
  graphics. Compared to the other package, including \pgfname, it is
  very small, which may or may not be an advantage. 
\item
  The |metapost| program is a very powerful alternative to
  \pgfname. However, it is an external program, which entails a
  bunch of problems. The time needed both to create a small graphic 
  and also to compile it is much greater than in \pgfname.
  The main problem with |metapost|, however, is the inclusion of
  labels. This is \emph{much} easier to achieve using \pgfname.
\item
  The |xfig| program is an important alternative to \tikzname\ for
  users who do not wish to ``program'' their graphics as is necessary
  with \tikzname\ and the other packages above. Their is a conversion
  program that will convert |xfig| graphics to both \tikzname\ and
  for \pgfname, but it is still under construction.
\end{enumerate}




\subsection{Utility Packages}

The \pgfname\ package comes along with a numer of utility package that
are not really about creating graphics and which can be used
independently of \pgfname. However, they are bundled with \pgfname,
partly out of convenience, partly because their functionality is
closely intertwined with \pgfname. These utility packages are: 
\begin{enumerate}
\item The |pgfkeys| package defines a powerful key management
  facility. It can be used completely independently of \pgfname.
\item The |pgffor| package defines a useful |\foreach| statement.
\item The |pgfcalendar| package defines macros for creating
  calendars. Typically, these calendars will be rendered using
  \pgfname's graphic engine, but you can use |pgfcalendar| also
  typeset calendars using normal text. The package also defines
  commands for ``working'' with dates.
\item The |pgfpages| package is used to assemble several pages into a
  single page. It provides commands for assembling several
  ``virtual pages'' into a single ``physical page.'' The idea is that
  whenever \TeX\ has a page ready for ``shipout,'' |pgfpages| interrupts
  this shipout and instead stores the page to be shipped out in a
  special box. When enough ``virtual pages'' have been accumulated in
  this way, they are scaled down and arranged on a ``physical page,''
  which then \emph{really} shipped out. This mechanism allows you to
  create ``two page on one page'' versions of a document directly inside
  \LaTeX\ without the use of any external programs. However,
  |pgfpages| can do quite a lot more than that. You can use it to put
  logos and watermark on pages, print up to 16 pages on one page, add
  borders to pages, and more. 
\end{enumerate}



\subsection{How to Read This Manual}

This manual describes both the design of the \pgfname\ system and
its usage. The organization is  very roughly according to
``user-friendliness.'' The commands and subpackages that are easiest
and most frequently used are described first, more low-level and
esoteric features are discussed later.

If you have not yet installed \pgfname, please read the installation
first. Second, it might be a good idea to read the tutorial. Finally,
you might wish to skim through the description of \tikzname. Typically, 
you will not need to read the sections on the basic layer. You will
only need to read the part on the system layer if you intend to write
your own frontend or if you wish to port \pgfname\ to a new driver.

The ``public'' commands and environments provided by the |pgf| package
are described throughout the text. In each such description, the
described command, environment or option is printed in red. Text shown
in green is optional and can be left out.


\subsection{Authors and Acknowledgements}
\label{section-authors}

The bulk of the \pgfname\ system and its documentation was written by
Till Tantau. The \pgfname\ mathematical engine was written and
documented by Mark Wibrow. Additionally, numerous people have
contributed to the \pgfname\ system by writing emails, spotting bugs,
or sending libraries. Many thanks to all these people, who are too
numerous to name them all!

\subsection{Getting Help}

When you need help with \pgfname\ and \tikzname, please do the
following:

\begin{enumerate}
\item
  Read the manual, at least the part that has to do with your problem.
\item
  If that does not solve the problem, try having a look at the
  sourceforge development page for \pgfname\ and \tikzname\ (see the
  title of this document). Perhaps someone has already reported a
  similar problem and someone has found a solution.
\item
  On the website you will find numerous forums for getting
  help. There, you can write to help forums, file bug reports, join
  mailing lists, and so on.
\item
  Before you file a bug report, especially a bug report concerning the
  installation, make sure that this is really a bug. In particular,
  have a look at the |.log| file that results when you \TeX\ your
  files. This |.log| file should show that all the right files are
  loaded from the right directories. Nearly all installation problems
  can be resolved by looking at the |.log| file.
\item
  \emph{As a last resort} you can try to email me (Till Tantau) or, if
  the problem concerns the mathematical engine, Mark Wibrow. I do
  not mind getting emails, I simply get way too many of them. Because
  of this, I cannot guarantee that your emails will be answered timely
  or even at all. Your chances that your problem will be fixed are
  somewhat higher if you mail to the \pgfname\ mailing list
  (naturally, I read this list and answer questions when I have the
  time).
\item
  Please, do not phone me in my office (unless, of course, you attend
  one of my lectures).
\end{enumerate}




\part{Tutorials and Guidelines}

{\Large \emph{by Till Tantau}}

\bigskip
\noindent
To help you get started with \tikzname, instead of a long installation
and configuration section, this manual starts with tutorials. They
explain all the basic and some of the more advanced features of the
system, without going into all the details. This part also contains
some guidelines on how you should proceed when creating graphics using
\tikzname. 

\vskip3cm

\begin{codeexample}[graphic=white,width=0pt]
\tikz \draw[thick,rounded corners=8pt]
  (0,0) -- (0,2) -- (1,3.25) -- (2,2) -- (2,0) -- (0,2) -- (2,2) -- (0,0) -- (2,0);
\end{codeexample}

\section{Tutorial: A Picture for Karl's Students}

This tutorial is intended for new users of \pgfname\ and \tikzname. It
does not give an exhaustive account of all the features of \tikzname\ or
\pgfname, just of those that you are likely to use right away.

Karl is a math and chemistry high-school teacher. He used to create
the graphics in his worksheets and exams using \LaTeX's |{picture}|
environment. While the results were acceptable, creating the graphics
often turned out to be a lengthy process. Also, there tended to be
problems with lines having slightly wrong angles and circles also
seemed to be hard to get right. Naturally, his students could not care
less whether the lines had the exact right angles and they find
Karl's exams too difficult no matter how nicely they were drawn. But
Karl was never entirely satisfied with the result.

Karl's son, who was even less satisfied with the results (he did not
have to take the exams, after all),  told Karl that he might wish
to try out a new package for creating graphics. A bit confusingly,
this package seems to have two names: First, Karl had to download and 
install a package called \pgfname. Then it turns out that inside this
package there is another package called \tikzname, which is supposed to
stand for ``\tikzname\ ist \emph{kein}  Zeichenprogramm.'' Karl finds this
all a bit strange and \tikzname\ seems to indicate that the package
does not do what he needs. However, having used \textsc{gnu}
software for quite some time and ``\textsc{gnu} not being Unix,''
there seems to be hope yet. His son assures him that \tikzname's name is
intended to warn people that \tikzname\ is not a program that you can
use to draw graphics with your mouse or tablet. Rather, it is more
like a ``graphics language.''


\subsection{Problem Statement}

Karl wants to put a graphic on the next worksheet for his
students. He is currently teaching his students about sine and
cosine. What he would like to have is something that looks like this
(ideally):

\noindent
\begin{tikzpicture}[scale=3,cap=round]
  % Local definitions
  \def\costhirty{0.8660256}

  % Colors
  \colorlet{anglecolor}{green!50!black}
  \colorlet{sincolor}{red}
  \colorlet{tancolor}{orange!80!black}
  \colorlet{coscolor}{blue}

  % Styles
  \tikzstyle axes=[]
  \tikzstyle important line=[very thick]
  \tikzstyle information text=[rounded corners,fill=red!10,inner sep=1ex]

  % The graphic
  \draw[style=help lines,step=0.5cm] (-1.4,-1.4) grid (1.4,1.4);
  
  \draw (0,0) circle (1cm);

  \begin{scope}[style=axes]
    \draw[->] (-1.5,0) -- (1.5,0) node[right] {$x$};
    \draw[->] (0,-1.5) -- (0,1.5) node[above] {$y$};

    \foreach \x/\xtext in {-1, -.5/-\frac{1}{2}, 1}
      \draw[xshift=\x cm] (0pt,1pt) -- (0pt,-1pt) node[below,fill=white] {$\xtext$};
  
    \foreach \y/\ytext in {-1, -.5/-\frac{1}{2}, .5/\frac{1}{2}, 1}
      \draw[yshift=\y cm] (1pt,0pt) -- (-1pt,0pt) node[left,fill=white] {$\ytext$};
  \end{scope}
    
  \filldraw[fill=green!20,draw=anglecolor] (0,0) -- (3mm,0pt) arc(0:30:3mm);
  \draw (15:2mm) node[anglecolor] {$\alpha$};
    
  \draw[style=important line,sincolor]
    (30:1cm) -- node[left=1pt,fill=white] {$\sin \alpha$} +(0,-.5);
  
  \draw[style=important line,coscolor]
    (0,0) -- node[below=2pt,fill=white] {$\cos \alpha$} (\costhirty,0);
  
  \draw[style=important line,tancolor] (1,0) --
    node [right=1pt,fill=white]
    {
      $\displaystyle \tan \alpha \color{black}=
      \frac{{\color{sincolor}\sin \alpha}}{\color{coscolor}\cos \alpha}$
    } (intersection of 0,0--30:1cm and 1,0--1,1) coordinate (t);

  \draw (0,0) -- (t);
  
  \draw[xshift=1.85cm] node [right,text width=6cm,style=information text]
    {
      The {\color{anglecolor} angle $\alpha$} is $30^\circ$ in the
      example ($\pi/6$ in radians). The {\color{sincolor}sine of
        $\alpha$}, which is the height of the red line, is
      \[
      {\color{sincolor} \sin \alpha} = 1/2.
      \]
      By the Theorem of Pythagoras we have ${\color{coscolor}\cos^2 \alpha} +
      {\color{sincolor}\sin^2\alpha} =1$. Thus the length of the blue
      line, which is the {\color{coscolor}cosine of $\alpha$}, must be
      \[
      {\color{coscolor}\cos\alpha} = \sqrt{1 - 1/4} = \textstyle
      \frac{1}{2} \sqrt 3. 
      \]%
      This shows that {\color{tancolor}$\tan \alpha$}, which is the
      height of the orange line, is  
      \[
      {\color{tancolor}\tan\alpha} = \frac{{\color{sincolor}\sin
          \alpha}}{\color{coscolor}\cos \alpha} = 1/\sqrt 3.
      \]%
    };
\end{tikzpicture}


\subsection{Setting up the Environment}

In \tikzname, to draw a picture, at the start of the picture
you need to tell \TeX\ or \LaTeX\ that you want to start a picture. In
\LaTeX\ this is done using the environment |{tikzpicture}|, in plain
\TeX\ you just use |\tikzpicture| to start the picture and
|\endtikzpicture| to end it.

\subsubsection{Setting up the Environment in \LaTeX}

Karl, being a \LaTeX\ user, thus sets up his file as follows:

\begin{codeexample}[code only]
\documentclass{article} % say
\usepackage{tikz}
\begin{document}
We are working on
\begin{tikzpicture}
  \draw (-1.5,0) -- (1.5,0);
  \draw (0,-1.5) -- (0,1.5);
\end{tikzpicture}.
\end{document}
\end{codeexample}

When executed, that is, run via |pdflatex| or via |latex| followed by
|dvips|, the resulting will contain something that looks like this:

\begin{codeexample}[width=7cm]
We are working on
\begin{tikzpicture}
  \draw (-1.5,0) -- (1.5,0);
  \draw (0,-1.5) -- (0,1.5);
\end{tikzpicture}.
\end{codeexample}

Admittedly, not quite the whole picture, yet, but we
do have the axes established. Well, not quite, but we have the lines
that make up the axes drawn. Karl suddenly has a sinking feeling
that the picture is still some way off. 

Let's have a more detailed look at the code. First, the package
|tikz| is loaded. This package is a so-called ``frontend'' to the
basic \pgfname\ system. The basic layer, which is also described in this
manual, is somewhat more, well, basic and thus harder to use. The
frontend makes things easier by providing a simpler syntax.

Inside the environment there are two |\draw| commands. They mean:
``The path, which is specified following the command up to the
semicolon, should be drawn.'' The first path is specified
as |(-1.5,0) -- (0,1.5)|, which means ``a straight line from the point
at position $(-1.5,0)$ to the point at position $(0,1.5)$.'' Here, the
positions are specified within a special coordinate system in which,
initially, one unit is 1cm.

Karl is quite pleased to note that the environment automatically
reserves enough space to encompass the picture.


\subsubsection{Setting up the Environment in Plain \TeX}

Karl's wife Gerda, who also happens to be a math teacher, is not a
\LaTeX\ user, but uses plain \TeX\ since she prefers to do things
``the old way.'' She can also use \tikzname. Instead of
|\usepackage{tikz}| she has to write |\input tikz.tex| and instead of
|\begin{tikzpicture}| she writes |\tikzpicture| and  instead of
  |\end{tikzpicture}| she writes |\endtikzpicture|. 

Thus, she would use:
\begin{codeexample}[code only]
%% Plain TeX file
\input tikz.tex
\baselineskip=12pt
\hsize=6.3truein
\vsize=8.7truein
We are working on
\tikzpicture
  \draw (-1.5,0) -- (1.5,0);
  \draw (0,-1.5) -- (0,1.5);
\endtikzpicture.
\bye
\end{codeexample}

Gerda can typeset this file using either |pdftex| or |tex| together
with |dvips|. \tikzname\ will automatically discern which driver she is
using. If she wishes to use |dvipdfm| together with |tex|, she 
either needs to modify the file |pgf.cfg| or can write
|\def\pgfsysdriver{pgfsys-dvipdfm.def}| somewhere \emph{before} she
inputs |tikz.tex| or |pgf.tex|.



\subsubsection{Setting up the Environment in Con\TeX t}

Karl's uncle Hans uses Con\TeX t. Like Gerda, Hans can also use
\tikzname. Instead of |\usepackage{tikz}| he says
|\usemodule[tikz]|. Instead of |\begin{tikzpicture}| he writes
  |\starttikzpicture| and  instead of |\end{tikzpicture}| he writes
|\stoptikzpicture|.  

His version of the example looks like this:
\begin{codeexample}[code only]
%% ConTeXt file
\usemodule[tikz]

We are working on
\starttikzpicture
  \draw (-1.5,0) -- (1.5,0);
  \draw (0,-1.5) -- (0,1.5);
\stoptikzpicture.
\end{codeexample}

Hans will now typeset this file in the usual way using |texexec|. 



\subsection{Straight Path Construction}

The basic building block of all pictures in \tikzname\ is the path. 
A \emph{path} is a series of straight lines and curves that are
connected (that is not the whole picture, but let us ignore the
complications for the moment). You start a path by specifying the
coordinates of the start position as a point in round brackets, as in
|(0,0)|. This is followed by a series of ``path extension
operations.'' The simplest is |--|, which we used already. It must be
followed by another coordinate and it extends the path in a straight
line to this new position. For example, if we were to turn the two
paths of the axes into one path, the following would result:

\begin{codeexample}[]
\tikz \draw (-1.5,0) -- (1.5,0) -- (0,-1.5) -- (0,1.5);
\end{codeexample}

Karl is a bit confused by the fact that there is no |{tikzpicture}|
environment, here. Instead, the little command |\tikz| is used. This
command either takes one argument (starting with an opening brace as in
|\tikz{\draw (0,0) -- (1.5,0)}|, which yields \tikz{\draw (0,0)
 --(1.5,0);}) or collects everything up to the next semicolon and
puts it inside a |{tikzpicture}| environment. As a rule of thumb, all
\tikzname\ graphic drawing commands must occur as an argument of |\tikz|
or inside a |{tikzpicture}| environment. Fortunately, the command
|\draw| will only be defined inside this environment, so there is
little chance that you will accidentally do something wrong here. 



\subsection{Curved Path Construction}

The next thing Karl wants to do is to draw the circle. For this,
straight lines obviously will not do. Instead, we need some way to
draw curves. For this, \tikzname\ provides a special syntax. One or two
``control points'' are needed. The math behind them is not quite
trivial, but here is the basic idea: Suppose you are at point $x$ and
the first control point is $y$. Then the curve will start ``going in
the direction of~$y$ at~$x$,'' that is, the tangent of the curve at $x$
will point toward~$y$. Next, suppose the curve should end at $z$ and
the second support point is $w$. Then the curve will, indeed, end at
$z$ and the tangent of the curve at point $z$ will go through $w$.

Here is an example (the control points have been added for clarity):
\begin{codeexample}[]
\begin{tikzpicture}
  \filldraw [gray] (0,0) circle (2pt)
                   (1,1) circle (2pt)
                   (2,1) circle (2pt)
                   (2,0) circle (2pt);
  \draw (0,0) .. controls (1,1) and (2,1) .. (2,0);
\end{tikzpicture}
\end{codeexample}

The general syntax for extending a path in a ``curved'' way is
|.. controls| \meta{first control point} |and| \meta{second control
  point} |..| \meta{end point}. You can leave out the |and|
\meta{second control point}, which causes the first one to be used 
twice.

So, Karl can now add the first half circle to the picture:

\begin{codeexample}[]
\begin{tikzpicture}
  \draw (-1.5,0) -- (1.5,0);
  \draw (0,-1.5) -- (0,1.5);
  \draw (-1,0) .. controls (-1,0.555) and (-0.555,1) .. (0,1)
               .. controls (0.555,1) and (1,0.555) .. (1,0);
\end{tikzpicture}
\end{codeexample}

Karl is happy with the result, but finds specifying circles in this
way to be extremely awkward. Fortunately, there is a much simpler way.


\subsection{Circle Path Construction}

In order to draw a circle, the path construction operation |circle| can
be used. This operation is followed by a radius in round brackets as in
the following example: (Note that the previous position is used as the
\emph{center} of the circle.)

\begin{codeexample}[]
\tikz \draw (0,0) circle (10pt);
\end{codeexample}

You can also append an ellipse to the path using the |ellipse|
operation. Instead of a single radius you can specify two of them, one
for the $x$-direction and one for the $y$-direction, separated by
|and|: 

\begin{codeexample}[]
\tikz \draw (0,0) ellipse (20pt and 10pt);
\end{codeexample}

To draw an ellipse whose axes are not horizontal and vertical, but
point in an arbitrary direction (a ``turned ellipse'' like \tikz
\draw[rotate=30] (0,0) ellipse (6pt and 3pt);) you can use
transformations, which are explained later. The code for the little
ellipse is |\tikz \draw[rotate=30] (0,0) ellipse (6pt and 3pt);|, by
the way. 

So, returning to Karl's problem, he can write
|\draw (0,0) circle (1cm);| to draw the circle:

\begin{codeexample}[]
\begin{tikzpicture}
  \draw (-1.5,0) -- (1.5,0);
  \draw (0,-1.5) -- (0,1.5);
  \draw (0,0) circle (1cm);
\end{tikzpicture}
\end{codeexample}


At this point, Karl is a bit alarmed that the circle is so small when
he wants the final picture to be much bigger. He is pleased to learn
that \tikzname\ has powerful transformation options and scaling
everything by a factor of three is very easy. But let us leave the
size as it is for the moment to save some space. 




\subsection{Rectangle Path Construction}

The next things we would like to have is the grid in the background.
There are several ways to produce it. For example, one might draw lots of
rectangles. Since rectangles are so common, there is a special syntax
for them: To add a rectangle to the current path, use the |rectangle|
path construction operation. This operation should be followed by another
coordinate and will append a rectangle to the path such that the
previous coordinate and the next coordinates are corners of the
rectangle. So, let us add two rectangles to the picture:

\begin{codeexample}[]
\begin{tikzpicture}
  \draw (-1.5,0) -- (1.5,0);
  \draw (0,-1.5) -- (0,1.5);
  \draw (0,0) circle (1cm);
  \draw (0,0) rectangle (0.5,0.5);
  \draw (-0.5,-0.5) rectangle (-1,-1);
\end{tikzpicture}
\end{codeexample}

While this may be nice in other situations, this is not really leading
anywhere with Karl's problem: First, we would need an awful lot of
these rectangles and then there is the border that is not ``closed.''

So, Karl is about to resort to simply drawing four vertical and four
horizontal lines using the nice |\draw| command, when he learns that
there is a |grid| path construction operation.



\subsection{Grid Path Construction}

The |grid| path operation adds a grid to the current path. It will add
lines making up a grid that fills the rectangle whose one corner is
the current point and whose other corner is the point following the
|grid| operation. For example, the code
|\tikz \draw[step=2pt] (0,0) grid (10pt,10pt);| produces \tikz
\draw[step=2pt] (0,0) grid (10pt,10pt);. Note how the optional
argument for |\draw| can be used to specify a grid width (there are
also |xstep| and |ystep| to define the steppings independently). As
Karl will learn soon, there are \emph{lots} of things that can be
influenced using such options.

For Karl, the following code could be used:

\begin{codeexample}[]
\begin{tikzpicture}
  \draw (-1.5,0) -- (1.5,0);
  \draw (0,-1.5) -- (0,1.5);
  \draw (0,0) circle (1cm);
  \draw[step=.5cm] (-1.4,-1.4) grid (1.4,1.4);
\end{tikzpicture}
\end{codeexample}

Having another look at the desired picture, Karl notices that it would
be nice for the grid to be more subdued. (His son told him that grids
tend to be distracting if they are not subdued.) To subdue the grid,
Karl adds two more options to the |\draw| command that draws the
grid. First, he uses the color |gray| for the grid lines. Second, he
reduces the line width to |very thin|. Finally, he swaps the ordering
of the commands so that the grid is drawn first and everything else on
top. 

\begin{codeexample}[]
\begin{tikzpicture}
  \draw[step=.5cm,gray,very thin] (-1.4,-1.4) grid (1.4,1.4);
  \draw (-1.5,0) -- (1.5,0);
  \draw (0,-1.5) -- (0,1.5);
  \draw (0,0) circle (1cm);
\end{tikzpicture}
\end{codeexample}


\subsection{Adding a Touch of  Style}

Instead of the options |gray,very thin| Karl could also have
said |style=help lines|. \emph{Styles} are predefined sets of options
that can be used to organize how a graphic is drawn. By saying
|style=help lines| you say ``use the style that I (or someone else)
has set for drawing help lines.'' If Karl decides, at some later
point, that grids should be drawn, say, using the color |blue!50|
instead of |gray|, he could say the following:
\begin{codeexample}[code only]
\tikzstyle help lines=[color=blue!50,very thin]
\end{codeexample}
Alternatively, he could have said the following:
\begin{codeexample}[code only]
\tikzstyle help lines+=[color=blue!50]
\end{codeexample}
This would have added the |color=blue!50| option. The |help lines|
style would now contain \emph{two} color options, but 
the second would override the first.

Using styles makes your graphics code more flexible. You can
change the way things look easily in a consistent manner.

To build a hierarchy of styles you can have one style use
another. So in order to define a style |Karl's grid| that is based on
the |grid| style Karl could say
\begin{codeexample}[code only]
\tikzstyle Karl's grid=[style=help lines,color=blue!50]
...
\draw[style=Karl's grid] (0,0) grid (5,5);
\end{codeexample}

You can also leave out the |style=|. Thus, whenever \tikzname\ encounters
an options that it does not know about, it will check whether this
option happens to be the name of a style. If so, the style is
used. Thus, Karl could also have written:
\begin{codeexample}[code only]
\tikzstyle Karl's grid=[help lines,color=blue!50]
...
\draw[Karl's grid] (0,0) grid (5,5);
\end{codeexample}

For some styles, like the |very thin| style, it is pretty clear what
the style does and there is no need to say |style=very thin|. For
other styles, like |help lines|, it seems more natural to me to say
|style=help lines|. But, mainly, this is a matter of taste.


\subsection{Drawing Options}

Karl wonders what other options there are that influence how a path is
drawn. He saw already that the |color=|\meta{color} option can be used
to set the line's color. The option |draw=|\meta{color} does nearly
the same, only it sets the color for the lines only and a different
color can be used for filling (Karl will need this when he fills the
arc for the angle).

He saw that the style |very thin| yields very thin lines. Karl is not
really surprised by this and neither is he surprised to learn that |thin|
yields thin lines,  |thick| yields thick lines, |very thick| yields
very thick lines, |ultra thick| yields really, really thick lines and
|ultra thin| yields lines that are so thin that low-resolution printers
and displays will have trouble showing them. He wonders what gives
lines of ``normal'' thickness. It turns out that |thin| is the correct
choice. This seems strange to Karl, but his son explains him that
\LaTeX\ has two commands called |\thinlines| and |\thicklines| and
that |\thinlines| gives the line width of ``normal'' lines, more
precisely, of the thickness that, say, the stem of a letter like ``T''
or ``i'' has. Nevertheless, Karl would like to know whether there is
anything ``in the middle'' between |thin| and |thick|. There is:
|semithick|.

Another useful thing one can do with lines is to dash or dot them. For
this, the two styles |dashed| and |dotted| can be used, yielding
\tikz \draw[dashed] (0,0) -- (10pt,0pt); and \tikz \draw[dotted] (0,0)
-- (10pt,0pt);. Both options also exist in a loose and a dense
version, called |loosely dashed|, |densely dashed|, |loosely dotted|,
and |densely dotted|. If he really, really  needs to, Karl can also
define much more complex dashing patterns with the |dash pattern|
option, but his son insists that dashing is to be used with utmost
care and mostly distracts. Karl's son claims that complicated dashing
patterns are evil. Karl's students do not care about dashing patterns. 



\subsection{Arc Path Construction}

Our next obstacle is to draw the arc for the angle. For this, the
|arc| path construction operation is useful, which draws part of a
circle or ellipse. This |arc| operation must be followed by a triple in 
rounded brackets, where the components of the triple are separated by
colons. The first two components are angles, the last one is a
radius. An example would be |(10:80:10pt)|, which means ``an arc from
10 degrees to 80 degrees on a circle of radius 10pt.'' Karl obviously
needs an arc from $0^\circ$ to $30^\circ$. The radius should be
something relatively small, perhaps around one third of the circle's
radius. This gives: |(0:30:3mm)|.

When one uses the arc path construction operation, the specified arc will
be added with its starting point at the current position. So, we first
have to ``get there.'' 

\begin{codeexample}[]
\begin{tikzpicture}
  \draw[step=.5cm,gray,very thin] (-1.4,-1.4) grid (1.4,1.4);
  \draw (-1.5,0) -- (1.5,0);
  \draw (0,-1.5) -- (0,1.5);
  \draw (0,0) circle (1cm);
  \draw (3mm,0mm) arc (0:30:3mm);
\end{tikzpicture}
\end{codeexample}

Karl thinks this is really a bit small and he cannot continue unless
he learns how to do scaling. For this, he can add the |[scale=3]|
option. He could add this option to each |\draw| command, but that
would be awkward. Instead, he adds it to the whole environment, which
causes this option to apply to everything within.

\begin{codeexample}[]
\begin{tikzpicture}[scale=3]
  \draw[step=.5cm,gray,very thin] (-1.4,-1.4) grid (1.4,1.4);
  \draw (-1.5,0) -- (1.5,0);
  \draw (0,-1.5) -- (0,1.5);
  \draw (0,0) circle (1cm);
  \draw (3mm,0mm) arc (0:30:3mm);
\end{tikzpicture}
\end{codeexample}

As for circles, you can specify ``two'' radii in order to get an
elliptical arc.

\begin{codeexample}[]
  \tikz \draw (0,0) arc (0:315:1.75cm and 1cm);
\end{codeexample}


\subsection{Clipping a Path}

In order to save space in this manual, it would be nice to clip Karl's
graphics a bit so that we can focus on the ``interesting''
parts. Clipping is pretty easy in \tikzname. You can use the |\clip|
command clip all subsequent drawing. It works like |\draw|, only it
does not draw anything, but uses the given path to clip everything
subsequently. 

\begin{codeexample}[]
\begin{tikzpicture}[scale=3]
  \clip (-0.1,-0.2) rectangle (1.1,0.75);
  \draw[step=.5cm,gray,very thin] (-1.4,-1.4) grid (1.4,1.4);
  \draw (-1.5,0) -- (1.5,0);
  \draw (0,-1.5) -- (0,1.5);
  \draw (0,0) circle (1cm);
  \draw (3mm,0mm) arc (0:30:3mm);
\end{tikzpicture}
\end{codeexample}

You can also do both at the same time: Draw \emph{and} clip a
path. For this, use the |\draw| command and add the |clip|
option. (This is not the whole picture: You can also use the |\clip|
command and add the |draw| option. Well, that is also not the whole
picture: In reality, |\draw| is just a shorthand for |\path[draw]|
and |\clip| is a shorthand for |\path[clip]| and you could also say
|\path[draw,clip]|.) Here is an example: 

\begin{codeexample}[]
\begin{tikzpicture}[scale=3]
  \clip[draw] (0.5,0.5) circle (.6cm);
  \draw[step=.5cm,gray,very thin] (-1.4,-1.4) grid (1.4,1.4);
  \draw (-1.5,0) -- (1.5,0);
  \draw (0,-1.5) -- (0,1.5);
  \draw (0,0) circle (1cm);
  \draw (3mm,0mm) arc (0:30:3mm);
\end{tikzpicture}
\end{codeexample}


\subsection{Parabola and Sine Path Construction}

Although Karl does not need them for his picture, he is pleased to
learn that there are |parabola| and |sin| and |cos| path operations for
adding parabolas and sine and cosine curves to the current path. For the
|parabola| operation, the current point will lie on the parabola as
well as the point given after the parabola operation. Consider
the following example:

\begin{codeexample}[]
\tikz \draw (0,0) rectangle (1,1)  (0,0) parabola (1,1);
\end{codeexample}

It is also possible to place the bend somewhere else:

\begin{codeexample}[]
\tikz \draw[x=1pt,y=1pt] (0,0) parabola bend (4,16) (6,12);
\end{codeexample}

The operations |sin| and |cos| add a sine or cosine curve in the interval
$[0,\pi/2]$ such that the previous current point is at the start of
the curve and the curve ends at the given end point. Here are two
examples:
\begin{codeexample}[]
A sine \tikz \draw[x=1ex,y=1ex] (0,0) sin (1.57,1); curve.
\end{codeexample}

\begin{codeexample}[]
\tikz \draw[x=1.57ex,y=1ex] (0,0) sin (1,1) cos (2,0) sin (3,-1) cos (4,0)
                            (0,1) cos (1,0) sin (2,-1) cos (3,0) sin (4,1);
\end{codeexample}



\subsection{Filling and Drawing}

Returning to the picture, Karl now wants the angle to be ``filled''
with a very light green. For this he uses |\fill| instead of
|\draw|. Here is what Karl does:

\begin{codeexample}[]
\begin{tikzpicture}[scale=3]
  \clip (-0.1,-0.2) rectangle (1.1,0.75);
  \draw[step=.5cm,gray,very thin] (-1.4,-1.4) grid (1.4,1.4);
  \draw (-1.5,0) -- (1.5,0);
  \draw (0,-1.5) -- (0,1.5);
  \draw (0,0) circle (1cm);
  \fill[green!20!white] (0,0) -- (3mm,0mm) arc (0:30:3mm) -- (0,0);
\end{tikzpicture}
\end{codeexample}

The color |green!20!white| means 20\% green and 80\% white mixed
together. Such color expression are possible since \pgfname\ uses Uwe
Kern's |xcolor| package, see the documentation of that package for
details on color expressions.

What would have happened, if Karl had not ``closed'' the path using
|--(0,0)| at the end? In this case, the path is closed automatically,
so this could have been omitted. Indeed, it would even have been
better to write the following, instead:
\begin{codeexample}[code only]
  \fill[green!20!white] (0,0) -- (3mm,0mm) arc (0:30:3mm) -- cycle;
\end{codeexample}
The |--cycle| causes the current path to be closed (actually the
current part of the current path) by smoothly joining the first and
last point. To appreciate the difference, consider the following
example:

\begin{codeexample}[]
\begin{tikzpicture}[line width=5pt]
  \draw (0,0) -- (1,0) -- (1,1) -- (0,0);
  \draw (2,0) -- (3,0) -- (3,1) -- cycle;
  \useasboundingbox (0,1.5); % make bounding box higher
\end{tikzpicture}
\end{codeexample}

You can also fill and draw a path at the same time using the
|\filldraw| command. This will first draw the path, then fill it. This
may not seem too useful, but you can specify different colors to be
used for filling and for stroking. These are specified as optional
arguments like this:

\begin{codeexample}[]
\begin{tikzpicture}[scale=3]
  \clip (-0.1,-0.2) rectangle (1.1,0.75);
  \draw[step=.5cm,gray,very thin] (-1.4,-1.4) grid (1.4,1.4);
  \draw (-1.5,0) -- (1.5,0);
  \draw (0,-1.5) -- (0,1.5);
  \draw (0,0) circle (1cm);
  \filldraw[fill=green!20!white, draw=green!50!black]
    (0,0) -- (3mm,0mm) arc (0:30:3mm) -- cycle;
\end{tikzpicture}
\end{codeexample}



\subsection{Shading}

Karl briefly considers the possibility of making the angle ``more
fancy'' by \emph{shading} it. Instead of filling the with a uniform
color, a smooth transition between different colors is used. For this,
|\shade| and |\shadedraw|, for shading and drawing at the same time,
can be used: 

\begin{codeexample}[]
  \tikz \shade (0,0) rectangle (2,1)  (3,0.5) circle (.5cm);
\end{codeexample}
The default shading is a smooth transition from gray to white. To
specify different colors, you can use options:

\begin{codeexample}[]
\begin{tikzpicture}[rounded corners,ultra thick]
  \shade[top color=yellow,bottom color=black] (0,0) rectangle +(2,1);
  \shade[left color=yellow,right color=black] (3,0) rectangle +(2,1);
  \shadedraw[inner color=yellow,outer color=black,draw=yellow] (6,0) rectangle +(2,1);
  \shade[ball color=green] (9,.5) circle (.5cm);
\end{tikzpicture}
\end{codeexample}

For Karl, the following might be appropriate:

\begin{codeexample}[]
\begin{tikzpicture}[scale=3]
  \clip (-0.1,-0.2) rectangle (1.1,0.75);
  \draw[step=.5cm,gray,very thin] (-1.4,-1.4) grid (1.4,1.4);
  \draw (-1.5,0) -- (1.5,0);
  \draw (0,-1.5) -- (0,1.5);
  \draw (0,0) circle (1cm);
  \shadedraw[left color=gray,right color=green, draw=green!50!black]
    (0,0) -- (3mm,0mm) arc (0:30:3mm) -- cycle;
\end{tikzpicture}
\end{codeexample}

However, he wisely decides that shadings usually only distract without
adding anything to the picture.


\subsection{Specifying Coordinates}

Karl now wants to add the sine and cosine lines. He knows already that
he can use the |color=| option to set the lines's colors. So, what is
the best way to specify the coordinates?

There are different ways of specifying coordinates. The easiest way is
to say something like |(10pt,2cm)|. This means 10pt in $x$-direction
and 2cm in $y$-directions. Alternatively, you can also leave out the
units as in |(1,2)|, which means ``one times the current $x$-vector
plus twice the current $y$-vector.'' These vectors default to 1cm in
the $x$-direction and 1cm in the $y$-direction, respectively.

In order to specify points in polar coordinates, use the notation
|(30:1cm)|, which means 1cm in direction 30 degree. This is obviously
quite useful to ``get to the point $(\cos 30^\circ,\sin 30^\circ)$ on
the circle.'' 

You can add a single |+| sign in front of a coordinate or two of
them as in |+(1cm,0cm)| or |++(0cm,2cm)|. Such coordinates are interpreted
differently: The first form means ``1cm upwards from the previous
specified position'' and the second means ``2cm to the right of the
previous specified position, making this the new specified position.''
For example, we can draw the sine line as follows:

\begin{codeexample}[]
\begin{tikzpicture}[scale=3]
  \clip (-0.1,-0.2) rectangle (1.1,0.75);
  \draw[step=.5cm,gray,very thin] (-1.4,-1.4) grid (1.4,1.4);
  \draw (-1.5,0) -- (1.5,0);
  \draw (0,-1.5) -- (0,1.5);
  \draw (0,0) circle (1cm);
  \filldraw[fill=green!20,draw=green!50!black]
    (0,0) -- (3mm,0mm) arc (0:30:3mm) -- cycle;
  \draw[red,very thick] (30:1cm) -- +(0,-0.5);
\end{tikzpicture}
\end{codeexample}

Karl used the fact $\sin 30^\circ = 1/2$. However, he very much
doubts that his students know this, so it would be nice to have a way
of specifying ``the point straight down from |(30:1cm)| that lies on
the $x$-axis.'' This is, indeed, possible using a special syntax: Karl
can write \verb!(30:1cm |- 0,0)!. In general, the meaning of
|(|\meta{p}\verb! |- !\meta{q}|)| is ``the intersection of a vertical
line through $p$ and a horizontal line through $q$.''

Next, let us draw the cosine line. One way would be to say
\verb!(30:1cm |- 0,0) -- (0,0)!. Another way is the following: we
``continue'' from where the sine ends: 

\begin{codeexample}[]
\begin{tikzpicture}[scale=3]
  \clip (-0.1,-0.2) rectangle (1.1,0.75);
  \draw[step=.5cm,gray,very thin] (-1.4,-1.4) grid (1.4,1.4);
  \draw (-1.5,0) -- (1.5,0);
  \draw (0,-1.5) -- (0,1.5);
  \draw (0,0) circle (1cm);
  \filldraw[fill=green!20,draw=green!50!black] (0,0) -- (3mm,0mm) arc
  (0:30:3mm) -- cycle;
  \draw[red,very thick]  (30:1cm) -- +(0,-0.5);
  \draw[blue,very thick] (30:1cm) ++(0,-0.5) -- (0,0);
\end{tikzpicture}
\end{codeexample}

Note the there is no |--| between |(30:1cm)| and |+(0,-0.5)|. In
detail, this path is interpreted as follows: ``First, the |(30:1cm)|
tells me to move by pen to $(\cos 30^\circ,1/2)$. Next, there comes
another coordinate specification, so I move my pen there without drawing
anything. This new point is half a unit down from the last position,
thus it is at $(\cos 30^\circ,0)$. Finally, I move the pen to the
origin, but this time drawing something (because of the |--|).''

To appreciate the difference between |+| and |++| consider the
following example:

\begin{codeexample}[]
\begin{tikzpicture}
  \def\rectanglepath{-- ++(1cm,0cm)  -- ++(0cm,1cm)  -- ++(-1cm,0cm) -- cycle}
  \draw (0,0) \rectanglepath;
  \draw (1.5,0) \rectanglepath;
\end{tikzpicture}
\end{codeexample}

By comparison, when using a single |+|, the coordinates are different:

\begin{codeexample}[]
\begin{tikzpicture}
  \def\rectanglepath{-- +(1cm,0cm)  -- +(1cm,1cm)  -- +(0cm,1cm) -- cycle}
  \draw (0,0) \rectanglepath;
  \draw (1.5,0) \rectanglepath;
\end{tikzpicture}
\end{codeexample}


Naturally, all of this could have been written more clearly and more
economically like this (either with a single of a double |+|): 
\begin{codeexample}[]
\tikz \draw (0,0) rectangle +(1,1)  (1.5,0) rectangle +(1,1);
\end{codeexample}



Karl is left with the line for $\tan \alpha$, which seems difficult to
specify using transformations and polar coordinates. For this he needs
another way of specifying coordinates: Karl can specify intersections
of lines as coordinates. The line for $\tan \alpha$ starts at $(1,0)$
and goes upward to a point that is at the intersection of a line going
``up'' and a line going from the origin through |(30:1cm)|. The syntax
for this point is the following:

\begin{codeexample}[code only]
\draw[very thick,orange] (1,0) -- (intersection of 1,0--1,1 and 0,0--30:1cm);
\end{codeexample}

In the following, two final examples of how to use relative
positioning are presented. Note that the transformation options,
which are explained later, are often more useful for shifting than
relative positioning. 

\begin{codeexample}[]
\begin{tikzpicture}[scale=0.5]
  \draw (0,0) -- (90:1cm) arc (90:360:1cm) arc (0:30:1cm) -- cycle;
  \draw (60:5pt) -- +(30:1cm) arc (30:90:1cm) -- cycle;

  \draw (3,0)  +(0:1cm) -- +(72:1cm) -- +(144:1cm) -- +(216:1cm) --
               +(288:1cm) -- cycle;
\end{tikzpicture}
\end{codeexample}



\subsection{Adding Arrow Tips}

Karl now wants to add the little arrow tips at the end of the axes. He has
noticed that in many plots, even in scientific journals, these arrow tips
seem to missing, presumably because the generating programs cannot
produce them. Karl thinks arrow tips belong at the end of axes. His
son agrees. His students do not care about arrow tips.

It turns out that adding arrow tips is pretty easy: Karl adds the option
|->| to the drawing commands for the axes:

\begin{codeexample}[]
\begin{tikzpicture}[scale=3]
  \clip (-0.1,-0.2) rectangle (1.1,1.51);
  \draw[step=.5cm,gray,very thin] (-1.4,-1.4) grid (1.4,1.4);
  \draw[->] (-1.5,0) -- (1.5,0);
  \draw[->] (0,-1.5) -- (0,1.5);
  \draw (0,0) circle (1cm);
  \filldraw[fill=green!20,draw=green!50!black] (0,0) -- (3mm,0mm) arc
  (0:30:3mm) -- cycle;
  \draw[red,very thick]    (30:1cm) -- +(0,-0.5);
  \draw[blue,very thick]   (30:1cm) ++(0,-0.5) -- (0,0);
  \draw[orange,very thick] (1,0) -- (intersection of 1,0--1,1 and 0,0--30:1cm);
\end{tikzpicture}
\end{codeexample}

If Karl had used the option |<-| instead of |->|, arrow tips would
have been put at the beginning of the path. The option |<->| puts
arrow tips at both ends of the path.

There are certain restrictions to the kind of paths to which arrow tips
can be added. As a rule of thumb, you can add arrow tips only to a
single open ``line.'' For example, you should not try to add tips to,
say, a rectangle or a circle. (You can try, but no guarantees as to what
will happen now or in future versions.) However, you can add arrow
tips to curved paths and to paths that have several segments, as in
the following examples:

\begin{codeexample}[]
\begin{tikzpicture}
  \draw [<->] (0,0) arc (180:30:10pt);
  \draw [<->] (1,0) -- (1.5cm,10pt) -- (2cm,0pt) -- (2.5cm,10pt);
\end{tikzpicture}
\end{codeexample}

Karl has a more detailed look at the arrow that \tikzname\ puts at the
end. It looks like this when he zooms it: \tikz { \useasboundingbox
  (0pt,-.5ex) rectangle (10pt,.5ex); \draw[->,line width=1pt] (0pt,0pt) --
  (10pt,0pt); }. The shape seems vaguely familiar and, indeed, this is
exactly the end of \TeX's standard arrow used in something like
$f\colon A \to B$.


Karl likes the arrow, especially since it is not ``as thick'' as the
arrows offered by many other packages. However, he expects that,
sometimes, he might need to use some other kinds of arrow.
To do so, Karl can say |>=|\meta{right arrow tip kind}, where
\meta{right arrow tip kind} is a special arrow tip specification. For
example, if Karl says |>=stealth|, then he tells \tikzname\
that he would like  ``stealth-fighter-like'' arrow tips: 

\begin{codeexample}[]
\begin{tikzpicture}[>=stealth]
  \draw [->] (0,0) arc (180:30:10pt);
  \draw [<<-,very thick] (1,0) -- (1.5cm,10pt) -- (2cm,0pt) -- (2.5cm,10pt);
\end{tikzpicture}
\end{codeexample}%>>

Karl wonders whether such a military name for the arrow type is really
necessary. He is not really mollified when his son tells him that
Microsoft's PowerPoint uses the same name. He decides to have his
students discuss this at some point.

In addition to |stealth| there are several other predefined arrow tip
kinds Karl can choose from, see
Section~\ref{section-library-arrows}. Furthermore, he can define
arrows types himself, if he needs new ones. 




\subsection{Scoping}

Karl saw already that there are numerous graphic options that affect how
paths are rendered. Often, he would like to apply certain options to
a whole set of graphic commands. For example, Karl might wish to draw
three paths using a |thick| pen, but would like everything else to
be drawn ``normally.''

If Karl wishes to set a certain graphic option for the whole picture,
he can simply pass this option to the |\tikz| command or to the
|{tikzpicture}| environment (Gerda would pass the options to
|\tikzpicture| and Hans passes them to |\starttikzpicture|). However,
if Karl wants to apply graphic options to a local group, he put these
commands inside a |{scope}| environment (Gerda uses |\scope| and
|\endscope|, Hans uses |\startscope| and |\stopscope|). This
environment takes graphic options as an optional argument and these
options apply to everything inside the scope, but not to anything outside.

Here is an example:

\begin{codeexample}[]
\begin{tikzpicture}[ultra thick]
  \draw (0,0) -- (0,1);
  \begin{scope}[thin]
    \draw (1,0) -- (1,1);
    \draw (2,0) -- (2,1);
  \end{scope}
  \draw (3,0) -- (3,1);  
\end{tikzpicture}
\end{codeexample}

Scoping has another interesting effect: Any changes to the clipping
area are local to the scope. Thus, if you say |\clip| somewhere inside
a scope, the effect of the |\clip| command ends at the end of the
scope. This is useful since there is no other way of ``enlarging'' the
clipping area.

Karl has also already seen that giving options to commands like
|\draw| apply only to that command. In turns out that the situation is
slightly more complex. First, options to a command like |\draw| are
not really options to the command, but they are ``path options'' and
can be given anywhere on the path. So, instead of
|\draw[thin] (0,0) -- (1,0);| one can also write
|\draw (0,0) [thin] -- (1,0);| or |\draw (0,0) -- (1,0) [thin];|; all
of these have the same effect. This might seem strange since in the
last case, it would appear that the |thin| should take effect only
``after'' the line from $(0,0)$ to $(1,0)$ has been draw. However,
most graphic options only apply to the whole path. Indeed, if you say
both |thin| and |thick| on the same path, the last option given will
``win.''

When reading the above, Karl notices that only ``most'' graphic
options apply to the whole path. Indeed, all transformation options do
\emph{not} apply to the whole path, but only to ``everything following
them on the path.'' We will have a more detailed look at this in a
moment. Nevertheless, all options given during a path construction
apply only to this path. 



\subsection{Transformations}

When you specify a  coordinate like |(1cm,1cm)|, where is that
coordinate placed on the page? To determine the position, \tikzname,
\TeX, and \textsc{pdf} or PostScript all apply certain transformations
to the given coordinate in order to determine the finally position on
the page. 

\tikzname\ provides numerous options that allow you to transform
coordinates in \pgfname's private coordinate system. For example, the
|xshift| option allows you to shift all subsequent points by a certain
amount:

\begin{codeexample}[]
\tikz \draw (0,0) -- (0,0.5) [xshift=2pt] (0,0) -- (0,0.5);
\end{codeexample}

It is important to note that you can change transformation ``in the
middle of a path,'' a feature that is not supported by \pdf\
or PostScript. The reason is that \pgfname\ keeps track of its own
transformation matrix.

Here is a more complicated example:
\begin{codeexample}[]
\begin{tikzpicture}[even odd rule,rounded corners=2pt,x=10pt,y=10pt]
  \filldraw[fill=examplefill] (0,0)   rectangle (1,1)
    [xshift=5pt,yshift=5pt]   (0,0)   rectangle (1,1)
                [rotate=30]   (-1,-1) rectangle (2,2);
\end{tikzpicture}
\end{codeexample}

The most useful transformations are |xshift| and |yshift| for
shifting, |shift| for shifting to a given point as in |shift={(1,0)}|
or |shift={+(0,0)}| (the braces are necessary so that \TeX\ does not
mistake the comma for separating options), |rotate| for rotating by a
certain angle (there is also a |rotate around| for rotating around a
given point), |scale| for scaling by a certain factor, |xscale| and
|yscale| for scaling only in the $x$- or $y$-direction (|xscale=-1| is
a flip), and |xslant| and |yslant| for slanting. If these
transformation and those that I have not mentioned are not
sufficient,  the |cm| option allows you to apply an arbitrary
transformation matrix. Karl's students, by the way, do not know what a
transformation matrix is. 



\subsection{Repeating Things: For-Loops}

Karl's next aim is to add little ticks on the axes at positions $-1$,
$-1/2$, $1/2$, and $1$. For this, it would be nice to use some kind of
``loop,'' especially since he wishes to do the same thing at each of
these positions. There are different packages for doing this. \LaTeX\
has its own internal command for this, |pstricks| comes along with the
powerful |\mulitdo| command. All of these can be used together with
\pgfname\ and \tikzname, so if you are familiar with them, feel free to
use them. \pgfname\ introduces yet another command, called |\foreach|,
which I introduced since I could never remember the syntax of the other
packages. |\foreach| is defined in the package |pgffor| and can be used
independently of \pgfname. \tikzname\ includes it automatically.

In its basic form, the |\foreach| command is easy to use:
\begin{codeexample}[]
\foreach \x in {1,2,3} {$x =\x$, }
\end{codeexample}

The general syntax is |\foreach| \meta{variable}| in {|\meta{list of
    values}|} |\meta{commands}. Inside the \meta{commands}, the
\meta{variable} will be assigned to the different values. If the
\meta{commands} do not start with a brace, everything up to the
next semicolon is used as \meta{commands}.

For Karl and the ticks on the axes, he could use the following code:

\begin{codeexample}[]
\begin{tikzpicture}[scale=3]
  \clip (-0.1,-0.2) rectangle (1.1,1.51);
  \draw[step=.5cm,gray,very thin] (-1.4,-1.4) grid (1.4,1.4);
  \filldraw[fill=green!20,draw=green!50!black] (0,0) -- (3mm,0mm) arc
  (0:30:3mm) -- cycle;
  \draw[->] (-1.5,0) -- (1.5,0);
  \draw[->] (0,-1.5) -- (0,1.5);
  \draw (0,0) circle (1cm);

  \foreach \x in {-1cm,-0.5cm,1cm}
    \draw (\x,-1pt) -- (\x,1pt);
  \foreach \y in {-1cm,-0.5cm,0.5cm,1cm}
    \draw (-1pt,\y) -- (1pt,\y);
\end{tikzpicture}
\end{codeexample}

As a matter of fact, there are many different ways of creating the
ticks. For example, Karl could have put the |\draw ...;| inside curly
braces. He could also have used, say,
\begin{codeexample}[code only]
\foreach \x in {-1,-0.5,1}
  \draw[xshift=\x cm] (0pt,-1pt) -- (0pt,1pt);
\end{codeexample}

Karl is curious what would happen in a more complicated situation
where there are, say, 20 ticks. It seems bothersome to explicitly
mention all these numbers in the set for |\foreach|. Indeed, it is
possible to use |...| inside the |\foreach| statement to iterate over 
a large number of values (which must, however, be dimensionless
real numbers) as in the following example: 

\begin{codeexample}[]
\tikz \foreach \x in {1,...,10}
        \draw (\x,0) circle (0.4cm);
\end{codeexample}

If you provide \emph{two} numbers before the |...|, the |\foreach|
statement will use their difference for the stepping:

\begin{codeexample}[]
\tikz \foreach \x in {-1,-0.5,...,1}
       \draw (\x cm,-1pt) -- (\x cm,1pt);
\end{codeexample}

We can also nest loops to create interesting effects:

\begin{codeexample}[]
\begin{tikzpicture}
  \foreach \x in {1,2,...,5,7,8,...,12}
    \foreach \y in {1,...,5}
    {
      \draw (\x,\y) +(-.5,-.5) rectangle ++(.5,.5);
      \draw (\x,\y) node{\x,\y};
    }
\end{tikzpicture}
\end{codeexample}

The |\foreach| statement can do even trickier stuff, but the above
gives the idea.




\subsection{Adding Text}

Karl is, by now, quite satisfied with the picture. However, the most
important parts, namely the labels, are still missing! 

\tikzname\ offers an easy-to-use and powerful system for adding text and,
more generally, complex shapes to a picture at specific positions. The
basic idea is the following: When \tikzname\ is constructing a path and
encounters the keyword |node| in the middle of a path, it
reads a \emph{node specification}. The keyword |node| is typically
followed by some options and then some text between curly braces. This
text is put inside a normal \TeX\ box (if the node specification
directly follows a coordinate, which is usually the case, \tikzname\ is
able to perform some magic so that it is even possible to use verbatim
text inside the boxes) and then placed at the current position, that
is, at the last specified position (possibly shifted a bit, according
to the given options). However, all nodes are drawn only after the
path has been completely drawn/filled/shaded/clipped/whatever.  

\begin{codeexample}[]
\begin{tikzpicture}
  \draw (0,0) rectangle (2,2);
  \draw (0.5,0.5) node [fill=examplefill]
                       {Text at \verb!node 1!}
     -- (1.5,1.5) node {Text at \verb!node 2!};
\end{tikzpicture}
\end{codeexample}

Obviously, Karl would not only like to place nodes \emph{on} the last
specified position, but also to the left or the 
right of these positions. For this, every node object that you
put in your picture is equipped with several \emph{anchors}. For
example, the |north| anchor is in the middle at the upper end of the shape,
the |south| anchor is at the bottom and the |north east| anchor is in
the upper right corner. When you given the option |anchor=north|, the
text will be placed such that this northern anchor will lie on the
current position and the text is, thus, below the current
position. Karl uses this to draw the ticks as follows:

\begin{codeexample}[]
\begin{tikzpicture}[scale=3]
  \clip (-0.6,-0.2) rectangle (0.6,1.51);
  \draw[step=.5cm,style=help lines] (-1.4,-1.4) grid (1.4,1.4);
  \filldraw[fill=green!20,draw=green!50!black]
    (0,0) -- (3mm,0mm) arc (0:30:3mm) -- cycle;
  \draw[->] (-1.5,0) -- (1.5,0);   \draw[->] (0,-1.5) -- (0,1.5);
  \draw (0,0) circle (1cm);

  \foreach \x in {-1,-0.5,1}
    \draw (\x cm,1pt) -- (\x cm,-1pt) node[anchor=north] {$\x$};
  \foreach \y in {-1,-0.5,0.5,1}
    \draw (1pt,\y cm) -- (-1pt,\y cm) node[anchor=east] {$\y$};
\end{tikzpicture}
\end{codeexample}

This is quite nice, already. Using these anchors, Karl can now add
most of the other text elements. However, Karl thinks that, though
``correct,'' it is quite counter-intuitive that in order to place something
\emph{below} a given point, he has to use the \emph{north} anchor. For
this reason, there is an option called |below|, which does the
same as |anchor=north|. Similarly, |above right| does the same as
|anchor=south east|. In addition, |below| takes an optional
dimension argument. If given, the shape will additionally be shifted
downwards by the given amount. So, |below=1pt| can be used to put
a text label below some point and, additionally shift it  1pt
downwards. 

Karl is not quite satisfied with the ticks. He would like to have
$1/2$ or $\frac{1}{2}$ shown instead of $0.5$, partly to show off the
nice capabilities of \TeX\ and \tikzname, partly because for positions
like $1/3$ or $\pi$ it is certainly very much preferable to have the
``mathematical'' tick there instead of just the ``numeric'' tick.
His students, on the other hand, prefer $0.5$ over $1/2$
since they are not too fond of fractions in general.

Karl now faces a problem: For the |\foreach| statement, the position
|\x| should still be given as |0.5| since \tikzname\ will not know where
|\frac{1}{2}| is supposed to be. On the other hand, the typeset text
should really be  |\frac{1}{2}|. To solve this problem, |\foreach|
offers a special syntax: Instead of having one variable |\x|, Karl can
specify two (or even more) variables separated by a slash as in
|\x / \xtext|. Then, the elements in the set over which |\foreach|
iterates must also be of the form \meta{first}|/|\meta{second}. In
each iteration, |\x| will be set to \meta{first} and |\xtext| will be
set to \meta{second}. If no \meta{second} is given, the \meta{first}
will be used again. So, here is the new code for the ticks: 

\begin{codeexample}[]
\begin{tikzpicture}[scale=3]
  \clip (-0.6,-0.2) rectangle (0.6,1.51);
  \draw[step=.5cm,style=help lines] (-1.4,-1.4) grid (1.4,1.4);
  \filldraw[fill=green!20,draw=green!50!black]
    (0,0) -- (3mm,0mm) arc (0:30:3mm) -- cycle;
  \draw[->] (-1.5,0) -- (1.5,0); \draw[->] (0,-1.5) -- (0,1.5);
  \draw (0,0) circle (1cm);

  \foreach \x/\xtext in {-1, -0.5/-\frac{1}{2}, 1}
    \draw (\x cm,1pt) -- (\x cm,-1pt) node[anchor=north] {$\xtext$};
  \foreach \y/\ytext in {-1, -0.5/-\frac{1}{2}, 0.5/\frac{1}{2}, 1}
    \draw (1pt,\y cm) -- (-1pt,\y cm) node[anchor=east] {$\ytext$};
\end{tikzpicture}
\end{codeexample}

Karl is quite pleased with the result, but his son points out that
this is still not perfectly satisfactory: The grid and the circle
interfere with the numbers and decrease their legibility. Karl is not
very concerned by this (his students do not even notice), but his son
insists that there is an easy solution: Karl can add the
|[fill=white]| option to fill out the background of the text shape
with a white color. 

The next thing Karl wants to do is to add the labels like $\sin
\alpha$. For this, he would like to place a label ``in the middle of
line.'' To do so, instead of specifying the label
|node {$\sin\alpha$}|  directly after one of the endpoints of the line
(which would place 
the label at that endpoint), Karl can give the label directly after
the |--|, before the coordinate. By default, this places the label in
the middle of the line, but the |pos=| options can be used to modify
this. Also, options like |near start| and |near end| can be used to
modify this position:


\begin{codeexample}[]
\begin{tikzpicture}[scale=3]
  \clip (-2,-0.2) rectangle (2,0.8);
  \draw[step=.5cm,gray,very thin] (-1.4,-1.4) grid (1.4,1.4);
  \filldraw[fill=green!20,draw=green!50!black] (0,0) -- (3mm,0mm) arc
  (0:30:3mm) -- cycle;
  \draw[->] (-1.5,0) -- (1.5,0) coordinate (x axis);
  \draw[->] (0,-1.5) -- (0,1.5) coordinate (y axis);
  \draw (0,0) circle (1cm);
    
  \draw[very thick,red]
    (30:1cm) -- node[left=1pt,fill=white] {$\sin \alpha$} (30:1cm |- x axis);
  \draw[very thick,blue]
    (30:1cm |- x axis) -- node[below=2pt,fill=white] {$\cos \alpha$} (0,0);
  \draw[very thick,orange] (1,0) -- node [right=1pt,fill=white]
    {$\displaystyle \tan \alpha \color{black}=
      \frac{{\color{red}\sin \alpha}}{\color{blue}\cos \alpha}$}
    (intersection of 0,0--30:1cm and 1,0--1,1) coordinate (t);

  \draw (0,0) -- (t);
  
  \foreach \x/\xtext in {-1, -0.5/-\frac{1}{2}, 1}
    \draw (\x cm,1pt) -- (\x cm,-1pt) node[anchor=north,fill=white] {$\xtext$};
  \foreach \y/\ytext in {-1, -0.5/-\frac{1}{2}, 0.5/\frac{1}{2}, 1}
    \draw (1pt,\y cm) -- (-1pt,\y cm) node[anchor=east,fill=white] {$\ytext$};
\end{tikzpicture}
\end{codeexample}

You can also position labels on curves and, by adding the |sloped|
option, have them rotated such that they match the line's slope. Here
is an example:

\begin{codeexample}[]
\begin{tikzpicture}
  \draw (0,0) .. controls (6,1) and (9,1) ..
    node[near start,sloped,above] {near start}
    node {midway}
    node[very near end,sloped,below] {very near end} (12,0);
\end{tikzpicture}
\end{codeexample}

It remains to draw the explanatory text at the right of the
picture. The main difficulty here lies in limiting the width of the
text ``label,'' which is quite long, so that line breaking is
used. Fortunately, Karl can use the option |text width=6cm| to get the
desired effect. So, here is the full code:

\begin{codeexample}[code only]
\begin{tikzpicture}[scale=3,cap=round]
  % Local definitions
  \def\costhirty{0.8660256}

  % Colors
  \colorlet{anglecolor}{green!50!black}
  \colorlet{sincolor}{red}
  \colorlet{tancolor}{orange!80!black}
  \colorlet{coscolor}{blue}

  % Styles
  \tikzstyle{axes}=[]
  \tikzstyle{important line}=[very thick]
  \tikzstyle{information text}=[rounded corners,fill=red!10,inner sep=1ex]

  % The graphic
  \draw[style=help lines,step=0.5cm] (-1.4,-1.4) grid (1.4,1.4);
  
  \draw (0,0) circle (1cm);

  \begin{scope}[style=axes]
    \draw[->] (-1.5,0) -- (1.5,0) node[right] {$x$} coordinate(x axis);
    \draw[->] (0,-1.5) -- (0,1.5) node[above] {$y$} coordinate(y axis);

    \foreach \x/\xtext in {-1, -.5/-\frac{1}{2}, 1}
      \draw[xshift=\x cm] (0pt,1pt) -- (0pt,-1pt) node[below,fill=white] {$\xtext$};
  
    \foreach \y/\ytext in {-1, -.5/-\frac{1}{2}, .5/\frac{1}{2}, 1}
      \draw[yshift=\y cm] (1pt,0pt) -- (-1pt,0pt) node[left,fill=white] {$\ytext$};
  \end{scope}
    
  \filldraw[fill=green!20,draw=anglecolor] (0,0) -- (3mm,0pt) arc(0:30:3mm);
  \draw (15:2mm) node[anglecolor] {$\alpha$};
    
  \draw[style=important line,sincolor]
    (30:1cm) -- node[left=1pt,fill=white] {$\sin \alpha$} (30:1cm |- x axis);
  
  \draw[style=important line,coscolor]
    (30:1cm |- x axis) -- node[below=2pt,fill=white] {$\cos \alpha$} (0,0);
  
  \draw[style=important line,tancolor] (1,0) -- node[right=1pt,fill=white] {
    $\displaystyle \tan \alpha \color{black}=
    \frac{{\color{sincolor}\sin \alpha}}{\color{coscolor}\cos \alpha}$}
    (intersection of 0,0--30:1cm and 1,0--1,1) coordinate (t);

  \draw (0,0) -- (t);
  
  \draw[xshift=1.85cm]
    node[right,text width=6cm,style=information text]
    {
      The {\color{anglecolor} angle $\alpha$} is $30^\circ$ in the
      example ($\pi/6$ in radians). The {\color{sincolor}sine of
        $\alpha$}, which is the height of the red line, is
      \[
      {\color{sincolor} \sin \alpha} = 1/2.
      \]
      By the Theorem of Pythagoras ...
    };
\end{tikzpicture}
\end{codeexample}


\section{Tutorial: A Petri-Net for Hagen}

In this second tutorial we explore the node mechanism of
\tikzname\ and \pgfname.

Hagen must give a talk tomorrow about his favorite formalism for
distributed systems: Petri nets! Hagen used to give his talks using a
blackboard and everyone seemed to be perfectly concent with
this. Unfortunately, his audience has been spoiled recently with fancy
projector-based presentations and there seems to be a certain amount
of peer pressure that this Petri nets should also be drawn using a
graphic program. One of the professors at his institutes recommends
\tikzname\ for this and Hagen decides to give it a try.


\subsection{Problem Statement}

For his talk, Hagen wishes to create a graphic that demonstrates how a
net with place capacities can be simulated by a net without
capacities. The graphic should look like this, ideally:

\begin{quote}
\begin{tikzpicture}[node distance=1.3cm,>=stealth',bend angle=45,auto]

  \tikzstyle{place}=[circle,thick,draw=blue!75,fill=blue!20,minimum size=6mm]
  \tikzstyle{red place}=[place,draw=red!75,fill=red!20]
  \tikzstyle{transition}=[rectangle,thick,draw=black!75,fill=black!20,minimum size=4mm]

  \tikzstyle{every label}=[red]

  \begin{scope}
    % First net
    \node [place,tokens=1] (w1)                                    {};
    \node [place] (c1) [below of=w1]                      {};
    \node [place] (s)  [below of=c1,label=above:$s\le 3$] {};
    \node [place] (c2) [below of=s]                       {};
    \node [place,tokens=1] (w2) [below of=c2]                      {};
    
    \node [transition] (e1) [left of=c1] {}
      edge [pre,bend left]                  (w1)
      edge [post,bend right]                (s)
      edge [post]                           (c1);

    \node [transition] (e2) [left of=c2] {}
      edge [pre,bend right]                 (w2)
      edge [post,bend left]                 (s)
      edge [post]                           (c2);
      
    \node [transition] (l1) [right of=c1] {}
      edge [pre]                            (c1)
      edge [pre,bend left]                  (s)
      edge [post,bend right] node[swap] {2} (w1);

    \node [transition] (l2) [right of=c2] {}
      edge [pre]                            (c2)
      edge [pre,bend right]                 (s)
      edge [post,bend left]  node {2}       (w2);
  \end{scope}
  
  \begin{scope}[xshift=6cm]
    % Second net
    \node [place,tokens=1]
                      (w1')                                                {};
    \node [place]     (c1') [below of=w1']                                 {};
    \node [red place] (s1') [below of=c1',xshift=-5mm,label=left:$s$]      {};
    \node [red place,tokens=3]
                      (s2') [below of=c1',xshift=5mm,label=right:$\bar s$] {};
    \node [place]     (c2') [below of=s1',xshift=5mm]                      {};
    \node [place,tokens=1]
                      (w2') [below of=c2']                                 {};
    
    \node [transition] (e1') [left of=c1'] {}
      edge [pre,bend left]                  (w1')
      edge [post]                           (s1')
      edge [pre]                            (s2')
      edge [post]                           (c1');

    \node [transition] (e2') [left of=c2'] {}
      edge [pre,bend right]                 (w2')
      edge [post]                           (s1')
      edge [pre]                            (s2')
      edge [post]                           (c2');
      
    \node [transition] (l1') [right of=c1'] {}
      edge [pre]                            (c1')
      edge [pre]                            (s1')
      edge [post]                           (s2')
      edge [post,bend right] node[swap] {2} (w1');

    \node [transition] (l2') [right of=c2'] {}
      edge [pre]                            (c2')
      edge [pre]                            (s1')
      edge [post]                           (s2')
      edge [post,bend left]  node {2}       (w2');
  \end{scope}

  \draw [-to,thick,snake=snake,segment amplitude=.4mm,segment length=2mm,line after snake=1mm]
    ([xshift=5mm]s -| l1) -- ([xshift=-5mm]s1' -| e1')
    node [above=1mm,midway,text width=3cm,text centered]
      {replacement of the \textcolor{red}{capacity} by \textcolor{red}{two places}};

  \begin{pgfonlayer}{background}
    \filldraw [line width=4mm,join=round,black!10]
      (w1.north  -| l1.east)  rectangle (w2.south  -| e1.west)
      (w1'.north -| l1'.east) rectangle (w2'.south -| e1'.west);
  \end{pgfonlayer}
\end{tikzpicture}
\end{quote}


\subsection{Setting up the Environment}

For the picture Hagen will need to load the \tikzname\ package as did
Karl in the previous tutorial. However, Hagen will also need to load
some additional  \emph{library packages} that Karl did not need. These
library packages contain additional definitions like extra arrow tips
that are typically not needed in a picture and that need to be
loaded explicitly.

Hagen will need to load four libraries: The arrow tip library for the
special arrow tip used in the graphic, the snake library with the
``snaking line'' in the middle, the background library for the two
rectangular areas that are behind the two main parts of the picture,
and the to-path library for easily creating the curved lines. 


\subsubsection{Setting up the Environment in \LaTeX}

When using \LaTeX\ use:

\begin{codeexample}[code only]
\documentclass{article} % say

\usepackage{tikz}
\usepackage{pgflibraryarrows}
\usepackage{pgflibrarysnakes}
\usepackage{pgflibrarytikzbackgrounds}
\usepackage{pgflibrarytikztopaths}

\begin{document}
\begin{tikzpicture}
  \draw (0,0) -- (1,1);
\end{tikzpicture}
\end{document}
\end{codeexample}


\subsubsection{Setting up the Environment in Plain \TeX}

When using plain \TeX\ use:

\begin{codeexample}[code only]
%% Plain TeX file
\input tikz.tex
\input pgflibraryarrows.tex
\input pgflibrarysnakes.tex
\input pgflibrarytikzbackgrounds.tex
\input pgflibrarytikztopaths.tex
\baselineskip=12pt
\hsize=6.3truein
\vsize=8.7truein
\tikzpicture
  \draw (0,0) -- (1,1);
\endtikzpicture
\bye
\end{codeexample}


\subsubsection{Setting up the Environment in Con\TeX t}

When using Con\TeX\ use:
\begin{codeexample}[code only]
%% ConTeXt file
\usemodule[tikz]
\usemodule[pgflibraryarrows]
\usemodule[pgflibrarysnakes]
\usemodule[pgflibrarytikzbackgrounds]
\usemodule[pgflibrarytikztopaths]

\starttikzpicture
  \draw (0,0) -- (1,1);
\stoptikzpicture
\end{codeexample}



\subsection{Introduction to Nodes}

In principle, we already know how to create the graphics that Hagen
desires (except perhaps for the snaked line, we will come to that): We
start with big light gray rectangle and then add lots of circles and
small rectangle, plus some arrows.

However, this approach has numerous disadvantages: First, it is hard
to change anything at a later stage. For example, if we decide to add
more places to the Petri nets (the circles are called places in Petri
net theory), all of the coordinates change and we need to recalculate
everything. Second, it is hard to read the code for the Petri net as
it just a long and complicated list of coordinates and drawing
commands -- the underlying structure of the Petri net is lost.

Fortunately, \tikzname\ offers a powerful mechanism for avoiding the
above problems: nodes. We already came across nodes in the previous
tutorial, where we used them to add labels to Karl's graphic. In the
present tutorial we will see that nodes are much more powerful.

A node is a small part of a picture. When a node is created, you
provide a position where the node should be drawn and a
\emph{shape}. A node of shape |circle| will be drawn as a circle, a
node of shape |rectangle| as a rectangle, and so on. A node may also
contain same text, which is why Karl used nodes to show text. Finally,
a node can get a \emph{name} for later reference.

In Hagen's picture we will use nodes for the places and for the
transitions of the Petri net (the places are the circles, the
transitions are the rectangles). Let us start with the upper half of
the left Petir net. In this upper half we have three places and two
transitions. Instead of drawing three circles and two rectangles, we
use three nodes of shape |circle| and two nodes of shape
|rectangle|.

\begin{codeexample}[]
\begin{tikzpicture}
  \path ( 0,2) node [shape=circle,draw] {}    
        ( 0,1) node [shape=circle,draw] {}    
        ( 0,0) node [shape=circle,draw] {}    
        ( 1,1) node [shape=rectangle,draw] {}
        (-1,1) node [shape=rectangle,draw] {};    
\end{tikzpicture}
\end{codeexample}

Hagen notes that this does not quite look like the final picture, but
it seems like a good first step.

Let us have a more detailed look at the code. The whole picture
consists of a single path. Ignoring the |node| operations there is not
much going on in this path: It is just a sequence of coordinates with
nothing ``happening'' between them. Indeed, even if something were to
happen like a line-to or a curve-to, the |\path| command would not
``do'' anything with the resulting path. So, all the magic must be in
the |node| commands.

In the previous tutorial we learned that a |node| will add a piece of
text at the last coordinate. Thus, each of the five nodes is added at
a different position. In the above code, this text is empty
(because of the empty |{}|). So, why do we see anything at all? The
answer is the |draw| option for the |node| operation: It causes the
``shape around the text'' to be drawn.

So, the code |(0,2) node [shape=circle,draw] {}| means the following:
``In the main path, add a move-to to the coordinate |(0,2)|. Then,
temporarily suspend the construction of the main path while the node
is build. This node will be a |circle| around an empty text. This
circle is to be |draw|n, but not filled or otherwise used. Once this
whole node is constructed, it is saved until after the 
main path is finished. Then, it is drawn.'' Then following
|(0,1) node [shape=circle,draw] {}| then has the following effect:
``Continue the main path with a move-to to |(0,1)|. Then construct a
node at this position also. This node is also shown after the main
path is finished.'' And so on.



\subsection{Placing Nodes Using the At Syntax}

Hagen now understands how the |node| operation adds nodes to the path,
but it seems a bit silly to create a path using the |\path| operation,
consisting of numerous superfluous move-to operations, only to place
nodes. He is pleased to learn that there are ways to add nodes in a
more sensible manner.

First, the |node| operation allows one to add
|at (|\meta{coordinate}|)| in order to directly specify where the node
should be placed, sidestepping the rule that nodes are placed on the
last coordinate. Hagen can then write the following:

\begin{codeexample}[]
\begin{tikzpicture}
  \path node at ( 0,2) [shape=circle,draw] {}    
        node at ( 0,1) [shape=circle,draw] {}    
        node at ( 0,0) [shape=circle,draw] {}    
        node at ( 1,1) [shape=rectangle,draw] {}
        node at (-1,1) [shape=rectangle,draw] {};    
\end{tikzpicture}
\end{codeexample}

Now Hagen is still left with a single empty path, but at least the
path no longer contains strange move-tos. It turns out that this can
be improved further: The |\node| command is an abbreviation for
|\path node|, which allows Hagen to write:

\begin{codeexample}[]
\begin{tikzpicture}
  \node at ( 0,2) [circle,draw] {};
  \node at ( 0,1) [circle,draw] {};   
  \node at ( 0,0) [circle,draw] {};   
  \node at ( 1,1) [rectangle,draw] {};
  \node at (-1,1) [rectangle,draw] {};    
\end{tikzpicture}
\end{codeexample}

Hagen likes this syntax much better than the previous one. Note that
Hagen has also omitted the |shape=| since, like |color=|, \tikzname\ 
allows you to omit the |shape=| if there is no confusion.



\subsection{Using Styles}

Feeling adventurous, Hagen tries to make the nodes look nicer. In the
final picture, the circles and rectangle should be filled with
different colors, resulting in the following code:

\begin{codeexample}[]
\begin{tikzpicture}[thick]
  \node at ( 0,2) [circle,draw=blue!50,fill=blue!20] {};
  \node at ( 0,1) [circle,draw=blue!50,fill=blue!20] {};   
  \node at ( 0,0) [circle,draw=blue!50,fill=blue!20] {};   
  \node at ( 1,1) [rectangle,draw=black!50,fill=black!20] {};
  \node at (-1,1) [rectangle,draw=black!50,fill=black!20] {};    
\end{tikzpicture}
\end{codeexample}

While this looks nicer in the picture, the code starts to get a bit
ugly. Ideally, we would like our code to transport the message ``there
are three places and two transitions'' and not so much which
filling colors should be used.

To solve this problem, Hagen uses styles. He defines a style for
places and another style for transitions:

\begin{codeexample}[]
\tikzstyle{place}=[circle,draw=blue!50,fill=blue!20,thick]
\tikzstyle{transition}=[rectangle,draw=black!50,fill=black!20,thick]
\begin{tikzpicture}
  \node at ( 0,2) [place] {};
  \node at ( 0,1) [place] {};   
  \node at ( 0,0) [place] {};   
  \node at ( 1,1) [transition] {};
  \node at (-1,1) [transition] {};    
\end{tikzpicture}
\end{codeexample}


\subsection{Node Size}

Before Hagen starts naming and connecting the nodes, let us first
make sure that the nodes get their final appearance. They are still
too small. Indeed, Hagen wonders why they have any size at all, after
all, the text is empty. The reason is than \tikzname\ automatically
adds some space around the text. The amount is set using the option
|inner sep|. So, to increase the size of the nodes, Hagen could write:

\begin{codeexample}[]
\tikzstyle{place}=[circle,draw=blue!50,fill=blue!20,thick]
\tikzstyle{transition}=[rectangle,draw=black!50,fill=black!20,thick]
\begin{tikzpicture}[inner sep=2mm]
  \node at ( 0,2) [place] {};
  \node at ( 0,1) [place] {};   
  \node at ( 0,0) [place] {};   
  \node at ( 1,1) [transition] {};
  \node at (-1,1) [transition] {};    
\end{tikzpicture}
\end{codeexample}

However, this is not really the best way to achieve the desired
effect. It is much better to use the |minimum size| option
instead. This option allows Hagen to specify a minimum size that the
node should have. If the nodes actually needs to be bigger because of
a longer text, it will be larger, but if the text is empty, then the
node will have |minimum size|. This option is also useful to ensure
that several nodes containing different amounts of text have the same
size. The options |minimum height| and |minimum width| allow you to
specify the minimum height and width independently. 

So, what Hagen needs to do is to provide |minimum size| for the
nodes. To be on the safe side, he also sets |inner sep=0pt|. This
ensures that the nodes will really have size |minimum size| and not,
for very small minimum sizes, the minimal size necessary to encompass
the automatically added space.

\begin{codeexample}[]
\tikzstyle{place}=[circle,draw=blue!50,fill=blue!20,thick,
                   inner sep=0pt,minimum size=6mm]
\tikzstyle{transition}=[rectangle,draw=black!50,fill=black!20,thick,
                        inner sep=0pt,minimum size=4mm]
\begin{tikzpicture}
  \node at ( 0,2) [place] {};
  \node at ( 0,1) [place] {};   
  \node at ( 0,0) [place] {};   
  \node at ( 1,1) [transition] {};
  \node at (-1,1) [transition] {};    
\end{tikzpicture}
\end{codeexample}




\subsection{Naming Nodes}

Hagen's next aim is to connect the nodes using arrows. This seems like
a tricky business since the arrows should not start in the middle of
the nodes, but somewhere on the border and Hagen would very much like
to avoid computing these positions by hand.

Fortunately, \pgfname\ will perform all the necessary calculations for
him. However, he first has to assign names to the nodes so that he can
reference them later on.

There are two ways to name a node. The first is the use the |name=|
option. The second method is to write the desired name in parentheses
after the |node| operation. Hagen thinks that this second method seems
strange, but he will soon change his opinion.

{
\tikzstyle{place}=[circle,draw=blue!50,fill=blue!20,thick,
                   inner sep=0pt,minimum size=6mm]
\tikzstyle{transition}=[rectangle,draw=black!50,fill=black!20,thick,
                        inner sep=0pt,minimum size=4mm]
\begin{codeexample}[]
% ... setup styles
\begin{tikzpicture}
  \node (waiting 1)  at ( 0,2)     [place] {};
  \node (critical 1) at ( 0,1)     [place] {};   
  \node (semaphore)  at ( 0,0)     [place] {};   
  \node (leave critical) at ( 1,1) [transition] {};
  \node (enter critical) at (-1,1) [transition] {};    
\end{tikzpicture}
\end{codeexample}
}

Hagen is pleased to note that the names help in understanding the
code. Names for nodes can be pretty arbitrary, but they should not
contain commas, periods, parentheses, colons, and some other special
characters. However, they can contain underscores and hyphens. 

The syntax for the |node| operation is quite liberal with respect to
the order in which node names, the |at| specifier, and the options
must come. Indeed, you can even have multiple option blocks between
the |node| and the text in curly braces, they accumulate. You can
rearrange them arbitrarily and perhaps the following might be preferable:

{
\tikzstyle{place}=[circle,draw=blue!50,fill=blue!20,thick,
                   inner sep=0pt,minimum size=6mm]
\tikzstyle{transition}=[rectangle,draw=black!50,fill=black!20,thick,
                        inner sep=0pt,minimum size=4mm]
\begin{codeexample}[]
\begin{tikzpicture}
  \node[place]      (waiting 1)      at ( 0,2) {};
  \node[place]      (critical 1)     at ( 0,1) {};   
  \node[place]      (semaphore)      at ( 0,0) {};   
  \node[transition] (leave critical) at ( 1,1) {};
  \node[transition] (enter critical) at (-1,1) {};    
\end{tikzpicture}
\end{codeexample}
}



\subsection{Placing Nodes Using Relative Placement}

Although Hagen still wishes to connect the nodes, he first wishes to
address another problem again: The placement of the nodes. Although he
likes the |at| syntax, in this particular case he would prefer placing
the nodes ``relative to each other.'' So, Hagen would like to say that
the |critical 1| node should be below the |waiting 1| node, wherever
the |waiting 1| node might be. There are different ways of achieving
this, but the nicest one in Hagen's case is the |below of| option:

{
\tikzstyle{place}=[circle,draw=blue!50,fill=blue!20,thick,
                   inner sep=0pt,minimum size=6mm]
\tikzstyle{transition}=[rectangle,draw=black!50,fill=black!20,thick,
                        inner sep=0pt,minimum size=4mm]
\begin{codeexample}[]
\begin{tikzpicture}
  \node[place]      (waiting)                            {};
  \node[place]      (critical)       [below of=waiting]  {};   
  \node[place]      (semaphore)      [below of=critical] {};   
  \node[transition] (leave critical) [right of=critical] {};
  \node[transition] (enter critical) [left of=critical]  {};    
\end{tikzpicture}
\end{codeexample}
}

The |below of| and similar options setup the position of the node in
such a manner that it is placed at the distance |node distance| in the
specified direction of the given direction. The |node distance| is the
distance between the centers of the nodes, not between the borders.

Even though the above code has the same effect the earlier code, Hagen
can pass it to his colleagues how will be able to just read and
understand it, perhaps without even having to see the picture.



\subsection{Adding Labels Next to Nodes}

Before we have a look at how Hagen can connect the nodes, let us add
the capacity ``$s \le 3$'' to the bottom node. For this, two
approaches are possible:
\begin{enumerate}
\item Hagen can just add a new node above the |north| anchor of the
  |semaphore| node.
{
\tikzstyle{place}=[circle,draw=blue!50,fill=blue!20,thick,
                   inner sep=0pt,minimum size=6mm]
\tikzstyle{transition}=[rectangle,draw=black!50,fill=black!20,thick,
                        inner sep=0pt,minimum size=4mm]
\begin{codeexample}[]
\begin{tikzpicture}
  \node[place]      (waiting)                            {};
  \node[place]      (critical)       [below of=waiting]  {};   
  \node[place]      (semaphore)      [below of=critical] {};   
  \node[transition] (leave critical) [right of=critical] {};
  \node[transition] (enter critical) [left of=critical]  {};    

  \node [red,above] at (semaphore.north) {$s\le 3$};   
\end{tikzpicture}
\end{codeexample}
}
This is a general approach that will ``always work.''

\item Hagen can use the special |label| option. This option is given
  to a |node| and it causes \emph{another} node to be added next to
  the node where the option is given. Here is the idea: When we
  construct the |semaphore| node, we wish to indicate that we want
  another node with the capacity above it. For this, we use the option
  |label=above:$s\le 3$|. This option is interpreted as follows: We
  want a node above the |semaphore| node and this node should read
  ``$s \le 3$.'' Instead of |above| we could also use things like
  |below left| before the colon or a number like |60|. 
{
\tikzstyle{place}=[circle,draw=blue!50,fill=blue!20,thick,
                   inner sep=0pt,minimum size=6mm]
\tikzstyle{transition}=[rectangle,draw=black!50,fill=black!20,thick,
                        inner sep=0pt,minimum size=4mm]
\begin{codeexample}[]
\begin{tikzpicture}
  \node[place]      (waiting)                            {};
  \node[place]      (critical)       [below of=waiting]  {};   
  \node[place]      (semaphore)      [below of=critical,
                                      label=above:$s\le3$] {};   
  \node[transition] (leave critical) [right of=critical] {};
  \node[transition] (enter critical) [left of=critical]  {};    
\end{tikzpicture}
\end{codeexample}
}
  It is also possible to give multiple |label| options, this causes
  multiple labels to be drawn.
\begin{codeexample}[]
\tikz
  \node [circle,draw,label=60:$60^\circ$,label=below:$-90^\circ$] {my circle};
\end{codeexample}
  Hagen is not fully satisfied with the |label| option since the label
  is not red. To achieve this, has has two options: First, he can
  redefine the |every label| style. Second, he can add options to the
  label's node. These options are given following the |label=|, so he
  would write |label=[red]above:$s\le3$|. However, this does not quite
  work since \TeX\ thinks that the |]| closes the whole option list of
  the |semaphore| node. So, Hagen has to add braces and writes
  |label={[red]above:$s\le3$}|. Since this looks a bit ugly, Hagen
  decides to redefine the |every label| style.
{
\tikzstyle{place}=[circle,draw=blue!50,fill=blue!20,thick,
                   inner sep=0pt,minimum size=6mm]
\tikzstyle{transition}=[rectangle,draw=black!50,fill=black!20,thick,
                        inner sep=0pt,minimum size=4mm]
\begin{codeexample}[]
\begin{tikzpicture}
  \tikzstyle{every label}=[red]    
  \node[place]      (waiting)                            {};
  \node[place]      (critical)       [below of=waiting]  {};   
  \node[place]      (semaphore)      [below of=critical,
                                      label=above:$s\le3$] {};   
  \node[transition] (leave critical) [right of=critical] {};
  \node[transition] (enter critical) [left of=critical]  {};    
\end{tikzpicture}
\end{codeexample}
}
\end{enumerate}



\subsection{Connecting Nodes}

It is now high time to connect the nodes. Let us start with something
simple, namely with the straight line from |enter critical| to
|critical|. We want this line to start at the right side of
|enter critical| and to end at the left side of |critical|. For
this, we can use the \emph{anchors} of the nodes. Every node defines a
whole bunch of anchors that lie on its border or inside it. For
example, the |center| anchor is at the center of the node, the |west|
anchor is on the left of the node, and so on. To access the coordinate
of a node, we use a coordinate that contains the node's name followed
by a dot, followed by the anchor's name:

{
\tikzstyle{place}=[circle,draw=blue!50,fill=blue!20,thick,
                   inner sep=0pt,minimum size=6mm]
\tikzstyle{transition}=[rectangle,draw=black!50,fill=black!20,thick,
                        inner sep=0pt,minimum size=4mm]
\begin{codeexample}[]
\begin{tikzpicture}
  \node[place]      (waiting)                            {};
  \node[place]      (critical)       [below of=waiting]  {};   
  \node[place]      (semaphore)      [below of=critical] {};   
  \node[transition] (leave critical) [right of=critical] {};
  \node[transition] (enter critical) [left of=critical]  {};    
  \draw [->] (critical.west) -- (enter critical.east);
\end{tikzpicture}
\end{codeexample}
}

Next, let us tackle the curve from |waiting| to |enter critical|. This
can be specified using curves and controls:

{
\tikzstyle{place}=[circle,draw=blue!50,fill=blue!20,thick,
                   inner sep=0pt,minimum size=6mm]
\tikzstyle{transition}=[rectangle,draw=black!50,fill=black!20,thick,
                        inner sep=0pt,minimum size=4mm]
\begin{codeexample}[]
\begin{tikzpicture}
  \node[place]      (waiting)                            {};
  \node[place]      (critical)       [below of=waiting]  {};   
  \node[place]      (semaphore)      [below of=critical] {};   
  \node[transition] (leave critical) [right of=critical] {};
  \node[transition] (enter critical) [left of=critical]  {};    
  \draw [->] (enter critical.east) -- (critical.west);
  \draw [->] (waiting.west) .. controls +(left:5mm) and +(up:5mm)
                            .. (enter critical.north);
\end{tikzpicture}
\end{codeexample}
}

Hagen sees how he can now add all his edges, but the whole process
seems a but awkward and not very flexible. Again, the code seems to
obscure the structure of the graphic rather than showing it.

So, let us start improving the code for the edges. First, Hagen can
leave out the anchors:

{
\tikzstyle{place}=[circle,draw=blue!50,fill=blue!20,thick,
                   inner sep=0pt,minimum size=6mm]
\tikzstyle{transition}=[rectangle,draw=black!50,fill=black!20,thick,
                        inner sep=0pt,minimum size=4mm]
\begin{codeexample}[]
\begin{tikzpicture}
  \node[place]      (waiting)                            {};
  \node[place]      (critical)       [below of=waiting]  {};   
  \node[place]      (semaphore)      [below of=critical] {};   
  \node[transition] (leave critical) [right of=critical] {};
  \node[transition] (enter critical) [left of=critical]  {};    
  \draw [->] (enter critical) -- (critical);
  \draw [->] (waiting) .. controls +(left:8mm) and +(up:8mm)
                       .. (enter critical);
\end{tikzpicture}
\end{codeexample}
}

Hagen is a bit surprised that this works. After all, how did
\tikzname\ know that the line from |enter critical| to |critical|
should actually start on the borders? Whenever \tikzname\ encounters a
whole node name as a ``coordinate,'' it tries to ``be smart'' about
the anchor that it should choose for this node. Depending on what
happens next, \tikzname\ will choose an anchor that lies on the border
of the node on a line to the next coordinate or control point. The
exact rules are a bit complex, but the chosen point will usually be
correct -- and when it is not, Hagen can still specify the desired
anchor by hand.

Hagen would now like to simplify the curve operation somehow. It turns
out that this can be accomplished using a special path operation: the
|to| operation. This operation takes many options (you can even define
new ones yourself). One pair of option is useful for Hagen: The pair
|in| and |out|. These options take angles at which a curve should
leave or reach the start or target coordinates. Without these options,
a straight line is drawn:

{
\tikzstyle{place}=[circle,draw=blue!50,fill=blue!20,thick,
                   inner sep=0pt,minimum size=6mm]
\tikzstyle{transition}=[rectangle,draw=black!50,fill=black!20,thick,
                        inner sep=0pt,minimum size=4mm]
\begin{codeexample}[]
\begin{tikzpicture}
  \node[place]      (waiting)                            {};
  \node[place]      (critical)       [below of=waiting]  {};   
  \node[place]      (semaphore)      [below of=critical] {};   
  \node[transition] (leave critical) [right of=critical] {};
  \node[transition] (enter critical) [left of=critical]  {};    
  \draw [->] (enter critical) to                 (critical);
  \draw [->] (waiting)        to [out=180,in=90] (enter critical);
\end{tikzpicture}
\end{codeexample}
}

There is another option for the |to| operation, that is even better
suited to Hagen's problem: The |bend right| option. This option also
takes an angle, but this angle only specifies the angle by which the
curve is bend to the right:

{
\tikzstyle{place}=[circle,draw=blue!50,fill=blue!20,thick,
                   inner sep=0pt,minimum size=6mm]
\tikzstyle{transition}=[rectangle,draw=black!50,fill=black!20,thick,
                        inner sep=0pt,minimum size=4mm]
\begin{codeexample}[]
\begin{tikzpicture}
  \node[place]      (waiting)                            {};
  \node[place]      (critical)       [below of=waiting]  {};   
  \node[place]      (semaphore)      [below of=critical] {};   
  \node[transition] (leave critical) [right of=critical] {};
  \node[transition] (enter critical) [left of=critical]  {};    
  \draw [->] (enter critical) to                 (critical);
  \draw [->] (waiting)        to [bend right=45] (enter critical);
  \draw [->] (enter critical) to [bend right=45] (semaphore);
\end{tikzpicture}
\end{codeexample}
}

It is now time for Hagen to learn about yet another way of specifying
edges: Using the |edge| path operation. This operation is very similar
to the |to| operation, but there is one important difference: Like a
node the edge generated by the |edge| operation is not part of the
main path, but is added only later. This may not seem very important,
but is has some nice consequences. For example, every edge can have
its own arrow tips and its own color and so one and, still, all the
edges can be given on the same path. This allows Hagen to write the
following: 


{
\tikzstyle{place}=[circle,draw=blue!50,fill=blue!20,thick,
                   inner sep=0pt,minimum size=6mm]
\tikzstyle{transition}=[rectangle,draw=black!50,fill=black!20,thick,
                        inner sep=0pt,minimum size=4mm]
\begin{codeexample}[]
\begin{tikzpicture}
  \node[place]      (waiting)                            {};
  \node[place]      (critical)       [below of=waiting]  {};   
  \node[place]      (semaphore)      [below of=critical] {};   
  \node[transition] (leave critical) [right of=critical] {};
  \node[transition] (enter critical) [left of=critical]  {}
    edge [->]               (critical)
    edge [<-,bend left=45]  (waiting)
    edge [->,bend right=45] (semaphore);
\end{tikzpicture}
\end{codeexample}
}

Each |edge| caused a new path to be constructed, consisting of a |to|
between the node |enter critical| and the node following the |edge|
command.

The finishing touch is to introduce two styles |pre| and |post| and to
use the |bend angle=45| option to set the bend angle once and for all:

{
\tikzstyle{place}=[circle,draw=blue!50,fill=blue!20,thick,
                   inner sep=0pt,minimum size=6mm]
\tikzstyle{transition}=[rectangle,draw=black!50,fill=black!20,thick,
                        inner sep=0pt,minimum size=4mm]
\begin{codeexample}[]
% Styles place and transition as before
\tikzstyle{pre}=[<-,shorten <=1pt,>=stealth',semithick]  
\tikzstyle{post}=[->,shorten >=1pt,>=stealth',semithick]  
\begin{tikzpicture}[bend angle=45]
  \node[place]      (waiting)                            {};
  \node[place]      (critical)       [below of=waiting]  {};   
  \node[place]      (semaphore)      [below of=critical] {};   

  \node[transition] (leave critical) [right of=critical] {}
    edge [pre]             (critical)
    edge [post,bend right] (waiting)
    edge [pre, bend left]  (semaphore);
  \node[transition] (enter critical) [left of=critical]  {}
    edge [post]            (critical)
    edge [pre, bend left]  (waiting)
    edge [post,bend right] (semaphore);
\end{tikzpicture}
\end{codeexample}
}




\subsection{Adding Labels Next to Lines}

The next thing that Hagen needs to add is the ``$2$'' at the arcs. For
this Hagen can use \tikzname's automatic node placement: By adding the
option |auto|, \tikzname\ will position nodes on curves and lines in
such a way that they are not on the curve but next to it. Adding
|swap| will mirror the label with respect to the line. Here is a
general example:

{
\begin{codeexample}[]
\begin{tikzpicture}[auto,bend right]
  \node (a) at (0:1) {$0^\circ$};
  \node (b) at (120:1) {$120^\circ$};
  \node (c) at (240:1) {$240^\circ$};

  \draw (a) to node {1} node [swap] {1'} (b)
        (b) to node {2} node [swap] {2'} (c)
        (c) to node {3} node [swap] {3'} (a);
\end{tikzpicture}
\end{codeexample}
}

What is happening here? The nodes are given somehow inside the |to|
operation! When this is done, the node is placed on the middle of the
curve or line created by the |to| operation. The |auto| option then
causes the node to be moved in such a way that it does not lie on the
curve, but next to it. In the example we provide even two nodes on
each |to| operation.

For Hagen that |auto| option is not really necessary since the two
``2'' labels could also easily be placed ``by hand.'' However, in a
complicated plot with numerous edges automatic placement can be a
blessing. 

{
\tikzstyle{place}=[circle,draw=blue!50,fill=blue!20,thick,
                   inner sep=0pt,minimum size=6mm]
\tikzstyle{transition}=[rectangle,draw=black!50,fill=black!20,thick,
                        inner sep=0pt,minimum size=4mm]
\tikzstyle{pre}=[<-,shorten <=1pt,>=stealth',semithick]  
\tikzstyle{post}=[->,shorten >=1pt,>=stealth',semithick]  
\begin{codeexample}[]
% Styles as before
\begin{tikzpicture}[bend angle=45]
  \node[place]      (waiting)                            {};
  \node[place]      (critical)       [below of=waiting]  {};   
  \node[place]      (semaphore)      [below of=critical] {};   

  \node[transition] (leave critical) [right of=critical] {}
    edge [pre]                                 (critical)
    edge [post,bend right] node[auto,swap] {2} (waiting)
    edge [pre, bend left]                      (semaphore);
  \node[transition] (enter critical) [left of=critical]  {}
    edge [post]                                (critical)
    edge [pre, bend left]                      (waiting)
    edge [post,bend right]                     (semaphore);
\end{tikzpicture}
\end{codeexample}
}



\subsection{Adding the Snaked Line and Multi-Line Text}

With the node mechanism Hagen can now easily create the two Petri
nets. What he is unsure of is how he can create the snaked line
between the nets.

For this he can use a \emph{snake}. Snakes a called thus since the
most basic form of a snake looks exactly like a snake. However, and
repeating pattern can be used as a snake like bumps or a saw or even
much more complicated stuff.

To draw the snake, Hagen only needs to set the |snake=snake| option on
the path. This causes all straight lines of the path to be replaced by
snakes. It is also possible to use snakes only in certain parts of a
path, but Hagen will not need this.

\begin{codeexample}[]
\begin{tikzpicture}
  \draw [->,snake=snake] (0,0) -- (2,0);
\end{tikzpicture}
\end{codeexample}

Well, that does not look quite right, yet. The problem is that the
snake happens to end exactly at the position where the arrow
begins. Fortunately, there is an option that helps here. Also, the
snake should be a bit smaller, which can be influenced by even more
options. 

\begin{codeexample}[]
\begin{tikzpicture}
  \draw [->,snake=snake,
         segment amplitude=.4mm,
         segment length=2mm,
         line after snake=1mm] (0,0) -- (3,0);
\end{tikzpicture}
\end{codeexample}

Now Hagen needs to add the text above the snake. This text is a bit
challenging since it is a multi-line text. To typeset such text, Hagen
needs to specify a width for the text and he needs to specify that the
text should be centered.


\begin{codeexample}[]
\begin{tikzpicture}
  \draw [->,snake=snake,
         segment amplitude=.4mm,
         segment length=2mm,
         line after snake=1mm] (0,0) -- (3,0)
    node [above,text width=3cm,text centered,midway]
    {
      replacement of the \textcolor{red}{capacity} by
      \textcolor{red}{two places}
    };
\end{tikzpicture}
\end{codeexample}



\subsection{Using Layers: The Background Rectangles}

Hagen still needs to add the background rectangles. These are a bit
tricky: Hagen would like to draw the rectangles \emph{after} the Petri
nets are finished. The reason is that only then can he conveniently
refer to the coordinates that make up the corners of the
rectangle. If Hagen draws the rectangle first, then he needs to know
the exact size of the Petri net -- which he does not.

The solution is to use \emph{layers}. When the background library is
loaded, Hagen can put parts of his picture inside a |{pgfonlayer}|
environment. Then this part of the picture becomes part of the layer
that is given as an argument to this environment. When the
|{tikzpicture}| environment ends, the layers are put on top of each
other, starting with the background layer. This causes everything
drawn on the background layer to be behind the main text.


{
\tikzstyle{place}=[circle,draw=blue!50,fill=blue!20,thick,
                   inner sep=0pt,minimum size=6mm]
\tikzstyle{transition}=[rectangle,draw=black!50,fill=black!20,thick,
                        inner sep=0pt,minimum size=4mm]
\tikzstyle{pre}=[<-,shorten <=1pt,>=stealth',semithick]  
\tikzstyle{post}=[->,shorten >=1pt,>=stealth',semithick]  
\begin{codeexample}[]
% Styles as before
\begin{tikzpicture}[bend angle=45]
  \node[place]      (waiting)                            {};
  \node[place]      (critical)       [below of=waiting]  {};   
  \node[place]      (semaphore)      [below of=critical] {};   

  \node[transition] (leave critical) [right of=critical] {}
    edge [pre]                                 (critical)
    edge [post,bend right] node[auto,swap] {2} (waiting)
    edge [pre, bend left]                      (semaphore);
  \node[transition] (enter critical) [left of=critical]  {}
    edge [post]                                (critical)
    edge [pre, bend left]                      (waiting)
    edge [post,bend right]                     (semaphore);

  \begin{pgfonlayer}{background}
    \filldraw [fill=black!30,draw=red]
                (semaphore.south -| enter critical.west)
      rectangle (waiting.north   -| leave critical.east);
  \end{pgfonlayer}
\end{tikzpicture}
\end{codeexample}
}




\subsection{The Complete Code}

Hagen has now finally put everything together. Only then does he learn
that there is already a library for drawing Petri nets! It turns out
that this library mainly provides the same definitions as Hagen
did. For example, it defines a |place| style in a similar way as Hagen
did. Adjusting the code so that it uses the library shortens Hagen
code a bit, as shown in the following.

First, Hagen needs less style definitions, but he still needs to
specify the colors of places and transitions.

\begin{codeexample}[code only]
\tikzstyle{every place}=     [minimum size=6mm,thick,draw=blue!75,fill=blue!20]
\tikzstyle{every transition}=[thick,draw=black!75,fill=black!20]

\tikzstyle{red place}=       [place,draw=red!75,fill=red!20]

\tikzstyle{every label}=     [red]
\begin{tikzpicture}[node distance=1.3cm,>=stealth',bend angle=45,auto]
\end{codeexample}

Now comes the code for the nets:

{
\tikzstyle{every place}=[minimum size=6mm,thick,draw=blue!75,fill=blue!20]
\tikzstyle{every transition}=[thick,draw=black!75,fill=black!20]

\tikzstyle{red place}=  [place,draw=red!75,fill=red!20]

\tikzstyle{every label}=[red]

\tikzstyle{every picture}=[node distance=1.3cm,>=stealth',bend angle=45,auto]
\begin{codeexample}[pre=\begin{tikzpicture},post=\end{tikzpicture}]
  \node [place,tokens=1] (w1)                                    {};
  \node [place]          (c1) [below of=w1]                      {};
  \node [place]          (s)  [below of=c1,label=above:$s\le 3$] {};
  \node [place]          (c2) [below of=s]                       {};
  \node [place,tokens=1] (w2) [below of=c2]                      {};
  
  \node [transition] (e1) [left of=c1] {}
    edge [pre,bend left]                  (w1)
    edge [post,bend right]                (s)
    edge [post]                           (c1);
  \node [transition] (e2) [left of=c2] {}
    edge [pre,bend right]                 (w2)
    edge [post,bend left]                 (s)
    edge [post]                           (c2);
  \node [transition] (l1) [right of=c1] {}
    edge [pre]                            (c1)
    edge [pre,bend left]                  (s)
    edge [post,bend right] node[swap] {2} (w1);
  \node [transition] (l2) [right of=c2] {}
    edge [pre]                            (c2)
    edge [pre,bend right]                 (s)
    edge [post,bend left]  node {2}       (w2);
\end{codeexample}
}

{
\tikzstyle{every place}=     [minimum size=6mm,thick,draw=blue!75,fill=blue!20]
\tikzstyle{every transition}=[thick,draw=black!75,fill=black!20]

\tikzstyle{red place}=  [place,draw=red!75,fill=red!20]

\tikzstyle{every label}=[red]

\tikzstyle{every picture}=[node distance=1.3cm,>=stealth',bend angle=45,auto]
\begin{codeexample}[pre=\begin{tikzpicture},post=\end{tikzpicture}]
  \begin{scope}[xshift=6cm]    
    \node [place,tokens=1]     (w1')                            {};
    \node [place]              (c1') [below of=w1']             {};
    \node [red place]          (s1') [below of=c1',xshift=-5mm]
            [label=left:$s$]                                    {};
    \node [red place,tokens=3] (s2') [below of=c1',xshift=5mm]
            [label=right:$\bar s$]                              {};
    \node [place]              (c2') [below of=s1',xshift=5mm]  {};
    \node [place,tokens=1]     (w2') [below of=c2']             {};
    
    \node [transition] (e1') [left of=c1'] {}
      edge [pre,bend left]                  (w1')
      edge [post]                           (s1')
      edge [pre]                            (s2')
      edge [post]                           (c1');
    \node [transition] (e2') [left of=c2'] {}
      edge [pre,bend right]                 (w2')
      edge [post]                           (s1')
      edge [pre]                            (s2')
      edge [post]                           (c2');
    \node [transition] (l1') [right of=c1'] {}
      edge [pre]                            (c1')
      edge [pre]                            (s1')
      edge [post]                           (s2')
      edge [post,bend right] node[swap] {2} (w1');
    \node [transition] (l2') [right of=c2'] {}
      edge [pre]                            (c2')
      edge [pre]                            (s1')
      edge [post]                           (s2')
      edge [post,bend left]  node {2}       (w2');
  \end{scope}
\end{codeexample}
}

The code for the background and the snake is the following:

\begin{codeexample}[code only]
  \draw [-to,thick,snake=snake,segment amplitude=.4mm,segment length=2mm,line after snake=1mm]
    ([xshift=5mm]s -| l1) -- ([xshift=-5mm]s1' -| e1')
    node [above=1mm,midway,text width=3cm,text centered]
      {replacement of the \textcolor{red}{capacity} by \textcolor{red}{two places}};

  \begin{pgfonlayer}{background}
    \filldraw [line width=4mm,join=round,black!10]
      (w1.north  -| l1.east)  rectangle (w2.south  -| e1.west)
      (w1'.north -| l1'.east) rectangle (w2'.south -| e1'.west);
  \end{pgfonlayer}
\end{tikzpicture}
\end{codeexample}

% Copyright 2006 by Till Tantau
%
% This file may be distributed and/or modified
%
% 1. under the LaTeX Project Public License and/or
% 2. under the GNU Free Documentation License.
%
% See the file doc/generic/pgf/licenses/LICENSE for more details.


\section{Tutorial: Euclid's Amber Version of the \emph{Elements}}

In this third tutorial we have a look at how \tikzname\ can be used to
draw geometric constructions.

Euclid is currently quite busy writing his new book series, whose
working title is ``Elements'' (Euclid is not quite sure whether this
title will convey the message of the series to future generations
correctly, but he intends to change the title before it goes to the
publisher). Up to know, he wrote down his text and graphics on
papyrus, but his publisher suddenly insists that he must submit in
electronic form. Euclid tries to argue with the publisher that 
electronics will only be discovered thousands of years later, but the
publisher informs him that the use of papyrus is no longer cutting edge
technology and Euclid will just have to keep up with modern tools.

Slightly disgruntled, Euclid starts converting his papyrus
entitled ``Book I, Proposition I'' to an amber version.  

\subsection{Book I, Proposition I}

The drawing on his papyrus looks like this:\footnote{The text is taken
from the wonderful interactive version of Euclid's Elements by David
E. Joyce, to be found on his website at Clark University.}

\bigskip
\noindent
\begin{tikzpicture}[thick,help lines/.style={thin,draw=black!50}]
  \def\A{\textcolor{input}{$A$}}
  \def\B{\textcolor{input}{$B$}}
  \def\C{\textcolor{output}{$C$}}
  \def\D{$D$}
  \def\E{$E$}
  
  \colorlet{input}{blue!80!black}
  \colorlet{output}{red!70!black}
  \colorlet{triangle}{orange}
  
  \coordinate [label=left:\A]
    (A) at ($ (0,0) + .1*(rand,rand) $);
  \coordinate [label=right:\B]
    (B) at ($ (1.25,0.25) + .1*(rand,rand) $);

  \draw [input] (A) -- (B);
  
  \node [name path=D,help lines,draw,label=left:\D] (D) at (A) [circle through=(B)] {};
  \node [name path=E,help lines,draw,label=right:\E] (E) at (B) [circle through=(A)] {};
  
  \path [name intersections={of=D and E,by={[label=above:\C]C}}];

  \draw [output] (A) -- (C);
  \draw [output] (B) -- (C);

  \foreach \point in {A,B,C}
    \fill [black,opacity=.5] (\point) circle (2pt);

  \begin{pgfonlayer}{background}
    \fill[triangle!80] (A) -- (C) -- (B) -- cycle;
  \end{pgfonlayer}
  
  \node [below right,text width=10cm,align=justify] at (4,3)
  {
    \small
    \textbf{Proposition I}\par
    \emph{To construct an \textcolor{triangle}{equilateral triangle}
      on a given \textcolor{input}{finite straight line}.}
    \par
    \vskip1em
    Let \A\B\ be the given \textcolor{input}{finite straight line}. It
    is required to construct an \textcolor{triangle}{equilateral
      triangle} on the \textcolor{input}{straight line}~\A\B. 

    Describe the circle \B\C\D\ with center~\A\ and radius \A\B. Again
    describe the circle \A\C\E\ with center~\B\ and radius \B\A. Join the
    \textcolor{output}{straight lines} \C\A\ and \C\B\ from the
    point~\C\ at which the circles cut one another to the points~\A\ and~\B.

    Now, since the point~\A\ is the center of the circle \C\D\B,
    therefore \A\C\ equals \A\B. Again, since the point \B\ is the
    center of the circle \C\A\E, therefore \B\C\ equals \B\A. But
    \A\C\ was proved equal to \A\B, therefore each of the straight
    lines \A\C\ and \B\C\ equals \A\B. And 
    things which equal the same thing also equal one another,
    therefore \A\C\ also equals \B\C. Therefore the three straight
    lines \A\C, \A\B, and \B\C\ equal one another. 
    Therefore the \textcolor{triangle}{triangle} \A\B\C\ is
    equilateral, and it has been  constructed on the given finite
    \textcolor{input}{straight line}~\A\B.  
  };
\end{tikzpicture}
\bigskip

Let us have a look at how Euclid can turn this into \tikzname\ code.

\subsubsection{Setting up the Environment}

As in the previous tutorials, Euclid needs to load \tikzname, together
with some libraries. These libraries are |calc|, |intersections|,
|through|, and |backgrounds|. Depending on which format he uses,
Euclid would use one of the following in the preamble: 

\begin{codeexample}[code only]
% For LaTeX:
\usepackage{tikz}
\usetikzlibrary{calc,intersections,through,backgrounds}
\end{codeexample}

\begin{codeexample}[code only]
% For plain TeX:
\input tikz.tex
\usetikzlibrary{calc,intersections,through,backgrounds}
\end{codeexample}

\begin{codeexample}[code only]
% For ConTeXt:
\usemodule[tikz]
\usetikzlibrary[calc,intersections,through,backgrounds]
\end{codeexample}


\subsubsection{The Line \emph{AB}}

The first part of the picture that Euclid wishes to draw is the line
$AB$. That is easy enough, something like |\draw (0,0) -- (2,1);|
might do. However, Euclid does not wish to reference the two points
$A$ and $B$ as $(0,0)$ and $(2,1)$ subsequently. Rather, he wishes to
just write |A| and |B|. Indeed, the whole point of his book is that
the points $A$ and $B$ can be arbitrary and all other points (like
$C$) are constructed in terms of their positions. It would not do
if Euclid were to write down the coordinates of $C$ explicitly.

So, Euclid starts with defining two coordinates using the
|\coordinate| command:
\begin{codeexample}[]
\begin{tikzpicture}
  \coordinate (A) at (0,0);
  \coordinate (B) at (1.25,0.25);

  \draw[blue] (A) -- (B);
\end{tikzpicture}
\end{codeexample}

That was easy enough. What is missing at this point are the labels for
the coordinates. Euclid does not want them \emph{on} the points, but
next to them. He decides to use the |label| option:
\begin{codeexample}[]
\begin{tikzpicture}
  \coordinate [label=left:\textcolor{blue}{$A$}]  (A) at (0,0);
  \coordinate [label=right:\textcolor{blue}{$B$}] (B) at (1.25,0.25);

  \draw[blue] (A) -- (B);
\end{tikzpicture}
\end{codeexample}

At this point, Euclid decides that it would be even nicer if the
points $A$ and $B$ were in some sense ``random.'' Then, neither Euclid
nor the reader can make the mistake of taking ``anything for granted''
concerning these position of these points. Euclid is pleased to learn
that there is a |rand| function in \tikzname\ that does exactly what
he needs: It produces a number between $-1$ and $1$. Since \tikzname\
can do a bit of math, Euclid can change the coordinates of the points
as follows:
\begin{codeexample}[code only]
\coordinate [...] (A) at (0+0.1*rand,0+0.1*rand);
\coordinate [...] (B) at (1.25+0.1*rand,0.25+0.1*rand);
\end{codeexample}

This works fine. However, Euclid is not quite satisfied since he would
prefer that the ``main coordinates'' $(0,0)$ and $(1.25,0.25)$ are
``kept separate'' from the perturbation
$0.1(\mathit{rand},\mathit{rand})$. This means, he would like to
specify that coordinate $A$ as ``The point that is at $(0,0)$ plus one
tenth of the vector  $(\mathit{rand},\mathit{rand})$.''

It turns out that the |calc| library allows him to do exactly this
kind of computation. When this library is loaded, you can use special
coordinates that start with |($| and end with |$)| rather than just
|(| and~|)|. Inside these special coordinates you can give a linear
combination of coordinates. (Note that the dollar signs are only
intended to signal that a ``computation'' is going on; no mathematical
typesetting is done.)

The new code for the coordinates is the following:

\begin{codeexample}[code only]
\coordinate [...] (A) at ($ (0,0) + .1*(rand,rand) $);
\coordinate [...] (B) at ($ (1.25,0.25) + .1*(rand,rand) $);
\end{codeexample}

Note that if a coordinate in such a computation has a factor (like
|.1|) you must place a |*| directly before the opening parenthesis of
the coordinate. You can nest such computations.



\subsubsection{The Circle Around \emph{A}}

The first tricky construction is the circle around~$A$. We will see
later how to do this in a very simple manner, but first let us do it
the ``hard'' way.

The idea is the following: We draw a circle around the point $A$ whose
radius is given by the length of the line $AB$. The difficulty lies in
computing the length of this line.

Two ideas ``nearly'' solve this problem: First, we can write
|($ (A) - (B) $)| for the vector that is the difference between $A$
and~$B$. All we need is the length of this vector. Second, given two
numbers $x$ and $y$, one can write |veclen(|$x$|,|$y$|)| inside a
mathematical expression. This gives the value $\sqrt{x^2+y^2}$, which
is exactly the desired length.

The only remaining problem is to access the $x$- and $y$-coordinate of
the vector~$AB$. For this, we need a new concept: the \emph{let
  operation}. A let operation can be given anywhere on a path where a
normal path operation like a line-to or a move-to is expected. The
effect of a let operation is to evaluate some coordinates and to
assign the results to special macros. These macros make it easy to
access the $x$- and $y$-coordinates of the coordinates.

Euclid would write the following:
\begin{codeexample}[]
\begin{tikzpicture}
  \coordinate [label=left:$A$]  (A) at (0,0);
  \coordinate [label=right:$B$] (B) at (1.25,0.25);
  \draw (A) -- (B);

  \draw (A) let
              \p1 = ($ (B) - (A) $)
            in
              circle ({veclen(\x1,\y1)});
\end{tikzpicture}
\end{codeexample}

Each assignment in a let operation starts with |\p|, usually followed
by a \meta{digit}. Then comes an equal sign and a coordinate. The
coordinate is evaluated and the result is stored internally. From
then on you can use the following expressions: 
\begin{enumerate}
\item |\x|\meta{digit} yields the $x$-coordinate of the resulting point.
\item |\y|\meta{digit} yields the $y$-coordinate of the resulting
  point.
\item |\p|\meta{digit} yields the same as |\x|\meta{digit}|,\y|\meta{digit}.
\end{enumerate}
You can have multiple assignments in a let operation, just separate
them with commas. In later assignments you can already use the results
of earlier assignments.

Note that |\p1| is not a coordinate in the usual sense. Rather, it
just expands to a string like |10pt,20pt|. So, you cannot write, for
instance, |(\p1.center)| since this would just expand to
|(10pt,20pt.center)|, which makes no sense.

Next, we want to draw both circles at the same time. Each time the
radius is |veclen(\x1,\y1)|. It seems natural to compute this radius
only once. For this, we can also use a let operation: Instead of
writing |\p1 = ...|, we write |\n2 = ...|. Here, ``n'' stands for
``number'' (while ``p'' stands for ``point''). The assignment of a
number should be followed by a number in curly braces.
\begin{codeexample}[]
\begin{tikzpicture}
  \coordinate [label=left:$A$]  (A) at (0,0);
  \coordinate [label=right:$B$] (B) at (1.25,0.25);
  \draw (A) -- (B);

  \draw let \p1 = ($ (B) - (A) $),
            \n2 = {veclen(\x1,\y1)}
        in
          (A) circle (\n2)
          (B) circle (\n2);
\end{tikzpicture}
\end{codeexample}
In the above example, you may wonder, what |\n1| would yield? The
answer is that it would be undefined -- the |\p|, |\x|, and |\y|
macros refer to the same logical point, while the |\n| macro has ``its
own namespace.'' We could even have replaced |\n2| in the example by
|\n1| and it would still work. Indeed, the digits following these
macros are just normal \TeX\ parameters. We could also use a longer
name, but then we have to use curly braces:
\begin{codeexample}[]
\begin{tikzpicture}
  \coordinate [label=left:$A$]  (A) at (0,0);
  \coordinate [label=right:$B$] (B) at (1.25,0.25);
  \draw (A) -- (B);

  \draw let \p1        = ($ (B) - (A) $),
            \n{radius} = {veclen(\x1,\y1)}
        in
          (A) circle (\n{radius})
          (B) circle (\n{radius});
\end{tikzpicture}
\end{codeexample}

At the beginning of this section it was promised that there is an
easier way to create the desired circle. The trick is to use the
|through| library. As the name suggests, it contains code for creating
shapes that go through a given point.

The option that we are looking for is |circle through|. This option is
given to a \emph{node} and has the following effects: First, it causes
the node's inner and outer separations to be set to zero. Then it sets
the shape of the node to |circle|. Finally, it sets the radius of the
node such that it goes through the parameter given to
|circle through|. This radius is computed in essentially the same way
as above.

\begin{codeexample}[]
\begin{tikzpicture}
  \coordinate [label=left:$A$]  (A) at (0,0);
  \coordinate [label=right:$B$] (B) at (1.25,0.25);
  \draw (A) -- (B);

  \node [draw,circle through=(B),label=left:$D$] at (A) {};
\end{tikzpicture}
\end{codeexample}


\subsubsection{The Intersection of the Circles}

Euclid can now draw the line and the circles. The final problem is to
compute the intersection of the two circles. This computation is a bit
involved if you want to do it ``by hand.'' Fortunately, the
intersection library allows us to compute the intersection of
arbitrary paths.

The idea is simple: First, you ``name'' two paths using the
|name path| option. Then, at some later point, you can use the option
|name intersections|, which creates coordinates called
|intersection-1|, |intersection-2|, and so on at all intersections of
the paths. Euclid assigns the names |D| and |E| to the paths of the
two circles (which happen to be the same names as the nodes
themselves, but nodes and their paths live in different
``namespaces''). 
\begin{codeexample}[]
\begin{tikzpicture}
  \coordinate [label=left:$A$]  (A) at (0,0);
  \coordinate [label=right:$B$] (B) at (1.25,0.25);
  \draw (A) -- (B);

  \node (D) [name path=D,draw,circle through=(B),label=left:$D$]  at (A) {};
  \node (E) [name path=E,draw,circle through=(A),label=right:$E$] at (B) {};

  % Name the coordinates, but do not draw anything:
  \path [name intersections={of=D and E}];
  
  \coordinate [label=above:$C$] (C) at (intersection-1);

  \draw [red] (A) -- (C);
  \draw [red] (B) -- (C);
\end{tikzpicture}
\end{codeexample}

It turns out that this can be further shortened: The
|name intersections| takes an optional argument |by|, which lets you
specify names for the coordinates and options for them. This creates
more compact code. Although Euclid does not need it for the current
picture, it is just a small step to computing the bisection of the line $AB$:

\begin{codeexample}[]
\begin{tikzpicture}
  \coordinate [label=left:$A$]  (A) at (0,0);
  \coordinate [label=right:$B$] (B) at (1.25,0.25);
  \draw [name path=A--B] (A) -- (B);

  \node (D) [name path=D,draw,circle through=(B),label=left:$D$]  at (A) {};
  \node (E) [name path=E,draw,circle through=(A),label=right:$E$] at (B) {};

  \path [name intersections={of=D and E, by={[label=above:$C$]C, [label=below:$C'$]C'}}];

  \draw [name path=C--C',red] (C) -- (C');

  \path [name intersections={of=A--B and C--C',by=F}];
  \node [fill=red,inner sep=1pt,label=-45:$F$] at (F) {};
\end{tikzpicture}
\end{codeexample}



\subsubsection{The Complete Code}

Back to Euclid's code. He introduces a few macros to make life
simpler, like a |\A| macro for typesetting a blue $A$. He also uses the
|background| layer for drawing the triangle behind everything at the
end. 

\begin{codeexample}[]
\begin{tikzpicture}[thick,help lines/.style={thin,draw=black!50}]
  \def\A{\textcolor{input}{$A$}}     \def\B{\textcolor{input}{$B$}}
  \def\C{\textcolor{output}{$C$}}    \def\D{$D$}
  \def\E{$E$}
  
  \colorlet{input}{blue!80!black}    \colorlet{output}{red!70!black}
  \colorlet{triangle}{orange}
  
  \coordinate [label=left:\A]  (A) at ($ (0,0) + .1*(rand,rand) $);
  \coordinate [label=right:\B] (B) at ($ (1.25,0.25) + .1*(rand,rand) $);

  \draw [input] (A) -- (B);
  
  \node [name path=D,help lines,draw,label=left:\D]   (D) at (A) [circle through=(B)] {};
  \node [name path=E,help lines,draw,label=right:\E]  (E) at (B) [circle through=(A)] {};
  
  \path [name intersections={of=D and E,by={[label=above:\C]C}}];

  \draw [output] (A) -- (C) -- (B);

  \foreach \point in {A,B,C}
    \fill [black,opacity=.5] (\point) circle (2pt);

  \begin{pgfonlayer}{background}
    \fill[triangle!80] (A) -- (C) -- (B) -- cycle;
  \end{pgfonlayer}
  
  \node [below right, text width=10cm,align=justify] at (4,3) {
    \small\textbf{Proposition I}\par
    \emph{To construct an \textcolor{triangle}{equilateral triangle}
      on a given \textcolor{input}{finite straight line}.}
    \par\vskip1em
    Let \A\B\ be the given \textcolor{input}{finite straight line}.  \dots
  };
\end{tikzpicture}
\end{codeexample}


\subsection{Book I, Proposition II}

The second proposition in the Elements is the following:

\bigskip\noindent
\begin{tikzpicture}[thick,help lines/.style={thin,draw=black!50}]
  \def\A{\textcolor{orange}{$A$}}   \def\B{\textcolor{input}{$B$}}
  \def\C{\textcolor{input}{$C$}}    \def\D{$D$}
  \def\E{$E$}                       \def\F{$F$}
  \def\G{$G$}                       \def\H{$H$}
  \def\K{$K$}                       \def\L{\textcolor{output}{$L$}}
  
  \colorlet{input}{blue!80!black}    \colorlet{output}{red!70!black}
  
  \coordinate [label=left:\A]  (A) at ($ (0,0) + .1*(rand,rand) $);
  \coordinate [label=right:\B] (B) at ($ (1,0.2) + .1*(rand,rand) $);
  \coordinate [label=above:\C] (C) at ($ (1,2) + .1*(rand,rand) $);

  \draw [input] (B) -- (C);
  \draw [help lines] (A) -- (B);

  \coordinate [label=above:\D] (D) at ($ (A)!.5!(B) ! {sin(60)*2} ! 90:(B) $);

  \draw [help lines] (D) -- ($ (D)!3.75!(A) $) coordinate [label=-135:\E] (E);
  \draw [help lines] (D) -- ($ (D)!3.75!(B) $) coordinate [label=-45:\F] (F);

  \node (H) at (B) [name path=H,help lines,circle through=(C),draw,label=135:\H] {};
  \path [name path=B--F] (B) -- (F);
  \path [name intersections={of=H and B--F}]
    coordinate [label=right:\G] (G) at (intersection-1);

  \node (K) at (D) [name path=K,help lines,circle through=(G),draw,label=135:\K] {};

  \path [name path=A to E line] (A) -- (E);
  \path [name intersections={of=K and A to E line}]
    coordinate [label=below:\L] (L) at (intersection-1);

  \draw [output] (A) -- (L);

  \foreach \point in {A,B,C,D,G,L}
    \fill [black,opacity=.5] (\point) circle (2pt);
  
  \node [below right, text width=9cm,align=justify] at (4,4) {
    \small\textbf{Proposition II}\par
    \emph{To place a \textcolor{output}{straight line} equal to a
      given \textcolor{input}{straight line} with 
      one end at a \textcolor{orange}{given point}.} 
    \par\vskip1em
    Let \A\ be the given point, and \B\C\ the given
    \textcolor{input}{straight line}. 
    It is required to place a \textcolor{output}{straight line} equal
    to the given \textcolor{input}{straight line} \B\C\ with one end
    at the point~\A.  

    Join the straight line \A\B\ from the point \A\ to the point \B, and
    construct the equilateral triangle \D\A\B\ on it.
    
    Produce the straight lines \A\E\ and \B\F\ in a straight line with
    \D\A\ and \D\B. Describe the circle \C\G\H\ with center \B\ and
    radius \B\C, and  again, describe the circle \G\K\L\ with center
    \D\ and radius \D\G. 	

    Since the point \B\ is the center of the circle \C\G\H, therefore
    \B\C\ equals \B\G. Again, since the point \D\ is the center of the
    circle \G\K\L, therefore \D\L\ equals \D\G. And in these \D\A\
    equals \D\B, therefore the remainder \A\L\ equals the remainder
    \B\G. But \B\C\ was also proved  equal to \B\G, therefore each of
    the straight lines \A\L\ and \B\C\ equals \B\G. And things which
    equal the same thing also equal one another, therefore \A\L\ also
    equals \B\C. 
    
    Therefore the \textcolor{output}{straight line} \A\L\ equal to the
    given \textcolor{input}{straight line} \B\C\  has been placed with
    one end at the \textcolor{orange}{given point}~\A.  
  };
\end{tikzpicture}




\subsubsection{Using Partway Calculations for the Construction of \emph{D}}

Euclid's construction starts with ``referencing'' Proposition~I for
the construction of the point~$D$. Now, while we could simply repeat the
construction, it seems a bit bothersome that one has to draw all these
circles and do all these complicated constructions.

For this reason, \tikzname\ supports some simplifications. First,
there is a simple syntax for computing a point that is ``partway'' on
a line from $p$ to~$q$: You place these two points in a coordinate
calculation -- remember, they start with |($| and end with |$)| -- and
then combine them using |!|\meta{part}|!|. A \meta{part} of |0| refers
to the \emph{first} coordinate, a \meta{part} of |1| refers to the
second coordinate, and a value in between refers to a point on the
line from $p$ to~$q$. Thus, the syntax is similar to the |xcolor|
syntax for mixing colors.

Here is the computation of the point in the middle of the line $AB$:
\begin{codeexample}[]
\begin{tikzpicture}
  \coordinate [label=left:$A$]  (A) at (0,0);
  \coordinate [label=right:$B$] (B) at (1.25,0.25);
  \draw (A) -- (B);
  \node [fill=red,inner sep=1pt,label=below:$X$] (X) at ($ (A)!.5!(B) $) {};
\end{tikzpicture}
\end{codeexample}

The computation of the point $D$ in Euclid's second proposition is a
bit more complicated. It can be expressed as follows: Consider the
line from $X$ to $B$. Suppose we 
rotate this line around $X$ for 90$^\circ$ and then stretch it by a
factor of $\sin(60^\circ)/2$. This yields the desired point~$D$. We
can do the stretching using the partway modifier above, for the
rotation we need a new modifier: the rotation modifier. The idea is
that the second coordinate in a partway computation can be prefixed by
an angle. Then the partway point is computed normally (as if no angle
were given), but the resulting point is rotated by this angle around
the first point.  

\begin{codeexample}[]
\begin{tikzpicture}
  \coordinate [label=left:$A$]  (A) at (0,0);
  \coordinate [label=right:$B$] (B) at (1.25,0.25);
  \draw (A) -- (B);
  \node [fill=red,inner sep=1pt,label=below:$X$] (X) at ($ (A)!.5!(B) $) {};
  \node [fill=red,inner sep=1pt,label=above:$D$] (D) at
    ($ (X) ! {sin(60)*2} ! 90:(B) $) {};
  \draw (A) -- (D) -- (B);
\end{tikzpicture}
\end{codeexample}

Finally, it is not necessary to explicitly name the point $X$. Rather,
again like in the |xcolor| package, it is possible to chain partway
modifiers:

\begin{codeexample}[]
\begin{tikzpicture}
  \coordinate [label=left:$A$]  (A) at (0,0);
  \coordinate [label=right:$B$] (B) at (1.25,0.25);
  \draw (A) -- (B);
  \node [fill=red,inner sep=1pt,label=above:$D$] (D) at
    ($ (A) ! .5 ! (B) ! {sin(60)*2} ! 90:(B) $) {};
  \draw (A) -- (D) -- (B);
\end{tikzpicture}
\end{codeexample}


\subsubsection{Intersecting a Line and a Circle}

The next step in the construction is to draw a circle around $B$
through $C$, which is easy enough to do using the |circle through|
option. Extending the lines $DA$ and $DB$ can be done using partway
calculations, but this time with a part value outside the range
$[0,1]$: 

\begin{codeexample}[]
\begin{tikzpicture}
  \coordinate [label=left:$A$]  (A) at (0,0);
  \coordinate [label=right:$B$] (B) at (0.75,0.25);
  \coordinate [label=above:$C$] (C) at (1,1.5);
  \draw (A) -- (B) -- (C);
  \coordinate [label=above:$D$] (D) at
    ($ (A) ! .5 ! (B) ! {sin(60)*2} ! 90:(B) $) {};
  \node (H) [label=135:$H$,draw,circle through=(C)] at (B) {};
  \draw (D) -- ($ (D) ! 3.5 ! (B) $) coordinate [label=below:$F$] (F);
  \draw (D) -- ($ (D) ! 2.5 ! (A) $) coordinate [label=below:$E$] (E);
\end{tikzpicture}
\end{codeexample}

We now face the problem of finding the point $G$, which is the
intersection of the line $BF$ and the circle $H$. One way is to use
yet another variant of the partway computation: Normally, a partway
computation has the form \meta{p}|!|\meta{factor}|!|\meta{q},
resulting in the point $(1-\meta{factor})\meta{p} +
\meta{factor}\meta{q}$. Alternatively, instead of \meta{factor} you
can also use a \meta{dimension} between the points. In this case, you
get the point that is \meta{dimension} removed from \meta{p} on the
straight line to \meta{q}.

We know that the point $G$ is on the way from $B$ to $F$. The distance
is given by the radius of the circle~$H$. Here is the code form
computing $H$:
{\tikzexternaldisable
\begin{codeexample}[pre={
\begin{tikzpicture}
  \coordinate [label=left:$A$]  (A) at (0,0);
  \coordinate [label=right:$B$] (B) at (0.75,0.25);
  \coordinate [label=above:$C$] (C) at (1,1.5);
  \draw (A) -- (B) -- (C);
  \coordinate [label=above:$D$] (D) at
    ($ (A) ! .5 ! (B) ! {sin(60)*2} ! 90:(B) $) {};
  \draw (D) -- ($ (D) ! 3.5 ! (B) $) coordinate [label=below:$F$] (F);
  \draw (D) -- ($ (D) ! 2.5 ! (A) $) coordinate [label=below:$E$] (E);
},post={\end{tikzpicture}}]
  \node (H) [label=135:$H$,draw,circle through=(C)] at (B) {};
  \path let \p1 = ($ (B) - (C) $) in
    coordinate [label=left:$G$] (G) at ($ (B) ! veclen(\x1,\y1) ! (F) $);
  \fill[red,opacity=.5] (G) circle (2pt);
\end{codeexample}

However, there is a simpler way: We can simply name the path of the
circle and of the line in question and then use |name intersections|
to compute the intersections.

\begin{codeexample}[pre={
\begin{tikzpicture}
  \coordinate [label=left:$A$]  (A) at (0,0);
  \coordinate [label=right:$B$] (B) at (0.75,0.25);
  \coordinate [label=above:$C$] (C) at (1,1.5);
  \draw (A) -- (B) -- (C);
  \coordinate [label=above:$D$] (D) at
    ($ (A) ! .5 ! (B) ! {sin(60)*2} ! 90:(B) $) {};
  \draw (D) -- ($ (D) ! 3.5 ! (B) $) coordinate [label=below:$F$] (F);
  \draw (D) -- ($ (D) ! 2.5 ! (A) $) coordinate [label=below:$E$] (E);
},post={\end{tikzpicture}}]
  \node (H) [name path=H,label=135:$H$,draw,circle through=(C)] at (B) {};
  \path [name path=B--F] (B) -- (F);
  \path [name intersections={of=H and B--F,by={[label=left:$G$]G}}];
  \fill[red,opacity=.5] (G) circle (2pt);
\end{codeexample}
}%

\subsubsection{The Complete Code}

\begin{codeexample}[]
\begin{tikzpicture}[thick,help lines/.style={thin,draw=black!50}]
  \def\A{\textcolor{orange}{$A$}}   \def\B{\textcolor{input}{$B$}}
  \def\C{\textcolor{input}{$C$}}    \def\D{$D$}
  \def\E{$E$}                       \def\F{$F$}
  \def\G{$G$}                       \def\H{$H$}
  \def\K{$K$}                       \def\L{\textcolor{output}{$L$}}
  
  \colorlet{input}{blue!80!black}    \colorlet{output}{red!70!black}
  
  \coordinate [label=left:\A]  (A) at ($ (0,0) + .1*(rand,rand) $);
  \coordinate [label=right:\B] (B) at ($ (1,0.2) + .1*(rand,rand) $);
  \coordinate [label=above:\C] (C) at ($ (1,2) + .1*(rand,rand) $);

  \draw [input] (B) -- (C);
  \draw [help lines] (A) -- (B);

  \coordinate [label=above:\D] (D) at ($ (A)!.5!(B) ! {sin(60)*2} ! 90:(B) $);

  \draw [help lines] (D) -- ($ (D)!3.75!(A) $) coordinate [label=-135:\E] (E);
  \draw [help lines] (D) -- ($ (D)!3.75!(B) $) coordinate [label=-45:\F] (F);

  \node (H) at (B) [name path=H,help lines,circle through=(C),draw,label=135:\H] {};
  \path [name path=B--F] (B) -- (F);
  \path [name intersections={of=H and B--F,by={[label=right:\G]G}}];

  \node (K) at (D) [name path=K,help lines,circle through=(G),draw,label=135:\K] {};
  \path [name path=A--E] (A) -- (E);
  \path [name intersections={of=K and A--E,by={[label=below:\L]L}}];

  \draw [output] (A) -- (L);

  \foreach \point in {A,B,C,D,G,L}
    \fill [black,opacity=.5] (\point) circle (2pt);

  % \node ...
\end{tikzpicture}
\end{codeexample}

% Copyright 2006 by Till Tantau
%
% This file may be distributed and/or modified
%
% 1. under the LaTeX Project Public License and/or
% 2. under the GNU Free Documentation License.
%
% See the file doc/generic/pgf/licenses/LICENSE for more details.


\section{Tutorial: Putting a Diagram in Chains}

In this tutorial we have a look at how chains and matrices can be used
to typeset a diagram.

Ilka, who has just got tenured for her professorship of Old and
Lovable Programming Languages, has recently dug up a technical report entitled
\emph{The Programming Language Pascal} in the dusty cellar of the
library of her university. Having been created in the good old times
using a pen and a rules, it looks like this:

% \bigskip
% \begin{tikzpicture}[
%   >=latex,thick,
%   /pgf/every decoration/.style={/tikz/sharp corners},
%   fuzzy/.style={decorate,decoration={random steps,segment length=0.5mm,amplitude=0.15pt}},
%   minimum size=6mm,line join=round,line cap=round,
%   terminal/.style={rectangle,draw,fill=white,fuzzy,rounded corners=3mm},
%   nonterminal/.style={rectangle,draw,fill=white,fuzzy},
%   node distance=3mm]

%   \ttfamily
%   \begin{scope}[start chain,
%                 every node/.style={on chain},
%                 terminal/.append style={join=by {->,shorten >=-1pt,fuzzy,decoration={post length=4pt}}},
%                 nonterminal/.append style={join=by {->,shorten >=-1pt,fuzzy,decoration={post length=4pt}}},
%                 support/.style={coordinate,join=by fuzzy}]
%     \node [support]             (start)        {};
%     \node [nonterminal]                        {unsigned integer};
%     \node [support]             (after ui)     {};
%     \node [terminal]                           {.};
%     \node [support]             (after dot)    {};
%     \node [terminal]                           {digit};
%     \node [support]             (after digit)  {};
%     \node [support]             (skip)         {};    
%     \node [support]             (before E)     {};
%     \node [terminal]                           {E};
%     \node [support]             (after E)      {};
%     \node [support,xshift=5mm]  (between)      {};
%     \node [support,xshift=5mm]  (before last)  {};
%     \node [nonterminal]                        {unsigned integer};
%     \node [support]             (after last)   {};
%     \node [join=by ->]          (end)          {};
%   \end{scope}
%   \node (plus)  [terminal,above=of between] {$+$};
%   \node (minus) [terminal,below=of between] {$-$};

%   \begin{scope}[->,decoration={post length=4pt},rounded corners=2mm,every path/.style=fuzzy]
%     \draw (after ui)    -- +(0,.7)  -| (skip);
%     \draw (after digit) -- +(0,-.7) -| (after dot);
%     \draw (before E)    -- +(0,-1.2) -| (after last);
%     \draw (after E)     |- (plus);
%     \draw (plus)        -| (before last);
%     \draw (after E)     |- (minus);
%     \draw (minus)       -| (before last);
%   \end{scope}
% \end{tikzpicture}
% \bigskip

For her next lecture, Ilka decides to redo this diagram, but this time
perhaps a bit ``cleaner'' and perhaps also bit ``cooler.''


\bigskip
\begin{tikzpicture}[
  >=stealth',thick,draw=black!70,fill=black!70,
  minimum size=6mm,line join=round,line cap=round,
  terminal/.style={font=\ttfamily,rectangle,very thick,draw=black!50,bottom
    color=black!20,top color=white,rounded corners=3mm},
  nonterminal/.style={font=\itshape,rectangle,very thick,draw=red!50!black!50,bottom
    color=red!50!black!20,top color=white},
  node distance=3mm,
  text height=8pt,text depth=2pt]

  \begin{scope}[start chain,
                every node/.style={on chain},
                terminal/.append style={join=by {->,shorten >=1pt}},
                nonterminal/.append style={join=by {->,shorten >=1pt}},
                support/.style={coordinate,join}]
    \node [support]             (start)        {};
    \node [nonterminal]                        {unsigned integer};
    \node [support]             (after ui)     {};
    \node [terminal]                           {.};
    \node [support]             (after dot)    {};
    \node [terminal]                           {digit};
    \node [support]             (after digit)  {};
    \node [support]             (skip)         {};    
    \node [support]             (before E)     {};
    \node [terminal]                           {E};
    \node [support]             (after E)      {};
    \node [support,xshift=5mm]  (between)      {};
    \node [support,xshift=5mm]  (before last)  {};
    \node [nonterminal]                        {unsigned integer};
    \node [support]             (after last)   {};
    \node [join=by ->]          (end)          {};
  \end{scope}
  \node (plus)  [terminal,above=of between] {+};
  \node (minus) [terminal,below=of between] {-};

  \begin{scope}[->,shorten >=1pt,decoration={post length=4pt},rounded corners]
    \draw (after ui)    -- +(0,.7)  -| (skip);
    \draw (after digit) -- +(0,-.7) -| (after dot);
    \draw (before E)    -- +(0,-1.2) -| (after last);
    \draw (after E)     |- (plus);
    \draw (plus)        -| (before last);
    \draw (after E)     |- (minus);
    \draw (minus)       -| (before last);
  \end{scope}
\end{tikzpicture}
\bigskip


Having read the previous tutorials, Ilka knows already how to setup
the environment for her graphic, namely using a |tikzpicture|
environment. She wonders which libraries she will need. She decides
that she will postpone the decision and add the necessary libraries as
needed as she constructs the picture.


\subsection{Styling the Nodes}

The bulk of this tutorial will be about arranging the nodes and
connecting them using chains, but let us start with setting up styles
for the nodes.

There are two kinds of nodes, namely what theoreticians like to call
\emph{terminals} and \emph{nonterminals}. For the terminals, Ilka
decides to use a black color, which visually shows that ``nothing
needs to be done about them.'' The nonterminals, which still need to
be ``processed'' further, get a bit of red mixed in.

Ilka starts with the simpler nonterminals, as there are no rounded
corners involved. Naturally, she sets up a style:

\tikzset{
  nonterminal/.style={
      % The shape:
      rectangle,
      % The size:
      minimum size=6mm,
      % The border:
      very thick,           
      draw=red!50!black!50,         % 50% red and 50% black,
                                    % and that mixed with 50% white
      % The filling:
      top color=white,              % a shading that is white at the top...
      bottom color=red!50!black!20, % and something else at the bottom
      % Font
      font=\itshape                 
}}
\begin{codeexample}[]
\begin{tikzpicture}[
    nonterminal/.style={
      % The shape:
      rectangle,
      % The size:
      minimum size=6mm,
      % The border:
      very thick,           
      draw=red!50!black!50,         % 50% red and 50% black,
                                    % and that mixed with 50% white
      % The filling:
      top color=white,              % a shading that is white at the top...
      bottom color=red!50!black!20, % and something else at the bottom
      % Font
      font=\itshape                 
    }]
  \node [nonterminal] {unsigned integer};
\end{tikzpicture}
\end{codeexample}
Ilka is pretty proud of the use of the |minimum size| option: As the
name suggests, this option ensures that the node is at least 6mm by
6mm, but it will expand in size as necessary to accommodate longer
text. By giving this option to all nodes, they will all have the same
height of 6mm.

Styling the terminals is a bit more difficult because of the round
corners. Ilka has several options how she can achieve them. Once way
is to use the |rounded corners| option. It gets a dimension as
parameter and causes all corners to be replaced by little arcs with
the given dimension as radius. By setting the radius to 3mm, she will
get exactly what she needs: circles, when the shapes are, indeed,
exactly 6mm by 6mm and otherwise half circles on the sides:

\begin{codeexample}[]
\begin{tikzpicture}[node distance=5mm,
                    terminal/.style={
                      % The shape:
                      rectangle,minimum size=6mm,rounded corners=3mm,
                      % The rest
                      very thick,draw=black!50,
                      top color=white,bottom color=black!20,
                      font=\ttfamily}]
  \node (dot)   [terminal]                {.};
  \node (digit) [terminal,right=of dot]   {digit};
  \node (E)     [terminal,right=of digit] {E};
\end{tikzpicture}
\end{codeexample}

Another option is to use a shape that is specially made for
typesetting rectangles with arc on the sides (she has to use the
|shapes.misc| library to use it). This shape gives Ilka
much more control over the appearance. For instance, she could have an
arc only on the left side, but she will not need this.
\begin{codeexample}[]
\begin{tikzpicture}[node distance=5mm,
                    terminal/.style={
                      % The shape:
                      rounded rectangle,
                      minimum size=6mm,
                      % The rest
                      very thick,draw=black!50,
                      top color=white,bottom color=black!20,
                      font=\ttfamily}]
  \node (dot)   [terminal]                {.};
  \node (digit) [terminal,right=of dot]   {digit};
  \node (E)     [terminal,right=of digit] {E};
\end{tikzpicture}
\end{codeexample}
Either method seems fine to Ilka.

At this point, she notices a problem. The baseline of the text in the
nodes is not aligned:
\tikzset{terminal/.style={
                      % The shape:
                      rounded rectangle,
                      minimum size=6mm,
                      % The rest
                      very thick,draw=black!50,
                      top color=white,bottom color=black!20,
                      font=\ttfamily}}
\begin{codeexample}[]
\begin{tikzpicture}[node distance=5mm]
  \node (dot)   [terminal]                {.};
  \node (digit) [terminal,right=of dot]   {digit};
  \node (E)     [terminal,right=of digit] {E};

  \draw [help lines] let \p1 = (dot.base),
                         \p2 = (digit.base),
                         \p3 = (E.base)
                     in (-.5,\y1) -- (3.5,\y1)
                        (-.5,\y2) -- (3.5,\y2)
                        (-.5,\y3) -- (3.5,\y3);                     
\end{tikzpicture}
\end{codeexample}
(Ilka has moved the style definition to the preamble by
saying |\tikzset{terminal/.style=...}|, so that she can use it in all
pictures.)

For the |digit| and the |E| the difference in the baselines is almost
imperceptible, but for the dot the problem is quite severe: It looks
more like a multiplication dot than a period.

Ilka toys with the idea of using the |base right=of...| option rather than
the |right=of...| to align the nodes in such a way that the baselines
are all on the same line (the |base right| option places a node
right of something so that the baseline is right of the baseline of
the other object). However, this does not have the desired effect:
\begin{codeexample}[]
\begin{tikzpicture}[node distance=5mm]
  \node (dot)   [terminal]                {.};
  \node (digit) [terminal,base right=of dot]   {digit};
  \node (E)     [terminal,base right=of digit] {E};
\end{tikzpicture}
\end{codeexample}
The nodes suddenly ``dance around''! There is no hope of changing the
position of text inside a node using anchors. Instead, Ilka must use a
trick: The problem of mismatching baselines is caused by the fact that
|.| and |digit| and |E| all have different heights and depth. If they
all had the same, they would all be positioned vertically in the same
manner. So, all Ilka needs to do is to use the |text height| and
|text depth| options to explicitly specify a height and depth for the
nodes.
\begin{codeexample}[]
\begin{tikzpicture}[node distance=5mm,
                    text height=1.5ex,text depth=.25ex]
  \node (dot)   [terminal]                {.};
  \node (digit) [terminal,right=of dot]   {digit};
  \node (E)     [terminal,right=of digit] {E};
\end{tikzpicture}
\end{codeexample}



\subsection{Aligning  the Nodes Using Positioning Options}

Ilka now has the ``styling'' of the nodes ready. The next problem is
to place them in the right places. There are several ways to do
this. The most straightforward is to simply explicitly place the nodes
at certain coordinates ``calculated by hand.'' For very simple
graphics this is perfectly alright, but it has several disadvantages:
\begin{enumerate}
\item For more difficult graphics, the calculation may become
  complicated.
\item Changing the next of the nodes may make it necessary to
  recalculate the coordinates.
\item The source code of the graphic is not very clear since the
  relationships between the positions of the nodes are not made
  explicit. 
\end{enumerate}

For these reasons, Ilka decides to try out different ways of arranging
the nodes on the page.

The first method is the use of \emph{positioning options}. To use
them, you need to load the |positioning| library. This gives you
access to advanced implementations of options like |above| or |left|,
since you can now say |above=of some node| in order to place a node
above of |some node|, with the borders separated by |node distance|.

Ilka can use this to draw the place the nodes in a long row:
\tikzset{terminal/.append style={text height=1.5ex,text depth=.25ex}}
\tikzset{nonterminal/.append style={text height=1.5ex,text
    depth=.25ex}}
\begin{codeexample}[]
\begin{tikzpicture}[node distance=5mm and 5mm]
  \node (ui1)   [nonterminal]                     {unsigned integer};
  \node (dot)   [terminal,right=of ui1]           {.};
  \node (digit) [terminal,right=of dot]           {digit};
  \node (E)     [terminal,right=of digit]         {E};
  \node (plus)  [terminal,above right=of E]       {+};
  \node (minus) [terminal,below right=of E]       {-};
  \node (ui2)   [nonterminal,below right=of plus] {unsigned integer};
\end{tikzpicture}
\end{codeexample}

For the plus and minus nodes, Ilka is a bit startled by their
placements. Shouldn't they be more to the right? The reason they are
placed in that manner is the following: The |north east| anchor of the
|E| node lies at the ``upper start of the right arc,'' which, a bit
unfortunately in this case, happens to be the top of the
node. Likewise, the |south west| anchor of the |+| node is actually at
its bottom and, indeed, the horizontal and vertical distances between
the top of the |E| node and the bottom of the |+| node are both 5mm.

There are several ways of fixing this problem (and with matrices,
described in a moment, this problem will completely disappear). The
easiest way is to simply add a little bit of horizontal shift by hand:
\begin{codeexample}[]
\begin{tikzpicture}[node distance=5mm and 5mm]
  \node (E)     [terminal]                                   {E};
  \node (plus)  [terminal,above right=of E,xshift=5mm]       {+};
  \node (minus) [terminal,below right=of E,xshift=5mm]       {-};
  \node (ui2)   [nonterminal,below right=of plus,xshift=5mm] {unsigned integer};
\end{tikzpicture}
\end{codeexample}

Now that the nodes have been placed, Ilka needs to add
connections. Here, some connections are more difficult than
other. Consider for instance the ``repeat'' line around the
|digit|. One way of describing this line is to say ``it starts a
little to the right of |digit| than goes down and then goes to the
left and finally ends at a point a little to the right of |digit|.''
Ilka can put this into code as follows:
\begin{codeexample}[]
\begin{tikzpicture}[node distance=5mm and 5mm]
  \node (dot)   [terminal]                        {.};
  \node (digit) [terminal,right=of dot]           {digit};
  \node (E)     [terminal,right=of digit]         {E};

  \path (dot)   edge[->] (digit)  % simple edges
        (digit) edge[->] (E);

  \draw [->]
     % start right of digit.east, that is, at the point that is the
     % linear combination of digit.east and the vector (2mm,0pt). We
     % use the ($ ... $) notation for computing linear combinations
     ($ (digit.east) + (2mm,0) $)  
     % Now go down
     -- ++(0,-.5)
     % And back to the left of digit.west
     -| ($ (digit.west) - (2mm,0) $);
\end{tikzpicture}
\end{codeexample}

Since Ilka needs this ``go up/down then horizontally and than up/down
to a target'' several times, it seems sensible to define a special
\emph{to-path} for this. Whenever the |edge| command is used, it
simply adds the current value of |to path| to the path. So, Ilka can
setup a style that contains the correct path:
\begin{codeexample}[]
\begin{tikzpicture}[node distance=5mm and 5mm,
    skip loop/.style={to path={-- ++(0,-.5) -| (\tikztotarget)}}]
  \node (dot)   [terminal]                        {.};
  \node (digit) [terminal,right=of dot]           {digit};
  \node (E)     [terminal,right=of digit]         {E};

  \path (dot)   edge[->]           (digit)  % simple edges
        (digit) edge[->]           (E)
        ($ (digit.east) + (2mm,0) $)
                edge[->,skip loop] ($ (digit.west) - (2mm,0) $);
\end{tikzpicture}
\end{codeexample}

Ilka can even go a step further and make her |skip look| style
parametrized. For this, the skip loop's vertical offset is passed as
parameter |#1|. Also, in the following code Ilka specifies the start
and targets differently, namely as the positions that are ``in the
middle between the nodes.''
\begin{codeexample}[]
\begin{tikzpicture}[node distance=5mm and 5mm,
    skip loop/.style={to path={-- ++(0,#1) -| (\tikztotarget)}}]
  \node (dot)   [terminal]                        {.};
  \node (digit) [terminal,right=of dot]           {digit};
  \node (E)     [terminal,right=of digit]         {E};

  \path (dot)   edge[->]                (digit)  % simple edges
        (digit) edge[->]                (E)
        ($ (digit.east)!.5!(E.west) $)
                edge[->,skip loop=-5mm] ($ (digit.west)!.5!(dot.east) $);
\end{tikzpicture}
\end{codeexample}


\subsection{Aligning  the Nodes Using Matrices}

Ilka is still bothered a bit by the placement of the plus and minus
nodes. Somehow, having to add an explicit |xshift| seems too much like
cheating.

A perhaps better way of positioning the nodes is to use a
\emph{matrix}. In \tikzname\ matrices can be used to align quite
arbitrary graphical objects in rows and columns. The syntax is very
similar to the use of arrays and tables in \TeX\ (indeed, internally
\TeX\ tables are used, but a lot of stuff is going on additionally).

In Ilka's graphic, there will be three rows: One row containing only
the plus node, one row containing the main nodes and one row
containing only the minus node.
\begin{codeexample}[]
\begin{tikzpicture}
  \matrix[row sep=1mm,column sep=5mm] {
    % First row:
      & & & & \node [terminal] {+}; & \\
    % Second row:
    \node [nonterminal] {unsigned integer}; &
    \node [terminal]    {.};                &
    \node [terminal]    {digit};            &
    \node [terminal]    {E};                &
                                            &
    \node [nonterminal] {unsigned integer}; \\
    % Third row:
      & & & & \node [terminal] {-}; & \\    
  };
\end{tikzpicture}
\end{codeexample}
That was easy! By toying around with the row and columns separations,
Ilka can achieve all sorts of pleasing arrangements of the nodes.

Ilka now faces the same connecting problem as before. This time, she
has an idea: She adds small nodes (they will be turned into
coordinates later on and be invisible) at all the places
where she would like connections to start and end.
\begin{codeexample}[]
\begin{tikzpicture}[point/.style={circle,inner sep=0pt,minimum size=2pt,fill=red},
                   skip loop/.style={to path={-- ++(0,#1) -| (\tikztotarget)}}]
  \matrix[row sep=1mm,column sep=2mm] {
    % First row:
    & & & & & & &  & & & & \node [terminal] {+};\\
    % Second row:
    \node (p1) [point]  {};                 &
    \node [nonterminal] {unsigned integer}; &
    \node (p2) [point]  {};                 &
    \node [terminal]    {.};                &
    \node (p3) [point]  {};                 &
    \node [terminal]    {digit};            &
    \node (p4) [point]  {};                 &
    \node (p5) [point]  {};                 &
    \node (p6) [point]  {};                 &
    \node [terminal]    {E};                &
    \node (p7) [point]  {};                 &
                                            &
    \node (p8) [point]  {};                 &
    \node [nonterminal] {unsigned integer}; &
    \node (p9) [point]  {};                 \\
    % Third row:
    & & & & & & &  & & & & \node [terminal] {-};\\
  };

  \path (p4) edge [->,skip loop=-5mm] (p3)
        (p2) edge [->,skip loop=5mm]  (p6);
\end{tikzpicture}
\end{codeexample}
Now, its only a small step to add all the missing edges.



\subsection{Connecting the Nodes using Chains}

Matrices allow Ilka to align the nodes nicely, but the connections are
not quite perfect. The problem is that the code does not really
reflect the paths that underlie the diagram.

For this reason, Ilka decides to try out \emph{chains} by including
the |chain| library. Basically, a chain is just a sequence of
(usually) connected nodes. The nodes can already have been constructed
or they can be constructed as the chain is constructed (or these
processes can be mixed).

Ilka starts with creating a chain from scratch. For this, she starts a
chain using the |start chain| option in a scope. Then, inside the
scope, she uses the |on chain| option on nodes to add them to the
chain.
\begin{codeexample}[]
\begin{tikzpicture}[start chain,node distance=5mm]
  \node [on chain,nonterminal]  {unsigned integer};
  \node [on chain,terminal]     {.};
  \node [on chain,terminal]     {digit};
  \node [on chain,terminal]     {E};
  \node [on chain,nonterminal]  {unsigned integer};
\end{tikzpicture}
\end{codeexample}
(Ilka will add the plus and minus nodes later.)

As can be seen, the nodes of a chain are placed in a row. This can be
changed, for instance by saying |start chain=going below| we get a
chain where each node is below the previous one.

The next step is to \emph{join} the nodes of the chain. For this, we
add the |join| option to each node. This joins the node with the
previous node (for the first node nothing happens).
\begin{codeexample}[]
\begin{tikzpicture}[start chain,node distance=5mm]
  \node [on chain,join,nonterminal]  {unsigned integer};
  \node [on chain,join,terminal]     {.};
  \node [on chain,join,terminal]     {digit};
  \node [on chain,join,terminal]     {E};
  \node [on chain,join,nonterminal]  {unsigned integer};
\end{tikzpicture}
\end{codeexample}
In order to get a arrow tip, we redefine the |every join| style. Also,
we move the |join| option to the |every on chain|
style so that we do not have to repeat them so often.
\begin{codeexample}[]
\begin{tikzpicture}[start chain,node distance=5mm, every on chain/.style={join}, every join/.style={->}]
  \node [on chain,nonterminal]  {unsigned integer};
  \node [on chain,terminal]     {.};
  \node [on chain,terminal]     {digit};
  \node [on chain,terminal]     {E};
  \node [on chain,nonterminal]  {unsigned integer};
\end{tikzpicture}
\end{codeexample}

It is now time to add the plus and minus signs. They obviously
\emph{branch off} the main chain. For this reason, we start a branch
for them using the |start branch| option.
\begin{codeexample}[]
\begin{tikzpicture}[start chain,node distance=5mm, every on chain/.style={join}, every join/.style={->}]
  \node [on chain,nonterminal]  {unsigned integer};
  \node [on chain,terminal]     {.};
  \node [on chain,terminal]     {digit};
  \node [on chain,terminal]     {E};
  \begin{scope}[start branch=plus]
    \node (plus)  [terminal,on chain=going above right] {+};
  \end{scope}
  \begin{scope}[start branch=minus]
    \node (minus) [terminal,on chain=going below right] {-};
  \end{scope}
  \node [nonterminal,on chain,join=with plus,join=with minus]  {unsigned integer};
\end{tikzpicture}
\end{codeexample}

Let us see, what is going on here. First, the |start branch| begins a
branch, starting at the |E| node. This is implicitly also the first
node on this branch. A branch is nothing different from a chain, which
is why the plus node is put on this branch using the |on chain|
option. However, this time we specify the placement of the node
explicitly using |going |\meta{direction}. This causes the plus sign
to be placed above and right of the |E| node. It is automatically
joined to its predecessor on the branch by the implicit |join|
option.

When the first branch ends, only the plus node has been added and the
current chain is the original chain once more and we are back to the
|E| node. Now we start a new branch for the minus node. After this
branch, the current branch is the |E| node once more.

Finally, the rightmost unsigned integer is added to the (main) chain,
which is why it is joined correctly with the |E| node. The two
additional |join| options get a special |with| parameter. This allows
you to join a node with a node other than the predecessor on the
chain. The  |with| should be followed by the name of a node.

Since Ilka will need scopes more often in the following, she includes
the |scopes| library. This allows her to replace |\begin{scope}|
  simply by an opening brace and  |\end{scope}| by the corresponding
closing brace. Also, in the following example we reference
the nodes |plus| and |minus| using
their automatic name: The $i$th node on a chain is called
|chain-|\meta{i}. For a branch \meta{branch}, the $i$th node is called
|chain/|meta{branch}|-|\meta{i}. The \meta{i} can be replaced by
|begin| and |end| to reference the first and (currently) last node on
the chain.

\begin{codeexample}[]
\begin{tikzpicture}[start chain,node distance=5mm, every on chain/.style={join}, every join/.style={->}]
  \node [on chain,nonterminal]  {unsigned integer};
  \node [on chain,terminal]     {.};
  \node [on chain,terminal]     {digit};
  \node [on chain,terminal]     {E};
  { [start branch=plus]
    \node (plus)  [terminal,on chain=going above right] {+};
  }
  { [start branch=minus]
    \node (minus) [terminal,on chain=going below right] {-};
  }
  \node [nonterminal,on chain,join=with chain/plus-end,join=with chain/minus-end]  {unsigned integer};
\end{tikzpicture}
\end{codeexample}

The next step is to add intermediate coordinate nodes in the same
manner as Ilka did for the matrix. For them, we change the |join|
style slightly, namely for these nodes we do not want an arrow
tip. This can be achieved either by (locally) changing the
|every join| style or, which is what is done in the below example, by
giving the desired style using |join=by ...|, where |...| is the style
to be used for the join.

\begin{codeexample}[]
\begin{tikzpicture}[start chain,node distance=5mm and 2mm,
                    every node/.style={on chain},    
                    nonterminal/.append style={join=by ->},    
                    terminal/.append style={join=by ->},    
                    point/.style={join=by -,circle,fill=red,minimum size=2pt,inner sep=0pt}]
  \node [point]            {};
  \node [nonterminal]      {unsigned integer};
  \node [point]            {};
  \node [terminal]         {.};
  \node [point]            {};
  \node [terminal]         {digit};
  \node [point]            {};
  \node [point]            {};
  \node [point]            {};
  \node [terminal]         {E};
  \node [point]            {};
  { [start branch=plus]
    \node (plus)  [terminal,yshift=7mm] {+};
  }
  { [start branch=minus]
    \node (minus) [terminal,yshift=-7mm] {-};
  }
  \node [point]            {};
  \node [point,join=with chain/plus-end by ->,join=with chain/minus-end by ->] {};
  \node [nonterminal]      {unsigned integer};
\end{tikzpicture}
\end{codeexample}

Still missing...

\subsection{Chains and Matrices}

Still missing...

% $Header: /cvsroot/pgf/pgf/doc/pgf/text-en/Attic/pgfmanual-en-guidelines.tex,v 1.1 2005/09/02 16:05:42 tantau Exp $

% Copyright 2005 by Till Tantau <tantau@cs.tu-berlin.de>.
%
% This program can be redistributed and/or modified under the terms
% of the GNU Public License, version 2.



\section{Guidelines on Graphics}

The present section is not about \pgfname\ or \tikzname, but about
general guidelines and principles concerning the creation of
graphics for scientific presentations, papers, and books.

The guidelines in this section come from different sources. Many of
them are just what I would like to claim is ``common sense,'' some
reflect my personal experience (though, hopefully, not my personal
preferences), some come from books (the bibliography is still missing,
sorry) on graphic design and typography. 
The most influential source  are the brilliant books
by Edward Tufte. While I do not agree with everything written in these
books, many of Tufte's arguments are so convincing that I decided to
repeat them in the following guidelines. 




\subsection{Should You Follow Guidelines?}

The first thing you should ask yourself when someone presents a bunch of
guidelines is: Should I really follow these guidelines? This is an
important questions, because there are good reasons not to follow
general guidelines.
\begin{itemize}
\item
  The person who setup the guidelines may have had other
  objectives than you do. For example, a guideline might say ``use the
  color red for emphasis.'' While this guideline makes perfect sense
  for, say, a presentation using a projector, red ``color'' has the
  \emph{opposite} effect of ``emphasis'' when printed using a
  black-and-white printer.

  Guidelines were almost always setup to address a specific
  situation. If you are not in this situation, following a guideline
  can do more harm than good.
\item
  The basic rule of typography is: ``Every rule can be broken, as long
  as you are \emph{aware}  that you are breaking a rule.'' This rule
  also applies to graphics. Phrased differently, the basic rule
  states: ``The only mistakes in typography are things done is
  ignorance.''

  When you are aware of a rule and when you decide that breaking the
  rule has a desirable effect, break the rule.
\item
  Some guidelines are simply \emph{wrong}, but everyone follows them
  out of tradition or is forced to do so. My favorite example is a 
  guideline a software company I used to work for has set in a big
  project: All programmers had to declare the parameters of functions
  in \emph{increasing order of size}. So, one-byte
  parameters should come first, then two-byte parameters, and so on. 

  This guideline is total nonsense. An (arguably) sensible guideline
  is ``parameters must be declared alphabetically'' so that parameters
  are easier to find. Another (arguably) sensible guideline is
  ``parameters must be declared in decreasing order of size'' so that
  less byte-alignment cache misses occur when the stack is
  accessed. The guideline the company used maximized cache misses and
  resulted in a more or less random ordering so that programmers
  constantly had to look up the parameter ordering.
\end{itemize}

So, before you apply a guideline or choose not to apply it, ask
yourself these questions: 
\begin{enumerate}
\item
  Does this guideline really address my situation?
\item
  If you do the opposite a guideline says you should do, will the
  advantages outweigh the disadvantages this guideline was supposed to
  prevent?  
\end{enumerate}



\subsection{Planning the Time Needed for the Creation of Graphics}

When you create a paper with numerous graphics, the time needed to
create these graphics becomes an important factor. How much time
should you calculate for the creation of graphics?

As a general rule, assume that a graphic will need as much time to
create as would a text of the same length. For example, when I
write a paper, I need about one hour per page for
the first draft. Later, I need between two and four hours per page
for revisions. Thus, I expect to need about half an hour for the
creation of \emph{a first draft} of a half page graphic. Later on, I
expect another one to two hours before the final graphic is finished.

In many publications, even in good journals, the authors and editors
have obviously  invested a lot of time on the text, but seem to 
have spend about five minutes to create all of the
graphics. Graphics often seem to have been added as an
``afterthought'' or look like a screen shot of whatever the authors's
statistical software shows them. As will be argued later on, the
graphics that programs like \textsc{gnuplot} produce by default are of
poor quality.

Creating informative graphics that help the reader and that fit
together with the main text is a difficult, lengthy process. 
\begin{itemize}
\item
  Treat graphics as first-class citizens of your papers. They deserve
  as much time and energy as the text does.
\item
  Arguably, the creation of graphics deserves \emph{even more} time
  than the writing of the main text since more attention will  be paid
  to the graphics and they will be looked at first. 
\item
  Plan as much time for the creation and revision of a graphic as you
  would plan for text of the same size.
\item
  Difficult graphics with a high information density may require even
  more time.
\item
  Very simple graphics will require less time, but most likely you do
  not want to have ``very simple graphics'' in your paper, anyway;
  just as you would not like to have a ``very simple text'' of the
  same size.  
\end{itemize}



\subsection{Workflow for Creating a Graphic}

When you write a (scientific) paper, you will most likely follow the
following pattern: You have some results/ideas that you would
like to report about. The creation of the paper will typically start
with compiling a rough outline. Then, the different sections are
filled with text to create a first draft. This draft is then revised
repeatedly until, often after substantial revision, a final paper
results. In a good journal paper there is typically not be a single 
sentence that has survived unmodified from the first draft.

Creating a graphics follows the same pattern:
\begin{itemize}
\item
  Decide on what the graphic should communicate. Make this a conscious
  decision, that is, determine ``What is the graphic supposed to tell
  the reader?''
\item
  Create an ``outline,'' that is, the rough overall ``shape'' of the
  graphic, containing the most crucial elements. Often, it is
  useful to do this using pencil and paper.
\item
  Fill out the finer details of the graphic to create a first
  draft.
\item
  Revise the graphic repeatedly along with the rest of the paper.
\end{itemize}




\subsection{Linking Graphics With the Main Text}

Graphics can be placed at different places in a text. Either, they can
be inlined, meaning they are somewhere ``in the middle of the text''
or they can be placed in standalone ``figures.'' Since printers (the
people) like to have their pages ``filled,'' (both for aesthetic and
economic reasons) standalone figures may traditionally be placed on
pages in the document far removed from the main text that refers to
them. \LaTeX\ and \TeX\ tend to encourage this ``drifting away'' of
graphics for technical reasons. 

When a graphic is inlined, it will more or less automatically be
linked with the main text in the sense that the labels of the graphic
will be implicitly explained by the surrounding text. Also, the main
text will typically make it clear what the graphic is about and what
is shown.

Quite differently, a standalone figure will often be viewed at a time
when the main text that this graphic belongs to either has not yet
been read or has been read some time ago. For this reason, you should
follow the following guidelines when creating standalone figures:
\begin{itemize}
\item
  Standalone figures should have a caption than should make them
  ``understandable by themselves.''

  For example, suppose a graphic shows an example of the different
  stages of a quicksort algorithm. Then the figure's caption should,
  at the very least, inform the reader that ``The figure shows the
  different stages of the quicksort algorithm introduced on page
  xyz.'' and not just ``Quicksort algorithm.''
\item
  A good caption adds as much context information as possible. For
  example, you could say: ``The figure shows the different stages of
  the quicksort algorithm introduced on page xyz. In the first line,
  the pivot element 5 is chosen. This causes\dots'' While this
  information can also be given in the main text, putting it in the
  caption will ensure that the context is kept. Do not feel afraid of
  a 5-line caption. (Your editor may hate you for this. Consider
  hating them back.)
\item
  Reference the graphic in your main text as in ``For an example of
  quicksort `in action,' see Figure~2.1 on page xyz.''
\item
  Most books on style and typography recommend that you do not use
  abbreviations as in ``Fig.~2.1'' but write ``Figure 2.1.''

  The main argument against abbreviations is that ``a period is too
  valuable to waste it on an abbreviation.'' The idea is that a period
  will make the reader assume that the sentence ends after ``Fig'' and
  it takes a ``conscious backtracking'' to realize that the sentence
  did not end after all.

  The argument in favor of abbreviations is that they save space.
  
  Personally, I am not really convinced by either argument. On the one
  hand, I have not yet seen any hard evidence that abbreviations slow 
  readers down. On the other hand,  abbreviating all ``Figure'' by
  ``Fig.''\ is most unlikely to save even a single line in  most
  documents.  

  I avoid abbreviations.
\end{itemize}



\subsection{Consistency Between Graphics and Text}

Perhaps the most common ``mistake'' people do when creating graphics
(remember that a ``mistake'' in design is always just ``ignorance'')
is to have a mismatch between the way their graphics look and the way 
their text looks.

It is quite common that authors use several different programs for
creating the graphics of a paper. An author might produce some plots
using \textsc{gnuplot}, a diagram using \textsc{xfig}, and include an
|.eps| graphic a coauthor contributed using some unknown program. All
these graphics will, most likely, use different line widths, different
fonts, and have different sizes. In addition, authors often use
options like |[height=5cm]| when including graphics to scale them to
some ``nice size.''

If the same approach were taken to writing the main text, every
section would be written in a different font at a different size. In
some sections all theorems would be underlined, in another they would
be printed all in uppercase letters, and in another in red. In
addition, the margins would be different on each page.

Readers and editors would not tolerate a text if it were written in
this fashion, but with graphics they often have to.

To create consistency between graphics and text, stick to the
following guidelines:
\begin{itemize}
\item
  Do not scale graphics.

  This means that when generating graphics using an external program,
  create them ``at the right size.''
\item
  Use the same font(s) both in graphics and the body text.
\item
  Use the same line width in text and graphics.

  The  ``line width'' for normal text is the width of the stem of
  letters like T{}. For \TeX, this is usually
  $0.4\,\mathrm{pt}$. However, some journals will not accept graphics
  with a normal line width below $0.5\,\mathrm{pt}$.
\item
  When using colors, use a consistent color coding in the text and in  
  graphics. For example, if red is supposed to alert the reader to
  something in the main text, use red also in graphics for important
  parts of the graphic. If blue is used for structural elements like 
  headlines and section titles, use blue also for structural elements
  of your graphic.

  However, graphics may also use a logical intrinsic color
  coding. For example, no matter what colors you normally use, readers
  will generally assume, say, that the color green as ``positive, go,
  ok'' and red as ``alert, warning, action.''
\end{itemize}

Creating consistency when using different graphic programs is almost
impossible. For this reason, you should consider sticking to a single
graphic program.


\subsection{Labels in Graphics}

Almost all graphics will contain labels, that is, pieces of text that
explain parts of the graphics. When placing labels, stick to the
following guidelines:

\begin{itemize}
\item
  Follow the rule of consistency when placing labels. You should do
  so in two ways: First, be consistent with the main text, that is,
  use the same font as the main text also for labels. Second, be
  consistent between labels, that is, if you format some labels in
  some particular way, format all labels in this way.
\item
  In addition to using the same fonts in text and graphics, you should
  also use the same notation. For example, if you write $1/2$ in your
  main text, also use ``$1/2$'' as labels in graphics, not
  ``0.5''. A $\pi$ is a ``$\pi$'' and not ``$3.141$''. Finally,
  $\mathrm e^{-\mathrm i \pi}$ is ``$\mathrm e^{-\mathrm i \pi}$'',
  not ``$-1$'', let alone ``-1''. 
\item
  Labels should be legible. They should not only have a reasonably
  large size, they also should not be obscured by lines or other
  text. This also applies to of lines and text \emph{behind} the
  labels.
\item
  Labels should be ``in  place.'' Whenever there is enough space,
  labels should be placed next to the thing they label. Only if
  necessary, add a (subdued) line from the label to the labeled
  object. Try to avoid labels that only reference explanations in
  external legends. Reader have to jump back and forth between the
  explanation and the object that is described. 
\item
  Consider subduing ``unimportant'' labels using, for example, a gray
  color. This will keep the focus on the actual graphic.
\end{itemize}



\subsection{Plots and Charts}

One of the most frequent kind of graphics, especially in scientific
papers, are \emph{plots}. They come in a large variety, including
simple line plots, parametric plots, three dimensional plots, pie
charts, and many more.

Unfortunately, plots are notoriously hard to get right. Partly, the
default settings of programs like \textsc{gnuplot} or Excel are to
blame for this since these programs make it very convenient to create
bad plots.

The first question you should ask yourself when creating a plot is the
following:
\begin{itemize}
\item
  Are there enough data points to merit a plot?
\end{itemize}

If the answer is ``not really,'' use a table.

A typical situation where a plot is unnecessary is when people present
a few numbers in a bar diagram. Here is a real-life example: At the
end of a seminar a lecturer asked the participants for feedback. Of
the 50 participants, 30 returned the feedback form. According to the
feedback, three participants considered the seminar ``very good,''
nine considered it  ``good,'' ten ``ok,'' eight ``bad,'' and no one thought 
that the seminar was ``very bad.''

A simple way of summing up this information is the following table:

\medskip
\begin{tabular}{lp{3.75cm}r}
  \emph{Rating given} & \raggedright\emph{Participants (out of 50) who gave this rating} &
  \emph{Percentage} \\[1.75em]
  ``very good'' & \hfil\hphantom{0}3\hfil & \hphantom{0}6\% \\
  ``good'' & \hfil\hphantom{0}9\hfil & 18\% \\
  ``ok'' & \hfil10\hfil & 20\% \\
  ``bad'' & \hfil\hphantom{0}8\hfil & 16\% \\
  ``very bad'' & \hfil\hphantom{0}0\hfil & \hphantom{0}0\% \\[2mm]
  none & \hfil20\hfil & 40\% \\
\end{tabular}

\bigskip
What the lecturer did was to visualize the data using a 3D bar
diagram. It looked like this:

\bigskip
\par
\begin{tikzpicture}[y=0.03cm,z=3mm]
  \foreach \y in {0,20,40,60,80,100}
    \draw[dashed] (0,\y,0) node[left] {\y} -- (0,\y,1)  -- (6,\y,1);

  \draw (0,0,0) -- (0,100,0)  (0,0,1) -- (0,100,1);
  \draw (0,0,0) -- (6,0,0);

  \foreach \x/\xtext/\height in {1/very good/10,2/good/30,3/ok/33,4/bad/27,5/very bad/0}
  {
    \draw (\x,0) node[rotate=90,anchor=east] {\xtext};

    \begin{scope}[xshift=\x cm]
      
    \filldraw[fill=blue!50] (-.3,0,0) rectangle (.3,\height,0);
    \filldraw[fill=blue!30] (.3,0,0) -- (.3,0,1) -- (.3,\height,1) -- (.3,\height,0) --cycle;
    \filldraw[fill=blue!20] (-.3,\height,0) -- (.3,\height,0) --
    (.3,\height,1) -- (-.3,\height,1) --cycle;
    \end{scope}
  }
\end{tikzpicture}
\bigskip

Both the table and the ``plot'' have about the same size. If your first
thought is ``the graphic looks nicer than the table,'' try to answer
the following questions based on the information in the table or in
the graphic: 
\begin{enumerate}
\item
  How many participants where there?
\item
  How many participants returned the feedback form?
\item
  What percentage of the participants returned the feedback form?
\item
  How many participants checked ``very good''?
\item
  What percentage out of all participants checked ``very good''?
\item
  Did more than a quarter of the participants check ``bad'' or ``very bad''?
\item
  What percentage of the participants that returned the form checked ``very good''?
\end{enumerate}

Sadly, the graphic does not allow us to answer \emph{a single one of these
  questions}. The table answers all of them directly, except for the last
one. In essence, the information density of the graphic is very
nearly zero. The table has a much higher information density; despite
the fact that it uses quite a lot of white space to present a few numbers.

Here is the list of things that went wrong with the 3D-bar diagram:
\begin{itemize}
\item
  The whole graphic is dominated by irritating background lines.
\item
  It is not clear what the numbers at the left mean; presumably
  percentages, but it might also be the absolute number of
  participants.
\item
  The labels at the bottom are rotated, making them hard to read.

  (In the real presentation that I saw, the text was rendered at a very 
  low resolution with about 10 by 6 pixels per letter with wrong
  kerning, making the rotated text almost impossible to read.)
\item
  The third dimension adds complexity to the graphic without adding
  information.
\item
  The three dimensional setup makes it much harder to gauge the height
  of the bars correctly. Consider the ``bad'' bar. It the number this
  bar stands for more than 20 or less? While the front of the bar is
  below the 20 line, the back of the bar (which counts) is above.
\item
  It is impossible to tell which  numbers are represented by the
  bars. Thus, the bars needlessly hide the information these bars are
  all about.
\item
  What do the bar heights add up to? Is it 100\% or 60\%?
\item
  Does the bar for ``very bad'' represent 0 or~1?
\item
  Why are the bars blue?
\end{itemize}

You might argue that in the example the exact numbers are not
important for the graphic. The important things is the ``message,''
which is that there are more ``very good'' and ``good'' ratings than
``bad'' and ``very bad.'' However, to convey this message either use a
sentence that says so or use a graphic that conveys this message more
clearly:  

\medskip
\par
\begin{tikzpicture}
  \colorlet{good}{green!75!black}
  \colorlet{bad}{red}
  \colorlet{neutral}{black!60}
  \colorlet{none}{white}

  \node[text centered,text width=3cm]{Ratings given by 50~participants};

  \begin{scope}[line width=4mm,rotate=270]
    \draw[good]          (-123:2cm) arc (-123:-101:2cm);
    \draw[good!60!white] (-36:2cm) arc (-36:-101:2cm);
    \draw[neutral]       (-36:2cm) arc (-36:36:2cm);
    \draw[bad!60!white]  (36:2cm)  arc (36:93:2cm);

    \newcount\mycount
    \foreach \angle in {0,72,...,3599}
    {
      \mycount=\angle\relax
      \divide\mycount by 10\relax
      \draw[black!15,thick] (\the\mycount:18mm) -- (\the\mycount:22mm);
    }
    
    \draw (0:2.2cm) node[below] {``ok'': 10 (20\%)};
    \draw (165:2.2cm) node[above] {none: 20 (40\%)};
    \draw (-111:2.2cm) node[left] {``very good'': 3 (6\%)};
    \draw (-68:2.2cm) node[left] {``good'': 9 (18\%)};
    \draw (65:2.2cm) node[right] {``bad'': 8 (16\%)};
    \draw (93:2.2cm) node[right] {``very bad'': 0 (0\%)};
  \end{scope}  
  \draw[gray] (0,0) circle (2.2cm) circle (1.8cm);
\end{tikzpicture}

\bigskip
The above graphic has about the same information density as the table
(about the same size and the same numbers are shown). In addition, one
can directly ``see'' that there are more good or very good ratings
than bad ones. One can also ``see'' that the number of people who gave
no rating at all is not negligible, which is quite common for feedback
forms. 

Charts are not always a good idea. Let us look at an example
that I redrew from a pie chart in \emph{Die Zeit}, June 4th, 2005:

\bigskip
\par
\begin{tikzpicture}
  \begin{scope}[xscale=3.2,yscale=1.2]

    \sffamily
    \coordinate (right border) at (2.0cm,-1.7cm);
    \coordinate (left border)  at (-2.5cm,2.1cm);

    \fill[black!25] ([xshift=-2mm,yshift=1.1cm]left border) rectangle ([xshift=2mm,yshift=-.3cm]right border);

    \node[below right,text width=10cm,inner sep=0pt] at ([yshift=.9cm,xshift=-1mm]left border)
    { {\color{black!75} \Large Kohle ist am wichtigsten}\\
      Energiemix bei der deutschen Stromerzeugung 2004};

    \filldraw[draw=gray,fill=white] ([xshift=-1mm]left border) node[below right,black]
      {\footnotesize Gesamte Netto-Stromerzeugung in Prozent, in
        Milliarden Kilowattstunden (Mrd.\ kWh)}
      rectangle ([xshift=1mm]right border);
    
    % The 3D stuff
    \pgfdeclarehorizontalshading{zeit}{100bp}
    {color(0pt)=(black);
      color(25bp)=(black);
      color(37bp)=(white);
      color(50bp)=(black);
      color(62bp)=(white);
      color(75bp)=(black);
      color(100bp)=(black)}

    \shadedraw[very thin,shading=zeit,yshift=-1.5mm] (0,0) circle (1cm);

    \fill[green!20!gray]   (0,0) -- (90:1cm) arc (90:-5:1cm);
    \fill[white!20!gray]   (0,0) -- (-5:1cm) arc (-5:-105:1cm);
    \fill[orange!20!gray]  (0,0) -- (-105:1cm) arc (-105:-180:1cm);
    \fill[orange!60!white] (0,0) -- (180:1cm) arc (180:150:1cm);
    \fill[black!75!white]  (0,0) -- (150:1cm) arc (150:145:1cm);
    \fill[blue!90!white]   (0,0) -- (145:1cm) arc (145:135:1cm);
    \fill[blue!50!white]   (0,0) -- (135:1cm) arc (135:92:1cm);
    \fill[yellow!50!black] (0,0) -- (92:1cm) arc (92:90:1cm);

    \begin{scope}[very thin]
      \draw (0,0) -- (90:1cm);
      \draw (0,0) -- (-5:1cm);
      \draw (0,0) -- (-105:1cm);
      \draw (0,0) -- (-180:1cm);
      \draw (0,0) -- (150:1cm);
      \draw (0,0) -- (145:1cm);
      \draw (0,0) -- (135:1cm);
      \draw (0,0) -- (92:1cm);
      
      \draw(0,0) circle (1cm);
    \end{scope}

    \node (Regenerative)   at (115:.75cm)  {\bfseries 9,4\%};
    \node (Kernenergie)    at (30:.5cm)   {\bfseries 27,8\%};
    \node (Braunkohle)     at (-45:.6cm)  {\bfseries 25,6\%};
    \node (Steinkohle)     at (-135:.6cm) {\bfseries 22,3\%};
    \node (Erdgas)         at (168:.75cm) {\bfseries 10,4\%};
    \coordinate (Mineral)  at (147:.9cm);
    \coordinate (Sonstige) at (140:.9cm);

    \small
    \draw (Regenerative.north) |- ([yshift=.25cm]Regenerative.north -| right border) coordinate (Regenerative label);
    \draw (91:.9cm) |- (Regenerative label);
    \node[above left] at (Regenerative label) {Regenerative\
      {\footnotesize (53,7 kWh)/davon} Wind \textbf{4,4\%}  \footnotesize (25,0 kWh)};

    \draw (Kernenergie.base east) -- (Kernenergie.base east -| right border) coordinate (Kernenergie label);
    \node[above left] at (Kernenergie label) {Kernenergie};
    \node[below left] at (Kernenergie label) {\footnotesize (158,4 kWh)};

    \draw (Braunkohle.south) |- ([yshift=-.75cm]Braunkohle.south -| right border) coordinate (Braunkohle label);
    \node[above left] at (Braunkohle label) {Braunkohle\ \ \footnotesize (146,0 kWh)};

    \draw (Steinkohle.south) |- ([yshift=-.75cm]Steinkohle.south -| left border) coordinate (Steinkohle label);
    \node[above right] at (Steinkohle label) {Steinkohle\ \ \footnotesize (127,1 kWh)};

    \draw (Erdgas.base west) -- (Erdgas.base west -| left border) coordinate (Erdgas label);
    \node[above right] at (Erdgas label) {Erdgas\ \ \footnotesize (59,2 kWh)};

    \draw (Mineral) -- (Mineral -| left border) coordinate (Mineral label);
    \node[above right] at (Mineral label) {Mineral\"olprodukte\ \
      \footnotesize (9,2 kWh) \  \ \normalsize\textbf{1,6\%}};

    \draw (Sonstige) |- (Regenerative label -| left border) coordinate (Sonstige label);
    \node[above right] at (Sonstige label) {Sonstige\ \
      \footnotesize (16,5 kWh) \hskip1.5cm\
      \normalsize\textbf{2,9\%}};
  \end{scope}    
\end{tikzpicture}

This graphic has been redrawn in \tikzname, but the original looks very
similar.

At first sight, the graphic looks  ``nice and informative,'' but there
are a lot of things that went wrong:

\begin{itemize}
\item
  The chart is three dimensional. However, the shadings add
  nothing ``information-wise,'' at best, they distract.
\item
  In a 3D-pie-chart the relative sizes are very strongly
  distorted. For example, the area taken up by the gray color of ``Braunkohle''
  is larger than the area taken up by the green color of
  ``Kernenergie'' \emph{despite the fact that the percentage of
    Braunkohle is less than the percentage of Kernenergie}.
\item
  The 3D-distortion gets worse for small areas. The area of
  ``Regenerative'' somewhat larger  than the area of ``Erdgas.''  
  The area of ``Wind'' is slightly smaller than the area of
  ``Mineral\"olprodukte'' \emph{although the percentage of Wind is
    nearly three times larger than the percentage of
    Mineral\"olprodukte.}

  In the last case, the different sizes are only partly due to
  distortion. The designer(s) of the original graphic have also made
  the ``Wind'' slice too small, even taking distortion into
  account. (Just compare the size of ``Wind'' to ``Regenerative'' in
  general.) 
\item
  According to its caption, this chart is supposed to inform us that
  coal was the most important energy source in Germany in
  2004. Ignoring the strong distortions caused by the superfluous and
  misleading 3D-setup, it takes quite a while for this message to get
  across. 

  Coal as an energy source is split up into two slices: one for
  ``Steinkohle'' and one for ``Braunkohle'' (two different kinds of
  coal). When you add them up, you see that the whole lower half of
  the pie chart is taken up by coal.

  The two areas for the different kinds of coal are not visually
  linked at all. Rather, two different colors are used, the labels are
  on different sides of the graphic. By comparison, ``Regenerative''
  and ``Wind'' are very closely linked.
\item
  The color coding of the graphic follows no logical pattern at
  all. Why is nuclear energy green? Regenerative energy is light blue,
  ``other sources'' are blue. It seems more like a joke that the area
  for ``Braunkohle'' (which literally translates to ``brown coal'') is
  stone gray, while the area for ``Steinkohle'' (which literally
  translates to ``stone coal'') is brown.
\item
  The area with the lightest color is used for ``Erdgas.'' This area
  stands out most because of the brighter color. However, for this
  chart ``Erdgas'' is not really important at all.
\end{itemize}
Edward Tufte calls graphics like the above ``chart junk.'' 

Here are a few recommendations that may help you avoid producing chart junk:
\begin{itemize}
\item
  Do not use 3D pie charts. They are \emph{evil}.
\item
  Consider using a table instead of a pie chart.
\item
  Due not apply colors randomly; use them to direct the readers's 
  focus and to group things.
\item
  Do not use background patterns, like a crosshatch or diagonal
  lines, instead of colors. They distract. Background patterns in
  information graphics are \emph{evil}.
\end{itemize}



\subsection{Attention and Distraction}

Pick up your favorite fiction novel and have a look at a typical
page. You will notice that the page is very uniform. Nothing is there
to distract the reader while reading; no large headlines, no bold
text, no large white areas. Indeed, even when the author does wish to
emphasize something, this is done using italic letters. Such letters
blend nicely with the main text---at a distance you will not be able to
tell whether a page contains italic letters, but you would notice a
single bold word immediately. The reason novels are typeset this way
is the following paradigm: Avoid distractions.

Good typography (like good organization) is something you do
\emph{not} notice. The job of typography is to make reading the text,
that is, ``absorbing'' its information content, as effortless as
possible. For a novel, readers absorb the content by reading the text
line-by-line, as if they were listening to someone telling the
story. In this situation anything on the page that distracts the eye
from  going quickly and evenly from line to line will make the text
harder to read.

Now, pick up your favorite weekly magazine or newspaper and have a
look at a typical 
page. You will notice that there is quite a lot ``going on'' on the
page. Fonts are used at different sizes and in different arrangements,
the text is organized in narrow columns, typically interleaved with
pictures. The reason magazines are typeset in this way is another
paradigm: Steer attention.

Readers will not read a magazine like a novel. Instead of reading a
magazine line-by-line, we use headlines and short abstracts to check
whether we want to read a certain article or not. The job of
typography is to steer our attention to these abstracts and headlines,
first. Once we have decided that we want to read an article, however,
we no longer tolerate distractions, which is why the main text of
articles is typeset exactly the same way as a novel.

The two principles ``avoid distractions'' and ``steer attention'' also
apply to graphics. When you design a graphic, you should eliminate
everything that will ``distract the eye.'' At the same time, you
should try to actively help the reader ``through the graphic'' by
using fonts/colors/line widths to highlight different parts.

Here is a non-exhaustive list of things that can distract readers:
\begin{itemize}
\item
  Strong contrasts will always be registered first by the eye. For
  example, consider the following two grids:

  \medskip\par
  \begin{tikzpicture}[x=40pt,y=40pt]
    \draw[step=10pt,gray] (0,0) grid +(1,1);
    \draw[step=2pt]      (2,0) grid +(1,1);
  \end{tikzpicture}

  \medskip
  Even though the left grid comes first in our normal reading order,
  the right one is much more likely to be seen first: The
  white-to-black contrast is higher than the gray-to-white
  contrast. In addition, there are more ``places'' adding to the
  overall contrast in the right grid.

  Things like grids and, more generally, help lines usually should not
  grab the attention of the readers and, hence, should be typeset with
  a low contrast to the background. Also, a loosely-spaced grid is
  less distracting than a very closely-spaced grid.
\item
  Dashed lines create many points at which there is black-to-white
  contrast. Dashed or dotted lines can be very distracting and, hence,
  should be avoided in general.

  Do not use different dashing patterns to differentiate curves in
  plots. You loose data points this way and the eye is not
  particularly good at ``grouping things according to a dashing
  pattern.'' The eye is \emph{much} better at grouping things
  according to colors.
\item
  Background patterns filling an area using  diagonal lines or
  horizontal and vertical lines or just dots are almost always
  distracting and, usually, serve no real purpose.
\item
  Background images and shadings distract and only seldom add
  anything of importance to a graphic.
\item
  Cute little cliparts can easily draw attention away from the
  data.
\end{itemize}





\part{Installation and Configuration}

{\Large \emph{by Till Tantau}}


\bigskip
\noindent
This part explains how the system is installed. Typically, someone has
already done so for your system, so this part can be skipped; but if
this is not the case and you are the poor fellow who has to do the
installation, read the present part. 


\vskip1cm

\begin{codeexample}[graphic=white]
\begin{tikzpicture}[->,>=stealth',shorten >=1pt,auto,node distance=2.8cm,on grid,semithick,
                    every state/.style={fill=red,draw=none,circular drop shadow,text=white}]

  \node[initial,state] (A)                    {$q_a$};
  \node[state]         (B) [above right=of A] {$q_b$};
  \node[state]         (D) [below right=of A] {$q_d$};
  \node[state]         (C) [below right=of B] {$q_c$};
  \node[state]         (E) [below=of D]       {$q_e$};

  \path (A) edge              node {0,1,L} (B)
            edge              node {1,1,R} (C)
        (B) edge [loop above] node {1,1,L} (B)  
            edge              node {0,1,L} (C)
        (C) edge              node {0,1,L} (D)
            edge [bend left]  node {1,0,R} (E)    
        (D) edge [loop below] node {1,1,R} (D)
            edge              node {0,1,R} (A)
        (E) edge [bend left]  node {1,0,R} (A);

   \node [right=1cm,text width=8cm] at (C)
   {
     The current candidate for the busy beaver for five states. It is
     presumed that this Turing machine writes a maximum number of
     $1$'s before halting among all Turing machines with five states
     and the tape alphabet $\{0, 1\}$. Proving this conjecture is an
     open research problem.
   };
\end{tikzpicture}
\end{codeexample}

% Copyright 2003 by Till Tantau <tantau@cs.tu-berlin.de>.
%
% This program can be redistributed and/or modified under the terms
% of the LaTeX Project Public License Distributed from CTAN
% archives in directory macros/latex/base/lppl.txt.


\section{Installation}

There are different ways of installing \pgfname, depending
on your system and needs, and you may need to install other
packages as well as, see below. Before installing, you may wish to
review the \textsc{gpl} license under which the package is
distributed, see Section~\ref{section-license}. 

Typically, the package will already be installed on your
system. Naturally, in this case you do not need to worry about the
installation process at all and you can skip the rest of this
section. 


\subsection{Package and Driver Versions}

This documentation is part of version \pgfversion\ of the \pgfname\
package. In order to run \pgfname, you need a reasonably recent 
\TeX\ installation. When using \LaTeX, you need the following packages
installed (newer versions should also work):
\begin{itemize}
\item
  |xcolor| version \xcolorversion.
\item
  |xkeyval| version \xkeyvalversion, if you wish to use \tikzname.
\end{itemize}
With plain \TeX, |xcolor| is not needed, but you obviously do not
get its (full) functionality. 

Currently, \pgfname\ supports the following backend drivers:
\begin{itemize}
\item
  |pdftex| version 0.14 or higher. Earlier versions do not work.
\item
  |dvips| version 5.94a or higher. Earlier versions may also work.
\item
  |dvipdfm| version 0.13.2c or higher. Earlier versions may also work.
\item
  |tex4ht| version 2003-05-05 or higher. Earlier versions may also work.
\item
  |vtex| version 8.46a or higher. Earlier versions may also work.
\item
  |textures| version 2.1 or higher. Earlier versions may also work.
\end{itemize}

Currently, \pgfname\ supports the following formats:
\begin{itemize}
\item
  |latex| with complete functionality.
\item
  |plain| with complete functionality, except for graphics inclusion,
  which works only for pdf\TeX.
\item
  |context| should work as |plain|, but I have not tried it.
\end{itemize}

For more details, see Section~\ref{section-formats}.



\subsection{Installing Prebundled Packages}

I do not create or manage prebundled packages of \pgfname, but,
fortunately, nice other people do. I cannot give detailed instructions
on how to install these packages, since I do not manage them, but I
\emph{can} tell you were to find them. If you have a problem with
installing, you might wish to have a look at the Debian page or the
Mik\TeX\ page first.


\subsubsection{Debian}

The command ``|aptitude install pgf|'' should do the trick. Sit back
and relax. In detail, the following packages are installed:  
\begin{verbatim}
http://packages.debian.org/pgf
http://packages.debian.org/latex-xcolor
\end{verbatim}


\subsubsection{MiKTeX}

For MiK\TeX, use the update wizard to install the (latest versions of
the) packages called |pgf|, |xcolor|, and |xkeyval|. 




\subsection{Installation in a texmf Tree}

For a permanent installation, you place the files of the
the \textsc{pgf} package in an appropriate |texmf| tree. 

When you ask \TeX\ to use a certain class or package, it usually looks
for the necessary files in so-called |texmf| trees. These trees
are simply huge directories that contain these files. By default,
\TeX\ looks for files in three different |texmf| trees:
\begin{itemize}
\item
  The root |texmf| tree, which is usually located at
  |/usr/share/texmf/| or |c:\texmf\| or somewhere similar.
\item
  The local  |texmf| tree, which is usually located at
  |/usr/local/share/texmf/| or |c:\localtexmf\| or somewhere similar.
\item
  Your personal  |texmf| tree, which is usually located in your home
  directory at |~/texmf/| or |~/Library/texmf/|.   
\end{itemize}

You should install the packages either in the local tree or in
your personal tree, depending on whether you have write access to the
local tree. Installation in the root tree can cause problems, since an
update of the whole \TeX\ installation will replace this whole tree.


\subsubsection{Installation that Keeps Everything Together}

Once you have located the right texmf tree, you must decide whether
you want to install \pgfname\ in such a way that ``all its files are
kept in one place'' or whether you want to be
``\textsc{tds}-compliant,'' where \textsc{tds} means ``\TeX\ directory
structure.''

If you want to keep ``everything in one place,'' inside the |texmf|
tree that you have chosen create a sub-sub-directory called
|texmf/tex/generic/pgf| or
|texmf/tex/generic/pgf-|\texttt{\pgfversion}, if you prefer. Then
place all files of the |pgf| package in this directory. Finally,
rebuild \TeX's filename database. This is done by running the command
|texhash| or |mktexlsr| (they are the same). In Mik\TeX, there is a
menu option to do this. 


\subsubsection{Installation that is TDS-Compliant}

While the above installation process is the most ``natural'' one and
although I would like to recommend it since it makes updating and
managing the \pgfname\ package easy, it is not
\textsc{tds}-compliant. If you want to be \textsc{tds}-compliant,
proceed as follows: (If you do not know what \textsc{tds}-compliant
means, you probably do not want to be \textsc{tds}-compliant.)

The |.tar| file of the |pgf| package contains the following files and
directories at its root: |README|, |doc|,  |generic|, |plain|, and
|latex|. You should ``merge'' each of the four directories with the
following directories |texmf/doc|, |texmf/tex/generic|,
|texmf/tex/plain|, and |texmf/tex/latex|. For example, in the |.tar|
file the |doc| directory contains just the directory |pgf|, and this
directory has to be moved to |texmf/doc/pgf|. The root |README| file
can be ignored since it is reproduced in |doc/pgf/README|.

You may also consider keeping everything in one place and using
symbolic links to point from the \textsc{tds}-compliant directories to
the central installation.

\vskip1em
For a more detailed explanation of the standard installation process
of packages, you might wish to consult
\href{http://www.ctan.org/installationadvice/}{|http://www.ctan.org/installationadvice/|}.
However, note that the \pgfname\ package does not come with a
|.ins| file (simply skip that part).


\subsection{Updating the Installation}

To update your installation from a previous version, all you need to
do is to replace everything in the directory |texmf/tex/generic/pgf|
with the files of the new version (or in all the directories where
|pgf| was installed, if you chose a \textsc{tds}-compliant
installation). The easiest way to do this is to first delete the old
version and then proceed as described above. Sometimes, there are
changes in the syntax of certain command from version to version. If
things no longer work that used to work, you may wish to have a look
at the release notes and at the change log. 


% $Header: /cvsroot/pgf/pgf/doc/generic/pgf/text-en/pgfmanual-en-license.tex,v 1.2 2006/10/10 07:37:25 tantau Exp $

% Copyright 2003, 2004 by Till Tantau <tantau@users.sourceforge.net>.
%
% This program can be redistributed and/or modified under the terms
% of the GNU Public License, version 2.


\section{License: The GNU Public License, Version 2}
\label{section-license}

The \pgfname\ package is distributed under the \textsc{gnu} public
license, version 2. In detail, this means the following (the following
text is copyrighted by the Free Software Foundation):

\subsubsection{Preamble}

The licenses for most software are designed to take away your freedom to
share and change it.  By contrast, the \textsc{gnu} General Public License is
intended to guarantee your freedom to share and change free software---to
make sure the software is free for all its users.  This General Public
License applies to most of the Free Software Foundation's software and to
any other program whose authors commit to using it.  (Some other Free
Software Foundation software is covered by the \textsc{gnu} Library General Public
License instead.)  You can apply it to your programs, too.

When we speak of free software, we are referring to freedom, not price.
Our General Public Licenses are designed to make sure that you have the
freedom to distribute copies of free software (and charge for this service
if you wish), that you receive source code or can get it if you want it,
that you can change the software or use pieces of it in new free programs;
and that you know you can do these things.

To protect your rights, we need to make restrictions that forbid anyone to
deny you these rights or to ask you to surrender the rights.  These
restrictions translate to certain responsibilities for you if you
distribute copies of the software, or if you modify it.

For example, if you distribute copies of such a program, whether gratis or
for a fee, you must give the recipients all the rights that you have.  You
must make sure that they, too, receive or can get the source code.  And
you must show them these terms so they know their rights.

We protect your rights with two steps: (1) copyright the software, and (2)
offer you this license which gives you legal permission to copy,
distribute and/or modify the software.

Also, for each author's protection and ours, we want to make certain that
everyone understands that there is no warranty for this free software.  If
the software is modified by someone else and passed on, we want its
recipients to know that what they have is not the original, so that any
problems introduced by others will not reflect on the original authors'
reputations.

Finally, any free program is threatened constantly by software patents.
We wish to avoid the danger that redistributors of a free program will
individually obtain patent licenses, in effect making the program
proprietary.  To prevent this, we have made it clear that any patent must
be licensed for everyone's free use or not licensed at all.

The precise terms and conditions for copying, distribution and
modification follow.

\subsubsection{Terms and Conditions For Copying, Distribution and
  Modification}

\begin{enumerate}

\addtocounter{enumi}{-1}

\item 
This License applies to any program or other work which contains a notice
placed by the copyright holder saying it may be distributed under the
terms of this General Public License.  The ``Program'', below, refers to
any such program or work, and a ``work based on the Program'' means either
the Program or any derivative work under copyright law: that is to say, a
work containing the Program or a portion of it, either verbatim or with
modifications and/or translated into another language.  (Hereinafter,
translation is included without limitation in the term ``modification''.)
Each licensee is addressed as ``you''.

Activities other than copying, distribution and modification are not
covered by this License; they are outside its scope.  The act of
running the Program is not restricted, and the output from the Program
is covered only if its contents constitute a work based on the
Program (independent of having been made by running the Program).
Whether that is true depends on what the Program does.

\item You may copy and distribute verbatim copies of the Program's source
  code as you receive it, in any medium, provided that you conspicuously
  and appropriately publish on each copy an appropriate copyright notice
  and disclaimer of warranty; keep intact all the notices that refer to
  this License and to the absence of any warranty; and give any other
  recipients of the Program a copy of this License along with the Program.

You may charge a fee for the physical act of transferring a copy, and you
may at your option offer warranty protection in exchange for a fee.

\item
You may modify your copy or copies of the Program or any portion
of it, thus forming a work based on the Program, and copy and
distribute such modifications or work under the terms of Section 1
above, provided that you also meet all of these conditions:

\begin{enumerate}

\item 
You must cause the modified files to carry prominent notices stating that
you changed the files and the date of any change.

\item
You must cause any work that you distribute or publish, that in
whole or in part contains or is derived from the Program or any
part thereof, to be licensed as a whole at no charge to all third
parties under the terms of this License.

\item
If the modified program normally reads commands interactively
when run, you must cause it, when started running for such
interactive use in the most ordinary way, to print or display an
announcement including an appropriate copyright notice and a
notice that there is no warranty (or else, saying that you provide
a warranty) and that users may redistribute the program under
these conditions, and telling the user how to view a copy of this
License.  (Exception: if the Program itself is interactive but
does not normally print such an announcement, your work based on
the Program is not required to print an announcement.)

\end{enumerate}


These requirements apply to the modified work as a whole.  If
identifiable sections of that work are not derived from the Program,
and can be reasonably considered independent and separate works in
themselves, then this License, and its terms, do not apply to those
sections when you distribute them as separate works.  But when you
distribute the same sections as part of a whole which is a work based
on the Program, the distribution of the whole must be on the terms of
this License, whose permissions for other licensees extend to the
entire whole, and thus to each and every part regardless of who wrote it.

Thus, it is not the intent of this section to claim rights or contest
your rights to work written entirely by you; rather, the intent is to
exercise the right to control the distribution of derivative or
collective works based on the Program.

In addition, mere aggregation of another work not based on the Program
with the Program (or with a work based on the Program) on a volume of
a storage or distribution medium does not bring the other work under
the scope of this License.

\item
You may copy and distribute the Program (or a work based on it,
under Section 2) in object code or executable form under the terms of
Sections 1 and 2 above provided that you also do one of the following:

\begin{enumerate}

\item
Accompany it with the complete corresponding machine-readable
source code, which must be distributed under the terms of Sections
1 and 2 above on a medium customarily used for software interchange; or,

\item
Accompany it with a written offer, valid for at least three
years, to give any third party, for a charge no more than your
cost of physically performing source distribution, a complete
machine-readable copy of the corresponding source code, to be
distributed under the terms of Sections 1 and 2 above on a medium
customarily used for software interchange; or,

\item
Accompany it with the information you received as to the offer
to distribute corresponding source code.  (This alternative is
allowed only for noncommercial distribution and only if you
received the program in object code or executable form with such
an offer, in accord with Subsubsection b above.)

\end{enumerate}


The source code for a work means the preferred form of the work for
making modifications to it.  For an executable work, complete source
code means all the source code for all modules it contains, plus any
associated interface definition files, plus the scripts used to
control compilation and installation of the executable.  However, as a
special exception, the source code distributed need not include
anything that is normally distributed (in either source or binary
form) with the major components (compiler, kernel, and so on) of the
operating system on which the executable runs, unless that component
itself accompanies the executable.

If distribution of executable or object code is made by offering
access to copy from a designated place, then offering equivalent
access to copy the source code from the same place counts as
distribution of the source code, even though third parties are not
compelled to copy the source along with the object code.

\item
You may not copy, modify, sublicense, or distribute the Program
except as expressly provided under this License.  Any attempt
otherwise to copy, modify, sublicense or distribute the Program is
void, and will automatically terminate your rights under this License.
However, parties who have received copies, or rights, from you under
this License will not have their licenses terminated so long as such
parties remain in full compliance.

\item
You are not required to accept this License, since you have not
signed it.  However, nothing else grants you permission to modify or
distribute the Program or its derivative works.  These actions are
prohibited by law if you do not accept this License.  Therefore, by
modifying or distributing the Program (or any work based on the
Program), you indicate your acceptance of this License to do so, and
all its terms and conditions for copying, distributing or modifying
the Program or works based on it.

\item
Each time you redistribute the Program (or any work based on the
Program), the recipient automatically receives a license from the
original licensor to copy, distribute or modify the Program subject to
these terms and conditions.  You may not impose any further
restrictions on the recipients' exercise of the rights granted herein.
You are not responsible for enforcing compliance by third parties to
this License.

\item
If, as a consequence of a court judgment or allegation of patent
infringement or for any other reason (not limited to patent issues),
conditions are imposed on you (whether by court order, agreement or
otherwise) that contradict the conditions of this License, they do not
excuse you from the conditions of this License.  If you cannot
distribute so as to satisfy simultaneously your obligations under this
License and any other pertinent obligations, then as a consequence you
may not distribute the Program at all.  For example, if a patent
license would not permit royalty-free redistribution of the Program by
all those who receive copies directly or indirectly through you, then
the only way you could satisfy both it and this License would be to
refrain entirely from distribution of the Program.

If any portion of this section is held invalid or unenforceable under
any particular circumstance, the balance of the section is intended to
apply and the section as a whole is intended to apply in other
circumstances.

It is not the purpose of this section to induce you to infringe any
patents or other property right claims or to contest validity of any
such claims; this section has the sole purpose of protecting the
integrity of the free software distribution system, which is
implemented by public license practices.  Many people have made
generous contributions to the wide range of software distributed
through that system in reliance on consistent application of that
system; it is up to the author/donor to decide if he or she is willing
to distribute software through any other system and a licensee cannot
impose that choice.

This section is intended to make thoroughly clear what is believed to
be a consequence of the rest of this License.

\item
If the distribution and/or use of the Program is restricted in
certain countries either by patents or by copyrighted interfaces, the
original copyright holder who places the Program under this License
may add an explicit geographical distribution limitation excluding
those countries, so that distribution is permitted only in or among
countries not thus excluded.  In such case, this License incorporates
the limitation as if written in the body of this License.

\item
The Free Software Foundation may publish revised and/or new versions
of the General Public License from time to time.  Such new versions will
be similar in spirit to the present version, but may differ in detail to
address new problems or concerns.

Each version is given a distinguishing version number.  If the Program
specifies a version number of this License which applies to it and ``any
later version'', you have the option of following the terms and conditions
either of that version or of any later version published by the Free
Software Foundation.  If the Program does not specify a version number of
this License, you may choose any version ever published by the Free Software
Foundation.

\item
If you wish to incorporate parts of the Program into other free
programs whose distribution conditions are different, write to the author
to ask for permission.  For software which is copyrighted by the Free
Software Foundation, write to the Free Software Foundation; we sometimes
make exceptions for this.  Our decision will be guided by the two goals
of preserving the free status of all derivatives of our free software and
of promoting the sharing and reuse of software generally.

\end{enumerate}

\subsubsection{No Warranty}

\begin{enumerate}

\addtocounter{enumi}{9}

\item
Because the program is licensed free of charge, there is no warranty
for the program, to the extent permitted by applicable law.  Except when
otherwise stated in writing the copyright holders and/or other parties
provide the program ``as is'' without warranty of any kind, either expressed
or implied, including, but not limited to, the implied warranties of
merchantability and fitness for a particular purpose.  The entire risk as
to the quality and performance of the program is with you.  Should the
program prove defective, you assume the cost of all necessary servicing,
repair or correction.

\item
In no event unless required by applicable law or agreed to in writing
will any copyright holder, or any other party who may modify and/or
redistribute the program as permitted above, be liable to you for damages,
including any general, special, incidental or consequential damages arising
out of the use or inability to use the program (including but not limited
to loss of data or data being rendered inaccurate or losses sustained by
you or third parties or a failure of the program to operate with any other
programs), even if such holder or other party has been advised of the
possibility of such damages.
\end{enumerate}

%%% Local Variables: 
%%% mode: latex
%%% TeX-master: "beameruserguide"
%%% End: 


% $Header: /cvsroot/pgf/pgf/doc/generic/pgf/text-en/pgfmanual-en-license.tex,v 1.2 2006/10/10 07:37:25 tantau Exp $

% Copyright 2003, 2004 by Till Tantau <tantau@users.sourceforge.net>.
%
% This program can be redistributed and/or modified under the terms
% of the GNU Public License, version 2.


\section{License: The GNU Public License, Version 2}
\label{section-license}

The \pgfname\ package is distributed under the \textsc{gnu} public
license, version 2. In detail, this means the following (the following
text is copyrighted by the Free Software Foundation):

\subsubsection{Preamble}

The licenses for most software are designed to take away your freedom to
share and change it.  By contrast, the \textsc{gnu} General Public License is
intended to guarantee your freedom to share and change free software---to
make sure the software is free for all its users.  This General Public
License applies to most of the Free Software Foundation's software and to
any other program whose authors commit to using it.  (Some other Free
Software Foundation software is covered by the \textsc{gnu} Library General Public
License instead.)  You can apply it to your programs, too.

When we speak of free software, we are referring to freedom, not price.
Our General Public Licenses are designed to make sure that you have the
freedom to distribute copies of free software (and charge for this service
if you wish), that you receive source code or can get it if you want it,
that you can change the software or use pieces of it in new free programs;
and that you know you can do these things.

To protect your rights, we need to make restrictions that forbid anyone to
deny you these rights or to ask you to surrender the rights.  These
restrictions translate to certain responsibilities for you if you
distribute copies of the software, or if you modify it.

For example, if you distribute copies of such a program, whether gratis or
for a fee, you must give the recipients all the rights that you have.  You
must make sure that they, too, receive or can get the source code.  And
you must show them these terms so they know their rights.

We protect your rights with two steps: (1) copyright the software, and (2)
offer you this license which gives you legal permission to copy,
distribute and/or modify the software.

Also, for each author's protection and ours, we want to make certain that
everyone understands that there is no warranty for this free software.  If
the software is modified by someone else and passed on, we want its
recipients to know that what they have is not the original, so that any
problems introduced by others will not reflect on the original authors'
reputations.

Finally, any free program is threatened constantly by software patents.
We wish to avoid the danger that redistributors of a free program will
individually obtain patent licenses, in effect making the program
proprietary.  To prevent this, we have made it clear that any patent must
be licensed for everyone's free use or not licensed at all.

The precise terms and conditions for copying, distribution and
modification follow.

\subsubsection{Terms and Conditions For Copying, Distribution and
  Modification}

\begin{enumerate}

\addtocounter{enumi}{-1}

\item 
This License applies to any program or other work which contains a notice
placed by the copyright holder saying it may be distributed under the
terms of this General Public License.  The ``Program'', below, refers to
any such program or work, and a ``work based on the Program'' means either
the Program or any derivative work under copyright law: that is to say, a
work containing the Program or a portion of it, either verbatim or with
modifications and/or translated into another language.  (Hereinafter,
translation is included without limitation in the term ``modification''.)
Each licensee is addressed as ``you''.

Activities other than copying, distribution and modification are not
covered by this License; they are outside its scope.  The act of
running the Program is not restricted, and the output from the Program
is covered only if its contents constitute a work based on the
Program (independent of having been made by running the Program).
Whether that is true depends on what the Program does.

\item You may copy and distribute verbatim copies of the Program's source
  code as you receive it, in any medium, provided that you conspicuously
  and appropriately publish on each copy an appropriate copyright notice
  and disclaimer of warranty; keep intact all the notices that refer to
  this License and to the absence of any warranty; and give any other
  recipients of the Program a copy of this License along with the Program.

You may charge a fee for the physical act of transferring a copy, and you
may at your option offer warranty protection in exchange for a fee.

\item
You may modify your copy or copies of the Program or any portion
of it, thus forming a work based on the Program, and copy and
distribute such modifications or work under the terms of Section 1
above, provided that you also meet all of these conditions:

\begin{enumerate}

\item 
You must cause the modified files to carry prominent notices stating that
you changed the files and the date of any change.

\item
You must cause any work that you distribute or publish, that in
whole or in part contains or is derived from the Program or any
part thereof, to be licensed as a whole at no charge to all third
parties under the terms of this License.

\item
If the modified program normally reads commands interactively
when run, you must cause it, when started running for such
interactive use in the most ordinary way, to print or display an
announcement including an appropriate copyright notice and a
notice that there is no warranty (or else, saying that you provide
a warranty) and that users may redistribute the program under
these conditions, and telling the user how to view a copy of this
License.  (Exception: if the Program itself is interactive but
does not normally print such an announcement, your work based on
the Program is not required to print an announcement.)

\end{enumerate}


These requirements apply to the modified work as a whole.  If
identifiable sections of that work are not derived from the Program,
and can be reasonably considered independent and separate works in
themselves, then this License, and its terms, do not apply to those
sections when you distribute them as separate works.  But when you
distribute the same sections as part of a whole which is a work based
on the Program, the distribution of the whole must be on the terms of
this License, whose permissions for other licensees extend to the
entire whole, and thus to each and every part regardless of who wrote it.

Thus, it is not the intent of this section to claim rights or contest
your rights to work written entirely by you; rather, the intent is to
exercise the right to control the distribution of derivative or
collective works based on the Program.

In addition, mere aggregation of another work not based on the Program
with the Program (or with a work based on the Program) on a volume of
a storage or distribution medium does not bring the other work under
the scope of this License.

\item
You may copy and distribute the Program (or a work based on it,
under Section 2) in object code or executable form under the terms of
Sections 1 and 2 above provided that you also do one of the following:

\begin{enumerate}

\item
Accompany it with the complete corresponding machine-readable
source code, which must be distributed under the terms of Sections
1 and 2 above on a medium customarily used for software interchange; or,

\item
Accompany it with a written offer, valid for at least three
years, to give any third party, for a charge no more than your
cost of physically performing source distribution, a complete
machine-readable copy of the corresponding source code, to be
distributed under the terms of Sections 1 and 2 above on a medium
customarily used for software interchange; or,

\item
Accompany it with the information you received as to the offer
to distribute corresponding source code.  (This alternative is
allowed only for noncommercial distribution and only if you
received the program in object code or executable form with such
an offer, in accord with Subsubsection b above.)

\end{enumerate}


The source code for a work means the preferred form of the work for
making modifications to it.  For an executable work, complete source
code means all the source code for all modules it contains, plus any
associated interface definition files, plus the scripts used to
control compilation and installation of the executable.  However, as a
special exception, the source code distributed need not include
anything that is normally distributed (in either source or binary
form) with the major components (compiler, kernel, and so on) of the
operating system on which the executable runs, unless that component
itself accompanies the executable.

If distribution of executable or object code is made by offering
access to copy from a designated place, then offering equivalent
access to copy the source code from the same place counts as
distribution of the source code, even though third parties are not
compelled to copy the source along with the object code.

\item
You may not copy, modify, sublicense, or distribute the Program
except as expressly provided under this License.  Any attempt
otherwise to copy, modify, sublicense or distribute the Program is
void, and will automatically terminate your rights under this License.
However, parties who have received copies, or rights, from you under
this License will not have their licenses terminated so long as such
parties remain in full compliance.

\item
You are not required to accept this License, since you have not
signed it.  However, nothing else grants you permission to modify or
distribute the Program or its derivative works.  These actions are
prohibited by law if you do not accept this License.  Therefore, by
modifying or distributing the Program (or any work based on the
Program), you indicate your acceptance of this License to do so, and
all its terms and conditions for copying, distributing or modifying
the Program or works based on it.

\item
Each time you redistribute the Program (or any work based on the
Program), the recipient automatically receives a license from the
original licensor to copy, distribute or modify the Program subject to
these terms and conditions.  You may not impose any further
restrictions on the recipients' exercise of the rights granted herein.
You are not responsible for enforcing compliance by third parties to
this License.

\item
If, as a consequence of a court judgment or allegation of patent
infringement or for any other reason (not limited to patent issues),
conditions are imposed on you (whether by court order, agreement or
otherwise) that contradict the conditions of this License, they do not
excuse you from the conditions of this License.  If you cannot
distribute so as to satisfy simultaneously your obligations under this
License and any other pertinent obligations, then as a consequence you
may not distribute the Program at all.  For example, if a patent
license would not permit royalty-free redistribution of the Program by
all those who receive copies directly or indirectly through you, then
the only way you could satisfy both it and this License would be to
refrain entirely from distribution of the Program.

If any portion of this section is held invalid or unenforceable under
any particular circumstance, the balance of the section is intended to
apply and the section as a whole is intended to apply in other
circumstances.

It is not the purpose of this section to induce you to infringe any
patents or other property right claims or to contest validity of any
such claims; this section has the sole purpose of protecting the
integrity of the free software distribution system, which is
implemented by public license practices.  Many people have made
generous contributions to the wide range of software distributed
through that system in reliance on consistent application of that
system; it is up to the author/donor to decide if he or she is willing
to distribute software through any other system and a licensee cannot
impose that choice.

This section is intended to make thoroughly clear what is believed to
be a consequence of the rest of this License.

\item
If the distribution and/or use of the Program is restricted in
certain countries either by patents or by copyrighted interfaces, the
original copyright holder who places the Program under this License
may add an explicit geographical distribution limitation excluding
those countries, so that distribution is permitted only in or among
countries not thus excluded.  In such case, this License incorporates
the limitation as if written in the body of this License.

\item
The Free Software Foundation may publish revised and/or new versions
of the General Public License from time to time.  Such new versions will
be similar in spirit to the present version, but may differ in detail to
address new problems or concerns.

Each version is given a distinguishing version number.  If the Program
specifies a version number of this License which applies to it and ``any
later version'', you have the option of following the terms and conditions
either of that version or of any later version published by the Free
Software Foundation.  If the Program does not specify a version number of
this License, you may choose any version ever published by the Free Software
Foundation.

\item
If you wish to incorporate parts of the Program into other free
programs whose distribution conditions are different, write to the author
to ask for permission.  For software which is copyrighted by the Free
Software Foundation, write to the Free Software Foundation; we sometimes
make exceptions for this.  Our decision will be guided by the two goals
of preserving the free status of all derivatives of our free software and
of promoting the sharing and reuse of software generally.

\end{enumerate}

\subsubsection{No Warranty}

\begin{enumerate}

\addtocounter{enumi}{9}

\item
Because the program is licensed free of charge, there is no warranty
for the program, to the extent permitted by applicable law.  Except when
otherwise stated in writing the copyright holders and/or other parties
provide the program ``as is'' without warranty of any kind, either expressed
or implied, including, but not limited to, the implied warranties of
merchantability and fitness for a particular purpose.  The entire risk as
to the quality and performance of the program is with you.  Should the
program prove defective, you assume the cost of all necessary servicing,
repair or correction.

\item
In no event unless required by applicable law or agreed to in writing
will any copyright holder, or any other party who may modify and/or
redistribute the program as permitted above, be liable to you for damages,
including any general, special, incidental or consequential damages arising
out of the use or inability to use the program (including but not limited
to loss of data or data being rendered inaccurate or losses sustained by
you or third parties or a failure of the program to operate with any other
programs), even if such holder or other party has been advised of the
possibility of such damages.
\end{enumerate}

%%% Local Variables: 
%%% mode: latex
%%% TeX-master: "beameruserguide"
%%% End: 

% Copyright 2006 by Till Tantau
%
% This file may be distributed and/or modified
%
% 1. under the LaTeX Project Public License and/or
% 2. under the GNU Free Documentation License.
%
% See the file doc/generic/pgf/licenses/LICENSE for more details.


\section{Input and Output Formats}
\label{section-formats}


\TeX\ was designed to be a flexible system. This is true both for the
\emph{input} for \TeX\ as well as for the \emph{output}. The present
section explains which input formats there are and how they are
supported by \pgfname. It also explains which different output formats
can be produced.



\subsection{Supported Input Formats}

\TeX\ does not prescribe exactly how your input should be
formatted. While it is \emph{customary} that, say, an opening brace
starts a scope in \TeX, this is by no means necessary. Likewise, it is
\emph{customary} that environments start with |\begin|, but \TeX\
could not really care less about the exact command name.

Even though \TeX\ can be reconfigured, users can not. For this reason,
certain \emph{input formats} specify a set of commands and conventions
how input for \TeX\ should be formatted. There are currently three
``major'' formats: Donald Knuth's original |plain| \TeX\ format,
Leslie Lamport's popular \LaTeX\ format, and Hans Hangen's Con\TeX t
format.


\subsubsection{Using the  \LaTeX\ Format}

Using \pgfname\ and \tikzname\ with the \LaTeX\ format is easy: You
say |\usepackage{pgf}| or |\usepackage{tikz}|. Usually, that is all
you need to do, all configuration will be done automatically and
(hopefully) correctly.

The style files used for the \LaTeX\ format reside in the subdirectory
|latex/pgf/| of the \pgfname-system. Mainly, what these files do is to
include files in the directory |generic/pgf|. For example, here is the
content of the file |latex/pgf/frontends/tikz.sty|:

\begin{codeexample}[code only]
% Copyright 2006 by Till Tantau
%
% This file may be distributed and/or modified
%
% 1. under the LaTeX Project Public License and/or
% 2. under the GNU Public License.
%
% See the file doc/generic/pgf/licenses/LICENSE for more details.


\RequirePackage{pgf,pgffor}

% Copyright 2006 by Till Tantau
%
% This file may be distributed and/or modified
%
% 1. under the LaTeX Project Public License and/or
% 2. under the GNU Public License.
%
% See the file doc/generic/pgf/licenses/LICENSE for more details.

\ProvidesPackageRCS[v\pgfversion] $Header: /cvsroot/pgf/pgf/generic/pgf/frontendlayer/Attic/tikz.code.tex,v 1.98 2007/10/29 15:27:31 tantau Exp $


% Always-present libraries:

\usepgflibrary{plothandlers}

% TikZ is a key family
\pgfkeys{/tikz/.is family}

\def\tikzset{\pgfqkeys{/tikz}}


\newdimen\tikz@lastx
\newdimen\tikz@lasty
\newdimen\tikz@lastxsaved
\newdimen\tikz@lastysaved

\newdimen\tikzleveldistance
\newdimen\tikzsiblingdistance

\newbox\tikz@figbox
\newbox\tikz@tempbox

\newcount\tikztreelevel
\newcount\tikznumberofchildren
\newcount\tikznumberofcurrentchild

\newcount\tikz@fig@count

\newif\iftikz@node@is@a@label
\newif\iftikz@snaked

\let\tikz@options=\pgfutil@empty
\def\tikz@addoption#1{\expandafter\def\expandafter\tikz@options\expandafter{\tikz@options#1}}
\def\tikz@addmode#1{\expandafter\def\expandafter\tikz@mode\expandafter{\tikz@mode#1}}
\def\tikz@addtransform#1{%
  \ifx\tikz@transform\relax%
    #1%
  \else%
    \expandafter\def\expandafter\tikz@transform\expandafter{\tikz@transform#1}%
  \fi%
}



% TikZ options:

% This command is supported for compatibility only:

\def\tikzoption#1{\pgfutil@ifnextchar[{\tikzoption@opt{#1}}{\tikzoption@noopt{#1}}}%}

\def\tikzoption@opt#1[#2]#3{\pgfkeysdef{/tikz/#1}{#3}\pgfkeyssetvalue{/tikz/#1/.@def}{#2}}
\def\tikzoption@noopt#1#2{\pgfkeysdef{/tikz/#1}{#2}\pgfkeyssetvalue{/tikz/#1/.@def}{\pgfkeysvaluerequired}}

% Baseline options
\tikzoption{baseline}[0pt]{\pgfutil@ifnextchar({\tikz@baseline@coordinate}{\tikz@baseline@simple}#1\@nil}%)
\def\tikz@baseline@simple#1\@nil{\pgfsetbaseline{#1}}
\def\tikz@baseline@coordinate#1\@nil{\pgfsetbaselinepointlater{\tikz@scan@one@point\pgfutil@firstofone#1}}

% Draw options
\tikzoption{line width}{\tikz@semiaddlinewidth{#1}}%

\def\tikz@semiaddlinewidth#1{\tikz@addoption{\pgfsetlinewidth{#1}}\pgfmathsetlength\pgflinewidth{#1}}

\tikzoption{cap}{\tikz@addoption{\csname pgfset#1cap\endcsname}}
\tikzoption{join}{\tikz@addoption{\csname pgfset#1join\endcsname}}
\tikzoption{miter limit}{\tikz@addoption{\pgfsetmiterlimit{#1}}}

\tikzoption{dash pattern}{% syntax: on 2pt off 3pt on 4pt ...
  \def\tikz@temp{#1}%
  \ifx\tikz@temp\pgfutil@empty%
    \def\tikz@dashpattern{}%
    \tikz@addoption{\pgfsetdash{}{0pt}}%
  \else%
    \def\tikz@dashpattern{}%
    \expandafter\tikz@scandashon\pgfutil@gobble#1o\@nil%
    \edef\tikz@temp{{\tikz@dashpattern}{\noexpand\tikz@dashphase}}%
    \expandafter\tikz@addoption\expandafter{\expandafter\pgfsetdash\tikz@temp}%
  \fi}
\tikzoption{dash phase}{%
  \def\tikz@dashphase{#1}%
  \edef\tikz@temp{{\tikz@dashpattern}{\noexpand\tikz@dashphase}}%
  \expandafter\tikz@addoption\expandafter{\expandafter\pgfsetdash\tikz@temp}%
}%
\def\tikz@dashphase{0pt}

\def\tikz@scandashon n#1o{%
  \expandafter\def\expandafter\tikz@dashpattern\expandafter{\tikz@dashpattern{#1}}%
  \pgfutil@ifnextchar\@nil{\pgfutil@gobble}{\tikz@scandashoff}}
\def\tikz@scandashoff ff#1o{%
  \expandafter\def\expandafter\tikz@dashpattern\expandafter{\tikz@dashpattern{#1}}%
  \pgfutil@ifnextchar\@nil{\pgfutil@gobble}{\tikz@scandashon}}

\tikzoption{draw opacity}{\tikz@addoption{\pgfsetstrokeopacity{#1}}}

% Double draw options
\tikzoption{double}[]{%
  \def\tikz@temp{#1}%
  \ifx\tikz@temp\tikz@nonetext%
    \tikz@addmode{\tikz@mode@doublefalse}%
  \else%
    \ifx\tikz@temp\pgfutil@empty%
    \else%
      \def\tikz@double@color{#1}%
    \fi%
    \tikz@addmode{\tikz@mode@doubletrue}%
  \fi}
\tikzoption{double distance}{%
  \pgfmathsetlength{\pgf@x}{#1}%
  \edef\tikz@double@width@distance{\the\pgf@x}%
  \tikz@addmode{\tikz@mode@doubletrue}}

\def\tikz@double@width@distance{0.6pt}
\def\tikz@double@color{white}

% Fill options

\tikzoption{even odd rule}[]{\tikz@addoption{\pgfseteorule}}
\tikzoption{nonzero rule}[]{\tikz@addoption{\pgfsetnonzerorule}}

\tikzoption{fill opacity}{\tikz@addoption{\pgfsetfillopacity{#1}}}


% Joined fill/draw options

\tikzoption{opacity}{\tikz@addoption{\pgfsetstrokeopacity{#1}\pgfsetfillopacity{#1}}}


% Main color options
\tikzoption{color}{%
  \tikz@addoption{%
    \ifx\tikz@fillcolor\pgfutil@empty%
      \ifx\tikz@strokecolor\pgfutil@empty%
      \else%
        \pgfsys@color@reset@inorderfalse%
        \let\tikz@strokecolor\pgfutil@empty%
        \let\tikz@fillcolor\pgfutil@empty%
      \fi%
    \else%
      \pgfsys@color@reset@inorderfalse%
      \let\tikz@strokecolor\pgfutil@empty%
      \let\tikz@fillcolor\pgfutil@empty%
    \fi%
    \pgfutil@colorlet{tikz@color}{#1}%
    \pgfutil@colorlet{.}{tikz@color}%
    \pgfsetcolor{.}%
    \pgfsys@color@reset@inordertrue%
  }%
  \def\tikz@textcolor{#1}}



% Rounding options
\tikzoption{rounded corners}[4pt]{\pgfsetcornersarced{\pgfpoint{#1}{#1}}}
\tikzoption{sharp corners}[]{\pgfsetcornersarced{\pgfpointorigin}}



% Coordinate options
\tikzoption{x}{\tikz@handle@vec{\pgfsetxvec}{\tikz@handle@x}#1\relax}
\tikzoption{y}{\tikz@handle@vec{\pgfsetyvec}{\tikz@handle@y}#1\relax}
\tikzoption{z}{\tikz@handle@vec{\pgfsetzvec}{\tikz@handle@z}#1\relax}

\def\tikz@handle@vec#1#2{\pgfutil@ifnextchar({\tikz@handle@coordinate#1}{\tikz@handle@single#2}}
\def\tikz@handle@coordinate#1{\tikz@scan@one@point#1}
\def\tikz@handle@single#1#2\relax{#1{#2}}
\def\tikz@handle@x#1{\pgfsetxvec{\pgfpoint{#1}{0pt}}}
\def\tikz@handle@y#1{\pgfsetyvec{\pgfpoint{0pt}{#1}}}
\def\tikz@handle@z#1{\pgfsetzvec{\pgfpoint{#1}{#1}}}


% Transformation options
\tikzoption{scale}{\tikz@addtransform{\pgftransformscale{#1}}}
\tikzoption{xscale}{\tikz@addtransform{\pgftransformxscale{#1}}}
\tikzoption{xslant}{\tikz@addtransform{\pgftransformxslant{#1}}}
\tikzoption{yscale}{\tikz@addtransform{\pgftransformyscale{#1}}}
\tikzoption{yslant}{\tikz@addtransform{\pgftransformyslant{#1}}}
\tikzoption{rotate}{\tikz@addtransform{\pgftransformrotate{#1}}}
\tikzoption{rotate around}{\tikz@addtransform{\tikz@rotatearound{#1}}}
\def\tikz@rotatearound#1{%
  \edef\tikz@temp{#1}% get rid of active stuff
  \expandafter\tikz@rotateparseA\tikz@temp%
}%
\def\tikz@rotateparseA#1:{%
  \def\tikz@temp@rot{#1}%
  \tikz@scan@one@point\tikz@rotateparseB%
}
\def\tikz@rotateparseB#1{%
  \pgf@process{#1}%
  \pgf@xc=\pgf@x%
  \pgf@yc=\pgf@y%
  \pgftransformshift{\pgfqpoint{\pgf@xc}{\pgf@yc}}%
  \pgftransformrotate{\tikz@temp@rot}%
  \pgftransformshift{\pgfqpoint{-\pgf@xc}{-\pgf@yc}}%
}

\tikzoption{shift}{\tikz@addtransform{\tikz@scan@one@point\pgftransformshift#1\relax}}
\tikzoption{xshift}{\tikz@addtransform{\pgftransformxshift{#1}}}
\tikzoption{yshift}{\tikz@addtransform{\pgftransformyshift{#1}}}
\tikzoption{cm}{\tikz@addtransform{\tikz@parse@cm#1\relax}}
\tikzoption{reset cm}[]{\tikz@addtransform{\pgftransformreset}}
\tikzoption{shift only}[]{\tikz@addtransform{\pgftransformresetnontranslations}}

\def\tikz@parse@cm#1,#2,#3,#4,{%
  \def\tikz@p@cm{{#1}{#2}{#3}{#4}}%
  \tikz@scan@one@point\tikz@parse@cmA}
\def\tikz@parse@cmA#1{%
  \expandafter\pgftransformcm\tikz@p@cm{#1}%
}



% Grid options
\tikzoption{xstep}{\def\tikz@grid@x{#1}}
\tikzoption{ystep}{\def\tikz@grid@y{#1}}
\tikzoption{step}{\tikz@handle@vec{\tikz@step@point}{\tikz@step@single}#1\relax}
\def\tikz@step@single#1{\def\tikz@grid@x{#1}\def\tikz@grid@y{#1}}
\def\tikz@step@point#1{\pgf@process{#1}\edef\tikz@grid@x{\the\pgf@x}\edef\tikz@grid@y{\the\pgf@y}}

\def\tikz@grid@x{1cm}
\def\tikz@grid@y{1cm}


% Path usage options
\newif\iftikz@mode@double
\newif\iftikz@mode@fill
\newif\iftikz@mode@draw
\newif\iftikz@mode@clip
\newif\iftikz@mode@boundary
\newif\iftikz@mode@shade
\let\tikz@mode=\pgfutil@empty

\def\tikz@nonetext{none}

\tikzoption{path only}[]{\let\tikz@mode=\pgfutil@empty}
\tikzoption{shade}[]{\tikz@addmode{\tikz@mode@shadetrue}}
\tikzoption{fill}[]{%
  \def\tikz@temp{#1}%
  \ifx\tikz@temp\tikz@nonetext%
    \tikz@addmode{\tikz@mode@fillfalse}%
  \else%
    \ifx\tikz@temp\pgfutil@empty%
    \else%
      \tikz@addoption{\pgfsetfillcolor{#1}}%
      \def\tikz@fillcolor{#1}%
    \fi%
    \tikz@addmode{\tikz@mode@filltrue}%
  \fi%
}
\tikzoption{draw}[]{%
  \def\tikz@temp{#1}%
  \ifx\tikz@temp\tikz@nonetext%
    \tikz@addmode{\tikz@mode@drawfalse}%
  \else%
    \ifx\tikz@temp\pgfutil@empty%
    \else%
      \tikz@addoption{\pgfsetstrokecolor{#1}}%
      \def\tikz@strokecolor{#1}%
    \fi%
    \tikz@addmode{\tikz@mode@drawtrue}%
  \fi%
}
\tikzoption{clip}[]{\tikz@addmode{\tikz@mode@cliptrue}}
\tikzoption{use as bounding box}[]{\tikz@addmode{\tikz@mode@boundarytrue}}

\tikzoption{save path}{\tikz@addmode{\pgfsyssoftpath@getcurrentpath#1\global\let#1=#1}}

\let\tikz@fillcolor=\pgfutil@empty
\let\tikz@strokecolor=\pgfutil@empty


% Pattern options
\tikzoption{pattern color}{\def\tikz@pattern@color{#1}}
\tikzoption{pattern}[]{%
  \def\tikz@temp{#1}%
  \ifx\tikz@temp\tikz@nonetext%
    \tikz@addmode{\tikz@mode@fillfalse}%
  \else%
    \ifx\tikz@temp\pgfutil@empty%
    \else%
      \tikz@addoption{\pgfsetfillpattern{#1}{\tikz@pattern@color}}%
      \def\tikz@pattern{#1}%
    \fi%
    \tikz@addmode{\tikz@mode@filltrue}%
  \fi%
}
\def\tikz@pattern@color{black}
\def\tikz@pattern{dots}


% Shading options
\tikzoption{shading}{\def\tikz@shading{#1}\tikz@addmode{\tikz@mode@shadetrue}}
\tikzoption{shading angle}{\def\tikz@shade@angle{#1}\tikz@addmode{\tikz@mode@shadetrue}}
\tikzoption{top color}{%
  \pgfutil@colorlet{tikz@axis@top}{#1}%
  \pgfutil@colorlet{tikz@axis@middle}{tikz@axis@top!50!tikz@axis@bottom}%
  \def\tikz@shading{axis}\def\tikz@shade@angle{0}\tikz@addmode{\tikz@mode@shadetrue}}
\tikzoption{bottom color}{%
  \pgfutil@colorlet{tikz@axis@bottom}{#1}%
  \pgfutil@colorlet{tikz@axis@middle}{tikz@axis@top!50!tikz@axis@bottom}%
  \def\tikz@shading{axis}\def\tikz@shade@angle{0}\tikz@addmode{\tikz@mode@shadetrue}}
\tikzoption{middle color}{%
  \pgfutil@colorlet{tikz@axis@middle}{#1}%
  \def\tikz@shading{axis}\tikz@addmode{\tikz@mode@shadetrue}}
\tikzoption{left color}{%
  \pgfutil@colorlet{tikz@axis@top}{#1}%
  \pgfutil@colorlet{tikz@axis@middle}{tikz@axis@top!50!tikz@axis@bottom}%
  \def\tikz@shading{axis}\def\tikz@shade@angle{90}\tikz@addmode{\tikz@mode@shadetrue}}
\tikzoption{right color}{%
  \pgfutil@colorlet{tikz@axis@bottom}{#1}%
  \pgfutil@colorlet{tikz@axis@middle}{tikz@axis@top!50!tikz@axis@bottom}%
  \def\tikz@shading{axis}\def\tikz@shade@angle{90}\tikz@addmode{\tikz@mode@shadetrue}}
\tikzoption{ball color}{\pgfutil@colorlet{tikz@ball}{#1}\def\tikz@shading{ball}\tikz@addmode{\tikz@mode@shadetrue}}
\tikzoption{inner color}{\pgfutil@colorlet{tikz@radial@inner}{#1}\def\tikz@shading{radial}\tikz@addmode{\tikz@mode@shadetrue}}
\tikzoption{outer color}{\pgfutil@colorlet{tikz@radial@outer}{#1}\def\tikz@shading{radial}\tikz@addmode{\tikz@mode@shadetrue}}

\def\tikz@shading{axis}
\def\tikz@shade@angle{0}

\pgfdeclareverticalshading[tikz@axis@top,tikz@axis@middle,tikz@axis@bottom]{axis}{100bp}{%
  color(0bp)=(tikz@axis@bottom);
  color(25bp)=(tikz@axis@bottom);
  color(50bp)=(tikz@axis@middle);
  color(75bp)=(tikz@axis@top);
  color(100bp)=(tikz@axis@top)}

\pgfutil@colorlet{tikz@axis@top}{gray}
\pgfutil@colorlet{tikz@axis@middle}{gray!50!white}
\pgfutil@colorlet{tikz@axis@bottom}{white}

\pgfdeclareradialshading[tikz@ball]{ball}{\pgfqpoint{-10bp}{10bp}}{%
 color(0bp)=(tikz@ball!15!white);
 color(9bp)=(tikz@ball!75!white);
 color(18bp)=(tikz@ball!70!black);
 color(25bp)=(tikz@ball!50!black);
 color(50bp)=(black)}

\pgfutil@colorlet{tikz@ball}{blue}

\pgfdeclareradialshading[tikz@radial@inner,tikz@radial@outer]{radial}{\pgfpointorigin}{%
 color(0bp)=(tikz@radial@inner);
 color(25bp)=(tikz@radial@outer);
 color(50bp)=(tikz@radial@outer)}

\pgfutil@colorlet{tikz@radial@inner}{gray}
\pgfutil@colorlet{tikz@radial@outer}{white}


% Pin options
\tikzoption{pin}{\pgfutil@ifnextchar[{\tikz@parse@pin}{\tikz@parse@pin[]}#1\pgf@nil}
\tikzoption{pin distance}{\def\tikz@pin@distance{#1}}
\tikzoption{pin edge}{\def\tikz@pin@edge@style{#1}}

\tikzoption{tikz@pin@post}[]{%
  \tikz@compute@direction{\tikz@label@angle}{\tikz@pin@distance}%
  \global\let\tikz@pin@edge@style@smuggle=\tikz@pin@edge@style%
}
\tikzoption{tikz@pre@pin@edge}[]{\def\pgf@marshal{\tikzstyle{tikz@pin@options}=}
  \expandafter\pgf@marshal\expandafter[\tikz@pin@edge@style@smuggle]%
}

\def\tikz@pin@distance{3ex}
\def\tikz@pin@edge@style{}

\def\tikz@parse@pin[#1]#2:#3\pgf@nil{%
  \tikz@add@after@node@path{\bgroup
    \pgfextra{\let\tikz@save@last@node=\tikzlastnode}%
    node
    [every pin,tikz@label@angle=#2,#1,at=(\tikzlastnode.\tikz@label@angle),%
    after node path={(\tikz@save@last@node) edge[every pin edge,tikz@pre@pin@edge,tikz@pin@options] (\tikzlastnode)},
    tikz@pin@post]
    {#3} \egroup}
}


% Label and pin options

\tikzoption{label}{\pgfutil@ifnextchar[{\tikz@parse@label}{\tikz@parse@label[]}#1\pgf@nil}
\tikzoption{label distance}{\def\tikz@label@distance{#1}}

\tikzoption{tikz@label@angle}{\def\tikz@label@angle{#1}\csname tikz@label@angle@is@#1\endcsname}

\tikzoption{tikz@label@post}[]{\tikz@compute@direction{\tikz@label@angle}{\tikz@label@distance}}

\def\tikz@label@distance{0pt}

\def\tikz@parse@label[#1]#2:#3\pgf@nil{%
  \tikz@add@after@node@path{
    \bgroup
    \pgfextra{\let\tikz@save@last@fig@name=\tikz@last@fig@name}%
    node
    [every label,%
    tikz@label@angle=#2,%
    #1,%
    at=(\tikzlastnode.\tikz@label@angle),tikz@label@post]%
    {#3}%
    \pgfextra{\global\let\tikz@last@fig@name=\tikz@save@last@fig@name}%
    \egroup%
  }
}

\expandafter\def\csname tikz@label@angle@is@right\endcsname{\def\tikz@label@angle{0}}
\expandafter\def\csname tikz@label@angle@is@above right\endcsname{\def\tikz@label@angle{45}}
\expandafter\def\csname tikz@label@angle@is@above\endcsname{\def\tikz@label@angle{90}}
\expandafter\def\csname tikz@label@angle@is@above left\endcsname{\def\tikz@label@angle{135}}
\expandafter\def\csname tikz@label@angle@is@left\endcsname{\def\tikz@label@angle{180}}
\expandafter\def\csname tikz@label@angle@is@below left\endcsname{\def\tikz@label@angle{225}}
\expandafter\def\csname tikz@label@angle@is@below\endcsname{\def\tikz@label@angle{270}}
\expandafter\def\csname tikz@label@angle@is@below right\endcsname{\def\tikz@label@angle{315}}

\def\tikz@compute@direction#1#2{%
  \let\tikz@do@auto@anchor=\relax
  \c@pgf@counta=#1\relax%
  \ifnum\c@pgf@counta<0\relax
    \advance\c@pgf@counta by 360\relax%
  \fi%
  \ifnum\c@pgf@counta>359\relax
    \advance\c@pgf@counta by-360\relax%
  \fi%
  \ifnum\c@pgf@counta<4\relax%
    \def\tikz@anchor{west}%
  \else\ifnum\c@pgf@counta<87\relax%
    \def\tikz@anchor{south west}%
  \else\ifnum\c@pgf@counta<94\relax%
    \def\tikz@anchor{south}%
  \else\ifnum\c@pgf@counta<177\relax%
    \def\tikz@anchor{south east}%
  \else\ifnum\c@pgf@counta<184\relax%
    \def\tikz@anchor{east}%
  \else\ifnum\c@pgf@counta<267\relax%
    \def\tikz@anchor{north east}%
  \else\ifnum\c@pgf@counta<274\relax%
    \def\tikz@anchor{north}%
  \else\ifnum\c@pgf@counta<357\relax%
    \def\tikz@anchor{north west}%
  \else%
    \def\tikz@anchor{west}%
  \fi\fi\fi\fi\fi\fi\fi\fi%
  \tikz@addtransform{\pgftransformshift{\pgfpointpolar{#1}{#2}}}%  
}



% General shape options
\tikzoption{name}{\edef\tikz@fig@name{#1}}

\tikzoption{at}{\tikz@scan@one@point\tikz@set@at#1}
\def\tikz@set@at#1{\def\tikz@node@at{#1}}%

\tikzoption{shape}{\edef\tikz@shape{#1}}

\tikzoption{nodes}{\tikzstyle{every node}+=[#1]}


% These are /pgf options now:

%\tikzoption{inner sep}{\def\pgfshapeinnerxsep{#1}\def\pgfshapeinnerysep{#1}}
%\tikzoption{inner xsep}{\def\pgfshapeinnerxsep{#1}}
%\tikzoption{inner ysep}{\def\pgfshapeinnerysep{#1}}

%\tikzoption{outer sep}{\def\pgfshapeouterxsep{#1}\def\pgfshapeouterysep{#1}}
%\tikzoption{outer xsep}{\def\pgfshapeouterxsep{#1}}
%\tikzoption{outer ysep}{\def\pgfshapeouterysep{#1}}

%\tikzoption{minimum width}{\def\pgfshapeminwidth{#1}}
%\tikzoption{minimum height}{\def\pgfshapeminheight{#1}}
%\tikzoption{minimum size}{\def\pgfshapeminwidth{#1}\def\pgfshapeminheight{#1}}

\tikzoption{aspect}{\pgfsetshapeaspect{#1}}

\tikzoption{after node path}{\tikz@add@after@node@path{#1}}%
\def\tikz@add@after@node@path#1{\expandafter\def\expandafter\tikz@after@node\expandafter{\tikz@after@node#1}}

\def\tikzaddafternodepathoption#1{%
  #1%
  \expandafter\def\expandafter\tikz@afternodepathoptions\expandafter{\tikz@afternodepathoptions#1}}

\let\tikz@afternodepathoptions=\pgfutil@empty

\tikzoption{anchor}{\def\tikz@anchor{#1}\let\tikz@do@auto@anchor=\relax}

\tikzoption{left}[]{\def\tikz@anchor{east}\tikz@possibly@transform{x}{-}{#1}}
\tikzoption{right}[]{\def\tikz@anchor{west}\tikz@possibly@transform{x}{}{#1}}
\tikzoption{above}[]{\def\tikz@anchor{south}\tikz@possibly@transform{y}{}{#1}}
\tikzoption{below}[]{\def\tikz@anchor{north}\tikz@possibly@transform{y}{-}{#1}}
\tikzoption{above left}[]%
  {\def\tikz@anchor{south east}%
    \tikz@possibly@transform{x}{-}{#1}\tikz@possibly@transform{y}{}{#1}}
\tikzoption{above right}[]%
  {\def\tikz@anchor{south west}%
    \tikz@possibly@transform{x}{}{#1}\tikz@possibly@transform{y}{}{#1}}
\tikzoption{below left}[]%
  {\def\tikz@anchor{north east}%
    \tikz@possibly@transform{x}{-}{#1}\tikz@possibly@transform{y}{-}{#1}}
\tikzoption{below right}[]%
  {\def\tikz@anchor{north west}%
    \tikz@possibly@transform{x}{}{#1}\tikz@possibly@transform{y}{-}{#1}}

\tikzoption{node distance}{\def\tikz@node@distance{#1}}
\def\tikz@node@distance{1cm}

\tikzoption{above of}{\tikz@of{#1}{90}}%
\tikzoption{below of}{\tikz@of{#1}{-90}}%
\tikzoption{left of}{\tikz@of{#1}{180}}%
\tikzoption{right of}{\tikz@of{#1}{0}}%
\tikzoption{above left of}{\tikz@of{#1}{135}}%
\tikzoption{below left of}{\tikz@of{#1}{-135}}%
\tikzoption{above right of}{\tikz@of{#1}{45}}%
\tikzoption{below right of}{\tikz@of{#1}{-45}}%

\def\tikz@of#1#2{%
  \def\tikz@anchor{center}%
  \let\tikz@do@auto@anchor=\relax%
  \tikz@addtransform{\pgftransformshift{\pgfpointpolar{#2}{\tikz@node@distance}}}%
  \def\tikz@node@at{\pgfpointanchor{#1}{center}}}
  
\tikzoption{transform shape}[true]{%
  \csname tikz@fullytransformed#1\endcsname%
  \iftikz@fullytransformed%
    \pgfresetnontranslationattimefalse%
  \else%
    \pgfresetnontranslationattimetrue%
  \fi%
}

\newif\iftikz@fullytransformed
\pgfresetnontranslationattimetrue%

\def\tikz@anchor{center}%
\def\tikz@shape{rectangle}%

\def\tikz@possibly@transform#1#2#3{%
  \let\tikz@do@auto@anchor=\relax%
  \def\tikz@test{#3}%
  \ifx\tikz@test\pgfutil@empty%
  \else%
    \pgfmathsetlength{\pgf@x}{#3}%
    \pgf@x=#2\pgf@x\relax%
    \edef\tikz@marshal{\noexpand\tikz@addtransform{%
        \expandafter\noexpand\csname  pgftransform#1shift\endcsname{\the\pgf@x}}}% 
    \tikz@marshal%
  \fi%
}


% Inter-picture options
\tikzoption{remember picture}[true]{\csname pgfrememberpicturepositiononpage#1\endcsname}
\tikzoption{overlay}[]{\pgf@relevantforpicturesizefalse}



% Line/curve label placement options
\tikzoption{sloped}[true]{\csname pgfslopedattime#1\endcsname}
\tikzoption{allow upside down}[true]{\csname pgfallowupsidedownattime#1\endcsname}

\tikzoption{pos}{\edef\tikz@time{#1}}

\tikzoption{auto}[]{\csname tikz@install@auto@anchor@#1\endcsname}
\tikzoption{swap}[]{%
  \def\tikz@temp{left}%
  \ifx\tikz@auto@anchor@direction\tikz@temp%
    \def\tikz@auto@anchor@direction{right}%
  \else%
    \def\tikz@auto@anchor@direction{left}%
  \fi%
}

\def\tikz@time{.5}

\def\tikz@install@auto@anchor@{\let\tikz@do@auto@anchor=\tikz@auto@anchor@on}
\def\tikz@install@auto@anchor@false{\let\tikz@do@auto@anchor=\relax}
\def\tikz@install@auto@anchor@left{\let\tikz@do@auto@anchor=\tikz@auto@anchor@on\def\tikz@auto@anchor@direction{left}}
\def\tikz@install@auto@anchor@right{\let\tikz@do@auto@anchor=\tikz@auto@anchor@on\def\tikz@auto@anchor@direction{right}}

\let\tikz@do@auto@anchor=\relax%

\def\tikz@auto@anchor@on{\csname tikz@auto@anchor@\tikz@auto@anchor@direction\endcsname}

\def\tikz@auto@anchor@left{\tikz@auto@pre\tikz@auto@anchor\tikz@auto@post}
\def\tikz@auto@anchor@right{\tikz@auto@pre\tikz@auto@anchor@prime\tikz@auto@post}

\def\tikz@auto@anchor@direction{left}

% Text options
\tikzoption{text}{\def\tikz@textcolor{#1}}
\tikzoption{font}{\def\tikz@textfont{#1}}
\tikzoption{text opacity}{\def\tikz@textopacity{#1}}
\tikzoption{text width}{\def\tikz@text@width{#1}}
\tikzoption{text height}{\def\tikz@text@height{#1}}
\tikzoption{text depth}{\def\tikz@text@depth{#1}}
\tikzoption{text ragged}[]%
{\def\tikz@text@action{\raggedright\rightskip\z@ plus2em \spaceskip.3333em \xspaceskip.5em\relax}}
\tikzoption{text badly ragged}[]{\def\tikz@text@action{\raggedright\relax}}
\tikzoption{text ragged left}[]%
{\def\tikz@text@action{\raggedleft\leftskip\z@ plus2em \spaceskip.3333em \xspaceskip.5em\relax}}
\tikzoption{text badly ragged left}[]{\def\tikz@text@action{\raggedleft\relax}}
\tikzoption{text justified}[]{\def\tikz@text@action{\leftskip\z@\rightskip\z@\relax}}
\tikzoption{text centered}[]{\def\tikz@text@action{%
  \leftskip\z@ plus2em%
  \rightskip\z@ plus2em%
  \spaceskip.3333em \xspaceskip.5em%
  \parfillskip=0pt%
  \let\\=\@centercr% for latex
  \relax}}
\tikzoption{text badly centered}[]%
{\def\tikz@text@action{%
  \let\\=\@centercr% for latex
  \parfillskip=0pt%
  \rightskip\@flushglue%
  \leftskip\@flushglue\relax}}

\let\tikz@text@width=\pgfutil@empty
\let\tikz@text@height=\pgfutil@empty
\let\tikz@text@depth=\pgfutil@empty
\let\tikz@textcolor=\pgfutil@empty
\let\tikz@textfont=\pgfutil@empty
\let\tikz@textopacity=\pgfutil@empty

\def\tikz@text@action{\raggedright\rightskip\z@ plus2em \spaceskip.3333em \xspaceskip.5em\relax}





% Arrow options
\tikzoption{arrows}{\tikz@processarrows{#1}}

\tikzoption{>}{%
  \tikz@set@pointed{\csname pgf@arrows@invert#1\endcsname}{#1}%
  \expandafter\tikz@processarrows\expandafter{\tikz@current@arrows}%
}

\tikzoption{shorten <}{\pgfsetshortenstart{#1}}
\tikzoption{shorten >}{\pgfsetshortenend{#1}}

\def\tikz@set@pointed#1#2{%
  \pgfutil@ifundefined{pgf@arrow@code@tikze@>@#2}
  {%
    \pgfarrowsdeclarealias{tikzs@<@#2}{tikze@>@#2}{#1}{#2}%
    \pgfarrowsdeclarereversed{tikzs@>@#2}{tikze@<@#2}{#1}{#2}%
    \pgfarrowsdeclarecombine*{tikz@|<@#2}{tikz@>|@#2}{#1}{#2}{|}{|}%
    \pgfarrowsdeclaredouble[\pgflinewidth]{tikzs@<<@#2}{tikze@>>@#2}{#1}{#2}%
    \pgfarrowsdeclarereversed{tikzs@>>@#2}{tikze@<<@#2}{tikzs@<<@#2}{tikze@>>@#2}%
  }{}%
  \pgfutil@namedef{tikz@special@arrow@start<}{tikzs@<@#2}%
  \pgfutil@namedef{tikz@special@arrow@end>}{tikze@>@#2}%
  \pgfutil@namedef{tikz@special@arrow@start>}{tikzs@>@#2}%
  \pgfutil@namedef{tikz@special@arrow@end<}{tikze@<@#2}%
  \pgfutil@namedef{tikz@special@arrow@start|<}{tikz@|<@#2}%
  \pgfutil@namedef{tikz@special@arrow@end>|}{tikz@>|@#2}%
  \pgfutil@namedef{tikz@special@arrow@start<<}{tikzs@<<@#2}%
  \pgfutil@namedef{tikz@special@arrow@end>>}{tikze@>>@#2}%
  \pgfutil@namedef{tikz@special@arrow@start>>}{tikzs@<<@#2}%
  \pgfutil@namedef{tikz@special@arrow@end<<}{tikze@>>@#2}%
}

\def\tikz@processarrows#1{%
  \def\tikz@current@arrows{#1}%
  \def\tikz@temp{#1}%
  \ifx\tikz@temp\pgfutil@empty%
  \else%
    \tikz@@processarrows#1\@nil
  \fi%
}
\def\tikz@@processarrows#1-#2\@nil{%
  \expandafter\ifx\csname tikz@special@arrow@start#1\endcsname\relax%
    \pgfsetarrowsstart{#1}
  \else%
    \pgfsetarrowsstart{\csname tikz@special@arrow@start#1\endcsname}%
  \fi%
  \expandafter\ifx\csname tikz@special@arrow@end#2\endcsname\relax%
    \pgfsetarrowsend{#2}
  \else%
    \pgfsetarrowsend{\csname tikz@special@arrow@end#2\endcsname}%
  \fi%
}

\tikz@set@pointed{\pgf@arrows@invertto}{to}
\def\tikz@current@arrows{-}

% Parabola options
\tikzoption{bend}{\tikz@scan@one@point\tikz@set@parabola@bend#1\relax}%
\tikzoption{bend pos}{\def\tikz@parabola@bend@factor{#1}}
\tikzoption{parabola height}{%
  \def\tikz@parabola@bend@factor{.5}%
  \def\tikz@parabola@bend{\pgfpointadd{\pgfpoint{0pt}{#1}}{\tikz@last@position@saved}}}

\def\tikz@parabola@bend{\tikz@last@position@saved}
\def\tikz@parabola@bend@factor{0}

\def\tikz@set@parabola@bend#1{\def\tikz@parabola@bend{#1}}

% Axis options
\tikzoption{domain}{\def\tikz@plot@domain{#1}\expandafter\tikz@plot@samples@recalc\tikz@plot@domain\relax}
\tikzoption{range}{\def\tikz@plot@range{#1}}

% Plot options
\tikzoption{smooth}[]{\let\tikz@plot@handler=\pgfplothandlercurveto}
\tikzoption{smooth cycle}[]{\let\tikz@plot@handler=\pgfplothandlerclosedcurve}
\tikzoption{sharp plot}[]{\let\tikz@plot@handler\pgfplothandlerlineto}

\tikzoption{tension}{\pgfsetplottension{#1}}

\tikzoption{xcomb}[]{\let\tikz@plot@handler=\pgfplothandlerxcomb}
\tikzoption{ycomb}[]{\let\tikz@plot@handler=\pgfplothandlerycomb}
\tikzoption{polar comb}[]{\let\tikz@plot@handler=\pgfplothandlerpolarcomb}

\tikzoption{raw gnuplot}[true]{\csname tikz@plot@raw@gnuplot#1\endcsname}
\tikzoption{prefix}{\def\tikz@plot@prefix{#1}}
\tikzoption{id}{\def\tikz@plot@id{#1}}

\tikzoption{samples}{\def\tikz@plot@samples{#1}\expandafter\tikz@plot@samples@recalc\tikz@plot@domain\relax}
\tikzoption{samples at}{\def\tikz@plot@samplesat{#1}}
\tikzoption{parametric}[true]{\csname tikz@plot@parametric#1\endcsname}

\tikzoption{variable}{\def\tikz@plot@var{#1}}

\tikzoption{only marks}[]{\let\tikz@plot@handler\pgfplothandlerdiscard}

\tikzoption{mark}{\def\tikz@plot@mark{#1}}
\tikzoption{mark options}{\def\tikz@plot@mark@options{#1}}
\tikzoption{mark size}{\pgfsetplotmarksize{#1}}

\tikzoption{mark indices}{\def\tikz@mark@list{#1}}
\tikzoption{mark phase}{\pgfsetplotmarkphase{#1}}
\tikzoption{mark repeat}{\pgfsetplotmarkrepeat{#1}}

\let\tikz@mark@list=\pgfutil@empty

\let\tikz@plot@mark@options=\pgfutil@empty

\let\tikz@plot@handler=\pgfplothandlerlineto
\let\tikz@plot@mark=\pgfutil@empty

\def\tikz@plot@samples{25}
\def\tikz@plot@domain{-5:5}
\def\tikz@plot@var{\x}
\def\tikz@plot@samplesat{-5,-4.6,...,5}
\def\tikz@plot@samples@recalc#1:#2\relax{%
  \pgfmathparse{#1}%
  \let\tikz@temp@start=\pgfmathresult%
  \pgfmathparse{#2}%
  \let\tikz@temp@end=\pgfmathresult%
  \pgfmathparse{\tikz@temp@start+(\tikz@temp@end-\tikz@temp@start)/\tikz@plot@samples}%
  \edef\tikz@plot@samplesat{\tikz@temp@start,\pgfmathresult,...,\tikz@temp@end}%
}


\def\tikz@plot@prefix{\jobname.}
\def\tikz@plot@id{pgf-plot}

\newif\iftikz@plot@parametric
\newif\iftikz@plot@raw@gnuplot


% To options
\tikzoption{to path}{\def\tikz@to@path{#1}}

\def\tikz@to@path{-- (\tikztotarget) \tikztonodes}



% Tree options
\tikzoption{edge from parent path}{\def\tikz@edge@to@parent@path{#1}}

\tikzoption{parent anchor}{\def\tikzparentanchor{.#1}\ifx\tikzparentanchor\tikz@border@text\let\tikzparentanchor\pgfutil@empty\fi}
\tikzoption{child anchor}{\def\tikzchildanchor{.#1}\ifx\tikzchildanchor\tikz@border@text\let\tikzchildanchor\pgfutil@empty\fi}

\tikzoption{level distance}{\pgfmathsetlength\tikzleveldistance{#1}}
\tikzoption{sibling distance}{\pgfmathsetlength\tikzsiblingdistance{#1}}

\tikzoption{growth function}{\let\tikz@grow=#1}
\tikzoption{growth parent anchor}{\def\tikz@growth@anchor{#1}}
\tikzoption{grow}{\tikz@set@growth{#1}\edef\tikz@special@level{\the\tikztreelevel}}%
\tikzoption{grow'}{\tikz@set@growth{#1}\tikz@swap@growth\edef\tikz@special@level{\the\tikztreelevel}}%

\def\tikz@growth@anchor{center}

\def\tikz@special@level{-1}% never

\def\tikz@swap@growth{%
  % Swap left and right
  \let\tikz@temp=\tikz@angle@grow@right%
  \let\tikz@angle@grow@right=\tikz@angle@grow@left%
  \let\tikz@angle@grow@left=\tikz@temp%
}%

\def\tikz@set@growth#1{%
  \let\tikz@grow=\tikz@grow@direction%
  \expandafter\ifx\csname tikz@grow@direction@#1\endcsname\relax%
    \c@pgf@counta=#1\relax%
  \else%
    \c@pgf@counta=\csname tikz@grow@direction@#1\endcsname%
  \fi%
  \edef\tikz@angle@grow{\the\c@pgf@counta}%
  \advance\c@pgf@counta by-90\relax%
  \edef\tikz@angle@grow@left{\the\c@pgf@counta}%
  \advance\c@pgf@counta by180\relax%
  \edef\tikz@angle@grow@right{\the\c@pgf@counta}%
}

\def\tikz@border@text{.border}
\let\tikzparentanchor=\pgfutil@empty
\let\tikzchildanchor=\pgfutil@empty
\def\tikz@edge@to@parent@path{(\tikzparentnode\tikzparentanchor) -- (\tikzchildnode\tikzchildanchor)}

\tikzleveldistance=15mm
\tikzsiblingdistance=15mm

\def\tikz@grow@direction@down{-90}
\def\tikz@grow@direction@up{90}
\def\tikz@grow@direction@left{180}
\def\tikz@grow@direction@right{0}

\def\tikz@grow@direction@south{-90}
\def\tikz@grow@direction@north{90}
\def\tikz@grow@direction@west{180}
\def\tikz@grow@direction@east{0}

\expandafter\def\csname tikz@grow@direction@north east\endcsname{45}
\expandafter\def\csname tikz@grow@direction@north west\endcsname{135}
\expandafter\def\csname tikz@grow@direction@south east\endcsname{-45}
\expandafter\def\csname tikz@grow@direction@south west\endcsname{-135}

\def\tikz@grow@direction{%
  \pgftransformshift{\pgfpointpolar{\tikz@angle@grow}{\tikzleveldistance}}%
  \ifnum\tikztreelevel=\tikz@special@level%
  \else%
    \pgf@xc=.5\tikzsiblingdistance%
    \c@pgf@counta=\tikznumberofchildren%
    \advance\c@pgf@counta by1\relax%
    \pgfutil@tempdima=\c@pgf@counta\pgf@xc%
    \pgftransformshift{\pgfpointpolar{\tikz@angle@grow@left}{\pgfutil@tempdima}}%
    \pgftransformshift{\pgfpointpolar{\tikz@angle@grow@right}{\tikznumberofcurrentchild\tikzsiblingdistance}}%
  \fi%
}

\tikzset{grow=down}


% Snake options
\tikzoption{snake}[]{%
  \def\tikz@@snake{#1}%
  \ifx\tikz@@snake\pgfutil@empty%
    \tikz@snakedtrue%
  \else%
    \ifx\tikz@@snake\tikz@nonetext%
      \tikz@snakedfalse%
    \else%
      \tikz@snakedtrue%
      \let\tikz@snake=\tikz@@snake%
    \fi%
  \fi}

\tikzoption{segment amplitude}{\pgfmathsetlength{\pgfsnakesegmentamplitude}{#1}}
\tikzoption{segment length}{\pgfmathsetlength{\pgfsnakesegmentlength}{#1}}
\tikzoption{segment angle}{\pgfmathparse{#1}\let\pgfsnakesegmentangle=\pgfmathresult}
\tikzoption{segment aspect}{\pgfmathparse{#1}\let\pgfsnakesegmentaspect=\pgfmathresult}

\tikzoption{segment object length}{\pgfmathparse{#1}\edef\pgfsnakesegmentobjectlength{\pgfmathresult pt}}

\tikzoption{raise snake}{\def\pgf@snake@raise{\pgftransformyshift{#1}}}
\tikzoption{mirror snake}[true]{%
  \csname if#1\endcsname
    \def\pgf@snake@mirror{\pgftransformyscale{-1}}%
  \else%
    \let\pgf@snake@mirror=\pgfutil@empty%
  \fi
}

\tikzoption{gap before snake}{\def\tikz@presnake{{moveto}{#1}}}
\tikzoption{line before snake}{\def\tikz@presnake{{lineto}{#1}}}

\tikzoption{gap after snake}{\def\tikz@postsnake{{moveto}{#1}}\def\tikz@mainsnakelength{\pgfsnakeremainingdistance+-#1}}
\tikzoption{line after snake}{\def\tikz@postsnake{{lineto}{#1}}\def\tikz@mainsnakelength{\pgfsnakeremainingdistance+-#1}}

\tikzoption{gap around snake}{%
  \def\tikz@presnake{{moveto}{#1}}%
  \def\tikz@postsnake{{moveto}{#1}}%
  \def\tikz@mainsnakelength{\pgfsnakeremainingdistance+-#1}%
}
\tikzoption{line around snake}{%
  \def\tikz@presnake{{lineto}{#1}}%
  \def\tikz@postsnake{{lineto}{#1}}%
  \def\tikz@mainsnakelength{\pgfsnakeremainingdistance+-#1}%
}
\let\pgf@snake@mirror=\pgfutil@empty
\let\pgf@snake@raise=\pgfutil@empty

\pgfsetsnakesegmenttransformation{\pgf@snake@mirror\pgf@snake@raise}

\def\tikz@snake{zigzag}

\let\tikz@presnake=\pgfutil@empty
\let\tikz@postsnake=\pgfutil@empty
\def\tikz@mainsnakelength{\pgfsnakeremainingdistance}


% Matrix options
\tikzoption{matrix}[true]{\csname tikz@is@matrix#1\endcsname}

\tikzoption{matrix anchor}{\def\tikz@matrix@anchor{#1}}

\tikzoption{column sep}{\def\pgfmatrixcolumnsep{#1}}
\tikzoption{row sep}{\def\pgfmatrixrowsep{#1}}

\tikzoption{cells}{\tikzstyle{every cell}+=[#1]}

\tikzoption{ampersand replacement}{\def\tikz@ampersand@replacement{#1}}

\newif\iftikz@is@matrix
\let\tikz@matrix@anchor=\pgfutil@empty
\let\tikz@ampersand@replacement=\pgfutil@empty

% Execute option

\tikzoption{execute at begin picture}{\expandafter\def\expandafter\tikz@atbegin@picture\expandafter{\tikz@atbegin@picture#1}}
\tikzoption{execute at end picture}{\expandafter\def\expandafter\tikz@atend@picture\expandafter{\tikz@atend@picture#1}}
\tikzoption{execute at begin scope}{\expandafter\def\expandafter\tikz@atbegin@scope\expandafter{\tikz@atbegin@scope#1}}
\tikzoption{execute at end scope}{\expandafter\def\expandafter\tikz@atend@scope\expandafter{\tikz@atend@scope#1}}
\tikzoption{execute at begin to}{\expandafter\def\expandafter\tikz@atbegin@to\expandafter{\tikz@atbegin@to#1}}
\tikzoption{execute at end to}{\expandafter\def\expandafter\tikz@atend@to\expandafter{\tikz@atend@to#1}}
\tikzoption{execute at begin node}{\expandafter\def\expandafter\tikz@atbegin@node\expandafter{\tikz@atbegin@node#1}}
\tikzoption{execute at end node}{\expandafter\def\expandafter\tikz@atend@node\expandafter{\tikz@atend@node#1}}
\tikzoption{execute at begin cell}{\expandafter\def\expandafter\tikz@atbegin@cell\expandafter{\tikz@atbegin@cell#1}}
\tikzoption{execute at end cell}{\expandafter\def\expandafter\tikz@atend@cell\expandafter{\tikz@atend@cell#1}}
\tikzoption{execute at empty cell}{\expandafter\def\expandafter\tikz@at@emptycell\expandafter{\tikz@at@emptycell#1}}

\let\tikz@atbegin@picture=\pgfutil@empty
\let\tikz@atend@picture=\pgfutil@empty
\let\tikz@atbegin@scope=\pgfutil@empty
\let\tikz@atend@scope=\pgfutil@empty
\let\tikz@atbegin@to=\pgfutil@empty
\let\tikz@atend@to=\pgfutil@empty
\let\tikz@atbegin@node=\pgfutil@empty
\let\tikz@atend@node=\pgfutil@empty
\let\tikz@atbegin@cell=\pgfutil@empty
\let\tikz@atend@cell=\pgfutil@empty
\let\tikz@at@emptycell=\pgfutil@empty




% Styles
\tikzoption{set style}{\tikzstyle#1}

% Handled in a special way.
\def\tikzstyle{\pgfutil@ifnextchar\bgroup\tikz@style@parseA\tikz@style@parseB}
\def\tikz@style@parseB#1={\tikz@style@parseA{#1}=}
\def\tikz@style@parseA#1#2=#3[#4]{% check for an optional argument
  \pgfutil@in@[{#2}%]
  \ifpgfutil@in@%
    \tikz@style@parseC{#1}#2={#4}%
  \else%
    \tikz@style@parseD{#1}#2={#4}%
  \fi%
}%

\def\tikz@style@parseC#1[#2]#3=#4{%
  \pgfkeys{/tikz/#1/.default={#2}}%
  \pgfutil@in@+{#3}%
  \ifpgfutil@in@%
    \pgfkeys{/tikz/#1/.append style={#4}}%
  \else%
    \pgfkeys{/tikz/#1/.style={#4}}%
  \fi}
\def\tikz@style@parseD#1#2=#3{%
  \pgfutil@in@+{#2}%
  \ifpgfutil@in@%
    \pgfkeys{/tikz/#1/.append style={#3}}%
  \else%
    \pgfkeys{/tikz/#1/.style={#3}}%
  \fi}


%
%
% Predefined styles
%
%

\tikzstyle{help lines}=              [color=gray,line width=0.2pt]

\tikzstyle{every picture}=           []
\tikzstyle{every path}=              []
\tikzstyle{every scope}=             []
\tikzstyle{every plot}=              []
\tikzstyle{every node}=              []
\tikzstyle{every child}=             []
\tikzstyle{every child node}=        []
\tikzstyle{every to}=                []
\tikzstyle{every cell}=              []
\tikzstyle{every matrix}=            []
\tikzstyle{every edge}=              [draw]
\tikzstyle{every label}=             [draw=none,fill=none]
\tikzstyle{every pin}=               [draw=none,fill=none]
\tikzstyle{every pin edge}=          [help lines]

\tikzstyle{ultra thin}=              [line width=0.1pt]
\tikzstyle{very thin}=               [line width=0.2pt]
\tikzstyle{thin}=                    [line width=0.4pt]
\tikzstyle{semithick}=               [line width=0.6pt]
\tikzstyle{thick}=                   [line width=0.8pt]
\tikzstyle{very thick}=              [line width=1.2pt]
\tikzstyle{ultra thick}=             [line width=1.6pt]

\tikzstyle{solid}=                   [dash pattern=]
\tikzstyle{dotted}=                  [dash pattern=on \pgflinewidth off 2pt]
\tikzstyle{densely dotted}=          [dash pattern=on \pgflinewidth off 1pt]
\tikzstyle{loosely dotted}=          [dash pattern=on \pgflinewidth off 4pt]
\tikzstyle{dashed}=                  [dash pattern=on 3pt off 3pt]
\tikzstyle{densely dashed}=          [dash pattern=on 3pt off 2pt]
\tikzstyle{loosely dashed}=          [dash pattern=on 3pt off 6pt]

\tikzstyle{transparent}=             [opacity=0]
\tikzstyle{ultra nearly transparent}=[opacity=0.05]
\tikzstyle{very nearly transparent}= [opacity=0.1]
\tikzstyle{nearly transparent}=      [opacity=0.25]
\tikzstyle{semitransparent}=         [opacity=0.5]
\tikzstyle{nearly opaque}=           [opacity=0.75]
\tikzstyle{very nearly opaque}=      [opacity=0.9]
\tikzstyle{ultra nearly opaque}=     [opacity=0.95]
\tikzstyle{opaque}=                  [opacity=1]

\tikzstyle{at start}=                [pos=0]
\tikzstyle{very near start}=         [pos=0.125]
\tikzstyle{near start}=              [pos=0.25]
\tikzstyle{midway}=                  [pos=0.5]
\tikzstyle{near end}=                [pos=0.75]
\tikzstyle{very near end}=           [pos=0.875]
\tikzstyle{at end}=                  [pos=1]

\tikzstyle{bend at start}=           [bend pos=0,bend={+(0,0)}]
\tikzstyle{bend at end}=             [bend pos=1,bend={+(0,0)}]

\tikzstyle{edge from parent}=        [draw]

\tikzstyle{snake triangles 45}=      [snake=triangles,segment object length=2.41421356\pgfsnakesegmentamplitude]
\tikzstyle{snake triangles 60}=      [snake=triangles,segment object length=1.73205081\pgfsnakesegmentamplitude]
\tikzstyle{snake triangles 90}=      [snake=triangles,segment object length=\pgfsnakesegmentamplitude]


%
% Setting keys
%

\pgfkeys{/tikz/style/.style=#1}

\pgfkeys{/tikz/.unknown/.code=%
  % Is it a pgf key?
  \let\tikz@key\pgfkeyscurrentname% 
  \pgfkeys{/pgf/\tikz@key/.try={#1}}%
  \ifpgfkeyssuccess%
  \else%
    \expandafter\pgfutil@in@\expandafter!\expandafter{\tikz@key}%
    \ifpgfutil@in@%
      % this is a color!
      \expandafter\tikz@addoption\expandafter{\expandafter\pgfutil@color\expandafter{\tikz@key}}%
      \edef\tikz@textcolor{\tikz@key}%
    \else%
      \pgfutil@doifcolorelse{\tikz@key}
      { %     
        \expandafter\tikz@addoption\expandafter{\expandafter\pgfutil@color\expandafter{\tikz@key}}%
        \edef\tikz@textcolor{\tikz@key}%
      }%
      {%
        % Ok, second chance: This might be an arrow specification:
        \expandafter\pgfutil@in@\expandafter-\expandafter{\tikz@key}
        \ifpgfutil@in@%
          % Ah, an arrow spec!
          \expandafter\tikz@processarrows\expandafter{\tikz@key}%
        \else%
          % Ok, third chance: A shape!
          \expandafter\ifx\csname pgf@sh@s@\tikz@key\endcsname\relax%
            \pgfkeys{/errors/unknown key={/tikz/\tikz@key}{#1}}%
          \else%
            \edef\tikz@shape{\tikz@key}%
          \fi%
        \fi%
      }%
    \fi%
  \fi%
}


%
% Main TikZ Environment
%

\def\tikzpicture{\pgfutil@ifnextchar[\tikz@picture{\tikz@picture[]}}%}
\def\tikz@picture[#1]{%
  \pgfpicture%
  \let\tikz@atbegin@picture=\pgfutil@empty%
  \let\tikz@atend@picture=\pgfutil@empty%
  \let\tikz@transform=\relax%
  \tikz@installcommands\scope[every picture,#1]%
  \tikz@atbegin@picture%
}
\def\endtikzpicture{%
    \tikz@atend@picture%
    \global\let\pgf@shift@baseline=\pgf@baseline%
    \global\let\pgf@remember@smuggle=\ifpgfrememberpicturepositiononpage%
    \endscope%
    \let\pgf@baseline=\pgf@shift@baseline%
    \let\ifpgfrememberpicturepositiononpage=\pgf@remember@smuggle%
  \endpgfpicture}

  

% Inlined picture
%
% #1 - some code to be put in a tikzpicture environment.
%
% If the command is not followed by braces, everything up to the next
% semicolon is used as argument.
%
% Example:
%
% The rectangle \tikz{\draw (0,0) rectangle (1em,1ex)} has width 1em and
% height 1ex.

\def\tikz{\pgfutil@ifnextchar[{\tikz@opt}{\tikz@opt[]}}
\def\tikz@opt[#1]{\tikzpicture[#1]\pgfutil@ifnextchar\bgroup{\tikz@}{\tikz@@}}
\def\tikz@#1{#1\endtikzpicture}
\def\tikz@@{%
  \let\tikz@next=\tikz@collectnormalsemicolon%
  \ifnum\the\catcode`\;=\active\relax%
    \let\tikz@next=\tikz@collectactivesemicolon%
  \fi%
  \tikz@next}
\def\tikz@collectnormalsemicolon#1;{#1;\endtikzpicture}
{
  \catcode`\;=\active
  \gdef\tikz@collectactivesemicolon#1;{#1;\endtikzpicture}
}



%
% Environment for scoping graphic state settings
%
\def\tikz@scope@env{\pgfutil@ifnextchar[\tikz@@scope@env{\tikz@@scope@env[]}}
\def\tikz@@scope@env[#1]{%
  \pgfscope%
  \begingroup%
  \let\tikz@atbegin@scope=\pgfutil@empty%
  \let\tikz@atend@scope=\pgfutil@empty%
  \let\tikz@options=\pgfutil@empty%
  \let\tikz@mode=\pgfutil@empty%
  \tikzset{every scope/.try,#1}%
  \tikz@options%
  \tikz@atbegin@scope%
}
\def\endtikz@scope@env{%
  \tikz@atend@scope%
  \endgroup%
  \endpgfscope%
}


%
% Install the abbreviated commands
%
\def\tikz@installcommands{%
  \ifnum\the\catcode`\;=\active\relax\expandafter\let\expandafter\tikz@origsemi\expandafter=\tikz@activesemicolon\fi%
  \ifnum\the\catcode`\:=\active\relax\expandafter\let\expandafter\tikz@origcolon\expandafter=\tikz@activecolon\fi%
  \ifnum\the\catcode`\|=\active\relax\expandafter\let\expandafter\tikz@origbar\expandafter=\tikz@activebar\fi%
  \let\tikz@origscope=\scope%
  \let\tikz@origendscope=\endscope%
  \let\tikz@origstartscope=\startscope%
  \let\tikz@origstopscope=\stopscope%
  \let\tikz@origpath=\path%
  \let\tikz@origagainpath=\againpath%
  \let\tikz@origdraw=\draw%
  \let\tikz@origpattern=\pattern%
  \let\tikz@origfill=\fill%
  \let\tikz@origfilldraw=\filldraw%
  \let\tikz@origshade=\shade%
  \let\tikz@origshadedraw=\shadedraw%
  \let\tikz@origclip=\clip%
  \let\tikz@origuseasboundingbox=\useasboundingbox%
  \let\tikz@orignode=\node%
  \let\tikz@origcoordinate=\coordinate%
  \let\tikz@origmatrix=\matrix%
  \let\tikz@origcalendar=\calendar%
  %
  \tikz@deactivatthings%
  %
  \let\scope=\tikz@scope@env%
  \let\endscope=\endtikz@scope@env%
  \let\startscope=\scope%
  \let\stopscope=\endscope%
  \let\path=\tikz@command@path%
  \let\againpath=\tikz@command@againpath%
  %
  \def\draw{\path[draw]}
  \def\pattern{\path[pattern]}
  \def\fill{\path[fill]}
  \def\filldraw{\path[fill,draw]}
  \def\shade{\path[shade]}
  \def\shadedraw{\path[shade,draw]}
  \def\clip{\path[clip]}
  \def\useasboundingbox{\path[use as bounding box]}
  \def\node{\tikz@path@overlay{node}}
  \def\coordinate{\tikz@path@overlay{coordinate}}
  \def\matrix{\tikz@path@overlay{node[matrix]}}
  \def\calendar{\tikz@lib@cal@calendar}%
}
\ifx\tikz@lib@cal@calendar\@undefined
\def\tikz@lib@cal@calendar{\PackageError{tikz}{You need to load the calendar library}{}}
\fi

\def\tikz@path@overlay#1{%
  \let\tikz@signal@path=\tikz@signal@path% for detection at begin of matrix cell
  \pgfutil@ifnextchar<{\tikz@path@overlayed{#1}}{\path #1}}
\def\tikz@path@overlayed#1<#2>{\path<#2> #1}

\def\tikz@uninstallcommands{%
  \ifnum\the\catcode`\;=\active\relax\expandafter\let\tikz@activesemicolon=\tikz@origsemi\fi%
  \ifnum\the\catcode`\:=\active\relax\expandafter\let\tikz@activecolon=\tikz@origcolon\fi%
  \ifnum\the\catcode`\|=\active\relax\expandafter\let\tikz@activebar=\tikz@origbar\fi%
  \let\scope=\tikz@origscope%
  \let\endscope=\tikz@origendscope%
  \let\startscope=\tikz@origstartscope%
  \let\stopscope=\tikz@origstopscope%
  \let\path=\tikz@origpath%
  \let\againpath=\tikz@origagainpath%
  \let\draw=\tikz@origdraw%
  \let\pattern=\tikz@origpattern%
  \let\fill=\tikz@origfill%
  \let\filldraw=\tikz@origfilldraw%
  \let\shade=\tikz@origshade%
  \let\shadedraw=\tikz@origshadedraw%
  \let\clip=\tikz@origclip%
  \let\useasboundingbox=\tikz@origuseasboundingbox%
  \let\node=\tikz@orignode%
  \let\coordinate=\tikz@origcoordinate%
  \let\matrix=\tikz@origmatrix%
  \let\calendar=\tikz@origcalendar%
}


{
  \catcode`\;=12
  \gdef\tikz@nonactivesemicolon{;}
  \catcode`\:=12
  \gdef\tikz@nonactivecolon{:}
  \catcode`\|=12
  \gdef\tikz@nonactivebar{|}
  \catcode`\;=\active
  \catcode`\:=\active
  \catcode`\|=\active
  \catcode`\"=\active
  \gdef\tikz@activesemicolon{;}%
  \gdef\tikz@activecolon{:}%
  \gdef\tikz@activebar{|}%
  \gdef\tikz@activequotes{"}%
  \gdef\tikz@deactivatthings{%
    \def;{\tikz@nonactivesemicolon}
    \def:{\tikz@nonactivecolon}
    \def|{\tikz@nonactivebar}
  }
}





% Constructs a path and draws/fills them according to the current
% settings.  

\def\tikz@command@path{%
  \let\tikz@signal@path=\tikz@signal@path% for detection at begin of matrix cell
  \pgfutil@ifnextchar[{\tikz@check@earg}%]
  {\pgfutil@ifnextchar<{\tikz@doopt}{\tikz@@command@path}}}
\def\tikz@signal@path{\tikz@signal@path}%
\def\tikz@check@earg[#1]{%
  \pgfutil@ifnextchar<{\tikz@swap@args[#1]}{\tikz@@command@path[#1]}}
\def\tikz@swap@args[#1]<#2>{\tikz@command@path<#2>[#1]}

\def\tikz@doopt{%
  \let\tikz@next=\tikz@eargnormalsemicolon%
  \ifnum\the\catcode`\;=\active\relax%
    \let\tikz@next=\tikz@eargactivesemicolon%
  \fi%
  \tikz@next}
\long\def\tikz@eargnormalsemicolon<#1>#2;{\only<#1>{\tikz@@command@path#2;}}
{
  \catcode`\;=\active
  \long\global\def\tikz@eargactivesemicolon<#1>#2;{\only<#1>{\tikz@@command@path#2;}}
}

\def\tikz@@command@path{%
  \edef\tikzscope@linewidth{\the\pgflinewidth}%
  \begingroup%
    \let\tikz@options=\pgfutil@empty%
    \let\tikz@mode=\pgfutil@empty%
    \let\tikz@moveto@waiting=\relax%
    \let\tikz@timer=\relax%
    \let\tikz@collected@onpath=\pgfutil@empty%
    \tikz@snakedfalse%
    \tikz@node@is@a@labelfalse%
    \tikz@expandcount=1000\relax%
    \tikz@lastx=0pt%
    \tikz@lasty=0pt%
    \tikz@lastxsaved=0pt%
    \tikz@lastysaved=0pt%
    \tikzset{every path/.try}%
    \tikz@scan@next@command%
}
\def\tikz@scan@next@command{%
  \ifx\tikz@collected@onpath\pgfutil@empty%
  \else%
    \tikz@invoke@collected@onpath%
  \fi%
  \afterassignment\tikz@handle\let\@let@token=%
}
\newcount\tikz@expandcount

% Central dispatcher for commands
\def\tikz@handle{%
  \let\@next=\tikz@expand%
    \ifx\@let@token(%)
      \let\@next=\tikz@movetoabs%
    \else%
      \ifx\@let@token+%
        \let\@next=\tikz@movetorel%
      \else%
        \ifx\@let@token-%
          \let\@next=\tikz@lineto%
        \else%
          \ifx\@let@token.%
            \let\@next=\tikz@dot%
          \else%
            \ifx\@let@token r%
              \let\@next=\tikz@rect%
            \else%
              \ifx\@let@token a%
                \let\@next=\tikz@arcA%
              \else%
                \ifx\@let@token[%]
                  \let\@next=\tikz@parse@options%
                \else%
                  \ifx\@let@token n%
                    \let\@next=\tikz@fig%
                  \else%
                    \ifx\@let@token\bgroup%
                      \let\@next=\tikz@beginscope%
                    \else%
                      \ifx\@let@token\egroup%
                        \let\@next=\tikz@endscope%
                      \else%
                        \ifx\@let@token;%
                          \let\@next=\tikz@finish%
                        \else%
                          \ifx\@let@token c%
                            \let\@next=\tikz@cchar%
                          \else%
                            \ifx\@let@token e%
                              \let\@next=\tikz@e@char%
                            \else%
                              \ifx\@let@token g%
                                \let\@next=\tikz@grid%
                              \else%
                                \ifx\@let@token s%
                                   \let\@next=\tikz@sine%
                                \else%
                                  \ifx\@let@token |%
                                     \let\@next=\tikz@vh@lineto%
                                  \else%
                                    \ifx\@let@token p%
                                      \let\@next=\tikz@pchar%
                                      \pgfsetmovetofirstplotpoint%
                                    \else%
                                      \ifx\@let@token t%
                                        \let\@next=\tikz@to%
                                      \else%
                                        \ifx\@let@token\pgfextra%
                                          \let\@next=\tikz@extra%
                                        \else%
                                          \ifx\@let@token\foreach%
                                            \let\@next=\tikz@foreach%
                                          \else%
                                            \ifx\@let@token\pgf@stop%
                                              \let\@next=\relax%
                                            \else%
                                              \ifx\@let@token\par%
                                                \let\@next=\tikz@scan@next@command%
                                              \fi%      
                                            \fi%      
                                          \fi%      
                                        \fi%      
                                      \fi%      
                                    \fi%  
                                  \fi%  
                                \fi%
                              \fi%
                            \fi%
                          \fi%
                        \fi%
                      \fi%  
                    \fi%  
                  \fi%  
                \fi%  
              \fi%
            \fi%
          \fi%
        \fi%
      \fi%
    \fi%
  \@next%
}

\def\tikz@pchar{\pgfutil@ifnextchar l{\tikz@plot}{\tikz@parabola}}
\def\tikz@cchar{%
  \pgfutil@ifnextchar i{\tikz@circle}%
  {\pgfutil@ifnextchar h{\tikz@children}{\tikz@cochar}}}%
\def\tikz@cochar o{%
  \pgfutil@ifnextchar o{\tikz@coordinate}{\tikz@cosine}}
\def\tikz@e@char{%
  \pgfutil@ifnextchar l{\tikz@ellipse}{\tikz@@e@char}}%
\def\tikz@@e@char dge{%
  \pgfutil@ifnextchar f{\tikz@edgetoparent}{\tikz@edge@plain}}%


\def\tikz@finish{%
  \tikz@mode@fillfalse%
  \tikz@mode@drawfalse%
  \tikz@mode@doublefalse%
  \tikz@mode@clipfalse%
  \tikz@mode@boundaryfalse%
  \edef\tikz@pathextend{%
    {\noexpand\pgfqpoint{\the\pgf@pathminx}{\the\pgf@pathminy}}%
    {\noexpand\pgfqpoint{\the\pgf@pathmaxx}{\the\pgf@pathmaxy}}%
  }%
  \tikz@mode% installs the mode settings
  % Rendering pipeline:  
  % 
  % Step 1: Setup options
  % 
  \ifx\tikz@options\pgfutil@empty%
  \else%
    \pgfsys@beginscope%
      \begingroup%
      \tikz@options%
  \fi%
  % 
  % Step 2: Do a fill if shade follows.
  %
  \iftikz@mode@fill%
    \iftikz@mode@shade%
      \pgfprocessround{\pgfsyssoftpath@currentpath}{\pgfsyssoftpath@currentpath}% change the current path
      \pgfsyssoftpath@invokecurrentpath%
      \pgfsys@fill%
      \tikz@mode@fillfalse% no more filling...
    \fi%
  \fi%
  % 
  % Step 3: Do a shade if necessary.
  %
  \iftikz@mode@shade%
    \pgfprocessround{\pgfsyssoftpath@currentpath}{\pgfsyssoftpath@currentpath}% change the current path
    \pgfshadepath{\tikz@shading}{\tikz@shade@angle}%
    \tikz@mode@shadefalse% no more shading...
  \fi%
  % 
  % Step 4: Double stroke, if necessary
  %
  \iftikz@mode@draw%
    \iftikz@mode@double%
      % Change line width
      \begingroup%
        \pgfsys@beginscope%
          \pgf@x=2\pgflinewidth%
          \advance\pgf@x by\tikz@double@width@distance%
          \pgflinewidth=\pgf@x%
          \pgfsetlinewidth{\the\pgflinewidth}%
    \fi%
  \fi%
  % 
  % Step 5: Do stroke/fill/clip as needed
  %
  \edef\tikz@temp{\noexpand\pgfusepath{%
    \iftikz@mode@fill fill,\fi%
    \iftikz@mode@draw draw,\fi%
    \iftikz@mode@clip clip,\fi%
    }}%
  \tikz@temp%
  \tikz@mode@fillfalse% no more filling
  % 
  % Step 6: Double stroke, if necessary
  %
  \iftikz@mode@draw%
    \iftikz@mode@double%
          \pgfsyssoftpath@setcurrentpath\pgf@last@used@path% reinstall
          \pgf@x=\tikz@double@width@distance%
          \pgfsetlinewidth{\the\pgf@x}%
          \pgfsetstrokecolor{\tikz@double@color}%
          \pgfsyssoftpath@flushcurrentpath%
          \pgfsys@stroke%
        \pgfsys@endscope%
        \pgf@add@arrows@as@needed
      \endgroup%
    \fi%
  \fi%
  \tikz@mode@drawfalse% no more stroking
  % 
  % Step 7: Add labels and nodes
  %
  \copy\tikz@figbox%
  \setbox\tikz@figbox=\box\voidb@x%
  %
  % Step 8: Close option brace
  %
  \ifx\tikz@options\pgfutil@empty%
  \else%
      \endgroup%
    \pgfsys@endscope%
    \iftikz@mode@clip%
      \PackageError{tikz}{Extra options not allowed for clipping path command.}{}%
    \fi%
  \fi%
  \iftikz@mode@clip%
    \aftergroup\pgf@relevantforpicturesizefalse%
  \fi%
  \iftikz@mode@boundary%
    \aftergroup\pgf@relevantforpicturesizefalse%
  \fi%
  \endgroup%
  \global\pgflinewidth=\tikzscope@linewidth%
}




\def\tikz@skip#1{\tikz@scan@next@command#1}
\def\tikz@expand{%
  \advance\tikz@expandcount by -1%
  \ifnum\tikz@expandcount<0\relax%
    \PackageError{tikz}{Giving up on this path. Did you forget a semicolon?}{}%
    \let\@next=\tikz@finish%
  \else%
    \let\@next=\tikz@@expand
  \fi%
  \@next}

\def\tikz@@expand{%
  \expandafter\tikz@scan@next@command\@let@token}



% Syntax for scopes: 
% {scoped path commands}

\def\tikz@beginscope{\begingroup\tikz@scan@next@command}
\def\tikz@endscope{%
  \global\setbox\tikz@tempbox=\copy\tikz@figbox%
  \endgroup%
  \setbox\tikz@figbox=\box\tikz@tempbox%
  \tikz@scan@next@command}


% Syntax for pgfextra: 
% \pgfextra {normal tex text}
% \pgfextra normal tex text \endpgfextra

\def\tikz@extra{\pgfutil@ifnextchar\bgroup\tikz@@extra\relax}
\long\def\tikz@@extra#1{#1\tikz@scan@next@command}
\let\endpgfextra=\tikz@scan@next@command

\def\pgfextra{pgfextra}


% Syntax for \foreach: 
% \foreach \var in {list} {path text}
%
% Example:
%
% \draw (0,0) \foreach \x in {1,2,3} {-- (\x,0) circle (1cm)} -- (5,5);

\def\tikz@foreach{%
  \def\pgffor@beginhook{\setbox\tikz@figbox=\box\tikz@tempbox\expandafter\tikz@scan@next@command\pgfutil@firstofone}%
  \def\pgffor@endhook{\pgfextra{%
      \xdef\tikz@foreach@save@lastx{\the\tikz@lastx}%
      \xdef\tikz@foreach@save@lasty{\the\tikz@lasty}%
      \xdef\tikz@foreach@save@lastxsaved{\the\tikz@lastxsaved}%
      \xdef\tikz@foreach@save@lastysaved{\the\tikz@lastysaved}%
      \global\setbox\tikz@tempbox=\copy\tikz@figbox\pgfutil@gobble}}%
  \def\pgffor@afterhook{%
    \tikz@lastx=\tikz@foreach@save@lastx%
    \tikz@lasty=\tikz@foreach@save@lasty%
    \tikz@lastxsaved=\tikz@foreach@save@lastxsaved%
    \tikz@lastysaved=\tikz@foreach@save@lastysaved%
    \setbox\tikz@figbox=\box\tikz@tempbox\tikz@scan@next@command}%
  \global\setbox\tikz@tempbox=\copy\tikz@figbox%
  \foreach}

  
% Syntax for againpath: 
% \againpath \somepathname

\def\tikz@command@againpath#1{%
  \pgfextra{%
    \pgfsyssoftpath@getcurrentpath\tikz@temp%
    \expandafter\pgfutil@g@addto@macro\expandafter\tikz@temp\expandafter{#1}%
    \pgfsyssoftpath@setcurrentpath\tikz@temp%
  }
}




%
% When this if is set, a just-scanned point is a shape and its border
% position still needs to be determined, depending on subsequent
% commands. 
%

\newif\iftikz@shapeborder


% Syntax for moveto: 
% <point>
\def\tikz@movetoabs{\tikz@moveto(}
\def\tikz@movetorel{\tikz@moveto+}
\def\tikz@moveto{%
  \tikz@scan@one@point{\tikz@@moveto}}
\def\tikz@@moveto#1{%
  \tikz@make@last@position{#1}%
  \iftikz@shapeborder%
    % ok, the moveto will have to wait. flag that we have a moveto in
    % wainting:
    \edef\tikz@moveto@waiting{\tikz@shapeborder@name}%
  \else%
    \pgfpathmoveto{\tikz@last@position}%
    \let\tikz@moveto@waiting=\relax%
  \fi%
  \tikz@scan@next@command%
}

\let\tikz@moveto@waiting=\relax % normally, nothing is waiting...

\def\tikz@flush@moveto{%
  \ifx\tikz@moveto@waiting\relax%
  \else%
    \pgfpathmoveto{\tikz@last@position}%
  \fi%
  \let\tikz@moveto@waiting=\relax%
}


\def\tikz@flush@moveto@toward#1#2#3{%
  % #1 = a point towards which the last moveto should be corrected
  % #2 = a dimension to which the corrected x-coordinate should be stored
  % #3 = a dimension for the corrected y-coordinate
  \ifx\tikz@moveto@waiting\relax%
    % do nothing
  \else%
    \pgf@process{\pgfpointshapeborder{\tikz@moveto@waiting}{#1}}%
    #2=\pgf@x%
    #3=\pgf@y%
    \edef\tikz@timer@start{\noexpand\pgfqpoint{\the\pgf@x}{\the\pgf@y}}%
    \pgfpathmoveto{\pgfqpoint{\pgf@x}{\pgf@y}}%
  \fi%
  \let\tikz@moveto@waiting=\relax%
}


%
% Collecting labels on the path 
%

\def\tikz@collect@coordinate@onpath#1coordinate
\def\tikz@@collect@coordinate@opt#1[#2]{%
  \pgfutil@ifnextchar({\tikz@@collect@coordinate#1[#2]}
\def\tikz@@collect@coordinate#1[#2](#3){%
  \tikz@collect@label@onpath#1node[shape=coordinate,#2](#3){}}

\def\tikz@collect@label@onpath#1node{%
  \expandafter\def\expandafter\tikz@collected@onpath\expandafter{\tikz@collected@onpath node}%
  \tikz@collect@label@scan#1}

\def\tikz@collect@label@scan#1{%  
  \pgfutil@ifnextchar({\tikz@collect@paran#1}%
  {\pgfutil@ifnextchar[{\tikz@collect@options#1}%
    {\pgfutil@ifnextchar\bgroup{\tikz@collect@arg#1}%
      {#1}}}%
}%}}

\def\tikz@collect@paran#1(#2){%
  \expandafter\def\expandafter\tikz@collected@onpath\expandafter{\tikz@collected@onpath(#2)}%
  \tikz@collect@label@scan#1%
}
\def\tikz@collect@options#1[#2]{%
  \expandafter\def\expandafter\tikz@collected@onpath\expandafter{\tikz@collected@onpath[#2]}%
  \tikz@collect@label@scan#1%
}
\def\tikz@collect@arg#1#2{%
  \expandafter\def\expandafter\tikz@collected@onpath\expandafter{\tikz@collected@onpath{#2}}%
  #1%
}


\def\tikz@invoke@collected@onpath{%
  \tikz@node@is@a@labeltrue%
  \let\tikz@temp=\tikz@collected@onpath%
  \let\tikz@collected@onpath=\pgfutil@empty%
  \expandafter\tikz@scan@next@command\tikz@temp\pgf@stop%
  \tikz@node@is@a@labelfalse%
}




% Syntax for lineto: 
% -- <point>

\def\tikz@lineto{%
  \pgfutil@ifnextchar |%
  {\expandafter\tikz@hv@lineto\pgfutil@gobble}%
  {\expandafter\pgfutil@ifnextchar\tikz@activebar{\expandafter\tikz@hv@lineto\pgfutil@gobble}%
    {\expandafter\tikz@lineto@mid\pgfutil@gobble}}}
\def\tikz@lineto@mid{%
  \pgfutil@ifnextchar n{\tikz@collect@label@onpath\tikz@lineto@mid}%
  {%
    \pgfutil@ifnextchar c{\tikz@close}{%
      \pgfutil@ifnextchar p{\pgfsetlinetofirstplotpoint\expandafter\tikz@plot\pgfutil@gobble}%
        {\tikz@scan@one@point{\tikz@@lineto}}}}}
\def\tikz@@lineto#1{%
  % Record the starting point for later labels on the path:
  \edef\tikz@timer@start{\noexpand\pgfqpoint{\the\tikz@lastx}{\the\tikz@lasty}}
  \iftikz@shapeborder%
    % ok, target is a shape. recalculate end
    \pgf@process{\pgfpointshapeborder{\tikz@shapeborder@name}{\tikz@last@position}}%
    \tikz@make@last@position{\pgfqpoint{\pgf@x}{\pgf@y}}%
    \tikz@flush@moveto@toward{\tikz@last@position}\pgf@x\pgf@y%
    \tikz@path@lineto{\tikz@last@position}%
    \edef\tikz@timer@end{\noexpand\pgfqpoint{\the\tikz@lastx}{\the\tikz@lasty}}%
    \tikz@make@last@position{#1}%
    \edef\tikz@moveto@waiting{\tikz@shapeborder@name}%    
  \else%
    % target is a reasonable point...
    % Record the starting point for later labels on the path:
    \tikz@make@last@position{#1}%
    \tikz@flush@moveto@toward{\tikz@last@position}\pgf@x\pgf@y%
    \tikz@path@lineto{\tikz@last@position}%
    \edef\tikz@timer@end{\noexpand\pgfqpoint{\the\tikz@lastx}{\the\tikz@lasty}}%
  \fi%
  \let\tikz@timer=\tikz@timer@line%
  \tikz@scan@next@command%
}

% snake or lineto?
\def\tikz@path@lineto#1{%
  \iftikz@snaked%
    {
      \pgfsyssoftpathmovetorelevantfalse%
      \pgfpathsnakesto{\tikz@presnake,{\tikz@snake}{\tikz@mainsnakelength},\tikz@postsnake}{#1}%
    }
  \else%
    \pgfpathlineto{#1}%
  \fi%
}

% snake or lineto?
\def\tikz@path@close#1{%
  \iftikz@snaked%
    {%
      \pgftransformreset%
      \pgfpathsnakesto{\tikz@presnake,{\tikz@snake}{\tikz@mainsnakelength},\tikz@postsnake}{#1}%
    }%
    \pgfpathclose%
  \else%
    \pgfpathclose%
  \fi%
}


% Syntax for lineto horizontal/vertical: 
% -| <point>

\def\tikz@hv@lineto{%
  \pgfutil@ifnextchar n
  {\tikz@collect@label@onpath\tikz@hv@lineto}
  {\pgfutil@ifnextchar c{\tikz@collect@coordinate@onpath\tikz@hv@lineto}%
    {\tikz@scan@one@point{\tikz@@hv@lineto}}}}
\def\tikz@@hv@lineto#1{%
  \edef\tikz@timer@start{\noexpand\pgfqpoint{\the\tikz@lastx}{\the\tikz@lasty}}%
  \pgf@yc=\tikz@lasty%
  \tikz@make@last@position{#1}%
  \tikz@flush@moveto@toward{\pgfqpoint{\tikz@lastx}{\pgf@yc}}\pgf@x\pgf@yc%
  \iftikz@shapeborder%
    % ok, target is a shape. have to work now:
    {%
      \pgf@process{\pgfpointshapeborder{\tikz@shapeborder@name}{\pgfqpoint{\tikz@lastx}{\pgf@yc}}}%
      \tikz@make@last@position{\pgfqpoint{\pgf@x}{\pgf@y}}%
      \tikz@path@lineto{\pgfqpoint{\tikz@lastx}{\pgf@yc}}%
      \tikz@path@lineto{\tikz@last@position}%
      \xdef\tikz@timer@end@temp{\noexpand\pgfqpoint{\the\tikz@lastx}{\the\tikz@lasty}}% move out of group
    }%
    \let\tikz@timer@end=\tikz@timer@end@temp%
    \edef\tikz@moveto@waiting{\tikz@shapeborder@name}%    
  \else%
    \tikz@path@lineto{\pgfqpoint{\tikz@lastx}{\pgf@yc}}%
    \tikz@path@lineto{\tikz@last@position}%
    \edef\tikz@timer@end{\noexpand\pgfqpoint{\the\tikz@lastx}{\the\tikz@lasty}}% move out of group
  \fi%
  \let\tikz@timer=\tikz@timer@hvline%
  \tikz@scan@next@command%
}

% Syntax for lineto vertical/horizontal: 
% |- <point>

\def\tikz@vh@lineto-{\tikz@vh@lineto@next}
\def\tikz@vh@lineto@next{%
  \pgfutil@ifnextchar n
  {\tikz@collect@label@onpath\tikz@vh@lineto@next}
  {\pgfutil@ifnextchar c{\tikz@collect@coordinate@onpath\tikz@vh@lineto@next}%
    {\tikz@scan@one@point\tikz@@vh@lineto}}}
\def\tikz@@vh@lineto#1{%
  \edef\tikz@timer@start{\noexpand\pgfqpoint{\the\tikz@lastx}{\the\tikz@lasty}}%
  \pgf@xc=\tikz@lastx%
  \tikz@make@last@position{#1}%
  \tikz@flush@moveto@toward{\pgfqpoint{\pgf@xc}{\tikz@lasty}}\pgf@xc\pgf@y%
  \iftikz@shapeborder%
    % ok, target is a shape. have to work now:
    {%
      \pgf@process{\pgfpointshapeborder{\tikz@shapeborder@name}{\pgfqpoint{\pgf@xc}{\tikz@lasty}}}%
      \tikz@make@last@position{\pgfqpoint{\pgf@x}{\pgf@y}}%
      \tikz@path@lineto{\pgfqpoint{\pgf@xc}{\tikz@lasty}}%
      \tikz@path@lineto{\tikz@last@position}%
      \xdef\tikz@timer@end@temp{\noexpand\pgfqpoint{\the\tikz@lastx}{\the\tikz@lasty}}% move out of group
    }%
    \let\tikz@timer@end=\tikz@timer@end@temp%
    \edef\tikz@moveto@waiting{\tikz@shapeborder@name}%    
  \else%
    \tikz@path@lineto{\pgfqpoint{\pgf@xc}{\tikz@lasty}}%
    \tikz@path@lineto{\tikz@last@position}%
    \edef\tikz@timer@end{\noexpand\pgfqpoint{\the\tikz@lastx}{\the\tikz@lasty}}%
  \fi%
  \let\tikz@timer=\tikz@timer@vhline%
  \tikz@scan@next@command%
}

% Syntax for cycle: 
% -- cycle
\def\tikz@close c{%
  \pgfutil@ifnextchar o{\tikz@collect@coordinate@onpath\tikz@lineto@mid c}% oops, a coordinate
  {\tikz@@close c}}%
\def\tikz@@close cycle{%
  \tikz@flush@moveto%
  \tikz@path@close{\expandafter\pgfpoint\pgfsyssoftpath@lastmoveto}%
  \def\pgfstrokehook{}%
  \let\tikz@timer=\@undefined%
  \tikz@scan@next@command%
}


% Syntax for options: 
% [options]
\def\tikz@parse@options#1]{%
  \tikzset{#1}%
  \tikz@scan@next@command%
}

% Syntax for edges:
% edge [options] (coordinate)
% edge [options] node {node text} (coordinate)
\def\tikz@edge@plain{%
  \begingroup%
    \tikz@to@use@whom%
    \let\tikz@to@or@edge@function=\tikz@do@edge%
    \tikz@to@or@edge}

% Syntax for to paths:
% to [options] (coordinate)
% to [options] node {node text} (coordinate)
\def\tikz@to o{%
  \tikz@to@use@last@coordinate%
  \let\tikz@to@or@edge@function=\tikz@do@to%
  \tikz@to@or@edge}
  
\def\tikz@to@or@edge{\pgfutil@ifnextchar[\tikz@@to@or@edge{\tikz@@to@or@edge[]}}%}
\def\tikz@@to@or@edge[#1]{%
  \def\tikz@@to@local@options{[#1]}%
  \let\tikz@collected@onpath=\pgfutil@empty%
  \tikz@@to@collect%
}
\def\tikz@@to@collect{%
  \pgfutil@ifnextchar(\tikz@@to@or@edge@coordinate
  {\pgfutil@ifnextchar n{\tikz@collect@label@onpath\tikz@@to@collect}%
    {\pgfutil@ifnextchar c{\tikz@collect@coordinate@onpath\tikz@@to@collect}
      {\PackageError{tikz}{( expected}{}%}
        \tikz@@to@or@edge@coordinate()}}}%
}

\def\tikz@@to@or@edge@coordinate(#1){%
  \def\tikztotarget{#1}%
  \tikz@to@or@edge@function%
}

\def\tikz@do@edge{%
  \setbox\tikz@figbox=\hbox\bgroup%
    \unhbox\tikz@figbox%
    \hbox\bgroup
      \bgroup%
        \pgfinterruptpath%
          \pgfscope%
            \let\tikz@transform=\pgfutil@empty%
            \let\tikz@options=\pgfutil@empty%
            \let\tikz@tonodes=\tikz@collected@onpath%
            \def\tikztonodes{{\pgfextra{\tikz@node@is@a@labeltrue}\tikz@tonodes}}%
            \let\tikz@collected@onpath=\pgfutil@empty%
            \tikz@options%
            \tikz@transform%            
            % Typeset node:
            \tikz@atbegin@to%
            \path[style=every edge]\tikz@@to@local@options(\tikztostart)\tikz@to@path;%
            \tikz@atend@to%
          \endpgfscope%
        \endpgfinterruptpath%
      \egroup
    \egroup%
  \egroup%
    \global\setbox\tikz@tempbox=\copy\tikz@figbox%
  \endgroup%
  \setbox\tikz@figbox=\box\tikz@tempbox%  
  \tikz@scan@next@command%  
}

\def\tikz@do@to{%
  \let\tikz@tonodes=\tikz@collected@onpath%
  \def\tikztonodes{{\pgfextra{\tikz@node@is@a@labeltrue}\tikz@tonodes}}%
  \let\tikz@collected@onpath=\pgfutil@empty%
  \tikz@scan@next@command%
  \pgfextra{\tikz@atbegin@to}%
  [style=every to]\tikz@@to@local@options\tikz@to@path%
  \pgfextra{\tikz@atend@to}%
}


\def\tikz@to@use@last@coordinate{%
  \iftikz@shapeborder%
    \edef\tikztostart{\tikz@shapeborder@name}%
  \else%
    \edef\tikztostart{\the\tikz@lastx,\the\tikz@lasty}%
  \fi%
}
\def\tikz@to@use@last@fig@name{%
  \edef\tikztostart{\tikz@to@last@fig@name}%
}



% Syntax for edge from parent: 
% edge from parent [options]
\def\tikz@edgetoparent from parent{\pgfutil@ifnextchar[\tikz@@edgetoparent{\tikz@@edgetoparent[]}}%}
\def\tikz@@edgetoparent[#1]{%
  \let\tikz@edge@to@parent@needed=\pgfutil@empty%
  \tikz@node@is@a@labeltrue%
  \tikz@scan@next@command [style=edge from parent,#1] \tikz@edge@to@parent@path%
}


% Syntax for bezier curves
% .. controls(point) and (point) .. (target)
% .. controls(point) .. (target) 
% .. (target) % currently not supported

\def\tikz@dot.{\tikz@@dot}%
\def\tikz@@dot{%
  \pgfutil@ifnextchar n%
  {\tikz@collect@label@onpath\tikz@@dot}%
  {\pgfutil@ifnextchar c{\tikz@curveto@double}{\tikz@curveto@auto}}}

\def\tikz@curveto@double co{%
  \pgfutil@ifnextchar o{\tikz@collect@coordinate@onpath\tikz@@dot co}
  {\tikz@cureveto@@double}}
\def\tikz@cureveto@@double ntrols#1{%
  \tikz@scan@one@point\tikz@curveA#1%
}
\def\tikz@curveA#1{%
  \edef\tikz@timer@start{\noexpand\pgfqpoint{\the\tikz@lastx}{\the\tikz@lasty}}%
  {%
    \tikz@make@last@position{#1}%
    \xdef\tikz@curve@first{\noexpand\pgfqpoint{\the\tikz@lastx}{\the\tikz@lasty}}%
  }%
  \pgfutil@ifnextchar a
  {\tikz@curveBand}%
  {\let\tikz@curve@second\tikz@curve@first\tikz@curveCdots}%
}
\def\tikz@curveBand and{%
  \tikz@scan@one@point\tikz@curveB%
}
\def\tikz@curveB#1{%
  \def\tikz@curve@second{#1}%
  \tikz@curveCdots}
\def\tikz@curveCdots{%
  \afterassignment\tikz@curveCdot\let\@next=}
\def\tikz@curveCdot.{%
  \ifx\@next.%
  \else%
    \PackageError{tikz}{Dot expected}{}%
  \fi%
  \tikz@updatecurrenttrue%
  \tikz@curveCcheck%
}
\def\tikz@curveCcheck{%
  \pgfutil@ifnextchar n%
  {\tikz@collect@label@onpath\tikz@curveCcheck}
  {\pgfutil@ifnextchar c{\tikz@collect@coordinate@onpath\tikz@curveCcheck}
    {\tikz@scan@one@point\tikz@curveC}}%
}
\def\tikz@curveC#1{%
  \tikz@make@last@position{#1}%
  \edef\tikz@curve@third{\noexpand\pgfqpoint{\the\tikz@lastx}{\the\tikz@lasty}}%
  {%
    \tikz@lastxsaved=\tikz@lastx%
    \tikz@lastysaved=\tikz@lasty%
    \tikz@make@last@position{\tikz@curve@second}%
    \xdef\tikz@curve@second{\noexpand\pgfqpoint{\the\tikz@lastx}{\the\tikz@lasty}}%
  }%
  %
  % Start recalculating things in case start and end are shapes.
  %
  % First, the start:
  \ifx\tikz@moveto@waiting\relax%
  \else%
    \pgf@process{\pgfpointshapeborder{\tikz@moveto@waiting}{\tikz@curve@first}}%
    \edef\tikz@timer@start{\noexpand\pgfqpoint{\the\pgf@x}{\the\pgf@y}}%
    \pgfpathmoveto{\pgfqpoint{\pgf@x}{\pgf@y}}%
  \fi%
  \let\tikz@timer@cont@one=\tikz@curve@first%
  \let\tikz@timer@cont@two=\tikz@curve@second%    
  % Second, the end:
  \iftikz@shapeborder%
    % ok, target is a shape. recalculate third
    {%
      \pgf@process{\pgfpointshapeborder{\tikz@shapeborder@name}{\tikz@curve@second}}%
      \tikz@make@last@position{\pgfqpoint{\pgf@x}{\pgf@y}}%
      \edef\tikz@curve@third{\noexpand\pgfqpoint{\the\tikz@lastx}{\the\tikz@lasty}}%
      \pgfpathcurveto{\tikz@curve@first}{\tikz@curve@second}{\tikz@curve@third}%
      \global\let\tikz@timer@end@temp=\tikz@curve@third% move out of group
    }%
    \let\tikz@timer@end=\tikz@timer@end@temp%
    \edef\tikz@moveto@waiting{\tikz@shapeborder@name}%    
  \else%
    \pgfpathcurveto{\tikz@curve@first}{\tikz@curve@second}{\tikz@curve@third}%
    \let\tikz@timer@end=\tikz@curve@third
    \let\tikz@moveto@waiting=\relax%
  \fi%
  \let\tikz@timer=\tikz@timer@curve%  
  \tikz@scan@next@command%
}


% Syntax for rectangles: 
% rectangle <corner point> 
\def\tikz@rect ectangle{%
  \tikz@flush@moveto%
  \edef\tikz@timer@start{\noexpand\pgfqpoint{\the\tikz@lastx}{\the\tikz@lasty}}%
  \tikz@@rect}%
\def\tikz@@rect{%
  \pgfutil@ifnextchar n
  {\tikz@collect@label@onpath\tikz@@rect}
  {\pgfutil@ifnextchar c{\tikz@collect@coordinate@onpath\tikz@@rect}%
    {
      \pgf@xa=\tikz@lastx\relax%
      \pgf@ya=\tikz@lasty\relax%
      \tikz@scan@one@point\tikz@rectB}}}
\def\tikz@rectB#1{%
  \tikz@make@last@position{#1}%
  \edef\tikz@timer@end{\noexpand\pgfqpoint{\the\tikz@lastx}{\the\tikz@lasty}}%
  \let\tikz@timer=\tikz@timer@line%  
  \pgfpathmoveto{\pgfqpoint{\pgf@xa}{\pgf@ya}}%
  \tikz@path@lineto{\pgfqpoint{\pgf@xa}{\tikz@lasty}}%
  \tikz@path@lineto{\pgfqpoint{\tikz@lastx}{\tikz@lasty}}%
  \tikz@path@lineto{\pgfqpoint{\tikz@lastx}{\pgf@ya}}%
  \iftikz@snaked% 
    \tikz@path@lineto{\pgfqpoint{\pgf@xa}{\pgf@ya}}%
  \fi%
  \pgfpathclose%
  \pgfpathmoveto{\pgfqpoint{\tikz@lastx}{\tikz@lasty}}%
  \def\pgfstrokehook{}%
  \tikz@scan@next@command%
}



% Syntax for grids: 
% grid <corner point> 
\def\tikz@grid rid{%
  \tikz@flush@moveto%
  \pgf@xa=\tikz@lastx\relax%
  \pgf@ya=\tikz@lasty\relax%
  \pgfutil@ifnextchar[{\tikz@gridA}{\tikz@gridA[]}}%}
\def\tikz@gridA[#1]{%
  \def\tikz@grid@options{#1}%
  \tikz@scan@one@point\tikz@gridB}%
\def\tikz@gridB#1{%
  \tikz@make@last@position{#1}%
  {%
    \expandafter\tikzset\expandafter{\tikz@grid@options}
    \tikz@checkunit{\tikz@grid@x}%
    \iftikz@isdimension%
      \pgf@process{\pgfpoint{\tikz@grid@x}{0pt}}%
    \else%
      \pgf@process{\pgfpointxy{\tikz@grid@x}{0}}%
    \fi%
    \pgf@xb=\pgf@x%
    \pgf@yb=\pgf@y%
    \tikz@checkunit{\tikz@grid@y}%
    \iftikz@isdimension%
      \pgf@process{\pgfpoint{0pt}{\tikz@grid@y}}%
    \else%
      \pgf@process{\pgfpointxy{0}{\tikz@grid@y}}%
    \fi%
    \advance\pgf@xb by\pgf@x%
    \advance\pgf@yb by\pgf@y%
    \pgfpathgrid[stepx=\pgf@xb,stepy=\pgf@yb]%
      {\pgfqpoint{\pgf@xa}{\pgf@ya}}{\pgfqpoint{\tikz@lastx}{\tikz@lasty}}%
  }
  \tikz@scan@next@command%
}



% Syntax for plot: 
% plot [local options] ...    % starts with a moveto
% -- plot [local options] ... % starts with a lineto
\def\tikz@plot lot{%
  \tikz@flush@moveto%
  \pgfutil@ifnextchar[{\tikz@@plot}{\tikz@@plot[]}}%}
\def\tikz@@plot[#1]{%
  \begingroup%
    \let\tikz@options=\pgfutil@empty%
    \tikzset{every plot/.try}%
    \tikzset{#1}%
    \pgfutil@ifnextchar f{\tikz@plot@f}%
    {\pgfutil@ifnextchar c{\tikz@plot@scan@points}%
      {\pgfutil@ifnextchar ({\tikz@plot@expression}{%
      \PackageError{tikz}{Cannot parse this plotting data}{}%
       \endgroup}}}}
\def\tikz@plot@f f{\pgfutil@ifnextchar i{\tikz@plot@file}{\tikz@plot@function}}

\def\tikz@plot@file ile#1{\def\tikz@plot@data{\pgfplotxyfile{#1}}\tikz@@@plot}%
\def\tikz@plot@scan@points coordinates#1{%
  \pgfplothandlerrecord\tikz@plot@data%
  \pgfplotstreamstart%
  \pgfutil@ifnextchar\pgf@stop{\pgfplotstreamend\expandafter\tikz@@@plot\pgfutil@gobble}
  {\tikz@scan@one@point\tikz@plot@next@point}%
  #1\pgf@stop%
}
\def\tikz@plot@next@point#1{%
  \pgfplotstreampoint{#1}%
  \pgfutil@ifnextchar\pgf@stop{\pgfplotstreamend\expandafter\tikz@@@plot\pgfutil@gobble}%
  {\tikz@scan@one@point\tikz@plot@next@point}%
}  
\def\tikz@plot@function unction#1{%
  \def\tikz@plot@filename{\tikz@plot@prefix\tikz@plot@id}%  
  \iftikz@plot@raw@gnuplot%
    \def\tikz@plot@data{\pgfplotgnuplot[\tikz@plot@filename]{#1}}%
  \else%
    \iftikz@plot@parametric%   
      \def\tikz@plot@data{\pgfplotgnuplot[\tikz@plot@filename]{%
          set samples \tikz@plot@samples;
          set parametric;
          plot [t=\tikz@plot@domain] #1}}%
    \else%
      \def\tikz@plot@data{\pgfplotgnuplot[\tikz@plot@filename]{%
          set samples \tikz@plot@samples;
          plot [x=\tikz@plot@domain] #1}}%
    \fi%
  \fi%
  \tikz@@@plot%
}

\def\tikz@plot@no@resample{%
  \pgfutil@IfFileExists{\tikz@plot@filename.table}%
  {\def\tikz@plot@data{\pgfplotxyfile{\tikz@plot@filename.table}}}%
  {}%
}

\def\tikz@plot@expression(#1){%
  \edef\tikz@plot@data{\noexpand\pgfplotfunction{\expandafter\noexpand\tikz@plot@var}{\tikz@plot@samplesat}}%
  \expandafter\def\expandafter\tikz@plot@data\expandafter{\tikz@plot@data{\tikz@scan@one@point\pgfutil@firstofone(#1)}}%
  \tikz@@@plot%
}

\def\tikz@@@plot{%
    \def\pgfplotlastpoint{\pgfpointorigin}%
    \tikz@plot@handler%
    \tikz@plot@data%
    \global\let\tikz@@@temp=\pgfplotlastpoint%
    \ifx\tikz@plot@mark\pgfutil@empty%
    \else%
      % Marks are drawn after the path.
      \setbox\tikz@figbox=\hbox{%
        \unhbox\tikz@figbox%
        \hbox{{%
          \pgfinterruptpath%
            \pgfscope%
              \let\tikz@options=\pgfutil@empty%
              \let\tikz@transform=\pgfutil@empty%
              \expandafter\tikzset\expandafter{\tikz@plot@mark@options}%
              \tikz@options%
              \ifx\tikz@mark@list\pgfutil@empty%
                \pgfplothandlermark{\tikz@transform\pgfuseplotmark{\tikz@plot@mark}}%
              \else
                \pgfplothandlermarklisted{\tikz@transform\pgfuseplotmark{\tikz@plot@mark}}{\tikz@mark@list}%
              \fi
              \tikz@plot@data%
            \endpgfscope
          \endpgfinterruptpath%
        }}%
      }%
    \fi%
    \global\setbox\tikz@tempbox=\copy\tikz@figbox%
  \endgroup%
  \setbox\tikz@figbox=\box\tikz@tempbox%  
  \tikz@make@last@position{\tikz@@@temp}%  
  \tikz@scan@next@command%
}


\pgfdeclareplotmark{ball}
{%
  \def\tikz@shading{ball}%
  \shade (0,0) circle (\pgfplotmarksize);%
}




% Syntax for cosine curves:
% cos <end of quarter-period>
\def\tikz@cosine s{\tikz@scan@one@point\tikz@@cosine}
\def\tikz@@cosine#1{%
  \tikz@flush@moveto%
  \pgf@process{#1}%
  \pgf@xc=\pgf@x%
  \pgf@yc=\pgf@y%
  \advance\pgf@xc by-\tikz@lastx%
  \advance\pgf@yc by-\tikz@lasty%
  \advance\tikz@lastx by\pgf@xc%
  \advance\tikz@lasty by\pgf@yc%
  \tikz@lastxsaved=\tikz@lastx%
  \tikz@lastysaved=\tikz@lasty%
  \tikz@updatecurrenttrue%
  \pgfpathcosine{\pgfqpoint{\pgf@xc}{\pgf@yc}}%
  \tikz@scan@next@command%
}

% Syntax for sine curves:
% sin <end of quarter-period>
\def\tikz@sine in{\tikz@scan@one@point\tikz@@sine}
\def\tikz@@sine#1{%
  \tikz@flush@moveto%
  \pgf@process{#1}%
  \pgf@xc=\pgf@x%
  \pgf@yc=\pgf@y%
  \advance\pgf@xc by-\tikz@lastx%
  \advance\pgf@yc by-\tikz@lasty%
  \advance\tikz@lastx by\pgf@xc%
  \advance\tikz@lasty by\pgf@yc%
  \tikz@lastxsaved=\tikz@lastx%
  \tikz@lastysaved=\tikz@lasty%
  \tikz@updatecurrenttrue%
  \pgfpathsine{\pgfqpoint{\pgf@xc}{\pgf@yc}}%
  \tikz@scan@next@command%
}

% Syntax for parabolas: 
% parabola[options] bend <coordinate> <coordinate>
\def\tikz@parabola arabola

\def\tikz@parabola@options[#1]{%
  \def\tikz@parabola@option{#1}%
  \pgfutil@ifnextchar b{\tikz@parabola@scan@bend}{\tikz@scan@one@point\tikz@parabola@semifinal}}
\def\tikz@parabola@scan@bend bend{\tikz@scan@one@point\tikz@parabola@scan@bendB}
\def\tikz@parabola@scan@bendB#1{%
  \def\tikz@parabola@bend{#1}%
  \tikz@scan@one@point\tikz@parabola@semifinal%
}
\def\tikz@parabola@semifinal#1{%
  \tikz@flush@moveto%
  % Save original start:
  \pgf@xb=\tikz@lastx%
  \pgf@yb=\tikz@lasty%
  \tikz@make@last@position{#1}%
  \pgf@xc=\tikz@lastx%
  \pgf@yc=\tikz@lasty%
  \begingroup% now calculate bend:
    \expandafter\tikzset\expandafter{\tikz@parabola@option}%
    \tikz@lastxsaved=\tikz@parabola@bend@factor\tikz@lastx%
    \tikz@lastysaved=\tikz@parabola@bend@factor\tikz@lasty%
    \advance\tikz@lastxsaved by\pgf@xb%
    \advance\tikz@lastysaved by\pgf@yb%
    \advance\tikz@lastxsaved by-\tikz@parabola@bend@factor\pgf@xb%
    \advance\tikz@lastysaved by-\tikz@parabola@bend@factor\pgf@yb%
    \expandafter\tikz@make@last@position\expandafter{\tikz@parabola@bend}%
    % Calculate delta from bend
    \advance\pgf@xc by-\tikz@lastx%
    \advance\pgf@yc by-\tikz@lasty%
    % Ok, now calculate delta to bend
    \advance\tikz@lastx by-\pgf@xb%
    \advance\tikz@lasty by-\pgf@yb%
    \xdef\tikz@parabola@b{{\noexpand\pgfqpoint{\the\tikz@lastx}{\the\tikz@lasty}}{\noexpand\pgfqpoint{\the\pgf@xc}{\the\pgf@yc}}}%
  \endgroup%
  \expandafter\pgfpathparabola\tikz@parabola@b%
  \tikz@scan@next@command%
}


% Syntax for circles:
% circle (radius)
%
% Syntax for ellipses:
% ellipse (x-radius and y-radius)
%
% radii can be dimensionless, then they are in the xy-system
\def\tikz@circle ircle{\tikz@flush@moveto\tikz@@circle}
\def\tikz@ellipse llipse{\tikz@flush@moveto\tikz@@circle}
\def\tikz@@circle{%
  \pgfutil@ifnextchar(\tikz@@@circle{%)
    \advance\tikz@expandcount by -1%
    \ifnum\tikz@expandcount<0\relax%
      \let\@next=\tikz@@circle@scangiveup%
    \else%
      \let\@next=\tikz@@circle@scanexpand%
    \fi%
    \@next%
  }%
}
\def\tikz@@circle@scanexpand{\expandafter\tikz@@circle}
\def\tikz@@circle@scangiveup#1{\PackageError{tikz}{Cannot parse this radius}{}#1{\tikz@scan@next@command}}
\def\tikz@@@circle(#1){%
  \pgfutil@in@{ and }{#1}%
  \ifpgfutil@in@%
    \tikz@@ellipseB(#1)%
  \else%
    \tikz@@ellipseB(#1 and #1)%
  \fi%
  \tikz@scan@next@command%
}
\def\tikz@@ellipseB(#1 and #2){%
  \tikz@checkunit{#1}%
  \iftikz@isdimension%
    \pgfpathellipse{\tikz@last@position}{\pgfpoint{#1}{0pt}}{\pgfpoint{0pt}{#2}}%
  \else%
    \pgfpathellipse{\tikz@last@position}{\pgfpointxy{#1}{0}}{\pgfpointxy{0}{#2}}%
  \fi%
}

% Syntax 1 for arcs:
% arc (start angle:end angle:radius)
%
% Syntax 2 for arcs:
% arc (start angle:end angle:x-radius and y-radius)
%
% radius can be dimensionless, then the arc is in the xy-coordinate system
\def\tikz@arcA rc{%
  \tikz@flush@moveto%
  \pgfutil@ifnextchar({\tikz@@arcto}{\expandafter\tikz@arcA\expandafter r\expandafter c}}

\def\tikz@@arcto(#1){%
  \edef\tikz@temp{(#1)}%
   \expandafter\tikz@@@arcto@check@slashand\tikz@temp%
}

\def\tikz@@@arcto@check@slashand(#1:#2:#3){%
  \pgfutil@in@{ and }{#3}%
  \ifpgfutil@in@% 
    \tikz@parse@arc@and(#1:#2:#3)%
  \else%
    \tikz@parse@arc@and(#1:#2:#3 and #3)%
  \fi%
}

\def\tikz@parse@arc@and(#1:#2:#3 and #4){%
  \tikz@checkunit{#3}%
  \iftikz@isdimension%
    \tikz@@@arcfinal{\pgfpatharc{#1}{#2}{#3 and #4}}
    {\pgfpointpolar{#1}{#3 and #4}}
    {\pgfpointpolar{#2}{#3 and #4}}%
  \else%
    \tikz@@@arcfinal{\pgfpatharcaxes{#1}{#2}{\pgfpointxy{#3}{0}}{\pgfpointxy{0}{#4}}}
    {\pgfpointpolarxy{#1}{#3 and #4}}{\pgfpointpolarxy{#2}{#3 and #4}}%
  \fi%
}

\def\tikz@@@arcfinal#1#2#3{%
  #1%
  \pgf@process{#2}%
  \advance\tikz@lastx by-\pgf@x%
  \advance\tikz@lasty by-\pgf@y%
  \pgf@process{#3}%
  \advance\tikz@lastx by\pgf@x%
  \advance\tikz@lasty by\pgf@y%
  \tikz@lastxsaved=\tikz@lastx%
  \tikz@lastysaved=\tikz@lasty%
  \tikz@scan@next@command%
}


% Syntax for coordinates:
% coordinate[options] (coordinate name) at (point)
% where ``at (point)'' is optional
\def\tikz@coordinate ordinate{%
  \pgfutil@ifnextchar[{\tikz@@coordinate@opt}{\tikz@@coordinate@opt[]}}
\def\tikz@@coordinate@opt[#1]
\def\tikz@@coordinate[#1](#2){%
  \pgfutil@ifnextchar a{\tikz@@coordinate@at[#1](#2)}
  {\tikz@fig ode[shape=coordinate,#1](#2){}}}
\def\tikz@@coordinate@at[#1](#2)at#3(#4){%
  \tikz@fig ode[shape=coordinate,#1](#2)at(#4){}}
  


% Syntax for nodes:
% node[options] (node name) {label text}
%
% all of [options], (node name) and {label text} are optional. There
% can be multiple options before the label text as in
% node[draw] (a) [rotate=10] {text}
%
% A label text always ``ends'' the node.
\def\tikz@fig ode{%
  \edef\tikz@save@line@width{\the\pgflinewidth}%
  \begingroup%
  \let\tikz@fig@name=\pgfutil@empty%
    \begingroup%
      \tikz@is@matrixfalse%
      \let\nodepart=\tikz@nodepart%
      \let\tikz@options=\pgfutil@empty%
      \let\tikz@after@node=\pgfutil@empty%
      \let\tikz@afternodepathoptions=\pgfutil@empty%
      \let\tikz@transform=\pgfutil@empty%
      \let\tikz@mode=\pgfutil@empty%
      \def\tikz@node@at{\pgfqpoint{\the\tikz@lastx}{\the\tikz@lasty}}%
      \iftikz@node@is@a@label%
      \else%
        \let\tikz@time=\pgfutil@empty%
      \fi%
      \tikzset{every node/.try}%
      \tikz@@scan@fig}%
\def\tikz@@scan@fig{%
  \pgfutil@ifnextchar a{\tikz@fig@scan@at}
  {\pgfutil@ifnextchar({\tikz@fig@scan@name}
    {\pgfutil@ifnextchar[{\tikz@fig@scan@options}%
      {\pgfutil@ifnextchar\bgroup{\tikz@fig@main}%
      {\PackageError{tikz}{A node must have a (possibly empty) label text}{}%
       \tikz@fig@main{}}}}}}%}}
\def\tikz@fig@scan@at at{%
  \tikz@scan@one@point\tikz@@fig@scan@at}
\def\tikz@@fig@scan@at#1{%
  \def\tikz@node@at{#1}\tikz@@scan@fig}%
\def\tikz@fig@scan@name(#1){\edef\tikz@fig@name{#1}\tikz@@scan@fig}%
\def\tikz@fig@scan@options[#1]{\tikzset{#1}\def\test{#1}\tikz@@scan@fig}%
\def\tikz@fig@main{\afterassignment\tikz@@fig@main\let\next=}
\def\tikz@@fig@main{%
    \pgfutil@ifundefined{pgf@sh@s@\tikz@shape}%
    {\PackageError{tikz}%
      {Unknown shape ``\tikz@shape.'' Using ``rectangle'' instead}{}%
      \def\tikz@shape{rectangle}}%
    {}%
    \tikzset{every \tikz@shape\space node/.try}%
    \iftikz@is@matrix%
      \let\tikz@next=\tikz@do@matrix%
    \else%
      \let\tikz@next=\tikz@do@fig%
    \fi%
    \tikz@next%  
}
\def\tikz@do@fig{%  
    \setbox\pgfnodeparttextbox=\hbox%
      \bgroup%
        \tikzset{every text node part/.try}%
        \ifx\tikz@textopacity\pgfutil@empty%
        \else%
          \pgfsetfillopacity{\tikz@textopacity}%
          \pgfsetstrokeopacity{\tikz@textopacity}%
        \fi%
        \pgfinterruptpicture%
          \tikz@textfont%  
          \ifx\tikz@text@width\pgfutil@empty%
          \else%
            \begingroup%
              \pgfutil@minipage[t]{\tikz@text@width}%
                \tikz@text@action%
          \fi%
          \tikz@atbegin@node%
          \bgroup%
            \aftergroup\unskip%
            \ifx\tikz@textcolor\pgfutil@empty%
            \else%
              \pgfutil@colorlet{.}{\tikz@textcolor}%
            \fi%
            \pgfsetcolor{.}%
            \setbox\tikz@figbox=\box\voidb@x%
            \tikz@uninstallcommands%
            \aftergroup\tikz@fig@collectresetcolor%
            \ignorespaces%
}
\def\tikz@fig@collectresetcolor{%
  \pgfutil@ifnextchar\reset@color%
  {\reset@color\afterassignment\tikz@fig@collectresetcolor\let\tikz@temp=}%
  {\tikz@fig@boxdone}%
}
\def\tikz@fig@boxdone{%
            \tikz@atend@node%
          \ifx\tikz@text@width\pgfutil@empty%
          \else%
              \pgfutil@endminipage%
            \endgroup%
          \fi%
        \endpgfinterruptpicture%
      \egroup%
    \pgfutil@ifnextchar c{\tikz@fig@mustbenamed\tikz@fig@continue}%
    {\pgfutil@ifnextchar[{\tikz@fig@mustbenamed\tikz@fig@continue}%
      {\pgfutil@ifnextchar t{\tikz@fig@mustbenamed\tikz@fig@continue}
        {\pgfutil@ifnextchar e{\tikz@fig@mustbenamed\tikz@fig@continue}
          {\ifx\tikz@after@node\pgfutil@empty\expandafter\tikz@fig@continue\else\expandafter\tikz@fig@mustbenamed\expandafter\tikz@fig@continue\fi}}}}}%}

\def\tikz@do@matrix{%
    \tikzset{every matrix/.try}%
    \tikz@node@transformations%
    \tikz@fig@mustbenamed%
    \setbox\tikz@figbox=\hbox\bgroup%
      \setbox\pgfutil@tempboxa=\copy\tikz@figbox%
      \unhbox\pgfutil@tempboxa%
      \hbox\bgroup\bgroup%
          \pgfinterruptpath%
            \pgfscope%
              \tikz@options%
              \setbox\tikz@figbox=\box\voidb@x%
              \let\tikzmatrixname=\tikz@fig@name%
              \edef\tikz@m@anchor{\ifx\tikz@matrix@anchor\pgfutil@empty\tikz@anchor\else\tikz@matrix@anchor\fi}%
              \expandafter\pgfutil@in@\expandafter{\expandafter.\expandafter}\expandafter{\tikz@m@anchor}%
              \ifpgfutil@in@%
                \expandafter\tikz@matrix@split\tikz@m@anchor\relax%
              \else%
                \def\tikz@matrix@shift{\pgfpointorigin}%  
              \fi%
              \let\tikz@transform=\relax%
              \pgfmatrix%
              {\tikz@shape}%
              {\tikz@m@anchor}%
              {\tikz@fig@name}%
              {%
                \pgfutil@tempdima=\pgflinewidth%
                {\begingroup\tikz@finish}%
                \global\pgflinewidth=\pgfutil@tempdima%
              }%
              {\tikz@matrix@shift}%
              {%
                \tikz@matrix@make@active@ampersand%
                \def\pgfmatrixbegincode{%
                  \pgfsys@beginscope%
                  \tikz@common@matrix@code%
                  \tikz@atbegin@cell%
                }%
                \def\tikz@common@matrix@code{%
                  \let\tikz@options=\pgfutil@empty%
                  \let\tikz@mode=\pgfutil@empty%
                  \tikzset{every cell/.try={\the\pgfmatrixcurrentrow}{\the\pgfmatrixcurrentcolumn}}%
                  \tikzset{column \the\pgfmatrixcurrentcolumn/.try}%
                  \ifodd\pgfmatrixcurrentcolumn%
                    \tikzset{every odd column/.try}%
                  \else%
                    \tikzset{every even column/.try}%
                  \fi%
                  \tikzset{row \the\pgfmatrixcurrentrow/.try}%
                  \ifodd\pgfmatrixcurrentrow%
                    \tikzset{every odd row/.try}%
                  \else%
                    \tikzset{every even row/.try}%
                  \fi%
                  \tikzset{row \the\pgfmatrixcurrentrow\space column \the\pgfmatrixcurrentcolumn/.try}%
                  \tikz@options%
                }%
                \def\pgfmatrixendcode{%
                  \tikz@atend@cell%
                  \pgfsys@endscope%
                }%
                \def\pgfmatrixemptycode{%
                  \pgfsys@beginscope%
                  \tikz@common@matrix@code%
                  \tikz@at@emptycell%
                  \pgfsys@endscope%
                }%
                \aftergroup\tikz@do@matrix@cont}%
              \bgroup%
}
\def\tikz@do@matrix@cont{%            
            \endpgfscope
          \endpgfinterruptpath%
      \egroup\egroup%
    \egroup%
    %
    \tikz@node@finish%
}

{%
  \catcode`\&=13
  \gdef\tikz@matrix@make@active@ampersand{%
    \ifx\tikz@ampersand@replacement\pgfutil@empty%
      \catcode`\&=13%
      \let&=\pgfmatrixnextcell%
    \else%
      \expandafter\let\tikz@ampersand@replacement=\pgfmatrixnextcell%
    \fi%
  }%
}%


\def\tikz@matrix@split#1.#2\relax{%
  \def\tikz@m@anchor{text}%
  \def\tikz@matrix@shift{\pgfpointanchor{#1}{#2}}%
}
  
\def\tikz@fig@continue{%
    \ifx\tikz@text@width\pgfutil@empty%
    \else%
      \pgfmathsetlength{\pgf@x}{\tikz@text@width}%
      \wd\pgfnodeparttextbox=\pgf@x%
    \fi%
    \ifx\tikz@text@height\pgfutil@empty%
    \else%
      \pgfmathsetlength{\pgf@x}{\tikz@text@height}%
      \ht\pgfnodeparttextbox=\pgf@x%
    \fi%
    \ifx\tikz@text@depth\pgfutil@empty%
    \else%
      \pgfmathsetlength{\pgf@x}{\tikz@text@depth}%
      \dp\pgfnodeparttextbox=\pgf@x%
    \fi%
    %
    % Node transformation
    %
    \tikz@node@transformations
    %
    \setbox\tikz@figbox=\hbox{%
      \setbox\pgfutil@tempboxa=\copy\tikz@figbox%
      \unhbox\pgfutil@tempboxa%
      \hbox{{%
          \pgfinterruptpath%
            \pgfscope%
              \tikz@options%
              \setbox\tikz@figbox=\box\voidb@x%
              \pgfmultipartnode{\tikz@shape}{\tikz@anchor}{\tikz@fig@name}{%
                \pgfutil@tempdima=\pgflinewidth%
                {\begingroup\tikz@finish}%
                \global\pgflinewidth=\pgfutil@tempdima%
              }%
            \endpgfscope
          \endpgfinterruptpath%
      }}%
    }%
    %
    \tikz@node@finish%
}


\def\tikz@fig@mustbenamed{%
  \ifx\tikz@fig@name\pgfutil@empty%
    % Assign a dummy name
    \global\advance\tikz@fig@count by1\relax
    \edef\tikz@fig@name{tikz@f@\the\tikz@fig@count}%
  \fi%
}

\def\tikz@node@transformations{
  % 
  % Possibly, we are ``online''
  % 
  \ifx\tikz@time\pgfutil@empty%
    \pgftransformshift{\tikz@node@at}%
    \iftikz@fullytransformed%
    \else%
      \pgftransformresetnontranslations%
    \fi%
  \else%
    \tikz@do@auto@anchor%
    \tikz@timer%
  \fi%
  % Invoke local transformations
  \tikz@transform%
}

\def\tikz@node@finish{%  
    \global\let\tikz@last@fig@name=\tikz@fig@name%
    \global\let\tikz@after@node@smuggle=\tikz@after@node%
    \global\let\tikz@afternodepathoptions@smuggle=\tikz@afternodepathoptions%
    % shift box outside group
    \global\setbox\tikz@tempbox=\copy\tikz@figbox%
  \endgroup\endgroup%
  \setbox\tikz@figbox=\box\tikz@tempbox%
  \pgflinewidth=\tikz@save@line@width%
  \let\tikz@to@last@fig@name=\tikz@last@fig@name%
  \let\tikz@to@use@whom=\tikz@to@use@last@fig@name%
  \let\tikzlastnode=\tikz@last@fig@name%
  \ifx\tikz@after@node@smuggle\pgfutil@empty%
  \else%
    \tikz@scan@next@command{\pgfextra{\tikz@afternodepathoptions@smuggle}\tikz@after@node@smuggle}\pgf@stop%
  \fi%
  \tikz@scan@next@command%
}
\let\tikz@fig@continue@orig=\tikz@fig@continue



% Syntax for parts of  nodes:
% node ... {... \nodepart{name} ... \nodepart{name} ...}

\def\tikz@nodepart#1{%
  \tikz@atend@node%
  \unskip%
  \gdef\tikz@nodepart@name{#1}%
  \global\let\tikz@fig@continue=\tikz@nodepart@continue%
  \pgfutil@ifnextchar x{\egroup\relax}{\egroup\relax}% gobble spaces
}
\def\tikz@nodepart@continue{%
  \global\let\tikz@fig@continue=\tikz@fig@continue@orig%
  % Now start new box:
   \expandafter\setbox\csname pgfnodepart\tikz@nodepart@name box\endcsname=\hbox%
      \bgroup%
        \tikzset{every \tikz@nodepart@name\space node part/.try}%
        \pgfinterruptpicture%
          \tikz@textfont%  
          \ifx\tikz@text@width\pgfutil@empty%
          \else%
            \begingroup%
              \pgfutil@minipage[t]{\tikz@text@width}%
                \tikz@text@action%
          \fi%
          \bgroup%
            \aftergroup\unskip%
            \ifx\tikz@textcolor\pgfutil@empty%
            \else%
              \pgfutil@colorlet{.}{\tikz@textcolor}%
            \fi%
            \pgfsetcolor{.}%
            \setbox\tikz@figbox=\box\voidb@x%
            \tikz@uninstallcommands%
            \tikz@atbegin@node%
            \aftergroup\tikz@fig@collectresetcolor%
            \ignorespaces%
}


% Auto placement

\def\tikz@auto@pre{%
  \begingroup
    \pgfresetnontranslationattimefalse
    \pgfslopedattimetrue%
    \pgfallowupsidedownattimetrue%
    \tikz@timer%
    \pgf@x=\pgf@pt@aa pt% 
    \pgf@y=\pgf@pt@ab pt%
    \pgfpointnormalised{}%
}

\def\tikz@auto@post{%
    \global\let\tikz@anchor@smuggle=\tikz@anchor%
  \endgroup%
  \let\tikz@anchor=\tikz@anchor@smuggle%
}

\def\tikz@auto@anchor{%
    \ifdim\pgf@x>0.05pt%
      \ifdim\pgf@y>0.05pt%
        \def\tikz@anchor{south east}%
      \else\ifdim\pgf@y<-0.05pt%
        \def\tikz@anchor{south west}%
      \else
        \def\tikz@anchor{south}%
      \fi\fi%
    \else\ifdim\pgf@x<-0.05pt%
      \ifdim\pgf@y>0.05pt%
        \def\tikz@anchor{north east}%
      \else\ifdim\pgf@y<-0.05pt%
        \def\tikz@anchor{north west}%
      \else
        \def\tikz@anchor{north}%
      \fi\fi%
    \else%
      \ifdim\pgf@y>0pt%
        \def\tikz@anchor{east}%
      \else%
        \def\tikz@anchor{west}%
      \fi%
    \fi\fi%
}

\def\tikz@auto@anchor@prime{%
    \ifdim\pgf@x>0.05pt%
      \ifdim\pgf@y>0.05pt%
        \def\tikz@anchor{north west}%
      \else\ifdim\pgf@y<-0.05pt%
        \def\tikz@anchor{north east}%
      \else
        \def\tikz@anchor{north}%
      \fi\fi%
    \else\ifdim\pgf@x<-0.05pt%
      \ifdim\pgf@y>0.05pt%
        \def\tikz@anchor{south west}%
      \else\ifdim\pgf@y<-0.05pt%
        \def\tikz@anchor{south east}%
      \else
        \def\tikz@anchor{south}%
      \fi\fi%
    \else%
      \ifdim\pgf@y>0pt%
        \def\tikz@anchor{west}%
      \else%
        \def\tikz@anchor{east}%
      \fi%
    \fi\fi%
}




% Syntax for trees:
% node {...} child [options] {...} child [options] {...} ...
% node {...} child [options] foreach \var in {list} [options] {...} ...

\def\tikz@children{%
  % Start collecting the children:
  \let\tikz@children@list=\pgfutil@empty%
  \tikznumberofchildren=0\relax%
  \tikz@collect@children c}

\def\tikz@collect@children{\pgfutil@ifnextchar c{\tikz@collect@children@cchar}{\tikz@children@collected}}
\def\tikz@collect@children@cchar c{\pgfutil@ifnextchar h{\tikz@collect@child}{\tikz@children@collected c}}
\def\tikz@collect@child hild{\pgfutil@ifnextchar[{\tikz@collect@childA}{\tikz@collect@childA[]}}%}
\def\tikz@collect@childA[#1]{\pgfutil@ifnextchar f{\tikz@collect@children@foreach[#1]}{\tikz@collect@childB[#1]}}
\def\tikz@collect@childB[#1]{%
  \advance\tikznumberofchildren by1\relax
  \expandafter\def\expandafter\tikz@children@list\expandafter{\tikz@children@list \tikz@childnode[#1]}%
  \pgfutil@ifnextchar\bgroup{\tikz@collect@child@code}{\tikz@collect@child@code{}}}
\def\tikz@collect@child@code#1{%
  \expandafter\def\expandafter\tikz@children@list\expandafter{\tikz@children@list{#1}}%
  \tikz@collect@children%
}
\def\tikz@collect@children@foreach[#1]foreach#2in#3{%
  \pgfutil@ifnextchar\bgroup{\tikz@collect@children@foreachA{#1}{#2}{#3}}{\tikz@collect@children@foreachA{#1}{#2}{#3}{}}}
\def\tikz@collect@children@foreachA#1#2#3#4{%
  \expandafter\def\expandafter\tikz@children@list\expandafter
    {\tikz@children@list\tikz@childrennodes[#1]{#2}{#3}{#4}}%
  \c@pgf@counta=\tikznumberofchildren%
  \foreach#2in{#3}%
  {%
    \global\advance\c@pgf@counta by1\relax%
  }%
  \tikznumberofchildren=\c@pgf@counta%
  \tikz@collect@children%
}
\long\def\tikz@children@collected{%
  \begingroup%
    \advance\tikztreelevel by 1\relax%
    \let\tikz@options=\pgfutil@empty%
    \let\tikz@transform=\pgfutil@empty%
    \tikzset{level/.try=\the\tikztreelevel,level \the\tikztreelevel/.try}%
    \tikz@transform%            
    \let\tikzparentnode=\tikz@last@fig@name%
    % Transform to center of node
    \pgftransformshift{\pgfpointanchor{\tikzparentnode}{\tikz@growth@anchor}}%
    \tikznumberofcurrentchild=0\relax%
    \tikz@children@list%
    \global\setbox\tikz@tempbox=\copy\tikz@figbox%
  \endgroup%
  \setbox\tikz@figbox=\box\tikz@tempbox%  
  \tikz@scan@next@command%
}


% Syntax for children:
%
% child [all children options] foreach \var in {values} [child options] {...}
\def\tikz@childrennodes[#1]#2#3#4{%
  \c@pgf@counta=\tikznumberofcurrentchild\relax%
  \setbox\tikz@tempbox=\box\tikz@figbox%
  \foreach#2in{#3}{%
    \tikznumberofcurrentchild=\c@pgf@counta\relax%
    \setbox\tikz@figbox=\box\tikz@tempbox%
    \tikz@childnode[#1]{#4}%
    % we must now make the current child number and the figbox survive
    % the group
    \global\c@pgf@counta=\tikznumberofcurrentchild\relax%
    \global\setbox\tikz@tempbox=\box\tikz@figbox%
  }%
  \tikznumberofcurrentchild=\c@pgf@counta\relax%
  \setbox\tikz@figbox=\box\tikz@tempbox%
}


% Syntax for child:
%
% child
%
% child[options]
%
% child[options] {node (name) {child node text} ...
%   edge from parent[options] node {label text} node {label text}}

\def\tikz@childnode[#1]#2{%
  \advance\tikznumberofcurrentchild by1\relax%
  \setbox\tikz@figbox=\hbox\bgroup%
    \unhbox\tikz@figbox%
    \hbox\bgroup\bgroup%
        \pgfinterruptpath%
          \pgfscope%
            \let\tikz@transform=\pgfutil@empty%
            \tikzset{every child/.try,#1}%
            \tikz@options%
            \tikz@transform%            
            \tikz@grow%
            % Typeset node:
            \edef\tikz@parent@node@name{[name=\tikzparentnode-\the\tikznumberofcurrentchild,style=every child node]}%
            \def\tikz@child@node@text{[shape=coordinate]{}}
            \tikz@parse@child@node#2\pgf@stop%
            \expandafter\expandafter\expandafter\node
            \expandafter\tikz@parent@node@name
              \tikz@child@node@text
              \pgfextra{\global\let\tikz@childnode@name=\tikz@last@fig@name};%
            \let\tikzchildnode=\tikz@childnode@name%
            {%
              \def\tikz@edge@to@parent@needed{edge from parent}
              \ifx\tikz@child@node@rest\pgfutil@empty%
                \path edge from parent;%
              \else%
                \path (0,0) \tikz@child@node@rest \tikz@edge@to@parent@needed;%
              \fi%
            }%
        \endpgfscope%
      \endpgfinterruptpath%
    \egroup\egroup%
  \egroup%
}

\def\tikz@parse@child@node{%
  \pgfutil@ifnextchar n{\tikz@parse@child@node@n}%
  {\pgfutil@ifnextchar c{\tikz@parse@child@node@c}%
    {\tikz@parse@child@node@rest}}}
\def\tikz@parse@child@node@rest#1\pgf@stop{\def\tikz@child@node@rest{#1}}
\def\tikz@parse@child@node@c c{\pgfutil@ifnextchar o{\tikz@parse@child@node@co}{\tikz@parse@child@node@rest c}}
\def\tikz@parse@child@node@co o{\pgfutil@ifnextchar o{\tikz@parse@child@node@coordinate}{\tikz@parse@child@node@rest co}}
\def\tikz@parse@child@node@coordinate ordinate{%
  \pgfutil@ifnextchar ({\tikz@@parse@child@node@coordinate}{%
    \def\tikz@child@node@text{[shape=coordinate]{}}%
    \tikz@parse@child@node@rest}}%}
\def\tikz@@parse@child@node@coordinate(#1){%
  \pgfutil@ifnextchar a{\tikz@p@c@n@c@at(#1)}{%
    \def\tikz@child@node@text{[shape=coordinate,name=#1]{}}%
    \tikz@parse@child@node@rest}}
\def\tikz@p@c@n@c@at(#1)at#2(#3){%
  \def\tikz@child@node@text{[shape=coordinate,name=#1]at(#3){}}%
  \tikz@parse@child@node@rest}%
\def\tikz@parse@child@node@n node{%
  \let\tikz@child@node@text=\pgfutil@empty%
  \tikz@p@c@s}%
\def\tikz@p@c@s}
\def\tikz@p@c@s@at at#1(#2){%
  \expandafter\def\expandafter\tikz@child@node@text\expandafter{\tikz@child@node@text at(#2)}
  \tikz@p@c@s}
\def\tikz@p@c@s@paran(#1){%
  \expandafter\def\expandafter\tikz@child@node@text\expandafter{\tikz@child@node@text(#1)}
  \tikz@p@c@s}
\def\tikz@p@c@s@bra[#1]{%
  \expandafter\def\expandafter\tikz@child@node@text\expandafter{\tikz@child@node@text[#1]}
  \tikz@p@c@s}
\def\tikz@p@c@s@group#1{%
  \expandafter\def\expandafter\tikz@child@node@text\expandafter{\tikz@child@node@text{#1}}
  \tikz@parse@child@node@rest}


%
% Timers
% 

\def\tikz@timer@line{%
  \pgftransformlineattime{\tikz@time}{\tikz@timer@start}{\tikz@timer@end}%
}

\def\tikz@timer@vhline{%
  \ifdim\tikz@time pt<0.5pt% first half
    \pgf@process{\tikz@timer@start}%
    \pgf@xa=\pgf@x%
    \pgf@ya=\pgf@y%
    \pgf@process{\tikz@timer@end}%
    \pgf@xb=\tikz@time pt%
    \pgf@xb=2\pgf@xb%    
    \edef\tikz@marshal{\noexpand\pgftransformlineattime{\pgf@sys@tonumber{\pgf@xb}}{\noexpand\tikz@timer@start}{%
        \noexpand\pgfqpoint{\the\pgf@xa}{\the\pgf@y}}}%
    \tikz@marshal%
  \else% second half
    \pgf@process{\tikz@timer@start}%
    \pgf@xa=\pgf@x%
    \pgf@ya=\pgf@y%
    \pgf@process{\tikz@timer@end}%
    \pgf@xb=\tikz@time pt%
    \pgf@xb=2\pgf@xb%
    \advance\pgf@xb by-1pt%
    \edef\tikz@marshal{\noexpand\pgftransformlineattime{\pgf@sys@tonumber{\pgf@xb}}%
      {\noexpand\pgfqpoint{\the\pgf@xa}{\the\pgf@y}}{\noexpand\tikz@timer@end}}%
    \tikz@marshal%
  \fi%
}

\def\tikz@timer@hvline{%
  \ifdim\tikz@time pt<0.5pt% first half
    \pgf@process{\tikz@timer@start}%
    \pgf@xa=\pgf@x%
    \pgf@ya=\pgf@y%
    \pgf@process{\tikz@timer@end}%
    \pgf@xb=\tikz@time pt%
    \pgf@xb=2\pgf@xb%    
    \edef\tikz@marshal{\noexpand\pgftransformlineattime{\pgf@sys@tonumber{\pgf@xb}}{\noexpand\tikz@timer@start}{%
        \noexpand\pgfqpoint{\the\pgf@x}{\the\pgf@ya}}}%
    \tikz@marshal%
  \else% second half
    \pgf@process{\tikz@timer@start}%
    \pgf@xa=\pgf@x%
    \pgf@ya=\pgf@y%
    \pgf@process{\tikz@timer@end}%
    \pgf@xb=\tikz@time pt%
    \pgf@xb=2\pgf@xb%
    \advance\pgf@xb by-1pt%
    \edef\tikz@marshal{\noexpand\pgftransformlineattime{\pgf@sys@tonumber{\pgf@xb}}%
      {\noexpand\pgfqpoint{\the\pgf@x}{\the\pgf@ya}}{\noexpand\tikz@timer@end}}%
    \tikz@marshal%
  \fi%
}

\def\tikz@timer@curve{%
  \pgftransformcurveattime{\tikz@time}{\tikz@timer@start}{\tikz@timer@cont@one}{\tikz@timer@cont@two}{\tikz@timer@end}%
}



%
% Coordinate systems
% 

\def\tikzdeclarecoordinatesystem#1#2{%
  \expandafter\def\csname tikz@parse@cs@#1\endcsname(##1){%
    \pgf@process{%
      #2%
      % Smuggle outside:
      \iftikz@shapeborder%
        \global\let\tikz@smuggle@a=\tikz@shapebordertrue%
      \else%
        \global\let\tikz@smuggle@a=\tikz@shapeborderfalse%
      \fi%
      \global\let\tikz@smubble@b=\tikz@shapeborder@name%
    }%
    \tikz@smuggle@a%
    \let\tikz@shapeborder@name=\tikz@smubble@b%
    \edef\tikz@return@coordinate{\noexpand\pgfqpoint{\the\pgf@x}{\the\pgf@y}}}%
}
\def\tikzaliascoordinatesystem#1#2{%
  \edef\pgf@marshal{\noexpand\let\expandafter\noexpand\csname
    tikz@parse@cs@#1\endcsname=\expandafter\noexpand\csname
    tikz@parse@cs@#2\endcsname}%
  \pgf@marshal%
}


% Default coodinate systems:

\tikzdeclarecoordinatesystem{canvas}
{%
  \tikzset{cs/.cd,x=0pt,y=0pt,#1}%
  \pgfpoint{\tikz@cs@x}{\tikz@cs@y}%
}

\tikzdeclarecoordinatesystem{canvas polar}
{%
  \tikzset{cs/.cd,angle=0,radius=0cm,#1}%
  \pgfpointpolar{\tikz@cs@angle}{\tikz@cs@xradius and \tikz@cs@yradius}%
}

\tikzdeclarecoordinatesystem{xyz}
{%
  \tikzset{cs/.cd,x=0,y=0,z=0,#1}%
  \pgfpointxyz{\tikz@cs@x}{\tikz@cs@y}{\tikz@cs@z}%
}

\tikzdeclarecoordinatesystem{xyz polar}
{%
  \tikzset{cs/.cd,angle=0,radius=0,#1}%
  \pgfpointpolarxy{\tikz@cs@angle}{\tikz@cs@xradius and \tikz@cs@yradius}%
}
\tikzaliascoordinatesystem{xy polar}{xyz polar}


\tikzdeclarecoordinatesystem{node}
{%
  \tikzset{cs/.cd,name=,anchor=none,angle=none,#1}%
  \ifx\tikz@cs@anchor\tikz@nonetext%
    \ifx\tikz@cs@angle\tikz@nonetext%
      \expandafter\ifx\csname pgf@sh@ns@\tikz@cs@node\endcsname\tikz@coordinate@text%
      \else
        \tikz@shapebordertrue%
        \edef\tikz@shapeborder@name{\tikz@cs@node}%
      \fi%
      \pgfpointanchor{\tikz@cs@node}{center}%
    \else%
      \pgfpointanchor{\tikz@cs@node}{\tikz@cs@angle}%
    \fi%
  \else%
    \pgfpointanchor{\tikz@cs@node}{\tikz@cs@anchor}%
  \fi%
}

\tikzdeclarecoordinatesystem{intersection}
{%
  \tikzset{cs/.cd,#1}%
  \expandafter\tikz@scan@one@point\expandafter\tikz@parse@intersection@a\tikz@cs@line@a@begin%
  \expandafter\tikz@scan@one@point\expandafter\tikz@parse@intersection@b\tikz@cs@line@a@end%
  \expandafter\tikz@scan@one@point\expandafter\tikz@parse@intersection@c\tikz@cs@line@b@begin%
  \expandafter\tikz@scan@one@point\expandafter\tikz@parse@intersection@d\tikz@cs@line@b@end%
  \edef\pgf@marshal{%
    {\noexpand\pgfpointintersectionoflines%
      {\noexpand\pgfqpoint{\the\pgf@xa}{\the\pgf@ya}}%
      {\noexpand\pgfqpoint{\the\pgf@xb}{\the\pgf@yb}}%
      {\noexpand\pgfqpoint{\the\pgf@xc}{\the\pgf@yc}}%
      {\noexpand\pgfqpoint{\the\pgf@x}{\the\pgf@y}}}}%
  \pgf@marshal%
}

\tikzdeclarecoordinatesystem{perpendicular}
{%
  \tikzset{cs/.cd,#1}%
  \expandafter\tikz@scan@one@point\expandafter\tikz@parse@intersection@a\tikz@cs@hori@line%
  \expandafter\tikz@scan@one@point\expandafter\tikz@parse@intersection@b\tikz@cs@vert@line%
  \pgfqpoint{\the\pgf@xb}{\the\pgf@ya}
}

\tikzdeclarecoordinatesystem{barycentric}
{%
  {%
    \pgf@xa=0pt% point
    \pgf@ya=0pt%
    \pgf@xb=0pt% sum
    \tikz@bary@dolist#1,=,%
    \pgfmathparse{1/\the\pgf@xb}%
    \global\pgf@x=\pgfmathresult\pgf@xa%
    \global\pgf@y=\pgfmathresult\pgf@ya%
  }%
}

\def\tikz@bary@dolist#1=#2,{%
  \def\tikz@temp{#1}%
  \ifx\tikz@temp\pgfutil@empty%
  \else
    \pgf@process{\pgfpointanchor{#1}{center}}%
    \pgfmathparse{#2}%
    \advance\pgf@xa by\pgfmathresult\pgf@x%
    \advance\pgf@ya by\pgfmathresult\pgf@y%
    \advance\pgf@xb by\pgfmathresult pt%
    \expandafter\tikz@bary@dolist%
  \fi%
}

\tikzset{cs/x/.store in=\tikz@cs@x}
\tikzset{cs/y/.store in=\tikz@cs@y}
\tikzset{cs/z/.store in=\tikz@cs@z}
\tikzset{cs/angle/.store in=\tikz@cs@angle}
\tikzset{cs/x radius/.store in=\tikz@cs@xradius}
\tikzset{cs/y radius/.store in=\tikz@cs@yradius}
\tikzset{cs/radius/.style={/tikz/cs/x radius=#1,/tikz/cs/y radius=#1}}
\tikzset{cs/name/.store in=\tikz@cs@node}
\tikzset{cs/anchor/.store in=\tikz@cs@anchor}

\tikzset{cs/first line/.code args={(#1)--(#2)}{\def\tikz@cs@line@a@begin{(#1)}\def\tikz@cs@line@a@end{(#2)}}}
\tikzset{cs/second line/.code args={(#1)--(#2)}{\def\tikz@cs@line@b@begin{(#1)}\def\tikz@cs@line@b@end{(#2)}}}

\tikzset{cs/horizontal line through/.store in=\tikz@cs@hori@line}
\tikzset{cs/vertical line through/.store in=\tikz@cs@vert@line}




%
% Coordinate management
%


% Last position visited
\def\tikz@last@position{\pgfqpoint{\tikz@lastx}{\tikz@lasty}}
\def\tikz@last@position@saved{\pgfqpoint{\tikz@lastxsaved}{\tikz@lastysaved}}

% Make given point the last position visited
\def\tikz@make@last@position#1{%
  \pgf@process{#1}%
  \tikz@lastx=\pgf@x\relax%
  \tikz@lasty=\pgf@y\relax%
  \iftikz@updatecurrent%
    \tikz@lastxsaved=\pgf@x\relax%
    \tikz@lastysaved=\pgf@y\relax%
  \fi%
  \tikz@updatecurrenttrue%
}

\newif\iftikz@updatecurrent
\tikz@updatecurrenttrue



% Scanner: Scans a point or a relative point. 
% It then calls the first parameter with the argument set to an
% appropriate pgf command representing that point.

\def\tikz@scan@one@point#1{%
  \let\tikz@to@use@whom=\tikz@to@use@last@coordinate%
  \tikz@shapeborderfalse%
  \pgfutil@ifnextchar+{\tikz@scan@relative#1}{\tikz@scan@absolute#1}}
\def\tikz@scan@absolute#1{%
  \pgfutil@ifnextchar({\tikz@scan@@absolute#1}%)
  {%
    \advance\tikz@expandcount by -1%
    \ifnum\tikz@expandcount<0\relax%
      \let\@next=\tikz@@scangiveup%
    \else%
      \let\@next=\tikz@@scanexpand%
    \fi%
    \@next{#1}%
  }%
}
\def\tikz@@scanexpand#1{\expandafter\tikz@scan@one@point\expandafter#1}
\def\tikz@@scangiveup#1{\PackageError{tikz}{Cannot parse this coordinate}{}#1{\pgfpointorigin}}
\def\tikz@scan@@absolute#1(#2){%
  \edef\tikz@temp{(#2)}%
  \expandafter\tikz@@scan@@absolute\expandafter#1\tikz@temp%
}
\def\tikz@@scan@@absolute#1({%
  \pgfutil@ifnextchar[% uhoh... options!
  {\def\tikz@scan@point@recall{#1}\tikz@scan@options}%
  {\tikz@@@scan@@absolute#1(}%
}

\def\tikz@scan@options[#1]#2{%
  \def\tikz@scan@point@options{#1}%
  \tikz@@@scan@@absolute\tikz@scan@handle@options(#2%
}

\def\tikz@scan@handle@options#1{%
  {%
    % Ok, compute point with options set and zero transformation
    % matrix:
    \pgftransformreset%
    \let\tikz@transform=\pgfutil@empty%
    \expandafter\tikzset\expandafter{\tikz@scan@point@options}%
    \tikz@transform%
    \pgf@process{\pgfpointtransformed{#1}}%
    \xdef\tikz@marshal{\expandafter\noexpand\tikz@scan@point@recall{\noexpand\pgfqpoint{\the\pgf@x}{\the\pgf@y}}}%
  }%
  \tikz@marshal%  
}

\def\tikz@@@scan@@absolute#1(#2){%
  \pgfutil@in@{intersection of}{#2}%
  \ifpgfutil@in@%
    \let\@next\tikz@parse@intersection%
  \else%
    \pgfutil@in@|{#2}%
    \ifpgfutil@in@
      \pgfutil@in@{-|}{#2}%
      \ifpgfutil@in@
        \let\@next\tikz@parse@hv%
      \else%
        \let\@next\tikz@parse@vh%
      \fi%
    \else%
      \pgfutil@in@{cs:}{#2}%
      \ifpgfutil@in@%
        \let\@next\tikz@parse@coordinatesystem%
      \else%
        \pgfutil@in@:{#2}%
        \ifpgfutil@in@
          \let\@next\tikz@parse@polar%
        \else%
          \pgfutil@in@,{#2}%
          \ifpgfutil@in@%      
            \let\@next\tikz@parse@regular%
          \else%
            \let\@next\tikz@parse@node%
          \fi%
        \fi%
      \fi%
    \fi%
  \fi%
  \@next#1(#2)%
}

\def\tikz@parse@coordinatesystem#1(#2 cs:#3){%
  \let\tikz@return@coordinate=\pgfpointorigin%
  \pgfutil@ifundefined{tikz@parse@cs@#2}
  {\PackageError{tikz}{Unknown coordinate system '#2'}{}}
  {\csname tikz@parse@cs@#2\endcsname(#3)}%
  \expandafter#1\expandafter{\tikz@return@coordinate}%
}


\newif\iftikz@isdimension
\def\tikz@checkunit#1{%
  \pgfmathparse{#1}%
  \let\iftikz@isdimension=\ifpgfmathunitsdeclared%
}
\def\tikz@@checkunit{\pgfutil@ifnextchar\tikz@unique{\tikz@checkunit@number}{\tikz@checkunit@dimension}}
\def\tikz@checkunit@number\tikz@unique{\tikz@isdimensionfalse}
\def\tikz@checkunit@dimension#1\tikz@unique{\tikz@isdimensiontrue}

\def\tikz@parse@polar#1(#2:#3){%
  \pgfutil@ifundefined{tikz@polar@dir@#2}
  {\tikz@@parse@polar#1(#2:#3)}
  {\tikz@@parse@polar#1(\csname tikz@polar@dir@#2\endcsname:#3)}%
}
\def\tikz@@parse@polar#1(#2:#3){%
  \pgfutil@in@{ and }{#3}%
  \ifpgfutil@in@%
    \edef\tikz@args{(#2:#3)}%
  \else%
    \edef\tikz@args{(#2:#3 and #3)}%
  \fi%
  \expandafter\tikz@@@parse@polar\expandafter#1\tikz@args%
}
\def\tikz@@@parse@polar#1(#2:#3 and #4){%
  \tikz@checkunit{#3}%
  \iftikz@isdimension%
    \def\tikz@next{#1{\pgfpointpolar{#2}{#3 and #4}}}%
  \else%
    \def\tikz@next{#1{\pgfpointpolarxy{#2}{#3 and #4}}}%
  \fi%
  \tikz@next%
}
\def\tikz@polar@dir@up{90}
\def\tikz@polar@dir@down{-90}
\def\tikz@polar@dir@left{180}
\def\tikz@polar@dir@right{0}
\def\tikz@polar@dir@north{90}
\def\tikz@polar@dir@south{-90}
\def\tikz@polar@dir@east{0}
\def\tikz@polar@dir@west{180}
\expandafter\def\csname tikz@polar@dir@north east\endcsname{45}
\expandafter\def\csname tikz@polar@dir@north west\endcsname{135}
\expandafter\def\csname tikz@polar@dir@south east\endcsname{-45}
\expandafter\def\csname tikz@polar@dir@south west\endcsname{-135}

\def\tikz@parse@regular#1(#2,#3){%
  \pgfutil@in@,{#3}%
  \ifpgfutil@in@%  
    \tikz@parse@splitxyz{#1}{#2}#3,%
  \else%
    \tikz@checkunit{#2}%
    \iftikz@isdimension%
      \def\@next{#1{\pgfpoint{#2}{#3}}}%
    \else%
      \def\@next{#1{\pgfpointxy{#2}{#3}}}%
    \fi%
  \fi%
  \@next%
}

\def\tikz@parse@splitxyz#1#2#3,#4,{%
  \def\@next{#1{\pgfpointxyz{#2}{#3}{#4}}}%
}

\def\tikz@coordinate@text{coordinate}

\def\tikz@parse@node#1(#2){%
  \pgfutil@in@.{#2}% Ok, flag this
  \ifpgfutil@in@
    \tikz@calc@anchor#2\tikz@stop%
  \else%
    \tikz@calc@anchor#2.center\tikz@stop% to be on the save side, in
                                % case iftikz@shapeborder is ignored...
    \expandafter\ifx\csname pgf@sh@ns@#2\endcsname\tikz@coordinate@text%
    \else
      \tikz@shapebordertrue%
      \def\tikz@shapeborder@name{#2}%
    \fi%
  \fi%
  \edef\tikz@marshal{\noexpand#1{\noexpand\pgfqpoint{\the\pgf@x}{\the\pgf@y}}}%
  \tikz@marshal%
}

\def\tikz@calc@anchor#1.#2\tikz@stop{%
  \pgfpointanchor{#1}{#2}%
}


\def\tikz@parse@hv#1(#2){%
  \pgfutil@in@{ -| }{#2}%
  \ifpgfutil@in@%
    \let\tikz@next=\tikz@parse@hvboth%
  \else%
    \pgfutil@in@{ -|}{#2}%
    \ifpgfutil@in@%
      \let\tikz@next=\tikz@parse@hvleft%
    \else%
      \pgfutil@in@{-| }{#2}%
      \ifpgfutil@in@%
        \let\tikz@next=\tikz@parse@hvright%
      \else%
        \let\tikz@next=\tikz@parse@hvdone%
      \fi%
    \fi%
  \fi%
  \tikz@next#1(#2)}
\def\tikz@parse@hvboth#1(#2 -| #3){\tikz@parse@vhdone#1(#3|-#2)}
\def\tikz@parse@hvleft#1(#2 -|#3){\tikz@parse@vhdone#1(#3|-#2)}
\def\tikz@parse@hvright#1(#2-| #3){\tikz@parse@vhdone#1(#3|-#2)}
\def\tikz@parse@hvdone#1(#2-|#3){\tikz@parse@vhdone#1(#3|-#2)}

\def\tikz@parse@vh#1(#2){%
  \pgfutil@in@{ |- }{#2}%
  \ifpgfutil@in@%
    \let\tikz@next=\tikz@parse@vhboth%
  \else%
    \pgfutil@in@{ |-}{#2}%
    \ifpgfutil@in@%
      \let\tikz@next=\tikz@parse@vhleft%
    \else%
      \pgfutil@in@{|- }{#2}%
      \ifpgfutil@in@%
        \let\tikz@next=\tikz@parse@vhright%
      \else%
        \let\tikz@next=\tikz@parse@vhdone%
      \fi%
    \fi%
  \fi%
  \tikz@next#1(#2)}
\def\tikz@parse@vhboth#1(#2 |- #3){\tikz@parse@vhdone#1(#2|-#3)}
\def\tikz@parse@vhleft#1(#2 |-#3){\tikz@parse@vhdone#1(#2|-#3)}
\def\tikz@parse@vhright#1(#2|- #3){\tikz@parse@vhdone#1(#2|-#3)}
\def\tikz@parse@vhdone#1(#2|-#3){%
  {%
    \tikz@@@scan@@absolute\tikz@parse@vh@mid(#2)%
    \tikz@@@scan@@absolute\tikz@parse@vh@end(#3)%
    \xdef\tikz@marshal{\noexpand#1{\noexpand\pgfqpoint{\the\pgf@xa}{\the\pgf@ya}}}%
  }%
  \tikz@shapeborderfalse%
  \tikz@marshal%
}
\def\tikz@parse@vh@mid#1{\pgf@process{#1}\pgf@xa=\pgf@x}
\def\tikz@parse@vh@end#1{\pgf@process{#1}\pgf@ya=\pgf@y}

\def\tikz@parse@intersection#1(intersection of #2--#3 and #4--#5){%
  {%
    \tikz@@@scan@@absolute\tikz@parse@intersection@a(#2)%
    \tikz@@@scan@@absolute\tikz@parse@intersection@b(#3)%
    \tikz@@@scan@@absolute\tikz@parse@intersection@c(#4)%
    \tikz@@@scan@@absolute\tikz@parse@intersection@d(#5)%
    \xdef\tikz@marshal{\noexpand#1{\noexpand\pgfpointintersectionoflines%
        {\noexpand\pgfqpoint{\the\pgf@xa}{\the\pgf@ya}}%
        {\noexpand\pgfqpoint{\the\pgf@xb}{\the\pgf@yb}}%
        {\noexpand\pgfqpoint{\the\pgf@xc}{\the\pgf@yc}}%
        {\noexpand\pgfqpoint{\the\pgf@x}{\the\pgf@y}}}}%
  }%
  \tikz@shapeborderfalse%
  \tikz@marshal%  
}

\def\tikz@parse@intersection@a#1{\pgf@process{#1}\pgf@xa=\pgf@x\pgf@ya=\pgf@y}
\def\tikz@parse@intersection@b#1{\pgf@process{#1}\pgf@xb=\pgf@x\pgf@yb=\pgf@y}
\def\tikz@parse@intersection@c#1{\pgf@process{#1}\pgf@xc=\pgf@x\pgf@yc=\pgf@y}
\def\tikz@parse@intersection@d#1{\pgf@process{#1}}

\def\tikz@scan@relative#1+{%
  \pgfutil@ifnextchar+{\tikz@scan@plusplus#1}{\tikz@scan@oneplus#1}}

\def\tikz@scan@plusplus#1+{%
  \def\tikz@doafter{#1}%
  \tikz@scan@absolute\tikz@add%
}
\def\tikz@add#1{%
  \tikz@doafter{\pgfpointadd{#1}{\tikz@last@position@saved}}%
}
\def\tikz@scan@oneplus#1{%
  \def\tikz@doafter{#1}%
  \tikz@updatecurrentfalse%
  \tikz@scan@absolute\tikz@add%
} 



% Loading further libraries

% Include a library file.
%
% #1 = List of names of library file.
%  
% Description:
%
% This command includes a list of TikZ library files. For each file X in the
% list, the file pgflibrarytikzX.code.tex is included, provided this has
% not been done earlier. 
%
% For the convenience of Context users, both round and square brackets
% are possible for the argument.
%
% Example:
%
% \usetikzlibrary{arrows}
% \usetikzlibrary[patterns,topaths]

\def\usetikzlibrary{\pgfutil@ifnextchar[{\use@tikzlibrary}{\use@@tikzlibrary}}%}
\def\use@tikzlibrary[#1]{\use@@tikzlibrary{#1}}
\def\use@@tikzlibrary#1{%
  \edef\pgf@list{#1}%
  \pgfutil@for\pgf@temp:=\pgf@list\do{%
    \expandafter\ifx\csname tikz@library@\pgf@temp @loaded\endcsname\relax%
      \expandafter\global\expandafter\let\csname tikz@library@\pgf@temp @loaded\endcsname=\pgfutil@empty%
      \expandafter\edef\csname tikz@library@#1@atcode\endcsname{\the\catcode`\@}
      \expandafter\edef\csname tikz@library@#1@barcode\endcsname{\the\catcode`\|}
      \catcode`\@=11
      \catcode`\|=12
      \input pgflibrarytikz\pgf@temp.code.tex
      \catcode`\@=\csname tikz@library@#1@atcode\endcsname
      \catcode`\|=\csname tikz@library@#1@barcode\endcsname
    \fi%
  }%
}


% Always-present libraries:

\usetikzlibrary{topaths}




\endinput


\endinput
\end{codeexample}

The files in the |generic/pgf| directory do the actual work.



\subsubsection{Using the Plain \TeX\ Format}

When using the plain \TeX\ format, you say |% Copyright 2006 by Till Tantau
%
% This file may be distributed and/or modified
%
% 1. under the LaTeX Project Public License and/or
% 2. under the GNU Public License.
%
% See the file doc/generic/pgf/licenses/LICENSE for more details.


\edef\pgfatcode{\the\catcode`\@}
\catcode`\@=11


\input pgfrcs.tex
\ProvidesPackageRCS $Header: /cvsroot/pgf/pgf/plain/pgf/basiclayer/pgf.tex,v 1.9 2008/01/13 10:35:47 vibrovski Exp $

\input pgfcore.tex
\input pgfbaseshapes.tex
\input pgfbaseplot.tex
\input pgfbaseimage.tex
\input pgfbaselayers.tex
\input pgfbasesnakes.tex
\input pgfbasedecorations.tex
\input pgfbasepatterns.tex
\input pgfbasematrix.tex

\catcode`\@=\pgfatcode

\endinput
| or
|% This file is tikz.tex

% Copyright 2003 by Till Tantau <tantau@cs.tu-berlin.de>.
%
% This program can be redistributed and/or modified under the terms
% of the LaTeX Project Public License Distributed from CTAN
% archives in directory macros/latex/base/lppl.txt.

\edef\tikzatcode{\the\catcode`\@}
\catcode`\@=11

\input xkeyval.tex
\input pgf.tex
\input pgffor.tex
\input pgflibraryplothandlers.tex
\input tikz.code.tex

\catcode`\@=\tikzatcode

\endinput
|. Then, instead of  |\begin{pgfpicture}| and
  |\end{pgfpicture}| you use  |\pgfpicture| and |\endpgfpicture|. 

Unlike for the \LaTeX\ format, \pgfname\ is not as good at discerning
the appropriate configuration for the plain \TeX\ format. In
particular, it can only automatically determine the correct output
format if you use |pdftex| or |tex| plus |dvips|. For all other output
formats you need to set the macro |\pgfsysdriver| to the correct
value. See the description of using output formats later on. 

\pgfname\ was originally written for use with \LaTeX\ and this shows
in a number of places. Nevertheless, the plain \TeX\ support is
reasonably good.

Like the \LaTeX\ style files, the plain \TeX\ files like |tikz.tex|
also just include the correct |tikz.code.tex| file.



\subsubsection{Using the Con\TeX t Format}

When using the Con\TeX t format, you say |\usemodule[pgf]| or
|\usemodule[tikz]|. As for the plain \TeX\ format you also have
to replace the start- and end-of-environment tags as follows: Instead
of  |\begin{pgfpicture}| and |\end{pgfpicture}| you use
|\startpgfpicture| and |\stoppgfpicture|; similarly, instead of
|\begin{tikzpicture}| and |\end{tikzpicture}| you use must now use
|\starttikzpicture| and |\stoptikzpicture|; and so on for other
environments. 

The Con\TeX t support is very similar to the plain \TeX\ support, so
the same restrictions apply: You may have to set the output
format directly and graphics inclusion may be a problem.

In addition to |pgf| and |tikz| there also exist modules like
|pgfcore|, |pgfbaseimage|, |pgflibrarysnakes| and so on. To
use them, you may need to include the module |pgfmod| first (the
modules |pgf| and |tikz| both include |pgfmod| for you, so typically
you can skip this). This special module is necessary since Con\TeX t
satanically restricts the length of module names to 6 characters
and \pgfname's long names are mapped
to cryptic 6-letter-names for you by the module |pgfmod|.





\subsection{Supported Output Formats}
\label{section-drivers}

An output format is a format in which \TeX\ outputs the text it has
typeset. Producing the output is (conceptually) a two-stage process:
\begin{enumerate}
\item
  \TeX\ typesets your text and graphics. The result of this
  typesetting is mainly a long list of letter--coordinate pairs, plus 
  (possibly) some ``special'' commands. This long list of pairs
  is written to something called a |.dvi|-file.
\item
  Some other program reads this |.dvi|-file and translates the
  letter--coordinate pairs into, say, PostScript commands for placing
  the given letter at the given coordinate.
\end{enumerate}

The classical example of this process is the combination of |latex|
and |dvips|. The |latex| program (which is just the |tex| program
called with the \LaTeX-macros preinstalled) produces a |.dvi|-file as
its output. The |dvips| program takes this output and produces a
|.ps|-file (a PostScript) file. Possibly, this file is further
converted using, say, |ps2pdf|, whose name is supposed to mean
``PostScript to PDF.'' Another example of programs using this
process is the combination of |tex| and |dvipdfm|. The |dvipdfm|
program takes a |.dvi|-file as 
input and translates the letter--coordinate pairs therein into
\pdf-commands, resulting in a |.pdf| file directly. Finally, the
|tex4ht| is also a program that takes a |.dvi|-file and produces an
output, this time it is a |.html| file. The programs |pdftex| and
|pdflatex| are special: They directly produce a |.pdf|-file without
the intermediate |.dvi|-stage. However, from the programmer's point of
view they behave exactly as if there where an intermediate stage.

Normally, \TeX\ only produces letter--coordinate pairs as its
``output.'' This obviously makes is difficult to draw, say, a
curve. For this, ``special'' commands can be used. Unfortunately,
these special commands are not the same for the different programs
that process the |.dvi|-file. Indeed, every program that takes a
|.dvi|-file as input has a totally different syntax for the special
commands.

One of the main jobs of \pgfname\ is to ``abstract way'' the
difference in the syntax of the different programs. However, this
means that support for each program has to be ``programmed,'' which is
a time-consuming and complicated process. 


\subsubsection{Selecting the Backend Driver}

When \TeX\ typesets your document, it does not know which program
you are going to use to transform the |.dvi|-file. If your |.dvi|-file
does not contain any special commands, this would be fine; but these
days almost all |.dvi|-files contain lots of special commands. It is
thus necessary to tell \TeX\ which program you are going to use later
on.

Unfortunately, there is no ``standard'' way of telling this to
\TeX. For the \LaTeX\ format a sophisticated mechanism exists inside
the |graphics| package and \pgfname\ plugs into this mechanism. For
other formats and when this plugging does not work as expected, it is
necessary to tell \pgfname\ directly which program you are going to
use. This is done by redefining the macro |\pgfsysdriver| to an
appropriate value \emph{before} you load |pgf|. If you are going to
use the |dvips| program, you set this macro to the value
|pgfsys-dvips.def|; if you use |pdftex| or |pdflatex|, you set it to
|pgfsys-pdftex.def|; and so on. In the following, details of the
support of the different programs are discussed.


\subsubsection{Producing PDF Output}

\pgfname\ supports three programs that produce \pdf\ output (\pdf\ means
``portable document format'' and was invented by the Adobe company):
|dvipdfm|, |pdftex|, and |vtex|. The |pdflatex| program is the same as the
|pdftex| program: it uses a different input format, but the output is
exactly the same.

\begin{filedescription}{pgfsys-pdftex.def}
  This is the driver file for use with pdf\TeX, that is, with the
  |pdftex| or |pdflatex| command. It includes
  |pgfsys-common-pdf.def|.

  This driver has the ``complete'' functionality. This means,
  everything \pgfname\ ``can do at all'' is implemented in this
  driver. 
\end{filedescription}

\begin{filedescription}{pgfsys-dvipdfm.def}
  This is a driver file for use with (|la|)|tex| followed by |dvipdfm|. It
  includes |pgfsys-common-pdf.def|.

  This driver supports most of \pgfname's features, but there are some
  restrictions:
  \begin{enumerate}
  \item
    In \LaTeX\ mode it uses |graphicx| for the graphics
    inclusion and does not support masking.
  \item
    In plain \TeX\ mode it does not support image inclusion.
  \item
    For remembering of pictures (inter-picture connections) you need
    to use a recent version of |pdftex| running in DVI-mode.
  \item
    Patterns are not (cannot) be supported.
  \item
    Functional shadings are not (cannot) be supported.
  \end{enumerate}
\end{filedescription}

\begin{filedescription}{pgfsys-xetex.def}
  This is a driver file for use with |xe|(|la|)|tex| followed by
  |xdvipdfmx|. This driver supports the same operations as the dvipdfm
  driver,  except that remembering of pictures (inter-picture
  connections)   always works.
\end{filedescription}

\begin{filedescription}{pgfsys-vtex.def}
  This is the driver file for use with the commercial \textsc{vtex}
  program. Even though it produces  \textsc{pdf} output, it
  includes |pgfsys-common-postscript.def|. Note that the
  \textsc{vtex} program can produce \emph{both} Postscript and
  \textsc{pdf} output, depending on the command line
  parameters. However, whether you produce Postscript or
  \textsc{pdf} output does not change anything with respect to the
  driver. 

  This driver supports most of \pgfname's features, except for
  the following restrictions:
  \begin{enumerate}
  \item
    In \LaTeX\ mode it uses |graphicx| for the graphics
    inclusion and does not support masking.
  \item
    In plain \TeX\ mode it does not support image inclusion.
  \item
    Shading is fully implemented, but yields the same quality as the
    implementation for |dvips|.
  \item
    Opacity is not supported.
  \item
    Remembering of pictures (inter-picture connections) is not
    supported. 
  \end{enumerate}
\end{filedescription}

It is also possible to produce a |.pdf|-file by first producing a
PostScript file (see below) and then using a PostScript-to-\pdf\
conversion program like |ps2pdf| or the Acrobat Distiller.


\subsubsection{Producing PostScript Output}

\begin{filedescription}{pgfsys-dvips.def}
  This is a driver file for use with (|la|)|tex| followed by
  |dvips|. It includes |pgfsys-common-postscript.def|.

  This driver also supports most of \pgfname's features, except for
  the following restrictions:
  \begin{enumerate}
  \item
    In \LaTeX\ mode it uses |graphicx| for the graphics
    inclusion and does not support masking.
  \item
    In plain \TeX\ mode it does not support image inclusion.
  \item
    Shading is fully implemented, but the results will not be 
    as good as with a driver producing |.pdf| as output. 
  \item
    Opacity works only in conjunction with newer versions of
    GhostScript. 
  \item
    For remembering of pictures (inter-picture connections) you need
    to use a recent version of |pdftex| running in DVI-mode.
  \end{enumerate}
\end{filedescription}

\begin{filedescription}{pgfsys-textures.def}
  This is a driver file for use with the \textsc{textures} program. It
  includes |pgfsys-common-postscript.def|. 

  This driver has exactly the same restrictions as the driver for
  |dvips|. 
\end{filedescription}

You can also use the |vtex| program together with |pgfsys-vtex.def| to
produce Postscript output.



\subsubsection{Producing HTML / SVG Output}

The |tex4ht| program converts |.dvi|-files to |.html|-files. While the
\textsc{html}-format cannot be used to draw graphics, the
\textsc{svg}-format can. Using the following driver, you can ask
\pgfname\ to produce an \textsc{svg}-picture for each \pgfname\
graphic in your text.

\begin{filedescription}{pgfsys-tex4ht.def}
  This is a driver file for use with the |tex4ht| program. It includes
  |pgfsys-common-svg.def|.

  When using this driver you should be aware of the following
  restrictions: 
  \begin{enumerate}
  \item
    In \LaTeX\ mode it uses |graphicx| for the graphics
    inclusion.    
  \item
    In plain \TeX\ mode it does not support image inclusion.
  \item
    Remembering of pictures (inter-picture connections) is not
    supported. 
  \item
    Text inside |pgfpicture|s is not supported very well. The reason
    is that the \textsc{svg} specification currently does not support
    text very well and it is also not possible to correctly ``escape
    back'' to \textsc{html}. All these problems will hopefully
    disappear in the future, but currently only two kinds of text work
    reasonably well: First, plain text without math mode, special
    characters or anything else special. Second, \emph{very} simple
    mathematical text that contains subscripts or superscripts. Even
    then, variables are not correctly set in italics and, in general,
    text simple does not look very nice.
  \item
    If you use text that contains anything special, even something as
    simple as |$\alpha$|, this may corrupt the graphic since |text4ht|
    does not always produce valid \textsc{xml} code. So, once more,
    \emph{stick to very simple node text inside graphics.} Sorry.
  \item
    Unlike for other output formats, the bounding box of a picture
    ``really crops'' the picture.
  \item
    Matrices do not work.
  \item
    Functional shadings are not supported.
  \end{enumerate}

  The driver basically works as follows: When a |{pgfpicture}| is
  started, appropriate |\special| commands are used to directed the
  output of |tex4ht| to a new file called |\jobname-xxx.svg|, where
  |xxx| is a number that is increased for each graphic. Then, till the
  end of the picture, each (system layer) graphic command creates a
  special that inserts appropriate \textsc{svg} literal text into the
  output file. The exact details are a bit complicated since the
  imaging model and the processing model of PostScript/\pdf\ and
  \textsc{svg} are not quite the same; but they are ``close enough''
  for \pgfname's purposes.
\end{filedescription}


\subsubsection{Producing Perfectly Portable DVI Output}

\begin{filedescription}{pgfsys-dvi.def}
  This is a driver file that can be used with any output driver,
  except for |tex4ht|.

  The driver will produce perfectly portable |.dvi| files by composing
  all pictures entirely of black rectangles, the basic and only graphic
  shape supported by the \TeX\ core. Even straight, but slanted lines
  are tricky to get right in this model (they need to be composed of
  lots of little squares).

  Naturally, \emph{very little} is possible with this driver. In fact,
  so little is possible that it is easier to list what is possible:
  \begin{itemize}
  \item Text boxes can be placed in the normal way.
  \item Lines and curves can be drawn (stroked). If they are not
    horizontal or vertical, they are composed of hundred of small
    rectangles.
  \item Lines of different width are supported.
  \item Transformations are supported.
  \end{itemize}
  Note that, say, even filling is not supported! (Let alone color or
  anything fancy.)

  This driver has only one real application: It might be useful when
  you only need horizontal or vertical lines in a picture. Then, the
  results are quite satisfactory.
\end{filedescription}



\part{Ti\emph{k}Z ist \emph{kein} Zeichenprogramm}
\label{part-tikz}

{\Large \emph{by Till Tantau}}


\bigskip
\noindent
\vskip3cm
\begin{codeexample}[graphic=white]
\begin{tikzpicture}
  \draw[fill=yellow] (0,0) -- (60:.75cm) arc (60:180:.75cm);
  \draw(120:0.4cm) node {$\alpha$};

  \draw[fill=green!30] (0,0) -- (right:.75cm) arc (0:60:.75cm);
  \draw(30:0.5cm) node {$\beta$};

  \begin{scope}[shift={(60:2cm)}]
    \draw[fill=green!30] (0,0) -- (180:.75cm) arc (180:240:.75cm);
    \draw (30:-0.5cm) node {$\gamma$};

    \draw[fill=yellow] (0,0) -- (240:.75cm) arc (240:360:.75cm);
    \draw (-60:0.4cm) node {$\delta$};
  \end{scope}

  \begin{scope}[thick]
    \draw  (60:-1cm) node[fill=white] {$E$} -- (60:3cm) node[fill=white] {$F$};
    \draw[red]                   (-2,0) node[left] {$A$} -- (3,0) node[right]{$B$};
    \draw[blue,shift={(60:2cm)}] (-3,0) node[left] {$C$} -- (2,0) node[right]{$D$};
  
    \draw[shift={(60:1cm)},xshift=4cm]
    node [right,text width=6cm,rounded corners,fill=red!20,inner sep=1ex]
    {
      When we assume that $\color{red}AB$ and $\color{blue}CD$ are
      parallel, i.\,e., ${\color{red}AB} \mathbin{\|} \color{blue}CD$,
      then $\alpha = \delta$ and $\beta = \gamma$.
    };
  \end{scope}
\end{tikzpicture}
\end{codeexample}



% Copyright 2006 by Till Tantau
%
% This file may be distributed and/or modified
%
% 1. under the LaTeX Project Public License and/or
% 2. under the GNU Free Documentation License.
%
% See the file doc/generic/pgf/licenses/LICENSE for more details.

\section{Design Principles}

This section describes the design principles behind the \tikzname\
frontend, where \tikzname\ means ``\tikzname\ ist \emph{kein}
Zeichenprogramm.'' To use \tikzname, as a \LaTeX\ user say
|\usepackage{tikz}| somewhere in the preamble, as a plain \TeX\ user
say |\input tikz.tex|. \tikzname's job is to make your life easier by
providing an easy-to-learn and easy-to-use syntax for describing
graphics. 

The commands and syntax of \tikzname\ were influenced by several
sources. The basic command names and the notion of  path operations is
taken from \textsc{metafont}, the option mechanism comes from
\textsc{pstricks}, the notion of styles is reminiscent of
\textsc{svg}. To make it all work together, some compromises were
necessary. I also added some ideas of my own, like snakes and coordinate
transformations.  

The following basic design principles underlie \tikzname:
\begin{enumerate}
\item Special syntax for specifying points.
\item Special syntax for path specifications.
\item Actions on paths.
\item Key-value syntax for graphic parameters.
\item Special syntax for nodes.
\item Special syntax for trees.
\item Grouping of graphic parameters.
\item Coordinate transformation system.
\end{enumerate}



\subsection{Special Syntax For Specifying Points}

\tikzname\ provides a special syntax for specifying points and
coordinates. In the simplest case, you provide two \TeX\ dimensions,
separated by commas, in round brackets as in |(1cm,2pt)|.

You can also specify a point in polar coordinates by using a colon
instead of a comma as in |(30:1cm)|, which means ``1cm in a 30
degrees direction.'' 

If you do not provide a unit, as in |(2,1)|, you specify a point in
\pgfname's $xy$-coordinate system. By default, the unit $x$-vector
goes 1cm to the right and the unit $y$-vector goes 1cm upward.

By specifying three numbers as in |(1,1,1)| you specify a point in
\pgfname's $xyz$-coordinate system.

It is also possible to use an anchor of a previously defined shape
as in |(first node.south)|.

You can add two plus signs before a coordinate as in
|++(1cm,0pt)|. This means ``1cm to the right of the last point
used.'' This allows you to easily specify relative movements. For
example, |(1,0) ++(1,0) ++(0,1)| specifies the three coordinates
|(1,0)|, then |(2,0)|, and |(2,1)|.

Finally, instead of two plus signs, you can also add a single
one. This also specifies a point in a relative manner, but it does
not ``change'' the current point used in subsequent relative
commands. For example, |(1,0) +(1,0) +(0,1)| specifies the three
coordinates |(1,0)|, then |(2,0)|, and |(1,1)|.

\subsection{Special Syntax For Path Specifications}

When creating a picture using \tikzname, your main job is the
specification of \emph{paths}. A path is a series of straight or curved
lines, which need not be connected. \tikzname\ makes it easy to
specify paths, partly using the syntax of \textsc{metapost}. For
example, to specify a triangular path you use
\begin{codeexample}[code only]
(5pt,0pt) -- (0pt,0pt) -- (0pt,5pt) -- cycle
\end{codeexample}
and you get \tikz \draw (5pt,0pt) -- (0pt,0pt) -- (0pt,5pt) -- cycle;
when you draw this path.

\subsection{Actions on Paths}

A path is just a series of straight and curved lines, but it is not
yet specified what should happen with it. One can \emph{draw} a
path, \emph{fill} a path, \emph{shade} it, \emph{clip} it, or do any
combination of these. Drawing (also known as \emph{stroking}) can be
thought of as taking a pen of a certain thickness and moving it
along the path, thereby drawing on the canvas. Filling means that
the interior of the path is filled with a uniform color. Obviously,
filling makes sense only for \emph{closed} paths and a path is
automatically closed prior to filling, if necessary.

Given a path as in |\path (0,0) rectangle (2ex,1ex);|, you can draw
it by adding the |draw| option as in
|\path[draw] (0,0) rectangle (2ex,1ex);|, which yields \tikz \path[draw]
(0,0) rectangle (2ex,1ex);. The |\draw| command is just an abbreviation for
|\path[draw]|. To fill a path, use the |fill| option or the |\fill|
command, which is an abbreviation for |\path[fill]|. The
|\filldraw| command is an abbreviation for
|\path[fill,draw]|. Shading is caused by the |shade| option (there
are |\shade| and |\shadedraw| abbreviations) and clipping by the
|clip| option. There is is also a |\clip| command, which does the
same as |\path[clip]|, but not commands like |\drawclip|. Use, say,
|\draw[clip]| or |\path[draw,clip]| instead.

All of these commands can only be used inside |{tikzpicture}|
environments. 

\tikzname\ allows you to use different colors for filling and
stroking.

\subsection{Key-Value Syntax for Graphic Parameters}

Whenever \tikzname\ draws or fills a path, a large number of graphic
parameters influenced the rendering. Examples include the colors
used, the dashing pattern, the clipping area, the line width, and
many others. In \tikzname, all these options are specified as lists
of so called key-value pairs, as in |color=red|, that are
passed as optional parameters to the path drawing and filling
commands. This usage is similar to \textsc{pstricks}. For
example, the following will draw a thick, red triangle;
\begin{codeexample}[]
\tikz \draw[line width=2pt,color=red] (1,0) -- (0,0) -- (1,0) -- cycle; 
\end{codeexample}

\subsection{Special Syntax for Specifying Nodes}
\tikzname\ introduces a special syntax for adding text or, more
generally, nodes to a graphic. When you specify a path, add nodes as
in the following example:
\begin{codeexample}[]
\tikz \draw (1,1) node {text} -- (2,2);
\end{codeexample}
Nodes are inserted at the current position of
the path, but only \emph{after} the path has been rendered. When
special options are given, as in
|\draw (1,1) node[circle,draw] {text};|, the text is not just put 
at the current position. Rather, it is surrounded by a circle and
this circle is ``drawn.'' 

You can add a name to a node for later reference either by using the
option   |name=|\meta{node name} or by stating the node name in
parentheses outside the text as in |node[circle](name){text}|.

Predefined shapes include |rectangle|, |circle|, and |ellipse|, but
it is possible (though a bit challenging) to define new shapes.

\subsection{Special Syntax for Specifying Trees}

In addition to the ``node syntax,'' \tikzname\ also introduces a
special syntax for drawing trees. The syntax is intergrated with the
special node syntax and only few new commands need to be remebered.
In essence, a |node| can be followed by any number of children, each
introduced by the keyword |child|. The children are nodes themselves,
each of which may have children in turn.

\begin{codeexample}[]
\begin{tikzpicture}
  \node {root}
    child {node {left}}
    child {node {right}
      child {node {child}}
      child {node {child}}
    };
\end{tikzpicture}
\end{codeexample}

Since trees are made up from nodes, it is possible to use options to
modify the way trees are drawn. Here are two examples of the above tree,
redrawn with different options:

\begin{codeexample}[]
\begin{tikzpicture}
  [edge from parent fork down,
   every node/.style={fill=red!30,rounded corners},
   edge from parent/.style={red,-o,thick,draw}]
  \node {root}
      child {node {left}}
      child {node {right}
        child {node {child}}
        child {node {child}}
      };
\end{tikzpicture}
\end{codeexample}

\begin{codeexample}[]
\begin{tikzpicture}
  [parent anchor=east,child anchor=west,grow=east,
   every node/.style={ball color=red,circle,text=white}
   edge from parent/.style={draw,dashed,thick,red}]
  \node {root}
      child {node {left}}
      child {node {right}
        child {node {child}}
        child {node {child}}
      };
\end{tikzpicture}
\end{codeexample}

\subsection{Grouping of Graphic Parameters}

Graphic parameters should often apply to several path drawing or
filling commands. For example, we may wish to draw numerous lines all
with the same line width of 1pt. For this, we put these commands
in a |{scope}| environment that takes the desired graphic options
as an optional parameter. Naturally, the specified graphic
parameters apply only to the drawing and filling commands inside the
environment. Furthermore, nested |{scope}| environments or
individual drawing commands can override the graphic parameters of
outer |{scope}| environments. In the following example, three red
lines, two green lines, and one blue line are drawn:

\begin{codeexample}[]
\begin{tikzpicture}
  \begin{scope}[color=red]
    \draw (0mm,10mm) -- (10mm,10mm);
    \draw (0mm, 8mm) -- (10mm, 8mm);
    \draw (0mm, 6mm) -- (10mm, 6mm);
  \end{scope}
  \begin{scope}[color=green]
    \draw             (0mm, 4mm) -- (10mm, 4mm);
    \draw             (0mm, 2mm) -- (10mm, 2mm);
    \draw[color=blue] (0mm, 0mm) -- (10mm, 0mm);
  \end{scope}
\end{tikzpicture}
\end{codeexample}

The |{tikzpicture}| environment itself also behaves like a
|{scope}| environment, that is, you can specify graphic parameters
using an optional argument. These optional apply to all commands in
the picture.


\subsection{Coordinate Transformation System}

\tikzname\ relies entirely on \pgfname's \emph{coordinate} transformation
system to perform transformations. \pgfname\ also supports
\emph{canvas} transformations, a more low-level transformation system,
but this system is not accessible from \tikzname. There are two reasons
for this: First, the canvas transformation must be used with great
care and often results in ``bad'' graphics with changing line width
and text in wrong sizes. Second, \pgfname\ looses track of where nodes
and shapes are positioned when canvas transformations are used.

For more details on the difference between coordinate transformations
and canvas transformations see
Section~\ref{section-design-transformations}. 

% Copyright 2006 by Till Tantau
%
% This file may be distributed and/or modified
%
% 1. under the LaTeX Project Public License and/or
% 2. under the GNU Free Documentation License.
%
% See the file doc/generic/pgf/licenses/LICENSE for more details.

\section[Hierarchical Structures: Package, Environments, Scopes, and Styles]
{Hierarchical Structures:\\
  Package, Environments, Scopes, and Styles}

The present section explains how your files should be structured when
you use \tikzname. On the top level, you need to include the |tikz|
package. In the main text, each graphic needs to be put in a
|{tikzpicture}| environment. Inside these environments, you can use
|{scope}| environments to create internal groups. Inside the scopes
you use |\path| commands to actually draw something. On all levels
(except for the package level), graphic options can be given that
apply to everything within the environment.



\subsection{Loading the Package and the Libraries}

\begin{package}{tikz}
  This package does not have any options.
  
  This will automatically load the \pgfname\ package and some other
  stuff that \tikzname\ needs (like the |xkeyval| package).

  \pgfname\ needs to know what \TeX\ driver you are intending to use. In
  most cases \pgfname\ is clever enough to determine the correct driver
  for you; this is true in particular if you \LaTeX. Currently, the only
  situation where \pgfname\ cannot know the driver ``by itself'' is when
  you use plain \TeX\ or Con\TeX t together with |dvipdfm|. In this case,
  you have to write |\def\pgfsysdriver{pgfsys-dvipdfm.def}|
  \emph{before} you input |tikz.tex|. 
\end{package}


\begin{command}{\usetikzlibrary\marg{list of libraries}}
  Once \tikzname\ has been loaded, you can use this command to load
  further libraries. The list of libraries should contain the names of
  libraries separated by commas. Instead of curly braces, you can also
  use square brackets, which is something Con\TeX t users will
  like. If you try to load a library a second time, nothing will
  happen. 

  \example |\usetikzlibrary{arrows}|

  The above command will load a whole bunch of extra arrow tip
  definitions.

  What this command does is to load the file
  |pgflibrarytikz|\meta{library}|.code.tex| for each \meta{library} in
  the \meta{list of libraries}. Thus, to write your own library file,
  all you need to do is to place a file of the appropriate name
  somewhere where \TeX\ can find it. \LaTeX, plain \TeX, and Con\TeX t
  users can then use your library.
\end{command}



\subsection{Creating a Picture}

\subsubsection{Creating a Picture Using an Environment}

The ``outermost'' scope of \tikzname\ is the |{tikzpicture}| 
environment. You may give drawing commands only inside this
environment, giving them outside (as is possible in many other
packages) will result in chaos.

In \tikzname, the way graphics are rendered is strongly influenced by
graphic options. For example, there is an option for setting the color used
for drawing, another for setting the color used for filling, and also
more obscure ones like the option  for setting the prefix used in the
filenames of temporary files written while plotting functions using an
external program. The graphic options are nearly always specified in a
so-called key-value style. (The ``nearly always'' refers to the name
of nodes, which can also be specified differently.) All graphic
options are local to the |{tikzpicture}| to which they apply.

\begin{environment}{{tikzpicture}\opt{\oarg{options}}}
  All \tikzname\ commands should be given inside this
  environment, except for the |\tikzstyle| command. Unlike other
  packages, it is not possible to use, say, |\pgfpathmoveto| outside
  this environment and doing so will result in chaos. For \tikzname,
  commands like |\path| are only defined inside this environment, so
  there is little chance that you will do something wrong here. 

  When this environment is encountered, the \meta{options} are
  parsed. All options given here will apply to the whole
  picture. 

  Next, the contents of the environment is processed and the graphic
  commands therein are put into a box. Non-graphic text is suppressed
  as well as possible, but non-\pgfname\ commands inside a
  |{tikzpicture}| environment should not produce any ``output'' since
  this may totally scramble the positioning system of the backend
  drivers. The suppressing of normal text, by the way, is done by
  temporarily switching the font to |\nullfont|. You can, however,
  ``escape back'' to normal \TeX\ typesetting. This happens, for
  example, when you specify a node.

  At the end of the environment, \pgfname\ tries to make a good guess
  at a good guess at the bounding box of the graphic and
  then resizes the box such that the box has this size. To ``make its
  guess,'' everytime \pgfname\ encounters a coordinate, it updates the
  bound box's size such that it encompasses all these
  coordinates. This will usually give a good 
  approximation at the bounding box, but will not always be
  accurate. First, the line thickness is not taken into
  account. Second, controls points of a curve often lie far
  ``outside'' the curve and make the bounding box too large. In this
  case, you should use the |[use as bounding box]| option.

  The following option influences the baseline of the resulting
  picture:
  \begin{itemize}
    \itemoption{baseline}\opt{|=|\meta{dimension or coordinate}}
    Normally, the lower end of the picture is put on the baseline of
    the surrounding text. For example, when you give the code
    |\tikz\draw(0,0)circle(.5ex);|, \pgfname\ will find out that the
    lower end of the picture is at $-.5\mathrm{ex}$ and that the upper
    end is at $.5\mathrm{ex}$. Then, the lower end will be put on the
    baseline, resulting in the following: \tikz\draw(0,0)circle(.5ex);.

    Using this option, you can specify that the picture should be
    raised or lowered such that the height \meta{dimension} is on the
    baseline. For example, |tikz[baseline=0pt]\draw(0,0)circle(.5ex);|
    yields \tikz[baseline=0pt]\draw(0,0)circle(.5ex); since, now, the
    baseline is on the height of the $x$-axis. If you omit the
    \meta{dimensions}, |0pt| is assumed as default.

    This options is often useful for ``inlined'' graphics as in
\begin{codeexample}[]
$A \mathbin{\tikz[baseline] \draw[->>] (0pt,.5ex) -- (3ex,.5ex);} B$
\end{codeexample}

    Instead of a \meta{dimension} you can also provide a coordinate in
    parantheses. Then the effect is to put the baseline on the
    $y$-coordinate that the give \meta{coordinate} has \emph{at the
      end of the picture}. This means that, at the end of the picture,
    the \meta{coordinate} is evaluated and then the baseline is set
    to the $y$-coordinate of the resulting point. This makes it easy
    to reference the $y$-coordinate of, say, the base line of nodes.
\begin{codeexample}[]
Hello
\tikz[baseline=(X.base)]
  \node [cross out,draw] (X) {world.};
\end{codeexample}

\begin{codeexample}[]
Top align:
\tikz[baseline=(current bounding box.north)]
  \draw (0,0) rectangle (1cm,1ex);
\end{codeexample}

    \itemoption{execute at begin picture}|=|\meta{code}
    This option can be used to install some code that will be executed
    at the beginning of the picture. This option must be
    given in the argument of the |{tikzpicture}| environment itself
    since this option will not have an effect otherwise. After all,
    the picture has already ``started'' later on.

    This option is mainly used in styles like the |every picture|
    style to execute certain code at the start  of a picture.

    \itemoption{execute at end picture}|=|\meta{code}
    This option installs some code that will be executed
    at the end of the picture. Using this option multiple times will
    cause the code to accumulate. This option must also be given in
    the optional argument of the |{tikzpicture}| environment.

\begin{codeexample}[]
\begin{tikzpicture}[execute at end picture=%
  {
    \begin{pgfonlayer}{background}
      \path[fill=yellow,rounded corners]
        (current bounding box.south west) rectangle
        (current bounding box.north east);
    \end{pgfonlayer}
  }]
  \node at (0,0) {X};
  \node at (2,1) {Y};
\end{tikzpicture}
\end{codeexample}
  \end{itemize}
  
  All options ``end'' at the end of the picture. To set an option
  ``globally'' you can use the following style:
  \begin{itemize}
    \itemstyle{every picture}
    This style is installed at the beginning of each picture.
\begin{codeexample}[code only]
\tikzstyle{every picture}=[semithick]
\end{codeexample}
  \end{itemize}
\end{environment}

In other \TeX\ format, you should use instead the following commands:

\begin{plainenvironment}{{tikzpicture}\opt{\oarg{options}}}
  This is the plain \TeX\ version of the environment.
\end{plainenvironment}

\begin{contextenvironment}{{tikzpicture}\opt{\oarg{options}}}
  This is the Con\TeX t version of the environment.
\end{contextenvironment}


\subsubsection{Creating a Picture Using a Command}

The following two commands are used for ``small'' graphics.

\begin{command}{\tikz\opt{\oarg{options}}\marg{commands}}
  This command places the \meta{commands} inside a
  |{tikzpicture}| environment and adds a semicolon at the end. This is
  just a convenience.

  The \meta{commands} may not contain a paragraph (an empty
  line). This is a precaution to ensure that users really use this
  command only for small graphics.

  \example |\tikz{\draw (0,0) rectangle (2ex,1ex)}| yields
  \tikz{\draw (0,0) rectangle (2ex,1ex);} 
\end{command}


\begin{command}{\tikz\opt{\oarg{options}}\meta{text}|;|}
  If the \meta{text} does not start with an opening brace, the end of
  the \meta{text} is the next semicolon that is encountered.

  \example |\tikz \draw (0,0) rectangle (2ex,1ex);| yields
  \tikz \draw (0,0) rectangle (2ex,1ex);
\end{command}



\subsubsection{Adding a Background}

By default, pictures do not have any background, that is, they are
``transparent'' on all parts on which you do not draw
anything. You may instead wish to have a colored background behind
your picture or a black frame around it or lines above and below it or
some other kind of decoration.

Since backgrounds are often not needed at all, the definition of
styles for adding backgrounds has been put in the library package
|pgflibrarytikzbackgrounds|. This package is documented in
Section~\ref{section-tikz-backgrounds}. 


\subsection{Using Scopes to Structure a Picture}

Inside a |{tikzpicture}| environment you can create scopes
using the |{scope}| environment. This environment is available only
inside the |{tikzpicture}| environment, so once more, there is little
chance of doing anything wrong.

\begin{environment}{{scope}\opt{\oarg{options}}}
  All \meta{options} are local to the \meta{environment
  contents}. Furthermore, the clipping path is also local to the
  environment, that is, any clipping done inside the environment
  ``ends'' at its end.

\begin{codeexample}[]
\begin{tikzpicture}
  \begin{scope}[red]
    \draw (0mm,0mm) -- (10mm,0mm);
    \draw (0mm,1mm) -- (10mm,1mm);
  \end{scope}
  \draw (0mm,2mm) -- (10mm,2mm);
  \begin{scope}[green]
    \draw (0mm,3mm) -- (10mm,3mm);
    \draw (0mm,4mm) -- (10mm,4mm);
    \draw[blue] (0mm,5mm) -- (10mm,5mm);
  \end{scope}
\end{tikzpicture}
\end{codeexample}
  
  The following style influences scopes:
  \begin{itemize}
    \itemstyle{every scope}
    This style is installed at the beginning of every scope. I do not
    know really know what this might be good for, but who knows?
  \end{itemize}

  The following options are useful for scopes:
  \begin{itemize}
    \itemoption{execute at begin scope}|=|\meta{code}
    This option install some code that will be executed
    at the beginning of the scope. This option must be
    given in the argument of the |{scope}| environment.

    The effect applies only to the current scope, not to subscopes.

    \itemoption{execute at end scope}|=|\meta{code}
    This option installs some code that will be executed
    at the end of the  current scope. Using this option multiple times
    will  cause the code to accumulate. This option must also be given
    in the optional argument of the |{scope}| environment. 

    Again, the effect applies only to the current scope, not to subscopes.
  \end{itemize}
\end{environment}

\begin{plainenvironment}{{scope}\opt{\oarg{options}}}
  Plain \TeX\ version of the environment.
\end{plainenvironment}

\begin{contextenvironment}{{scope}\opt{\oarg{options}}}
  Con\TeX t version of the environment.
\end{contextenvironment}



\subsection{Using Scopes Inside Paths}

The |\path| command, which is described in much more detail in later
sections, also takes graphic options. These options are local to the
path. Furthermore, it is possible to create local scopes within a
path simply by using curly braces as in
\begin{codeexample}[]
\tikz \draw (0,0) -- (1,1)
           {[rounded corners] -- (2,0) -- (3,1)}
           -- (3,0) -- (2,1);
\end{codeexample}

Note that many options apply only to the path as a whole and cannot be
scoped in this way. For example, it is not possible to scope the
|color| of the path. See the explanations in the section on paths for
more details.

Finally, certain elements that you specify in the argument to the
|\path| command also take local options. For example, a node
specification takes options. In this case, the options apply only to
the node, not to the surrounding path.



\subsection{Using Styles to Manage How Pictures Look}

There is a way of organizing sets of graphic options ``orthogonally''
to the normal scoping mechanism. For example, you might wish all your
``help lines'' to be drawn in a certain way like, say, gray and thin
(do \emph{not} dash them, that distracts). For this, you can use
\emph{styles}.

A style is simply a set of graphic options that is predefined at some
point. Once a style has been defined, it can be used anywhere using
the |style| option:

\begin{itemize}
  \itemoption{style}|=|\meta{style name}
  invokes all options that are currently set in the \meta{style
    name}. An example of a style is the predefined |help lines| style,
  which you should use for lines in the background like grid lines or
  construction lines. You can easily define new styles and modify
  existing ones.
\begin{codeexample}[]
\begin{tikzpicture}
  \draw                   (0,0) grid +(2,2);
  \draw[style=help lines] (2,0) grid +(2,2);
\end{tikzpicture}
\end{codeexample}
\end{itemize}


\begin{command}{\tikzstyle\meta{style name}\opt{|+|}|=[|\meta{options}|]|}
  This command defines the style \meta{style name}. Whenever it is
  used using the |style=|\meta{style name} command, the \meta{options}
  will be invoked. It is permissible that a style invokes another
  style using the |style=| command inside the \meta{options}, which
  allows you to build hierarchies of styles. Naturally, you should
  \emph{not} create cyclic dependencies.

  If the style already has a predefined meaning, it will
  unceremoniously be redefined without a warning.
\begin{codeexample}[]
\tikzstyle{help lines}=[blue!50,very thin]
\begin{tikzpicture}
  \draw                   (0,0) grid +(2,2);
  \draw[style=help lines] (2,0) grid +(2,2);
\end{tikzpicture}
\end{codeexample}

  If the optional |+| is given, the options are \emph{added} to the
  existing definition:
\begin{codeexample}[]
\tikzstyle{help lines}+=[dashed]% aaarghhh!!!
\begin{tikzpicture}
  \draw                   (0,0) grid +(2,2);
  \draw[style=help lines] (2,0) grid +(2,2);
\end{tikzpicture}
\end{codeexample}
\end{command}

It is also possible to set a style using an option:
\begin{itemize}
  \itemoption{set style}|={|\marg{style name}\opt{|+|}|=[|\meta{options}|]}|
  This option has the same effect as saying |\tikzstyle| before the
  argument of the option. 
\begin{codeexample}[]
\begin{tikzpicture}[set style={{help lines}+=[dashed]}]
  \draw                   (0,0) grid +(2,2);
  \draw[style=help lines] (2,0) grid +(2,2);
\end{tikzpicture}
\end{codeexample}
\end{itemize}



% Copyright 2006 by Till Tantau
%
% This file may be distributed and/or modified
%
% 1. under the LaTeX Project Public License and/or
% 2. under the GNU Free Documentation License.
%
% See the file doc/generic/pgf/licenses/LICENSE for more details.

\section{Specifying Coordinates}


\subsection{Overview}

A \emph{coordinate} is a position on the canvas on which your picture
is drawn. \tikzname\ uses a special syntax for specifying
coordinates. Coordinates are always put in round brackets. The general
syntax is 
\declare{|(|\opt{|[|\meta{options}|]|}\meta{coordinate  specification}|)|}. 

The \meta{coordinate specification} specified coordinates using one of
many different possible \emph{coordinate systems}. Examples are the
Cartesian coordinate system or polar coordinates or spherical
coordinates. No matter which coordinate system is used, in the end, a
specific point on the canvas is represented by the coordinate.

There are two ways of specifying which coordinate system should be used:
\begin{description}
\item[Explicitly] You can specify the coordinate system explicitly. To
  do so, you give the name of the coordinate system at the beginning,
  followed by |cs:|, which stands for ``coordinate system,'' followed
  by a specification of the coordinate using the key-value
  syntax. Thus, the general syntax for \meta{coordinate specification}
  in the explicit case is |(|\meta{coordinate system}| cs:|\meta{list
    of key-value pairs specific to the coordinate system}|)|.
\item[Implicitly] The explicit specification is often too verbose when
  numerous coordinates should be given. Because of this, for the
  coordinate systems that you are likely to use often a special syntax
  is provided. \tikzname\ will notice when you use a coordinate
  specified in a special syntax and will choose the correct coordinate
  system automatically.
\end{description}

Here is an example in which explicit the coordinate systems are
specified explicitly:
\begin{codeexample}[]
\begin{tikzpicture}
  \draw[help lines] (0,0) grid (3,2);
  \draw (canvas cs:x=0cm,y=2mm)
     -- (canvas polar cs:radius=2cm,angle=30);
\end{tikzpicture}
\end{codeexample}
In the next example, the coordinate systems are implicit:
\begin{codeexample}[]
\begin{tikzpicture}
  \draw[help lines] (0,0) grid (3,2);
  \draw (0cm,2mm) -- (30:2cm);
\end{tikzpicture}
\end{codeexample}

It is possible to give options that apply only to a single
coordinate, although this makes sense for transformation options
only. To give transformation options for a single coordinate, give
these options at the beginning in brackets:
\begin{codeexample}[]
\begin{tikzpicture}
  \draw[help lines] (0,0) grid (3,2);
  \draw      (0,0) -- (1,1);
  \draw[red] (0,0) -- ([xshift=3pt] 1,1);
  \draw      (1,0) -- +(30:2cm);
  \draw[red] (1,0) -- +([shift=(135:5pt)] 30:2cm);
\end{tikzpicture}
\end{codeexample}


\subsection{Coordinate Systems}

\subsubsection{Canvas, XYZ, and Polar Coordinate Systems}

Let us start with the basic coordinate systems.

\begin{coordinatesystem}{canvas}
  The simplest way of specifying a coordinate is to use the |canvas|
  coordinate system. You provide a dimension $d_x$ using the |x=|
  option and another dimension $d_y$ using the |y=| option. The position on
  the canvas is located at the position that is $d_x$ to the right and
  $d_y$ above the origin.

  \begin{key}{/tikz/cs/x=\meta{dimension} (initially 0pt)}
    Distance by which the coordinate
    is to the right of the origin. You can also write things like
    |1cm+2pt| since the mathematical engine is used to evaluate the
    \meta{dimension}.
  \end{key}

  \begin{key}{/tikz/cs/y=\meta{dimension} (initially 0pt)}
    Distance by which the coordinate
    is above the origin.
  \end{key}

\begin{codeexample}[]
\begin{tikzpicture}
  \draw[help lines] (0,0) grid (3,2);

  \fill (canvas cs:x=1cm,y=1.5cm)    circle (2pt);
  \fill (canvas cs:x=2cm,y=-5mm+2pt) circle (2pt);
\end{tikzpicture}
\end{codeexample}

  To specify a coordinate in the coordinate system implicitly, you use
  two dimensions that are separated by a comma as in |(0cm,3pt)| or
  |(2cm,\textheight)|. 
\begin{codeexample}[]
\begin{tikzpicture}
  \draw[help lines] (0,0) grid (3,2);

  \fill (1cm,1.5cm)    circle (2pt);
  \fill (2cm,-5mm+2pt) circle (2pt);
\end{tikzpicture}
\end{codeexample}
\end{coordinatesystem}


\begin{coordinatesystem}{xyz}
  The |xyz| coordinate system allows you to specify a point as a
  multiple of three vectors called the $x$-, $y$-, and
  $z$-vectors.  By default, the $x$-vector points 1cm to the right,
  the $y$-vector points 1cm upwards, but this can be changed
  arbitrarily as explained in Section~\ref{section-xyz}. The default
  $z$-vector points to $\bigl(-3.85\textrm{mm},-3.85\textrm{mm}\bigr)$.

  To specify the factors by which the vectors should be multiplied
  before being added, you use the following three options:  
  \begin{key}{/tikz/cs/x=\meta{factor} (initially 0)}
    Factor by which the $x$-vector is multiplied.
  \end{key}
  \begin{key}{/tikz/cs/y=\meta{factor} (initially 0)}
    Works like |x|.
  \end{key}
  \begin{key}{/tikz/cs/z=\meta{factor} (initially 0)}
    Works like |x|.
  \end{key}

\begin{codeexample}[]
\begin{tikzpicture}[->]
  \draw (0,0) -- (xyz cs:x=1);
  \draw (0,0) -- (xyz cs:y=1);
  \draw (0,0) -- (xyz cs:z=1);
\end{tikzpicture}
\end{codeexample}

  This coordinate system can also be selected implicitly. To do so,
  you just provide two or three comma-separated factors (not
  dimensions). 
\begin{codeexample}[]
\begin{tikzpicture}[->]
  \draw (0,0) -- (1,0);
  \draw (0,0) -- (0,1,0);
  \draw (0,0) -- (0,0,1);
\end{tikzpicture}
\end{codeexample}
\end{coordinatesystem}

\emph{Note:} It is possible to use coordinates like |(1,2cm)|, which
are neither |canvas| coordinates nor |xyz| coordinates. The rule is
the following: If a coordinate is of the implicit form
|(|\meta{x}|,|\meta{y}|)|, then \meta{x} and \meta{y} are checked,
independently, whether they have a dimension or whether they are
dimensionless. If both have a dimension, the |canvas| coordinate
system is used. If both lack a dimension, the |xyz| coordinate system
is used. If \meta{x} has a dimension and \meta{y} has not, then the
sum of two coordinate |(|\meta{x}|,0pt)| and |(0,|\meta{y}|)| is
used. If \meta{y} has a dimension and \meta{x} has not, then the sum
of two coordinate |(|\meta{x}|,0)| and |(0pt,|\meta{y}|)| is used.

\emph{Note furthermore:} An expression like |(2+3cm,0)| does
\emph{not} mean the same as |(2cm+3cm,0)|. Instead, if \meta{x} or
\meta{y} internally uses a mixture of dimensions and dimensionless
values, then all dimensionless values are ``upgraded'' to dimensions
by interpreting them as |pt|. So, |2+3cm| is the same dimension as
|2pt+3cm|. 

\begin{coordinatesystem}{canvas polar}
  The |canvas polar| coordinate system allows you to specify
  polar coordinates. You provide an angle using the |angle=| option
  and a radius using the |radius=| option. This yields the point on
  the canvas that is at the given radius distance from the origin at
  the given degree. A degree of zero points to the right, a degree of
  90 upward.
  \begin{key}{/tikz/cs/angle=\meta{degrees}}
    The angle of the coordinate.
    The angle must always be given in degrees and should be between
    $-360$ and $720$.
  \end{key}
  \begin{key}{/tikz/cs/radius=\meta{dimension}}
    The distance from the origin.
  \end{key}
  \begin{key}{/tikz/cs/x radius=\meta{dimension}}
    A polar coordinate is,
    after all, just a point on a circle of the given \meta{radius}. When
    you provide an $x$-radius and also a $y$-radius, you specify an
    ellipse instead of a circle. The |radius| option has the same effect
    as specifying identical |x radius| and |y radius| options.
  \end{key}
  \begin{key}{/tikz/cs/y radius=\meta{dimension}}
    Works like |x radius|.
  \end{key}
\begin{codeexample}[]
\tikz \draw (0,0) -- (canvas polar cs:angle=30,radius=1cm);
\end{codeexample}

  The implicit form for canvas polar coordinates is the following: 
  you specify the angle and the distance, separated by a colon as in
  |(30:1cm)|. 

\begin{codeexample}[]
\tikz \draw    (0cm,0cm) -- (30:1cm) -- (60:1cm) -- (90:1cm)
            -- (120:1cm) -- (150:1cm) -- (180:1cm);
\end{codeexample}

  Two different radii are specified by writing |(30:1cm and 2cm)|.

  For the implicit form, instead of an angle given as a number you can
  also use certain words. For example, |up| is the same as |90|, so
  that you can write |\tikz \draw (0,0) -- (2ex,0pt) -- +(up:1ex);|
  and get \tikz \draw (0,0) -- (2ex,0pt) -- +(up:1ex);. Apart from |up|
  you can use |down|, |left|, |right|, |north|, |south|, |west|, |east|,
  |north east|, |north west|, |south east|, |south west|, all of which
  have their natural meaning.
\end{coordinatesystem}

\begin{coordinatesystem}{xyz polar}
  This coordinate system work similarly to the |canvas polar|
  system. However, the radius and the angle are interpreted in the
  $xy$-coordinate system, not in the canvas system. More detailed,
  consider the circle or ellipse whose half axes are given by the
  current $x$-vector and the current $y$-vector. Then, consider the
  point that lies at a given angle on this ellipse, where an angle of
  zero is the same as the $x$-vector and an angle of 90 is the
  $y$-vector. Finally, multiply the resulting vector by the given
  radius factor. Voil�.
  \begin{key}{/tikz/cs/angle=\meta{degrees}}
    The angle of the coordinate
    interpreted in the ellipse whose axes are the $x$-vector and the
    $y$-vector.
  \end{key}
  \begin{key}{/tikz/cs/radius=\meta{factor}}
    A factor by which the $x$-vector
    and $y$-vector are multiplied prior to forming the ellipse.
  \end{key}
  \begin{key}{/tikz/cs/x radius=\meta{dimension}} A specific factor by
    which only the $x$-vector is multiplied.
  \end{key}
  \begin{key}{/tikz/cs/y radius=\meta{dimension}}
    Works like |x radius|.
  \end{key}
\begin{codeexample}[]
\begin{tikzpicture}[x=1.5cm,y=1cm]
  \draw[help lines] (0cm,0cm) grid (3cm,2cm);

  \draw (0,0) -- (xyz polar cs:angle=0,radius=1);
  \draw (0,0) -- (xyz polar cs:angle=30,radius=1);
  \draw (0,0) -- (xyz polar cs:angle=60,radius=1);
  \draw (0,0) -- (xyz polar cs:angle=90,radius=1);

  \draw (xyz polar cs:angle=0,radius=2)
     -- (xyz polar cs:angle=30,radius=2)
     -- (xyz polar cs:angle=60,radius=2)
     -- (xyz polar cs:angle=90,radius=2);
 \end{tikzpicture}
\end{codeexample}

  The implicit version of this option is the same as the implicit
  version of |canvas polar|, only you do not provide a unit.

\begin{codeexample}[]
\tikz[x={(0cm,1cm)},y={(-1cm,0cm)}]
  \draw  (0,0) -- (30:1) -- (60:1) -- (90:1)
             -- (120:1) -- (150:1) -- (180:1);
\end{codeexample}
\end{coordinatesystem}

\begin{coordinatesystem}{xy polar}
  This is just an alias for |xyz polar|, which some people might
  prefer as there is no z-coordinate involved in the |xyz polar|
  coordinates.   
\end{coordinatesystem}


\subsubsection{Barycentric Systems}
\label{section-barycentric-coordinates}

In the barycentric coordinate system a point is expressed as the
linear combination of multiple vectors. The idea is that you specify
vectors $v_1$, $v_2$, \dots, $v_n$ and numbers $\alpha_1$, $\alpha_2$,
\dots, $\alpha_n$. Then the barycentric coordinate specified by these
vectors and numbers is
\begin{align*}
  \frac{\alpha_1 v_1 + \alpha_2 v_2 + \cdots + \alpha_n v_n}{\alpha_1
    + \alpha_2 + \cdots + \alpha_n}
\end{align*}

The |barycentric cs| allows you to specify such coordinates easily.

\begin{coordinatesystem}{barycentric}
  For this coordinate system, the \meta{coordinate specification}
  should be a comma-separated list of expressions of the form
  \meta{node name}|=|\meta{number}. Note that (currently) the list
  should not contain any spaces before or after the \meta{node name}
  (unlike normal key-value pairs). 

  The specified coordinate is now computed as follows: Each pair
  provides one vector and a number. The vector is the |center| anchor
  of the \meta{node name}. The number is the \meta{number}. Note that
  (currently) you cannot specify a different anchor, so that in order
  to use, say, the |north| anchor of a node you first have to create a
  new coordinate at this north anchor. (Using for instance
  \texttt{\string\coordinate (mynorth) at (mynode.north);}.)

\begin{codeexample}[]
\begin{tikzpicture}
  \coordinate (content)   at (90:3cm);
  \coordinate (structure) at (210:3cm);
  \coordinate (form)      at (-30:3cm);
    
  \node [above]       at (content)   {content oriented};
  \node [below left]  at (structure) {structure oriented};
  \node [below right] at (form)      {form oriented};

  \draw [thick,gray] (content.south) -- (structure.north east) -- (form.north west) -- cycle;

  \small
  \node at (barycentric cs:content=0.5,structure=0.1 ,form=1)    {PostScript};
  \node at (barycentric cs:content=1  ,structure=0   ,form=0.4)  {DVI};
  \node at (barycentric cs:content=0.5,structure=0.5 ,form=1)    {PDF};
  \node at (barycentric cs:content=0  ,structure=0.25,form=1)    {CSS};
  \node at (barycentric cs:content=0.5,structure=1   ,form=0)    {XML};
  \node at (barycentric cs:content=0.5,structure=1   ,form=0.4)  {HTML};
  \node at (barycentric cs:content=1  ,structure=0.2 ,form=0.8)  {\TeX};
  \node at (barycentric cs:content=1  ,structure=0.6 ,form=0.8)  {\LaTeX};
  \node at (barycentric cs:content=0.8,structure=0.8 ,form=1)    {Word};
  \node at (barycentric cs:content=1  ,structure=0.05,form=0.05) {ASCII};
\end{tikzpicture}
\end{codeexample}
\end{coordinatesystem}

\subsubsection{Node Coordinate System}
\label{section-node-coordinates}

In \pgfname\ and in \tikzname\ it is quite easy to define a node that you
wish to reference at a later point. Once you have defined a node,
there are different ways of referencing points of the node. To do so,
you use the following coordinate system:

\begin{coordinatesystem}{node}
  This coordinate system is used to reference a specific point inside
  or on the border of a previously defined node. It can be used in
  different ways, so let us go over them one by one.

  You can use three options to specify which coordinate you mean:
  \begin{key}{/tikz/cs/name=\meta{node name}}
    Specifies the node in which you which to specify a coordinate. The
    \meta{node name} is 
    the name that was previously used to name the node using the
    |name=|\meta{node name} option or the special node name syntax.
  \end{key}
  \begin{key}{/tikz/anchor=\meta{anchor}}
    Specifies an anchor of the node. Here is an example: 
\begin{codeexample}[]
\begin{tikzpicture}
  \node (shape)   at (0,2)  [draw] {|class Shape|};
  \node (rect)    at (-2,0) [draw] {|class Rectangle|};
  \node (circle)  at (2,0)  [draw] {|class Circle|};
  \node (ellipse) at (6,0)  [draw] {|class Ellipse|};

  \draw (node cs:name=circle,anchor=north) |- (0,1);
  \draw (node cs:name=ellipse,anchor=north) |- (0,1);
  \draw[-open triangle 90] (node cs:name=rect,anchor=north)
        |- (0,1) -| (node cs:name=shape,anchor=south);
\end{tikzpicture}
\end{codeexample}
  \end{key}
  \begin{key}{/tikz/cs/angle=\meta{degrees}}
    It is also possible to provide an angle \emph{instead} of an
    anchor. This coordinate refers to a point of the node's
    border where a ray shot from the center
    in the given angle hits the border. Here is an example:
\begin{codeexample}[]
\begin{tikzpicture}
  \node (start) [draw,shape=ellipse] {start};
  \foreach \angle in {-90, -80, ..., 90}
    \draw (node cs:name=start,angle=\angle)
      .. controls +(\angle:1cm) and +(-1,0) .. (2.5,0);
  \end{tikzpicture}
\end{codeexample}
  \end{key}

  It is possible to provide \emph{neither} the |anchor=| option nor
  the |angle=| option. In this case, \tikzname\ will calculate an
  appropriate border position for you. Here is an example: 

\begin{codeexample}[]
\begin{tikzpicture}
  \path (0,0)  node(a) [ellipse,rotate=10,draw] {An ellipse}
        (3,-1) node(b) [circle,draw]            {A circle};
  \draw[thick] (node cs:name=a) -- (node cs:name=b);
\end{tikzpicture}
\end{codeexample}

  \tikzname\ will be reasonably clever at determining the border points that
  you ``mean,'' but, naturally, this may fail in some situations. If
  \tikzname\ fails to determine an appropriate border point, the center will
  be used instead.

  Automatic computation of anchors works only with the line-to operations
  |--|, the vertical/horizontal versions \verb!|-! and \verb!-|!, and
  with the curve-to operation |..|. For other path commands, such as
  |parabola| or |plot|, the center will be used. If this is not desired,
  you should give a named anchor or an angle anchor.
  
  Note that if you use an automatic coordinate for both the start and
  the end of a line-to, as in |--(node cs:name=b)--|, then \emph{two}
  border   coordinates are computed with a move-to between them. This
  is usually   exactly what you want.
  
  If you use relative coordinates together with automatic anchor
  coordinates, the relative coordinates are computed relative to
  the node's center, not relative to the border point. Here is an
  example:

\begin{codeexample}[]
\tikz \draw (0,0) node(x) [draw] {Text}
            rectangle (1,1)
            (node cs:name=x) -- +(1,1);
\end{codeexample}

Similarly, in the following examples both control points are $(1,1)$:

\begin{codeexample}[]
\tikz \draw (0,0) node(x) [draw] {X}
            (2,0) node(y) {Y}
            (node cs:name=x) .. controls +(1,1) and +(-1,1) ..
            (node cs:name=y);
\end{codeexample}

  The implicit way of specifying the node coordinate system is to
  simply use the name of the node in parentheses as in |(a)| or to
  specify a name together with an anchor or an angle separated by a
  dot as in |(a.north)| or |(a.10)|.

  Here is a more complete example:
\begin{codeexample}[]
\begin{tikzpicture}[fill=blue!20]
  \draw[help lines] (-1,-2) grid (6,3);
  \path (0,0)  node(a) [ellipse,rotate=10,draw,fill]    {An ellipse}
        (3,-1) node(b) [circle,draw,fill]               {A circle}
        (2,2)  node(c) [rectangle,rotate=20,draw,fill]  {A rectangle}
        (5,2)  node(d) [rectangle,rotate=-30,draw,fill] {Another rectangle};
  \draw[thick] (a.south) -- (b) -- (c) -- (d);
  \draw[thick,red,->] (a) |- +(1,3) -| (c) |- (b);       
  \draw[thick,blue,<->] (b) .. controls +(right:2cm) and +(down:1cm) .. (d);       
\end{tikzpicture}
\end{codeexample}
\end{coordinatesystem}


% Deprecated:
           
% \subsubsection{Intersection Coordinate Systems}

% Often you wish to specify a point that is on the
% intersection of two lines or shapes. For this, the following
% coordinate system is useful:

% \begin{coordinatesystem}{intersection}
%   First, you must specify two objects that should be
%   intersected. These ``objects'' can either be lines or the shapes of
%   nodes. There are two option to specify the first object:
%   \begin{key}{/tikz/cs/first line={\ttfamily\char`\{}|(|\meta{first
%           coordinate}|)--(|\meta{second coordinate}|)|{\ttfamily\char`\}}}
%     Specifies that the first object is a line that goes from
%     \meta{first coordinate} to meta{second coordinate}.
%   \end{key}
%   Note that you have to write |--| between the coordinate, but this
%   does not mean that anything is added to the path. This is simply a
%   special syntax.
%   \begin{key}{/tikz/cs/first node=\meta{node}}
%     Specifies that the first object is a previously defined node named
%     \meta{node}.
%   \end{key}
  
%   To specify the second object, you use one of the following keys:
%   \begin{key}{/tikz/cs/second line={\ttfamily\char`\{}|(|\meta{first
%           coordinate}|)--(|\meta{second coordinate}|)|{\ttfamily\char`\}}}
%     As above.
%   \end{key}
%   \begin{key}{/tikz/cs/second node=\meta{node}}
%     Specifies that the second object is a previously defined node
%     named \meta{node}.
%   \end{key}

%   Since it is possible that two objects have multiple intersections,
%   you may need to specify which solution you want:
%   \begin{key}{/tikz/cs/solution=\meta{number} (initially 1)}
%     Specifies which solution should be used. Numbering starts with 1.
%   \end{key}
%   The coordinate specified in this way is the \meta{number}th
%   intersection of the two objects.  If the objects do not intersect,
%   an error may occur.

% \begin{codeexample}[]
% \begin{tikzpicture}
%   \draw[help lines] (0,0) grid (3,2);
%   \draw (0,0) coordinate (A) -- (3,2) coordinate (B)
%         (1,2)                -- (3,0);

%   \fill[red] (intersection cs:
%     first line={(A)--(B)},
%     second line={(1,2)--(3,0)}) circle (2pt);
% \end{tikzpicture}
% \end{codeexample}

%   The implicit way of specifying this coordinate system is to write
%   \declare{|(intersection |\opt{\meta{number}}| of |\meta{first
%       object}%
%     | and |\meta{second object}|)|}. Here, \meta{first obejct} either
%   has the form \meta{$p_1$}|--|\meta{$p_2$} or it is just a node
%   name. Likewise for \meta{second object}. Note that there are \emph{no}
%   parentheses around the $p_i$. Thus, you would write
%   |(intersection of A--B and 1,2--3,0)|  for the intersection of the
%   line through the coordinates |A| and |B| and the line through the
%   points $(1,2)$ and $(3,0)$. You would write 
%   |(intersection 2 of c_1 and c_2)| for the second
%   intersection of the node named |c_1| and the node named
%   |c_2|.

%   \tikzname\ needs an explicit algorithm for computing the
%   intersection of two shapes and such an algorithm is available only
%   for few shapes. Currently, the following intersection will be
%   computed correctly:
%   \begin{itemize}
%   \item a line and a line
%   \item a |circle| node and a line (in any order)
%   \item a |circle| and a |circle|
%   \end{itemize}
% \begin{codeexample}[]
% \begin{tikzpicture}[scale=.25]
%   \coordinate [label=-135:$a$] (a) at ($ (0,0)   + (rand,rand) $);
%   \coordinate [label=45:$b$]   (b) at ($ (3,2) + (rand,rand) $);

%   \coordinate [label=-135:$u$] (u) at (-1,1);
%   \coordinate [label=45:$v$]   (v) at (6,0);

%   \draw (a) -- (b)
%         (u) -- (v);

%   \node (c1) at (a) [draw,circle through=(b)] {};
%   \node (c2) at (b) [draw,circle through=(a)] {};

%   \coordinate [label=135:$c$] (c) at (intersection 2 of c1 and c2);
%   \coordinate [label=-45:$d$] (d) at (intersection of u--v and c2);
%   \coordinate [label=135:$e$] (e) at (intersection of u--v and a--b);

%   \foreach \p in {a,b,c,d,e,u,v}
%     \fill [opacity=.5] (\p) circle (8pt);
% \end{tikzpicture}
% \end{codeexample}
% \end{coordinatesystem}

           
\subsubsection{Tangent Coordinate Systems}

\begin{coordinatesystem}{tangent}
  This coordinate system, which is available only when the \tikzname\
  library |calc| is loaded, allows you to compute the point that lies
  tangent to a shape. In detail, consider a \meta{node} and a
  \meta{point}. Now, draw a straight line from the \meta{point} so
  that it ``touches'' the \meta{node} (more formally, so that it is
  \emph{tangent} to this \meta{node}). The point where the line
  touches the shape is the point referred to by the |tangent|
  coordinate system.

  The following options may be given:
  \begin{key}{/tikz/cs/node=\meta{node}}
    This key specifies the node on whose border the tangent should
    lie. 
  \end{key}
  \begin{key}{/tikz/cs/point=\meta{point}}
    This key specifes the point through which the tangent should go.
  \end{key}
  \begin{key}{/tikz/cs/solution=\meta{number}}
    Specifies which solution should be used if there are more than one.
  \end{key}

  A special algorithm is needed in order to compute the tangent for a
  given shape. Currently, tangents can be computed for nodes whose
  shape is one of the following:
  \begin{itemize}
  \item |coordinate|
  \item |circle|
  \end{itemize}

\begin{codeexample}[]
\begin{tikzpicture}
  \draw[help lines] (0,0) grid (3,2);

  \coordinate (a) at (3,2);

  \node [circle,draw] (c) at (1,1) [minimum size=40pt] {$c$};
  
  \draw[red] (a)  -- (tangent cs:node=c,point={(a)},solution=1) --
       (c.center) -- (tangent cs:node=c,point={(a)},solution=2) -- cycle;
\end{tikzpicture}
\end{codeexample}

  There is no implicit syntax for this coordinate system.
\end{coordinatesystem}



\subsubsection{Defining New Coordinate Systems}

While the set of coordinate systems that \tikzname\ can parse via
their special syntax is fixed, it is possible and quite easy to define
new explicitly named coordinate systems. For this, the following
commands are used:

\begin{command}{\tikzdeclarecoordinatesystem\marg{name}\marg{code}}
  This command declares a new coordinate system named \meta{name} that
  can later on be used by writing
  |(|\meta{name}| cs:|\meta{arguments}|)|. When \tikzname\ encounters a coordinate
  specified in this way, the \meta{arguments} are passed to
  \meta{code} as argument |#1|.

  It is now the job of \meta{code} to make sense of the
  \meta{arguments}. At the end of \meta{code}, the two \TeX\ dimensions
  |\pgf@x| and |\pgf@y| should be have the $x$- and $y$-canvas
  coordinate of the coordinate.

  It is not necessary, but customary, to parse \meta{arguments} using
  the key-value syntax. However, you can also parse it in any way you
  like.

  In the following example, a coordinate system |cylindrical| is
  defined.
\begin{codeexample}[]
\makeatletter
\define@key{cylindricalkeys}{angle}{\def\myangle{#1}}    
\define@key{cylindricalkeys}{radius}{\def\myradius{#1}}    
\define@key{cylindricalkeys}{z}{\def\myz{#1}}
\tikzdeclarecoordinatesystem{cylindrical}%
{%
  \setkeys{cylindricalkeys}{#1}%
  \pgfpointadd{\pgfpointxyz{0}{0}{\myz}}{\pgfpointpolarxy{\myangle}{\myradius}}
}
\begin{tikzpicture}[z=0.2pt]
  \draw [->] (0,0,0) -- (0,0,350);
  \foreach \num in {0,10,...,350}
    \fill (cylindrical cs:angle=\num,radius=1,z=\num) circle (1pt);
\end{tikzpicture}
\end{codeexample}
\end{command}

\begin{command}{\tikzaliascoordinatesystem\marg{new name}\marg{old name}}
  Creates an alias of \meta{old name}.  
\end{command}



\subsection{Coordinates at Intersections}
\label{section-intersection-coordinates}

You will wish to compute the intersection of two paths. For the
special and frequent case of two perpendicular lines, a special
coordinate system called |perpendicular| is available. For more
general cases, the |intersection| library can be used.


\subsubsection{Intersections of Perpendicular Lines}

A frequent special case of path intersections is the intersection of a 
vertical line going through a point $p$ and a horizontal line going
through some other point $q$. For this situation there is a useful 
coordinate system.

\begin{coordinatesystem}{perpendicular}
  You can specify the two lines using the following keys:

  \begin{key}{/tikz/cs/horizontal line through={\ttfamily\char`\{}|(|\meta{coordinate}|)|{\ttfamily\char`\}}}
    Specifies that one line is a horizontal line that goes through the
    given coordinate.
  \end{key}
  \begin{key}{/tikz/cs/vertical line through={\ttfamily\char`\{}|(|\meta{coordinate}|)|{\ttfamily\char`\}}}
    Specifies that the other line is vertical and goes through the
    given coordinate.  
  \end{key}

  However, in almost all cases you should, instead, use the implicit
  syntax. Here, you write \declare{|(|\meta{p}\verb! |- !\meta{q}|)|} or
  \declare{|(|\meta{q}\verb! -| !\meta{p}|)|}.

  For example, \verb!(2,1 |- 3,4)! and  \verb!(3,4 -| 2,1)! both yield
  the same as \verb!(2,4)! (provided the $xy$-coordinate system has not
  been modified). 

  The most useful application of the syntax is to draw a line up to some
  point on a vertical or horizontal line. Here is an example:

\begin{codeexample}[]
\begin{tikzpicture}
  \path (30:1cm) node(p1) {$p_1$}   (75:1cm) node(p2) {$p_2$};

  \draw (-0.2,0) -- (1.2,0) node(xline)[right] {$q_1$};
  \draw (2,-0.2) -- (2,1.2) node(yline)[above] {$q_2$};

  \draw[->] (p1) -- (p1 |- xline);
  \draw[->] (p2) -- (p2 |- xline);
  \draw[->] (p1) -- (p1 -| yline);
  \draw[->] (p2) -- (p2 -| yline);
\end{tikzpicture}
\end{codeexample}
\end{coordinatesystem}


\subsubsection{Intersections of Arbitrary Paths}

\begin{tikzlibrary}{intersections}
  This library enables the calculation of intersections of
  two arbitrary paths. However, due to the low accuracy of
  \TeX, the paths should not be ``too complicated''.
  In particular, you should not try to intersect paths consisting 
  lots of very small segments such as plots or decorated paths.
\end{tikzlibrary}

To find the intersections of two paths in \tikzname, they must be
``named''. A ``named path'' is, quite simply, a path that has been 
named using the following key:
  
\begin{keylist}{%
	/tikz/name path=\meta{name},
	/tikz/name path global=\meta{name}}
  The effect of this key is that, after the path has been constructed, 
  just before it is used, it is associated with \meta{name}. For |name path|,
  this association survives beyond the final semi-colon of the path 
  but not the end of the surrounding scope. For |name path global|, the association
  will survive beyond any scope as well. Handle with care.
  
  Any paths created by nodes on the (main) path are ignored, unless
  this key is explicitly used. If the same \meta{name} is used for the
  main path and the node path(s), then the paths will be added
  together and then associated with \meta{name}.
\end{keylist}

To find the intersection of named paths, the following key is used:

\begin{key}{/tikz/name intersections=\marg{options}}
  This key changes the key path to |/tikz/intersection| and processes
  \meta{options}. These options determine, among other things,
  which paths to use for the intersection. Having processed the 
  options, any intersections are then found. A coordinate is created 
  at each intersection, which by default, will be named 
  |intersection-1|, |intersection-2|, and so on. 
  Optionally, the prefix |intersection| can be changed, and the 
  total number of intersections stored in a \TeX-macro. 

\begin{codeexample}[]
\begin{tikzpicture}[every node/.style={opacity=1, black, above left}]
  \draw [help lines] grid (3,2);
  \draw [name path=ellipse] (2,0.5) ellipse (0.75cm and 1cm);
  \draw [name path=rectangle, rotate=10] (0.5,0.5) rectangle +(2,1);
  \fill [red, opacity=0.5, name intersections={of=ellipse and rectangle}]
    (intersection-1) circle (2pt) node {1}
    (intersection-2) circle (2pt) node {2};
\end{tikzpicture}
\end{codeexample}

The following keys can be used in \meta{options}:
  
\begin{key}{/tikz/intersection/of=\meta{name path 1}| and |\meta{name path 2}}
  This key is used to specify the names of the paths to use for
  the intersection.
\end{key}

\begin{key}{/tikz/intersection/name=\meta{prefix} (initially intersection)}
  This key specifies the prefix name for the coodinate nodes placed
  at each intersection.
\end{key}

\begin{key}{/tikz/intersection/total=\meta{macro}}
  This key will mean than the total number of intersections found
  will be stored in \meta{macro}.
\end{key}

\begin{codeexample}[]
\begin{tikzpicture}
  \clip (-2,-2) rectangle (2,2);
  \draw [name path=curve 1] (-2,-1) .. controls (8,-1) and (-8,1) .. (2,1);
  \draw [name path=curve 2] (-1,-2) .. controls (-1,8) and (1,-8) .. (1,2);
  
  \fill [name intersections={of=curve 1 and curve 2, name=i, total=\t}]
        [red, opacity=0.5, every node/.style={above left, black, opacity=1}] 
        \foreach \s in {1,...,\t}{(i-\s) circle (2pt) node {\footnotesize\s}};
\end{tikzpicture}
\end{codeexample}

  
  \begin{key}{/tikz/intersection/by=\meta{comma-separated list}}
    This key allows you to specify a list of names for the intersection
    coordinates. The intersection coordinates will still be named
    \meta{prefix}|-|\meta{number}, but additionally the first
    coordinate will also be named by the first element of the
    \meta{comma-separated list}. What happens is that the
    \meta{comma-separated list} is passed to the |\foreach| statement
    and for \meta{list member} a coordinate is created at the
    already-named intersection.
\begin{codeexample}[]
\begin{tikzpicture}
  \clip (-2,-2) rectangle (2,2);
  \draw [name path=curve 1] (-2,-1) .. controls (8,-1) and (-8,1) .. (2,1);
  \draw [name path=curve 2] (-1,-2) .. controls (-1,8) and (1,-8) .. (1,2);
  
  \fill [name intersections={of=curve 1 and curve 2, by={a,b}}]
        (a) circle (2pt)
        (b) circle (2pt);
\end{tikzpicture}
\end{codeexample}    

    You can also use the |...| notation of the |\foreach| statement
    inside the \meta{comma-separated list}.

    In case an element of the \meta{comma-separated list} starts with
    options in square brackets, these options are used when the
    coordinate is created. A coordinate name can still, but need not,
    follow the  options. This
    makes it easy to add labels to intersections: 
\begin{codeexample}[]
\begin{tikzpicture}
  \clip (-2,-2) rectangle (2,2);
  \draw [name path=curve 1] (-2,-1) .. controls (8,-1) and (-8,1) .. (2,1);
  \draw [name path=curve 2] (-1,-2) .. controls (-1,8) and (1,-8) .. (1,2);
  
  \fill [name intersections={
          of=curve 1 and curve 2,
          by={[label=center:a],[label=center:...],[label=center:i]}}];
\end{tikzpicture}
\end{codeexample}
  \end{key}

  \begin{key}{/tikz/intersection/sort by=\meta{path name}}
By default, the intersections are simply returned in the order that 
the intersection algorithm finds them. Unfortunately, this is not 
necessarily a ``helpful'' ordering. This key can be used to sort
the intersections along the path specified by \meta{path name},
which should be one of the paths mentioned in the 
|/tikz/intersection/of| key.

\begin{codeexample}[]
\begin{tikzpicture}
\clip (-0.5,-0.75) rectangle (3.25,2.25);
\foreach \pathname/\shift in {line/0cm, curve/2cm}{
  \tikzset{xshift=\shift}
  \draw [->, name path=curve] (1,1.5) .. controls (-1,1) and (2,0.5) .. (0,0);
  \draw [->, name path=line]  (0,-.5) -- (1,2) ;
  \fill [name intersections={of=line and curve,sort by=\pathname, name=i}]
    [red, opacity=0.5, every node/.style={left=.25cm, black, opacity=1}]
    \foreach \s in {1,2,3}{(i-\s) circle (2pt) node {\footnotesize\s}};
}
\end{tikzpicture}
\end{codeexample}

  \end{key}
\end{key}




\subsection{Relative and Incremental Coordinates}


\subsubsection{Specifying Relative Coordinates}

You can prefix coordinates by |++| to make them ``relative.'' A
coordinate such as |++(1cm,0pt)| means ``1cm to the right of the
previous position.'' Relative coordinates are often useful in
``local'' contexts:

\begin{codeexample}[]
\begin{tikzpicture}
  \draw (0,0)     -- ++(1,0) -- ++(0,1) -- ++(-1,0) -- cycle;
  \draw (2,0)     -- ++(1,0) -- ++(0,1) -- ++(-1,0) -- cycle;
  \draw (1.5,1.5) -- ++(1,0) -- ++(0,1) -- ++(-1,0) -- cycle;
\end{tikzpicture}
\end{codeexample}

Instead of |++| you can also use a single |+|. This also specifies a
relative coordinate, but it does not ``update'' the current point for
subsequent usages of relative coordinates. Thus, you can use this
notation to specify numerous points, all relative to the same
``initial'' point:

\begin{codeexample}[]
\begin{tikzpicture}
  \draw (0,0)     -- +(1,0) -- +(1,1) -- +(0,1) -- cycle;
  \draw (2,0)     -- +(1,0) -- +(1,1) -- +(0,1) -- cycle;
  \draw (1.5,1.5) -- +(1,0) -- +(1,1) -- +(0,1) -- cycle;
\end{tikzpicture}
\end{codeexample}

There is a special situation, where relative coordinates are
interpreted differently. If you use a relative coordinate as a control
point of a B�zier curve, the following rule applies: First, a relative
first control point is taken relative to the beginning of the
curve. Second, a relative second control point is taken relative to
the end of the curve. Third, a relative end point of a curve is taken
relative to the start of the curve.

This special behavior makes it easy to specify that a curve should
``leave or arrives from a certain direction'' at the start or end. In
the following example, the curve ``leaves'' at $30^\circ$ and
``arrives'' at $60^\circ$: 

\begin{codeexample}[]
\begin{tikzpicture}
  \draw (1,0) .. controls +(30:1cm) and +(60:1cm) .. (3,-1);
  \draw[gray,->] (1,0) -- +(30:1cm);
  \draw[gray,<-] (3,-1) -- +(60:1cm);
\end{tikzpicture}
\end{codeexample}


\subsubsection{Relative Coordinates and Scopes}
\label{section-scopes-relative}
An interesting question is, how do relative coordinates behave in the
presence of scopes? That is, suppose we use curly braces in a path to
make part of it ``local,'' how does that affect the current position?
On the one hand, the current position certainly changes since the
scope only affects options, not the path itself. On the other hand, it
may be useful to ``temporarily escape'' from the updating of the
current point.

Since both interpretations of how the current point and scopes should
``interact'' are useful, there is a (local!) option that allows you to
decide which you need.

\begin{key}{/tikz/current point is local=\opt{\meta{boolean}} (initially
    false)}
  Normally, the scope path operation has no effect on the current
  point. That is, curly braces on a path have no effect on the current
  position:
\begin{codeexample}[]
\begin{tikzpicture}
  \draw      (0,0) -- ++(1,0)   -- ++(0,1)   -- ++(-1,0);
  \draw[red] (2,0) -- ++(1,0) { -- ++(0,1) } -- ++(-1,0);
\end{tikzpicture}
\end{codeexample}
  If you set this key to |true|, this behaviour changes. In this case,
  at the end of a group created on a path, the last current position
  reverts to whatever value it had at the beginning of the scope. More
  precisely, when \tikzname\ encounters |}| on a path, it checks
  whether at this particular moment the key is set to |true|. If so,
  the current position reverts to the value is had when the matching
  |{| was read.
\begin{codeexample}[]
\begin{tikzpicture}
  \draw      (0,0) -- ++(1,0)   -- ++(0,1)   -- ++(-1,0);
  \draw[red] (2,0) -- ++(1,0)
     { [current point is local] -- ++(0,1) } -- ++(-1,0);
\end{tikzpicture}
\end{codeexample}  
  In the above example, we could also have given the option outside
  the scope, for instance as a parameter to the whole scope.
\end{key}


\subsection{Coordinate Calculations}

\begin{tikzlibrary}{calc}
  You need to load this library in order to use the coordinate
  calculation functions described in the present section.
\end{tikzlibrary}


It is possible to do some basic calculations that involve
coordinates. In essence, you can add and subtract coordinates, scale
them, compute midpoints, and do projections. For instance,
|($(a) + 1/3*(1cm,0)$)| is the coordinate that is $1/3$cm to the right
of the point |a|:
\begin{codeexample}[]
\begin{tikzpicture}
  \draw [help lines] (0,0) grid (3,2);

  \node (a) at (1,1) {A};
  \fill [red] ($(a) + 1/3*(1cm,0)$) circle (2pt);
\end{tikzpicture}
\end{codeexample}



\subsubsection{The General Syntax}

The general syntax is the following:

\begin{quote}
  \declare{|(|\opt{|[|\meta{options}|]|}|$|\meta{coordinate computation}|$)|}. 
\end{quote}

As you can see, the syntax uses the \TeX\ math symbol |$| to %$
indicate that a ``mathematical computation'' is involved. However, the |$| %$
has no other effect, in particular, no mathematical text is typeset.

The \meta{coordinate computation} has the following structure:
\begin{enumerate}
\item
  It starts with
  \begin{quote}
    \opt{\meta{factor}|*|}\meta{coordinate}\opt{\meta{modifiers}} 
  \end{quote}
\item
  This is optionally followed by |+| or |-| and then another
  \begin{quote}
    \opt{\meta{factor}|*|}\meta{coordinate}\opt{\meta{modifiers}} 
  \end{quote}
\item
  This is once more followed by |+| or |-| and another of the above
  modified coordinate; and so on.
\end{enumerate}

In the following, the syntax of factors and of the different modifiers
is explained in detail.


\subsubsection{The Syntax of Factors}

The \meta{factor}s are optional and detected
by checking whether the \meta{coordinate computation} starts with a
|(|. Also, after each $\pm$ a \meta{factor} is present if, and only
if, the |+| or |-| sign is not directly followed by~|(|.

If a \meta{factor} is present, it is evaluated using the
|\pgfmathparse| macro. This means that you can use pretty complicated
computations inside a factor. A \meta{factor} may even contain opening
parentheses, which creates a complication: How does \tikzname\ know
where a \meta{factor} ends and where a coordinate starts? For
instance, if the beginning of a \meta{coordinate computation} is
|2*(3+4|\dots, it is not clear whether |3+4| is part of a
\meta{coordinate} or part of a \meta{factor}. Because of this, the
following rule is used: Once it has been determined, that a
\meta{factor} is present, in principle, the \meta{factor} contains
everything up to the next occurrence of |*(|. Note that there is no
space between the asterisk and the parenthesis.

It is permissible to put the \meta{factor} is curly braces. This can
be used whenever it is unclear where the \meta{factor} would end. 

Here are some examples of coordinate specifications that consist of
exactly one \meta{factor} and one \meta{coordinate}:
\begin{codeexample}[]
\begin{tikzpicture}
  \draw [help lines] (0,0) grid (3,2);

  \fill [red] ($2*(1,1)$) circle (2pt);
  \fill [green] (${1+1}*(1,.5)$) circle (2pt);
  \fill [blue] ($cos(0)*sin(90)*(1,1)$) circle (2pt);
  \fill [black] (${3*(4-3)}*(1,0.5)$) circle (2pt);
\end{tikzpicture}
\end{codeexample}



\subsubsection{The Syntax of Partway Modifiers}

A \meta{coordinate} can be followed by different \meta{modifiers}. The
first kind of modifier is the \emph{partway modifier}. The syntax
(which is loosely inspired by Uwe Kern's |xcolor| package) is the
following:
\begin{quote}
  \meta{coordinate}\declare{|!|\meta{number}|!|\opt{\meta{angle}|:|}\meta{second coordinate}}
\end{quote}
One could write for instance
\begin{codeexample}[code only]
(1,2)!.75!(3,4)
\end{codeexample}
The meaning of this is: ``Use the coordinate that is three quarters on
the way from |(1,2)| to |(3,4)|.'' In general, \meta{coordinate
  x}|!|\meta{number}|!|\meta{coordinate y} yields the coordinate
$(1-\meta{number})\meta{coordinate x} + \meta{number} \meta{coordinate
  y}$. Note that this is a bit different from the way the
\meta{number} is interpreted in the |xcolor| package: First, you use a
factor between $0$ and $1$, not a percentage, and, second, as the
\meta{number} approaches $1$, we approach the second coordinate, not
the first. It is permissible to use \meta{numbers} that are smaller
than $0$ or larger than $1$. The \meta{number} is evaluated using the
|\pgfmathparse| command and, thus, it can involve complicated
computations. 

\begin{codeexample}[]
\begin{tikzpicture}
  \draw [help lines] (0,0) grid (3,2);

  \draw (1,0) -- (3,2);
  
  \foreach \i in {0,0.2,0.5,0.9,1}
    \node at ($(1,0)!\i!(3,2)$) {\i};
\end{tikzpicture}
\end{codeexample}

The \meta{second coordinate} may be prefixed by an \meta{angle},
separated with a colon, as in |(1,1)!.5!60:(2,2)|. The general meaning
of \meta{a}|!|\meta{factor}|!|\meta{angle}|:|\meta{b} is ``First,
consider the line from \meta{a} to \meta{b}. Then rotate this line by
\meta{angle} \emph{around the point \meta{a}}. Then the two endpoints
of this line will be \meta{a} and some point \meta{c}. Use this point
\meta{c} for the subsequent computation, namely the partway
computation.''

Here are two examples:
\begin{codeexample}[]
\begin{tikzpicture}
  \draw [help lines] (0,0) grid (3,3);

  \coordinate (a) at (1,0);
  \coordinate (b) at (3,2);

  \draw[->] (a) -- (b);

  \coordinate (c) at ($ (a)!1! 10:(b) $);

  \draw[->,red] (a) -- (c);

  \fill ($ (a)!.5! 10:(b) $) circle (2pt);
\end{tikzpicture}
\end{codeexample}


\begin{codeexample}[]
\begin{tikzpicture}
  \draw [help lines] (0,0) grid (4,4);

  \foreach \i in {0,0.1,...,2}
    \fill ($(2,2) !\i! \i*180:(3,2)$) circle (2pt);
\end{tikzpicture}
\end{codeexample}


You can repeatedly apply modifiers. That is, after any modifier
you can add another (possibly different) modifier.

\begin{codeexample}[]
\begin{tikzpicture}
  \draw [help lines] (0,0) grid (3,2);

  \draw (0,0) -- (3,2);
  \draw[red] ($(0,0)!.3!(3,2)$) -- (3,0);
  \fill[red] ($(0,0)!.3!(3,2)!.7!(3,0)$) circle (2pt);
\end{tikzpicture}
\end{codeexample}


\subsubsection{The Syntax of Distance Modifiers}

A \emph{distance modifier} has nearly the same syntax as a partway
modifier, only you use a \meta{dimension} (something like |1cm|)
instead of a \meta{factor} (something like |0.5|):
\begin{quote}
  \meta{coordinate}\declare{|!|\meta{dimension}|!|\opt{\meta{angle}|:|}\meta{second coordinate}}
\end{quote}

When you write \meta{a}|!|\meta{dimension}|!|\meta{b}, this means the
following: Use the point that is distanced \meta{dimension} from
\meta{a} on the straight line from \meta{a} to \meta{b}. Here is an example:
\begin{codeexample}[]
\begin{tikzpicture}
  \draw [help lines] (0,0) grid (3,2);

  \draw (1,0) -- (3,2);
  
  \foreach \i in {0cm,1cm,15mm}
    \node at ($(1,0)!\i!(3,2)$) {\i};
\end{tikzpicture}
\end{codeexample}

As before, if you use a \meta{angle}, the \meta{second coordinate} is
rotated by this much around the \meta{coordinate} before it is used.

The combination of an \meta{angle} of |90| degrees with a distance can
be used to ``offset'' a point relative to a line. Suppose, for
instance, that you have computed a point |(c)| that lies somewhere on
a line from |(a)| to~|(b)| and you now wish to offset this point by
|1cm| so that the distance from this offset point to the line is
|1cm|. This can be achieved as follows:
\begin{codeexample}[]
\begin{tikzpicture}
  \draw [help lines] (0,0) grid (3,2);

  \coordinate (a) at (1,0);
  \coordinate (b) at (3,1);

  \draw (a) -- (b);

  \coordinate (c) at ($ (a)!.25!(b) $);
  \coordinate (d) at ($ (c)!1cm!90:(b) $);

  \draw [<->] (c) -- (d) node [sloped,midway,above] {1cm};
\end{tikzpicture}
\end{codeexample}



\subsubsection{The Syntax of Projection Modifiers}

The projection modifier is also similar to the above modifiers: It also
gives a point on a line from the \meta{coordinate} to the \meta{second
  coordinate}. However, the \meta{number} or \meta{dimension} is replaced by a
\meta{projection coordinate}:
\begin{quote}
  \meta{coordinate}\declare{|!|\meta{projection coordinate}|!|\opt{\meta{angle}|:|}\meta{second coordinate}}
\end{quote}

Here is an example:
\begin{codeexample}[code only]
(1,2)!(0,5)!(3,4)
\end{codeexample}

The effect is the following: We project the \meta{projection
  coordinate} orthogonally onto to the line from \meta{coordinate} to
\meta{second coordinate}. This makes it easy to compute projected
points: 
\begin{codeexample}[]
\begin{tikzpicture}
  \draw [help lines] (0,0) grid (3,2);

  \coordinate (a) at (0,1);
  \coordinate (b) at (3,2);
  \coordinate (c) at (2.5,0);

  \draw (a) -- (b) -- (c) -- cycle;

  \draw[red]    (a) -- ($(b)!(a)!(c)$);
  \draw[orange] (b) -- ($(a)!(b)!(c)$);
  \draw[blue]   (c) -- ($(a)!(c)!(b)$);
\end{tikzpicture}
\end{codeexample}

% Copyright 2005 by Till Tantau <tantau@cs.tu-berlin.de>.
%
% This program can be redistributed and/or modified under the terms
% of the LaTeX Project Public License Distributed from CTAN
% archives in directory macros/latex/base/lppl.txt.


\section{Syntax for Path Specifications}

A \emph{path} is a series of straight and curved line segments. It is
specified following a |\path| command and the specification must
follow a special syntax, which is described in the subsections of the
present section.


\begin{command}{\path\meta{specification}|;|}
  This command is available only inside a |{tikzpicture}| environment.

  The \meta{specification} is a long stream of \emph{path
  operations}. Most of these path operations tell \tikzname\ how the path
  is build. For example, when you write |--(0,0)|, you use a
  \emph{line-to operation} and it means ``continue the path from
  wherever you are to the origin.''

  At any point where \tikzname\ expects a path operation, you can also
  give some graphic options, which is a list of options in brackets,
  such as |[rounded corners]|. These options can have different
  effects:
  \begin{enumerate}
  \item
    Some options take ``immediate'' effect and apply to all subsequent
    path operations on the path. For example, the |rounded corners|
    option will round all following corners, but not the corners
    ``before'' and if the |sharp corners| is given later on the path
    (in a new set of brackets), the rounding effect will end.

\begin{codeexample}[]
\tikz \draw (0,0) -- (1,1)
           [rounded corners] -- (2,0) -- (3,1)
           [sharp corners] -- (3,0) -- (2,1);
\end{codeexample}
    Another example are the transformation options, which also apply
    only to subsequent coordinates.
  \item
    The options that have immediate effect can be ``scoped'' by
    putting part of a path in curly braces. For example, the above
    example could also be written as follows:

\begin{codeexample}[]
\tikz \draw (0,0) -- (1,1)
           {[rounded corners] -- (2,0) -- (3,1)}
           -- (3,0) -- (2,1);
\end{codeexample}
  \item
    Some options only apply to the path as a whole. For example, the
    |color=| option for determining the color used for, say, drawing
    the path always applies to all parts of the path. If several
    different colors are given for different parts of the path, only
    the last one (on the outermost scope) ``wins'':
 
\begin{codeexample}[]
\tikz \draw (0,0) -- (1,1)
           [color=red] -- (2,0) -- (3,1)
           [color=blue] -- (3,0) -- (2,1);
\end{codeexample}

    Most options are of this type. In the above example, we would have
    had to ``split up'' the path into several |\path| commands:
\begin{codeexample}[]
\tikz{\draw (0,0) -- (1,1);
      \draw [color=red] (2,0) -- (3,1);
      \draw [color=blue] (3,0) -- (2,1);}
\end{codeexample}
  \end{enumerate}

  By default, the |\path| command does ``nothing'' with the
  path, it just ``throws it away.'' Thus, if you write
  |\path(0,0)--(1,1);|, nothing is drawn 
  in your picture. The only effect is that the area occupied by the
  picture is (possibly) enlarged so that the path fits inside the
  area. To actually ``do'' something with the path, an option like
  |draw| or |fill| must be given somewhere on the path. Commands like
  |\draw| do this implicitly.
  
  Finally, it is also possible to give \emph{node specifications} on a
  path. Such specifications can come at different locations, but they
  are always allowed when a normal path operation could follow. A node
  specification starts with |node|. Basically, the effect is to
  typeset the node's text as normal \TeX\ text and to place
  it at the ``current location'' on the path. The details are explained
  in Section~\ref{section-nodes}.

  Note, however, that the nodes are \emph{not} part of the path in any
  way. Rather, after everything has been done with the path what is
  specified by the path options (like filling and drawing the path due
  to a |fill| and a |draw| option somewhere in the
  \meta{specification}), the nodes are added in a post-processing
  step.   
  
  The following style influences scopes:
  \begin{itemize}
    \itemstyle{every path}
    This style is installed at the beginning of every path. This can
    be useful for (temporarily) adding, say, the |draw| option to
    everything in a scope.
\begin{codeexample}[]
\begin{tikzpicture}[fill=examplefill] % only sets the color
  \tikzstyle{every path}=[draw]           % all paths are drawn
  \fill  (0,0) rectangle +(1,1);
  \shade (2,0) rectangle +(1,1);
\end{tikzpicture}
\end{codeexample}
  \end{itemize}
\end{command}




\subsection{The Move-To Operation}

The perhaps simplest operation is the move-to operation, which is
specified by just giving a coordinate where a path operation is
expected.

\begin{pathoperation}[noindex]{}{\meta{coordinate}}
  \index{empty@\protect\meta{empty} path operation}%
  \index{Path operations!empty@\protect\texttt{\meta{empty}}}%
  The move-to operation normally starts a path at a certain
  point. This does not cause a line segment to be created, but it  
  specifies the starting point of the next segment. If a path is
  already under construction, that is, if several segments have
  already been created, a move-to operation will start a new part of the
  path that is not connected to any of the previous segments.

\begin{codeexample}[]
\begin{tikzpicture}
  \draw (0,0) --(2,0) (0,1) --(2,1);
\end{tikzpicture}
\end{codeexample}

  In the specification |(0,0) --(2,0) (0,1) --(2,1)| two move-to
  operations are specified: |(0,0)| and |(0,1)|. The other two
  operations, namely |--(2,0)| and |--(2,1)| are line-to operations,
  described next.
\end{pathoperation}


\subsection{The Line-To Operation}


\subsubsection{Straight Lines}

\begin{pathoperation}{--}{\meta{coordinate}}
  The line-to operation extends the current path from the current
  point in a straight line to the given coordinate. The ``current
  point'' is the endpoint of the previous drawing operation or the point
  specified by a prior move-to operation.

  You use two minus signs followed by a coordinate in round
  brackets. You can add spaces before and after the~|--|.

  When a line-to operation is used and some path segment has just been
  constructed, for example by another line-to operation, the two line
  segments become joined. This means that if they are drawn, the point
  where they meet is ``joined'' smoothly. To appreciate the difference,
  consider the following two examples: In the left example, the path
  consists of two path segments that are not joined, but that happen to
  share a point, while in the right example a smooth join is shown.

\begin{codeexample}[]
\begin{tikzpicture}[line width=10pt]
  \draw (0,0) --(1,1)  (1,1) --(2,0);
  \draw (3,0) -- (4,1) -- (5,0);
  \useasboundingbox (0,1.5); % make bounding box higher
\end{tikzpicture}
\end{codeexample}

\end{pathoperation}


\subsubsection{Horizontal and Vertical Lines}

Sometimes you want to connect two points via straight lines that are
only horizontal and vertical. For this, you can use two path
construction operations.

{\catcode`\|=12
\begin{pathoperation}[noindex]{-|}{\meta{coordinate}}
  \index{--1@\protect\texttt{-\protect\pgfmanualbar} path operation}%
  \index{Path operations!--1@\protect\texttt{-\protect\pgfmanualbar}}%
  This operation means ``first horizontal, then vertical.''

  \begin{codeexample}[]
\begin{tikzpicture}
  \draw (0,0) node(a) [draw] {A}  (1,1) node(b) [draw] {B};
  \draw (a.north) |- (b.west);
  \draw[color=red] (a.east) -| (2,1.5) -| (b.north);
\end{tikzpicture}
\end{codeexample}
\end{pathoperation}
\begin{pathoperation}[noindex]{|-}{\meta{coordinate}}
  \index{--2@\protect\texttt{\protect\pgfmanualbar-} path operation}%
  \index{Path operations!--2@\protect\texttt{\protect\pgfmanualbar-}}%
  This operations means  ``first vertical, then horizontal.''
\end{pathoperation}
}


\subsubsection{Snaked Lines}
\label{section-tikz-snakes}

The line-to operation can not only be used to append straight lines to
the path, but also ``snaked'' lines (called thus because they look a
little bit like snakes seen from above).

\tikzname\ and \pgfname\ use a concept that I termed \emph{snakes}
for appending such ``squiggly'' lines. A snake specifies a way of
extending a path between two points in a ``fancy manner.''

Normally, a snake will just connect the start point to the end point
without starting new subpaths. Thus, a path containing a snaked line
can, nevetheless, still be used for filling. However, this is not
always the case. Some snakes consist of numerous unconnected
segments. ``Lines'' consisting of such snakes cannot be used as the
borders of enclosed areas.

Here are some examples of snakes in action:

\begin{codeexample}[]
\begin{tikzpicture}[thick]
  \draw                                        (0,3)   -- (3,3);
  \draw[snake=zigzag]                          (0,2.5) -- (3,2.5);
  \draw[snake=brace]                           (0,2)   -- (3,2);
  \draw[snake=triangles]                       (0,1.5) -- (3,1.5);
  \draw[snake=coil,segment length=4pt]         (0,1)   -- (3,1);
  \draw[snake=coil,segment aspect=0]           (0,.5)  -- (3,.5);
  \draw[snake=expanding waves,segment angle=7] (0,0)   -- (3,0);
\end{tikzpicture}
\end{codeexample}

\begin{codeexample}[]
\begin{tikzpicture}
  \filldraw[fill=red!20,snake=bumps] (0,0) rectangle (3,2);
\end{tikzpicture}
\end{codeexample}

\begin{codeexample}[]
\begin{tikzpicture}
  \filldraw[fill=blue!20]              (0,3)
  [snake=saw]                       -- (3,3)
  [snake=coil,segment aspect=0]     -- (2,1)
  [snake=bumps]                     -| (0,3);
\end{tikzpicture}
\end{codeexample}

No special path operation is needed to use a snake. Instead, you use
the following option to ``switch on'' snaking:

\begin{itemize}
  \itemoption{snake}\opt{|=|\meta{snake name}}
  This option causes the snake \meta{snake name} to be used for
  subsequent line-to operations. So, whenever you use the |--| syntax
  to specify that a straight line should be added to the path, a snake
  to this path will be added instead. Snakes will also be used when
  you use the \verb!-|! and \verb!|-! syntax and also when you use the
  |rectangle| operation. Snakes will \emph{not} be used when you use
  the curve-to operation nor when any other ``curved'' line is added
  to the path.

  This option has to be given anew for each path. However, you can
  also leave out the \meta{snake name}. In this case, the enclosing
  scope's \meta{snake name} is used. Thus, you can specify a
  ``standard'' snake name for scope and then just say |\draw[snake]|
  every time this snake should actually be used.

  The \meta{snake name} |none| is special. It can be used to switch
  off snaking after it has been switched on on a path.

  A bit strangely, no valid \meta{snake names} are defined by
  \tikzname\ by default. Instead, you have to include the library
  package |pgflibrarysnakes|. This package defines numerous snakes,
  see Section~\ref{section-library-snakes} for the complete list.
\end{itemize}

Most snakes can be configured. For example, for a snake that looks
like a sine curve, you might wish to change the amplitude or the
frequency. There are numerous options that influence these
parameters. Not all options apply to all snakes, see
Section~\ref{section-library-snakes} once more for details.

\begin{itemize}
  \itemoption{gap before snakes}|=|\meta{dimension}
  This option allows you to add a certain ``gap'' to the snake at its
  beginning. The snake will not start at the current point; instead
  the start point of the snake is move be \meta{dimension} in the
  direction of the target.
\begin{codeexample}[]
\begin{tikzpicture}
  \draw[help lines] (0,0) grid (3,2);
  \draw[snake=zigzag]                      (0,1) -- ++(3,1);
  \draw[snake=zigzag,gap before snake=1cm] (0,0) -- ++(3,1);
\end{tikzpicture}
\end{codeexample}
  \itemoption{gap after snake}|=|\meta{dimension}
  This option has the same effect as |gap before snake|, only it
  affects the end of the snake, which will ``end early.''
  \itemoption{gap around snake}|=|\meta{dimension}
  This option sets the gap before and after the gap to
  \meta{dimension}. 
\begin{codeexample}[]
\begin{tikzpicture}
  \draw[help lines] (0,0) grid (3,2);
  \draw[snake=brace]                      (0,1) -- ++(3,1);
  \draw[snake=brace,gap around snake=5mm] (0,0) -- ++(3,1);
\end{tikzpicture}
\end{codeexample}
  \itemoption{line before snake}|=|\meta{dimension}
  This option works like |gap before snake|, only it will connect the
  current point with a straight line to the start of the snake.
\begin{codeexample}[]
\begin{tikzpicture}
  \draw[help lines] (0,0) grid (3,2);
  \draw[snake=zigzag]                       (0,1) -- ++(3,1);
  \draw[snake=zigzag,line before snake=1cm] (0,0) -- ++(3,1);
\end{tikzpicture}
\end{codeexample}
  \itemoption{line after snake}|=|\meta{dimension}
  Works line |gap after snake|, only it adds a straight line.
  \itemoption{line around snake}|=|\meta{dimension}
  Works line |gap around snake|, only it adds straight lines.
  \itemoption{raise snake}|=|\meta{dimension}
  This option can be used with all snakes. It will offset the snake by
  ``raising'' it by \meta{dimension}. A negative \meta{dimension} will
  lower the snake. Raising and lowering is always relative to the line
  along which the snake is drawn. Here is an example:
\begin{codeexample}[]
\begin{tikzpicture}
  \node (a) {A};
  \node (b) at (2,1) {B};
  \draw                                  (a) -- (b);
  \draw[snake=brace]                     (a) -- (b);
  \draw[snake=brace,raise snake=5pt,red] (a) -- (b);
\end{tikzpicture}
\end{codeexample}
  \itemoption{mirror snake}
  This option causes the snake to be ``reflected along the path.''
  This is best understood by looking at an example:
\begin{codeexample}[]
\begin{tikzpicture}
  \node (a) {A};
  \node (b) at (2,1) {B};
  \draw                                     (a) -- (b);
  \draw[snake=brace]                        (a) -- (b);
  \draw[snake=brace,mirror snake,red,thick] (a) -- (b);
\end{tikzpicture}
\end{codeexample}
  This option can be used with every snake and can be combined with
  the |raise snake| option.
  \itemoption{segment amplitude}|=|\meta{dimension}
  This option sets the ``amplitude'' of the snake. For a snake that is
  a sine wave this would be the amplitude of this line. For other
  snakes this value typically describes how far the snakes ``rises
  above'' or ``falls below'' the path. For some snakes, this value is
  ignored. 
\begin{codeexample}[]
\begin{tikzpicture}
  \node (a) {A}   node (b) at (2,1) {B}  node (c) at (2,-1) {C};
  \draw[snake=zigzag]                                 (a) -- (b);
  \draw[snake=zigzag,segment amplitude=5pt,red,thick] (a) -- (c);
\end{tikzpicture}
\end{codeexample}
  \itemoption{segment length}|=|\meta{dimension}
  This option sets the length of each ``segment'' of a snake. For a
  sine wave this would be the wave length, for other snakes it is the
  length of each ``repetitive part'' of the snake.
\begin{codeexample}[]
\begin{tikzpicture}
  \node (a) {A}   node (b) at (2,1) {B}  node (c) at (2,-1) {C};
  \draw[snake=zigzag]                               (a) -- (b);
  \draw[snake=zigzag,segment length=20pt,red,thick] (a) -- (c);
\end{tikzpicture}
\end{codeexample}
\begin{codeexample}[]
\begin{tikzpicture}
  \node (a) {A}   node (b) at (2,1) {B}  node (c) at (2,-1) {C};
  \draw[snake=bumps]                               (a) -- (b);
  \draw[snake=bumps,segment length=20pt,red,thick] (a) -- (c);
\end{tikzpicture}
\end{codeexample}
  \itemoption{segment object length}|=|\meta{dimension}
  This option sets the length of the objects inside each segment of a
  snake. This option is only used for snakes in which each segment
  contains an object like a triangle or a star. 
\begin{codeexample}[]
\begin{tikzpicture}
  \node (a) {A}   node (b) at (2,1) {B}  node (c) at (2,-1) {C};
  \draw[snake=triangles]                                     (a) -- (b);
  \draw[snake=triangles,segment object length=8pt,red,thick] (a) -- (c);
\end{tikzpicture}
\end{codeexample}
  \itemoption{segment angle}|=|\meta{degrees}
  This option sets an angle that is interpreted in a snake-specific
  way. For example, the |waves| and |expanding waves| snakes interpret
  this as (half the) opening angle of the wave. The |border| snake
  uses this value for the angle of the little ticks.
\begin{codeexample}[]
\begin{tikzpicture}[segment amplitude=10pt]
  \node (a) {A}   node (b) at (2,0) {B};
  \draw[snake=border]                            (a) -- (b);
  \draw[snake=border,segment angle=20,red,thick] (a) -- (b);
\end{tikzpicture}
\end{codeexample}
\begin{codeexample}[]
\begin{tikzpicture}[segment amplitude=10pt]
  \node (a)            {A}   node (b)  at (2,0)  {B};
  \node (a') at (0,-1) {A}   node (b') at (2,-1) {B};
  \draw[snake=expanding waves]                            (a)  -- (b);
  \draw[snake=expanding waves,segment angle=20,red,thick] (a') -- (b');
\end{tikzpicture}
\end{codeexample}
  \itemoption{segment aspect}|=|\meta{ratio}
  This option sets an aspect ratio that is interpreted in a
  snake-specific way. For example, for the |coils| snake this
  describes the ``direction'' from which the coil is viewed.
\begin{codeexample}[]
\begin{tikzpicture}[segment amplitude=5pt,segment length=5pt]
  \node (a) {A}   node (b) at (2,1) {B}  node (c) at (2,-1) {C};
  \draw[snake=coil]                            (a) -- (b);
  \draw[snake=coil,segment aspect=0,red,thick] (a) -- (c);
\end{tikzpicture}
\end{codeexample}
\end{itemize}

It is possible to define new snakes, but this cannot be done inside
\tikzname. You need to use the command |\pgfdeclaresnake| from the
basic level directly, see Section~\ref{section-base-snakes}.

The following styles define combinations of segment settings that may
be useful:
\begin{itemize}
  \itemstyle{snake triangles 45}
  Installs a snake the consists of little triangles with an opening
  angle of $45^\circ$.
  \itemstyle{snake triangles 60}
  Installs a snake the consists of little triangles with an opening
  angle of $60^\circ$.
  \itemstyle{snake triangles 90}
  Installs a snake the consists of little triangles with an opening
  angle of $90^\circ$.
\end{itemize}



\subsection{The Curve-To Operation}

The curve-to operation allows you to extend a path using a B�zier
curve.

\begin{pathoperation}{..}{\declare{|controls|}\meta{c}\opt{|and|\meta{d}}\declare{|..|\meta{y}}}
  This operation extends the current path from the current
  point, let us call it $x$, via a curve to a the current point~$y$.
  The curve is a cubic B�zier curve. For such a curve, 
  apart from $y$, you also specify two control points $c$ and $d$. The
  idea is that the curve starts at $x$, ``heading'' in the direction
  of~$c$. Mathematically spoken, the tangent of the curve at $x$ goes
  through $c$. Similarly, the curve ends at $y$, ``coming from'' the
  other control point,~$d$. The larger the distance between $x$ and~$c$
  and between $d$ and~$y$, the larger the curve will be.

  If the ``|and|\meta{d}'' part is not given, $d$ is assumed to be
  equal to $c$.

\begin{codeexample}[]
\begin{tikzpicture}
  \draw[line width=10pt] (0,0) .. controls (1,1) .. (4,0)
                               .. controls (5,0) and (5,1) .. (4,1);
  \draw[color=gray] (0,0) -- (1,1) -- (4,0) -- (5,0) -- (5,1) -- (4,1);
\end{tikzpicture}
\end{codeexample}

  As with the line-to operation, it makes a difference whether two curves
  are joined because they resulted from consecutive curve-to or line-to
  operations, or whether they just happen to have the same ending:

\begin{codeexample}[]
\begin{tikzpicture}[line width=10pt]
  \draw (0,0) -- (1,1) (1,1) .. controls (1,0) and (2,0) .. (2,0);
  \draw (3,0) -- (4,1) .. controls (4,0) and (5,0) .. (5,0);
  \useasboundingbox (0,1.5); % make bounding box higher
\end{tikzpicture}
\end{codeexample}
\end{pathoperation}


\subsection{The Cycle Operation}

\begin{pathoperation}{--cycle}{}
  This operation adds a straight line from the current
  point to the last point specified by a move-to operation. Note that
  this need not be the beginning of the path. Furthermore, a smooth join
  is created between the first segment created after the last move-to
  operation and the straight line appended by the cycle operation.

  Consider the following example. In the left example, two triangles are
  created using three straight lines, but they are not joined at the
  ends. In the second example cycle operations are used.

\begin{codeexample}[]
\begin{tikzpicture}[line width=10pt]
  \draw (0,0) -- (1,1) -- (1,0) -- (0,0) (2,0) -- (3,1) -- (3,0) -- (2,0);
  \draw (5,0) -- (6,1) -- (6,0) -- cycle (7,0) -- (8,1) -- (8,0) -- cycle;
  \useasboundingbox (0,1.5); % make bounding box higher
\end{tikzpicture}
\end{codeexample}
\end{pathoperation}



\subsection{The Rectangle Operation}

A rectangle can obviously be created using four straight lines and a
cycle operation. However, since rectangles are needed so often, a
special syntax is available for them.

\begin{pathoperation}{rectangle}{\meta{corner}}
  When this operation is used, one corner will be the current point,
  another corner is given by \meta{corner}, which becomes the new
  current point.

\begin{codeexample}[]
\begin{tikzpicture}
  \draw (0,0) rectangle (1,1);
  \draw (.5,1) rectangle (2,0.5) (3,0) rectangle (3.5,1.5) -- (2,0);
\end{tikzpicture}
\end{codeexample}
\end{pathoperation}


\subsection{Rounding Corners}

All of the path construction operations mentioned up to now are
influenced by the following option:
\begin{itemize}
  \itemoption{rounded corners}\opt{|=|\meta{inset}}
  When this option is in force, all corners (places where a line is
  continued either via line-to or a curve-to operation) are replaced by
  little arcs so that the corner becomes smooth. 

\begin{codeexample}[]
\tikz \draw [rounded corners] (0,0) -- (1,1)
           -- (2,0) .. controls (3,1) .. (4,0);
\end{codeexample}

  The \meta{inset} describes how big the corner is. Note that the
  \meta{inset} is \emph{not} scaled along if you use a scaling option
  like |scale=2|. 

\begin{codeexample}[]
\begin{tikzpicture}
  \draw[color=gray,very thin] (10pt,15pt) circle (10pt);
  \draw[rounded corners=10pt] (0,0) -- (0pt,25pt) -- (40pt,25pt);
\end{tikzpicture}
\end{codeexample}

  You can switch the rounded corners on and off ``in the middle of
  path'' and different corners in the same path can have different
  corner radii:

\begin{codeexample}[]
\begin{tikzpicture}
  \draw (0,0) [rounded corners=10pt] -- (1,1) -- (2,1)
                     [sharp corners] -- (2,0)
               [rounded corners=5pt] -- cycle;
\end{tikzpicture}
\end{codeexample}

Here is a rectangle with rounded corners:
\begin{codeexample}[]
\tikz \draw[rounded corners=1ex] (0,0) rectangle (20pt,2ex);
\end{codeexample}

  You should be aware, that there are several pitfalls when using this
  option. First, the rounded corner will only be an arc (part of a
  circle) if the angle is $90^\circ$. In other cases, the rounded
  corner will still be round, but ``not as nice.''

  Second, if there are very short line segments in a path, the
  ``rounding'' may cause inadverted effects. In such case it may be
  necessary to temporarily switch off the rounding using
  |sharp corners|. 

  \itemoption{sharp corners}
  This options switches off any rounding on subsequent corners of the
  path.   
\end{itemize}



\subsection{The Circle and Ellipse Operations}

A circle can be approximated well using four B�zier curves. However,
it is difficult to do so correctly. For this reason, a special syntax
is available for adding such an approximation of a circle to the
current path.

\begin{pathoperation}{circle}{|(|\meta{radius}|)|}
  The center of the circle is given by the current point. The new
  current point of the path will remain to be the center of the
  circle.  
\end{pathoperation}

\begin{pathoperation}{ellipse}{|(|\meta{half width}| and |\meta{half height}|)|}
  Note that you can add spaces after |ellipse|, but you have to place
  spaces around |and|.

\begin{codeexample}[]
\begin{tikzpicture}
  \draw (1,0) circle (.5cm);
  \draw (3,0) ellipse (1cm and .5cm) -- ++(3,0) circle (.5cm)
    -- ++(2,-.5) circle (.25cm);
\end{tikzpicture}
\end{codeexample}
\end{pathoperation}


\subsection{The Arc Operation}

The \emph{arc operation} allows you to add an arc to the current
path.
\begin{pathoperation}{arc}{|(|\meta{start angle}|:|\meta{end
    angle}|:|\meta{radius}\opt{| and |\meta{half height}}|)|}
  The arc operation adds a part of a circle of the given radius
  between the given angles. The arc will start at the current point
  and will end at the end of the arc.

  \begin{codeexample}[]
\begin{tikzpicture}
  \draw (0,0) arc (180:90:1cm) -- (2,.5) arc (90:0:1cm);
  \draw (4,0) -- +(30:1cm) arc (30:60:1cm) -- cycle;
  \draw (8,0) arc (0:270:1cm and .5cm) -- cycle;
\end{tikzpicture}
\end{codeexample}

\begin{codeexample}[]
\begin{tikzpicture}
  \draw (-1,0) -- +(3.5,0);
  \draw (1,0) ++(210:2cm) -- +(30:4cm);
  \draw (1,0) +(0:1cm) arc (0:30:1cm);      
  \draw (1,0) +(180:1cm) arc (180:210:1cm);
  \path (1,0) ++(15:.75cm) node{$\alpha$};
  \path (1,0) ++(15:-.75cm) node{$\beta$};
\end{tikzpicture}
\end{codeexample}
\end{pathoperation}


\subsection{The Grid Operation}

You can add a grid to the current path using the |grid| path
operation. 

\begin{pathoperation}{grid}{\opt{\oarg{options}}\meta{corner}}
  This operations adss a grid filling a rectangle whose two corners
  are given by \meta{corner} and by the previous coordinate. Thus, the
  typical way in which a grid is drawn is |\draw (1,1) grid (3,3);|,
  which yields a grid filling the rectangle whose corners are at
  $(1,1)$ and $(3,3)$. All coordinate transformations apply to the
  grid.

\begin{codeexample}[]
\tikz[rotate=30] \draw[step=1mm] (0,0) grid (2,2);
\end{codeexample}

  The \meta{options}, which are local to the |grid| operation, can be
  used to influence the appearance of the grid. The stepping of the
  grid is governed by the following options: 

\begin{itemize}
  \itemoption{step}|=|\meta{number or dimension or coordinate} sets the stepping in both the
  $x$ and $y$-direction. If a dimension is provided, this is used
  directly. If a number is provided, this number is interpreted in the
  $xy$-coordinate system. For example, if you provide the number |2|,
  then the $x$-step is twice the $x$-vector and the $y$-step is twice
  the $y$-vector set by the |x=| and |y=| options. Finally, if you
  provide a coordinate, then the $x$-part of this coordinate will be
  used as the $x$-step and the $y$-part will be used as the
  $y$-coordinate.

\begin{codeexample}[]
\begin{tikzpicture}[x=.5cm]
  \draw[thick] (0,0) grid [step=1]     (3,2);
  \draw[red]   (0,0) grid [step=.75cm] (3,2);
\end{tikzpicture}
\begin{tikzpicture}
  \draw        (0,0) circle (1);
  \draw[blue]  (0,0) grid [step=(45:1)] (3,2);
\end{tikzpicture}
\end{codeexample}  

  A complication arises when the $x$- and/or $y$-vector do not point
  along the axes. Because of this, the actual rule for computing the
  $x$-step and the $y$-step is the following: As the $x$- and
  $y$-steps we use the $x$- and $y$-components or the following two
  vectors: The first vector is either $(\meta{x-grid-step-number},0)$
  or $(\meta{x-grid-step-dimension},0\mathrm{pt})$, the second vector
  is  $(0,\meta{y-grid-step-number})$ or
  $(0\mathrm{pt},\meta{x-grid-step-dimension})$. 
  
  \itemoption{xstep}|=|\meta{dimension or number} sets the stepping in the
  $x$-direction. 
\begin{codeexample}[]
\tikz \draw (0,0) grid [xstep=.5,ystep=.75] (3,2);
\end{codeexample}
\itemoption{ystep}|=|\meta{dimension or number} sets the stepping in the
  $y$-direction. 
\end{itemize}

  It is important to note that the grid is always ``phased'' such that
  it contains the point $(0,0)$ if that point happens to be inside the
  rectangle. Thus, the grid does \emph{not} always have an intersection
  at the corner points; this occurs only if the corner points are
  multiples of the stepping. Note that due to rounding errors, the
  ``last'' lines of a grid may be omitted. In this case, you have to
  add an epsilon to the corner points.

  The following style is useful for drawing grids:
\begin{itemize}
  \itemstyle{help lines}
  This style makes lines ``subdued'' by using thin gray lines for
  them. However, this style is not installed automatically and you
  have to say for example:
\begin{codeexample}[]
\tikz \draw[style=help lines] (0,0) grid (3,3);
\end{codeexample}
\end{itemize}
\end{pathoperation}



\subsection{The Parabola Operation}

The |parabola| path operation continues the current path with a
parabola. A parabola is a (shifted and scaled) curve defined by the
equation $f(x) = x^2$ and looks like this: \tikz \draw (-1ex,1.5ex)
parabola[parabola height=-1.5ex] +(2ex,0ex);.

\begin{pathoperation}{parabola}{\opt{\oarg{options}|bend|\meta{bend
        coordinate}}\meta{coordinate}}
  This operation adds a parabola through the current point and the
  given \meta{coordinate}. If the |bend| is given, it specifies where
  the bend should go; the \meta{options} can also be used to specify
  where the bend is. By default, the bend is at the old current point. 

\begin{codeexample}[]
\begin{tikzpicture}
  \draw               (0,0) rectangle                (1,1.5)
                      (0,0) parabola                 (1,1.5);
  \draw[xshift=1.5cm] (0,0) rectangle                (1,1.5)
                      (0,0) parabola[bend at end]    (1,1.5);
  \draw[xshift=3cm]   (0,0) rectangle                (1,1.5)
                      (0,0) parabola bend (.75,1.75) (1,1.5);
\end{tikzpicture}
\end{codeexample}

  The following options influence parabolas:
\begin{itemize}
  \itemoption{bend}|=|\meta{coordinate}
  Has the same effect as saying |bend|\meta{coordinate} outside the
  \meta{options}. The option specifies that the bend of the parabola
  should be at the given \meta{coordinate}. You have to take care
  yourself that the bend position is a ``valid'' position; which means
  that if there is no parabola of the form $f(x) = a x^2 + b x + c$
  that goes through the old current point, the given bend, and the new
  current point, the result will not be a parabola.

  There is one special property of the \meta{coordinate}: When a
  relative coordinate is given like |+(0,0)|, the position relative
  to which this coordinate is ``flexible.'' More precisely, this
  position lies somewhere on a line from the old current point to the
  new current point. The exact position depends on the next
  option.

  \itemoption{bend pos}|=|\meta{fraction}
  Specifies where the ``previous'' point is relative to which the bend
  is calculated. The previous point will be at the \meta{fraction}th
  part of the line from the old current point to the new current
  point.

  The idea is the following: If you say |bend pos=0| and
  |bend +(0,0)|, the bend will be at the old current point. If you say
  |bend pos=1| and |bend +(0,0)|, the bend will be at the new current
  point. If you say |bend pos=0.5| and |bend +(0,2cm)| the bend will
  be 2cm above the middle of the line between the start and end
  point. This is most useful in situations such as the following:
\begin{codeexample}[]
\begin{tikzpicture}
  \draw[help lines] (0,0) grid (3,2);
  \draw (-1,0) parabola[bend pos=0.5] bend +(0,2) +(3,0);
\end{tikzpicture}
\end{codeexample}

  In the above example, the |bend +(0,2)| essentially means ``a
  parabola that is 2cm high'' and |+(3,0)| means ``and 3cm wide.''
  Since this situation arises often, there is a special shortcut
  option:
  \itemoption{parabola height}|=|\meta{dimension} This option has the
  same effect as if you had written the following instead:
  |[bend pos=0.5,bend={+(0pt,|\meta{dimension}|)}]|. 
\begin{codeexample}[]
\begin{tikzpicture}
  \draw[help lines] (0,0) grid (3,2);
  \draw (-1,0) parabola[parabola height=2cm] +(3,0);
\end{tikzpicture}
\end{codeexample}
\end{itemize}

The following styles are useful shortcuts:
\begin{itemize}
  \itemstyle{bend at start} This places the bend at the start of a
  parabola. It is a shortcut for the following options:
  |bend pos=0,bend={+(0,0)}|. 
  \itemstyle{bend at end} This places the bend at the end of a
  parabola.
\end{itemize}
\end{pathoperation}


\subsection{The Sine and Cosine Operation}

The |sin| and |cos| operations are similar to the |parabola|
operation. They, too, can be used to draw (parts of) a sine or cosine
curve.

\begin{pathoperation}{sin}{\meta{coordinate}}
  The effect of |sin| is to draw a scaled and shifted version of a sine
  curve in the interval $[0,\pi/2]$. The scaling and shifting is done in
  such a way that the start of the sine curve in the interval is at the
  old current point and that the end of the curve in the interval is at
  \meta{coordinate}. Here is an example that should clarify this:

\begin{codeexample}[]
\tikz \draw (0,0) rectangle (1,1)     (0,0) sin (1,1)
            (2,0) rectangle +(1.57,1) (2,0) sin +(1.57,1);
\end{codeexample}
\end{pathoperation}

\begin{pathoperation}{cos}{\meta{coordinate}}
  This operation works similarly, only a cosine in the interval
  $[0,\pi/2]$ is drawn. By correctly alternating |sin| and |cos|
  operations, you can create a complete sine or cosine curve:

\begin{codeexample}[]
\begin{tikzpicture}[xscale=1.57]
  \draw (0,0) sin (1,1) cos (2,0) sin (3,-1) cos (4,0) sin (5,1);
  \draw[color=red] (0,1.5) cos (1,0) sin (2,-1.5) cos (3,0) sin (4,1.5) cos (5,0);
\end{tikzpicture}
\end{codeexample}
\end{pathoperation}

Note that there is no way to (conveniently) draw an interval on a sine
or cosine curve whose end points are not multiples of $\pi/2$.



\subsection{The Plot Operation}

The |plot| operation can be used to append a line or curve to the path
that goes through a large number of coordinates. These coordinates are
either given in a simple list of coordinates or they are read from
some file.

The syntax of the |plot| comes in different versions.

\begin{pathoperation}{--plot}{\meta{further arguments}}
  This operation plots the curve through the coordinates specified in
  the \meta{further arguments}. The current (sub)path is simply
  continued, that is, a line-to operation to the first point of the
  curve is implicitly added. The details of the \meta{further
    arguments}  will be explained in a moment.
\end{pathoperation}

\begin{pathoperation}{plot}{\meta{further arguments}}
  This operation plots the curve through the coordinates specified in
  the \meta{further arguments} by first ``moving'' to the first
  coordinate of the curve.
\end{pathoperation}

The \meta{further arguments} are used in three different ways to
specifying the coordinates of the points to be plotted:

\begin{enumerate}
\item
  \opt{|--|}|plot|\oarg{local options}\declare{|coordinates{|\meta{coordinate
    1}\meta{coordinate 2}\dots\meta{coordinate $n$}|}|}
\item
  \opt{|--|}|plot|\oarg{local options}\declare{|file{|\meta{filename}|}|}
\item
  \opt{|--|}|plot|\oarg{local options}\declare{|function{|\meta{gnuplot formula}|}|}
\end{enumerate}

These different ways are explained in the following.


\subsubsection{Plotting Points Given Inline}

In the first two cases, the points are given directly in the \TeX-file
as in the following example:

\begin{codeexample}[]
\tikz \draw plot coordinates {(0,0) (1,1) (2,0) (3,1) (2,1) (10:2cm)};
\end{codeexample}

Here is an example showing the difference between |plot| and |--plot|:

\begin{codeexample}[]
\begin{tikzpicture}
  \draw (0,0) -- (1,1) plot coordinates {(2,0)  (4,0)};
  \draw[color=red,xshift=5cm]
        (0,0) -- (1,1) -- plot coordinates {(2,0)  (4,0)};
\end{tikzpicture}
\end{codeexample}


\subsubsection{Plotting Points Read From an External File}

The second way of specifying points is to put them in an external
file named \meta{filename}. Currently, the only file format that
\tikzname\ allows is the following: Each line of the \meta{filename}
should contain one line starting with two numbers, separated by a
space. Anything following the two numbers on the line is
ignored. Also, lines starting with a |%| or a |#| are ignored as well
as empty lines. (This is exactly the format that \textsc{gnuplot}
produces when you say |set terminal table|.) If necessary, more
formats will be supported in the future, but it is usually easy to
produce a file containing data in this form.

\begin{codeexample}[]
\tikz \draw plot[mark=x,smooth] file {plots/pgfmanual-sine.table};
\end{codeexample}

The file |plots/pgfmanual-sine.table| reads:
\begin{codeexample}[code only]
#Curve 0, 20 points
#x y type
0.00000 0.00000  i
0.52632 0.50235  i
1.05263 0.86873  i
1.57895 0.99997  i
...
9.47368 -0.04889  i
10.00000 -0.54402  i
\end{codeexample}
It was produced from the following source, using |gnuplot|:
\begin{codeexample}[code only]
set terminal table
set output "../plots/pgfmanual-sine.table"
set format "%.5f"
set samples 20
plot [x=0:10] sin(x)
\end{codeexample}

The \meta{local options} of the |plot| operation are local to each
plot and do not affect other plots ``on the same path.'' For example,
|plot[yshift=1cm]| will locally shift the plot 1cm upward. Remember,
however, that most options can only be applied to paths as a
whole. For example, |plot[red]| does not have the effect of making the
plot red. After all, you are trying to ``locally'' make part of the
path red, which is not possible.

\subsubsection{Plotting a Function}
\label{section-tikz-gnuplot}

Often, you will want to plot points that are given via a function like
$f(x) = x \sin x$. Unfortunately, \TeX\ does not really have enough
computational power to generate the points on such a function
efficiently (it is a text processing program, after all). However,
if you allow it, \TeX\ can try to call external programs that can
easily produce the necessary points. Currently, \tikzname\ knows how to
call \textsc{gnuplot}.

When \tikzname\ encounters your operation
|plot[id=|\meta{id}|] function{x*sin(x)}| for 
the first time, it will create a file called
\meta{prefix}\meta{id}|.gnuplot|, where \meta{prefix} is |\jobname.| by
default, that is, the name of you main |.tex| file. If no \meta{id} is
given, it will be empty, which is alright, but it is better when each
plot has a unique \meta{id} for reasons explained in a moment. Next,
\tikzname\ writes some initialization code into this file followed by
|plot x*sin(x)|. The initialization code sets up things 
such that the |plot| operation will write the coordinates into another
file called \meta{prefix}\meta{id}|.table|. Finally, this table file
is read as if you had said |plot file{|\meta{prefix}\meta{id}|.table}|. 

For the plotting mechanism to work, two conditions must be met:
\begin{enumerate}
\item
  You must have allowed \TeX\ to call external programs. This is often
  switched off by default since this is a security risk (you might,
  without knowing, run a \TeX\ file that calls all sorts of ``bad''
  commands). To enable this ``calling external programs'' a command
  line option must be given to the \TeX\ program. Usually, it is
  called something like |shell-escape| or |enable-write18|. For
  example, for my |pdflatex| the option |--shell-escape| can be
  given.
\item
  You must have installed the |gnuplot| program and \TeX\ must find it
  when compiling your file.
\end{enumerate}

Unfortunately, these conditions will not always be met. Especially if
you pass some source to a coauthor and the coauthor does not have
\textsc{gnuplot} installed, he or she will have trouble compiling your
files.

For this reason, \tikzname\ behaves differently when you compile your
graphic for the second time: If upon reaching
|plot[id=|\meta{id}|] function{...}| the file \meta{prefix}\meta{id}|.table|
already exists \emph{and} if the \meta{prefix}\meta{id}|.gnuplot| file
contains what \tikzname\ thinks that it ``should'' contain, the |.table|
file is immediately read without trying to call a |gnuplot|
program. This approach has the following advantages: 
\begin{enumerate}
\item
  If you pass a bundle of your |.tex| file and all |.gnuplot| and
  |.table| files to someone else, that person can \TeX\ the |.tex|
  file without having to have |gnuplot| installed.
\item
  If the |\write18| feature is switched off for security reasons (a
  good idea), then, upon the first compilation of the |.tex| file, the
  |.gnuplot| will still be generated, but not the |.table|
  file. You can then simply call |gnuplot| ``by hand'' for each
  |.gnuplot| file, which will produce all necessary |.table| files.
\item
  If you change the function that you wish to plot or its
  domain, \tikzname\ will automatically try to regenerate the |.table|
  file.
\item
  If, out of laziness, you do not provide an |id|, the same |.gnuplot|
  will be used for different plots, but this is not a problem since
  the |.table| will automatically be regenerated for each plot
  on-the-fly. \emph{Note: If you intend to share your files with
  someone else, always use an id, so that the file can by typeset
  without having \textsc{gnuplot} installed.} Also, having unique ids
  for each plot will improve compilation speed since no external
  programs need to be called, unless it is really necessary.
\end{enumerate}

When you use |plot function{|\meta{gnuplot formula}|}|, the \meta{gnuplot
  formula} must be given in the |gnuplot| syntax, whose details are
beyond the scope of this manual. Here is the ultra-condensed
essence: Use |x| as the variable and use the C-syntax for normal
plots, use |t| as the variable for parametric plots. Here are some examples:

\begin{codeexample}[]
\begin{tikzpicture}[domain=0:4]
  \draw[very thin,color=gray] (-0.1,-1.1) grid (3.9,3.9);
  
  \draw[->] (-0.2,0) -- (4.2,0) node[right] {$x$};
  \draw[->] (0,-1.2) -- (0,4.2) node[above] {$f(x)$};
  
  \draw[color=red]    plot[id=x]   function{x}           node[right] {$f(x) =x$};
  \draw[color=blue]   plot[id=sin] function{sin(x)}      node[right] {$f(x) = \sin x$};
  \draw[color=orange] plot[id=exp] function{0.05*exp(x)} node[right] {$f(x) = \frac{1}{20} \mathrm e^x$};
\end{tikzpicture}
\end{codeexample}


The following options influence the plot:

\begin{itemize}
  \itemoption{samples}|=|\meta{number}
  sets the number of samples used in the plot. The default is 25.
  \itemoption{domain}|=|\meta{start}|:|\meta{end}
  sets the domain between which the samples are taken. The default is
  |-5:5|. 
  \itemoption{parametric}\opt{|=|\meta{true or false}}
  sets whether the plot is a parametric plot. If true, then |t| must
  be used instead of |x| as the parameter and two comma-separated
  functions must be given in the \meta{gnuplot formula}. An example is
  the following:
\begin{codeexample}[]
\tikz \draw[scale=0.5,domain=-3.141:3.141,smooth]
  plot[parametric,id=parametric-example] function{t*sin(t),t*cos(t)};
\end{codeexample}
  
  \itemoption{id}|=|\meta{id}
  sets the identifier of the current plot. This should be a unique
  identifier for each plot (though things will also work if it is not,
  but not as well, see the explanations above). The \meta{id} will be
  part of a filename, so it should not contain anything fancy like |*|
  or |$|.%$
  \itemoption{prefix}|=|\meta{prefix}
  is put before each plot file name. The default is |\jobname.|, but
  if you have many plots, it might be better to use, say |plots/| and
  have all plots placed in a directory. You have to create the
  directory yourself.
  \itemoption{raw gnuplot}
  causes the \meta{gnuplot formula} to be passed on to
  \textsc{gnuplot} without setting up the samples or the |plot|
  operation. Thus, you could write
\begin{codeexample}[code only]
plot[raw gnuplot,id=raw-example] function{set samples 25; plot sin(x)}
\end{codeexample}
  This can be 
  useful for complicated things that need to be passed to
  \textsc{gnuplot}. However, for really complicated situations you
  should create a special external generating \textsc{gnuplot} file
  and use the |file|-syntax to include the table ``by hand.''
\end{itemize}

The following styles influence the plot:
\begin{itemize}
  \itemstyle{every plot}
  This style is installed in each plot, that is, as if you always said
\begin{codeexample}[code only]
  plot[style=every plot,...]
\end{codeexample}
 This is most useful for globally setting a prefix for all plots by saying:
\begin{codeexample}[code only]
\tikzstyle{every plot}=[prefix=plots/]
\end{codeexample}
\end{itemize}



\subsubsection{Placing Marks on the Plot}

As we saw already, it is possible to add \emph{marks} to a plot using
the |mark| option. When this option is used, a copy of the plot
mark is placed on each point of the plot. Note that the marks are
placed \emph{after} the whole path has been drawn/filled/shaded. In
this respect, they are handled like text nodes. 

In detail, the following options govern how marks are drawn:
\begin{itemize}
  \itemoption{mark}|=|\meta{mark mnemonic}
  Sets the mark to a mnemonic that has previously been defined using
  the |\pgfdeclareplotmark|. By default, |*|, |+|, and |x| are available,
  which draw a filled circle, a plus, and a cross as marks. Many more
  marks become available when the library |pgflibraryplotmarks| is
  loaded. Section~\ref{section-plot-marks} lists the available plot
  marks.

  One plot mark is special: the |ball| plot mark is available only
  it \tikzname. The |ball color| determines the balls's color. Do not use
  this option with a large number of marks since it will take very long
  to render in PostScript.
  
  \begin{tabular}{lc}
    Option & Effect \\\hline \vrule height14pt width0pt
    \plotmarkentrytikz{ball}
  \end{tabular}

  \itemoption{mark repeat}|=|\meta{r}
  This option tells \tikzname\ that only every $r$th mark should be
  drawn.
  
\begin{codeexample}[]
\tikz \draw plot[mark=x,mark repeat=3,smooth] file {plots/pgfmanual-sine.table};
\end{codeexample}

  \itemoption{mark phase}|=|\meta{p}
  This option tells \tikzname\ that the first mark to be draw should
  be the $p$th, followed by the $(p+r)$th, then the $(p+2r)$th, and so
  on.
  
\begin{codeexample}[]
\tikz \draw plot[mark=x,mark repeat=3,mark phase=6,smooth] file {plots/pgfmanual-sine.table};
\end{codeexample}

  \itemoption{mark indices}|=|\meta{list}
  This option allows you to specify explicitly the indices at which a
  mark should be placed. Counting starts with 1. You can use the
  |\foreach| syntax, that is, |...| can be used.
    
\begin{codeexample}[]
\tikz \draw plot[mark=x,mark indices={1,4,...,10,11,12,...,16,20},smooth]
  file {plots/pgfmanual-sine.table};
\end{codeexample}
  
  \itemoption{mark size}|=|\meta{dimension}
  Sets the size of the plot marks. For circular plot marks,
  \meta{dimension} is the radius, for other plot marks
  \meta{dimension} should be about half the width and height.

  This option is not really necessary, since you achieve the same
  effect by specifying |scale=|\meta{factor} as a local option, where
  \meta{factor} is the quotient of the desired size and the default
  size. However, using |mark size| is a bit faster and more natural. 

  \itemoption{mark options}|=|\meta{options}
  These options are applied to marks when they are drawn. For example,
  you can scale (or otherwise transform) the plot mark or set its
  color. 
\begin{codeexample}[]
\tikz \fill[fill=blue!20]
  plot[mark=triangle*,mark options={color=blue,rotate=180}]
    file{plots/pgfmanual-sine.table} |- (0,0);
\end{codeexample}
\end{itemize}



\subsubsection{Smooth Plots, Sharp Plots, and Comb Plots}

There are different things the |plot| operation can do with the points
it reads from a file or from the inlined list of points. By default,
it will connect these points by straight lines. However, you can also
use options to change the behavior of |plot|.

\begin{itemize}
  \itemoption{sharp plot}
  This is the default and causes the points to be connected by
  straight lines. This option is included only so that you can
  ``switch back'' if you ``globally'' install, say, |smooth|.
  
  \itemoption{smooth}
  This option causes the points on the path to be connected using a
  smooth curve:

\begin{codeexample}[]
\tikz\draw plot[smooth] file{plots/pgfmanual-sine.table};
\end{codeexample}

  Note that the smoothing algorithm is not very intelligent. You will
  get the best results if the bending angles are small, that is, less
  than about $30^\circ$ and, even more importantly, if the distances
  between points are about the same all over the plotting path.

  \itemoption{tension}|=|\meta{value}
  This option influences how ``tight'' the smoothing is. A lower value
  will result in sharper corners, a higher value in more ``round''
  curves. A value of $1$ results in a circle if four points at
  quarter-positions on a circle are given. The default is $0.55$. The
  ``correct'' value depends on the details of plot.
  
\begin{codeexample}[]
\begin{tikzpicture}[smooth cycle]
  \draw                 plot[tension=0.2]
    coordinates{(0,0) (1,1) (2,0) (1,-1)};
  \draw[yshift=-2.25cm] plot[tension=0.5]
    coordinates{(0,0) (1,1) (2,0) (1,-1)};
  \draw[yshift=-4.5cm]  plot[tension=1]
    coordinates{(0,0) (1,1) (2,0) (1,-1)};
\end{tikzpicture}
\end{codeexample}
  
  \itemoption{smooth cycle}
  This option causes the points on the path to be connected using a
  closed smooth curve. 

\begin{codeexample}[]
\tikz[scale=0.5]
  \draw plot[smooth cycle] coordinates{(0,0) (1,0) (2,1) (1,2)}
        plot               coordinates{(0,0) (1,0) (2,1) (1,2)} -- cycle;
\end{codeexample}

  \itemoption{ycomb}
  This option causes the |plot| operation to interpret the plotting
  points differently. Instead of connecting them, for each point of
  the plot a straight line is added to the path from the $x$-axis to the point,
  resulting in a sort of ``comb'' or ``bar diagram.''

\begin{codeexample}[]
\tikz\draw[ultra thick] plot[ycomb,thin,mark=*] file{plots/pgfmanual-sine.table};
\end{codeexample}

\begin{codeexample}[]
\begin{tikzpicture}[ycomb]
  \draw[color=red,line width=6pt]
    plot coordinates{(0,1) (.5,1.2) (1,.6) (1.5,.7) (2,.9)};
  \draw[color=red!50,line width=4pt,xshift=3pt]
    plot coordinates{(0,1.2) (.5,1.3) (1,.5) (1.5,.2) (2,.5)};
\end{tikzpicture}
\end{codeexample}

  \itemoption{xcomb}
  This option works like |ycomb| except that the bars are horizontal. 

\begin{codeexample}[]
\tikz \draw plot[xcomb,mark=x] coordinates{(1,0) (0.8,0.2) (0.6,0.4) (0.2,1)};
\end{codeexample}

  \itemoption{polar comb}
  This option causes a line from the origin to the point to be added
  to the path for each plot point.

\begin{codeexample}[]
\tikz \draw plot[polar comb,
     mark=pentagon*,mark options={fill=white,draw=red},mark size=4pt]
   coordinates {(0:1cm) (30:1.5cm) (160:.5cm) (250:2cm) (-60:.8cm)};
\end{codeexample}


  \itemoption{only marks}
  This option causes only marks to be shown; no path segments are
  added to the actual path. This can be useful for quickly adding some
  marks to a path.

\begin{codeexample}[]
\tikz \draw (0,0) sin (1,1) cos (2,0)
  plot[only marks,mark=x] coordinates{(0,0) (1,1) (2,0) (3,-1)};
\end{codeexample}
\end{itemize}



\subsection{The To Path Operation}

The |to| operation is used to add a user-defined path
from the previous coordinate to the following coordinate. When you
write |(a) to (b)|, a straight line is added from |a|
to |b|, exactly as if you had written |(a) -- (b)|. However, if you
write |(a) to [out=135,in=45] (b)| a curve is added to the path,
which leaves at an angle of 135$^\circ$ at |a| and arrives at an angle
of 45$^\circ$ at |b|. This is because the options |in| and |out|
trigger a special path to be used instead of the straight line. 

\begin{pathoperation}{to}{\opt{|[|\meta{options}|]|}
    \opt{\meta{nodes}} |(|\meta{coordinate}|)|}

  This path operation inserts the path current set via the |to path|
  option at the current position. The \meta{options} can be used to
  modify (perhaps implicitly) the |to path| and to setup how the path
  will be rendered. 

  Before the |to path| is inserted, a number of macros are setup that
  can ``help'' the |to path|. These are |\tikztostart|,
  |\tikztotarget|, and |\tikztonodes|; they are explained in the
  following. 
  
  \medskip
  \textbf{Start and Target Coordinates.}\ \ 
  The |to| operation is always followed by a \meta{coordinate}, called
  the target coordinate. The macro |\tikztotarget| is set to this
  coordinate (without the parantheses). There is also a \emph{start
    coordinate}, which is the coordinate preceding the |to|
  operation. This coordinate can be accessed via the macro
  |\tikztostart|. In the following example, for the first |to|, the
  macro |\tikztostart| is |0pt,0pt| and the |\tikztotarget| is
  |0,2|. For the second |to|, the macro |\tikztostart| is |10pt,10pt|
  and |\tikztotarget| is |a|.
  
\begin{codeexample}[]
\begin{tikzpicture}
  \draw[help lines] (0,0) grid (3,2);
  
  \draw      (0,0)       to (0,2);
  \node      (a)         at (2,2) {a};
  \draw[red] (10pt,10pt) to (a);
\end{tikzpicture}
\end{codeexample}


  \medskip
  \textbf{Nodes on tos.}\ \
  It is possible to add nodes to the paths constructed by a |to|
  operation. To do so, you specify the nodes between the |to|
  keyword and the coordinate (if there are options to the |to|
  operation, these come first). The effect of |(a) to node {x} (b)| 
  (typically) is the same as if you had written
  |(a) -- node {x} (b)|, namely that the node is placed on the
  to. This can be used to add labels to tos: 
  
\begin{codeexample}[]
\begin{tikzpicture}
  \draw (0,0) to node [sloped,above] {x} (3,2);

  \draw (0,0) to[out=90,in=180] node [sloped,above] {x} (3,2);
\end{tikzpicture}
\end{codeexample}

  \medskip
  \textbf{Styles for nodes.}\ \
  In addition to the \meta{options} given after the |to| operation,
  the following style is also set at the beginning of the to path:
  \begin{itemize}
    \itemstyle{every to}
    This style is installed at the beginning of every to. By
    default, it is set to |draw|. 
\begin{codeexample}[]
\begin{tikzpicture}
  \tikzstyle{every to}=[draw,dashed]
  \path (0,0) to (3,2);
\end{tikzpicture}
\end{codeexample}
  \end{itemize}

  \medskip
  \textbf{Options.}\ \ 
  The \meta{options} given with the |to| allow you to influence the
  appearance of the |to path|. Mostly, these options are used to
  change the |to path|. This can be used to change the path from a 
  straight line to, say, a curve.

  The path used is set using the following option:
  \begin{itemize}
    \itemoption{to path}|=|\meta{path}
    Whenever an |to| operation is used, the \meta{path} is
    inserted. More precisely, the following path is added:

    \begin{quote}
      |[every to,|\meta{options}|] |\meta{path}
    \end{quote}
  
    The \meta{options} are the options given to the |to| operation,
    the \meta{path} is the path set by this option |to path|.

    Inside the \meta{path}, different macros are used to reference the
    from- and to-coordinates. In detail, these are:
    \begin{itemize}
    \item \declare{|\tikztostart|} will expand to the from-coordinate
      (without the parantheses).
    \item \declare{|\tikztotarget|} will expand to the to-coordinate.
    \item \declare{|\tikztonodes|} will expand to the nodes between
      the |to| operation and the coordinate. Furthermore, these
      nodes will have the |pos| option set implicitly.      
    \end{itemize}

    Let us have a look at a simple example. The standard straight line
    for an to is achieved by the following \meta{path}:
    \begin{quote}
      |-- (\tikztotarget) \tikztonodes|
    \end{quote}

    Indeed, this is the default setting for the path. When we write
    |(a) to (b)|, the \meta{path} will expand to |(a) -- (b)|, when
    we write
    \begin{quote}
      |(a) to[red] node {x} (b)|
    \end{quote}
    the \meta{path} will expand to
    \begin{quote}
      |(a) -- (b) node[pos] {x}|
    \end{quote}

    It is not possible to specify the path
    \begin{quote}
      |-- \tikztonodes (\tikztotarget)|
    \end{quote}
    since \tikzname\ does not allow one to have a macro after |--|
    that expands to a node.

    Now let us have a look at how we can modify the \meta{path}
    sensibly. The simplest way is to use a curve.
    
\begin{codeexample}[]
\begin{tikzpicture}[to path={
    .. controls +(1,0) and +(1,0) .. (\tikztotarget) \tikztonodes}]

  \node (a) at (0,0) {a};
  \node (b) at (2,1) {b};
  \node (c) at (1,2) {c};
                  
  \draw (a) to node {x} (b)
        (a) to          (c);
\end{tikzpicture}
\end{codeexample}

    Here is another example:

\begin{codeexample}[]
\tikzstyle{my loop}=[->,to path={
  .. controls +(80:1) and +(100:1) .. (\tikztotarget) \tikztonodes}]
\tikzstyle{my state}=          [circle,draw]
                           
\begin{tikzpicture}[shorten >=2pt]
  \node [my state] (a) at (210:1) {$q_a$};
  \node [my state] (b) at (330:1) {$q_b$};
                  
  \draw (a) to           node[below]       {1} (b)
            to [my loop] node[above right] {0} (b);
\end{tikzpicture}
\end{codeexample}

    \itemoption{execute at begin to}|=|\meta{code}
    The \meta{code} is executed prior to the to. This can be used to
    draw one or more additional paths or to do additional
    computations.

    \itemoption{executed at end to}|=|\meta{code}
    Works like the previous option, only this code is executed after
    the to path has been added.

  \itemstyle{every to}
    This style is installed at the beginning of every to. It is empty
    by default.
  \end{itemize}
\end{pathoperation}

There are a number of predefined |to path|s, see
Section~\ref{library-to-paths} for a reference.


  

\subsection{The Scoping Operation}

When \tikzname\ encounters and opening or a closing brace (|{| or~|}|) at
some point where a path operation should come, it will open or close a
scope. All options that can be applied ``locally'' will be scoped
inside the scope. For example, if you apply a transformation like
|[xshift=1cm]| inside the scoped area, the shifting only applies to
the scope. On the other hand, an option like |color=red| does not have
any effect inside a scope since it can only be applied to the path as
a whole. 


\subsection{The Node Operation}

There are teo more operations that can be found in paths:
|node| and |edge|. The first is used to add a so-called node to a
path. This operation is special in the following sense: It does not 
change the current path in any way. In other words, this operation 
is not really a path operation, but has an effect that is
``external''  to the path. The |edge| operation has similar effect in
that it adds something \emph{after} the main parth has been
drawn. However, it works like the |to| operation, that is, it adds a
|to| path to the picture after the main path has been drawn.

Since these operations are quite complex, they are described in the
separate Section~\ref{section-nodes}.  




% Copyright 2006 by Till Tantau
%
% This file may be distributed and/or modified
%
% 1. under the LaTeX Project Public License and/or
% 2. under the GNU Free Documentation License.
%
% See the file doc/generic/pgf/licenses/LICENSE for more details.

\section{Actions on Paths}

\subsection{Overview}

Once a path has been constructed, different things can be done with
it. It can be drawn (or stroked) with a ``pen,'' it can be filled with
a color or shading, it can be used for clipping subsequent drawing, it
can be used to specify the extend of the picture---or  any
combination of these actions at the same time.

To decide what is to be done with a path, two methods can be
used. First, you can use a special-purpose command like |\draw| to
indicate that the path should be drawn. However, commands like |\draw|
and |\fill| are just abbreviations for special cases of the more
general method: Here, the |\path| command is used to specify the
path. Then, options encountered on the path indicate what should be
done with the path.

For example, |\path (0,0) circle (1cm);| means ``This is a path
consisting of a circle around the origin. Do not do anything with it
(throw it away).'' However, if the option |draw| is encountered
anywhere on the path, the circle will be drawn. ``Anywhere'' is any
point on the path where an option can be given, which is everywhere
where a path command like |circle (1cm)| or |rectangle (1,1)| or even
just |(0,0)| would also be allowed. Thus, the following commands all
draw the same circle:
\begin{codeexample}[code only]
\path [draw] (0,0) circle (1cm);
\path (0,0) [draw] circle (1cm);
\path (0,0) circle (1cm) [draw];
\end{codeexample}
Finally, |\draw (0,0) circle (1cm);| also draws a path, because
|\draw| is an abbreviation for |\path [draw]| and thus the command
expands to the first line of the above example.

Similarly, |\fill| is an abbreviation for |\path[fill]| and
|\filldraw| is an abbreviation for the command
|\path[fill,draw]|. Since options accumulate, the following commands
all have the same effect: 
\begin{codeexample}[code only]
\path [draw,fill]   (0,0) circle (1cm);
\path [draw] [fill] (0,0) circle (1cm);
\path [fill] (0,0) circle (1cm) [draw];
\draw [fill] (0,0) circle (1cm);
\fill (0,0) [draw] circle (1cm);
\filldraw (0,0) circle (1cm);
\end{codeexample}

In the following subsection the different actions are explained that
can be performed on a path. The following commands are abbreviations for
certain sets of actions, but for many useful combinations there are no
abbreviations:

\begin{command}{\draw}
  Inside |{tikzpicture}| this is an abbreviation for |\path[draw]|.
\end{command}

\begin{command}{\fill}
  Inside |{tikzpicture}| this is an abbreviation for |\path[fill]|.
\end{command}

\begin{command}{\filldraw}
  Inside |{tikzpicture}| this is an abbreviation for |\path[fill,draw]|.
\end{command}

\begin{command}{\pattern}
  Inside |{tikzpicture}| this is an abbreviation for |\path[pattern]|.
\end{command}

\begin{command}{\shade}
  Inside |{tikzpicture}| this is an abbreviation for |\path[shade]|.
\end{command}

\begin{command}{\shadedraw}
  Inside |{tikzpicture}| this is an abbreviation for |\path[shade,draw]|.
\end{command}

\begin{command}{\clip}
  Inside |{tikzpicture}| this is an abbreviation for |\path[clip]|.
\end{command}

\begin{command}{\useasboundingbox}
  Inside |{tikzpicture}| this is an abbreviation for |\path[use as bounding box]|.
\end{command}



\subsection{Specifying a Color}

The most unspecific option for setting colors is the following:

\begin{key}{/tikz/color=\meta{color name}}
  \indexoption{color option}%
  This option sets the color that is used for fill, drawing, and text
  inside the current scope. Any special settings for filling colors or
  drawing colors are immediately ``overruled'' by this option.

  The \meta{color name} is the name of a previously defined color. For
  \LaTeX\ users, this is just a normal ``\LaTeX-color'' and the
  |xcolor| extensions are allows. Here is an example:

\begin{codeexample}[]
\tikz \fill[color=red!20] (0,0) circle (1ex);
\end{codeexample}

  It is possible to ``leave out'' the |color=| part and you can also
  write:
\begin{codeexample}[]
\tikz \fill[red!20] (0,0) circle (1ex);
\end{codeexample}
  What happens is that every option that \tikzname\ does not know, like
  |red!20|, gets a ``second chance'' as a color name.

  For plain \TeX\ users, it is not so easy to specify colors since
  plain \TeX\ has no ``standardized'' color naming
  mechanism. Because of this, \pgfname\ emulates the |xcolor| package,
  though the emulation is \emph{extremely basic} (more precisely, what
  I could hack together in two hours or so). The emulation allows you
  to do the following:
  \begin{itemize}
  \item Specify a new color using |\definecolor|. Only the two color
    models |gray| and |rgb| are supported\footnote{Con\TeX t users should be aware that \texttt{\textbackslash definecolor} has a different meaning in Con\TeX t. There is a low-level equivalent named \texttt{\textbackslash pgfutil@definecolor} which can be used instead.}.%
    \example |\definecolor{orange}{rgb}{1,0.5,0}|
  \item Use |\colorlet| to define a new color based on an old
    one. Here, the |!| mechanism is supported, though only ``once''
    (use multiple |\colorlet| for more fancy colors).
    \example |\colorlet{lightgray}{black!25}|
  \item Use |\color|\marg{color name} to set the color in the current
    \TeX\ group. |\aftergroup|-hackery is used to restore the color
    after the group.
  \end{itemize}
\end{key}

As pointed out above, the |color=| option applies to ``everything''
(except to shadings), which is not always what you want. Because of
this, there are several more specialized color options. For example,
the |draw=| option sets the color used for drawing, but does not
modify the color used for filling. These color options are documented
where the path action they influence is described.


\subsection{Drawing a Path}

You can draw a path using the following option:
\begin{key}{/tikz/draw=\meta{color} (default \normalfont is scope's color setting)}
  Causes the path to be drawn. ``Drawing'' (also known as
  ``stroking'') can be thought of as picking up a pen and moving it
  along the path, thereby leaving ``ink'' on the canvas.

  There are numerous parameters that influence how a line is drawn,
  like the thickness or the dash pattern. These options are explained
  below.

  If the optional \meta{color} argument is given, drawing is done
  using the given \meta{color}. This color can be different from the
  current filling color, which allows you to draw and fill a path with
  different colors. If no \meta{color} argument is given, the last
  usage of the |color=| option is used.

  If the special color name |none| is given, this option causes
  drawing to be ``switched off.'' This is useful if a style has
  previously switched on drawing and you locally wish to undo this
  effect. 

  Although this option is normally used on paths to indicate that the
  path should be drawn, it also makes sense to use the option with a
  |{scope}| or |{tikzpicture}| environment. However, this will
  \emph{not} cause all path to drawn. Instead, this just sets the
  \meta{color} to be used for drawing paths inside the environment.

\begin{codeexample}[]
\begin{tikzpicture}
  \path[draw=red] (0,0) -- (1,1) -- (2,1) circle (10pt);
\end{tikzpicture}
\end{codeexample}
\end{key}


The following subsections list the different options that influence
how a path is drawn. All of these options only have an effect if the
|draw| options is given (directly or indirectly).

\subsubsection{Graphic Parameters: Line Width, Line Cap, and Line Join}

\label{section-cap-joins}

\begin{key}{/tikz/line width=\meta{dimension} (initially 0.4pt)}
  Specifies the line width. Note the space.

\begin{codeexample}[]
  \tikz \draw[line width=5pt] (0,0) -- (1cm,1.5ex);
\end{codeexample}
\end{key}

There are a number of predefined styles that provide more ``natural''
ways of setting the line width. You can also redefine these
styles. 

\begin{stylekey}{/tikz/ultra thin}
  Sets the line width to 0.1pt.
\begin{codeexample}[]
  \tikz \draw[ultra thin] (0,0) -- (1cm,1.5ex);
\end{codeexample}
\end{stylekey}

\begin{stylekey}{/tikz/very thin}
  Sets the line width to 0.2pt.
\begin{codeexample}[]
  \tikz \draw[very thin] (0,0) -- (1cm,1.5ex);
\end{codeexample}
\end{stylekey}

\begin{stylekey}{/tikz/thin}
  Sets the line width to 0.4pt.
\begin{codeexample}[]
  \tikz \draw[thin] (0,0) -- (1cm,1.5ex);
\end{codeexample}
\end{stylekey}

\begin{stylekey}{/tikz/semithick}
  Sets the line width to 0.6pt.
\begin{codeexample}[]
  \tikz \draw[semithick] (0,0) -- (1cm,1.5ex);
\end{codeexample}
\end{stylekey}

\begin{stylekey}{/tikz/thick}
  Sets the line width to 0.8pt.
\begin{codeexample}[]
  \tikz \draw[thick] (0,0) -- (1cm,1.5ex);
\end{codeexample}
\end{stylekey}

\begin{stylekey}{/tikz/very thick}
  Sets the line width to 1.2pt.
\begin{codeexample}[]
  \tikz \draw[very thick] (0,0) -- (1cm,1.5ex);
\end{codeexample}
\end{stylekey}

\begin{stylekey}{/tikz/ultra thick}
  Sets the line width to 1.6pt.
\begin{codeexample}[]
  \tikz \draw[ultra thick] (0,0) -- (1cm,1.5ex);
\end{codeexample}
\end{stylekey}


\begin{key}{/tikz/line cap=\meta{type} (initially butt)}
  Specifies how lines ``end.'' Permissible \meta{type} are |round|,
  |rect|, and |butt|. They have the following effects:

\begin{codeexample}[]
\begin{tikzpicture}
  \begin{scope}[line width=10pt]
    \draw[line cap=rect]  (0,0 ) -- (1,0);
    \draw[line cap=butt]  (0,.5) -- (1,.5);
    \draw[line cap=round] (0,1 ) -- (1,1);
  \end{scope}
  \draw[white,line width=1pt]
    (0,0 ) -- (1,0) (0,.5) -- (1,.5) (0,1 ) -- (1,1);
\end{tikzpicture}
\end{codeexample}
\end{key}

\begin{key}{/tikz/line join=\meta{type} (initially miter)}
  Specifies how lines ``join.'' Permissible \meta{type} are |round|,
  |bevel|, and |miter|. They have the following effects:

\begin{codeexample}[]
\begin{tikzpicture}[line width=10pt]
  \draw[line join=round] (0,0) -- ++(.5,1) -- ++(.5,-1);
  \draw[line join=bevel] (1.25,0) -- ++(.5,1) -- ++(.5,-1); 
  \draw[line join=miter] (2.5,0) -- ++(.5,1) -- ++(.5,-1); 
  \useasboundingbox (0,1.5); % make bounding box bigger
\end{tikzpicture}
\end{codeexample}

  \begin{key}{/tikz/miter limit=\meta{factor} (initially 10)}
    When you use the miter join and there is a very sharp corner (a
    small angle), the miter join may protrude very far over the actual
    joining point. In this case, if it were to protrude by 
    more than \meta{factor} times the line width, the miter join is
    replaced by a bevel join. 

\begin{codeexample}[]
\begin{tikzpicture}[line width=5pt]
  \draw                 (0,0) -- ++(5,.5) -- ++(-5,.5);
  \draw[miter limit=25] (6,0) -- ++(5,.5) -- ++(-5,.5);
  \useasboundingbox (14,0); % make bounding box bigger
\end{tikzpicture}
\end{codeexample}
  \end{key}
\end{key}

\subsubsection{Graphic Parameters: Dash Pattern}

\begin{key}{/tikz/dash pattern=\meta{dash pattern}}
  Sets the dashing pattern. The syntax is the same as in
  \textsc{metafont}. For example following pattern
  |on 2pt off 3pt on 4pt off 4pt| means ``draw
  2pt, then leave out 3pt, then draw 4pt once more, then leave out 4pt
  again, repeat''. 

\begin{codeexample}[]
\begin{tikzpicture}[dash pattern=on 2pt off 3pt on 4pt off 4pt]
  \draw (0pt,0pt) -- (3.5cm,0pt);
\end{tikzpicture}
\end{codeexample}
\end{key}

\begin{key}{/tikz/dash phase=\meta{dash phase} (initially 0pt)}
  Shifts the start of the dash pattern by \meta{phase}.

\begin{codeexample}[]
\begin{tikzpicture}[dash pattern=on 20pt off 10pt]
  \draw[dash phase=0pt] (0pt,3pt) -- (3.5cm,3pt);
  \draw[dash phase=10pt] (0pt,0pt) -- (3.5cm,0pt);
\end{tikzpicture}
\end{codeexample}
\end{key}

As for the line thickness, some predefined styles allow you to set the
dashing conveniently.

\begin{stylekey}{/tikz/solid}
  Shorthand for setting a solid line as ``dash pattern.'' This is the default.

\begin{codeexample}[]
\tikz \draw[solid] (0pt,0pt) -- (50pt,0pt);
\end{codeexample}
\end{stylekey}

\begin{stylekey}{/tikz/dotted}
  Shorthand for setting a dotted dash pattern.

\begin{codeexample}[]
\tikz \draw[dotted] (0pt,0pt) -- (50pt,0pt);
\end{codeexample}
\end{stylekey}

\begin{stylekey}{/tikz/densely dotted}
  Shorthand for setting a densely dotted dash pattern.

\begin{codeexample}[]
\tikz \draw[densely dotted] (0pt,0pt) -- (50pt,0pt);
\end{codeexample}
\end{stylekey}

\begin{stylekey}{/tikz/loosely dotted}
  Shorthand for setting a loosely dotted dash pattern.

\begin{codeexample}[]
\tikz \draw[loosely dotted] (0pt,0pt) -- (50pt,0pt);
\end{codeexample}
\end{stylekey}

\begin{stylekey}{/tikz/dashed}
  Shorthand for setting a dashed dash pattern.

\begin{codeexample}[]
\tikz \draw[dashed] (0pt,0pt) -- (50pt,0pt);
\end{codeexample}
\end{stylekey}

\begin{stylekey}{/tikz/densely dashed}
  Shorthand for setting a densely dashed dash pattern.

\begin{codeexample}[]
\tikz \draw[densely dashed] (0pt,0pt) -- (50pt,0pt);
\end{codeexample}
\end{stylekey}

\begin{stylekey}{/tikz/loosely dashed}
  Shorthand for setting a loosely dashed dash pattern.

\begin{codeexample}[]
\tikz \draw[loosely dashed] (0pt,0pt) -- (50pt,0pt);
\end{codeexample}
\end{stylekey}


\subsubsection{Graphic Parameters: Draw Opacity}

When a line is drawn, it will normally ``obscure'' everything behind
it as if you has used perfectly opaque ink. It is also possible to ask
\tikzname\ to use an ink that is a little bit (or a big bit)
transparent using the |draw opacity| option. This is explained in
Section~\ref{section-tikz-transparency} on transparency in more detail.



\subsubsection{Graphic Parameters: Arrow Tips}

When you draw a line, you can add arrow tips at the ends. It is
only possible to add one arrow tip at the start and one at the end. If
the path consists of several segments, only the last segment gets
arrow tips. The behavior for paths that are closed is not specified
and may change in the future.

\begin{key}{/tikz/arrows=\meta{start arrow kind}|-|\meta{end arrow kind}}
  This option sets the start and end arrow tips (an empty value as in |->|
  indicates that no arrow tip should be drawn at the start).%
  \indexoption{arrows}

  \emph{Note: Since the arrow option is so often used, you can leave
    out the text |arrows=|.} What happens is that every option that
  contains a |-| is interpreted as an arrow specification.

\begin{codeexample}[]
\begin{tikzpicture}
  \draw[->]        (0,0)   -- (1,0);
  \draw[o-stealth] (0,0.3) -- (1,0.3);
\end{tikzpicture}
\end{codeexample}

  The permissible values are all predefined arrow tips, though
  you can also define new arrow tip kinds as explained in
  Section~\ref{section-arrows}. This is often necessary to obtain
  ``double'' arrow tips and arrow tips that have a fixed size. You
  need to load the |arrows| library if you need arrow tips other than
  the default ones, see Section~\ref{section-library-arrows}.

  One arrow tip kind is special: |>| (and all arrow tip kinds containing the
  arrow tip kind such as |<<| or \verb!>|!). This arrow tip type is not  
  fixed. Rather, you can redefine it using the |>=| option, see
  below. 

  \example You can also combine arrow tip types as in
\begin{codeexample}[]
\begin{tikzpicture}[thick]
  \draw[to reversed-to]   (0,0) .. controls +(.5,0) and +(-.5,-.5) .. +(1.5,1);
  \draw[[-latex reversed] (1,0) .. controls +(.5,0) and +(-.5,-.5) .. +(1.5,1);
  \draw[latex-)]          (2,0) .. controls +(.5,0) and +(-.5,-.5) .. +(1.5,1);
  \useasboundingbox (-.1,-.1) rectangle (3.1,1.1); % make bounding box bigger
\end{tikzpicture}
\end{codeexample}
\end{key}

\begin{key}{/tikz/>=\meta{end arrow kind}}
  This option can be used to redefine the ``standard'' arrow tip |>|. The
  idea is that different people have different ideas what arrow tip kind
  should normally be used. I prefer the arrow tip of \TeX's |\to| command
  (which is used in things like $f\colon A \to B$). Other people will
  prefer \LaTeX's standard arrow tip, which looks like this: \tikz
  \draw[-latex] (0,0) -- (10pt,1ex);. Since the arrow tip kind |>| is
  certainly the most ``natural'' one to use, it is kept free of any
  predefined meaning. Instead, you can change it by saying |>=to| to
  set the ``standard'' arrow tip kind to \TeX's arrow tip, whereas |>=latex|
  will set it to \LaTeX's arrow tip and |>=stealth| will use a
  \textsc{pstricks}-like arrow tip.

  Apart from redefining the arrow tip kind |>| (and |<| for the start),
  this option also redefines the following arrow tip kinds: |>| and |<| as
  the swapped version of \meta{end arrow kind}, |<<| and |>>| as
  doubled versions, |>>| and |<<| as swapped doubled versions, %>>
  and \verb!|<! and \verb!>|! as arrow tips ending with a vertical bar.

\begin{codeexample}[]
\begin{tikzpicture}[scale=2]
  \begin{scope}[>=latex]
    \draw[->]    (0pt,6ex) -- (1cm,6ex);
    \draw[>->>]  (0pt,5ex) -- (1cm,5ex); 
    \draw[|<->|] (0pt,4ex) -- (1cm,4ex);
  \end{scope}
  \begin{scope}[>=diamond]
    \draw[->]    (0pt,2ex) -- (1cm,2ex);
    \draw[>->>]  (0pt,1ex) -- (1cm,1ex);
    \draw[|<->|] (0pt,0ex) -- (1cm,0ex);
  \end{scope} 
\end{tikzpicture}
\end{codeexample} 
% <<
\end{key}

\begin{key}{/tikz/shorten >=\meta{dimension} (initially 0pt)}
  This option will shorten the end of lines by the given
  \meta{dimension}. If you specify an arrow tip, lines are already
  shortened a bit such that the arrow tip touches the specified endpoint
  and does not ``protrude over'' this point. Here is an example:

\begin{codeexample}[]
\begin{tikzpicture}[line width=20pt]
  \useasboundingbox (0,-1.5) rectangle (3.5,1.5);
  \draw[red]        (0,0) -- (3,0);
  \draw[gray,->]    (0,0) -- (3,0);
\end{tikzpicture}
\end{codeexample}

  The |shorten >| option allows you to shorten the end on the line
  \emph{additionally} by the given distance. This option can also be
  useful if you have not specified an arrow tip at all.

\begin{codeexample}[]
\begin{tikzpicture}[line width=20pt]
  \useasboundingbox (0,-1.5) rectangle (3.5,1.5);
  \draw[red]                     (0,0) -- (3,0);
  \draw[-to,shorten >=10pt,gray] (0,0) -- (3,0);
\end{tikzpicture}
\end{codeexample}
\end{key}


\begin{key}{/tikz/shorten <=\meta{dimension}}
  Works like |shorten >|, but for the start.
\end{key}



\subsubsection{Graphic Parameters: Double Lines and Bordered Lines}

\begin{key}{/tikz/double=\meta{core color} (default white)}
  This option causes ``two'' lines to be drawn instead of a single
  one. However, this is not what really happens. In reality, the path
  is drawn twice. First, with the normal drawing color, secondly with
  the \meta{core color}, which is normally |white|. Upon the second
  drawing, the line width is reduced. The net effect is that it
  appears as if two lines had been drawn and this works well even with
  complicated, curved paths:

\begin{codeexample}[]
\tikz \draw[double]
  plot[smooth cycle] coordinates{(0,0) (1,1) (1,0) (0,1)};
\end{codeexample}

  You can also use the doubling option to create an effect in which a
  line seems to have a certain ``border'':

\begin{codeexample}[]
\begin{tikzpicture}
  \draw (0,0) -- (1,1);
  \draw[draw=white,double=red,very thick] (0,1) -- (1,0);
\end{tikzpicture}
\end{codeexample}
\end{key}

\begin{key}{/tikz/double distance=\meta{dimension} (initially 0.6pt)}
  Sets the distance the ``two'' lines are spaced apart. In reality,
  this is the thickness of the line that is used 
  to draw the path for the second time. The thickness of the
  \emph{first} time the path is drawn is twice the normal line width
  plus the given \meta{dimension}. As a side-effect, this option
  ``selects'' the |double| option.

\begin{codeexample}[]
\begin{tikzpicture}
  \draw[very thick,double]              (0,0) arc (180:90:1cm);
  \draw[very thick,double distance=2pt] (1,0) arc (180:90:1cm);
  \draw[thin,double distance=2pt]       (2,0) arc (180:90:1cm);
\end{tikzpicture}
\end{codeexample}
\end{key}

\begin{key}{/tikz/double distance between line centers=\meta{dimension}}
  This option works like |double distance|, only the distance is not
  the distance between (inner) borders of the two main lines, ut
  between their centers. Thus, the thickness the
  \emph{first} time the path is drawn is the normal line width
  plus the given \meta{dimension}, while the line width of the
  \emph{second} line that is drawn is \meta{dimension} minus the
  normal line width. As a side-effect, this option ``selects'' the
  |double| option. 

\begin{codeexample}[]
\begin{tikzpicture}[double distance between line centers=3pt]
  \foreach \lw in {0.5,1,1.5,2,2.5}
    \draw[line width=\lw pt,double] (\lw,0) -- ++(4mm,0);
\end{tikzpicture}
\end{codeexample}
\begin{codeexample}[]
\begin{tikzpicture}[double distance=3pt]
  \foreach \lw in {0.5,1,1.5,2,2.5}
    \draw[line width=\lw pt,double] (\lw,0) -- ++(4mm,0);
\end{tikzpicture}
\end{codeexample}
\end{key}

\begin{stylekey}{/tikz/double equal sign distance}
  This style selects a double line distance such that it corresponds
  to the distance of the two lines in an equal sign.
\begin{codeexample}[]
\Huge $=\implies$\tikz[baseline,double equal sign distance]
                    \draw[double,thick,-implies](0,0.55ex) --++(3ex,0);
\end{codeexample}
\begin{codeexample}[]
\normalsize $=\implies$\tikz[baseline,double equal sign distance]
                          \draw[double,-implies](0,0.6ex) --++(3ex,0);
\end{codeexample}
\begin{codeexample}[]
\tiny $=\implies$\tikz[baseline,double equal sign distance]
                   \draw[double,very thin,-implies](0,0.5ex) -- ++(3ex,0);
\end{codeexample}  
\end{stylekey}

  




\subsection{Filling a Path}
\label{section-rules}
To fill a path, use the following option:
\begin{key}{/tikz/fill=\meta{color} (default \normalfont is scope's color setting)}
  This option causes the path to be filled. All unclosed parts of the
  path are first closed, if necessary. Then, the area enclosed by the
  path is filled with the current filling color, which is either the
  last color set using the general |color=| option or the optional
  color \meta{color}. For self-intersection paths and for paths
  consisting of several closed areas, the ``enclosed area'' is
  somewhat complicated to define and two different definitions exist,
  namely the nonzero winding number rule and the even odd rule, see
  the explanation of these options, below.

  Just as for the |draw| option, setting \meta{color} to |none|
  disables filling locally.

\begin{codeexample}[]
\begin{tikzpicture}
  \fill (0,0) -- (1,1) -- (2,1);
  \fill (4,0) circle (.5cm)  (4.5,0) circle (.5cm);
  \fill[even odd rule] (6,0) circle (.5cm)  (6.5,0) circle (.5cm);
  \fill (8,0) -- (9,1) -- (10,0) circle (.5cm);
\end{tikzpicture}
\end{codeexample}

  If the |fill| option is used together with the |draw| option (either
  because both are given as options or because a |\filldraw| command
  is used), the path is filled \emph{first}, then the path is drawn
  \emph{second}. This is especially useful if different colors are
  selected for drawing and for filling. Even if the same color is
  used, there is a difference between this command and a plain 
  |fill|: A ``filldrawn'' area will be slightly larger than a filled
  area because of the thickness of the ``pen.''

\begin{codeexample}[]
\begin{tikzpicture}[fill=examplefill,line width=5pt]
  \filldraw (0,0) -- (1,1) -- (2,1);
  \filldraw (4,0) circle (.5cm)  (4.5,0) circle (.5cm);
  \filldraw[even odd rule] (6,0) circle (.5cm)  (6.5,0) circle (.5cm);
  \filldraw (8,0) -- (9,1) -- (10,0) circle (.5cm);
\end{tikzpicture}
\end{codeexample}
\end{key}



\subsubsection{Graphic Parameters: Fill Pattern}

\label{section-fill-pattern}
Instead of filling a path with a single solid color, it is also
possible to fill it with a \emph{tiling pattern}. Imagine a small tile
that contains a simple picture like a star. Then these tiles are
(conceptually) repeated infinitely in all directions, but clipped
against the path.

Tiling patterns come in two variants: \emph{inherently 
  colored patterns} and \emph{form-only patterns}. An inherently colored
pattern is, say, a red star with a black border and will always look
like this. A form-only pattern may have a different color each time
it is used, only the form of the pattern will stay the same. As such,
form-only patterhns do not have any colors of their own, but when it
is used the current \emph{pattern color} is used as its color.

Patterns are not overly flexible. In particular, it is not possible to
change the size or orientation of a pattern without declaring a new
pattern. For complicated case, it may be easier to use two nested
|\foreach| statements to simulate a pattern, but patterns are rendered
\emph{much} more quickly than simulated ones.

\begin{key}{/tikz/pattern=\meta{name} (default \normalfont is scope's pattern)}
  This option causes the path to be filled with a pattern. If the
  \meta{name} is given, this pattern is used, otherwise the pattern
  set in the enclosing scope is used. As for the |draw| and |fill|
  options, setting \meta{name} to |none| disables filling locally.

  The pattern works like a fill color. In particular, setting a new
  fill color will fill the path with a solid color once more.

  Strangely, no \meta{name}s are permissible by default. You need to
  load for instance |pgflibrarypatterns|, see
  Section~\ref{section-library-patterns}, to install predefined
  patterns.  
  
\begin{codeexample}[]
\begin{tikzpicture}
  \draw[pattern=dots] (0,0) circle (1cm);
  \draw[pattern=fivepointed stars] (0,0) rectangle (3,1);
\end{tikzpicture}
\end{codeexample}
\end{key}

\begin{key}{/tikz/pattern color=\meta{color}}
  This option is used to set the color to be used for form-only
  patterns. This option has no effect on inherently colored patterns. 
  
\begin{codeexample}[]
\begin{tikzpicture}
  \draw[pattern color=red,pattern=fivepointed stars]  (0,0) circle (1cm);
  \draw[pattern color=blue,pattern=fivepointed stars] (0,0) rectangle (3,1);
\end{tikzpicture}
\end{codeexample}

\begin{codeexample}[]
\begin{tikzpicture}
  \def\mypath{(0,0) -- +(0,1) arc (180:0:1.5cm) -- +(0,-1)}
  \fill   [red]                                \mypath;
  \pattern[pattern color=white,pattern=bricks] \mypath;
\end{tikzpicture}
\end{codeexample}
\end{key}


\subsubsection{Graphic Parameters: Interior Rules}

The following two options can be used to decide how interior points
should be determined:
\begin{key}{/tikz/nonzero rule}
  If this rule is used (which is the default), the following method is
  used to determine whether a given point is ``inside'' the path: From
  the point, shoot a ray in some direction towards infinity (the
  direction is chosen such that no strange borderline cases
  occur). Then the ray may hit the path. Whenever it hits the path, we
  increase or decrease a counter, which is initially zero. If the ray
  hits the path as the path goes ``from left to right'' (relative to
  the ray), the counter is increased, otherwise it is decreased. Then,
  at the end, we check whether the counter is nonzero (hence the
  name). If so, the point is deemed to lie ``inside,'' otherwise it is
  ``outside.'' Sounds complicated? It is.

\begin{codeexample}[]
\begin{tikzpicture}
  \filldraw[fill=examplefill]
  % Clockwise rectangle
  (0,0) -- (0,1) -- (1,1) -- (1,0) -- cycle
  % Counter-clockwise rectangle
  (0.25,0.25) -- (0.75,0.25) -- (0.75,0.75) -- (0.25,0.75) -- cycle;

  \draw[->] (0,1) -- (.4,1);
  \draw[->] (0.75,0.75) -- (0.3,.75);

  \draw[->] (0.5,0.5) -- +(0,1) node[above] {crossings: $-1+1 = 0$};

  \begin{scope}[yshift=-3cm]
    \filldraw[fill=examplefill]
    % Clockwise rectangle
    (0,0) -- (0,1) -- (1,1) -- (1,0) -- cycle
    % Clockwise rectangle
    (0.25,0.25) -- (0.25,0.75) -- (0.75,0.75) -- (0.75,0.25) -- cycle;

    \draw[->] (0,1) -- (.4,1);
    \draw[->] (0.25,0.75) -- (0.4,.75);
      
    \draw[->] (0.5,0.5) -- +(0,1) node[above] {crossings: $1+1 = 2$};
  \end{scope}
\end{tikzpicture}
\end{codeexample}
\end{key}

\begin{key}{/tikz/even odd rule}
  This option causes a different method to be used for determining the
  inside and outside of paths. While it is less flexible, it turns out
  to be more intuitive.

  With this method, we also shoot rays from the point for which we
  wish to determine whether it is inside or outside the filling
  area. However, this time we only count how often we ``hit'' the path
  and declare the point to be ``inside'' if the number of hits is odd.

  Using the even-odd rule, it is easy to ``drill holes'' into a path.
  
\begin{codeexample}[]
\begin{tikzpicture}
  \filldraw[fill=examplefill,even odd rule]
    (0,0) rectangle (1,1) (0.5,0.5) circle (0.4cm);
  \draw[->] (0.5,0.5) -- +(0,1) [above] node{crossings: $1+1 = 2$};
\end{tikzpicture}
\end{codeexample}
\end{key}



\subsubsection{Graphic Parameters: Fill Opacity}

\label{section-fill-opacity}
Analogously to the |draw opacity|, you can also set the filling
opacity. Please see Section~\ref{section-tikz-transparency} for more
details. 


\subsection{Generalized Filling: Using Arbitrary Pictures to Fill a Path}

Sometimes you wish to ``fill'' a path with something even more
complicated than a pattern, let alone a single color. For instance,
you might wish to use an image to fill the path or some other,
complicated drawing. In principle, this effect can be achieved
by first using the path for clipping and then, subsequently, drawing
the desired image or picture. However, there is an option that makes
this process much easier:

\begin{key}{/tikz/path picture=\meta{code}}
  When this option is given on a path and when the \meta{code} is not
  empty, the following happens: After all other ``filling'' operations
  are done with the path, which are caused by the options |fill|,
  |pattern| and  |shade|, a local scope is opened and the path is
  temporarily installed as a clipping path. Then, the \meta{code} is
  executed, which can now draw something. Then, the local scope ends
  and, possibly, the path is stroked, provided the |draw| option has
  been given.

  As with other keys like |fill| or |draw| this option needs to be
  given on a path, setting the |path picture| outside a path has not
  effect (the path picture is cleared at the beginning of each path).

  The \meta{code} can be any normal \tikzname\ code like |\draw ...|
  or |\node ...|. As always, when you include an external graphic you
  need to put it inside a |\node|.

  Note that no special actions are taken to transform the origin in
  any way. This means that the coordinate |(0,0)| is still where is
  was when the path was being constructed and not -- as one might
  expect -- at the lower left corner of the path. However, you can use
  the followin special node to access the size of the path:
  \begin{predefinednode}{path picture bounding box}
    This node is of shape |rectangle|. Its size and position are those
    of |current path bounding box| just before the \meta{code}
    of the path picture started to be executed. The \meta{code} can
    construct its own paths, so accessing the 
    |current path bounding box| inside the \meta{code} yields the
    bounding box of any path that is currently being constructed
    inside the \meta{code}.
  \end{predefinednode}

\begin{codeexample}[]
\begin{tikzpicture}
  \draw [help lines] (0,0) grid (3,2);
  \filldraw [fill=blue!10,draw=blue,thick] (1.5,1) circle (1)
    [path picture={
      \node at (path picture bounding box.center) {
        This is a loong text.
      };}
    ];
\end{tikzpicture}
\end{codeexample}

\begin{codeexample}[]
\begin{tikzpicture}[cross/.style={path picture={
      \draw[black]
            (path picture bounding box.south east) --
            (path picture bounding box.north west)
            (path picture bounding box.south west) --
            (path picture bounding box.north east); 
    }}]    
  \draw [help lines] (0,0) grid (3,2);
  \filldraw [cross,fill=blue!10,draw=blue,thick] (1,1) circle (1);
  \path     [cross,top color=red,draw=red,thick] (2,0) -- (3,2) -- (3,0);
\end{tikzpicture}
\end{codeexample}

\begin{codeexample}[]
  \begin{tikzpicture}[path image/.style={
      path picture={
        \node at (path picture bounding box.center) {
          \includegraphics[height=3cm]{#1}
        };}}]
  \draw     [help lines] (0,0) grid (3,2);
  
  \draw [path image=brave-gnu-world-logo.jpg,draw=blue,thick]
          (0,1) circle (1);
  \draw [path image=brave-gnu-world-logo.jpg,draw=red,very thick,->]
          (1,0) parabola[parabola height=2cm] (3,0);
  
\end{tikzpicture}
\end{codeexample}
\end{key}


\subsection{Shading a Path}

You can shade a path using the |shade| option. A shading is like a
filling, only the shading changes its color smoothly from one color to
another.

\begin{key}{/tikz/shade}
  Causes the path to be shaded using the currently selected shading
  (more on this later). If this option is used together with the
  |draw| option, then the path is first shaded, then drawn.

  It is not an error to use this option together with the |fill|
  option, but it makes no sense.

\begin{codeexample}[]
\tikz \shade (0,0) circle (1ex);
\end{codeexample}

\begin{codeexample}[]
\tikz \shadedraw (0,0) circle (1ex);
\end{codeexample}
\end{key}

For some shadings it is not really clear how they can ``fill'' the
path. For example, the |ball| shading normally looks like this: \tikz
\shade[shading=ball] (0,0) circle (0.75ex);. How is this supposed to
shade a rectangle? Or a triangle?

To solve this problem, the predefined shadings like |ball| or |axis|
fill a large rectangle completely in a sensible way. Then, when the
shading is used to ``shade'' a path, what actually happens is that the
path is temporarily used for clipping and then the rectangular shading
is drawn, scaled and shifted such that all parts of the path are
filled.

The default shading is a smooth transition from gray
to white and from above to bottom. However, other shadings are also
possible, for example a shading that will sweep a color from the
center to the corners outward. To choose the shading, you can use the
|shading=| option, which will also automatically invoke the |shade|
option. Note that this does \emph{not} change the shading color, only
the way the colors sweep. For changing the colors, other options are
needed, which are explained below.

\begin{key}{/tikz/shading=\meta{name}}
  This selects a shading named \meta{name}. The following shadings are
  predefined: |axis|, |radial|, and |ball|.
\begin{codeexample}[]
\tikz \shadedraw [shading=axis] (0,0) rectangle (1,1);
\tikz \shadedraw [shading=radial] (0,0) rectangle (1,1);
\tikz \shadedraw [shading=ball] (0,0) circle (.5cm);
\end{codeexample}

  The shadings as well as additional shadings are described in more
  detail in Section~\ref{section-library-shadings}.

  To change the color of a shading, special options are needed like
  |left color|, which sets the color of an axis shading from left to
  right. These options implicitly also select the right shading type,
  see the following example
\begin{codeexample}[]
\tikz \shadedraw [left color=red,right color=blue]
    (0,0) rectangle (1,1);
\end{codeexample}

  For a complete list of the possible options see
  Section~\ref{section-library-shadings} once more.

  \begin{key}{/tikz/shading angle=\meta{degrees} (initially 0)}
    This option rotates the shading (not the path!) by the given
    angle. For example, we can turn a top-to-bottom axis shading into a
    left-to-right shading by rotating it by $90^\circ$.

\begin{codeexample}[]
\tikz \shadedraw [shading=axis,shading angle=90] (0,0) rectangle (1,1);
\end{codeexample}
  \end{key}
\end{key}

You can also define new shading types yourself. However, for this, you
need to use the basic layer directly, which is, well, more basic and
harder to use. Details on how to create a shading appropriate for
filling paths are given in Section~\ref{section-shading-a-path}.



\subsection{Establishing a Bounding Box}

\pgfname\ is reasonably good at keeping track of the size of your picture
and reserving just the right amount of space for it in the main
document. However, in some cases you may want to say things like
``do not count this for the picture size'' or ``the picture is
actually a little large.'' For this you can use the option
|use as bounding box| or the command |\useasboundingbox|, which is just
a shorthand for |\path[use as bounding box]|.

\begin{key}{/tikz/use as bounding box}
  Normally, when this option is given on a path, the bounding box of
  the present path is used to determine the size of the picture and
  the size of all \emph{subsequent} paths are
  ignored. However, if there were previous path operations that have
  already established a larger bounding box, it will not be made
  smaller by this operation (consider the |\pgfresetboundingbox| command
  to reset the previous bounding box).

  In a sense, |use as bounding box| has the same effect as clipping
  all subsequent drawing against the current path---without actually
  doing the clipping, only making \pgfname\ treat everything as if it
  were clipped.

  The first application of this option is to have a |{tikzpicture}|
  overlap with the main text:

\begin{codeexample}[]
Left of picture\begin{tikzpicture}
  \draw[use as bounding box] (2,0) rectangle (3,1);
  \draw (1,0) -- (4,.75);
\end{tikzpicture}right of picture.
\end{codeexample}

  In a second application this option can be used to get better
  control over the white space around the picture:
  
\begin{codeexample}[]
Left of picture
\begin{tikzpicture}
  \useasboundingbox (0,0) rectangle (3,1);
  \fill (.75,.25) circle (.5cm);
\end{tikzpicture}
right of picture.
\end{codeexample}

  Note: If this option is used on a path inside a \TeX\ group (scope),
  the effect ``lasts'' only till the end of the scope. Again, this
  behavior is the same as for clipping.


  Consider using |\useasboundingbox| together with |\pgfresetboundingbox| in order to replace the bounding box with a new one.
\end{key}

There is a node that allows you to get the size of the current
bounding box. The |current bounding box| node has the |rectangle|
shape and its size is always the size of the current 
bounding box.

Similarly, the |current path bounding box| node has the |rectangle|
hape and the size of the bounding box of the current path.


\begin{codeexample}[]
\begin{tikzpicture}
  \draw[red] (0,0) circle (2pt);
  \draw[red] (2,1) circle (3pt);

  \draw (current bounding box.south west) rectangle
        (current bounding box.north east);

  \draw[red] (3,-1) circle (4pt);

  \draw[thick] (current bounding box.south west) rectangle
               (current bounding box.north east);
\end{tikzpicture}
\end{codeexample}


Occasionally, you may want to align multiple |tikzpicture| environments horizontally and/or vertically at some prescribed position. The vertical alignment can be realized by means of the |baseline| option since \TeX\ supports the concept of box depth natively. For horizontal alignment, things are slightly more involved. The following approach is realized by means of negative |\hspace|s before and/or after the picture, thereby removing parts of the picture. However, the actual amount of negative horizontal space is provided by means of image coordinates using the |trim left| and |trim right| keys:

\begin{key}{/tikz/trim left=\meta{dimension or coordinate or \texttt{default}} (default 0pt)}
	The |trim left| key tells \pgfname\space to discard everything which is left of the provided \meta{dimension or coordinate}. Here, \meta{dimension} is a single $x$ coordinate of the picture and \meta{coordinate} is a point with $x$ and $y$ coordinates (but only its $x$ coordinate will be used). The effect is the same as if you issue |\hspace{-s}| where |s| is the difference of the picture's bounding box lower left $x$ coordinate and the $x$ coordinate specified as \meta{dimension or coordinate}:
\begin{codeexample}[]
Text before image.%
	\begin{tikzpicture}[trim left]
		\draw (-1,-1) grid (3,2);
		\fill (0,0) circle (5pt);
	\end{tikzpicture}%
Text after image.
\end{codeexample}
	Since |trim left| uses the default |trim left=0pt|, everything left of $x=0$ is removed from the bounding box.

	The following example has once the relative long label $-1$ and once the shorter label $1$. Horizontal alignment is established with |trim left|:
\begin{codeexample}[pre={\vbox\bgroup\hsize=5cm},post=\egroup,width=8cm]
\begin{tikzpicture}
	\draw (0,1) -- (0,0) -- (1,1) -- cycle;
	\fill (0,0) circle (2pt);
	\node[left] at (0,0) {$-1$};
\end{tikzpicture}
\par
\begin{tikzpicture}
	\draw (0,1) -- (0,0) -- (1,1) -- cycle;
	\fill (0,0) circle (2pt);
	\node[left] at (0,0) {$1$};
\end{tikzpicture}
\par
\begin{tikzpicture}[trim left]
	\draw (0,1) -- (0,0) -- (1,1) -- cycle;
	\fill (0,0) circle (2pt);
	\node[left] at (0,0) {$-1$};
\end{tikzpicture}
\par
\begin{tikzpicture}[trim left]
	\draw (0,1) -- (0,0) -- (1,1) -- cycle;
	\fill (0,0) circle (2pt);
	\node[left] at (0,0) {$1$};
\end{tikzpicture}
\end{codeexample}
	
	Use |trim left=default| to reset the value.
\end{key}

\begin{key}{/tikz/trim right=\meta{dimension or coordinate or \texttt{default}}}
	This key is similar to |trim left|: it discards everything which is right of the provided \meta{dimension or coordinate}. As for |trim left|, \meta{dimension} denotes a single $x$ coordinate of the picture and \meta{coordinate} a coordinate with $x$ and $y$ value (although only its $x$ component will be used).

	We use the same example from above and add |trim right|:
\begin{codeexample}[]
Text before image.%
	\begin{tikzpicture}[trim left, trim right=2cm, baseline]
		\draw (-1,-1) grid (3,2);
		\fill (0,0) circle (5pt);
	\end{tikzpicture}%
Text after image.
\end{codeexample}
	In addition to |trim left=0pt|, we also discard everything which is right of $x$|=2cm|. Furthermore, the |baseline| key supports vertical alignment as well (using the $y$|=0cm| baseline).

	Use |trim right=default| to reset the value.
\end{key}

Note that |baseline|, |trim left| and |trim right| are currently the \emph{only} supported way of truncated bounding boxes which are compatible with image externalization (see the |external| library for details).

\begin{key}{/pgf/trim lowlevel=\mchoice{true,false} (initially false)}
	This affects only the basic level image externalization: the initial configuration |trim lowlevel=false| stores the normal image, without trimming, and the trimming into a separate file. This allows reduced bounding boxes without clipping the rest away. The |trim lowlevel=true| information causes the image externalization to store the trimmed image, possibly resulting in clipping.
\end{key}

\subsection{Clipping and Fading (Soft Clipping)}

\emph{Clipping path} means that all painting on the page is restricted
to a certain area. This area need not be rectangular, rather an
arbitrary path can be used to specify this area. The |clip| option,
explained below, is used to specify the region that is to be used for
clipping.

A \emph{fading} (a term that I propose, fadings are commonly known
as soft masks, transparency masks, opacity masks or soft clips) is
similar to clipping, but a fading allows parts of the picture to be
only ``half clipped.'' This means that a fading can specify that newly
painted pixels should be partly transparent. The specification
and handling of fadings is a bit complex and it is detailed in
Section~\ref{section-tikz-transparency}, which is devoted to
transparency in general.

\begin{key}{/tikz/clip}
  This option causes all subsequent drawings to be clipped against the
  current path and the size of subsequent paths will not be important
  for the picture size.  If you clip against a self-intersecting path,
  the even-odd rule or  the nonzero winding number rule is used to
  determine whether a point is inside or outside the clipping region.

  The clipping path is a graphic state parameter, so it will be reset
  at the end of the current scope. Multiple clippings accumulate, that
  is, clipping is always done against the intersection of all clipping
  areas that have been specified inside the current scopes. The only
  way of enlarging the clipping area is to end a |{scope}|.

\begin{codeexample}[]
\begin{tikzpicture}
  \draw[clip] (0,0) circle (1cm);
  \fill[red] (1,0) circle (1cm);
\end{tikzpicture}
\end{codeexample}

  It  is usually a \emph{very} good idea to apply the |clip| option only
  to the first path command in a scope. 

  If you ``only wish to clip'' and do not wish to draw anything, you can
  use the |\clip| command, which is a shorthand for |\path[clip]|.

\begin{codeexample}[]
\begin{tikzpicture}
  \clip (0,0) circle (1cm);
  \fill[red] (1,0) circle (1cm);
\end{tikzpicture}
\end{codeexample}

  To keep clipping local, use |{scope}| environments as in the
  following example:

\begin{codeexample}[]
\begin{tikzpicture}
  \draw (0,0) -- ( 0:1cm);
  \draw (0,0) -- (10:1cm);
  \draw (0,0) -- (20:1cm);
  \draw (0,0) -- (30:1cm);
  \begin{scope}[fill=red]
    \fill[clip] (0.2,0.2) rectangle (0.5,0.5);
    
    \draw (0,0) -- (40:1cm);
    \draw (0,0) -- (50:1cm);
    \draw (0,0) -- (60:1cm);
  \end{scope}
  \draw (0,0) -- (70:1cm);
  \draw (0,0) -- (80:1cm);
  \draw (0,0) -- (90:1cm);
\end{tikzpicture}
\end{codeexample}

  There is a slightly annoying catch: You cannot specify certain graphic
  options for the command used for clipping. For example, in the above
  code we could not have moved the |fill=red| to the |\fill|
  command. The reasons for this have to do with the internals of the
  \pdf\ specification. You do not want to know the details. It is best
  simply not to specify any options for these 
  commands. 
\end{key}



\subsection{Doing Multiple Actions on a Path}

If more than one of the basic actions like drawing, clipping and
filling are requested, they are automatically applied in a sensible
order: First, a path is filled, then drawn, and then clipped (although
it took Apple two mayor revisions of their operating system to get
this right\dots). Sometimes, however, you need finer control over what
is done with a path. For instance, you might wish to first fill a path
with a color, then repaint the path with a pattern and then repaint it
with yet another pattern. In such cases you can use the following two
options:

\begin{key}{/tikz/preactions=\meta{options}}
  This option can be given to a |\path| command (or to derived
  commands like |\draw| which internally call |\path|). Similarly to
  options like |draw|, this option only has an effect when given to a
  |\path| or as part of the options of a |node|; as an option to a
  |{scope}| it has no effect.

  When this option is used on a |\path|, the effect is the following:
  When the path has been completely constructed and is about to be
  used, a scope is created. Inside this scope, the path is used but
  not with the original path options, but with \meta{options}
  instead. Then, the path is used in the usual manner. In other words,
  the path is used twice: Once with \meta{options} in force and then
  again with the normal path options in force.

  Here is an example in which the path consists of a rectangle. The
  main action is to draw this path in red (which is why we see a red
  rectangle). However, the preaction is to draw the path in blue,
  which is why we see a blue rectangle behind the red rectangle.
\begin{codeexample}[]
\begin{tikzpicture}
  \draw[help lines] (0,0) grid (3,2);

  \draw
    [preaction={draw,line width=4mm,blue}]
    [line width=2mm,red] (0,0) rectangle (2,2);
\end{tikzpicture}
\end{codeexample}

  Note that when the preactions are preformed, then the path is
  already ``finished.'' In particular, applying a coordinate
  transformation to the path has no effect. By comparison, applying a
  canvas transformation does have an effect. Let us use this to add a
  ``shadow'' to a path. For this, we use the preaction to fill the
  path in gray, shifted a bit to the right and down:

\begin{codeexample}[]
\begin{tikzpicture}
  \draw[help lines] (0,0) grid (3,2);
  \draw
    [preaction={fill=black,opacity=.5,
                transform canvas={xshift=1mm,yshift=-1mm}}]
    [fill=red] (0,0) rectangle (1,2)
               (1,2) circle (5mm);
\end{tikzpicture}
\end{codeexample}

  Naturally, you would normally create a style |shadow| that contains
  the above code. The shadow library, see
  Section~\ref{section-libs-shadows}, contains predefined shadows of
  this kind.

  It is possible to use the |preaction| option multiple times. In this
  case, for each use of the |preaction| option, the path is used again
  (thus, the \meta{options} do not accumulate in a single usage of the
  path). The path is used in the order of |preaction| options given.

  In the following example, we use one |preaction| to add a shadow and
  another to provide a shading, while the main action is to use a
  pattern. 
\begin{codeexample}[]
\begin{tikzpicture}
  \draw[help lines] (0,0) grid (3,2);
  \draw [pattern=fivepointed stars]
    [preaction={fill=black,opacity=.5,
                transform canvas={xshift=1mm,yshift=-1mm}}]
    [preaction={top color=blue,bottom color=white}]
               (0,0) rectangle (1,2)
               (1,2) circle (5mm);
\end{tikzpicture}
\end{codeexample}

  A complicated application is shown in the following example, where
  the path is used several times with different fadings and shadings
  to create a special visual effect:
\begin{codeexample}[]
\begin{tikzpicture}
  [
    % Define an interesting style
    button/.style={
      % First preaction: Fuzzy shadow
      preaction={fill=black,path fading=circle with fuzzy edge 20 percent,
                 opacity=.5,transform canvas={xshift=1mm,yshift=-1mm}},
      % Second preaction: Background pattern
      preaction={pattern=#1,
                 path fading=circle with fuzzy edge 15 percent},
      % Third preaction: Make background shiny
      preaction={top color=white,
                 bottom color=black!50,
                 shading angle=45,
                 path fading=circle with fuzzy edge 15 percent,
                 opacity=0.2},
      % Fourth preaction: Make edge especially shiny
      preaction={path fading=fuzzy ring 15 percent,
                 top color=black!5,
                 bottom color=black!80,
                 shading angle=45},
      inner sep=2ex
    },
    button/.default=horizontal lines light blue,
    circle
  ]

  \draw [help lines] (0,0) grid (4,3);

  \node [button] at (2.2,1) {\Huge Big};
  \node [button=crosshatch dots light steel blue,
         text=white] at (1,1.5) {Small};
\end{tikzpicture}
\end{codeexample}
\end{key}

\begin{key}{/tikz/postaction=\meta{options}}
  The postactions work in the same way as the preactions, only they
  are applied \emph{after} the main action has been taken. Like
  preactions, multiple |postaction| options may be given to a |\path|
  command, in which case the path is reused several times, each time
  with a different set of options in force.

  If both pre- and postactions are specified, then the preactions are
  taken first, then the main action, and then the post actions.

  In the first example, we use a postaction to draw the path, after it
  has already been drawn:
\begin{codeexample}[]
\begin{tikzpicture}
  \draw[help lines] (0,0) grid (3,2);

  \draw
    [postaction={draw,line width=2mm,blue}]
    [line width=4mm,red,fill=white] (0,0) rectangle (2,2);
\end{tikzpicture}
\end{codeexample}

  In another example, we use a postaction to ``colorzie'' a path: 

\begin{codeexample}[]
\begin{tikzpicture}
  \draw[help lines] (0,0) grid (3,2);
  \draw
    [postaction={path fading=south,fill=white}]
    [postaction={path fading=south,fading angle=45,fill=blue,opacity=.5}]
    [left color=black,right color=red,draw=white,line width=2mm]
               (0,0) rectangle (1,2)
               (1,2) circle (5mm);
\end{tikzpicture}
\end{codeexample}
\end{key}



\subsection{Decorating and Morphing a Path}

Before a path is used, it is possible to first ``decorate'' and/or
``morph'' it. Morphing means that the path is replaced by another path
that slightly varied. Such morphings are a special case
of the more general ``decorations'' described in detail in
Section~\ref{section-tikz-decorations}. For instance, in the following
example the path is drawn twice: Once normally and then in a morphed
(=decorated) manner. 
\begin{codeexample}[]
\begin{tikzpicture}
  \draw (0,0) rectangle (3,2);
  \draw [red, decorate, decoration=zigzag]
        (0,0) rectangle (3,2);
\end{tikzpicture}
\end{codeexample}

Naturally, we could have combined this into a single command using
pre- or postaction. It is also possible to deform shapes:
\begin{codeexample}[]
\begin{tikzpicture}
  \node [circular drop shadow={shadow scale=1.05},minimum size=3.13cm,
         decorate, decoration=zigzag,
         fill=blue!20,draw,thick,circle] {Hello!};
\end{tikzpicture}
\end{codeexample}


% Copyright 2006 by Till Tantau
%
% This file may be distributed and/or modified
%
% 1. under the LaTeX Project Public License and/or
% 2. under the GNU Free Documentation License.
%
% See the file doc/generic/pgf/licenses/LICENSE for more details.

\section{Nodes and Edges}

\label{section-nodes}

\subsection{Overview}

In the present section, the usage of \emph{nodes} in
\tikzname\ is explained. A node is typically a rectangle or circle or
another simple shape with some text on it. 

Nodes are added to paths using the special path
operation |node|. Nodes \emph{are not part of the path
  itself}. Rather, they are added to the picture after the path has
been drawn. 

In Section~\ref{section-nodes-basic} the basic syntax of the node
operation is explained, followed in Section~\ref{section-nodes-multi}
by the syntax for multi-part nodes, which are nodes that contain
several different text parts. After this, the different options for
the text in nodes are explained. In
Section~\ref{section-nodes-anchors} the concept of \emph{anchors} is
introduced along with their usage. In
Section~\ref{section-nodes-transformations} the different ways
transformations affect nodes are
studied. Sections~\ref{section-nodes-placing-1}
and~\ref{section-nodes-placing-2} are about placing nodes on or next
to straight lines and curves. In
Section~\ref{section-nodes-connecting} it is explained how a node can
be used as a ``pseudo-coordinate.'' Section~\ref{section-nodes-edges}
introduces the |edge| operation, which
works similar to the |to| operation and also similar to the |node|
operation. Finally, Section~\ref{section-nodes-executing} explains the
special |after node path| options.


\subsection{Nodes and Their Shapes}

\label{section-nodes-basic}

In the simplest case, a node is just some text that is
placed at some coordinate. However, a node can also have a border
drawn around it or have a more complex background and
foreground. Indeed, some nodes do not have a text at all, but consist
solely of the background. You can name nodes so that you can reference
their coordinates later in the same picture or, if certain precautions
are taken as explained in Section~\ref{section-cross-picture-tikz},
also in different pictures.

There are no special \TeX\ commands for adding a node to a picture; rather,
there is path operation called |node| for this. Nodes are created
whenever \tikzname\ encounters |node| or |coordinate| at a point on a
path where it would expect a normal path operation (like |-- (1,1)| or
|sin (1,1)|). It is also possible to give node specifications
\emph{inside} certain path operations as explained later.

The node operation is typically followed by some options, which apply
only to the node. Then, you can optionally \emph{name} the node by
providing a name in round braces. Lastly, for the |node| operation you
must provide some label text for the node in curly braces, while for
the |coordinate| operation you may not. The node is placed at the
current position of the path \emph{after the path has been
  drawn}. Thus, all nodes are drawn ``on top'' of the path and
retained until the path is complete. If there are several nodes on a
path, they are drawn on top of the path in the order they are
encountered. 

\begin{codeexample}[]
\tikz \fill[fill=examplefill]
     (0,0) node {first node}
  -- (1,1) node {second node}
  -- (0,2) node {third node};
\end{codeexample}

The syntax for specifying nodes is the following:
\begin{pathoperation}{node}{\opt{|[|\meta{options}|]|}\opt{|(|\meta{name}|)|}%
    \opt{|at(|\meta{coordinate}|)|}\opt{\marg{text}}}
  The effect of |at| is to place the node at the coordinate given
  after |at| and not, as would normally be the case, at the last
  position. The |at| syntax is not available when a node is given
  inside a path operation (it would not make any sense, there).
  
  The |(|\meta{name}|)| is a name for later reference and it is
  optional. You may also add the option |name=|\meta{name} to the
  \meta{option} list; it has the same effect.

  \begin{key}{/tikz/name=\meta{node name}}
    Assigns a name to the node for later reference. Since this is a
    ``high-level'' name (drivers never know of it), you can use spaces,
    number, letters, or whatever you like when naming a node. Thus, you
    can name a node just |1| or perhaps |start of chart| or even
    |y_1|. Your node name should \emph{not} contain any punctuation like
    a dot, a comma, or a colon since these are used to detect what kind
    of coordinate you mean when you reference a node. 
  \end{key}

  \begin{key}{/tikz/at=\meta{coordinate}}
    This is another way of specifying ath |at| coordinate. Note that,
    typically, you will have to enclose the \meta{coordinate} in curly
    braces so that a comma inside the \meta{coordinate} does not
    confuse \TeX.
  \end{key}

  The \meta{options} is an optional list of options that \emph{apply
    only to the node} and have no effect outside. The other way round,
  most ``outside'' options also apply to the node, but not all. For
  example, the ``outside'' rotation does not apply to nodes (unless some
  special options are used, sigh). Also, the outside path action, like
  |draw| or |fill|, never applies to the node and must be given in the
  node (unless some special other options are used, deep sigh).

  As mentioned before, we can add a border and even a background to a
  node:  
\begin{codeexample}[]
\tikz \fill[fill=examplefill]
      (0,0) node {first node}
   -- (1,1) node[draw] {second node}
   -- (0,2) node[fill=red!20,draw,double,rounded corners] {third node};
\end{codeexample}

  The ``border'' is actually just a special case of a much more general
  mechanism. Each node has a certain \emph{shape} which, by default, is
  a rectangle. However, we can also ask \tikzname\ to use a circle shape
  instead or an ellipse shape (you have to include |pgflibraryshapes| for
  the latter shape): 

\begin{codeexample}[]
\tikz \fill[fill=examplefill]
      (0,0) node{first node}
   -- (1,1) node[ellipse,draw] {second node}
   -- (0,2) node[circle,fill=red!20] {third node};
\end{codeexample}

  In the future, there might be much more complicated shapes available
  such as, say, a shape for a resistor or a shape for a \textsc{uml}
  class. Unfortunately, creating new shapes is a bit tricky and makes
  it necessary to use the basic layer directly. Life is hard.

  To select the shape of a node, the following option is used:
  \begin{key}{/tikz/shape=\meta{shape name} (initially rectangle)}
    Select the shape either of the current node or, when this option is
    not given inside a node but somewhere outside, the shape of all
    nodes in the current scope.%
    \indexoption{\meta{shape name}}

    Since this option is used often, you can leave out the
    |shape=|. When \tikzname\ encounters an option like |circle|
    that it does not know, it will, after everything else has failed,
    check whether this option is the name of some shape. If so, that
    shape is selected as if you had said |shape=|\meta{shape name}.

    By default, the following shapes are available: |rectangle|,
    |circle|, |coordinate|, and, when the package |pgflibraryshapes| is
    loaded, also |ellipse|. Details of these shapes, like their anchors
    and size options, are discussed in Section~\ref{section-the-shapes}.
  \end{key}
  
  The following styles influences how nodes are rendered:
  \begin{stylekey}{/tikz/every node (initially \normalfont empty)}
    This style is installed at the beginning of every node. 
\begin{codeexample}[]
\begin{tikzpicture}[every node/.style={draw}]
  \draw (0,0) node {A} -- (1,1) node {B};
\end{tikzpicture}
\end{codeexample}
  \end{stylekey}
  \begin{stylekey}{/tikz/every \meta{shape} node (initially \normalfont empty)}
    These styles are installed at the beginning of a node of a given
    \meta{shape}. For example, |every rectangle node| is used for
    rectangle nodes, and so on.
\begin{codeexample}[]
\begin{tikzpicture}
  [every rectangle node/.style={draw}, 
   every circle node/.style={draw,double}]
  \draw (0,0) node[rectangle] {A} -- (1,1) node[circle] {B};
\end{tikzpicture}
\end{codeexample}
  \end{stylekey}
\end{pathoperation}

There is a special syntax for specifying ``light-weighed'' nodes:

\begin{pathoperation}{coordinate}{\opt{|[|\meta{options}|]|}|(|\meta{name}|)|\opt{|at(|\meta{coordinate}|)|}}
  This has the same effect as

  |node[shape=coordinate][|\meta{options}|](|\meta{name}|)at(|\meta{coordinate}|){}|,
  
  where the |at| part might be missing.
\end{pathoperation}

Since nodes are often the only path operation on paths, there are two
special commands for creating paths containing only a node:

\begin{command}{\node}
  Inside |{tikzpicture}| this is an abbreviation for |\path node|.
\end{command}

\begin{command}{\coordinate}
  Inside |{tikzpicture}| this is an abbreviation for |\path coordinate|.
\end{command}


\subsubsection{Predefined Shapes}

\label{section-nodes-predefined}

\label{section-the-shapes}

\pgfname\ and \tikzname\ define three shapes, by default:
\begin{itemize}
\item
  |rectangle|,
\item
  |circle|, and
\item
  |coordinate|.
\end{itemize}
By loading library packages, you can define more shapes like ellipses
or diamonds; see Section~\ref{section-libs-shapes} for the complete
list of shapes. 

\label{section-tikz-coordinate-shape}
The |coordinate| shape is handled in a special way by \tikzname. When
a node |x| whose shape is |coordinate| is used as a coordinate |(x)|,
this has the same effect as if you had said |(x.center)|. None  of the
special ``line shortening rules'' apply in this case. This can be
useful since, normally, the line shortening causes paths to be
segmented and they cannot be used for filling. Here is an example that
demonstrates the difference: 
\begin{codeexample}[]
\begin{tikzpicture}[every node/.style={draw}]
  \path[yshift=1.5cm,shape=rectangle]
    (0,0) node(a1){} (1,0) node(a2){}
    (1,1) node(a3){} (0,1) node(a4){};
  \filldraw[fill=examplefill] (a1) -- (a2) -- (a3) -- (a4);
  
  \path[shape=coordinate]
    (0,0) coordinate(b1) (1,0) coordinate(b2)
    (1,1) coordinate(b3) (0,1) coordinate(b4);
  \filldraw[fill=examplefill] (b1) -- (b2) -- (b3) -- (b4);
\end{tikzpicture}
\end{codeexample}



\subsubsection{Common Options: Separations, Margins, Padding and
  Border Rotation}

\label{section-shape-seps}
\label{section-shape-common-options}

The exact behaviour of shapes differs, shapes defined for more
special purposes (like a, say, transistor shape) will have even more
custom behaviors. However, there are some options that apply to most
shapes:

\begin{key}{/pgf/inner sep=\meta{dimension} (initially .3333em)}
  \keyalias{tikz}
  An additional (invisible) separation space of \meta{dimension} will
  be added inside the shape, between the text and the shape's
  background path. The effect is as if you had added appropriate
  horizontal and vertical skips at the beginning and end of the text
  to make it a bit ``larger.''

  For those familiar with \textsc{css}, this is the same as
  \emph{padding}. 

\begin{codeexample}[]
\begin{tikzpicture}
  \draw (0,0)     node[inner sep=0pt,draw] {tight}
        (0cm,2em) node[inner sep=5pt,draw] {loose}
        (0cm,4em) node[fill=examplefill]   {default};
\end{tikzpicture}
\end{codeexample}
\end{key}

\begin{key}{/pgf/inner xsep=\meta{dimension} (initially .3333em)}
  \keyalias{tikz}
  Specifies the inner separation in the $x$-direction, only.
\end{key}

\begin{key}{/pgf/inner ysep=\meta{dimension} (initially .3333em)}
  \keyalias{tikz}
  Specifies the inner separation in the $y$-direction, only.
\end{key}  

\begin{key}{/pgf/outer sep=\meta{dimension} (initially .5\string\pgflinewidth)}
  \keyalias{tikz}
  This option adds an additional (invisible) separation space of
  \meta{dimension} outside the background path. The main effect of
  this option is that all anchors will move a little ``to the
  outside.''

  For those familiar with \textsc{css}, this is same as \emph{margin}.

  The default for this option is half the line width. When the default
  is used and when the background path is draw, the anchors will lie
  exactly on the ``outside border'' of the path (not on the path
  itself). When the shape is filled, but not drawn, this may not be
  desirable. In this case, the |outer sep| should be set to zero
  point. 
\begin{codeexample}[]
\begin{tikzpicture}
  \draw[line width=5pt]
    (0,0) node[outer sep=0pt,fill=examplefill]     (f) {filled}
    (2,0) node[inner sep=.5\pgflinewidth+2pt,draw] (d) {drawn};

  \draw[->] (1,-1) -- (f);
  \draw[->] (1,-1) -- (d);  
\end{tikzpicture}
\end{codeexample}
\end{key}

\begin{key}{/pgf/outer xsep=\meta{dimension} (initially .5\string\pgflinewidth)}
  \keyalias{tikz}
  Specifies the outer separation in the $x$-direction, only.
\end{key}

\begin{key}{/pgf/outer ysep=\meta{dimension} (initially .5\string\pgflinewidth)}
  \keyalias{tikz}
  Specifies the outer separation in the $y$-direction, only.
\end{key}

\begin{key}{/pgf/minimum height=\meta{dimension} (initially 0pt)}
  \keyalias{tikz}
  This option ensures that the height of the shape (including the
  inner, but ignoring the outer separation) will be at least
  \meta{dimension}. Thus, if the text plus the inner separation is not
  at least as large as \meta{dimension}, the shape will be enlarged 
  appropriately. However, if the text is already larger than
  \meta{dimension}, the shape will not be shrunk.
\begin{codeexample}[]
\begin{tikzpicture}
  \draw (0,0) node[minimum height=1cm,draw] {1cm}
        (2,0) node[minimum height=0cm,draw] {0cm};
\end{tikzpicture}
\end{codeexample}
\end{key}

\begin{key}{/pgf/minimum width=\meta{dimension} (initially 0pt)}
  \keyalias{tikz}
  Same as |minimum height|, only for the width.
\begin{codeexample}[]
\begin{tikzpicture}
  \draw (0,0) node[minimum height=2cm,minimum width=3cm,draw] {$3 \times 2$};
\end{tikzpicture}
\end{codeexample}
\end{key}

\begin{key}{/pgf/minimum size=\meta{dimension}}
  \keyalias{tikz}
  Sets both the minimum height and width at the same time.
\begin{codeexample}[]
\begin{tikzpicture}
  \draw (0,0)  node[minimum size=2cm,draw] {square};
  \draw (0,-2) node[minimum size=2cm,draw,circle] {circle};
\end{tikzpicture}
\end{codeexample}
\end{key}

\begin{key}{/pgf/shape aspect=\meta{aspect ratio}}
  \keyalias{tikz}
  Sets a desired aspect ratio for the shape. For the |diamond| shape,
  this option sets the ratio between width and height of the shape.
\begin{codeexample}[]
\begin{tikzpicture}
  \draw (0,0)  node[shape aspect=1,diamond,draw] {aspect 1};
  \draw (0,-2) node[shape aspect=2,diamond,draw] {aspect 2};
\end{tikzpicture}
\end{codeexample}
\end{key}

\label{section-rotating-shape-borders}

Some shapes (but not all), support a special kind of rotation. This 
rotation affects only the border of a shape and is independent of the 
node contents, but \emph{in addition} to any other transformations.
	
\begin{codeexample}[]
\tikzstyle{every node}=[dart, shape border uses incircle, 
  inner sep=1pt, draw]
\begin{tikzpicture}
  \foreach \a/\b/\c in {A/0/0, B/45/0, C/0/45, D/45/45}
    \node [shape border rotate=\b, rotate=\c] at (\b/36,-\c/36) {\a};
\end{tikzpicture}
\end{codeexample}

There are two types of rotation: restricted and unrestricted. Which 
type of rotation is applied is determined by on how the shape border 
is constructed. If the shape border is contructed using an incircle, 
that is, a circle that tightly fits the node contents (including 
the |inner sep|), then the rotation can be unrestricted. If, however,
the border is constructed using the natural dimensions of the node
contents, the rotation is restricted to integer multiples of 90 
degrees.

Why should there be two kinds of rotation and border construction?
Borders constructed using the natural dimensions of the node contents
provide a much tighter fit to the node contents, but to maintain 
this tight fit, the border rotation must be restricted to integer 
multiples of 90 degrees. By using an incircle, unrestricted rotation
is possible, but the border will not make a very tight fit to the
node contents. 
	
\begin{codeexample}[]
\tikzstyle{every node}=[isosceles triangle, draw]
\begin{tikzpicture}
  \node {abc};
  \node [shape border uses incircle] at (2,0) {abc};
\end{tikzpicture}
\end{codeexample}

There are \pgfname{} keys determine how a shape border is 
contructed, and to specify its rotation.
It should be noted that not all shapes support these keys, so 
reference should be made to the documentation for individual 
shapes. 
	
\begin{key}{/pgf/shape border uses incircle=\opt{\meta{boolean}}
    (default true)}
  \keyalias{tikz}
  Determines if the border of a shape is constructed using the 
  incircle. If no value is given \meta{boolean} will take the default
  value |true|.
\end{key}


\begin{key}{/pgf/shape border rotate=\meta{angle} (initially 0)}
  \keyalias{tikz}
  Rotates the border of a shape independently of the node contents,
  but in addition to any other transformations. If the shape 
  border is not constructed using the incircle, the rotation will be
  rounded to the nearest integer multiple of 90 degrees when the
  shape is drawn. 
\end{key}

Note that if the border of the shape is rotated, 
the compass point anchors, and `text box' anchors (including 
|mid east|, |base west|, and so on), \emph{do not rotate}, but the 
other anchors do:
	
\begin{codeexample}[]
\tikzstyle{every node}=[shape=trapezium, draw, shape border uses incircle]
\begin{tikzpicture}
  \node at (0,0)  (A) {A};
  \node [shape border rotate=30] at (1.5,0) (B) {B};
  \foreach \s/\t in 
    {left side/base east, bottom side/north, bottom left corner/base}{
       \fill[red]  (A.\s) circle(1.5pt) (B.\s) circle(1.5pt);
       \fill[blue] (A.\t) circle(1.5pt) (B.\t) circle(1.5pt);
  }
\end{tikzpicture}
\end{codeexample}

Finally, a somewhat unfortunate side-effect of rotating shape borders 
is that the supporting shapes do not distinguish between 
|outer xsep| and |outer ysep|, and typically, the larger of the 
two values will be used. 




\subsection{Multi-Part Nodes}

\label{section-nodes-multi}

Most nodes just have a single simple text label. However, nodes of a
more complicated shapes might be made up from several \emph{node
  parts}. For example, in automata theory a so-called Moore state has
a state name, drawn in the upper part of the state circle, and an
output text, drawn in the lower part of the state circle. These two
parts are quite independent. Similarly, a \textsc{uml} class shape
would have a name part, a method part, and an attributes
part. Different molecule shape might use parts for the different atoms
to be drawn at the different positions, and so on.

Both \pgfname\ and \tikzname\ support such multipart nodes. On the
lower level, \pgfname\ provides a system for specifying that a shape
consists of several parts. On the \tikzname\ level, you specify the
different node parts by using the following command:

\begin{command}{\nodepart\marg{part name}}
  This command can only be used inside the \meta{text} argument of a
  |node| path operation. It works a little bit like a |\part| command
  in \LaTeX. It will stop the typesetting of whatever node part was
  typeset until now and then start putting all following text into the
  node part named \meta{part name}---until another |\partname| is
  encountered or until the node \meta{text} ends.

\begin{codeexample}[]
\begin{tikzpicture}
  \node [circle split,draw,double,fill=red!20]
  {
    % No \nodepart has been used, yet. So, the following is put in the
    % ``text'' node part by default.
    $q_1$ 
    \nodepart{lower} % Ok, end ``text'' part, start ``output'' part
    $00$
  }; % output part ended.
\end{tikzpicture}
\end{codeexample}

  You will have to lookup which parts are defined by a shape.

  The following styles influences node parts:
  \begin{stylekey}{/tikz/every \meta{part name} node part (initially
      \normalfont empty)}
    This style is installed at the beginning of every node part named
    \meta{part name}. 
\begin{codeexample}[]
\tikz [every lower node part/.style={red}]
  \node [circle split,draw] {$q_1$ \nodepart{lower} $00$};
\end{codeexample}
  \end{stylekey}
\end{command}



\subsection{Options for the Text in  Nodes}

\label{section-nodes-options}

The simplest option for the text in nodes is its color. Normally, this
color is just the last color installed using |color=|, possibly
inherited from another scope. However, it is possible to specificly
set the color used for text using the following option:

\begin{key}{/tikz/text=\meta{color}}
  Sets the color to be used for text labels. A |color=| option
  will immediately override this option.
\begin{codeexample}[]
\begin{tikzpicture}
  \draw[red]       (0,0) -- +(1,1) node[above]     {red};
  \draw[text=red]  (1,0) -- +(1,1) node[above]     {red};
  \draw            (2,0) -- +(1,1) node[above,red] {red};
\end{tikzpicture}
\end{codeexample}
\end{key}

Just like the color itself, you may also wish to set the opacity of
the text only. For this, use the option |text opacity| option, which
is detailed in Section~\ref{section-tikz-transparency}.

Next, you may wish to adjust the font used for the text. Use the
following option for this:
\begin{key}{/tikz/font=\meta{font commands}}
  Sets the font used for text labels. 
\begin{codeexample}[]
\begin{tikzpicture}
  \draw[font=\itshape] (1,0) -- +(1,1) node[above] {italic};
\end{tikzpicture}
\end{codeexample}
  A perhaps more useful example is the following:

\begin{codeexample}[]
\tikz [every text node part/.style={font=\itshape}, 
       every lower node part/.style={font=\footnotesize}]
  \node [circle split,draw] {state \nodepart{lower} output};
\end{codeexample}
\end{key}


Normally, when a node is typeset, all the text you give in the braces
is but in one long line (in an |\hbox|, to be precise) and the node
will become as wide as necessary.

You can change this behaviour using the following options. They allow
you to limit the width of a node (naturally, at the expense of its
height).

\begin{key}{/tikz/text width=\meta{dimension}}
  This option will put the text of a node in a box of the given width
  (more precisely, in a |{minipage}| of this width; for plain \TeX\ a
  rudimentary ``minipage emulation'' is used).

  If the node text is not as wide as \meta{dimension}, it will
  nevertheless be put in a box of this width. If it is larger, line
  breaking will be done.

  By default, when this option is given, a ragged right border will be
  used. This is sensible since, typically, these boxes are narrow and
  justifying the text looks ugly.
\begin{codeexample}[]
\tikz \draw (0,0) node[fill=examplefill,text width=3cm]
  {This is a demonstration text for showing how line breaking works.};  
\end{codeexample}
\end{key}

\begin{key}{/tikz/text justified}
  Causes the text to be justified instead of (right)ragged. Use this
  only with pretty broad nodes.
{%
\hbadness=10000
\begin{codeexample}[]
\tikz \draw (0,0) node[fill=examplefill,text width=3cm,text justified]
  {This is a demonstration text for showing how line breaking works.};  
\end{codeexample}
}
  In the above example, \TeX\ complains (rightfully) about three very
  badly typeset lines. (For this manual I asked \TeX\ to stop
  complaining by using |\hbadness=10000|, but this is a foul deed,
  indeed.) 
\end{key}

\begin{key}{/tikz/text ragged}
  Causes the text to be typeset with a ragged right. This uses the
  original plain \TeX\ definition of a ragged right border, in which
  \TeX\ will try to balance the right border as well as possible. This
  is the default.
\begin{codeexample}[]
\tikz \draw (0,0) node[fill=examplefill,text width=3cm,text ragged]
  {This is a demonstration text for showing how line breaking works.};  
\end{codeexample}
\end{key}

\begin{key}{/tikz/text badly ragged}
  Causes the right border to be ragged in the \LaTeX-style, in which
  no balancing occurs. This looks ugly, but it may be useful for very
  narrow boxes and when you wish to avoid hyphenations.
\begin{codeexample}[]
\tikz \draw (0,0) node[fill=examplefill,text width=3cm,text badly ragged]
  {This is a demonstration text for showing how line breaking works.};  
\end{codeexample}
\end{key}

\begin{key}{/tikz/text centered}
  Centers the text, but tries to balance the lines.
\begin{codeexample}[]
\tikz \draw (0,0) node[fill=examplefill,text width=3cm,text centered]
  {This is a demonstration text for showing how line breaking works.};  
\end{codeexample}
\end{key}

\begin{key}{/tikz/text badly centered}
  Centers the text, without balancing the lines.
\begin{codeexample}[]
\tikz \draw (0,0) node[fill=examplefill,text width=3cm,text badly centered]
  {This is a demonstration text for showing how line breaking works.};  
\end{codeexample}
\end{key}

In addition to changing the width of nodes, you can also change the
height of nodes. This can be done in two ways: First, you can use the
option |minimum height|, which ensures that the height of the whole
node is at least the given height (this option is described in more
detail later). Second, you can use the option |text height|, which
sets the height of the text itself, more precisely, of the \TeX\ text
box of the text. Note that the |text height| typically is not the
height of the shape's box: In addition to the |text height|, an
internal |inner sep| is added as extra space and the text depth is
also taken into account.

I recommend using |minimum size| instead of |text height| except for
special situations.

\begin{key}{/tikz/text height=\meta{dimension}}
  Sets the height of the text boxes in shapes. Thus, when you write
  something like |node {text}|, the |text| is first typeset, resulting
  in some box of a certain height. This height is then replaced by the
  height |text height|. The resulting box is then used to determine
  the size of the shape, which will typically be larger. When you
  write |text height=| without specifying anything, the ``natural''
  size of the text box remains unchanged.
\begin{codeexample}[]
\tikz \node[draw]                  {y};    
\tikz \node[draw,text height=10pt] {y};       
\end{codeexample}
\end{key}

\begin{key}{/tikz/text depth=\meta{dimension}}
  This option works like |text height|, only for the depth of the text
  box. This option is mostly useful when you need to ensure a uniform
  depth of text boxes that need to be aligned. 
\end{key}




\subsection{Positioning Nodes}

\label{section-nodes-anchors}

When you place a node at some coordinate, the node is centered on this
coordinate by default. This is often undesirable and it would be
better to have the node to the right or above the actual coordinate.


\subsubsection{Positioning Nodes Using Anchors}

\pgfname\ uses a so-called anchoring mechanism to give you a very fine
control over the placement. The idea is simple: Imaging a node of
rectangular shape of a certain size. \pgfname\ defines numerous anchor
positions in the shape. For example to upper right corner is called,
well, not ``upper right anchor,'' but the |north east| anchor of the
shape. The center of the shape has an anchor called |center| on top of
it, and so on. Here are some examples (a complete list is given in
Section~\ref{section-the-shapes}).

\medskip\noindent
\begin{tikzpicture}
  \path node[minimum height=2cm,minimum width=5cm,fill=blue!25](x) {Big node};
  \fill (x.north)      circle (2pt) node[above] {|north|}
        (x.north east) circle (2pt) node[above] {|north east|}
        (x.north west) circle (2pt) node[above] {|north west|}
        (x.west) circle (2pt)       node[left]  {|west|}
        (x.east) circle (2pt)       node[right] {|east|}
        (x.base) circle (2pt)       node[below] {|base|};
\end{tikzpicture}

Now, when you place a node at a certain coordinate, you can ask \tikzname\
to place the node shifted around in such a way that a certain
anchor is at the coordinate. In the following example, we ask \tikzname\
to shift the first node such that its  |north east| anchor is at
coordinate |(0,0)| and that the |west| anchor of the second node is at
coordinate |(1,1)|.

\begin{codeexample}[]
\tikz \draw           (0,0) node[anchor=north east] {first node}
            rectangle (1,1) node[anchor=west] {second node};
\end{codeexample}

Since the default anchor is |center|, the default behaviour is to
shift the node in such a way that it is centered on the current
position.

\begin{key}{/tikz/anchor=\meta{anchor name}}
  Causes the node to be shifted such that it's anchor \meta{anchor
  name} lies on the current coordinate.

  The only anchor that is present in all shapes is |center|. However,
  most shapes will at least define anchors in all ``compass
  directions.'' Furthermore, the standard shapes also define a |base|
  anchor, as well as |base west| and |base east|, for placing things on
  the baseline of the text.
  
  The standard shapes also define a |mid| anchor (and |mid west| and
  |mid east|). This anchor is half the height of the character ``x''
  above the base line. This anchor is useful for vertically centering
  multiple nodes that have different heights and depth. Here is an
  example:
\begin{codeexample}[]
\begin{tikzpicture}[scale=3,transform shape]
  % First, center alignment -> wobbles
  \draw[anchor=center] (0,1)  node{x} -- (0.5,1)  node{y} -- (1,1)  node{t};
  % Second, base alignment -> no wobble, but too high
  \draw[anchor=base]   (0,.5) node{x} -- (0.5,.5) node{y} -- (1,.5) node{t};
  % Third, mid alignment
  \draw[anchor=mid]    (0,0)  node{x} -- (0.5,0)  node{y} -- (1,0)  node{t};
\end{tikzpicture}
\end{codeexample}
\end{key}



\subsubsection{Basic Placement Options}

Unfortunately, while perfectly logical, it is often rather
counter-intuitive that in order to place a node \emph{above} a given
point, you need to specify the |south| anchor. For this reason, there
are some useful options that allow you to select the standard anchors
more intuitively:

\begin{key}{/tikz/above=\meta{offset} (default 0pt)}
  Does the same as |anchor=south|. If the \meta{offset} is specified,
  the node is additionally shifted upwards by the given
  \meta{offset}. 
\begin{codeexample}[]
\tikz \fill (0,0) circle (2pt) node[above] {above};
\end{codeexample}
\begin{codeexample}[]
\tikz \fill (0,0) circle (2pt) node[above=2pt] {above};
\end{codeexample}
\end{key}

\begin{key}{/tikz/below=\meta{offset} (default 0pt)}
  Similar to |above|.
\end{key}

\begin{key}{/tikz/left=\meta{offset} (default 0pt)}
  Similar to |above|.
\end{key}

\begin{key}{/tikz/right=\meta{offset} (default 0pt)}
  Similar to |above|.
\end{key}

\begin{key}{/tikz/above left}
  Does the same as |anchor=south east|. Note that giving both |above|
  and |left| options does not have the same effect as |above left|,
  rather only the last |left| ``wins.'' Actually, this option also
  takes an \meta{offset} parameter, but using this parameter without
  using the |positioning| library is deprecated. (The |positioning|
  library changes the meaning of this parameter to something more
  sensible.) 
\begin{codeexample}[]
\tikz \fill (0,0) circle (2pt) node[above left] {above left};
\end{codeexample}
\end{key}

\begin{key}{/tikz/above right}
  Similar to  |above left|.
\begin{codeexample}[]
\tikz \fill (0,0) circle (2pt) node[above right] {above right};
\end{codeexample}
\end{key}

\begin{key}{/tikz/below left}
  Similar to |above left|.
\end{key}
\begin{key}{/tikz/below right}
  Similar to |above left|.
\end{key}

% A second set of options behaves similarly, namely the |above of|,
% |below of|, and so on options. They cause the same anchors to be set
% as the options without |of|, however, their parameter is different:
% You must provide the name of another node. The current node will then
% be placed, say, above this specified node at a distance given by the
% option |node distance|. 
% \begin{key}{/tikz/above of=\meta{node}}
%   This option causes the node to be placed at the distance
%   |node distance| above of \meta{node}. The anchor is |center|.
% \begin{codeexample}[]
% \begin{tikzpicture}[node distance=1cm]
%   \draw[help lines] (0,0) grid (3,2);
%   \node (a)                    {a};
%   \node (b) [above of=a]       {b};
%   \node (c) [above of=b]       {c};
%   \node (d) [right of=c]       {d};
%   \node (e) [below right of=d] {e};
% \end{tikzpicture}
% \end{codeexample}
% \end{key}

% \begin{key}{/tikz/above left of=\meta{node}}
%   Works like |above of|, only the node is now put above and left. The
%   |node distance| is the Euclidean distance between the two nodes, not
%   the $L_1$-distance.
% \end{key}

% \begin{key}{/tikz/above right of=\meta{node}}
%   Works similarly.
% \end{key}
% \begin{key}{/tikz/left of=\meta{node}}
%   Works similarly.
% \end{key}
% \begin{key}{/tikz/right of=\meta{node}}
%   Works similarly.
% \end{key}
% \begin{key}{/tikz/below of=\meta{node}}
%   Works similarly.
% \end{key}
% \begin{key}{/tikz/below left of=\meta{node}}
%   Works similarly.
% \end{key}
% \begin{key}{/tikz/below right of=\meta{node}}
%   Works similarly.
% \end{key}
% \begin{key}{/tikz/node distance=\meta{dimension}}
%   Sets the distance between nodes that are placed using the
%   |... of| options. Note that this distance is the distance between
%   the centers of the nodes, not the distance between their borders. 
% \end{key}



\subsubsection{Advanced Placement Options}

While the standard placement options suffice for simple cases, the
|positioning| library offers more convenient placement options.

\begin{tikzlibrary}{positioning}
  The library defines additional options for placing nodes
  conveniently. It also redefines the standard options like |above| so
  that they give you better control of node placement.
\end{tikzlibrary}

When this library is loaded, the options like |above| or |above left| behave
differently.

\begin{key}{/tikz/above=\opt{\meta{specification}} (default 0pt)}
  With the |positioning| library loaded, the |above| option does not
  take a simple \meta{dimension} as its paramter. Rather, it can
  (also) take a more elaborate \meta{specification} as parameter. This
  \meta{specification} has the following general form: It starts with
  an optional \meta{shifting part} and is followed by an optional
  \meta{of-part}. Let us start with the \meta{shifting part}, which
  can have three forms:
  \begin{enumerate}
  \item It can simply be a \declare{\meta{dimension}} (or a mathematical
    expression that evaluates to a dimension) like |2cm| or
    |3cm/2+4cm|. In this case, the following happens: the node's
    anchor is set to |south| and the node is vertically shifted
    upwards by \meta{dimension}.
\begin{codeexample}[]
\begin{tikzpicture}
  \draw[help lines] (0,0) grid (2,2);
  \node at (1,1) [above=2pt+3pt,draw] {above};
\end{tikzpicture}
\end{codeexample}
    This use of the |above| option is the same as if the |positioning|
    library were not loaded.
  \item It can be a \declare{\meta{number}} (that is, any mathematical
    expression that does not include a unit like |pt| or
    |cm|). Examples are |2| or |3+sin(60)|. In this case, the anchor
    is also set to |south| and the node is vertically shifted by
    the vertical component of the coordinate |(0,|\meta{number}|)|.
\begin{codeexample}[]
\begin{tikzpicture}
  \draw[help lines] (0,0) grid (2,2);
  \node at (1,1) [above=.2,draw] {above};
  % south border of the node is now 2mm above (1,1)
\end{tikzpicture}
\end{codeexample}
  \item It can be of the form \declare{\meta{number or
        dimension 1}| and |\meta{number or dimension 2}}. This specification
    does not make particular sense for the |above| option, it is much
    more useful for options like |above left|. The reason it is
    allowed for the |above| option is that it is sometimes
    automatically used, as explained later.

    The effect of this option is the following. First, the point
    |(|\meta{number of dimension 2}|,|\meta{number or dimension 1}|)|
    is computed (note the inversed order), using the normal rules for
    evaluating such a coordinate, yielding some position. Then, the
    node is shifted by the vertical component of this point. The
    anchor is set to |south|.
\begin{codeexample}[]
\begin{tikzpicture}
  \draw[help lines] (0,0) grid (2,2);
  \node at (1,1) [above=.2 and 3mm,draw] {above};
  % south border of the node is also 2mm above (1,1)
\end{tikzpicture}
\end{codeexample}
  \end{enumerate}
  The \meta{shifting part} can optionally be followed by a
  \meta{of-part}, which has one of the following forms:
  \begin{enumerate}
  \item The \meta{of-part} can be declare{|of |\meta{coordinate}},
    where\meta{coordinate} is \emph{not} in parentheses and it is
    \emph{not} just a node name. An example would be
    |of somenode.north| or |of 2,3|. In this case, the
    following happens: First, the node's |at| parameter is set to the
    \meta{coordinate}. Second, the node is shifted according to the
    \meta{shift-part}. Third, the anchor is set to |south|. 

    Here is a basic example:
\begin{codeexample}[]
\begin{tikzpicture}[every node/.style=draw]
  \draw[help lines] (0,0) grid (2,2);
  \node (somenode) at (1,1) {some node};
  
  \node [above=5mm of somenode.north east] {\tiny 5mm of somenode.north east};
  \node [above=1cm of somenode.north]      {\tiny 1cm of somenode.north};
\end{tikzpicture}
\end{codeexample}
    As can be seen the |above=5mm of somenode.north east| option does,
    indeed, place the node 5mm above the north east anchor of
    |somenode|. The same effect could have been achieved writing
    |above=5mm| followed by |at=(somenode.north east)|.

    If the \meta{shift-part} is missing, the shift is not zero, but
    rather the value of the |node distance| key is used, see below.
  \item The \meta{of-part} can have be |of |\meta{node name}. An
    example would be |of somenode|. In this case, the following
    usually happens: 
    \begin{itemize}
    \item The anchor is set to |south|.
    \item The node is shifted according to the \meta{shifting part}
      or, if it is missing, according to the value of |node distance|.
    \item The node's |at| parameter is set to \meta{node
        name}|.north|. 
    \end{itemize}
    The net effect of all this is that the new node will be placed in
    such a way that the distance between is south border and
    \meta{node name}'s north border is exactly the given distance.
\begin{codeexample}[]
\begin{tikzpicture}[every node/.style=draw]
  \draw[help lines] (0,0) grid (2,2);
  \node (some node) at (1,1) {some node};
  
  \node (other node) [above=1cm of some node] {\tiny above=1cm of some node};

  \draw [<->] (some node.north) -- (other node.south)
                                node [midway,right,draw=none] {1cm};
\end{tikzpicture}
\end{codeexample}
    It is possible to change the behaviour of this
    \meta{specification} rather drastically, using the following key:
    \begin{key}{/tikz/on grid=\meta{boolean} (initially false)}
      When this key is set to |true|, an \meta{of-part} of the
      current form behaves differently: The anchors set for the
      current node as well as the anchor used for other \meta{node
        name} are set the |center|.

      This has the following effect: When you say
      |above=1cm of somenode| with |on grid| set to true, the new node
      will be placed in such a way that its center is 1cm above the
      center of |somenode|. Repeatedly placing nodes in this way will
      result in nodes that are centered on ``grid coordinate,'' hence
      the name of the option. 
\begin{codeexample}[]
\begin{tikzpicture}[every node/.style=draw]
  \draw[help lines] (0,0) grid (2,3);

  % Not gridded
  \node (a1) at (0,0) {not gridded};
  \node (b1) [above=1cm of a1] {fooy};
  \node (c1) [above=1cm of b1] {a};

  % gridded
  \node (a2) at (2,0) {gridded};
  \node (b2) [on grid,above=1cm of a2] {fooy};
  \node (c2) [on grid,above=1cm of b2] {a};
\end{tikzpicture}
\end{codeexample}
    \end{key}
  \end{enumerate}

  \begin{key}{/tikz/node distance=\meta{shifting part} (initially 1cm and 1cm)}
    The value of this key is used as \meta{shifting part} is used if
    and only if a \meta{of-part} is present, but no \meta{shifting
      part}. 
\begin{codeexample}[]
\begin{tikzpicture}[every node/.style=draw,node distance=5mm]
  \draw[help lines] (0,0) grid (2,3);

  % Not gridded
  \node (a1) at (0,0) {not gridded};
  \node (b1) [above=of a1] {fooy};
  \node (c1) [above=of b1] {a};

  % gridded
  \begin{scope}[on grid]
    \node (a2) at (2,0) {gridded};
    \node (b2) [above=of a2] {fooy};
    \node (c2) [above=of b2] {a};
  \end{scope}
\end{tikzpicture}
\end{codeexample}
  \end{key}
\end{key}

\begin{key}{/tikz/below=\opt{\meta{specification}}}
  This key is redefined in the same manner as |above|.
\end{key}

\begin{key}{/tikz/left=\opt{\meta{specification}}}
  This key is redefined in the same manner as |above|, only all
  vertical shifts are replaced by horizontal shifts.
\end{key}

\begin{key}{/tikz/right=\opt{\meta{specification}}}
  This key is redefined in the same manner as |left|.
\end{key}

\begin{key}{/tikz/above left=\opt{\meta{specification}}}
  This key is also redefined in a manner similar to the above, but
  behaviour of the \meta{shifting part} is more complicated:
  \begin{enumerate}
  \item When the \meta{shifting part} is of the form \meta{number or
      dimension}| and |\meta{number or dimension}, it has
    (essentially) the effect of shifting the node vertically upwards
    by the first \meta{number or dimension} and to the left by the
    second. To be more precise, the coordinate |(|\meta{second number
      or dimension}|,|\meta{first number or dimension}|)| is computed
    and then the node is shifted vertically by the $y$-part of
    the resulting coordinate and horizontally be the negated $x$-part
    of the result. (This is exactly what you expect, except possibly
    when you have used the |x| and |y| options to modify the
    |xy|-coordinate system so that the unit  vectors no longer point
    in the expected directions.)
  \item When the \meta{shifting part} is of the form \meta{number or
      dimension}, the node is shifted by this \meta{number or
      dimension} in the direction of $135^\circ$. This means that
    there is a difference between a \meta{shifting part} of |1cm| and
    of |1cm and 1cm|: In the second case, the node is shifted by 1cm
    upward and 1cm to the left; in the first case it is shifted by
    $\frac{1}{2}\sqrt{2}$cm upward and by the same amount to the
    left. A more mathematical way of phrasing this is the following: A
    plain \meta{dimension} is measured in the $l_2$-norm, while a
    \meta{dimension}| and |\meta{dimension} is measured in the
    $l_1$-norm. 
  \end{enumerate}
  The following example should help to illustrate the difference:
\begin{codeexample}[]
\begin{tikzpicture}[every node/.style={draw,circle}]
  \draw[help lines] (0,0) grid (2,5);
  \begin{scope}[node distance=5mm]
    \node (a) at (1,1) {a};
    \node [left=of a] {1};       \node [right=of a] {2};
    \node [above=of a] {3};      \node [below=of a] {4};
    \node [above left=of a] {5}; \node [above right=of a] {6};
    \node [below left=of a] {7}; \node [below right=of a] {8};
  \end{scope}
  \begin{scope}[node distance=5mm and 5mm]
    \node (b) at (1,4) {b};
    \node [left=of b] {1};       \node [right=of b] {2};
    \node [above=of b] {3};      \node [below=of b] {4};
    \node [above left=of b] {5}; \node [above right=of b] {6};
    \node [below left=of b] {7}; \node [below right=of b] {8};
  \end{scope}
\end{tikzpicture}
\end{codeexample}  
\begin{codeexample}[]
\begin{tikzpicture}[every node/.style={draw,rectangle}]
  \draw[help lines] (0,0) grid (2,5);
  \begin{scope}[node distance=5mm]
    \node (a) at (1,1) {a};
    \node [left=of a] {1};       \node [right=of a] {2};
    \node [above=of a] {3};      \node [below=of a] {4};
    \node [above left=of a] {5}; \node [above right=of a] {6};
    \node [below left=of a] {7}; \node [below right=of a] {8};
  \end{scope}
  \begin{scope}[node distance=5mm and 5mm]
    \node (b) at (1,4) {b};
    \node [left=of b] {1};       \node [right=of b] {2};
    \node [above=of b] {3};      \node [below=of b] {4};
    \node [above left=of b] {5}; \node [above right=of b] {6};
    \node [below left=of b] {7}; \node [below right=of b] {8};
  \end{scope}
\end{tikzpicture}
\end{codeexample}  
\begin{codeexample}[]
\begin{tikzpicture}[every node/.style={draw,rectangle},on grid]
  \draw[help lines] (0,0) grid (4,4);
  \begin{scope}[node distance=1]
    \node (a) at (2,3) {a};
    \node [left=of a] {1};       \node [right=of a] {2};
    \node [above=of a] {3};      \node [below=of a] {4};
    \node [above left=of a] {5}; \node [above right=of a] {6};
    \node [below left=of a] {7}; \node [below right=of a] {8};
  \end{scope}
  \begin{scope}[node distance=1 and 1]
    \node (b) at (2,0) {b};
    \node [left=of b] {1};       \node [right=of b] {2};
    \node [above=of b] {3};      \node [below=of b] {4};
    \node [above left=of b] {5}; \node [above right=of b] {6};
    \node [below left=of b] {7}; \node [below right=of b] {8};
  \end{scope}
\end{tikzpicture}
\end{codeexample}  
\end{key}

\begin{key}{/tikz/below left=\opt{\meta{specification}}}
  Works similar to |above left|.
\end{key}
\begin{key}{/tikz/above left=\opt{\meta{specification}}}
  Works similar to |above left|.
\end{key}
\begin{key}{/tikz/above right=\opt{\meta{specification}}}
  Works similar to |above left|.
\end{key}

The |positioning| package also introduces the following new placement
keys:
\begin{key}{/tikz/base left=\opt{\meta{specification}}}
  This key works like the |left| key, only instead of the |east| anchor,
  the |base east| anchor is used and, when the second form of an
  \meta{of-part} is used, the corresponding |base west| anchor.

  This key is useful for chaining together nodes so that their base
  lines are aligned.
\begin{codeexample}[]
\begin{tikzpicture}[node distance=1ex]
  \draw[help lines] (0,0) grid (3,1);
  \huge
  \node (X) at (0,1)     {X};
  \node (a) [right=of X] {a};
  \node (y) [right=of a] {y};
  
  \node (X) at (0,0)          {X};
  \node (a) [base right=of X] {a};
  \node (y) [base right=of a] {y};
\end{tikzpicture}
\end{codeexample}  
\end{key}
\begin{key}{/tikz/base right=\opt{\meta{specification}}}
  Works like |base left|.
\end{key}
\begin{key}{/tikz/mid left=\opt{\meta{specification}}}
  Works like |base left|, but with |mid east| and |mid west| anchors
  instead of |base east| and |base west|.
\end{key}
\begin{key}{/tikz/mid right=\opt{\meta{specification}}}
  Works like |mid left|.
\end{key}




\subsubsection{Arranging Nodes Using a Chains and Matrices}

The simple |above| and |right| options may not always suffice for
arranging a large number of nodes. For such situations \tikzname\
offers two libraries that make positioning easier: The |chains|
library and the |matrix| library. The first is mostly useful for
creating ``chains of nodes'' and, more generally, ``flows.'' The
second allows you to arrange multiple nodes in rows and columns. These
methods for positioning nodes are described in two separate
Sections~\ref{section-matrices} and~\ref{section-chains}. 


\subsection{Fitting Nodes to a Set of Coordinates}

\label{section-nodes-fitting}

It is sometimes desirable that the size and position of a node is not
given using anchors and size parameters, rather one would sometimes
have a box be placed and be sized such that it ``is just large enough
to contain this, that, and that point.'' This situation typically
arises when a picture has been drawn an, afterwards, parts of the
picture are supposed to be encircled or hilighted.

In this situation the |fit| option from the |fit| library is useful,
see Section~\ref{section-library-fit} for a the details. The idea is
that you may give the |fit| option to a node. The |fit| option expects
a list of coordinates (one after the other without commas) as its
parameter. The effect will be that the node's text area has exactly
the necessary size so that it contains all the given coordinates. Here
is an example:

\begin{codeexample}[]
\begin{tikzpicture}[level distance=8mm]
  \node (root) {root}
    child { node (a) {a} }
    child { node (b) {b}
      child { node (d) {d} }
      child { node (e) {e} } }
    child { node (c) {c} };

  \node[draw=red,inner sep=0pt,thick,ellipse,fit=(root) (b) (d) (e)] {};
  \node[draw=blue,inner sep=0pt,thick,ellipse,fit=(b) (c) (e)] {};
\end{tikzpicture}
\end{codeexample}

If you want to fill the fitted node you will usually have to place it
on a background layer.

\begin{codeexample}[]
\begin{tikzpicture}[level distance=8mm]
  \node (root) {root}
    child { node (a) {a} }
    child { node (b) {b}
      child { node (d) {d} }
      child { node (e) {e} } }
    child { node (c) {c} };

  \begin{pgfonlayer}{background}
    \node[fill=red!20,inner sep=0pt,ellipse,fit=(root) (b) (d) (e)] {};
    \node[fill=blue!20,inner sep=0pt,ellipse,fit=(b) (c) (e)] {};
  \end{pgfonlayer}
\end{tikzpicture}
\end{codeexample}

\subsection{Transformations}

\label{section-nodes-transformations}


It is possible to transform nodes, but, by default, transformations do
not apply to nodes. The reason is that you usually do \emph{not} want
your text to be scaled or rotated even if the main graphic is
transformed. Scaling text is evil, rotating slightly less so.

However, sometimes you \emph{do} wish to transform a node, for
example, it certainly sometimes makes sense to rotate a node by
90 degrees. There are two ways in which you can achieve this:

\begin{enumerate}
\item
  You can use the following option:
  \begin{key}{/tikz/transform shape}
    Causes the current ``external'' transformation matrix to be
    applied to the shape. For example, if you said
    |\tikz[scale=3]| and then say |node[transform shape] {X}|, you
    will get a ``huge'' X in your graphic.
  \end{key}
\item
  You can give transformation option \emph{inside} the option list of
  the node. \emph{These} transformations always apply to the node.
\begin{codeexample}[]
\begin{tikzpicture}[every node/.style={draw}]    
  \draw[help lines](0,0) grid (3,2);
  \draw            (1,0) node{A}
                   (2,0) node[rotate=90,scale=1.5] {B};
  \draw[rotate=30] (1,0) node{A}
                   (2,0) node[rotate=90,scale=1.5] {B};
  \draw[rotate=60] (1,0) node[transform shape] {A}
                   (2,0) node[transform shape,rotate=90,scale=1.5] {B};
\end{tikzpicture}
\end{codeexample}
\end{enumerate}




\subsection{Placing Nodes on a Line or Curve Explicitly}

\label{section-nodes-placing-1}

Until now, we always placed node on a coordinate that is mentioned in
the path. Often, however, we wish to place nodes on ``the middle'' of
a line and we do not wish to compute these coordinates ``by hand.''
To facilitate such placements, \tikzname\ allows you to specify that a
certain node should be somewhere ``on'' a line. There are two ways of
specifying this: Either explicitly by using the |pos| option or
implicitly by placing the node ``inside'' a path operation. These two
ways are described in the following.

\label{section-pos-option}

\begin{key}{/tikz/pos=\meta{fraction}}
  When this option is given, the node is not anchored on the last
  coordinate. Rather, it is anchored on some point on the line from
  the previous coordinate to the current point. The \meta{fraction}
  dictates how ``far'' on the line the point should be. A
  \meta{fraction} or 0 is the previous coordinate, 1 is the current
  one, everything else is in between. In particular, 0.5 is the
  middle.  

  Now, what is ``the previous line''? This depends on the previous
  path construction operation.

  In the simplest case, the previous path operation was a ``line-to''
  operation, that is, a  |--|\meta{coordinate} operation:
\begin{codeexample}[]
\tikz \draw (0,0) -- (3,1)
    node[pos=0]{0} node[pos=0.5]{1/2} node[pos=0.9]{9/10};
\end{codeexample}

  The next case is the curve-to operation (the |..| operation). In this
  case, the ``middle'' of the curve, that is, the position |0.5| is
  not necessarily the point at the exact half distance on the
  line. Rather, it is some point at ``time'' 0.5 of a point traveling
  from the start of the curve, where it is at time 0, to the end of
  the curve, which it reaches at time 0.5. The ``speed'' of the point
  depends on the length of the support vectors (the vectors that
  connect the start and end points to the control points). The exact
  math is a bit complicated (depending on your point of view, of
  course); you may wish to consult a good book on computer graphics
  and B�zier curves if you are intrigued. 
\begin{codeexample}[]
  \tikz \draw (0,0) .. controls +(right:3.5cm) and +(right:3.5cm) .. (0,3)
    \foreach \p in {0,0.125,...,1} {node[pos=\p]{\p}};
\end{codeexample}

  Another interesting case are the horizontal/vertical line-to operations
  \verb!|-! and \verb!-|!. For them, the position (or time) |0.5| is
  exactly the corner point.

\begin{codeexample}[]
\tikz \draw (0,0) |- (3,1)
  node[pos=0]{0} node[pos=0.5]{1/2} node[pos=0.9]{9/10};
\end{codeexample}

\begin{codeexample}[]
\tikz \draw (0,0) -| (3,1)
  node[pos=0]{0} node[pos=0.5]{1/2} node[pos=0.9]{9/10};
\end{codeexample}

  For all other path construction operations, \emph{the position
  placement does not work}, currently. This will hopefully change in
  the future (especially for the arc operation).  
\end{key}

\begin{key}{/tikz/auto=\meta{left or right} (default \normalfont is scope's setting)}
  This option causes an anchor positions to be calculated
  automatically according to the following rule. Consider a line
  between to points. If the \meta{direction} is |left|, then the
  anchor is chosen such that the node is to the left of this line. If
  the \meta{direction} is |right|, then the node is to the right of
  this line. Leaving out \meta{direction} causes automatic placement
  to be enabled with the last value of |left| or |right| used. A
  \meta{direction} of |false| disables automatic placement. This
  happens also  whenever an anchor is given explicitly by the
  |anchor| option or by one of the |above|, |below|, etc.\ options.

  This option only has an effect for nodes that are placed on lines or
  curves. 

\begin{codeexample}[]
\begin{tikzpicture}
  [scale=.8,auto=left,every node/.style={circle,fill=blue!20}]
  \node (a) at (-1,-2) {a};
  \node (b) at ( 1,-2) {b};
  \node (c) at ( 2,-1) {c};
  \node (d) at ( 2, 1) {d};
  \node (e) at ( 1, 2) {e};
  \node (f) at (-1, 2) {f};
  \node (g) at (-2, 1) {g};
  \node (h) at (-2,-1) {h};

  \foreach \from/\to in {a/b,b/c,c/d,d/e,e/f,f/g,g/h,h/a}
    \draw [->] (\from) -- (\to)
               node[midway,fill=red!20] {\from--\to};
\end{tikzpicture}
\end{codeexample}
\end{key}

\begin{key}{/tikz/swap}
  This option exchanges the roles of |left| and |right| in automatic
  placement. That is, if |left| is the current |auto| placement,
  |right| is set instead and the other way round.
\begin{codeexample}[]
\begin{tikzpicture}[auto]
  \draw[help lines,use as bounding box] (0,-.5) grid (4,5);  

  \draw (0.5,0) .. controls (9,6) and (-5,6) .. (3.5,0)
    \foreach \pos in {0,0.1,0.2,0.3,0.4,0.5,0.6,0.7,0.8,0.9,1}
      {node [pos=\pos,swap,fill=red!20] {\pos}}
    \foreach \pos in {0.025,0.2,0.4,0.6,0.8,0.975}
      {node [pos=\pos,fill=blue!20] {\pos}};
\end{tikzpicture}
\end{codeexample}
\begin{codeexample}[]
\begin{tikzpicture}[shorten >=1pt,node distance=2cm,auto]
  \draw[help lines] (0,0) grid (3,2);

  \node[state] (q_0)                      {$q_0$};
  \node[state] (q_1) [above right of=q_0] {$q_1$};
  \node[state] (q_2) [below right of=q_0] {$q_2$};
  \node[state] (q_3) [below right of=q_1] {$q_3$};

  \path[->] (q_0) edge              node        {0} (q_1)
                  edge              node [swap] {1} (q_2)
            (q_1) edge              node        {1} (q_3)
                  edge [loop above] node        {0} ()
            (q_2) edge              node [swap] {0} (q_3)
                  edge [loop below] node        {1} ();
\end{tikzpicture}
\end{codeexample}
\end{key}

\begin{key}{/tikz/sloped}
  This option causes the node to be rotated such that a horizontal
  line becomes a tangent to the curve. The rotation is normally 
  done in such a way that text is never ``upside down.'' To get
  upside-down text, use can use |[rotate=180]| or
  |[allow upside down]|, see below.
\begin{codeexample}[]
\tikz \draw (0,0) .. controls +(up:2cm) and +(left:2cm) .. (1,3)
    \foreach \p in {0,0.25,...,1} {node[sloped,above,pos=\p]{\p}};
\end{codeexample}
\begin{codeexample}[]
\begin{tikzpicture}[->]
  \draw (0,0)   -- (2,0.5) node[midway,sloped,above] {$x$};
  \draw (2,-.5) -- (0,0)   node[midway,sloped,below] {$y$};
\end{tikzpicture}
\end{codeexample}
\end{key}


\begin{key}{/tikz/allow upside down=\meta{boolean} (default true, initially false)}
  If set to |true|, \tikzname\ will not ``righten'' upside down text. 
\begin{codeexample}[]
\tikz [allow upside down]
  \draw (0,0) .. controls +(up:2cm) and +(left:2cm) .. (1,3)
    \foreach \p in {0,0.25,...,1} {node[sloped,above,pos=\p]{\p}};
\end{codeexample}
\begin{codeexample}[]
\begin{tikzpicture}[->,allow upside down]
  \draw (0,0)   -- (2,0.5) node[midway,sloped,above] {$x$};
  \draw (2,-.5) -- (0,0)   node[midway,sloped,below] {$y$};
\end{tikzpicture}
\end{codeexample}
\end{key}


There exist styles for specifying positions a bit less ``technically'':
\begin{stylekey}{/tikz/midway}
  This has the same effect as |pos=0.5|.
\begin{codeexample}[]
\tikz \draw (0,0) .. controls +(up:2cm) and +(left:3cm) .. (1,5)
       node[at end]          {|at end|}
       node[very near end]   {|very near end|}
       node[near end]        {|near end|}
       node[midway]          {|midway|}
       node[near start]      {|near start|}
       node[very near start] {|very near start|}
       node[at start]        {|at start|};
\end{codeexample}
\end{stylekey}

\begin{stylekey}{/tikz/near start}
  Set to |pos=0.25|.
\end{stylekey}

\begin{stylekey}{/tikz/near end}
  Set to |pos=0.75|.
\end{stylekey}

\begin{stylekey}{/tikz/very near start}
  Set to |pos=0.125|.
\end{stylekey}

\begin{stylekey}{/tikz/very near end}
  Set to |pos=0.875|.
\end{stylekey}

\begin{stylekey}{/tikz/at start}
  Set to |pos=0|.
\end{stylekey}

\begin{stylekey}{/tikz/at end}
  Set to |pos=1|.
\end{stylekey}


\subsection{Placing Nodes on a Line or Curve Implicitly}

\label{section-nodes-placing-2}

When you wish to place a node on the line |(0,0) -- (1,1)|,
it is natural to specify the node not following the |(1,1)|, but
``somewhere in the middle.'' This is, indeed, possible and you can
write |(0,0) -- node{a} (1,1)| to place a node midway between |(0,0)| and
|(1,1)|.

What happens is the following: The syntax of the line-to path
operation is actually |--|
\opt{|node|\meta{node specification}}\meta{coordinate}. (It is even
possible to give multiple nodes in this way.) When the optional
|node| is encountered, that is, 
when the |--| is directly followed by |node|, then the
specification(s) are read and ``stored away.'' Then, after the
\meta{coordinate} has finally been reached, they are inserted again,
but with the |pos| option set.

There are two things to note about this: When a node specification is
``stored,'' its catcodes become fixed. This means that you cannot use
overly complicated verbatim text in them. If you really need, say, a
verbatim text, you will have to put it in a normal node following the
coordinate and add the |pos| option.

Second, which |pos| is chosen for the node? The position is inherited
from the surrounding scope. However, this holds only for nodes
specified in this implicit way. Thus, if you add the option
|[near end]| to a scope, this does not mean that \emph{all} nodes given
in this scope will be put on near the end of lines. Only the nodes
for which an implicit |pos| is added will be placed near the
end. Typically, this is what you want. Here are some examples that
should make this clearer:

\begin{codeexample}[]
\begin{tikzpicture}[near end]
  \draw (0cm,4em) -- (3cm,4em) node{A};    
  \draw (0cm,3em) --           node{B}          (3cm,3em);
  \draw (0cm,2em) --           node[midway] {C} (3cm,2em);
  \draw (0cm,1em) -- (3cm,1em) node[midway] {D} ;
\end{tikzpicture}
\end{codeexample}

Like the line-to operation, the curve-to operation |..| also allows you to
specify nodes ``inside'' the operation. After both the first |..| and
also after the second |..| you can place node specifications. Like for
the |--| operation, these will be collected and then reinserted after
the operation with the |pos| option set.


\subsection{The Label and Pin Options}

In addition to the |node| path operation, nodes can also be added
using the |label| and the |pin| option. This is mostly useful
for simple nodes. 

\begin{key}{/tikz/label=\opt{|[|\meta{options}|]|}\meta{angle}|:|\meta{text}}
  When this option is given to a |node| operation, it causes
  \emph{another} node to be added to the path after the current node
  has been finished. This extra node will have the text
  \meta{text}. It is placed according to the following rule: Suppose
  the |node| currently under construction is called |main node| and let us
  call the label node |label node|. Then the anchor of |label node| is
  placed at |main node.|\meta{angle}. The anchor that is chosen
  depends on the \meta{angle}. If the \meta{angle} lies between
  $-3^\circ$ and $+3^\circ$, then the anchor |west| is chosen, which
  causes |label node| to be placed right of the right end
  |main node|. If \meta{angle} lies between $4^\circ$ and $86^\circ$,
  the anchor |south west| is chosen, causing the |label node| to be
  placed above and right of the |main node|; and so on.
\begin{codeexample}[]
\tikz
  \node [circle,draw,label=60:$60^\circ$,label=below:$-90^\circ$] {my circle};
\end{codeexample}

  As can be seen in the above example, instead of specifying
  \meta{angle} as a number, it is also possible to use |left|,
  |right|, |above|, |above left|, and so on.

  You can pass \meta{options} to the node |label node|. For this, you
  provide the options in square brackets before the \meta{angle}. If you
  do so, you need to add braces around the whole argument of the
  |label| option and this is also the case if you have brackets or
  commas or semicolons or anything special in the \meta{text}.
\begin{codeexample}[]
\tikz \node [circle,draw,label={[red]above:X}] {my circle};
\end{codeexample}

\begin{codeexample}[]
\begin{tikzpicture}
  \node [circle,draw,label={[name=label node]above left:$a,b$}] {};
  \draw (label node) -- +(1,1);
\end{tikzpicture}
\end{codeexample}

  If you provide multiple |label| options, then multiple extra label
  nodes are added in the order they are given.

  The following styles influence how labels are drawn:
  \begin{key}{/tikz/label distance=\meta{distance} (initially 0pt)}
    The \meta{distance} is additionally inserted between the main node
    and the label node.
\begin{codeexample}[]
\tikz[label distance=5mm]
  \node [circle,draw,label=right:X,
                     label=above right:Y,
                     label=above:Z]       {my circle};
\end{codeexample}
  \end{key}
  \begin{stylekey}{/tikz/every label (initially \normalfont empty)}
    This style is used in every node created by the |label|
    option. The default is |draw=none,fill=none|.
  \end{stylekey}
\end{key}

\begin{key}{/tikz/pin=\opt{|[|\meta{options}|]|}\meta{angle}|:|\meta{text}}
  This is option is quite similar to the |label| option, but there is
  one difference: In addition to adding a extra node to the picture, 
  it also adds an edge from this node to the main node. This causes
  the node to look like a pin that has been added to the main node: 
\begin{codeexample}[]
\tikz \node [circle,fill=blue!50,minimum size=1cm,pin=60:$q_0$] {};
\end{codeexample}

  The meaning of the \meta{options} and the \meta{angle} and the
  \meta{text} is exactly the same as for the |node| option. Only, the
  options and styles the influence the way pins look are different:
  \begin{key}{/tikz/pin distance=\meta{distance} (initially 3ex)}
    This \meta{distance} is used instead of the |label distance| for
    the distance between the main node and the label node.
\begin{codeexample}[]
\tikz[pin distance=1cm]
  \node [circle,draw,pin=right:X,
                     pin=above right:Y,
                     pin=above:Z]       {my circle};
\end{codeexample}
  \end{key}
  \begin{stylekey}{/tikz/every pin (initially \normalfont draw=none,fill=none)}
    This style is used in every node created by the |pin|
    option.
  \end{stylekey}
  \begin{stylekey}{/tikz/every pin edge (initially help lines)}
    This style is used in every edge created by the |pin| optins.
\begin{codeexample}[]
\tikz [pin distance=15mm,
       every pin edge/.style={<-,shorten <=1pt,decorate,
                              decoration={snake,pre length=4pt}}]
  \node [circle,draw,pin=right:X,
                     pin=above right:Y,
                     pin=above:Z]       {my circle};
\end{codeexample}
  \end{stylekey}

  \begin{key}{/tikz/pin edge=\meta{options} (initially \normalfont empty)}
    This option can be used to set the options that are to be used
    in the edge created by the |pin| option. 
\begin{codeexample}[]
\tikz[pin distance=10mm]
  \node [circle,draw,pin={[pin edge={blue,thick}]right:X},
                     pin=above:Z]       {my circle};
\end{codeexample}
\begin{codeexample}[]
\tikz [every pin edge/.style={},
       initial/.style={pin={[pin distance=5mm,
                             pin edge={<-,shorten <=1pt}]left:start}}]
  \node [circle,draw,initial] {my circle};
\end{codeexample}
  \end{key}
\end{key}


\subsection{Connecting Nodes: Using Nodes as Coordinates}

\label{section-nodes-connecting}

Once you have defined a node and given it a name, you can use this
name to reference it. This can be done in two ways, see also
Section~\ref{section-node-coordinates}. Suppose you have said
|\path(0,0) node(x) {Hello World!};| in order to define a node named |x|. 
\begin{enumerate}
\item
  Once the node |x| has been defined, you can use
  |(x.|\meta{anchor}|)| wherever you would normally use a normal
  coordinate. This will yield the position at which the given
  \meta{anchor} is in the picture. Note that transformations do not
  apply to this coordinate, that is, |(x.north)| will be the northern
  anchor of |x| even if you have said |scale=3| or |xshift=4cm|. This
  is usually what you would expect.
\item
  You can also just use |(x)| as a coordinate. In most cases, this
  gives the same coordinate as |(x.center)|. Indeed, if the |shape| of
  |x| is |coordinate|, then |(x)| and |(x.center)| have exactly the
  same effect.

  However, for most other shapes, some path construction operations like
  |--| try to be ``clever'' when this they are asked to draw a line
  from such a coordinate or to such a coordinate. When you say
  |(x)--(1,1)|, the |--| path operation will not draw a line from the center
  of |x|, but \emph{from the border} of |x| in the direction going
  towards |(1,1)|. Likewise, |(1,1)--(x)| will also have the line
  end on the border in the direction coming from |(1,1)|.

  In addition to |--|, the curve-to path operation |..| and the path
  operations \verb!-|! and \verb!|-! will also handle nodes without
  anchors correctly. Here is an example, see also
  Section~\ref{section-node-coordinates}:
\begin{codeexample}[]
\begin{tikzpicture}
  \path (0,0) node             (x) {Hello World!}
        (3,1) node[circle,draw](y) {$\int_1^2 x \mathrm d x$};

  \draw[->,blue]   (x) -- (y);
  \draw[->,red]    (x) -| node[near start,below] {label} (y);
  \draw[->,orange] (x) .. controls +(up:1cm) and +(left:1cm) .. node[above,sloped] {label} (y);
\end{tikzpicture}
\end{codeexample}
\end{enumerate}




\subsection{Connecting Nodes: Using the Edge Operation}

\label{section-nodes-edges}

The |edge| operation works like a |to| operation that is added after
the main path has been drawn, much like a node is added after the main
path has been drawn. This allows you to have each |edge| to have a
different appearance. As the |node| operation, an |edge| temporarily
suspends the construction of the current path and a new path $p$ is 
constructed. This new path $p$ will be drawn after the main path has
been drawn. Note that $p$ can be totally different from the main
path with respect to its options. Also note that if there are
several |to| and/or |node| operations in the main path, each
creates its own path(s) and they are drawn in the order that they
are encountered on the path.

\begin{pathoperation}{edge}{\opt{|[|\meta{options}|]|}
    \opt{\meta{nodes}} |(|\meta{coordinate}|)|}
  The effect of the |edge| operation is that after the main path the
  following path is added to the picture:
  \begin{quote}
    |\path[every edge,|\meta{options}|] (\tikztostart) |\meta{path}|;|
  \end{quote}
  Here, \meta{path} is the |to path|. Note that, unlike the path added
  by the |to| operation, the |(\tikztostart)| is added before the
  \meta{path} (which is unnecessary for the |to| operation, since this
  coordinate is already part of the main path).

  The |\tikztostart| is the last coordinate on the path just before
  the |edge| operation, just as for the |node| or |to| operations. 
  However, there is one exception to this rule: If the |edge|
  operation is directly preceded by a |node| operation, then this
  just-declared node is the start coordinate (and not, as would
  normally be the case, the coordinate where this just-declared node
  is placed -- a small, but subtle difference). In this regard, |edge|
  differs from both |node| and |to|.
  
  If there are several |edge| operations in a row, the start coordinate
  is the same for all of them as their target coordiantes are not,
  after all, part of the main path. The start coordinate is, thus, the
  coordinate preceding the first |edge| operation. This is
  similar to nodes insofar as the |edge| operation does not modify the
  current path at all. In particular, it does not change the last
  coordinate visited, see the following example:
  
\begin{codeexample}[]
\begin{tikzpicture}
  \node (a) at   (0:1) {$a$};
  \node (b) at  (90:1) {$b$} edge [->]     (a);
  \node (c) at (180:1) {$c$} edge [->]     (a)
                             edge [<-]     (b);
  \node (d) at (270:1) {$d$} edge [->]     (a)
                             edge [dotted] (b)
                             edge [<-]     (c);  
\end{tikzpicture}
\end{codeexample}

  A different way of specifying the above graph using the |edge|
  operation is the following:  

\begin{codeexample}[]
\begin{tikzpicture}
  \foreach \name/\angle in {a/0,b/90,c/180,d/270}
    \node (\name) at (\angle:1) {$\name$};

  \path[->] (b) edge (a)
                edge (c)
                edge [-,dotted] (d)
            (c) edge (a)
                edge (d)
            (d) edge (a);
\end{tikzpicture}
\end{codeexample}

  As can be seen, the path of the |edge| operation inherits the
  options from the main path, but you can locally overrule them.

\begin{codeexample}[]
\begin{tikzpicture}
  \foreach \name/\angle in {a/0,b/90,c/180,d/270}
    \node (\name) at (\angle:1.5) {$\name$};

  \path[->] (b) edge            node[above right]  {$5$}     (a)
                edge                                         (c)
                edge [-,dotted] node[below,sloped] {missing} (d)
            (c) edge                                         (a)
                edge                                         (d)
            (d) edge [red]      node[above,sloped] {very}  
                                node[below,sloped] {bad}     (a);
\end{tikzpicture}
\end{codeexample}

  Instead of |every to|, the style |every edge| is installed at the
  beginning of the main path.
  \begin{stylekey}{/tikz/every edge (inititially draw)}
    Executed for each |edge|.
\begin{codeexample}[]
\begin{tikzpicture}[every to/.style={draw,dashed}]
  \path (0,0) to (3,2);
\end{tikzpicture}
\end{codeexample}
  \end{stylekey}
\end{pathoperation}


\subsection{Referencing Nodes Outside the Current Pictures}

\label{section-cross-picture-tikz}

\subsubsection{Referencing a Node in a Different Picture}

It is possible (but not quite trivial) to reference nodes in pictures
other than the current one. This means that you can create a picture
and a node therein and, later, you can draw a line from some other
position to this node. 

To reference nodes in different pictures, proceed as follows:
\begin{enumerate}
\item You need to add the |remember picture| option to all pictures
  that contain nodes that you wish to reference and also to all
  pictures from which you wish to reference a node in another
  picture.
\item You need to add the |overlay| option to paths or to whole
  pictures that contain references to nodes in different
  pictures. (This option switches the computation of the
  bounding box off.)
\item You need to use a driver that supports picture remembering and
  you need to run \TeX\ twice.
\end{enumerate}
(For more details on what is going on behind the scenes, see
Section~\ref{section-cross-pictures-pgf}.)

Let us have a look at the effect of these options.
\begin{key}{/tikz/remember picture=\meta{boolean} (initially false)}
  This option tells \tikzname\ that it should attempt to remember the
  position of the current picture on the page. This attempt may fail
  depending on which backend driver is used. Also, even if remembering
  works, the position may only be available on a second run of \TeX.

  Provided that remebering works, you may consider saying
\begin{codeexample}[code only]
\tikzsytle{every picture}+=[remember picture]
\end{codeexample}
  to make \tikzname\ remember all pictures. This will add one line in
  the |.aux| file for each picture in your document -- which typically
  is not very much. Then, you do not have to worry about remembered
  pictures at all.
\end{key}

\begin{key}{/tikz/overlay}
  This option is mainly intended for use when nodes in other pictures
  are referenced, but you can also use it in other situations. The
  effect of this option is that everything within the current scope is
  not taken into consideration when the bounding box of the current
  picture is computed.

  You need to specify this option on all paths (or at least on all
  parts of paths) that contain a reference to a node in another
  picture. The reason is that, otherwise, \tikzname\ will attempt to
  make the current picture large enough to encompass \emph{the node in
    the other picture}. However, on a second run of \TeX\ this will
  create an even bigger picture, leading to larger and larger
  pictures. Unless you know what you are doing, I suggest specifying
  the |overlay| option with all pictures that contain references to
  other pictures.
\end{key}

Let us now have a look at a few examples. These examples work only if
this document is processed with a driver that supports picture
remembering.
\medskip

\noindent\begin{minipage}{\textwidth}
Inside the current text we place two pictures, containing nodes named
|n1| and |n2|, using
\begin{codeexample}[code only]
\tikz[remember picture] \node[circle,fill=red!50] (n1) {};
\end{codeexample}
which yields \tikz[remember picture] \node[circle,fill=red!50] (n1)
{};, and
\begin{codeexample}[code only]
\tikz[remember picture] \node[fill=blue!50] (n2) {};
\end{codeexample}
yielding the node \tikz[remember picture] \node[fill=blue!50] (n2)
{};. To connect these nodes, we create another picture using the
|overlay| option and also the |remember picture| option.
\begin{codeexample}[]
\begin{tikzpicture}[remember picture,overlay]
  \draw[->,very thick] (n1) -- (n2);
\end{tikzpicture}
\end{codeexample}
Note that the last picture is seemingly empty. What happens is that it
has zero size and contains an arrow that lies well outside its bounds.
As a last example, we connect a node in another picture to the first
two nodes. Here, we provide the |overlay| option only with the line
that we do not wish to count as part of the picture.
\begin{codeexample}[]
\begin{tikzpicture}[remember picture]
  \node (c) [circle,draw] {Big circle};
  
  \draw [overlay,->,very thick,red,opacity=.5]
    (c) to[bend left] (n1) (n1) -| (n2);
\end{tikzpicture}
\end{codeexample}
\end{minipage}


\subsubsection{Referencing the Current Page Node -- Absolute Positioning}

There is a special node called |current page| that can be used to
access the current page. It is a node of shape rectangle whose
|south west| anchor is the lower left corner of the page and whose
|north east| anchor is the upper right corner of the page. While this
node is handled in a special way internally, you can reference it as
if it were defined in some remembered picture other than the current
one. Thus, by giving the |remembered picture| and the |overlay|
options to a picture, you can position nodes \emph{absolutely} on a
page.

The first example places some text in the lower left corner of the
current page:
\begin{codeexample}[]
\begin{tikzpicture}[remember picture,overlay]
  \node [xshift=1cm,yshift=1cm] at (current page.south west)
        [text width=7cm,fill=red!20,rounded corners,above right]
  {
    This is an absolutely positioned text in the
    lower left corner. No shipout-hackery is used.
  };
\end{tikzpicture}
\end{codeexample}

The next example adds a circle in the middle of the page.
\begin{codeexample}[]
\begin{tikzpicture}[remember picture,overlay]
  \draw [line width=1mm,opacity=.25]
    (current page.center) circle (3cm);
\end{tikzpicture}
\end{codeexample}

The final example overlays some text over the page (depending on where
this example is found on the page, the text may also be behind the
page).
\begin{codeexample}[]
\begin{tikzpicture}[remember picture,overlay]
  \node [rotate=60,scale=10,text opacity=0.2]
    at (current page.center) {Example};
\end{tikzpicture}
\end{codeexample}



\subsection{Late Code and Options}

All options given to a node only locally affect this one node. While
this is a blessing in most cases, you may sometimes want to cause
options to have effects ``later'' on. The other way round, you may
sometimes note ``only later'' that some options should be added to the
options of a node. The present section describes ways of achieving
these effects. 


\subsubsection{Executing Code After Nodes}

\label{section-nodes-executing}

It is possible to add a path right after a node using the option
|after node path|. The idea is that a style might use this option to
add some additional stuff to the node that has just been
typeset. Examples of such styles include the |label| option and the
|pin| option. 

\begin{key}{/tikz/after node path=\meta{path}}
  The \meta{path} is added to the main path right after the node, as
  if you had given the path thereafter. This option can only be given
  inside the option list of a node and multiple calls of this option
  accumulate. 

  Inside the \meta{path} you have access to the node that has just
  been created via the macro \declare{|\tikzlastnode|}.
\begin{codeexample}[]
\tikz
  \draw node [draw,after node path={(\tikzlastnode) circle (2cm)}]
    {hello};
\end{codeexample}

  Note that in the above example, if we had written |\path| instead of
  |\draw|, the circle would not have been drawn since the circle is
  part of the main path, not part of the node itself. 
\end{key}

\begin{command}{\tikzaddafternodepathoption\marg{code}}
  This command allows you to specify that the \meta{code} should be
  executed at the beginning of the |after node path| of the current
  node. The code will also be executed immediately, but also again at
  the beginning of an |after node path|.
\end{command}


\subsection{Late Options}

A \emph{late option} for a node is an option that is given a long time
after the node has already been constructed.

\begin{key}{/tikz/late options=\meta{options}}
  This option can be given on a path (but not as an argument to a
  |node| path command). It has the following effect: An already
  \meta{existing node} is determined (in a way to be described in a moment)
  and, then, the \meta{options} are executed in a local scope. Most of
  these options will have no effect since you \emph{cannot change the
    appearance of the node,} that is, you cannot change a red node
  into a green node using late options. However, giving the
  the |after node path| option inside the \meta{options} (directly or
  indirectly) does have the desired effect: The after node path gets
  executed with the |\tikzlastnode| set to the determined node.

  The net effect of all this is that you can provide, say, the |label|
  option inside the \meta{options} to a add a label to a node that has
  already been constructed. Likewise, you can use the |on chain|
  option to make an already \meta{existing node} part of a chain.

  The \meta{existing node} is determined as follows: If the
  |name=|\meta{existing node} option is used inside the
  \meta{options}, then this name is used. Otherwise, if the last
  coordinate on the current path was of the form |(|\meta{existing
    node}|)|, then this \meta{existing node} name is used. Otherwise,
  an error results.

\begin{codeexample}[]
\begin{tikzpicture}
  \node (a) [draw,circle] {Hello};
  \path (a) [late options={label=above:world}];
\end{tikzpicture}
\end{codeexample}  
\end{key}


%%% Local Variables: 
%%% mode: latex
%%% TeX-master: "pgfmanual"
%%% End: 

% Copyright 2006 by Till Tantau
%
% This file may be distributed and/or modified
%
% 1. under the LaTeX Project Public License and/or
% 2. under the GNU Free Documentation License.
%
% See the file doc/generic/pgf/licenses/LICENSE for more details.

\section{Matrices and Alignment}

\label{section-matrices}

\subsection{Overview}

When creating pictures, one often faces the problem of correctly
aligning parts of the picture. For example, you might wish that the
base lines of certain nodes should be on the same line and some
further nodes should be below these nodes with, say, their centers on
a vertical lines. There are different ways of solving such
problems. For example, by making clever use of anchors, nearly all
such alignment problems can be solved. However, this often leads to
complicated code. An often simpler way is to use \emph{matrices},
the use of which is explaied in the current section.

A \tikzname\ matrix is similar to \LaTeX's |{tabular}| or
|{array}| environment, only instead of text each cell contains a
little picture or a node. The sizes of the cells are automatically
adjusted such that they are large enough to contain all the cell
contents.

Matrices are a powerful tool and they need to handled with some care.
For impatient readers who skip the rest of this section: you
\emph{must} end \emph{every} row with |\\|. In particular, the last
row \emph{must} be ended with |\\|.

Many of the ideas implemented in \tikzname's matrix support are due to
Mark Wibrow -- many thanks to Mark at this point!



\subsection{Matrices are Nodes}

Matrices are special in many ways, but for most purposes matrices are
treated like nodes. This means, that you use the |node| path command
to create a matrix and you only use a special option, namely the
|matrix| option, to signal that the node will contain a
matrix. Instead of the usual \TeX-box that makes up the |text| part of
the node's shape, the matrix is used. Thus, in particular, a matrix
can have a shape, this shape can be drawn or filled, it can be used in
a tree, and so on. Also, you can refer to the different anchors of a
matrix. 

\begin{key}{/tikz/matrix=\meta{true or false} (default true)}
  This option can be passed to a |node| path command. It signals that
  the node will contain a matrix.
\begin{codeexample}[]
\begin{tikzpicture}
  \draw[help lines] (0,0) grid (4,2);
  \node [matrix,fill=red!20,draw=blue,very thick] (my matrix) at (2,1)
  {
    \draw (0,0)   circle (4mm); & \node[rotate=10] {Hello};        \\
    \draw (0.2,0) circle (2mm); & \fill[red]   (0,0) circle (3mm); \\
  };

  \draw [very thick,->] (0,0) |- (my matrix.west);
\end{tikzpicture}
\end{codeexample}
  The exact syntax of the matrix is explained in the course of this
  section.
  \begin{stylekey}{/tikz/every matrix (initially \normalfont empty)}
    This style is used in every matrix.
  \end{stylekey}
\end{key}

Even more so than nodes, matrices will often be the only object on a
path. Because of this, there is a special abbreviation for creating matrices:

\begin{command}{\matrix}
  Inside |{tikzpicture}| this is an abbreviation for |\path node[matrix]|.
\end{command}

Even though matrices are nodes, some options do not have the same
effect as for normal nodes:
\begin{enumerate}
\item Rotations and scaling have no effect on a matrix as a whole
  (however, you can still transform the contents of the cells
  normally). Before the matrix is typeset, the rotational and scaling
  part of the transformation matrix is reset.
\item For multi-part shapes you can only set the |text| part of the
  node. 
\item All options starting with |text| such as |text width| have no
  effect.
\end{enumerate}



\subsection{Cell Pictures}

A matrix consists of rows of \emph{cells}. Each row (including the
last one!) is ended by the command |\\|. The character |&| is used
to separate cells. Inside each cell, you must place commands for
drawing a picture, called the \emph{cell picture} in the
following. (However, cell pictures are not enclosed in a complete
|{pgfpicture}| environment, they are a bit more light-weight. The main
difference is that cell pictures cannot have layers.) It is not
necessary to specify beforehand how many rows or columns there are
going to be and if a row contains less cell pictures than another
line, empty cells are automatically added as needed.


\subsubsection{Alignment of Cell Pictures}

For each cell picture a bounding box is computed. These bounding boxes
and the origins of the cell pictures determine how the cells are
aligned. Let us start with the rows: Consider the cell pictures on the first
row. Each has a bounding box and somewhere inside this bounding box
the origin of the cell picture can be found (the origin might even lie
outside the bounding box, but let us ignore this problem for the
moment). The cell pictures are then shifted around such that all
origins lie on the same horizontal line. This may make it necessary to
shift some cell pictures upwards and other downwards, but it can be
done and this yields the vertical alignment of the cell pictures this
row. The top of the row is then given by the top of the ``highest''
cell picture in the row, the bottom of the row is given by the bottom
of the lowest cell picture. (To be more precise, the height of the row
is the maximum $y$-value of any of the bounding boxes and the depth of
the row is the negated minimum $y$-value of the bounding boxes).

\begin{codeexample}[]
\begin{tikzpicture}
  [every node/.style={draw=black,anchor=base,font=\huge}]

  \matrix [draw=red]
  {
    \node {a}; \fill[blue] (0,0) circle (2pt); &
    \node {X}; \fill[blue] (0,0) circle (2pt); &
    \node {g}; \fill[blue] (0,0) circle (2pt); \\
  };
\end{tikzpicture}
\end{codeexample}

Each row is aligned in this fashion: For each row the cell pictures
are vertically aligned such that the origins lie on the same
line. Then the second row is placed below the first row such that the
bottom of the first row touches the top of the second row (unless a
|row sep| is used to add a bit of space). Then the bottom of the
second row touches the top of the third row, and so on. Typically,
each row will have an individual height and depth.

\begin{codeexample}[]
\begin{tikzpicture}
  [every node/.style={draw=black,anchor=base}]

  \matrix [draw=red]
  {
    \node {a}; & \node {X}; & \node {g}; \\
    \node {a}; & \node {X}; & \node {g}; \\
  };

  \matrix [row sep=3mm,draw=red] at (0,-2)
  {
    \node {a}; & \node {X}; & \node {g}; \\
    \node {a}; & \node {X}; & \node {g}; \\
  };
\end{tikzpicture}
\end{codeexample}

Let us now have a look at the columns. The rules for how the pictures
on any given column are aligned are very similar to the row
alignment: Consider all cell pictures in the first column. Each is
shifted horizontally such that the origins lie on the same vertical
line. Then, the left end of the column is at the left end of the
bounding box that protrudes furthest to the left. The right end of the
column is at the right end of the bounding box that protrudes furthest
to the left. This fixes the horizontal alignment of the cell pictures
in the first column and the same happens the cell pictures in the
other columns. Then, the right end of the first column touches the
left end of the second column (unless |column sep| is used). The right
end of the second column touches the left end of the third column, and
so on. (Internally, two columns are actually used to achieve the
desired horizontal alignment, but that is only and implementation
detail.) 

\begin{codeexample}[]
\begin{tikzpicture}[every node/.style={draw}]
  \matrix [draw=red]
  {
    \node[left]  {Hallo}; \fill[blue] (0,0) circle (2pt); \\
    \node        {X};     \fill[blue] (0,0) circle (2pt); \\
    \node[right] {g};     \fill[blue] (0,0) circle (2pt); \\
  };
\end{tikzpicture}
\end{codeexample}

\begin{codeexample}[]
\begin{tikzpicture}[every node/.style={draw}]
  \matrix [draw=red,column sep=1cm]
  {
    \node {8}; & \node{1}; & \node {6}; \\
    \node {3}; & \node{5}; & \node {7}; \\
    \node {4}; & \node{9}; & \node {2}; \\
  };
\end{tikzpicture}
\end{codeexample}



\subsubsection{Setting and Adjusting Column and Row Spacing}

There are different ways of setting and adjusting the spacing between
columns and rows. First, you can use the options |column sep| and
|row sep| to set a default spacing for all rows and all
columns. Second, you can add options to the |&| character and the |\\|
command to adjust the spacing between two specific columns or
rows. Additionally, you can specify whether the space between two
columns or rows should be considered between the origins of cells in
the column or row or between their borders. 

\begin{key}{/tikz/column sep=\meta{spacing list}}
  This option sets a default space that is added between every two
  columns. This space can be positive or negative and is zero by
  default. The \meta{spacing list} normally contains a single
  dimension like |2pt|.
\begin{codeexample}[]
\begin{tikzpicture}
  \matrix [draw,column sep=1cm,nodes=draw]
  {
    \node(a) {123}; & \node (b) {1};   & \node {1}; \\
    \node    {12};  & \node     {12};  & \node {1}; \\
    \node(c) {1};   & \node (d) {123}; & \node {1}; \\
  };
  \draw [red,thick]  (a.east) -- (a.east |- c)
                     (d.west) -- (d.west |- b);
  \draw [<->,red,thick] (a.east) -- (d.west |- b)
    node [above,midway] {1cm};
\end{tikzpicture}
\end{codeexample}
  More generally, the \meta{spacing list} may contain a whole list of
  numbers, separated by commas, and occurrences of the two key words
  |between origins| and |between borders|. The effect of specifying
  such a list is the following: First, all numbers occurring in the
  list are simply added to compute the final spacing. Second,
  concerning the two keywords, the last occurrence of one of the keywords is
  important. If the last occurrence is |between borders| or if neither
  occurs, then the space is inserted between the two columns
  normally. However, if the last occurs is |between origins|, then the 
  following happens: The distance between the columns is adjusted such
  that the difference between the origins of all the cells in the
  first column (remember that they all lie on straight line) and the
  origins of all the cells in the second column is exactly the given
  distance.

  \emph{The |between origins| option can only be used for columns
    mentioned in the first row, that is, you cannot specify this
    option for columns introduced only in later rows.}
  
\begin{codeexample}[]
\begin{tikzpicture}
  \matrix [draw,column sep={1cm,between origins},nodes=draw]
  {
    \node(a) {123}; & \node (b) {1};   & \node {1}; \\
    \node    {12};  & \node     {12};  & \node {1}; \\
    \node    {1};   & \node     {123}; & \node {1}; \\
  };
  \draw [<->,red,thick] (a.center) -- (b.center) node [above,midway] {1cm};
\end{tikzpicture}
\end{codeexample}
\end{key}

\begin{key}{/tikz/row sep=\meta{spacing list}}
  This option works like |column sep|, only for rows. Here, too, you
  can specify whether the space is added between the lower end of the
  first row and the upper end of the second row, or whether the space
  is computed between the origins of the two rows.
\begin{codeexample}[]
\begin{tikzpicture}
  \matrix [draw,row sep=1cm,nodes=draw]
  {
    \node (a) {123}; & \node {1};   & \node {1}; \\
    \node (b) {12};  & \node {12};  & \node {1}; \\
    \node     {1};   & \node {123}; & \node {1}; \\
  };
  \draw [<->,red,thick] (a.south) -- (b.north) node [right,midway] {1cm};
\end{tikzpicture}
\end{codeexample}
\begin{codeexample}[]
\begin{tikzpicture}
  \matrix [draw,row sep={1cm,between origins},nodes=draw]
  {
    \node (a) {123}; & \node {1};   & \node {1}; \\
    \node (b) {12};  & \node {12};  & \node {1}; \\
    \node     {1};   & \node {123}; & \node {1}; \\
  };
  \draw [<->,red,thick] (a.center) -- (b.center) node [right,midway] {1cm};
\end{tikzpicture}
\end{codeexample}
\end{key}

The row-end command |\\| allows you to provide an optional
argument, which must be a dimension. This dimension will be added to
the list in |row sep|. This means that, firstly, any numbers you list
in this argument will be added as an extra row separation between the
line being ended and the next line and, secondly, you can use the
keywords |between origins| and |between borders| to locally overrule
the standard setting for this line pair.
\begin{codeexample}[]
\begin{tikzpicture}
  \matrix [row sep=1mm]
  {
    \draw (0,0) circle (2mm); & \draw (0,0) circle (2mm); \\
    \draw (0,0) circle (2mm); & \draw (0,0) circle (2mm); \\[-1mm]
    \draw (0,0) coordinate (a) circle (2mm); &
    \draw (0,0) circle (2mm); \\[1cm,between origins]
    \draw (0,0) coordinate (b) circle (2mm); &
    \draw (0,0) circle (2mm); \\
  };
  \draw [<->,red,thick] (a.center) -- (b.center) node [right,midway] {1cm};
\end{tikzpicture}
\end{codeexample}

The cell separation character |&| also takes an optional
argument, which must also be a spacing list. This spacing list is
added to the |column sep| having a similar effect as the option for
the |\\| command for rows.

This optional spacing list can only be given the first time
a new column is started (usually in the first row), subsequent usages
of this option in later rows have no effect. 
\begin{codeexample}[]
\begin{tikzpicture}
  \matrix [draw,nodes=draw,column sep=1mm]
  {
    \node {8}; &[2mm] \node{1}; &[-1mm] \node {6}; \\
    \node {3}; &      \node{5}; &       \node {7}; \\
    \node {4}; &      \node{9}; &       \node {2}; \\
  };
\end{tikzpicture}
\end{codeexample}
\begin{codeexample}[]
\begin{tikzpicture}
  \matrix [draw,nodes=draw,column sep=1mm]
  {
    \node {8}; &[2mm] \node(a){1}; &[1cm,between origins] \node(b){6}; \\
    \node {3}; &      \node   {5}; &                      \node   {7}; \\
    \node {4}; &      \node   {9}; &                      \node   {2}; \\
  };
  \draw [<->,red,thick] (a.center) -- (b.center) node [above,midway] {11mm};
\end{tikzpicture}
\end{codeexample}
\begin{codeexample}[]
\begin{tikzpicture}
  \matrix [draw,nodes=draw,column sep={1cm,between origins}]
  {
    \node (a) {8}; & \node (b) {1}; &[between borders] \node (c) {6}; \\
    \node     {3}; & \node     {5}; &                  \node     {7}; \\
    \node     {4}; & \node     {9}; &                  \node     {2}; \\
  };
  \draw [<->,red,thick] (a.center) -- (b.center) node [above,midway] {10mm};
  \draw [<->,red,thick] (b.east) -- (c.west) node [above,midway] {10mm};
\end{tikzpicture}
\end{codeexample}




\subsubsection{Cell Styles and Options}

For following style and option are useful for changing the appearance
of the all cell pictures:

\begin{stylekey}{/tikz/every cell=\marg{row}\marg{column} (initially \normalfont empty)}
  This style is installed at the beginning of each cell picture with
  the two parameters being the current \meta{row} and \meta{column} of
  the cell. Note that setting this style to |draw| will \emph{not}
  cause all nodes to be drawn since the |draw| option has to be passed
  to each node individually.

  Inside this style (and inside all cells), the current \meta{row} and
  \meta{column} number are also accessible via the counters
  |\pgfmatrixcurrentrow| and |\pgfmatrixcurrentcolumn|.   
\end{stylekey}

\begin{key}{/tikz/cells=\meta{options}}
  This key adds the \meta{options} to the style |every cell|. It mainly
  just a shorthand for the code
  |every cell/.append style=|\meta{options}.
\end{key}

\begin{key}{/tikz/nodes=\meta{options}}
  This key adds the \meta{options} to the style |every node|. It mainly
  just a shorthand for the code |every node/.append style=|\meta{options}.

  The main use of this option is the install some options for the
  nodes \emph{inside} the matrix that should not apply to the matrix
  \emph{itself}. 

\begin{codeexample}[]
\begin{tikzpicture}
  \matrix [nodes={fill=blue!20,minimum size=5mm}]
  {
    \node {8}; & \node{1}; & \node {6}; \\
    \node {3}; & \node{5}; & \node {7}; \\
    \node {4}; & \node{9}; & \node {2}; \\
  };
\end{tikzpicture}
\end{codeexample}
\end{key}

The next set of styles can be used to change the appearance of certain
rows, columns, or cells. If more than one of these styles is defined,
they are executed in the below order (the |every cell| style is
executed before all of the below).
\begin{stylekey}{/tikz/column \meta{number}}
  This style is used for every cell in column \meta{number}.
\end{stylekey}

\begin{stylekey}{/tikz/every odd column}
  This style is used for every cell in an odd column.
\end{stylekey}

\begin{stylekey}{/tikz/every even column}
  This style is used for every cell in an even column.
\end{stylekey}

\begin{stylekey}{/tikz/row \meta{number}}
  This style is used for every cell in row \meta{number}.
\end{stylekey}

\begin{stylekey}{/tikz/every odd row}
  This style is used for every cell in an odd row.
\end{stylekey}

\begin{stylekey}{/tikz/every even row}
  This style is used for every cell in an even row.
\end{stylekey}

\begin{stylekey}{/tikz/row \meta{row number} column \meta{column number}}
  This style is used for the cell in row \meta{row number} and column
  \meta{column number}.
\end{stylekey}


\begin{codeexample}[]
\begin{tikzpicture}
  [row 1/.style={red},
   column 2/.style={green!50!black},
   row 3 column 3/.style={blue}]
    
  \matrix
  {
    \node {8}; & \node{1}; & \node {6}; \\
    \node {3}; & \node{5}; & \node {7}; \\
    \node {4}; & \node{9}; & \node {2}; \\
  };
\end{tikzpicture}
\end{codeexample}

You can use the |column |\meta{number} option to change the alignment
for different columns.

\begin{codeexample}[]
\begin{tikzpicture}
  [column 1/.style={anchor=base west},
   column 2/.style={anchor=base east},
   column 3/.style={anchor=base}]
  \matrix
  {
    \node {123}; & \node{456}; & \node {789}; \\
    \node {12}; & \node{45}; & \node {78}; \\
    \node {1}; & \node{4}; & \node {7}; \\
  };
\end{tikzpicture}
\end{codeexample}


In many matrices all cell pictures have nearly the same code. For
example, cells typically start with |\node{| and end |};|. The
following options allow you to execute such code in all cells:

\begin{key}{/tikz/execute at begin cell=\meta{code}}
  The code will be executed at the beginning of each nonempty cell.
\end{key}
\begin{key}{/tikz/execute at end cell=\meta{code}}
  The code will be executed at the end of each nonempty cell.
\end{key}
\begin{key}{/tikz/execute at empty cell=\meta{code}}
  The code will be executed inside each empty cell.
\end{key}

\begin{codeexample}[]
\begin{tikzpicture}
  [matrix of nodes/.style={
     execute at begin cell=\node\bgroup,
     execute at end cell=\egroup;%
   }]
  \matrix [matrix of nodes]
  {
    8 & 1 & 6 \\
    3 & 5 & 7 \\
    4 & 9 & 2 \\
  };
\end{tikzpicture}
\end{codeexample}
\begin{codeexample}[]
\begin{tikzpicture}
  [matrix of nodes/.style={
     execute at begin cell=\node\bgroup,
     execute at end cell=\egroup;,%
     execute at empty cell=\node{--};%
   }]
  \matrix [matrix of nodes]
  {
    8 & 1 &   \\
    3 &   & 7 \\
      &   & 2 \\
  };
\end{tikzpicture}
\end{codeexample}

The |matrix| library defines a number of styles that make use of the
above options.




\subsection{Anchoring a Matrix}

Since matrices are nodes, they can be anchored in the usual fashion
using the |anchor| option. However, there are two ways to influence
this placement further. First, the following option is often useful:

\begin{key}{/tikz/matrix anchor=\meta{anchor}}
  This option has the same effect as |anchor|, but the option applies
  only to the matrix itself, not to the cells inside. If you just say
  |anchor=north| as an option to the matrix node, all nodes inside
  matrix will also have this anchor, unless it is explicitly set
  differently for each node. By comparison, |matrix anchor| sets the
  anchor for the matrix, but for the nodes inside the value of
  |anchor| remain unchanged.

\begin{codeexample}[]
\begin{tikzpicture}
  \matrix [matrix anchor=west] at (0,0)
  {
    \node {123}; \\ % still center anchor
    \node {12}; \\
    \node {1}; \\
  };
  \matrix [anchor=west] at (0,-2)
  {
    \node {123}; \\ % inherited west anchor
    \node {12}; \\
    \node {1}; \\
  };
\end{tikzpicture}
\end{codeexample}
\end{key}

The second way to anchor a matrix is to use \emph{an anchor of a node
  inside the matrix}. For this, the |anchor| option has a special
effect when given as an argument to a matrix:

\begin{key}{/tikz/anchor=\meta{anchor or node.anchor}}
  Normally, the argument of this option refers to anchor of the matrix
  node, which is the node than includes all of the stuff of the
  matrix. However, you can also provide an argument of the form
  \meta{node}|.|\meta{anchor} where \meta{node} must be node defined
  inside the matrix and \meta{anchor} is an anchor of this node. In
  this case, the whole matrix is shifted around in such a way that
  this particular anchor of this particular node lies at the |at|
  position of the matrix. The same is true for |matrix anchor|.

\begin{codeexample}[]
\begin{tikzpicture}
  \draw[help lines] (0,0) grid (3,2);
  \matrix[matrix anchor=inner node.south,anchor=base,row sep=3mm] at (1,1)
  {
    \node {a}; & \node             {b}; & \node {c}; & \node {d}; \\
    \node {a}; & \node(inner node) {b}; & \node {c}; & \node {d}; \\
    \node {a}; & \node             {b}; & \node {c}; & \node {d}; \\
  };
  \draw (inner node.south) circle (1pt);
\end{tikzpicture}
\end{codeexample}
\end{key}


\subsection{Considerations Concerning Active Characters}

Even though \tikzname\ seems to use |&| to separate cells, \pgfname\ actually
uses a different command to separate cells, namely the command
|\pgfmatrixnextcell| and using a normal |&| character will normally
fail. What happens is that, \tikzname\ makes |&| an active character
and then defines this character to be equal to
|\pgfmatrixnextcell|. In most situations this will work 
nicely, but sometimes |&| cannot be made active; for
instance because the matrix is used in an argument of some macro or
the matrix contains nodes that contain normal |{tabular}|
environments. In this case you can use the following option to avoid
having to type |\pgfmatrixnextcell| each time:

\begin{key}{/tikz/ampersand replacement=\meta{macro name or empty}}
  If a macro name is provided, this macro will be defined to be equal
  to |\pgfmatrixnextcell| inside matrices and |&| will not be made
  active. For instance, you could say |ampersand replacement=\&| and
  then use \& to separate columns as in the following example:
\begin{codeexample}[]
\tikz
  \matrix [ampersand replacement=\&]
  {
    \draw (0,0)   circle (4mm); \& \node[rotate=10] {Hello};        \\
    \draw (0.2,0) circle (2mm); \& \fill[red]   (0,0) circle (3mm); \\
  };
\end{codeexample}
\end{key}


\subsection{Examples}

The following examples are adapted from code by Mark Wibrow. The first
two redraw pictures from Timothy van Zandt's PSTricks documentation: 

{\catcode`\|=12
\begin{codeexample}[]
\begin{tikzpicture} 
  \matrix [matrix of math nodes,row sep=1cm]
  { 
    |(U)| U &[2mm]                       &[8mm]    \\ 
            &      |(XZY)| X \times_Z Y  &      |(X)| X \\ 
            &      |(Y)|   Y             &      |(Z)| Z \\
  }; 
  \begin{scope}[every node/.style={midway,auto,font=\scriptsize}] 
    \draw [double, dashed] (U)   -- node {$x$} (X); 
    \draw                  (X)   -- node {$p$} (X -| XZY.east)
                           (X)   -- node {$f$} (Z)
                                 -- node {$g$} (Y)
                                 -- node {$q$} (XZY)
                                 -- node {$y$} (U);
   \end{scope}
\end{tikzpicture} 
\end{codeexample}

\begin{codeexample}[]
\begin{tikzpicture}[>=stealth,->,shorten >=2pt,looseness=.5,auto] 
  \matrix [matrix of math nodes,
           column sep={2cm,between origins},
           row sep={3cm,between origins},
           nodes={circle, draw, minimum size=7.5mm}]
  { 
            & |(A)| A &         \\ 
    |(B)| B & |(E)| E & |(C)| C \\ 
            & |(D)| D           \\
  }; 
  \begin{scope}[every node/.style={font=\small\itshape}]
    \draw (A) to [bend left] node [midway]   {g} (B); 
    \draw (B) to [bend left] node [midway]   {f} (A); 
    \draw (D) --             node [midway]   {c} (B); 
    \draw (E) --             node [midway]   {b} (B); 
    \draw (E) --             node [near end] {a} (C); 
    \draw [-,line width=8pt,draw=graphicbackground] 
          (D) to [bend right, looseness=1] (A); 
    \draw (D) to [bend right, looseness=1] 
            node [near start] {b} node [near end] {e} (A); 
  \end{scope}
\end{tikzpicture}
\end{codeexample}

\begin{codeexample}[]
\begin{tikzpicture} 
  \matrix (network)
    [matrix of nodes,%
     nodes in empty cells,
     nodes={outer sep=0pt,circle,minimum size=4pt,draw},
     column sep={1cm,between origins},
     row sep={1cm,between origins}]
  {
                  &                &                 & \\ 
                  &                &                 & \\ 
    |[draw=none]| & |[xshift=1mm]| & |[xshift=-1mm]|   \\
  }; 
  \foreach \a in {1,...,4}{ 
    \draw (network-3-2) -- (network-2-\a); 
    \draw (network-3-3) -- (network-2-\a); 
    \draw [-stealth] ([yshift=5mm]network-1-\a.north) -- (network-1-\a); 
    \foreach \b in {1,...,4} 
      \draw (network-1-\a) -- (network-2-\b); 
  } 
  \draw [stealth-] ([yshift=-5mm]network-3-2.south) -- (network-3-2); 
  \draw [stealth-] ([yshift=-5mm]network-3-3.south) -- (network-3-3); 
\end{tikzpicture} 
\end{codeexample}

The following example is adapted from code written by Kjell Magne
Fauske, which is based on the following paper: K.~Bossley, M.~Brown, 
and C.~Harris, Neurofuzzy identification of an autonomous underwater
vehicle, \emph{International Journal of Systems Science}, 1999, 30, 901--913.

\begin{codeexample}[]
\begin{tikzpicture}
  [auto, 
   decision/.style={diamond, draw=blue, thick, fill=blue!20, 
                    text width=4.5em,align=flush center,
                    inner sep=1pt}, 
   block/.style   ={rectangle, draw=blue, thick, fill=blue!20, 
                    text width=5em,align=center, rounded corners,
                    minimum height=4em},
   line/.style    ={draw, thick, -latex',shorten >=2pt},
   cloud/.style   ={draw=red, thick, ellipse,fill=red!20,
                    minimum height=2em}]
  
  \matrix [column sep=5mm,row sep=7mm]
  {
    % row 1
      \node [cloud] (expert)   {expert}; & 
      \node [block] (init)     {initialize model}; & 
      \node [cloud] (system)   {system}; \\
    % row 2
      & \node [block] (identify) {identify candidate model}; & \\ 
    % row 3
      \node [block] (update)   {update model};  & 
      \node [block] (evaluate) {evaluate candidate models}; & \\ 
    % row 4
      & \node [decision] (decide) {is best candidate}; & \\ 
    % row 5
      & \node [block] (stop)      {stop}; & \\
  }; 
  \begin{scope}[every path/.style=line]
    \path          (init)     -- (identify); 
    \path          (identify) -- (evaluate); 
    \path          (evaluate) -- (decide); 
    \path          (update)   |- (identify); 
    \path          (decide)   -| node [near start] {yes} (update); 
    \path          (decide)   -- node [midway] {no} (stop); 
    \path [dashed] (expert)   -- (init); 
    \path [dashed] (system)   -- (init); 
    \path [dashed] (system)   |- (evaluate); 
  \end{scope}
\end{tikzpicture} 
\end{codeexample}
}

%%% Local Variables: 
%%% mode: latex
%%% TeX-master: "pgfmanual"
%%% End: 

% Copyright 2006 by Till Tantau
%
% This file may be distributed and/or modified
%
% 1. under the LaTeX Project Public License and/or
% 2. under the GNU Free Documentation License.
%
% See the file doc/generic/pgf/licenses/LICENSE for more details.

\section{Making Trees Grow}

\label{section-trees}


\subsection{Introduction to the  Child Operation}

\emph{Trees} are a common way of visualizing hierarchical
structures. A simple tree looks like this:
\begin{codeexample}[]
\begin{tikzpicture}
  \node {root}
    child {node {left}}
    child {node {right}
      child {node {child}}
      child {node {child}}
    };
\end{tikzpicture}
\end{codeexample}

Admittedly, in reality trees are more likely to grow \emph{upward} and
not downward as above. You can tell whether the author of a paper is a
mathematician or a computer scientist by looking at the direction
their trees grow. A computer scientist's trees will grow downward
while a mathematician's tree will grow upward. Naturally, the
\emph{correct} way is the mathematician's way, which can be specify as
follows: 
\begin{codeexample}[]
\begin{tikzpicture}
  \node {root} [grow'=up]
    child {node {left}}
    child {node {right}
      child {node {child}}
      child {node {child}}
    };
\end{tikzpicture}
\end{codeexample}

In \tikzname, trees are specified by adding \emph{children} to a
node on a path using the |child| operation:

\begin{pathoperation}{child}{\opt{\oarg{options}}%
    \opt{|foreach|\meta{variables}|in|\marg{values}}\opt{\marg{child path}}} 
  This operation should directly follow a completed |node| operation
  or another |child| operation, although it is permissible that the
  first |child| operation is preceded by options (we will come to
  that).

  When a |node| operation like |node {X}| is followed by |child|,
  \tikzname\ starts counting the number of child nodes that follow the
  original |node {X}|. For this, it scans the input and stores away each
  |child| and its arguments until it reaches a path operation that is
  not a |child|. Note that this will fix the character codes of all
  text inside the child arguments, which means, in essence, that you
  cannot use verbatim text inside the nodes inside a |child|. Sorry. 

  Once the children have been collected and counted, \tikzname\ starts
  generating the child nodes. For each child of a parent node
  \tikzname\ computes an appropriate position where the child is
  placed. For each child, the coordinate system is transformed so that
  the origin is at this position. Then the \meta{child path} is
  drawn. Typically, the child path just consists of a |node|
  specification, which results in a node being drawn at the child's
  position. Finally, an edge is drawn from the first node in the
  \meta{child path} to the parent node.

  The optional |foreach| part (note that there is no backslash before
  |foreach|) allows you to specify multiple children in a single
  |child| command. The idea is the following: A |\foreach| statement
  is (internally) used to iterate over the list of \meta{values}. For
  each value in this list, a new |child| is added to the node. The
  syntax for \meta{variables} and for \meta{values} is the same as for
  the |\foreach| statement, see Section~\ref{section-foreach}. For
  example, when you say 
\begin{codeexample}[code only]
node {root} child [red] foreach \name in {1,2} {node {\name}}
\end{codeexample}
  the effect will be the same as if you had said
\begin{codeexample}[code only]
node {root} child[red] {node {1}} child[ref] {node {2}}
\end{codeexample}
  When you write 
\begin{codeexample}[code only]
node {root} child[\pos] foreach \name/\pos in {1/left,2/right} {node[\pos] {\name}}
\end{codeexample}
  the effect will be the same as for
\begin{codeexample}[code only]
node {root} child[left] {node[left] {1}} child[right] {node[right] {2}}
\end{codeexample}

  You can nest things as in the following example:
\begin{codeexample}[]
\begin{tikzpicture}[level distance=4mm]
  \tikzstyle{level 1}=[sibling distance=8mm]
  \tikzstyle{level 2}=[sibling distance=4mm]
  \tikzstyle{level 3}=[sibling distance=2mm]
  \coordinate
    child foreach \x in {0,1}
      {child foreach \y in {0,1} 
        {child foreach \z in {0,1}}};
\end{tikzpicture}
\end{codeexample}

  The details and options for this operation are described in the rest
  of this present section.
\end{pathoperation}



\subsection{Child Paths and the Child Nodes}

For each |child| of a root node, its \meta{child path} is inserted at
a specific location in the picture (the placement rules are discussed
in Section~\ref{section-tree-placement}). The first node in the
\meta{child path}, if it exists, is special and called the \emph{child
  node}. If there is no first node in the \meta{child path}, that is,
if the \meta{child path} is missing (including the curly braces) or if
it does not start with |node| or with |coordinate|, then an empty
child node of shape |coordinate| is automatically added.

Consider the example |\node {x} child {node {y}} child;|. For the
first child, the \meta{child path} has the child node |node {y}|. For
the second child, no child node is specified and, thus, it is just
|coordinate|.

As for any normal node, you can give the child node a name, shift it 
around, or use options to influence how it is rendered.
\begin{codeexample}[]
\begin{tikzpicture}
  \node[rectangle,draw] {root}
    child {node[circle,draw] (left node) {left}}
    child {node[ellipse,draw] (right node) {right}};
  \draw[dashed,->] (left node) -- (right node);
\end{tikzpicture}
\end{codeexample}

In many cases, the \meta{child path} will just consist of a
specification of a child node and, possibly, children of this child
node. However, the node specification may be followed by arbitrary
other material that will be added to the picture, transformed to the
child's coordinate system. For your convenience, a move-to |(0,0)|
operation is inserted automatically at the beginning of the path. Here
is an example: 

\begin{codeexample}[]
\begin{tikzpicture}
  \node {root}
    child {[fill] circle (2pt)}
    child {[fill] circle (2pt)};
\end{tikzpicture}    
\end{codeexample}


At the end of the \meta{child path} you may add a special path
operation called |edge from parent|. If this operation is not given by
yourself somewhere on the path, it will be automatically added at the
end. This option causes a connecting edge from the parent node to the
child node to be added to the path. By giving options to this
operation you can influence how the edge is rendered. Also, nodes
following the |edge from parent| operation will be placed on this
edge, see Section~\ref{section-edge-from-parent} for details.

To sum up:
\begin{enumerate}
\item
  The child path starts with a node specification. If it is not there,
  it is added automatically.
\item
  The child path ends with a |edge from parent| operation, possibly
  followed by nodes to be put on this edge. If the operation is not
  given at the end, it is added automatically.
\end{enumerate}



\subsection{Naming Child Nodes}

Child nodes can be named like any other node using either the |name|
option or the special syntax in which the name of the node is placed
in round parentheses between the |node| operation and the node's
text.

If you do not assign a name to a child node, \tikzname\ will
automatically assign a name as follows: Assume that the name of the
parent node is, say, |parent|. (If you did not assign a
name to the parent, \tikzname\ will do so itself, but that name will
not be user-accessible.) The first child
of |parent| will be named |parent-1|, the second child is named
|parent-2|, and so on.

This naming convention works recursively. If the second child
|parent-2| has children, then the first of these children will be
called |parent-2-1| and the second |parent-2-2| and so on.

If you assign a name to a child node yourself, no name is generated
automatically (the node does not have two names). However, ``counting
continues,'' which means that the third child of |parent| is called
|parent-3| independently of whether you have assigned names to the
first and/or second child of |parent|.

Here is an example:

\begin{codeexample}[]
\begin{tikzpicture}
  \node (root) {root}
    child 
    child {
      child {coordinate (special)}
      child
    };
  \node at (root-1) {root-1};
  \node at (root-2) {root-2};
  \node at (special) {special};
  \node at (root-2-2) {root-2-2};
\end{tikzpicture}
\end{codeexample}

\subsection{Specifying Options for Trees and Children}

Each |child| may have its own \meta{options}, which apply to ``the
whole child,'' including all of its grandchildren. Here is an
example:

\begin{codeexample}[]
\begin{tikzpicture}[thick]
  \tikzstyle{level 2}=[sibling distance=10mm]
  \coordinate
    child[red]   {child child}
    child[green] {child child[blue]};
\end{tikzpicture}
\end{codeexample}

The options of the root node have no effect on the children since
the options of a node are always ``local'' to that node. Because of
this, the edges in the following tree are black, not red.
  
\begin{codeexample}[]
\begin{tikzpicture}[thick]
  \node [red] {root}
    child
    child;
\end{tikzpicture}
\end{codeexample}
  This raises the problem of how to set options for \emph{all}
  children. Naturally, you could always set options for the whole path
  as in |\path [red] node {root} child child;| but this is bothersome
  in some situations. Instead, it is easier to give the options
  \emph{before the first child} as follows:
\begin{codeexample}[]
\begin{tikzpicture}[thick]
  \node [red] {root}
    [green] % option applies to all children
    child
    child;
\end{tikzpicture}
\end{codeexample}

Here is the set of rules:
\begin{enumerate}
\item
  Options for the whole tree are given before the root node.
\item
  Options for the root node are given directly to the |node| operation
  of the root.
\item
  Options for all children can be given between the root node and the
  first child.
\item
  Options applying to a specific child path are given as options to
  the |child| operation.
\item
  Options applying to the node of a child, but not to the whole child
  path, are given as options to the |node| command inside the
  \meta{child path}.
\end{enumerate}

\begin{codeexample}[code only]
\begin{tikzpicture}
  \path
    [...]             % Options apply to the whole tree
    node[...] {root}  % Options apply to the root node only
      [...]           % Options apply to all children
      child[...]      % Options apply to this child and all its children
      {
        node[...] {}  % Options apply to the child node only
        ...
      }
      child[...]      % Options apply to this child and all its children
    ;
\end{tikzpicture}
\end{codeexample}

There are additional styles that influence how children are rendered:
\begin{itemize}
  \itemstyle{every child}
  This style is used at the beginning of each child, as if you had
  given the options to the |child| operation.
  \itemstyle{every child node}
  This style is used at the beginning of each child node in addition
  to the |every node| style.
  \itemstyle{level \meta{number}}
  This style is used at the beginning of each set of children, where
  \meta{number} is the current level in the current tree. For example,
  when you say |\node {x} child child;|, then the style |level 1| is
  used before the first |child|. If this first |child| has children
  itself, then |level 2| would be used for them.

\begin{codeexample}[]
\begin{tikzpicture}
  \tikzstyle{level 1}=[sibling distance=20mm]
  \tikzstyle{level 2}=[sibling distance=5mm]
  \node {root}
    child { child child }
    child { child child child };
\end{tikzpicture}
\end{codeexample}
\end{itemize}




\subsection{Placing Child Nodes}

\label{section-tree-placement}

Perhaps the most difficult part in drawing a tree is the correct
layout of the children. Typically, the children have different sizes
and it is not easy to arrange them in such a manner that not too much
space is wasted, the children do not overlap, and they are either 
evenly spaced or their centers are evenly distributed. Calculating
good positions is especially difficult since a good position for the
first child may depend on the size of the last child.

In \tikzname, a comparatively simple approach is taken to placing the
children. In order to compute a child's position, all that is taken
into account is the number of the current child in the list of
children and the number of children in this list. Thus, if a node has
five children, then there is a fixed position for the first child, a
position for the second child, and so on. These positions \emph{do not
  depend on the size of the children} and, hence, children can easily
overlap. However, since you can use options to shift individual
children a bit, this is not as great a problem as it may seem.

Although the placement of the children only depends on their number in
the list of children and the total number of children, everything else
about the placement is highly configurable. You can change the
distance between children (appropriately called the
|sibling distance|) and the distance between levels of the tree. These
distances may change from level to level. The direction in which the
tree grows can be changed globally and for parts of the tree. You can
even specify your own ``growth function'' to arrange children on a
circle or along special lines or curves. 

The default growth function works as follows: Assume that we are given
a node and five children. These children will be placed on a line with
their centers (or, more generally, with their anchors) spaced apart by
the current |sibling distance|. The line is 
orthogonal to the current \emph{direction of growth}, which is set
with the |grow| and |grow'| option (the latter option reverses the
ordering of the children). The distance from the line to the parent node
is given by the |level distance|.

{\catcode`\|=12
\begin{codeexample}[]
\begin{tikzpicture}
  \path [help lines]
    node (root) {root}
    [grow=-10]
    child {node {1}}
    child {node {2}}
    child {node {3}}
    child {node {4}};

  \draw[|<->|,thick] (root-1.center)
    -- node[above,sloped] {sibling distance} (root-2.center);

  \draw[|<->|,thick] (root.center) 
    -- node[above,sloped] {level distance} +(-10:\tikzleveldistance);
\end{tikzpicture}
\end{codeexample}
}

Here is a detailed description of the options:
\begin{itemize}
  \itemoption{level distance}|=|\meta{distance}
  This option allows you to change the distance between different
  levels of the tree, more precisely, between the parent and the line
  on which its children are arranged. When given to a single child,
  this will set the distance for this child only.
 
\begin{codeexample}[]
\begin{tikzpicture}
  \node {root}
    [level distance=20mm]
    child
    child {
      [level distance=5mm]
      child
      child
      child
    }
    child[level distance=10mm];  
\end{tikzpicture}
\end{codeexample}
 
\begin{codeexample}[]
\begin{tikzpicture}
  \tikzstyle{level 1}=[level distance=10mm]    
  \tikzstyle{level 2}=[level distance=5mm]    
  \node {root}
    child
    child {
      child
      child[level distance=10mm]
      child
    }
    child;
\end{tikzpicture}
\end{codeexample}

  \itemoption{sibling distance}|=|\meta{distance}
  This option specifies the distance between the anchors of the
  children of a parent node.   

\begin{codeexample}[]
\begin{tikzpicture}[level distance=4mm]
  \tikzstyle{level 1}=[sibling distance=8mm]
  \tikzstyle{level 2}=[sibling distance=4mm]
  \tikzstyle{level 3}=[sibling distance=2mm]
  \coordinate
     child {
       child {child child}
       child {child child}
     }
     child {
       child {child child}
       child {child child}
     };
\end{tikzpicture}
\end{codeexample}

\begin{codeexample}[]
\begin{tikzpicture}[level distance=10mm]
  \tikzstyle{every node}=[fill=red!60,circle,inner sep=1pt]
  \tikzstyle{level 1}=[sibling distance=20mm,
    set style={{every node}+=[fill=red!45]}]
  \tikzstyle{level 2}=[sibling distance=10mm,
    set style={{every node}+=[fill=red!30]}]
  \tikzstyle{level 3}=[sibling distance=5mm,
    set style={{every node}+=[fill=red!15]}]
  \node {31}
     child {node {30}
       child {node {20}
         child {node {5}}
         child {node {4}}
       }
       child {node {10}
         child {node {9}}
         child {node {1}}
       }
     }
     child {node {20}
       child {node {19}
         child {node {1}}
         child[fill=none] {edge from parent[draw=none]}
       }
       child {node {18}}
     };
\end{tikzpicture}
\end{codeexample}
  
  \itemoption{grow}|=|\meta{direction}
  This option is used to define the \meta{direction} in which the tree
  will grow. The \meta{direction} can either be an angle in degrees or
  one of the following special text strings: |down|, |up|, |left|,
  |right|, |north|, |south|, |east|, |west|, |north east|,
  |north west|, |south east|, and |south west|. All of these have
  ``their obvious meaning,'' so, say, |south west| is the same as the
  angle $-135^\circ$.

  As a side effect, this option installs the default growth function.

  In addition to setting the direction, this option also has a
  seemingly strange effect: It sets the sibling distance for the
  current level to |0pt|, but leaves the sibling distance for later
  levels unchanged.

  This somewhat strange behaviour has a highly desirable effect: If
  you give this option before the list of children of a node starts,
  the ``current level'' is still the parent level. Each child will be
  on a later level and, hence, the sibling distance will be as
  specified originally. This will cause the children to be neatly
  aligned in a line orthogonal to the given \meta{direction}. However,
  if you give this option locally to a single child, then ``current
  level'' will be the same as the child's level. The zero sibling
  distance will then cause the child to be placed exactly at a point
  at distance |level distance| in the direction
  \meta{direction}. However, the children of the child will be placed
  ``normally'' on a line orthogonal to the \meta{direction}.

  These placement effects are best demonstrated by some examples:
\begin{codeexample}[]
\tikz \node {root} [grow=right] child child;
\end{codeexample}

\begin{codeexample}[]
\tikz \node {root} [grow=south west] child child;
\end{codeexample}

\begin{codeexample}[]
\begin{tikzpicture}[level distance=10mm,sibling distance=5mm]
  \node {root}
    [grow=down]
    child
    child
    child[grow=right] {
      child child child
    };  
\end{tikzpicture}
\end{codeexample}

\begin{codeexample}[]
\begin{tikzpicture}[level distance=2em]
  \node {C}
    child[grow=up]    {node {H}}
    child[grow=left]  {node {H}}
    child[grow=down]  {node {H}}
    child[grow=right] {node {C}
        child[grow=up]    {node {H}}
        child[grow=right] {node {H}}
        child[grow=down]  {node {H}}
      edge from parent[double]
        coordinate (wrong)
    };
  \draw[<-,red] ([yshift=-2mm]wrong) -- +(0,-1)
    node[below]{This is wrong!};  
\end{tikzpicture}
\end{codeexample}

\begin{codeexample}[]
\begin{tikzpicture}
  \node[rectangle,draw] (a) at (0,0) {start node};
  \node[rectangle,draw] (b) at (2,1) {end};

  \draw (a) -- (b)
    node[coordinate,midway] {}
      child[grow=100,<-] {node[above] {the middle is here}};
\end{tikzpicture}
\end{codeexample}

  \itemoption{grow'}|=|\meta{direction}
  This option has the same effect as |grow|, only the children are
  arranged in the opposite order.
  \itemoption{growth function}|=|\meta{macro name}
  This rather low-level option allows you to set a new growth
  function. The \meta{macro name} must be the name of a macro without
  parameters. This macro will be called for each child of a node.

  The effect of executing the macro should be the following: It should
  transform the coordinate system in such a way that the origin
  becomes the place where the current child should be anchored. When
  the macro is called, the current coordinate system will be setup
  such that the anchor of the parent node is in the origin. Thus, in
  each call, the \meta{macro name} must essentially do a shift to the
  child's origin. When the macro is called, the \TeX\ counter
  |\tikznumberofchildren| will be set to the total number of children
  of the parent node and the counter |\tikznumberofcurrentchild| will
  be set to the number of the current child.

  The macro may, in addition to shifting the coordinate system, also
  transform the coordinate system further. For example, it could be
  rotated or scaled.

  Additional growth functions are defined in the library, see 
  Section~\ref{section-tree-library}.
\end{itemize}



\subsection{Edges From the Parent Node}

\label{section-edge-from-parent}

Every child node is connected to its parent node via a special kind of
edge called the |edge from parent|. This edge is added to the
\meta{child path} when the following path operation is encountered:

\begin{pathoperation}{edge from parent}{\opt{\oarg{options}}}
  This path operation can only be used inside \meta{child paths} and
  should be given at the end, possibly followed by node specifications
  (we will come to that). If a \meta{child path} does not contain this
  operation, it will be added at the end of the \meta{child path}
  automatically.

  This operation has several effects. The most important is that it
  inserts the current ``edge from parent path'' into the child
  path. The edge from parent path can be set using the following
  option:
  \begin{itemize}
    \itemoption{edge from parent path}|=|\meta{path}
    This options allows you to set the edge from parent path to a new
    path. The default for this path is the following:
    \begin{codeexample}[code only]
(\tikzparentnode\tikzparentanchor) -- (\tikzchildnode\tikzchildanchor)      
    \end{codeexample}
    The |\tikzparentnode| is a macro that will expand to the name of
    the parent node. This works even when you have not assigned a name
    to the parent node, in this case an internal name is automatically
    generated. The |\tikzchildnode| is a macro that expands to the
    name of the child node. The two |...anchor| macros are empty by
    default. So, what is essentially inserted is just the path segment
    |(\tikzparentnode) -- (\tikzchildnode)|; which is exactly an edge
    from the parent to the child.

    You can modify this edge from parent path to achieve all sorts of
    effects. For example, we could replace the straight line by a
    curve as follows:
\begin{codeexample}[]
\begin{tikzpicture}[edge from parent path=
  {(\tikzparentnode.south) .. controls +(0,-1) and +(0,1)
                           .. (\tikzchildnode.north)}]
  \node {root}
    child {node {left}}
    child {node {right}
      child {node {child}}
      child {node {child}}
    };
\end{tikzpicture}
\end{codeexample}

    Further useful edge from parent paths are defined in the tree
    library, see Section~\ref{section-tree-library}.

    As said before, the anchors in the default edge from parent path
    are empty. However, you can set them using the following options:
    \begin{itemize}
      \itemoption{child anchor}|=|\meta{anchor}
      Specifies the anchor where the edge from parent meets the child
      node by setting the macro |\tikzchildanchor| to
      |.|\meta{anchor}.

      If you specify |border| as the \meta{anchor}, then the macro
      |\tikzchildanchor| is set to the empty string. The effect of
      this is that the edge from the parent will meet the child on the
      border at an automatically calculated position.
\begin{codeexample}[]
\begin{tikzpicture}
  \node {root}
    [child anchor=north]
    child {node {left} edge from parent[dashed]}
    child {node {right}
      child {node {child}}
      child {node {child} edge from parent[draw=none]}
    };
\end{tikzpicture}
\end{codeexample}
      \itemoption{parent anchor}|=|\meta{anchor}
      This option works the same way as the |child anchor|, only for
      the parent.
    \end{itemize}
  \end{itemize}

  Besides inserting the edge from parent path, the |edge from parent|
  operation has another effect: The \meta{options} are inserted
  directly before the edge from parent path and the following style is
  also installed prior to inserting the path:
  \begin{itemize}
    \itemstyle{edge from parent}
    This style is inserted right before the edge from parent path and
    before the \meta{options} are inserted. By default, it just draws
    the edge from parent, but you can use it to make the edge look
    different. 
\begin{codeexample}[]
\begin{tikzpicture}
  \tikzstyle{edge from parent}=[draw,red,thick]    
  \node {root}
    child {node {left} edge from parent[dashed]}
    child {node {right}
      child {node {child}}
      child {node {child} edge from parent[draw=none]}
    };
\end{tikzpicture}
\end{codeexample}
  \end{itemize}

  Note: The \meta{options} inserted before the edge from parent path
  is added \emph{apply to the whole child path}. Thus, it is not
  possible to, say, draw a circle in red as part of the child path and
  then have an edge to parent in blue. However, as always, the child
  node is a node and can be drawn in a totally different way.

  Finally, the |edge from parent| operation has one more effect: It
  causes all nodes \emph{following} the operation to be placed on the
  edge. This is the same effect as if you had added the |pos| option
  to all these nodes, see also Section~\ref{section-pos-option}.

  As an example, consider the following code:
\begin{codeexample}[code only]
\node (root) {} child {node (child) {} edge to parent node {label}};    
\end{codeexample}
  The |edge to parent| operation and the following |node| operation
  will, together, have the same effect as if we had said:
\begin{codeexample}[code only]
(root) -- (child) node [pos=0.5] {label}
\end{codeexample}

  Here is a more complicated example:
\begin{codeexample}[]
\begin{tikzpicture}
  \node {root}
    child {
      node {left}
      edge from parent
        node[left] {a}
        node[right] {b}
    }
    child {
      node {right}
        child {
          node {child}
          edge from parent
            node[left] {c}
        }
        child {node {child}}
      edge from parent
        node[near end] {x}
    };
\end{tikzpicture}
\end{codeexample}

\end{pathoperation}



%%% Local Variables: 
%%% mode: latex
%%% TeX-master: "pgfmanual-pdftex-version"
%%% End: 

% Copyright 2007 by Till Tantau
%
% This file may be distributed and/or modified
%
% 1. under the LaTeX Project Public License and/or
% 2. under the GNU Free Documentation License.
%
% See the file doc/generic/pgf/licenses/LICENSE for more details.


\section{Plots of Functions}

\label{section-tikz-plots}

\subsection{When Should One Use \tikzname\ for Generating Plots? }

\label{section-why-pgname-for-plots}

There exist many powerful programs that produce plots, examples are
\textsc{gnuplot} or \textsc{mathematica}. These programs can produce
two different kinds of output: First, they can output a complete plot
picture in a certain format (like \pdf) that includes all low-level
commands necessary for drawing the complete plot (including axes and
labels). Second, they can usually also produce ``just plain data'' in
the form of a long list of coordinates. Most of the powerful programs
consider it a to be ``a bit boring'' to just output tabled data and
very much prefer to produce fancy pictures. Nevertheless, when coaxed,
they can also provide the plain data.

\emph{Note that is often not necessary to use \tikzname\ for plots.}
Programs like \textsc{gnuplot} can produce very sophisticated plots
and it is usually much easier to simply include these plots as a
finished \textsc{pdf} or PostScript graphics.

However, there are a number of reasons why you may wish to invest time
and energy into mastering the \pgfname\ commands for creating plots:

\begin{itemize}
\item
  Virtually all plots produced by ``external programs'' use different
  fonts from the one used in your document.
\item
  Even worse, formulas will look totally different, if they can be
  rendered at all.
\item
  Line width will usually be too large or too small.
\item
  Scaling effects upon inclusion can create a mismatch between sizes
  in the plot and sizes in the text.
\item
  The automatic grid generated by most programs is mostly
  distracting. 
\item
  The automatic ticks generated by most programs are cryptic
  numerics. (Try adding a tick reading ``$\pi$'' at the right point.)
\item
  Most programs make it very easy to create ``chart junk'' in a most
  convenient fashion.  All show, no content.
\item
  Arrows and plot marks will almost never match the arrows used in the
  rest of the document.
\end{itemize}

The above list is not exhaustive, unfortunately.


\subsection{The Plot Path Operation}

The |plot| path operation can be used to append a line or curve to the path
that goes through a large number of coordinates. These coordinates are
either given in a simple list of coordinates, read from some file, or
they are computed on the fly.

The syntax of the |plot| comes in different versions.

\begin{pathoperation}{--plot}{\meta{further arguments}}
  This operation plots the curve through the coordinates specified in
  the \meta{further arguments}. The current (sub)path is simply
  continued, that is, a line-to operation to the first point of the
  curve is implicitly added. The details of the \meta{further
    arguments}  will be explained in a moment.
\end{pathoperation}

\begin{pathoperation}{plot}{\meta{further arguments}}
  This operation plots the curve through the coordinates specified in
  the \meta{further arguments} by first ``moving'' to the first
  coordinate of the curve.
\end{pathoperation}

The \meta{further arguments} are used in three different ways to
specifying the coordinates of the points to be plotted:

\begin{enumerate}
\item
  \opt{|--|}|plot|\oarg{local options}\declare{|coordinates{|\meta{coordinate
    1}\meta{coordinate 2}\dots\meta{coordinate $n$}|}|}
\item
  \opt{|--|}|plot|\oarg{local options}\declare{|file{|\meta{filename}|}|}
\item
  \opt{|--|}|plot|\oarg{local options}\declare{\meta{coordinate expression}}
\item
  \opt{|--|}|plot|\oarg{local options}\declare{|function{|\meta{gnuplot formula}|}|}
\end{enumerate}

These different ways are explained in the following.


\subsection{Plotting Points Given Inline}

In the first two cases, the points are given directly in the \TeX-file
as in the following example:

\begin{codeexample}[]
\tikz \draw plot coordinates {(0,0) (1,1) (2,0) (3,1) (2,1) (10:2cm)};
\end{codeexample}

Here is an example showing the difference between |plot| and |--plot|:

\begin{codeexample}[]
\begin{tikzpicture}
  \draw (0,0) -- (1,1) plot coordinates {(2,0)  (4,0)};
  \draw[color=red,xshift=5cm]
        (0,0) -- (1,1) -- plot coordinates {(2,0)  (4,0)};
\end{tikzpicture}
\end{codeexample}


\subsection{Plotting Points Read From an External File}

The second way of specifying points is to put them in an external
file named \meta{filename}. Currently, the only file format that
\tikzname\ allows is the following: Each line of the \meta{filename}
should contain one line starting with two numbers, separated by a
space. Anything following the two numbers on the line is
ignored. Also, lines starting with a |%| or a |#| are ignored as well
as empty lines. (This is exactly the format that \textsc{gnuplot}
produces when you say |set terminal table|.) If necessary, more
formats will be supported in the future, but it is usually easy to
produce a file containing data in this form.

\begin{codeexample}[]
\tikz \draw plot[mark=x,smooth] file {plots/pgfmanual-sine.table};
\end{codeexample}

The file |plots/pgfmanual-sine.table| reads:
\begin{codeexample}[code only]
#Curve 0, 20 points
#x y type
0.00000 0.00000  i
0.52632 0.50235  i
1.05263 0.86873  i
1.57895 0.99997  i
...
9.47368 -0.04889  i
10.00000 -0.54402  i
\end{codeexample}
It was produced from the following source, using |gnuplot|:
\begin{codeexample}[code only]
set terminal table
set output "../plots/pgfmanual-sine.table"
set format "%.5f"
set samples 20
plot [x=0:10] sin(x)
\end{codeexample}

The \meta{local options} of the |plot| operation are local to each
plot and do not affect other plots ``on the same path.'' For example,
|plot[yshift=1cm]| will locally shift the plot 1cm upward. Remember,
however, that most options can only be applied to paths as a
whole. For example, |plot[red]| does not have the effect of making the
plot red. After all, you are trying to ``locally'' make part of the
path red, which is not possible.

\subsection{Plotting a Function}
\label{section-tikz-plot}

When you plot a function, the coordinates of the plot data can be
computed by evaluating a mathematical expression. Since \pgfname\
comes with a mathematical engine, you can specify this expression and
then have \tikzname\ produce the desired coordinates for you,
automatically.

Since this case is quite common when plotting a function, the syntax
is easy: Following the |plot| command and its local options, you
directly provide a \meta{coordinate expression}. It looks like a
normal coordinate, but inside you may use a special macro, which is
|\x| by default, but this can be changed using the |variable|
option. The \meta{coordinate expression} is then evaluated for
different values for |\x| and the resulting coordinates are plotted.

Note that you will often have to put the $x$- or $y$-coordinate inside
braces, namely whenever you use an expression involving a paranthesis.

The following options influence how the \meta{coordinate expression}
is evaluated:
\begin{key}{/tikz/variable=\meta{macro} (initially x)}
  Sets the macro whose value is set to the different values when
  \meta{coordinate expression} is evaluated.
\end{key}

\begin{key}{/tikz/samples=\meta{number} (initially 25)}
  Sets the number of samples used in the plot.
\end{key}

\begin{key}{/tikz/domain=\meta{start}|:|\meta{end} (initially -5:5)}
  Sets the domain between which the samples are taken.
\end{key}

\begin{key}{/tikz/samples at=\meta{sample list}}
  This option specifies a list of positions for which the
  variable should be evaluated. For instance, you can say
  |samples at={1,2,8,9,10}| to have the variable evaluated exactly for
  values $1$, $2$, $8$, $9$, and $10$. You can use the |\foreach|
  syntax, so you can use |...| inside the \meta{sample list}.

  When this option is used, the |samples| and |domain| option are
  overruled. The other ways round, setting either |samples| or
  |domain| will overrule this option.
\end{key}

\begin{codeexample}[]
\begin{tikzpicture}[domain=0:4]
  \draw[very thin,color=gray] (-0.1,-1.1) grid (3.9,3.9);
  
  \draw[->] (-0.2,0) -- (4.2,0) node[right] {$x$};
  \draw[->] (0,-1.2) -- (0,4.2) node[above] {$f(x)$};
  
  \draw[color=red]    plot (\x,\x)             node[right] {$f(x) =x$};
  \draw[color=blue]   plot (\x,{sin(\x r)})    node[right] {$f(x) = \sin x$};
  \draw[color=orange] plot (\x,{0.05*exp(\x)}) node[right] {$f(x) = \frac{1}{20} \mathrm e^x$};
\end{tikzpicture}
\end{codeexample}

\begin{codeexample}[]
\tikz \draw[scale=0.5,domain=-3.141:3.141,smooth,variable=\t]
  plot ({\t*sin(\t r)},{\t*cos(\t r)});
\end{codeexample}

\begin{codeexample}[]
\tikz \draw[domain=0:360,smooth,variable=\t]
  plot ({sin(\t)},\t/360,{cos(\t)});
\end{codeexample}


\subsection{Plotting a Function Using Gnuplot}
\label{section-tikz-gnuplot}

Often, you will want to plot points that are given via a function like
$f(x) = x \sin x$. Unfortunately, \TeX\ does not really have enough
computational power to generate the points on such a function
efficiently (it is a text processing program, after all). However,
if you allow it, \TeX\ can try to call external programs that can
easily produce the necessary points. Currently, \tikzname\ knows how to
call \textsc{gnuplot}.

When \tikzname\ encounters your operation
|plot[id=|\meta{id}|] function{x*sin(x)}| for 
the first time, it will create a file called
\meta{prefix}\meta{id}|.gnuplot|, where \meta{prefix} is |\jobname.| by
default, that is, the name of you main |.tex| file. If no \meta{id} is
given, it will be empty, which is alright, but it is better when each
plot has a unique \meta{id} for reasons explained in a moment. Next,
\tikzname\ writes some initialization code into this file followed by
|plot x*sin(x)|. The initialization code sets up things 
such that the |plot| operation will write the coordinates into another
file called \meta{prefix}\meta{id}|.table|. Finally, this table file
is read as if you had said |plot file{|\meta{prefix}\meta{id}|.table}|. 

For the plotting mechanism to work, two conditions must be met:
\begin{enumerate}
\item
  You must have allowed \TeX\ to call external programs. This is often
  switched off by default since this is a security risk (you might,
  without knowing, run a \TeX\ file that calls all sorts of ``bad''
  commands). To enable this ``calling external programs'' a command
  line option must be given to the \TeX\ program. Usually, it is
  called something like |shell-escape| or |enable-write18|. For
  example, for my |pdflatex| the option |--shell-escape| can be
  given.
\item
  You must have installed the |gnuplot| program and \TeX\ must find it
  when compiling your file.
\end{enumerate}

Unfortunately, these conditions will not always be met. Especially if
you pass some source to a coauthor and the coauthor does not have
\textsc{gnuplot} installed, he or she will have trouble compiling your
files.

For this reason, \tikzname\ behaves differently when you compile your
graphic for the second time: If upon reaching
|plot[id=|\meta{id}|] function{...}| the file \meta{prefix}\meta{id}|.table|
already exists \emph{and} if the \meta{prefix}\meta{id}|.gnuplot| file
contains what \tikzname\ thinks that it ``should'' contain, the |.table|
file is immediately read without trying to call a |gnuplot|
program. This approach has the following advantages: 
\begin{enumerate}
\item
  If you pass a bundle of your |.tex| file and all |.gnuplot| and
  |.table| files to someone else, that person can \TeX\ the |.tex|
  file without having to have |gnuplot| installed.
\item
  If the |\write18| feature is switched off for security reasons (a
  good idea), then, upon the first compilation of the |.tex| file, the
  |.gnuplot| will still be generated, but not the |.table|
  file. You can then simply call |gnuplot| ``by hand'' for each
  |.gnuplot| file, which will produce all necessary |.table| files.
\item
  If you change the function that you wish to plot or its
  domain, \tikzname\ will automatically try to regenerate the |.table|
  file.
\item
  If, out of laziness, you do not provide an |id|, the same |.gnuplot|
  will be used for different plots, but this is not a problem since
  the |.table| will automatically be regenerated for each plot
  on-the-fly. \emph{Note: If you intend to share your files with
  someone else, always use an id, so that the file can by typeset
  without having \textsc{gnuplot} installed.} Also, having unique ids
  for each plot will improve compilation speed since no external
  programs need to be called, unless it is really necessary.
\end{enumerate}

When you use |plot function{|\meta{gnuplot formula}|}|, the \meta{gnuplot
  formula} must be given in the |gnuplot| syntax, whose details are
beyond the scope of this manual. Here is the ultra-condensed
essence: Use |x| as the variable and use the C-syntax for normal
plots, use |t| as the variable for parametric plots. Here are some examples:

\begin{codeexample}[]
\begin{tikzpicture}[domain=0:4]
  \draw[very thin,color=gray] (-0.1,-1.1) grid (3.9,3.9);
  
  \draw[->] (-0.2,0) -- (4.2,0) node[right] {$x$};
  \draw[->] (0,-1.2) -- (0,4.2) node[above] {$f(x)$};
  
  \draw[color=red]    plot[id=x]   function{x}           node[right] {$f(x) =x$};
  \draw[color=blue]   plot[id=sin] function{sin(x)}      node[right] {$f(x) = \sin x$};
  \draw[color=orange] plot[id=exp] function{0.05*exp(x)} node[right] {$f(x) = \frac{1}{20} \mathrm e^x$};
\end{tikzpicture}
\end{codeexample}


The plot in influenced by the following options: First, the options
|samples| and |domain| explained earlier. Second, there are some more
specialized options.

\begin{key}{/tikz/parametric=\meta{boolena} (default true)}
  Sets whether the plot is a parametric plot. If true, then |t| must
  be used instead of |x| as the parameter and two comma-separated
  functions must be given in the \meta{gnuplot formula}. An example is
  the following:
\begin{codeexample}[]
\tikz \draw[scale=0.5,domain=-3.141:3.141,smooth]
  plot[parametric,id=parametric-example] function{t*sin(t),t*cos(t)};
\end{codeexample}
\end{key}

\begin{key}{/tikz/id=\meta{id}}
  Sets the identifier of the current plot. This should be a unique
  identifier for each plot (though things will also work if it is not,
  but not as well, see the explanations above). The \meta{id} will be
  part of a filename, so it should not contain anything fancy like |*|
  or |$|.%$
\end{key}

\begin{key}{/tikz/prefix=\meta{prefix}}
  The \meta{prefix} is put before each plot file name. The default is
  |\jobname.|, but 
  if you have many plots, it might be better to use, say |plots/| and
  have all plots placed in a directory. You have to create the
  directory yourself.
\end{key}

\begin{key}{/tikz/raw gnuplot}
  This key causes the \meta{gnuplot formula} to be passed on to
  \textsc{gnuplot} without setting up the samples or the |plot|
  operation. Thus, you could write
\begin{codeexample}[code only]
plot[raw gnuplot,id=raw-example] function{set samples 25; plot sin(x)}
\end{codeexample}
  This can be 
  useful for complicated things that need to be passed to
  \textsc{gnuplot}. However, for really complicated situations you
  should create a special external generating \textsc{gnuplot} file
  and use the |file|-syntax to include the table ``by hand.''
\end{key}

The following styles influence the plot:
\begin{stylekey}{/tikz/every plot (initially \normalfont empyt)}
  This style is installed in each plot, that is, as if you always said
\begin{codeexample}[code only]
  plot[every plot,...]
\end{codeexample}
 This is most useful for globally setting a prefix for all plots by saying:
\begin{codeexample}[code only]
\tikzset{every plot/.style={prefix=plots/}}
\end{codeexample}
\end{stylekey}



\subsection{Placing Marks on the Plot}

As we saw already, it is possible to add \emph{marks} to a plot using
the |mark| option. When this option is used, a copy of the plot
mark is placed on each point of the plot. Note that the marks are
placed \emph{after} the whole path has been drawn/filled/shaded. In
this respect, they are handled like text nodes. 

In detail, the following options govern how marks are drawn:
\begin{key}{/tikz/mark=\meta{mark mnemonic}}
  Sets the mark to a mnemonic that has previously been defined using
  the |\pgfdeclareplotmark|. By default, |*|, |+|, and |x| are available,
  which draw a filled circle, a plus, and a cross as marks. Many more
  marks become available when the library |pgflibraryplotmarks| is
  loaded. Section~\ref{section-plot-marks} lists the available plot
  marks.

  One plot mark is special: the |ball| plot mark is available only
  it \tikzname. The |ball color| determines the balls's color. Do not use
  this option with a large number of marks since it will take very long
  to render in PostScript.
  
  \begin{tabular}{lc}
    Option & Effect \\\hline \vrule height14pt width0pt
    \plotmarkentrytikz{ball}
  \end{tabular}
\end{key}

\begin{key}{/tikz/mark repeat=\meta{r}}
  This option tells \tikzname\ that only every $r$th mark should be
  drawn.
  
\begin{codeexample}[]
\tikz \draw plot[mark=x,mark repeat=3,smooth] file {plots/pgfmanual-sine.table};
\end{codeexample}
\end{key}

\begin{key}{/tikz/mark phase=\meta{p}}
  This option tells \tikzname\ that the first mark to be draw should
  be the $p$th, followed by the $(p+r)$th, then the $(p+2r)$th, and so
  on.
  
\begin{codeexample}[]
\tikz \draw plot[mark=x,mark repeat=3,mark phase=6,smooth] file {plots/pgfmanual-sine.table};
\end{codeexample}
\end{key}

\begin{key}{/tikz/mark indices=\meta{list}}
  This option allows you to specify explicitly the indices at which a
  mark should be placed. Counting starts with 1. You can use the
  |\foreach| syntax, that is, |...| can be used.
    
\begin{codeexample}[]
\tikz \draw plot[mark=x,mark indices={1,4,...,10,11,12,...,16,20},smooth]
  file {plots/pgfmanual-sine.table};
\end{codeexample}
\end{key}

\begin{key}{/tikz/mark size=\meta{dimension}}
  Sets the size of the plot marks. For circular plot marks,
  \meta{dimension} is the radius, for other plot marks
  \meta{dimension} should be about half the width and height.

  This option is not really necessary, since you achieve the same
  effect by specifying |scale=|\meta{factor} as a local option, where
  \meta{factor} is the quotient of the desired size and the default
  size. However, using |mark size| is a bit faster and more natural. 
\end{key}

\begin{key}{/tikz/mark options=\meta{options}}
  These options are applied to marks when they are drawn. For example,
  you can scale (or otherwise transform) the plot mark or set its
  color. 
\begin{codeexample}[]
\tikz \fill[fill=blue!20]
  plot[mark=triangle*,mark options={color=blue,rotate=180}]
    file{plots/pgfmanual-sine.table} |- (0,0);
\end{codeexample}
\end{key}



\subsection{Smooth Plots, Sharp Plots, Jump Plots, Comb Plots and Bar Plots}

There are different things the |plot| operation can do with the points
it reads from a file or from the inlined list of points. By default,
it will connect these points by straight lines. However, you can also
use options to change the behavior of |plot|.

\begin{key}{/tikz/sharp plot}
  This is the default and causes the points to be connected by
  straight lines. This option is included only so that you can
  ``switch back'' if you ``globally'' install, say, |smooth|.
\end{key}

\begin{key}{/tikz/smooth}
  This option causes the points on the path to be connected using a
  smooth curve:

\begin{codeexample}[]
\tikz\draw plot[smooth] file{plots/pgfmanual-sine.table};
\end{codeexample}

  Note that the smoothing algorithm is not very intelligent. You will
  get the best results if the bending angles are small, that is, less
  than about $30^\circ$ and, even more importantly, if the distances
  between points are about the same all over the plotting path.
\end{key}

\begin{key}{/tikz/tension=\meta{value}}
  This option influences how ``tight'' the smoothing is. A lower value
  will result in sharper corners, a higher value in more ``round''
  curves. A value of $1$ results in a circle if four points at
  quarter-positions on a circle are given. The default is $0.55$. The
  ``correct'' value depends on the details of plot.
  
\begin{codeexample}[]
\begin{tikzpicture}[smooth cycle]
  \draw                 plot[tension=0.2]
    coordinates{(0,0) (1,1) (2,0) (1,-1)};
  \draw[yshift=-2.25cm] plot[tension=0.5]
    coordinates{(0,0) (1,1) (2,0) (1,-1)};
  \draw[yshift=-4.5cm]  plot[tension=1]
    coordinates{(0,0) (1,1) (2,0) (1,-1)};
\end{tikzpicture}
\end{codeexample}
\end{key}

\begin{key}{/tikz/smooth cycle}
  This option causes the points on the path to be connected using a
  closed smooth curve. 

\begin{codeexample}[]
\tikz[scale=0.5]
  \draw plot[smooth cycle] coordinates{(0,0) (1,0) (2,1) (1,2)}
        plot               coordinates{(0,0) (1,0) (2,1) (1,2)} -- cycle;
\end{codeexample}
\end{key}

\begin{key}{/tikz/const plot}
  This option causes the points on the path to be connected using piecewise constant series of lines:

\begin{codeexample}[]
\tikz\draw plot[const plot] file{plots/pgfmanual-sine.table};
\end{codeexample}
\end{key}

\begin{key}{/tikz/const plot mark left}
  Just an alias for |/tikz/const plot|.
\begin{codeexample}[]
\tikz\draw plot[const plot mark left,mark=*] file{plots/pgfmanual-sine.table};
\end{codeexample}
\end{key}

\begin{key}{/tikz/const plot mark right}
  A variant of |/tikz/const plot| which places its mark on the right ends:
\begin{codeexample}[]
\tikz\draw plot[const plot mark right,mark=*] file{plots/pgfmanual-sine.table};
\end{codeexample}
\end{key}


\begin{key}{/tikz/jump mark left}
  This option causes the points on the path to be drawn using piecewise constant, non-connected series of lines. If there are any marks, they will be placed on left open ends:

\begin{codeexample}[]
\tikz\draw plot[jump mark left, mark=*] file{plots/pgfmanual-sine.table};
\end{codeexample}
\end{key}

\begin{key}{/tikz/jump mark right}
  This option causes the points on the path to be drawn using piecewise constant, non-connected series of lines. If there are any marks, they will be placed on right open ends:

\begin{codeexample}[]
\tikz\draw plot[jump mark right, mark=*] file{plots/pgfmanual-sine.table};
\end{codeexample}
\end{key}

\begin{key}{/tikz/ycomb}
  This option causes the |plot| operation to interpret the plotting
  points differently. Instead of connecting them, for each point of
  the plot a straight line is added to the path from the $x$-axis to the point,
  resulting in a sort of ``comb'' or ``bar diagram.''

\begin{codeexample}[]
\tikz\draw[ultra thick] plot[ycomb,thin,mark=*] file{plots/pgfmanual-sine.table};
\end{codeexample}

\begin{codeexample}[]
\begin{tikzpicture}[ycomb]
  \draw[color=red,line width=6pt]
    plot coordinates{(0,1) (.5,1.2) (1,.6) (1.5,.7) (2,.9)};
  \draw[color=red!50,line width=4pt,xshift=3pt]
    plot coordinates{(0,1.2) (.5,1.3) (1,.5) (1.5,.2) (2,.5)};
\end{tikzpicture}
\end{codeexample}
\end{key}


\begin{key}{/tikz/xcomb}
  This option works like |ycomb| except that the bars are horizontal. 

\begin{codeexample}[]
\tikz \draw plot[xcomb,mark=x] coordinates{(1,0) (0.8,0.2) (0.6,0.4) (0.2,1)};
\end{codeexample}
\end{key}


\begin{key}{/tikz/polar comb}
  This option causes a line from the origin to the point to be added
  to the path for each plot point.

\begin{codeexample}[]
\tikz \draw plot[polar comb,
     mark=pentagon*,mark options={fill=white,draw=red},mark size=4pt]
   coordinates {(0:1cm) (30:1.5cm) (160:.5cm) (250:2cm) (-60:.8cm)};
\end{codeexample}
\end{key}

\begin{key}{/tikz/ybar}
  This option produces fillable bar plots. It is thus very similar to |ycomb|, but it employs rectangular shapes instead of line-to operations. It thus allows to use any fill- or pattern style.

\begin{codeexample}[]
\tikz\draw[draw=blue,fill=blue!60!black] plot[ybar] file{plots/pgfmanual-sine.table};
\end{codeexample}

\begin{codeexample}[]
\begin{tikzpicture}[ybar]
  \draw[color=red,fill=red!80,bar width=6pt]
    plot coordinates{(0,1) (.5,1.2) (1,.6) (1.5,.7) (2,.9)};
  \draw[color=red!50,fill=red!20,bar width=4pt,xshift=3pt]
    plot coordinates{(0,1.2) (.5,1.3) (1,.5) (1.5,.2) (2,.5)};
\end{tikzpicture}
\end{codeexample}
\end{key}

\begin{key}{/tikz/xbar}
  This option works like |ybar| except that the bars are horizontal. 

\begin{codeexample}[]
\tikz \draw[pattern=north west lines] plot[xbar] 
   coordinates{(1,0) (0.4,1) (1.7,2) (1.6,3)};
\end{codeexample}
\end{key}

\begin{key}{/tikz/bar width=\meta{dimension} (initially 10pt)}
  Sets the bar width for |xbar| and |ybar|.
\end{key}


\begin{key}{/tikz/only marks}
  This option causes only marks to be shown; no path segments are
  added to the actual path. This can be useful for quickly adding some
  marks to a path.

\begin{codeexample}[]
\tikz \draw (0,0) sin (1,1) cos (2,0)
  plot[only marks,mark=x] coordinates{(0,0) (1,1) (2,0) (3,-1)};
\end{codeexample}
\end{key}


% Copyright 2006 by Till Tantau
%
% This file may be distributed and/or modified
%
% 1. under the LaTeX Project Public License and/or
% 2. under the GNU Free Documentation License.
%
% See the file doc/generic/pgf/licenses/LICENSE for more details.


\section{Transparency}

\label{section-tikz-transparency}


\subsection{Overview}

Normally, when you paint something using any of \tikzname's commands
(this includes stroking, filling, shading, patterns, and images), the
newly painted objects totally obscure whatever was painted earlier in
the same area.

You can change this behaviour by using something that can be thought
of as ``(semi)transparent colors.'' Such colors do not completely
obscure the background, rather they blend the background with the new
color. At first sight, using such semitransparent colors might seem quite
straightforward, but the math going on in the background is quite
involved and the correct handling of transparency fills some 64 pages
in the PDF specification. 

In the present section, we start with the different ways of specifying
``how transparent'' newly drawn objects should be. The simplest way is
to just specify a percentage like ``60\% transparent.'' A much more
general way is to use something that I call a \emph{fading,} also
known as a soft mask or a mask.

At the end of the section we adress the problem of creating so-called
\emph{transparency groups}. This problem arises when you paint over a
position several times with a semitransparent color. Sometimes you
want the effect to accumulate, sometimes you do not.

\emph{Note:} Transparency is best supported by the pdf\TeX\
driver. The \textsc{svg} driver also has some support. For PostScript
output, opacity is rendered correctly only with the most recent
versions of GhostScript. Printers and other programs will typically
ignore the opacity setting. 



\subsection{Specifying a Uniform Opacity}

Specifying a stroke and/or fill opacity is quite easy using the
following options.


\begin{key}{/tikz/draw opacity=\meta{value}}
  This option sets ``how transparent'' lines should be. A value of |1|
  means ``fully opaque'' or ``not transparent at all,'' a value of |0|
  means ``fully transparent'' or ``invisible.'' A value of |0.5|
  yields lines that are semitransparent.

  Note that when you use PostScript as your output format,
  this option works only with recent versions of GhostScript.
   
\begin{codeexample}[]
\begin{tikzpicture}[line width=1ex]
  \draw (0,0) -- (3,1);
  \filldraw [fill=examplefill,draw opacity=0.5] (1,0) rectangle (2,1);
\end{tikzpicture}
\end{codeexample}
\end{key}

Note that the |draw opacity| options only sets the opacity of drawn
lines. The opacity of fillings is set using the option
|fill opacity| (documented in Section~\ref{section-fill-opacity}. The
option |opacity| sets both at the same time. 

\begin{key}{/tikz/opacity=\meta{value}}
  Sets both the drawing and filling opacity to \meta{value}.

  The following predefined styles make it easier to use this option:
  \begin{stylekey}{/tikz/transparent}
    Makes everything totally transparent and, hence, invisible.

\begin{codeexample}[]
\tikz{\fill[red]             (0,0)   rectangle (1,0.5);
      \fill[transparent,red] (0.5,0) rectangle (1.5,0.25); }
\end{codeexample}
  \end{stylekey}

  \begin{stylekey}{/tikz/ultra nearly transparent}
    Makes everything, well, ultra nearly transparent.

\begin{codeexample}[]
\tikz{\fill[red]                      (0,0)   rectangle (1,0.5);
      \fill[ultra nearly transparent] (0.5,0) rectangle (1.5,0.25); }
\end{codeexample}
  \end{stylekey}

  \begin{stylekey}{/tikz/very nearly transparent}
\begin{codeexample}[]
\tikz{\fill[red]                     (0,0)   rectangle (1,0.5);
      \fill[very nearly transparent] (0.5,0) rectangle (1.5,0.25); }
\end{codeexample}
  \end{stylekey}

  \begin{stylekey}{/tikz/nearly transparent}
\begin{codeexample}[]
\tikz{\fill[red]                (0,0)   rectangle (1,0.5);
      \fill[nearly transparent] (0.5,0) rectangle (1.5,0.25); }
\end{codeexample}
  \end{stylekey}

  \begin{stylekey}{/tikz/semitransparent} 
\begin{codeexample}[]
\tikz{\fill[red]             (0,0)   rectangle (1,0.5);
      \fill[semitransparent] (0.5,0) rectangle (1.5,0.25); }
\end{codeexample}
  \end{stylekey}

  \begin{stylekey}{/tikz/nearly opaque}   
\begin{codeexample}[]
\tikz{\fill[red]           (0,0)   rectangle (1,0.5);
      \fill[nearly opaque] (0.5,0) rectangle (1.5,0.25); }
\end{codeexample}
  \end{stylekey}
 
  \begin{stylekey}{/tikz/very nearly opaque} 
\begin{codeexample}[]
\tikz{\fill[red]                (0,0)   rectangle (1,0.5);
      \fill[very nearly opaque] (0.5,0) rectangle (1.5,0.25); }
\end{codeexample}
  \end{stylekey}

  \begin{stylekey}{/tikz/ultra nearly opaque}
\begin{codeexample}[]
\tikz{\fill[red]                 (0,0)   rectangle (1,0.5);
      \fill[ultra nearly opaque] (0.5,0) rectangle (1.5,0.25); }
\end{codeexample}
  \end{stylekey}

  \begin{stylekey}{/tikz/opaque}
    This yields completely opaque drawings, which is the default.
\begin{codeexample}[]
\tikz{\fill[red]    (0,0)   rectangle (1,0.5);
      \fill[opaque] (0.5,0) rectangle (1.5,0.25); }
\end{codeexample}
  \end{stylekey}
\end{key}


\begin{key}{/tikz/fill opacity=\meta{value}}
  This option sets the opacity of fillings. In addition to filling
  operations, this opacity also applies to text and images.

  Note, again, that when you use PostScript as your output format,
  this option works only with recent versions of GhostScript.
  
\begin{codeexample}[]
\begin{tikzpicture}[thick,fill opacity=0.5]
  \filldraw[fill=red]   (0:1cm)    circle (12mm);
  \filldraw[fill=green] (120:1cm)  circle (12mm);
  \filldraw[fill=blue]  (-120:1cm) circle (12mm);
\end{tikzpicture}
\end{codeexample}

\begin{codeexample}[]
\begin{tikzpicture}
  \fill[red] (0,0) rectangle (3,2);

  \node                   at (0,0) {\huge A};
  \node[fill opacity=0.5] at (3,2) {\huge B};
\end{tikzpicture}
\end{codeexample}
\end{key}

\begin{key}{/tikz/text opacity=\meta{value}}
  Sets the opacity of text labels, overriding the |fill opacity| setting. 
\begin{codeexample}[]
\begin{tikzpicture}[every node/.style={fill,draw}]
  \draw[line width=2mm,blue!50,cap=round] (0,0) grid (3,2);

  \node[opacity=0.5] at (1.5,2) {Upper node};
  \node[draw opacity=0.8,fill opacity=0.2,text opacity=1]
    at (1.5,0) {Lower node};
\end{tikzpicture}
\end{codeexample}
\end{key}


Note the following effect: If you setup a certain opacity for stroking
or filling and you stroke or fill the same area twice, the effect
accumulates:

\begin{codeexample}[]
\begin{tikzpicture}[fill opacity=0.5]
  \fill[red] (0,0) circle (1);
  \fill[red] (1,0) circle (1);
\end{tikzpicture}
\end{codeexample}

Often, this is exactly what you intend, but not always. You can use
transparency groups, see the end of this section, to change this.


\subsection{Fadings}

For complicated graphics, uniform transparency settings are not always
sufficient. Suppose, for instance, that while you paint a picture, you
want the transparency to vary smoothly from completely opaque to
completely transparent. This is a ``shading-like'' transparency. For
such a form of transparency I will use the term \emph{fading} (as a
noun). They are also known as \emph{soft masks}, \emph{opacity masks},
\emph{masks}, or \emph{soft clips}.


\subsubsection{Creating Fadings}

How do we specify a fading? This is a bit of an art since the
underlying mechanism is quite powerful, but a bit difficult to use.

Let us start with a bit of terminology. A \emph{fading} specifies for
each point of an area to transparency of the point. This transparency
can by any number between 0 and 1. A \emph{fading picture} is a normal
graphic that, in a way to be described in a moment, determines the
transparency of points inside the fading. Each fading has an
underlying fading picture. 

The fading picture is a normal graphic drawn using any of the normal
graphic drawing commands. A fading and its fading picture are related
as follows: Given any point of the fading, the transparency of this
point is determined by the lumonisity of the fading picture at the
same position. The luminosity of a point determines ``how bright'' the
point is. The brighter the point in the fading picture, the more
opaque is the point in the fading. In particular, a white point of the
fading picture is completely opaque in the fading and a black point of
the fading picture is completely transparent in the fading. (The
background of the fading picture is always transparent in the fading
as if the background where black.)

It is rather counter-intuitive that a \emph{white} pixel of the fading
picture will be \emph{opaque} in the fading and a \emph{black} pixel
will be \emph{transparent}. For this reason, \tikzname\ defines a
color called |transparent| that is the same as |black|. The nice thing
about this definition is that the color
|transparent!|\meta{percentage} in the fading picture yields a
pixel that is \meta{percantage} per cent transparent in the fading. 

Turning a fading picture into a normal picture is achieved using the
following commands, which are \emph{only defined in the library},
namely the library |fadings|. So, to use them, you have to say
|\usetikzlibrary{fadings}| first.

\begin{environment}{{tikzfadingfrompicture}\oarg{options}}
  This command works like a |{tikzpicture}|, only the picture is not
  shown, but instead a fading is defined based on this picture. To set
  the name of the picture, use the |name| option (which is normally
  used to set the name of a node).
  \begin{key}{/tikz/name=\marg{name}}
    Use this option with the |{tikzfadingfrompicture}| environment to
    set the name of the fading. You \emph{must} provide this option.
  \end{key}
  
  The following shading is 2cm by 2cm and changes gets more and more
  transparent from left to right, but is 50\% transparent for a large
  circle in the middle.
\begin{codeexample}[]
\begin{tikzfadingfrompicture}[name=fade right]
  \shade[left color=transparent!0,
         right color=transparent!100] (0,0) rectangle (2,2);
  \fill[transparent!50] (1,1) circle (0.7);
\end{tikzfadingfrompicture}

% Now we use the fading in another picture:
\begin{tikzpicture}
  % Background
  \fill [black!20] (-1.2,-1.2) rectangle (1.2,1.2);
  \pattern [pattern=checkerboard,pattern color=black!30]
                   (-1.2,-1.2) rectangle (1.2,1.2);
  
  \fill [path fading=fade right,red] (-1,-1) rectangle (1,1);
\end{tikzpicture}
\end{codeexample}
  In the next example we create a fading picture that contains some
  text. When the fading is used, we only see the shading ``through
  it.'' 
\begin{codeexample}[]
\begin{tikzfadingfrompicture}[name=tikz]
  \node [text=transparent!20]
    {\fontfamily{ptm}\fontsize{45}{45}\bfseries\selectfont Ti\emph{k}Z};
\end{tikzfadingfrompicture}

% Now we use the fading in another picture:
\begin{tikzpicture}
  \fill [black!20] (-2,-1) rectangle (2,1);
  \pattern [pattern=checkerboard,pattern color=black!30]
                   (-2,-1) rectangle (2,1);

  \shade[path fading=tikz,fit fading=false,
         left color=blue,right color=black]
    (-2,-1) rectangle (2,1);
\end{tikzpicture}
\end{codeexample}
\end{environment}

\begin{plainenvironment}{{tikzfadingfrompicture}\oarg{options}}
  The plain\TeX\ version of the environment.
\end{plainenvironment}

\begin{contextenvironment}{{tikzfadingfrompicture}\oarg{options}}
  The Con\TeX t version of the environment.
\end{contextenvironment}

\begin{command}{\tikzfading\oarg{options}}
  This command is used to define a fading similarly to that way a
  shading is defined. In the \meta{options} you should 
  \begin{enumerate}
  \item use the |name=|\meta{name} option to set a name for the fading,
  \item use the |shading| option to set the name of the shading that
    you wish to use,
  \item extra options for setting the colors of the shading (typically
    you will set them to the color |transparent!|\meta{percentage}).
  \end{enumerate}
  Then, a new fading named \meta{name} will be created based on the
  shading.

\begin{codeexample}[]
\tikzfading[name=fade right,
            left color=transparent!0,
            right color=transparent!100]

% Now we use the fading in another picture:
\begin{tikzpicture}
  % Background
  \fill [black!20] (-1.2,-1.2) rectangle (1.2,1.2);
  \path [pattern=checkerboard,pattern color=black!30]
                   (-1.2,-1.2) rectangle (1.2,1.2);

  \fill [red,path fading=fade right] (-1,-1) rectangle (1,1);
\end{tikzpicture}
\end{codeexample}  

\begin{codeexample}[]
\tikzfading[name=fade out,
            inner color=transparent!0,
            outer color=transparent!100]

% Now we use the fading in another picture:
\begin{tikzpicture}
  % Background
  \fill [black!20] (-1.2,-1.2) rectangle (1.2,1.2);
  \path [pattern=checkerboard,pattern color=black!30]
                   (-1.2,-1.2) rectangle (1.2,1.2);

  \fill [blue,path fading=fade out] (-1,-1) rectangle (1,1);
\end{tikzpicture}
\end{codeexample}  
\end{command}



\subsubsection{Fading a Path}

Aa fading specifies for each pixel of a certain area how transparent
this pixel will be. The following options are used to install such a
fading for the current scope or path. 

\pgfdeclarefading{fade down}{%
  \tikzset{top color=pgftransparent!0,bottom color=pgftransparent!100}
  \pgfuseshading{axis}
}
\pgfdeclarefading{fade inside}{%
  \tikzset{inner color=pgftransparent!90,outer color=pgftransparent!30}
  \pgfuseshading{radial}
}

\begin{key}{/tikz/path fading=\meta{name} (default \normalfont scope's setting)}
  This option tells \tikzname\ that the current path should be faded
  with the fading \meta{name}. If no \meta{name} is given, the 
  \meta{name} set for the whole scope is used. Similarly to options
  like |draw| or |fill|, this option is reset for each path, so you
  have to add it to each path that should be faded. You can also
  specify |none| as \meta{name}, in which case fading for the path
  will be switched off in case it has been switched on by previous
  options or styles.
\begin{codeexample}[]
\begin{tikzpicture}[path fading=south]
  % Checker board
  \fill [black!20] (0,0) rectangle (4,3);
  \pattern [pattern=checkerboard,pattern color=black!30]
                   (0,0) rectangle (4,3);

  \fill [color=blue]                   (0.5,1.5) rectangle +(1,1);
  \fill [color=blue,path fading=north] (2.5,1.5) rectangle +(1,1);

  \fill [color=red,path fading]        (1,0.75) ellipse (.75 and .5);
  \fill [color=red]                    (3,0.75) ellipse (.75 and .5);
\end{tikzpicture}
\end{codeexample}

  \begin{key}{/tikz/fit fading=\meta{boolean} (default true, initially true)}
    When set to |true|, the fading is shifted and resized (in exactly
    the same way as a shading) so that is covers the current
    path. When set to |false|, the fading is only shifted so that it
    is centered on the path's center, but it is not resized. This can
    be useful for special-purpose fadings, for instance when you use a
    fading to ``punsh out'' something.
  \end{key}

  \begin{key}{/tikz/fading transform=\meta{transformation options}}
    The \meta{transformation options} are applied to the fading before
    it is used. For instance, if \meta{transformation options} is set
    to |rotate=90|, the fading is rotated by 90 degrees.
\begin{codeexample}[]
\begin{tikzpicture}[path fading=fade down]
  % Checker board
  \fill [black!20] (0,0) rectangle (4,1.5);
  \path [pattern=checkerboard,pattern color=black!30] (0,0) rectangle (4,1.5);

  \fill [red,path fading,fading transform={rotate=90}]
    (1,0.75) ellipse (.75 and .5);
  \fill [red,path fading,fading transform={rotate=30}]
    (3,0.75) ellipse (.75 and .5);
\end{tikzpicture}
\end{codeexample}
  \end{key}
  
  \begin{key}{/tikz/fading angle=\meta{degree}}
    A shortcut for |fading transform={rotate=|\meta{degree}|}|.
  \end{key}

  Note that you can ``fade just about anything.'' In particular, you
  can fade a shading.
  
\begin{codeexample}[]
\begin{tikzpicture}
  % Checker board
  \fill [black!20] (0,0) rectangle (4,4);
  \path [pattern=checkerboard,pattern color=black!30] (0,0) rectangle (4,4);

  \shade [ball color=blue,path fading=south] (2,2) circle (1.8);
\end{tikzpicture}
\end{codeexample}

  The |fade inside| of the following example more transparent in the middle than on the
  outside.

\begin{codeexample}[]
\tikzfading[name=fade inside,
            inner color=transparent!80,
            outer color=transparent!30]    
\begin{tikzpicture}
  % Checker board
  \fill [black!20] (0,0) rectangle (4,4);
  \path [pattern=checkerboard,pattern color=black!30] (0,0) rectangle (4,4);

  \shade [ball color=red] (3,3) circle (0.8);
  \shade [ball color=white,path fading=fade inside] (2,2) circle (1.8);
\end{tikzpicture}
\end{codeexample}

  Note that adding the |path fading| option to a node fades the
  (background) path, not the text itself. To fade the text, you need
  to use a scope fading (see below).
\end{key}

Note that using fadings in conjunction with patterns can create
visually rather pleasing effects:
\begin{codeexample}[]
\tikzfading[name=middle,
            top color=transparent!50,
            bottom color=transparent!50,
            middle color=transparent!20]
\begin{tikzpicture}
  \node      [circle with shadow,
              pattern=horizontal lines dark blue,
              path fading=south,
              minimum size=3.6cm] {};
  \pattern   [path fading=north,
              pattern=horizontal lines dark gray]
    (0,0) circle (1.8cm);
  \pattern   [path fading=middle,
              pattern=crosshatch dots light steel blue]
    (0,0) circle (1.8cm);
\end{tikzpicture}
\end{codeexample}


\subsubsection{Fading a Scope}

In addtion to fading individual paths, you may also wish to ``fade a
scope,'' that is, you may wish to install a fading that is used
globally to specify the transparency for all objects drawn inside a
scope. This effect can also be thought of as a ``soft clip'' and it
works in a similar way: You add the |scope fading| option to a path in
a scope -- typically the first one -- and then all subsequent drawings
in the scope are faded. You will use a |transparency group| in
conjunction, see the end of this section.

\begin{key}{/tikz/scope fading=\meta{fading}}
  In principle, this key works in excatly the same way as the
  |path fading| key. The only difference is, that the effect of the
  fading will persist after the current path till the end of the
  scope. Thus, the \meta{fading} is applied to all subsequent drawings
  in the current scope, not just to the current path. In this regard,
  the option works very much like the |clip| option. (Note, however,
  that, unlike the |clip| option, fadings to not accumulate unless a
  transparency group is used.)

  The keys |fit fading| and |fading transform| have the same effect as
  for |path fading|. Also that, just as for |path fading|, providing
  the |scope fading| option with a |{scope}| only sets the name of the
  fading to be used. You have to explicitly provide the |scope fading|
  with a path to actually install a fading.
  
\begin{codeexample}[]
\begin{tikzpicture}
  \fill [black!20] (-2,-2) rectangle (2,2);
  \pattern [pattern=checkerboard,pattern color=black!30]
                   (-2,-2) rectangle (2,2);

  % The bounding box of the shading:
  \draw [red] (-50bp,-50bp) rectangle (50bp,50bp);

  \path [scope fading=south,fit fading=false] (0,0);
  % fading is centered at its natural size

  \fill[red]   ( 90:1) circle (1);
  \fill[green] (210:1) circle (1);
  \fill[blue]  (330:1) circle (1);
\end{tikzpicture}
\end{codeexample}

  In the following example we resize the fading to the size of the
  whole picture:
\begin{codeexample}[]
\begin{tikzpicture}
  \fill [black!20] (-2,-2) rectangle (2,2);
  \pattern [pattern=checkerboard,pattern color=black!30]
                   (-2,-2) rectangle (2,2);

  \path [scope fading=south] (-2,-2) rectangle (2,2);

  \fill[red]   ( 90:1) circle (1);
  \fill[green] (210:1) circle (1);
  \fill[blue]  (330:1) circle (1);
\end{tikzpicture}
\end{codeexample}

  Scope fadings are also needed if you wish to fade a node.
\begin{codeexample}[]
\tikz \node [scope fading=south,fading angle=45,text width=3.5cm]
{
  This is some text that will fade out as we go right
  and down. It is pretty hard to achieve this effect in
  other ways.
};    
\end{codeexample}

\end{key}


\subsection{Transparency Groups}

Consider the following cross and sign. They ``look wrong'' because we
can see how they were constructed, while this is not really part of
the desired effect. 

\begin{codeexample}[]
\begin{tikzpicture}[opacity=.5]
  \draw [line width=5mm] (0,0) -- (2,2);
  \draw [line width=5mm] (2,0) -- (0,2);
\end{tikzpicture}
\end{codeexample}

\begin{codeexample}[]
\begin{tikzpicture}
  \node at (0,0) [forbidden sign,line width=2ex,draw=red,fill=white] {Smoking};

  \node [opacity=.5]
        at (2,0) [forbidden sign,line width=2ex,draw=red,fill=white] {Smoking};
\end{tikzpicture}
\end{codeexample}

Transparency groups are used to render them correctly:

\begin{codeexample}[]
\begin{tikzpicture}[opacity=.5]
  \begin{scope}[transparency group]
    \draw [line width=5mm] (0,0) -- (2,2);
    \draw [line width=5mm] (2,0) -- (0,2);
  \end{scope}
\end{tikzpicture}
\end{codeexample}

\begin{codeexample}[]
\begin{tikzpicture}
  \node at (0,0) [forbidden sign,line width=2ex,draw=red,fill=white] {Smoking};

  \begin{scope}[opacity=.5,transparency group]
    \node at (2,0) [forbidden sign,line width=2ex,draw=red,fill=white]
      {Smoking};
  \end{scope}
\end{tikzpicture}
\end{codeexample}

\begin{key}{/tikz/transparency group}
  This option can be given to a |scope|. It will have the following
  effect: The scope's contents is stroked/filled
  ``ignoring any outside transparency.'' This means, all previous
  transparency settings are ignored (you can still set transparency
  inside the group, but never mind). For instance, in the forbidden
  sign example, the whole sign is first painted (conceptually) like
  the image on the left hand side. Note that some pixels of the sign
  are painted multiple times (up to three times), but only the last
  color ``wins.''

  Then, when the scope is finished, it is painted as a whole. The  
  \emph{fill} transparency settings are now applied to the resulting
  picutre. For instance, the pixel that has been painted three times
  is just red at the end, so this red color will be blended with
  whatever is ``behind'' the group on the page.

  Note that, depending on the driver, it is possible to directly put
  objects in a transparency group that lie outside the picture. This
  has to do with internal bounding box computations.
  Section~\ref{section-transparency} explains how to sidestep this
  problem.   
\end{key}



%%% Local Variables: 
%%% mode: latex
%%% TeX-master: "pgfmanual"
%%% End: 

% Copyright 2008 by Mark Wibrow
%
% This file may be distributed and/or modified
%
% 1. under the LaTeX Project Public License and/or
% 2. under the GNU Free Documentation License.
%
% See the file doc/generic/pgf/licenses/LICENSE for more details.

\section{Decorated Paths}

\label{section-tikz-decorations}


\subsection{Overview}

Decorations are a general concept to make (sub)paths ``more
interesting.'' Before we have a look at the details, let us have a
look at some examples:

\begin{codeexample}[]
\begin{tikzpicture}[thick]
  \draw                                                (0,3)   -- (3,3);
  \draw[decorate,decoration=zigzag]                    (0,2.5) -- (3,2.5);
  \draw[decorate,decoration=brace]                     (0,2)   -- (3,2);
  \draw[decorate,decoration=triangles]                 (0,1.5) -- (3,1.5);
  \draw[decorate,decoration={coil,segment length=4pt}] (0,1)   -- (3,1);
  \draw[decorate,decoration={coil,aspect=0}]           (0,.5)  -- (3,.5);
  \draw[decorate,decoration={expanding waves,angle=7}] (0,0)   -- (3,0);
\end{tikzpicture}
\end{codeexample}

\begin{codeexample}[]
\begin{tikzpicture}
  \node [fill=red!20,draw,decorate,decoration={bumps,mirror},
         minimum height=1cm]
  {Bumpy};
\end{tikzpicture}
\end{codeexample}

\begin{codeexample}[]
\begin{tikzpicture}
  \filldraw[fill=blue!20]                    (0,3)
  decorate [decoration=saw]             { -- (3,3) }
  decorate [decoration={coil,aspect=0}] { -- (2,1) }
  decorate [decoration=bumps]           { -| (0,3) }; 
\end{tikzpicture}
\end{codeexample}

\begin{codeexample}[]
\begin{tikzpicture}
  \node [fill=yellow!50,draw,thick, minimum height=2cm, minimum width=3cm,
         decorate, decoration={random steps,segment length=3pt,amplitude=1pt}]
    {Saved from trash};
\end{tikzpicture}
\end{codeexample}

The general idea of decorations is the following: First, you construct
a path using the usual path construction commands. The resulting path
is, in essence, a series of straight and curved lines. Instead of
directly using this path for filling or drawing, you can then specify
that it should form the basis for a decoration. In this case,
depending on which decoration you use, a new path is constructed
``along'' the path you specified. For instance, with the |zigzag|
decoration, the new path is a zigzagging line that goes along the old
path.

Let us have a look at an example: In the first picture, we see a path
that consists of a line, an arc, and a line. In the second picture,
this path has been used as the basis of a decoration.

\begin{codeexample}[]
\tikz \fill
  [fill=blue!20,draw=blue,thick] (0,0) -- (2,1) arc (90:-90:.5) -- cycle;
\end{codeexample}
\begin{codeexample}[]
\tikz \fill [decorate,decoration={zigzag}]
  [fill=blue!20,draw=blue,thick] (0,0) -- (2,1) arc (90:-90:.5) -- cycle;
\end{codeexample}

It is also possible to decorate only a subpath (the exact syntax will
be explained later in this section).
\begin{codeexample}[]
\tikz \fill [decoration={zigzag}]
  [fill=blue!20,draw=blue,thick] (0,0) -- (2,1)
    decorate { arc (90:-90:.5) } -- cycle;
\end{codeexample}

The |zigzag| decoration will be called a \emph{path
  morphing} decoration because it morphs a path into a different, but
topologically equivalent path. Not all decorations are path
morphing; rather there are three kinds of decorations.


\begin{enumerate}
\item The just-mentioned \emph{path morphing} decorations morph the
  path in the sense that what used to be a straight  line might
  afterwards be a squiggly line or might have bumps. However, a line
  is still and a line and path deforming decorations do not change
  the number of subpaths. 

  Examples of such decorations are the |snake| or the |zigzag|
  decoration. Many such decorations are defined in the library
  |decorations.pathmorphing|.
  
\item \emph{Path replacing} decorations completely replace the
  path by a different path that is only ``loosely based'' on the
  original path. For instance, the |crosses| decoration replaces a path
  by a path consisting of a sequence of crosses. Note how in the
  following example filling the path has no effect since the path
  consist only of (numerous) unconnected straight line subpaths:
\begin{codeexample}[]
\tikz \fill [decorate,decoration={crosses}]
  [fill=blue!20,draw=blue,thick] (0,0) -- (2,1) arc (90:-90:.5) -- cycle;
\end{codeexample}

  Examples of path replacing decorations are |crosses| or |ticks| or
  |shape backgrounds|. Such decorations are defined in the library
  |decorations.pathreplacing|, but also in |decorations.shapes|.
  
\item \emph{Path removing} decorations completely remove the
  to-be-decorated path. Thus, they have no effect on the main path
  that is being constructed. Instead, they typically have numerous
  \emph{side  effects}. For instance, they might ``write some text''
  along the (removed) path or they might place nodes along this
  path. Note that for such decorations the path usage command for the
  main path have no influence on how the decoration looks like.

\begin{codeexample}[]
\tikz \fill [decorate,decoration={text along path,
               text=This is a text along a path. Note how the path is lost.}]
  [fill=blue!20,draw=blue,thick] (0,0) -- (2,1) arc (90:-90:.5) -- cycle;
\end{codeexample}
\end{enumerate}

Decorations are defined in different decoration libraries, see
Section~\ref{section-library-decorations} for details. It is also
possible to define your own decorations, see
Section~\ref{section-base-decorations}, but you need to use the
\pgfname\ basic layer and a bit of theory is involved.

Decorations can be used to decorate already decorated paths. In the
following three graphics, we start with a simple path, then decorate
it once, and then decorate the decorated path once more.

\begin{codeexample}[]
\tikz \fill [fill=blue!20,draw=blue,thick]
  (0,0) rectangle (3,2);
\end{codeexample}
\begin{codeexample}[]
\tikz \fill [fill=blue!20,draw=blue,thick]
  decorate[decoration={zigzag,segment length=10mm,amplitude=2.5mm}]
    { (0,0) rectangle (3,2) };
\end{codeexample}
\begin{codeexample}[]
\tikz \fill [fill=blue!20,draw=blue,thick]
  decorate[decoration={crosses,segment length=2mm}] {
    decorate[decoration={zigzag,segment length=10mm,amplitude=2.5mm}] {
      (0,0) rectangle (3,2) 
    }
  };
\end{codeexample}

One final word of warning: Decorations can be pretty slow to
typeset and they can be inaccurate. The reason is that \pgfname\ has
to a \emph{lot} of rather difficult computations in the background and
\TeX\ is not very good at doing math. Decorations are fastest when
applied to straight line segments, but even then they are much slower
than other alternative. For instance, the |ticks| decoration can be
simulated by clever use of a dashing pattern and the dashing pattern
will literally be thousands of times faster to typeset. However, for
most decorations there are no real alternatives.

\begin{tikzlibrary}{decorations}
  In order to use decorations, you first have to load a decoration
  library. This |decoration| library defines the basic options
  described in the following, but it does not define any new
  decorations. This is done by libraries like
  |decorations.text|. Since these more specialized libraries include
  the |decoration| library automatically, you usually do not have to
  bother about it.
\end{tikzlibrary}



\subsection{Decorating a Subpath Using the Decorate Path Command}

The most general way to decorate a (sub)path is the following path
command.

\begin{pathoperation}{decorate}{\opt{\oarg{options}}\marg{subpath}}
  This path operation causes the \meta{subpath} to be
  decorated using the current decoration. Depending on the decoration,
  this may or may not extend the current path.
\begin{codeexample}[]
\begin{tikzpicture}
  \draw [help lines] grid (3,2);
  \draw decorate [decoration={name=zigzag}]
         { (0,0) .. controls (0,2) and (3,0) .. (3,2) |- (0,0) };
\end{tikzpicture}
\end{codeexample}
  The path can include straight lines, curves,
  rectangles, arcs, circles, ellipses, and even already decorated
  paths (that is, you can nest applications of the |decorate| path
  command, see below).

  Closed subpaths (like  rectangles or circles) may not be decorated
  succesfully with ``continuous'' decorations (those that do not
  create multiple segmented subpaths). In addition, due to the limits
  on the precision in  \TeX, some inaccuraces in positioning when
  crossing subpath boundaries may occasionally be found.

  You can use nodes normally inside the \meta{subpath}.
\begin{codeexample}[]
\begin{tikzpicture}
  \draw [help lines] grid (3,2);
  \draw decorate [decoration={name=zigzag}]
    { (0,0) -- (2,2) node (hi) [left,draw=red] {Hi!} arc(90:0:1)};

  \draw [blue] decorate [decoration={crosses}] {(3,0) -- (hi)};
\end{tikzpicture}
\end{codeexample}
  
  The following key is used to select the decoration and also to
  select further ``rendering options'' for the decoration.

  \begin{key}{/pgf/decoration=\meta{decoration options}}
    \keyalias{tikz}
    This option is used to specify which decoration is used and how it
    will look like. Note that his key will \emph{not} cause any
    decorations to be applied, immediately. It takes the |decorate| path
    command or the |decorate| option to actually decorate a path. The
    |decoration| option is only used to specify which decoration should
    be used, in principle. You can also use this option at the
    beginning of a picture or a scope to specify the decoration to be
    used with each invocation of the |decorate| path
    command. Naturally, any local options of the |decorate| path
    command override these ``global'' options.
\begin{codeexample}[]
\begin{tikzpicture}[decoration=zigzag]
  \draw       decorate                      {(0,0) -- (3,2)};
  \draw [red] decorate [decoration=crosses] {(0,2) -- (3,0)};
\end{tikzpicture}
\end{codeexample}
    
    The \meta{decoration options} are special options
    (which have the path prefix |/pgf/decoration/|) that determine the
    properties of the decoration. Which options are appropriate for a
    decoration depend strongly on the decoration, you will have to look
    up the appropriate options in the documentation of the decoration,
    see Section~\ref{section-library-decorations}.
    
    There is one option (available only in \tikzname) that is special:
    \begin{key}{/pgf/decoration/name=\meta{name} (initially none)}
      Use this key to set which decoration is to be used. The
      \meta{name} can both be a decoration or a meta-decoration (you
      need to worry about the difference only if you wish to define
      your own decorations).
      
      If you set \meta{name} to |none|, no decorations are added.
\begin{codeexample}[]
\begin{tikzpicture}
  \draw [help lines] grid (3,2);
  \draw decorate [decoration={name=zigzag}]
         { (0,0) .. controls (0,2) and (3,0) .. (3,2) };
\end{tikzpicture}
\end{codeexample}
      Since this option is used so often, you can also leave out the
      |name=| part. Thus, the above example can be rewritten more
      succinctly: 
\begin{codeexample}[]
\begin{tikzpicture}
  \draw [help lines] grid (3,2);
  \draw decorate [decoration=zigzag]
         { (0,0) .. controls (0,2) and (3,0) .. (3,2) };
\end{tikzpicture}
\end{codeexample}
      In general, when \meta{decoration options} are parsed, for each
      unknown key it is checked whether that key happens to be a
      (meta-)decoration and, if so, the |name| option is executed for
      this key.
    \end{key}

    Further options allow you to adjust the position of decorations
    relative to the to-be-decorated path. See
    Section~\ref{section-decorations-adjust} below for details.
  \end{key}

  Recall that some decorations actually completely remove the
  to-be-decorated path. In such cases, the construction of the main
  path is resumed after the |decorate| path command ends.
  
\begin{codeexample}[]
\begin{tikzpicture}[decoration={text along path,text=
      around and around and around and around we go}]

  \draw (0,0) -- (1,1) decorate { -- (2,1) } -- (3,0);
\end{tikzpicture}
\end{codeexample}

  It is permissible to nest |decorate| commands. In this case, the
  path resulting from the first decoration process is used as the
  to-be-decorated path for the second decoration process. This is
  especially useful for drawing fractals. The |Koch snowflake|
  decoration replaces a straight line like \tikz\draw (0,0) -- (1,0);
  by \tikz[decoration=Koch snowflake] \draw decorate{(0,0) --
    (1,0)};. Repeatedly applying this transformation to a triangle
  yields a fractal that looks a bit like a snowflake, hence the name. 
\begin{codeexample}[]
\begin{tikzpicture}[decoration=Koch snowflake,draw=blue,fill=blue!20,thick]
  \filldraw (0,0) -- ++(60:1) -- ++(-60:1) -- cycle ;
  \filldraw decorate{ (0,-1) -- ++(60:1) -- ++(-60:1) -- cycle };
  \filldraw decorate{ decorate{ (0,-2.5) -- ++(60:1) -- ++(-60:1) -- cycle }};
\end{tikzpicture}
\end{codeexample}
\end{pathoperation}



\subsection{Decorating a Complete Path}

You may sometimes wish to decorate a path over whose construction you
have no control. For instance, the path of the background of a node is
created without your having a chance to issue a |decorate| path
command. In such cases you can use the following option, which allows
you to decorate a path ``after the fact.''

\begin{key}{/tikz/decorate=\opt{\meta{boolean}} (default true)}
  When this key is set, the whole path is decorated after it has been
  finished. The decoration used for decorating the path is set via the
  |decoration| way, in exactly the same way as for the |decorate| path
  command. Indeed, the following two commands have the same effect:
  \begin{enumerate}
  \item |\path decorate[|\meta{options}|] {|\meta{path}|};|
  \item |\path [decorate,|\meta{options}|] |\meta{path}|;|
  \end{enumerate}
  The main use or the |decorate| option is the you can also use it
  with the nodes. It then causes the background path of the node to be
  decorated. Note that you decorate a background path only once in
  this manner. That is, in contrast to the |decorate| path command you
  cannot apply this option twice (this would just set it to |true|,
  once more).

\begin{codeexample}[]
\begin{tikzpicture}[decoration=zigzag]
  \draw [help lines] (0,0) grid (3,5);
  
  \draw [fill=blue!20,decorate] (1.5,4) circle (1cm);

  \node at (1.5,2.5) [fill=red!20,decorate,ellipse] {Ellipse};

  \node at (1.5,1) [inner sep=6mm,fill=red!20,decorate,ellipse,decoration=
    {text along path,text={This is getting silly}}] {Ellipse};
\end{tikzpicture}
\end{codeexample}

  In the last example, the |text along path| decoration removes the
  path. In such cases it is useful to use a pre- or postaction to
  cause the decoration to be applied only before or after the main
  path has been used. Incidentally, this is another application of the
  |decorate| option that you cannot achieve with the decorate path
  command. 
\begin{codeexample}[]
\begin{tikzpicture}[decoration=zigzag]
  \node at (1.5,1) [inner sep=6mm,fill=red!20,ellipse,
    postaction={decorate,decoration=
    {text along path,text={This is getting silly}}}] {Ellipse};
\end{tikzpicture}
\end{codeexample}
  Here is more useful example, where a postaction is used to add the
  path after the main path has been drawn.
\begin{codeexample}[]
\catcode`\|12
\begin{tikzpicture}
\draw [help lines] grid (3,2);
\fill [draw=red,fill=red!20,
         postaction={decorate,decoration={raise=2pt,text along path,
           text=around and around and around and around we go}}] 
  (0,1) arc (180:-180:1.5cm and 1cm);
\end{tikzpicture}
\end{codeexample} 
\end{key}


\subsection{Adjusting Decorations}

\label{section-decorations-adjust}

\subsubsection{Positioning Decorations Relative to the To-Be-Decorate Path}

The following option, which are only available with \tikzname, allow
you to modify the positioning of decorations relative to the
to-be-decorated path.

\begin{key}{/pgf/decoration/raise=\meta{dimension} (initially 0pt)}
  The segments of the decoration are raised by \meta{dimension}
  relative to the to-be-decorated path. More precisely, the segments
  of the path are offset by this much ``to the left'' of the path as
  we travel along the path. This raising is done after and in addition
  to any transformations set using the |transform| option (see below).

  A negative \meta{dimension} will offset the decoration ``to the
  right'' of the to-be-decorated path.
\begin{codeexample}[]
\begin{tikzpicture}
  \draw [help lines] (0,0) grid (3,2);

  \draw (0,0) -- (1,1) arc (90:0:2 and 1);
  \draw      decorate [decoration=crosses]
        { (0,0) -- (1,1) arc (90:0:2 and 1) };
  \draw[red] decorate [decoration={crosses,raise=5pt}]
        { (0,0) -- (1,1) arc (90:0:2 and 1) };
\end{tikzpicture}
\end{codeexample}
\end{key}

\begin{key}{/pgf/decoration/mirror=\opt{\meta{boolean}}}
  Causes the segments of the decoration to be mirrored along the
  to-be-decorated path. This is done after and in addition to any
  transformations set using the |transform| and/or |raise| options.
\begin{codeexample}[]
\begin{tikzpicture}
  \node (a)          {A};
  \node (b) at (2,1) {B};
  \draw                                                    (a) -- (b);
  \draw[decorate,decoration=brace]                         (a) -- (b);
  \draw[decorate,decoration={brace,mirror},red]            (a) -- (b);
  \draw[decorate,decoration={brace,mirror,raise=5pt},blue] (a) -- (b);
\end{tikzpicture}
\end{codeexample}
\end{key}


\begin{key}{/pgf/decoration/transform=\meta{transformations}}
  This key allows you to specify general \meta{transformations} to be
  applied to the segments of a decoration. These transformations are
  applied before and independently of |raise| and |mirror|
  transformations. The \meta{transformations} should be normal
  \tikzname\ transformations like |shift| or |rotate|.

  In the following example the |shift only| transformation is used to
  make sure that the crosses are \emph{not} sloped along the path.
\begin{codeexample}[]
\begin{tikzpicture}
  \draw [help lines] (0,0) grid (3,2);

  \draw (0,0) -- (1,1) arc (90:0:2 and 1);
  \draw[red,very thick] decorate [decoration={
               crosses,transform={shift only},shape size=1.5mm}]
        { (0,0) -- (1,1) arc (90:0:2 and 1) };
\end{tikzpicture}
\end{codeexample}
\end{key}


\subsubsection{Starting and Ending Decorations Early or Late}

You sometimes may wish to ``end'' a decoration a bit early on the
path. For instance, you might wish a |snake| decoration to stop 5mm
before the end of the path and to continue in a straight line. There
are different ways of achieving this effect, but the easiest may be
the |pre| and |post| options, which only have an effect in
\tikzname. Note, however, that they can only be used with decorations,
not with meta-decorations.

\begin{key}{/pgf/decoration/pre=\meta{decoration} (initially lineto)}
  This key sets a decoration that should be used before the main
  decoration starts. The \meta{decoration} will be used for a length
  of |pre length|, which |0pt| by default. Thus, for the |pre| option
  to have any effect, you also need to set the |pre length| option.
\begin{codeexample}[]
\begin{tikzpicture}
\tikz [decoration={zigzag,pre=lineto,pre length=1cm}]
  \draw [decorate] (0,0) -- (2,1) arc (90:0:1);
\end{tikzpicture}
\end{codeexample}
\begin{codeexample}[]
\begin{tikzpicture}
\tikz [decoration={zigzag,pre=moveto,pre length=1cm}]
  \draw [decorate] (0,0) -- (2,1) arc (90:0:1);
\end{tikzpicture}
\end{codeexample}
\begin{codeexample}[]
\begin{tikzpicture}
\tikz [decoration={zigzag,pre=crosses,pre length=1cm}]
  \draw [decorate] (0,0) -- (2,1) arc (90:0:1);
\end{tikzpicture}
\end{codeexample}

  Note that the default |pre| option is |lineto|, not |curveto|. This
  means that the default |pre| decoration will not follow curves (for
  efficiency reasons). Change the |pre| key to |curveto| if you have a
  curved path. 
\begin{codeexample}[]
\begin{tikzpicture}
\tikz [decoration={zigzag,pre length=3cm}]
  \draw [decorate] (0,0) -- (2,1) arc (90:0:1);
\end{tikzpicture}
\end{codeexample}
\begin{codeexample}[]
\begin{tikzpicture}
\tikz [decoration={zigzag,pre=curveto,pre length=3cm}]
  \draw [decorate] (0,0) -- (2,1) arc (90:0:1);
\end{tikzpicture}
\end{codeexample}
\end{key}

\begin{key}{/pgf/decoration/pre length=\meta{dimension} (initially 0pt)}
  This key sets the distance along which the pre-decoration should be
  used. If you do not need/wish a pre-decoration, set this key to
  |0pt| (exactly this string, not just to something that evaluated to
  the same things such as |0cm|).
\end{key}

\begin{key}{/pgf/decorations/post=\meta{decoration} (initially
    lineto)}
  Works like |pre|, only for the end of the decoration.  
\end{key}

\begin{key}{/pgf/decorations/post length=\meta{dimension} (initially
    0pt)}
  Works like |pre length|, only for the end of the decoration.  
\end{key}

Here is a typical example that shows how these keys can be used:

\begin{codeexample}[]
\begin{tikzpicture}
  [decoration=snake,
   line around/.style={decoration={pre length=#1,post length=#1}}]

  \draw[->,decorate]                  (0,0)    -- ++(3,0);
  \draw[->,decorate,line around=5pt]  (0,-5mm) -- ++(3,0);
  \draw[->,decorate,line around=1cm]  (0,-1cm) -- ++(3,0);
\end{tikzpicture}
\end{codeexample}



\endinput
% Copyright 2006 by Till Tantau
%
% This file may be distributed and/or modified
%
% 1. under the LaTeX Project Public License and/or
% 2. under the GNU Free Documentation License.
%
% See the file doc/generic/pgf/licenses/LICENSE for more details.

\section{Transformations}

\pgfname\ has a powerful transformation mechanism that is similar to
the transformation capabilities of \textsc{metafont}. The present
section explains how you can access it in \tikzname.


\subsection{The Different Coordinate Systems}

It is a long process from  a coordinate like, say, $(1,2)$ or
$(1\mathrm{cm},5\mathrm{pt})$, to the position a point is finally
placed on the display or paper. In order to find out where the point
should go, it is constantly ``transformed,'' which means that it is
mostly shifted around and possibly rotated, slanted, scaled, and
otherwise mutilated. 

In detail, (at least) the following transformations are applied to a
coordinate like $(1,2)$ before a point on the screen is chosen:
\begin{enumerate}
\item
  \pgfname\ interprets a coordinate like $(1,2)$  in its
  $xy$-coordinate system as ``add the current $x$-vector once and the
  current $y$-vector twice to obtain the new point.''
\item
  \pgfname\ applies its coordinate transformation matrix to the
  resulting coordinate. This yields the final position of the point 
  inside the picture.
\item
  The backend driver (like |dvips| or |pdftex|) adds transformation
  commands such the coordinate is shifted to the correct position in
  \TeX's page coordinate system.
\item
  \textsc{pdf} (or PostScript) apply the canvas transformation
  matrix to the point, which can once more change the position on the
  page. 
\item
  The viewer application or the printer applies the device
  transformation matrix to transform the coordinate to its final pixel
  coordinate on the screen or paper.  
\end{enumerate}

In reality, the process is even more involved, but the above should
give the idea: A point is constantly transformed by changes of the
coordinate system.

In \tikzname, you only have access to the first two coordinate systems:
The $xy$-coordinate system and the coordinate transformation matrix
(these will be explained later). \pgfname\ also allows you to change
the canvas transformation matrix, but you have to use commands of
the core layer directly to do so and you ``better know what you are
doing'' when you do this. The moment you start modifying the
canvas matrix, \pgfname\ immediately looses track of all
coordinates and shapes, anchors, and bounding box computations will no
longer work.


\subsection{The XY- and XYZ-Coordinate Systems}
\label{section-xyz}

The first and easiest coordinate systems are \pgfname's $xy$- and
$xyz$-coordinate systems. The idea is very simple: Whenever you
specify a coordinate like |(2,3)| this means $2v_x + 3v_y$, where
$v_x$ is the current \emph{$x$-vector} and $v_y$ is the current
\emph{$y$-vector}. Similarly, the coordinate |(1,2,3)| means $v_x +
2v_y + 3v_z$.

Unlike other packages, \pgfname\ does not insist that $v_x$ actually
has a $y$-component of $0$, that is, that it is a horizontal
vector. Instead, the $x$-vector can point anywhere you
want. Naturally, \emph{normally} you will want the $x$-vector to point
horizontally.

One undesirable effect of this flexibility is that it is not possible
to provide mixed coordinates as in $(1,2\mathrm{pt})$. Life is hard.

To change the $x$-, $y$-, and $z$-vectors, you can use the following
options:

\begin{key}{/tikz/x=\meta{value} (initially 1cm)}
  If \meta{value} is a dimension, the $x$-vector of
  \pgfname's $xyz$-coordinate system is setup to point 
  \meta{value} to the right, that is, to $(\meta{value},0pt)$.

\begin{codeexample}[]
\begin{tikzpicture}
  \draw                  (0,0)   -- +(1,0);
  \draw[x=2cm,color=red] (0,0.1) -- +(1,0);
\end{tikzpicture}
\end{codeexample}    

\begin{codeexample}[]
\tikz \draw[x=1.5cm] (0,0) grid (2,2);
\end{codeexample}    

  The last example shows that the size of steppings in grids, just like
  all other dimensions, are not affected by the $x$-vector. After all,
  the $x$-vector is only used to determine the coordinate of the upper
  right corner of the grid.

  If \meta{value} is a coordinate, the $x$-vector of
  \pgfname's $xyz$-coordinate system to the specified coordinate. If
  \meta{value} contains a comma, it must be put in braces. 

\begin{codeexample}[]
\begin{tikzpicture}
  \draw                            (0,0) -- (1,0);
  \draw[x={(2cm,0.5cm)},color=red] (0,0) -- (1,0);
\end{tikzpicture}
\end{codeexample}

  You can use this, for example, to exchange the meaning of the $x$- and
  $y$-coordinate.

\begin{codeexample}[]
\begin{tikzpicture}[smooth]
  \draw plot coordinates{(1,0) (2,0.5) (3,0) (3,1)};
  \draw[x={(0cm,1cm)},y={(1cm,0cm)},color=red]
        plot coordinates{(1,0) (2,0.5) (3,0) (3,1)};
\end{tikzpicture}
\end{codeexample}
\end{key}

\begin{key}{/tikz/y=\meta{value} (initially 1cm)}
  Works like the |x=| option, only if \meta{value} is a dimension, the
  resulting vector points to $(0,\meta{value})$.
\end{key}

\begin{key}{/tikz/z=\meta{value} (initially \normalfont$-\sqrt{2}$cm)}
  Works like the |y=| option, but now a dimension is means the point
  $(\meta{value},\meta{value})$.

\begin{codeexample}[]
\begin{tikzpicture}[z=-1cm,->,thick]
  \draw[color=red] (0,0,0) -- (1,0,0);
  \draw[color=blue] (0,0,0) -- (0,1,0);
  \draw[color=orange] (0,0,0) -- (0,0,1);
\end{tikzpicture}
\end{codeexample}
\end{key}



\subsection{Coordinate Transformations}

\pgfname\ and \tikzname\ allow you to specify \emph{coordinate
  transformations}. Whenever you specify a coordinate as in |(1,0)| or
|(1cm,1pt)| or |(30:2cm)|, this coordinate is first
``reduced'' to a position of the form ``$x$ points to the right and
  $y$ points upwards.'' For example, |(1in,5pt)| is reduced to
``$72\frac{72}{100}$ points to the right and 5 points upwards'' and
|(90:100pt)| means ``0pt to the right and 100 points upwards.''

The next step is to apply the current \emph{coordinate transformation
  matrix} to the coordinate. For example, the coordinate
transformation matrix might currently be set such that it adds a
certain constant to the $x$ value. Also, it might be setup such that
it, say, exchanges the $x$ and $y$ value. In general, any
``standard'' transformation like translation, rotation, slanting, or
scaling or any combination thereof is possible. (Internally, \pgfname\
keeps track of a coordinate transformation matrix very much like the
concatenation matrix used by \textsc{pdf} or PostScript.)

\begin{codeexample}[]
\begin{tikzpicture}
  \draw[help lines] (0,0) grid (3,2);
  \draw (0,0) rectangle (1,0.5);
  \begin{scope}[xshift=1cm]
    \draw             [red]    (0,0) rectangle (1,0.5);
    \draw[yshift=1cm] [blue]   (0,0) rectangle (1,0.5);
    \draw[rotate=30]  [orange] (0,0) rectangle (1,0.5);
  \end{scope}
\end{tikzpicture}
\end{codeexample}

The most important aspect of the coordinate transformation matrix is
\emph{that it applies to coordinates only!} In particular, the
coordinate transformation has no effect on things like the line width
or the dash pattern or the shading angle. In certain cases, it is not
immediately clear whether the coordinate transformation matrix
\emph{should} apply to a certain dimension. For example, should the
coordinate transformation matrix apply to grids? (It does.) And what
about the size of arced corners? (It does not.) The general rule is
``If there is no `coordinate' involved, even `indirectly,' the matrix
is not applied.'' However, sometimes, you simply have to try or look
it up in the documentation whether the matrix will be applied.

Setting the matrix cannot be done directly. Rather, all you can do is
to ``add'' another transformation to the current matrix. However, all
transformations are local to the current \TeX-group. All
transformations are added using graphic options, which are described
below.

Transformations apply immediately when they are encountered ``in the
middle of a path'' and they apply only to the coordinates on the path
following the transformation option. 

\begin{codeexample}[]
\tikz \draw (0,0) rectangle (1,0.5) [xshift=2cm] (0,0) rectangle (1,0.5);
\end{codeexample}

A final word of warning: You should refrain from using ``aggressive''
transformations like a scaling of a factor of 10000. The reason is
that all transformations are done using \TeX, which has a fairly low
accuracy. Furthermore, in certain situations it is necessary that
\tikzname\ \emph{inverts} the current transformation matrix and this will
fail if the transformation matrix is badly conditioned or even
singular (if you do not know what singular matrices are, you are blessed).   

\begin{key}{/tikz/shift={\ttfamily\char`\{}\meta{coordinate}{\ttfamily\char`\}}}
  Adds the \meta{coordinate} to all coordinates.
\begin{codeexample}[]
\begin{tikzpicture}
  \draw[help lines] (0,0) grid (3,2);
  \draw                       (0,0) -- (1,1) -- (1,0);
  \draw[shift={(1,1)},blue]   (0,0) -- (1,1) -- (1,0);
  \draw[shift={(30:1cm)},red] (0,0) -- (1,1) -- (1,0);
\end{tikzpicture}
\end{codeexample}
\end{key}

\begin{key}{/tikz/shift only}
  This option does not take any parameter. Its effect is to cancel all
  current transformations except for the shifting. This means that the
  origin will remain where it is, but any rotation around the origin
  or scaling relative to the origin or skewing will no longer have an
  effect.

  This option is useful in situtations where a complicated
  transformation is used to ``get to a position,'' but you then wish
  to draw something ``normal'' at this position. 

\begin{codeexample}[]
\begin{tikzpicture}
  \draw[help lines] (0,0) grid (3,2);
  \draw                                      (0,0) -- (1,1) -- (1,0);
  \draw[rotate=30,xshift=2cm,blue]           (0,0) -- (1,1) -- (1,0);
  \draw[rotate=30,xshift=2cm,shift only,red] (0,0) -- (1,1) -- (1,0);
\end{tikzpicture}
\end{codeexample}
\end{key}

\begin{key}{/tikz/xshift=\meta{dimension}}
  Adds \meta{dimension} to the $x$ value of all coordinates.  
\begin{codeexample}[]
\begin{tikzpicture}
  \draw[help lines] (0,0) grid (3,2);
  \draw                   (0,0) -- (1,1) -- (1,0);
  \draw[xshift=2cm,blue]  (0,0) -- (1,1) -- (1,0);
  \draw[xshift=-10pt,red] (0,0) -- (1,1) -- (1,0);
\end{tikzpicture}
\end{codeexample}
\end{key}

\begin{key}{/tikz/yshift=\meta{dimension}}
  Adds \meta{dimension} to the $y$ value of all coordinates.
\end{key}

\begin{key}{/tikz/scale=\meta{factor}}
  Multiplies all coordinates by the given \meta{factor}. The
  \meta{factor} should not be excessively large in absolute terms or
  very near to zero.
\begin{codeexample}[]
\begin{tikzpicture}
  \draw[help lines] (0,0) grid (3,2);
  \draw               (0,0) -- (1,1) -- (1,0);
  \draw[scale=2,blue] (0,0) -- (1,1) -- (1,0);
  \draw[scale=-1,red] (0,0) -- (1,1) -- (1,0);
\end{tikzpicture}
\end{codeexample}
\end{key}

\begin{key}{/tikz/scale around={\ttfamily\char`\{}\meta{factor}|:|\meta{coordinate}{\ttfamily\char`\}}}
  Scales the coordinate system by \meta{factor}, put with the ``origin
  of scaling'' centered on \meta{coordinate} rather than the origin. 
\begin{codeexample}[]
\begin{tikzpicture}
  \draw[help lines] (0,0) grid (3,2);
  \draw                             (0,0) -- (1,1) -- (1,0);
  \draw[scale=2,blue]               (0,0) -- (1,1) -- (1,0);
  \draw[scale around={2:(1,1)},red] (0,0) -- (1,1) -- (1,0);
\end{tikzpicture}
\end{codeexample}
\end{key}

\begin{key}{/tikz/xscale=\meta{factor}}
  Multiplies only the $x$-value of all coordinates by the given
  \meta{factor}. 
\begin{codeexample}[]
\begin{tikzpicture}
  \draw[help lines] (0,0) grid (3,2);
  \draw                (0,0) -- (1,1) -- (1,0);
  \draw[xscale=2,blue] (0,0) -- (1,1) -- (1,0);
  \draw[xscale=-1,red] (0,0) -- (1,1) -- (1,0);
\end{tikzpicture}
\end{codeexample}
\end{key}

\begin{key}{/tikz/yscale=\meta{factor}}
  Multiplies only the $y$-value of all coordinates by \meta{factor}.
\end{key}

\begin{key}{/tikz/xslant=\meta{factor}}
  Slants the coordinate horizontally by the given \meta{factor}:
\begin{codeexample}[]
\begin{tikzpicture}
  \draw[help lines] (0,0) grid (3,2);
  \draw                (0,0) -- (1,1) -- (1,0);
  \draw[xslant=2,blue] (0,0) -- (1,1) -- (1,0);
  \draw[xslant=-1,red] (0,0) -- (1,1) -- (1,0);
\end{tikzpicture}
\end{codeexample}
\end{key}


\begin{key}{/tikz/yslant=\meta{factor}}
  Slants the coordinate vertically by the given \meta{factor}:
\begin{codeexample}[]
\begin{tikzpicture}
  \draw[help lines] (0,0) grid (3,2);
  \draw                (0,0) -- (1,1) -- (1,0);
  \draw[yslant=2,blue] (0,0) -- (1,1) -- (1,0);
  \draw[yslant=-1,red] (0,0) -- (1,1) -- (1,0);
\end{tikzpicture}
\end{codeexample}
\end{key}


\begin{key}{/tikz/rotate=\meta{degree}}
  Rotates the coordinate system by \meta{degree}:
\begin{codeexample}[]
\begin{tikzpicture}
  \draw[help lines] (0,0) grid (3,2);
  \draw                 (0,0) -- (1,1) -- (1,0);
  \draw[rotate=40,blue] (0,0) -- (1,1) -- (1,0);
  \draw[rotate=-20,red] (0,0) -- (1,1) -- (1,0);
\end{tikzpicture}
\end{codeexample}
\end{key}

\begin{key}{/tikz/rotate around={\ttfamily\char`\{}\meta{degree}|:|\meta{coordinate}{\ttfamily\char`\}}}
  Rotates the coordinate system by \meta{degree} around the point
  \meta{coordinate}.
\begin{codeexample}[]
\begin{tikzpicture}
  \draw[help lines] (0,0) grid (3,2);
  \draw                                (0,0) -- (1,1) -- (1,0);
  \draw[rotate around={40:(1,1)},blue] (0,0) -- (1,1) -- (1,0);
  \draw[rotate around={-20:(1,1)},red] (0,0) -- (1,1) -- (1,0);
\end{tikzpicture}
\end{codeexample}
\end{key}


\begin{key}{/tikz/cm={\ttfamily\char`\{}\meta{$a$}|,|\meta{$b$}|,|\meta{$c$}|,|\meta{$d$}|,|\meta{coordinate}{\ttfamily\char`\}}}
  applies the following transformation to all coordinates: Let $(x,y)$
  be the coordinate to be transformed and let \meta{coordinate}
  specify the point $(t_x,t_y)$. Then the new coordinate is given by
  $\left(\begin{smallmatrix} a & b \\ c & d\end{smallmatrix}\right)
  \left(\begin{smallmatrix} x \\ y \end{smallmatrix}\right) +
  \left(\begin{smallmatrix} t_x \\ t_y
  \end{smallmatrix}\right)$. Usually, you do not use this option
  directly. 
\begin{codeexample}[]
\begin{tikzpicture}
  \draw[help lines] (0,0) grid (3,2);
  \draw                             (0,0) -- (1,1) -- (1,0);
  \draw[cm={1,1,0,1,(0,0)},blue]    (0,0) -- (1,1) -- (1,0);
  \draw[cm={0,1,1,0,(1cm,1cm)},red] (0,0) -- (1,1) -- (1,0);
\end{tikzpicture}
\end{codeexample}
\end{key}

\begin{key}{/tikz/reset cm}
  Completely resets the coordinate transformation matrix to the
  identity matrix. This will destroy not only the transformations
  applied in the current scope, but also all transformations inherited
  from surrounding scopes. Do not use this option, unless you really,
  really know what you are doing.
\end{key}




\subsection{Canvas Transformations}

A \emph{canvas transformation}, see
Section~\ref{section-design-transformations} for details, is best
thought of as a transformation in which the drawing canvas is
stretched or rotated. Imaging writing something on a balloon (the
canvas) and then blowing air into the balloon: Not only does the text
become larger, the thin lines also become larger. In particular, if
you scale the canvas by a factor of two, all lines are twice as
thick.

Canvas transformations should be used with great care. In most
circumstances you do \emph{not} want line widths to change in a
picture as this creates visual inconsistency.

Just as important, when
you use canvas transformations \emph{\pgfname\ looses track of
  positions of nodes and of picture sizes} since it does not take the
effect of canvas transformations into account when it computes
coordinates of nodes (you not, however, rely on this; it may change in
the future).

Finally, not that a canvas transformation always applies to a path as
a whole, it is not possible (as for coordinate transformations) to use
different transformations in different parts of a path.

In short, you should not use canvas transformations unless you really
know what you are doing.

\begin{key}{/tikz/transform canvas=\meta{options}}
  The \meta{options} should contain coordinate transformations options
  like |scale| or |xshift|. Multiple options can be given, their
  effects accumulate in the usual manner. The effect of these
  \meta{options} (immediately) changes the current canvas
  transformation matrix. The coordinate transformation matrix is not
  changed. Tracking of the picture size is (locally) switched off and
  the node coordinate will no longer be correct.
\begin{codeexample}[]
\begin{tikzpicture}
  \draw[help lines] (0,0) grid (3,2);
  \draw                                    (0,0) -- (1,1) -- (1,0);
  \draw[transform canvas={scale=2},blue]   (0,0) -- (1,1) -- (1,0);
  \draw[transform canvas={rotate=180},red] (0,0) -- (1,1) -- (1,0);
\end{tikzpicture}
\end{codeexample}
\end{key}




\part{Libraries}
\label{part-libraries}

{\Large \emph{by Till Tantau}}


\bigskip
\noindent
In this part the library packages are documented. They
provide additional predefined graphic objects like new arrow heads or
new plot marks, but also sometimes extensions of the basic \pgfname\
or \tikzname\ system. The libraries are not loaded by default since
many users will not need them.  

\medskip
\noindent
\begin{codeexample}[graphic=white]
\tikzset{
  ld/.style={level distance=#1},lw/.style={line width=#1},  
  level 1/.style={ld=4.5mm, trunk,          lw=1ex ,sibling angle=60},
  level 2/.style={ld=3.5mm, trunk!80!leaf a,lw=.8ex,sibling angle=56},
  level 3/.style={ld=2.75mm,trunk!60!leaf a,lw=.6ex,sibling angle=52},
  level 4/.style={ld=2mm,   trunk!40!leaf a,lw=.4ex,sibling angle=48},
  level 5/.style={ld=1mm,   trunk!20!leaf a,lw=.3ex,sibling angle=44},
  level 6/.style={ld=1.75mm,leaf a,         lw=.2ex,sibling angle=40},
}
\pgfarrowsdeclare{leaf}{leaf}
  {\pgfarrowsleftextend{-2pt} \pgfarrowsrightextend{1pt}}
{
  \pgfpathmoveto{\pgfpoint{-2pt}{0pt}}
  \pgfpatharc{150}{30}{1.8pt}
  \pgfpatharc{-30}{-150}{1.8pt}    
  \pgfusepathqfill
}

\newcommand{\logo}[5]
{
  \colorlet{border}{#1}
  \colorlet{trunk}{#2}
  \colorlet{leaf a}{#3}
  \colorlet{leaf b}{#4}
  \begin{tikzpicture}
    \scriptsize\scshape
    \draw[border,line width=1ex,yshift=.3cm,
          yscale=1.45,xscale=1.05,looseness=1.42]
      (1,0) to [out=90, in=0]    (0,1)  to [out=180,in=90]  (-1,0)
            to [out=-90,in=-180] (0,-1) to [out=0,  in=-90] (1,0) -- cycle;

    \coordinate (root) [grow cyclic,rotate=90]
    child {
      child [line cap=round] foreach \a in {0,1} {
        child foreach \b in {0,1} {
          child foreach \c in {0,1} {
            child foreach \d in {0,1} {
              child foreach \leafcolor in {leaf a,leaf b}
                { edge from parent [color=\leafcolor,-#5] }
        } } }
      } edge from parent [shorten >=-1pt,serif cm-,line cap=butt]
    };

    \node [text centered,text width=2cm,below] at (0pt,-.5ex)
    { \textcolor{border}{T}heoretical \\ \textcolor{border}{C}omputer \\
      \textcolor{border}{S}cience };
  \end{tikzpicture}
}
\begin{minipage}{3cm}
  \logo{green!80!black}{green!25!black}{green}{green!80}{leaf}\\
  \logo{green!50!black}{black}{green!80!black}{red!80!green}{leaf}\\
  \logo{red!75!black}{red!25!black}{red!75!black}{orange}{leaf}\\
  \logo{black!50}{black}{black!50}{black!25}{}
\end{minipage}
\end{codeexample}

% Copyright 2006 by Till Tantau
%
% This file may be distributed and/or modified
%
% 1. under the LaTeX Project Public License and/or
% 2. under the GNU Free Documentation License.
%
% See the file doc/generic/pgf/licenses/LICENSE for more details.


\section{Arrow Tip Library}
\label{section-library-arrows}

\begin{pgflibrary}{arrows}
  The package defines additional arrow tips, which are described
  below. Note that neither the standard packages nor
  this package defines an arrow name containing |>| or |<|. These are
  left for the user to defined as he or she sees fit.
\end{pgflibrary}

The arrow tips |to|, |stealth|, |latex|, |space|, their reversed
forms, and \verb!|! are predefined, but listed below for completeness,
nevertheless. 


\subsection{Mathematical Arrow Tips}

\begin{tabular}{ll}
  \symarrow{to} \\
  \symarrow{to reversed} \\
  \symarrowdouble{implies} \\
\end{tabular}


\subsection{Triangular Arrow Tips}

\begin{tabular}{ll}
  \symarrowdouble{latex} \\
  \symarrowdouble{latex reversed}  \\
  \symarrow{latex'} \\
  \symarrow{latex' reversed}  \\
  \symarrowdouble{stealth} \\
  \symarrowdouble{stealth reversed}  \\
  \symarrow{stealth'} \\
  \symarrow{stealth' reversed}\\
  \symarrow{triangle 90} \\
  \symarrow{triangle 90 reversed}   \\
  \symarrow{triangle 60} \\
  \symarrow{triangle 60 reversed}   \\
  \symarrow{triangle 45} \\
  \symarrow{triangle 45 reversed}   \\
  \symarrow{open triangle 90} \\
  \symarrow{open triangle 90 reversed}   \\
  \symarrow{open triangle 60} \\
  \symarrow{open triangle 60 reversed}   \\
  \symarrow{open triangle 45} \\
  \symarrow{open triangle 45 reversed}   \\
\end{tabular}


\subsection{Barbed Arrow Tips}

\begin{tabular}{ll}
  \symarrow{angle 90} \\
  \symarrow{angle 90 reversed}   \\
  \symarrow{angle 60} \\
  \symarrow{angle 60 reversed}   \\
  \symarrow{angle 45} \\
  \symarrow{angle 45 reversed}   \\
  \symarrow{hooks} \\
  \symarrow{hooks reversed} \\
\end{tabular}


\subsection{Bracket-Like Arrow Tips}

{
\bigskip
\catcode`\|=12
\begin{tabular}{ll}
  \sarrow{[}{]} \\
  \sarrow{]}{[} \\
  \sarrow{(}{)} \\
  \sarrow{)}{(} \\
  \index{*vbar@\protect\texttt{\protect\myvbar} arrow tip}%
  \index{Arrow tips!*vbar@\protect\texttt{\protect\myvbar}}
  \texttt{\char`\|-\char`\|} & yields thick  
  \begin{tikzpicture}[arrows={|-|},thick]
    \useasboundingbox (0pt,-0.5ex) rectangle (1cm,2ex);
    \draw (0,0) -- (1,0);
  \end{tikzpicture} and thin
  \begin{tikzpicture}[arrows={|-|},thin]
    \useasboundingbox (0pt,-0.5ex) rectangle (1cm,2ex);
    \draw (0,0) -- (1,0);
  \end{tikzpicture}
\end{tabular}
}

\subsection{Circle and Diamond Arrow Tips}


\begin{tabular}{ll}
  \symarrow{o} \\
  \symarrow{*} \\
  \symarrow{diamond} \\
  \symarrow{open diamond}   \\
\end{tabular}



\subsection{Serif-Like Arrow Tips}

\begin{tabular}{ll}
  \symarrow{serif cm}
\end{tabular}


\subsection{Partial Arrow Tips}

\begin{tabular}{ll}
  \symarrow{left to} \\
  \symarrow{left to reversed} \\
  \symarrow{right to} \\
  \symarrow{right to reversed} \\
  \symarrow{left hook} \\
  \symarrow{left hook reversed} \\
  \symarrow{right hook} \\
  \symarrow{right hook reversed}
\end{tabular}



\subsection{Line Caps}

\begin{tabular}{ll}
  \carrow{round cap} \\
  \carrow{butt cap} \\
  \carrow{triangle 90 cap} \\
  \carrow{triangle 90 cap reversed} \\
  \carrow{fast cap} \\
  \carrow{fast cap reversed} \\
\end{tabular}


\subsection{Spacing Tips}

The spacing arrow tips are useful for combining them with other arrows
to get arrows that do not touch the end of the line.

\begin{tabular}{ll}
  \symarrow{space} \\
\end{tabular}


%%% Local Variables: 
%%% mode: latex
%%% TeX-master: "pgfmanual-pdftex-version"
%%% End: 

% Copyright 2006 by Till Tantau
%
% This file may be distributed and/or modified
%
% 1. under the LaTeX Project Public License and/or
% 2. under the GNU Free Documentation License.
%
% See the file doc/generic/pgf/licenses/LICENSE for more details.


\section{Automata Drawing Library}

\begin{tikzlibrary}{automata}
  This packages provides shapes and styles for drawing finite state
  automata and Turing machines. 
\end{tikzlibrary}


\subsection{Drawing Automata}

The automata drawing library is intended to make it easy to draw
finite automata and Turing machines. It does not cover every
situation imaginable, but most finite automata and Turing machines
found in text books can be drawn in a nice and convenient fashion
using this library. 

To draw an automaton, proceed as follows:
\begin{enumerate}
\item For each state of the automaton, there should be one node with
  the option |state|.
\item To place the states, you can either use absolute positions or
  relative positions, using options like |above of| or |right of|.
\item Give a unique name to each state node.
\item Accepting and initial states are indicated by adding the
  options |accepting| and |initial|, respectively, to the state
  nodes.
\item Once the states are fixed, the edges can be added. For this, the
  |edge| operation is most useful. It is, however, also possible to
  add edges after each node has been placed.
\item For loops, use the |edge [loop]| operation.
\end{enumerate}

Let us now see how this works for a real example. Let us consider a
nondeterminsitic four state automaton that checks whether an contains
the sequence $0^*1$ or the sequence $1^*0$. 
\begin{codeexample}[]
\begin{tikzpicture}[shorten >=1pt,node distance=2cm,auto]
  \draw[help lines] (0,0) grid (3,2);

  \node[state,initial]  (q_0)                      {$q_0$};
  \node[state]          (q_1) [above right of=q_0] {$q_1$};
  \node[state]          (q_2) [below right of=q_0] {$q_2$};
  \node[state,accepting](q_3) [below right of=q_1] {$q_3$};

  \path[->] (q_0) edge              node        {0} (q_1)
                  edge              node [swap] {1} (q_2)
            (q_1) edge              node        {1} (q_3)
                  edge [loop above] node        {0} ()
            (q_2) edge              node [swap] {0} (q_3)
                  edge [loop below] node        {1} ();
\end{tikzpicture}
\end{codeexample}


\subsection{States With and Without Output}

The |state| style actually just ``selects'' a default underlying
style. Thus, you can define multiple new complicated state style and
then simply set the |state| style to your given style to get the
desired kind of styles.

By default, the following state styles are defined:
\begin{stylekey}{/tikz/state without output}
  This node style causes nodes to be drawn circles. Also, this style
  calls |every state|.
\end{stylekey}

\begin{stylekey}{/tikz/state with output}
  This node style causes nodes to be drawn as split circles, that is,
  using the |circle split| shape. In the upper part of the shape you
  have the name of the style, in the lower part the output is
  placed. To specify the output, use the command |\nodepart{lower}|
  inside the node. This style also calls |every state|.
\begin{codeexample}[]
\begin{tikzpicture}
  \draw[help lines] (0,0) grid (3,2);

  \node[state without output] {$q_0$};
  
  \node[state with output] at (2,0) {$q_1$ \nodepart{lower} $00$};
\end{tikzpicture}
\end{codeexample}
\end{stylekey}

\begin{stylekey}{/tikz/state (initially state without output)}
  You should redefine it to something else, if you wish to use states
  of a different nature.
\begin{codeexample}[]
\begin{tikzpicture}[state/.style=state with output]
  \node[state]          {$q_0$ \nodepart{lower} $11$};
  \node[state] at (2,0) {$q_1$ \nodepart{lower} $00$};
\end{tikzpicture}
\end{codeexample}
\end{stylekey}

\begin{stylekey}{/tikz/every state (initially \normalfont empyt)}
  This style is used by |state with output| and also by
  |state without output|. By default, it does nothing, but you can use
  it to make your state look more fancy:
\begin{codeexample}[]
\begin{tikzpicture}[shorten >=1pt,node distance=2cm,>=stealth',
    every state/.style={draw=blue!50,very thick,fill=blue!20}]

  \node[state,initial]  (q_0)                      {$q_0$};
  \node[state]          (q_1) [above right of=q_0] {$q_1$};
  \node[state]          (q_2) [below right of=q_0] {$q_2$};

  \path[->] (q_0) edge              node [above left]  {0} (q_1)
                  edge              node [below left]  {1} (q_2)
            (q_1) edge [loop above] node               {0} ()
            (q_2) edge [loop below] node               {1} ();
\end{tikzpicture}
\end{codeexample}
\end{stylekey}


\subsection{Initial and Accepting States}

The styles |initial| and |accepting| are similar to the |state| style
as they also just select an ``underlying'' style, which installs the
actual settings for initial and accepting states.

Let us start with the initial states.
\begin{stylekey}{/tikz/initial (initially initial by arrow)}
  This style is used to draw initial states.
\end{stylekey}
\begin{stylekey}{/tikz/initial by arrow}
  This style causes an arrow and, possibly, some text to be added to
  the node. The arrow points from the text to the node. The node text
  and the direction and the distance can be set using the following
  key:
  \begin{key}{/tikz/initial text=\meta{text} (initially start)}
    This key sets the text to be used. Use an empty text to suppress
    all text.
  \end{key}
  \begin{key}{/tikz/initial where=\meta{direction} (initially left)}
    Set the place where the text should be shown. Allowed values are
    |above|, |below|, |left|, and |right|.
  \end{key}
  \begin{key}{/tikz/intial distance=\meta{distance} (initially 3ex)}
    Sets the length of the arrow leading from the text to the state
    node.
  \end{key}
  \begin{stylekey}{/tikz/every initial by arrow (initially \normalfont empty)}
    This style is executed at the beginning of every path that contains
    the arrow and the text. You can use it to, say, make the text red or
    whatever.
  \end{stylekey}
\begin{codeexample}[]
\begin{tikzpicture}[every initial by arrow/.style={text=red,->>}]
  \node[state,initial,initial distance=2cm] {$q_0$};
\end{tikzpicture}
\end{codeexample}
  %<<
\end{stylekey}
\begin{stylekey}{/tikz/initial above}
  This is a shorthand for |initial by arrow,initial where=above|.
\end{stylekey}
\begin{stylekey}{/tikz/initial below}
  Works similarly to the previous option.
\end{stylekey}
\begin{stylekey}{/tikz/initial left}
  Works similarly to the previous option.
\end{stylekey}
\begin{stylekey}{/tikz/initial right}
  Works similarly to the previous option.
\end{stylekey}

\begin{stylekey}{/tikz/initial by diamond}
  This style uses a diamond to indicate an initial node. 
\end{stylekey}

For the accepting states, the sitation is similar: There is also an
|accepting| style that selects the way accepting states are
rendered. There are now two options: First,
|accepting by arrow|, which works the same way as |initial by arrow|,
only with the direction of arrow reversed, and |accepting by double|,
where accepting states get a double line around them.

\begin{stylekey}{/tikz/accepting (initially accepting by double)}
  This style is used to draw accepting states.  You can replace
  this by the style |accepting by arrow| to get accepting states with
  an arrow leaving them.
\end{stylekey}

\begin{stylekey}{/tikz/accepting by double}
  This style causes a double line to be drawn arond a state.
\end{stylekey}

\begin{stylekey}{/tikz/accepting by arrow}
  This style causes an arrow and, possibly, some text to be added to
  the node. The arrow points to the text from the node.

  The same options as for initial states can be used, only with
  |initial| replaced by |accepting|:
  \begin{key}{/tikz/accepting text=\meta{text} (initially \normalfont empty)}
    This key sets the text to be used.
  \end{key}
  \begin{key}{/tikz/accepting where=\meta{direction} (initially right)}
    Set the place where the text should be shown. Allowed values are
    |above|, |below|, |left|, and |right|.
  \end{key}
  \begin{key}{/tikz/intial distance=\meta{distance} (initially 3ex)}
    Sets the length of the arrow leading from the text to the state
    node.
  \end{key}
  \begin{stylekey}{/tikz/every accepting by arrow (initially \normalfont empty)}
    Executed at the beginning of every path that contains
    the arrow and the text.
  \end{stylekey}  
\begin{codeexample}[]
\begin{tikzpicture}
  [shorten >=1pt,node distance=2cm,>=stealth',initial text=,
   every state/.style={draw=blue!50,very thick,fill=blue!20},
   accepting/.style=accepting by arrow]

  \node[state,initial]  (q_0)                      {$q_0$};
  \node[state]          (q_1) [above right of=q_0] {$q_1$};
  \node[state]          (q_2) [below right of=q_0] {$q_2$};
  \node[state,accepting](q_3) [below right of=q_1] {$q_3$};

  \path[->] (q_0) edge              node [above left]  {0} (q_1)
                  edge              node [below left]  {1} (q_2)
            (q_1) edge              node [above right] {1} (q_3)
                  edge [loop above] node               {0} ()
            (q_2) edge              node [below right] {0} (q_3)
                  edge [loop below] node               {1} ();
\end{tikzpicture}
\end{codeexample}
\end{stylekey}

\begin{stylekey}{/tikz/accepting above}
  This is a shorthand for |accepting by arrow,accepting where=above|.
\end{stylekey}
\begin{stylekey}{/tikz/accepting below}
  Works similarly to the previous option.
\end{stylekey}
\begin{stylekey}{/tikz/accepting left}
  Works similarly to the previous option.
\end{stylekey}
\begin{stylekey}{/tikz/accepting right}
  Works similarly to the previous option.
\end{stylekey}



\subsection{Examples}

In the following example, we once more typeset the automaton presented
in the previous sections. This time, we use the following rule for
accepting/initial state: Initial states are red, accepting states are
green, and normal states are orange. Then, we must find a path from a
red state to a green state. 

\begin{codeexample}[]
\begin{tikzpicture}[shorten >=1pt,node distance=2cm,>=stealth',thick,
    every state/.style={fill,draw=none,orange,text=white},
    accepting/.style  ={green!50!black,text=white},
    initial/.style    ={red,text=white}]

  \node[state,initial]  (q_0)                      {$q_0$};
  \node[state]          (q_1) [above right of=q_0] {$q_1$};
  \node[state]          (q_2) [below right of=q_0] {$q_2$};
  \node[state,accepting](q_3) [below right of=q_1] {$q_3$};

  \path[->] (q_0) edge              node [above left]  {0} (q_1)
                  edge              node [below left]  {1} (q_2)
            (q_1) edge              node [above right] {1} (q_3)
                  edge [loop above] node               {0} ()
            (q_2) edge              node [below right] {0} (q_3)
                  edge [loop below] node               {1} ();
\end{tikzpicture}
\end{codeexample}

The next example is the current candidate for the five-state busiest
beaver:

\begin{codeexample}[]
\begin{tikzpicture}[->,>=stealth',shorten >=1pt,%
                    auto,node distance=2cm,semithick,
                    inner sep=2pt,bend angle=45]
  \node[initial,state] (A)                    {$q_a$};
  \node[state]         (B) [above right of=A] {$q_b$};
  \node[state]         (D) [below right of=A] {$q_d$};
  \node[state]         (C) [below right of=B] {$q_c$};
  \node[state]         (E) [below of=D]       {$q_e$};

  \path [every node/.style={font=\footnotesize}]
        (A) edge              node {0,1,L} (B)
            edge              node {1,1,R} (C)
        (B) edge [loop above] node {1,1,L} (B)  
            edge              node {0,1,L} (C)
        (C) edge              node {0,1,L} (D)
            edge [bend left]  node {1,0,R} (E)    
        (D) edge [loop below] node {1,1,R} (D)
            edge              node {0,1,R} (A)
        (E) edge [bend left]  node {1,0,R} (A);
\end{tikzpicture}
\end{codeexample}


%%% Local Variables: 
%%% mode: latex
%%% TeX-master: "pgfmanual-pdftex-version"
%%% End: 

% Copyright 2003 by Till Tantau <tantau@cs.tu-berlin.de>.
%
% This program can be redistributed and/or modified under the terms
% of the LaTeX Project Public License Distributed from CTAN
% archives in directory macros/latex/base/lppl.txt.




\section{Background Library}

\label{section-tikz-backgrounds}

\begin{tikzlibrary}{backgrounds}
  This library defines ``backgrounds'' for pictures. This does not
  refer to background pictures, but rather to frames drawn around and
  behind pictures. For example, this package allows you to just add
  the |framed| option to a picture to get a rectangular box around
  your picture or |gridded| to put a grid behind your picture.
\end{tikzlibrary}

When this package is loaded, the following styles become available:
\begin{itemize}
  \itemstyle{show background rectangle}
  This style causes a rectangle to be drawn behind your graphic. This
  style option must be given to the |{tikzpicture}| environment or to
  the |\tikz| command.
\begin{codeexample}[]
\begin{tikzpicture}[show background rectangle]
  \draw (0,0) ellipse (10mm and 5mm);
\end{tikzpicture}
\end{codeexample}
  The size of the background rectangle is determined as follows:
  We start with the bounding box of the picture. Then, a certain
  separator distance is added on the sides. This distance can be
  different for the $x$- and $y$-directions and can be set using the
  following options:
  \begin{itemize}
    \itemoption{inner frame xsep}|=|\meta{dimension}
    Sets the additional horizontal separator distance for the
    background rectangle. The default is |1ex|.
    \itemoption{inner frame ysep}|=|\meta{dimension}
    Same for the vertical separator distance.
    \itemoption{inner frame sep}|=|\meta{dimension}
    sets the horizontal and vertical separator distances
    simultaneously. 
  \end{itemize}
  The following two styles make setting the inner separator a bit
  easier to remember:
  \begin{itemize}
    \itemstyle{tight background} Sets the inner frame separator to
    0pt. The background rectangle will have the size of the bounding
    box. 
    \itemstyle{loose background} Sets the inner frame separator to 2ex.
  \end{itemize}
    
  You can influence how the background rectangle is rendered by setting
  the following style:
  \begin{itemize}
    \itemstyle{background rectangle}
    This style dictates how the background rectangle is drawn or
    filled. By default this style is set to |draw|, which causes the
    path of the background rectangle to be drawn in the usual
    way. Setting this style to, say, |fill=blue!20| causes a light
    blue background to be added to the picture. You can also use more
    fancy settings as shown in the following example:
\begin{codeexample}[]
\tikzstyle{background rectangle}=
  [double,ultra thick,draw=red,top color=blue,rounded corners]      
\begin{tikzpicture}[show background rectangle]
  \draw (0,0) ellipse (10mm and 5mm);
\end{tikzpicture}
\end{codeexample}
    Naturally, no one in their right mind would use the above, but
    here is a nice background: 
\begin{codeexample}[]
\tikzstyle{background rectangle}=
  [draw=blue!50,fill=blue!20,rounded corners=1ex]      
\begin{tikzpicture}[show background rectangle]
  \draw (0,0) ellipse (10mm and 5mm);
\end{tikzpicture}
\end{codeexample}
\end{itemize}
  \itemstyle{framed}
  This is a shorthand for |show background rectangle|.
  \itemstyle{show background grid}
  This style behaves similarly to the |show background rectangle|
  style, but it will not use a rectangle path, but a grid. The lower
  left and upper right corner of the grid is computed in the same way
  as for the background rectangle:
\begin{codeexample}[]
\begin{tikzpicture}[show background grid]
  \draw (0,0) ellipse (10mm and 5mm);
\end{tikzpicture}
\end{codeexample}
  You can influence the background grid by setting
  the following style:
  \begin{itemize}
    \itemstyle{background grid}
    This style dictates how the background grid path is drawn. The
    default is |draw,help lines|. 
\begin{codeexample}[]
\tikzstyle{background grid}=[thick,draw=red,step=.5cm]
\begin{tikzpicture}[show background grid]
  \draw (0,0) ellipse (10mm and 5mm);
\end{tikzpicture}
\end{codeexample}
  This option can be combined with the |framed| option (use the
  |framed| option first):
\begin{codeexample}[]
\tikzstyle{background grid}=[thick,draw=red,step=.5cm]
\tikzstyle{background rectangle}=[rounded corners,fill=yellow]
\begin{tikzpicture}[framed,gridded]
  \draw (0,0) ellipse (10mm and 5mm);
\end{tikzpicture}
\end{codeexample}
  \itemstyle{gridded}
  This is a shorthand for |show background grid|.
  \end{itemize}
  \itemstyle{show background top}
  This style causes a single line to be drawn at the top of the
  background rectangle. Normally, the line coincides exactly with the
  top line of the background rectangle:
\begin{codeexample}[]
\tikzstyle{background rectangle}=[fill=yellow]    
\begin{tikzpicture}[framed,show background top]
  \draw (0,0) ellipse (10mm and 5mm);
\end{tikzpicture}
\end{codeexample}
  The following option allows you to lengthen (or shorten) the line:
  \begin{itemize}
    \itemoption{outer frame xsep}|=|\meta{dimension}
    The \meta{dimension} is added at the left and right side of the
    line. 
\begin{codeexample}[]
\tikzstyle{background rectangle}=[fill=yellow]    
\begin{tikzpicture}
    [framed,show background top,outer frame xsep=1ex]
  \draw (0,0) ellipse (10mm and 5mm);
\end{tikzpicture}
\end{codeexample}
    \itemoption{outer frame ysep}|=|\meta{dimension}
    This option does not apply to the top line, but to the left and
    right lines, see below.
    \itemoption{outer frame sep}|=|\meta{dimension}
    Sets both the $x$- and $y$-separation.
\begin{codeexample}[]
\tikzstyle{background rectangle}=[fill=blue!20]    
\begin{tikzpicture}
  [outer frame sep=1ex,%
   show background top,%
   show background bottom,%
   show background left,%
   show background right]
  \draw (0,0) ellipse (10mm and 5mm);
\end{tikzpicture}
\end{codeexample}
  \end{itemize}
  You can influence how the line is drawn grid by setting
  the following style:
  \begin{itemize}
    \itemstyle{background top}
    Default is |draw|.
\begin{codeexample}[]
\tikzstyle{background rectangle}=[fill=blue!20]    
\tikzstyle{background top}=[draw=blue!50,line width=1ex]
\begin{tikzpicture}[framed,show background top]
  \draw (0,0) ellipse (10mm and 5mm);
\end{tikzpicture}
\end{codeexample}
  \end{itemize}
  \itemstyle{show background bottom}
  works like the style for the top line.
  \itemstyle{show background left}
  works like the style for the top line.
  \itemstyle{show background right}
  works like the style for the top line.
\end{itemize}



%%% Local Variables: 
%%% mode: latex
%%% TeX-master: "pgfmanual-pdftex-version"
%%% End: 

% Copyright 2006 by Till Tantau
%
% This file may be distributed and/or modified
%
% 1. under the LaTeX Project Public License and/or
% 2. under the GNU Free Documentation License.
%
% See the file doc/generic/pgf/licenses/LICENSE for more details.


\section{Calendar Library}

\label{section-calender-snakes}

\begin{pgflibrary}{calendar}
  The library defines the |\calendar| command, which can be used to
  typeset calendars. The command relies on the |\pgfcalendar| command
  from the |pgfcalendar| package, which is loaded automatically.

  The |\calendar| command is quite configurable, allowing you to
  produce all kinds of different calendars.
\end{pgflibrary}


\subsection{Calendar Command}

The core command for creating calendars in \tikzname\ is the
following:
\begin{command}{\calendar \meta{calendar specification}|;|}
  The syntax for this command is similar to commands like |\node| or
  |\matrix|. However, it has its complete own parser and only those
  commands described in the following will be recognized, nothing
  else. Note, furthermore, that a \meta{calendar specification} is not
  a path specification, indeed, no path is created for the calendar.

  
\end{command}


\subsection{Day Arrangements}


\subsection{Month Labels}


\subsection{Examples}

In the following, some example calendars are shown that come either
from real applications or are just nice to look at.

Let us start with a year-2100-countdown, in which we cross out dates
as we approach the big celebration. For
this, we set the shape to |strike out| for these dates.

\begin{codeexample}[leave comments]
\begin{tikzpicture}
  \calendar
  [
    dates=2099-12-01 to 2100-01-last,
    week list,inner sep=2pt,month label above centered,
    month text=\%mt \%y0
  ]
  if (at most=2099-12-29) [nodes={strike out,draw}]
  if (weekend)            [black!50,nodes={draw=none}]
  ;
\end{tikzpicture}
\end{codeexample}

The next calendar shows a deadline, which is 10 days in the future
from the current date. The last three days before the deadline are in
red, because we really should be done by then. All days on which we
can no longer work on the project are crossed out.

\begin{codeexample}[leave comments]
\begin{tikzpicture}
  \calendar
  [
    dates=\year-\month-\day+-25 to \year-\month-\day+25,
    week list,inner sep=2pt,month label above centered,
    month text=\textit{\%mt \%y0}
  ]
  if (at least=\year-\month-\day) {}
    else [nodes={strike out,draw}]
  if (at most=\year-\month-\day+7)
    [green!50!black]
  if (between=\year-\month-\day+8 and \year-\month-\day+10)
    [red]
  if (Sunday)
    [gray,nodes={draw=none}]
  ;
\end{tikzpicture}
\end{codeexample}

The following example is a futuristic calendar that is all about circles:

\begin{codeexample}[]
\sffamily

\colorlet{winter}{blue}  
\colorlet{spring}{green!60!black}  
\colorlet{summer}{orange}  
\colorlet{fall}{red}  

% A counter, since TikZ is not clever enough (yet) to handle
% arbitrary angle systems.
\newcount\mycount

\begin{tikzpicture}[transform shape]
  \tikzstyle{every day}=[anchor=mid,font=\fontsize{6}{6}\selectfont]
  \node{\normalsize\the\year};
  \foreach \month/\monthcolor in
    {1/winter,2/winter,3/spring,4/spring,5/spring,6/summer,
     7/summer,8/summer,9/fall,10/fall,11/fall,12/winter}
  {
    % Computer angle:
    \mycount=\month
    \advance\mycount by -1
    \multiply\mycount by 30
    \advance\mycount by -90

    % The actual calendar
    \calendar at (\the\mycount:6.4cm)
    [
      dates=\the\year-\month-01 to \the\year-\month-last,
    ]
    if (day of month=1) {\color{\monthcolor}\tikzmonthcode}
    if (Sunday) [red]
    if (all) 
    {
      % Again, compute angle
      \mycount=1
      \advance\mycount by -\pgfcalendarcurrentday
      \multiply\mycount by 11
      \advance\mycount by 90
      \pgftransformshift{\pgfpointpolar{\mycount}{1.4cm}}
    };
  }
\end{tikzpicture}
\end{codeexample}



%%% Local Variables: 
%%% mode: latex
%%% TeX-master: "pgfmanual-pdftex-version"
%%% End: 

% Copyright 2008 by Till Tantau
%
% This file may be distributed and/or modified
%
% 1. under the LaTeX Project Public License and/or
% 2. under the GNU Free Documentation License.
%
% See the file doc/generic/pgf/licenses/LICENSE for more details.


\section{Chains}

\label{section-chains}

\begin{tikzlibrary}{chains}
  This library defines options for creating chains.
\end{tikzlibrary}


\subsection{Overview}

Chains provide a convenient way of creating sequences of nodes that
are connected to each other in a ``chainwise fashion.'' More
generally, they can be used to position nodes of a branching network
in a systematic manner. An alternative way is to use matrices, see
Section~\ref{section-matrices}, but chains can also be used 
to describe the connections between nodes that have already been
connected using, say, a matrix. 

The basic idea behind a chain is the following: A chain is a
sequence of nodes where each node is placed relative to the
preceding one. Typically, each node is joined to the preceding one
using an edge.

Chains can branch and join once more, which allows you to create
sophisticated networks.


\subsection{Creating a Chain}

Typically, you construct one chain at a time, but it is
permissible to have construct multiple chains simultaneously. In this
case, the chain must be named differently and you must specify for
each node which chain it belongs to.

The first step toward creating a chain is to use the |start chain|
key.

\begin{key}{/tikz/start chain=\opt{\meta{chain name}}\opt{\meta{direction}}}
  This key should, but need not, be given as an option to a scope
  enclosing all nodes of the chain. Typically, this will be a |scope|
  or the whole |tikzpicture|, but it might just be a path on which all
  nodes are placed.

  The key starts a chain named \meta{chain name}; if the chain already
  exists, it is ``restarted.'' If no \meta{chain name} is given, the
  default value |chain| will be used instead. Chains are started
  globally, that is, it is possible to continue a chain after a scope
  has ended.

  The \meta{direction} is used to determine the placement rule for
  nodes on the chain. If it is omitted, the current value of the
  following key is used:
  \begin{key}{/tikz/chain default direction=\meta{direction}
      (initially going right)}
    This \meta{direction} is used in a |chain| option, if no other
    \meta{direction} is specified.
  \end{key}

  The \meta{direction} can have two different forms:
  \declare{|going |\meta{options}} or
  \declare{|placed |\meta{options}}. The effect of these rules will be
  explained in the description of the |on chain| option. Right now,
  just remember that the \meta{direction} you provide with the |chain|
  option applies to the whole chain.
  
  This option also sets the following key to \meta{chain name} (or to
  the default name |chain|, if no \meta{chain name} is given), but you
  can also set this key to another value later on yourself:
  \begin{key}{/tikz/chain=\meta{name}}
    This key (locally) keeps track of the current chain that is being
    constructed. This name will be used by other options like
    |on chain| if no chain name is given.

    So: If you do not provide a \meta{chain name} for the
    |start chain| option, |chain| is used as the name, while if you do not
    provide a \meta{chain name} for the |on chain| command, the value
    of the |chain| key is used.
  \end{key}
  Other than this, this key has no further effect. In particular, to
  place nodes on the chain, you must use the |on chain| option,
  described next.

\begin{codeexample}[]
\begin{tikzpicture}[start chain]
  % The chain is called just "chain"
  \node [on chain] {A};
  \node [on chain] {B};
  \node [on chain] {C};
\end{tikzpicture}
\end{codeexample}

\begin{codeexample}[]
\begin{tikzpicture}
  % Same as above, using the scope shorthand
  { [start chain]
    \node [on chain] {A};
    \node [on chain] {B};
    \node [on chain] {C};
  }
\end{tikzpicture}
\end{codeexample}

\begin{codeexample}[]
\begin{tikzpicture}[start chain=1 going right,
                    start chain=2 going below,
                    node distance=5mm,
                    every node/.style=draw]
  \node [on chain=1] {A};
  \node [on chain=1] {B};
  \node [on chain=1] {C};

  \node [on chain=2] at (0,-1) {0};
  \node [on chain=2] {1};
  \node [on chain=2] {2};
  
  \node [on chain=1] {D}; 
\end{tikzpicture}
\end{codeexample}
\end{key}


\subsection{Nodes on a Chain}

\begin{key}{/tikz/on chain=\opt{\meta{chain name}}\opt{\meta{direction}}}
  This key should be given as an option to a node. When the option is
  used, the \meta{chain name} must previously have been declared for
  the current scope using the |start chain| option. If \meta{chain name} is
  the empty string, the current value of the key |chain| is used. 

  The \meta{direction} part is optional. If present it overrides the
  \meta{direction} used for this node, otherwise the \meta{direction}
  that was given to the original |start chain| option is used.

  The effects of this option are the following:
  \begin{enumerate}
  \item An internal counter (there is one, local, counter
    for each chain) is increased. This counter reflects the current
    number of the node in the chain, where the first node is node 1,
    the second is node 2, and so on.

    This value of this internal counter is globally stored in the
    macro \declare{|\tikzchaincount|}.
  \item If the node does not yet have a name, (having been given using
    the |at| option or the at-syntax), the name of the node is set to
    \meta{chain name}|-|\meta{value of the internal chain
      counter}. For instance, if the chain is called |nums|, the first
    node would be named |nums-1|, the second |nums-2|, and so on. For
    the default chain name |chain|, the first node is named |chain-1|,
    the second |chain-2|, and so on.
  \item Independently of whether the name has been provided
    automatically or via the |at| option, the name of the node is
    globally stored in the macro \declare{|\tikzchaincurrent|}.
  \item Except for the first node, the macro
    \declare{|\tikzchainprevious|} is now globally set to the name of
    the node of the previous node on the chain. For this first node,
    this macro is globally set to the empty string.
  \item Except possibly for the first node of the chain, the placement
    rule is now executed. The placement rule is just a \tikzname\ option
    that is applied automatically to each node on the chain. Depending
    on the form of the \meta{direction} parameter (either the locally
    given one or the one given to the |chain| option), different
    things happen.

    First, it makes a difference whether the \meta{direction} starts
    with |going| or with |placed|. The difference is that in the first
    case, the placement rule is not applied to the first node of the
    chain, while in the second case the placement rule is applied also
    to this first node. The idea is that a |going|-direciton indicates
    that we are ``going somewhere relative to the previous node''
    whereas a |placed| indicates that we are ``placing nodes according
    to their number.''

    Independently of which form is used, the \meta{text} inside
    \meta{direction} that follows |going| or |placed| (separated by a
    compulsory space) can have two different effects:
    \begin{enumerate}
    \item If it contains an equal sign, then this \meta{text} is used
      as the placement rule, that is, it is simply executed.  
    \item If it does not contain an equal sign, then
      \meta{text}|=of \tikzchainprevious| is used as the placement
      rule. 
    \end{enumerate}

    Note that in the first case, inside the \meta{text} you have
    access to |\tikzchainprevious| and |\tikzchaincount| for doing
    your positioning calculations.
    complicated placings.
\begin{codeexample}[]
\begin{tikzpicture}[start chain=circle placed {at=(\tikzchaincount*30:1.5)}]
  \foreach \i in {1,...,10}
    \node [on chain] {\i};
  
  \draw (circle-1) -- (circle-10);
\end{tikzpicture}
\end{codeexample}
  \item
    The following style is executed:
    \begin{stylekey}{/tikz/every on chain}
      This key is executed for every node of a chain, including the
      first one. 
    \end{stylekey}
  \end{enumerate}

  Recall that the standard replacement rule has a form like
  |right=of (\tikzchainprevious)|. This means that each 
  new node is placed to the right of the previous one, spaced by the
  current value of |node distance|.
\begin{codeexample}[]
\begin{tikzpicture}[start chain,node distance=5mm]
  \node [draw,on chain] {};
  \node [draw,on chain] {Hallo};
  \node [draw,on chain] {Welt};
\end{tikzpicture}
\end{codeexample}

  The optional \meta{direction} allows us to temporarily change the
  direction in the middle of a chain:
\begin{codeexample}[]
\begin{tikzpicture}[start chain,node distance=5mm]
  \node [draw,on chain] {Hello};
  \node [draw,on chain] {World};
  \node [draw,on chain=going below] {,};
  \node [draw,on chain] {this};
  \node [draw,on chain] {is};
\end{tikzpicture}
\end{codeexample}

  You can also use more complicated computations in the \meta{direction}:
\begin{codeexample}[]
\begin{tikzpicture}[start chain=going {at=(\tikzchainprevious),shift=(30:1)}]
  \draw [help lines] (0,0) grid (3,2);
  \node [draw,on chain] {1};
  \node [draw,on chain] {Hello};
  \node [draw,on chain] {World};
  \node [draw,on chain] {.};
\end{tikzpicture}
\end{codeexample}
\end{key}

\begin{key}{/tikz/redirect chain=\opt{\meta{chain}}\meta{new direction}}
  This option allows you to change the \meta{direction} of the
  \meta{chain}. For all subsequent nodes on the \meta{chain}, the
  \meta{new direction} is used (except if they, in turn, locally set
  their own direction). If the \meta{chain} is omitted, the value of
  the key |current chain| is used.  
\begin{codeexample}[]
\begin{tikzpicture}[start chain=going right,node distance=5mm]
  \node [draw,on chain] {Hello};
  \node [draw,on chain] {World};
  \node [draw,redirect chain=going below,on chain] {,};
  \node [draw,on chain] {this};
  \node [draw,on chain] {is};
\end{tikzpicture}
\end{codeexample}
\end{key}

\begin{command}{\chainin}
  Description missing...
\end{command}


\subsection{Joining Nodes on a Chain}

\begin{key}{/tikz/join=\opt{|with |\meta{with} }\opt{|by |\meta{options}}}
  When this key is given to any node on a chain (except possibly for
  the first node), an |edge| command is added after the node. The
  |with| part specifies which node should be used for the start point
  of the edge; if the |with| part is omitted, the |\tikzchainprevious|
  is used. This |edge| command gets the \meta{options} as parameter
  and the current node as its target. If there is no
  previous node and no |with| is given, no |edge| command gets
  executed.  
  \begin{stylekey}{/tikz/every join}
    This style is executed each time this command is used.
  \end{stylekey}

  Note that is makes sense to call this option several times for a
  node, in order to connect it to several nodes.
\begin{codeexample}[]
\begin{tikzpicture}[start chain,node distance=5mm,
                    every join/.style={->,red}]
  \node [draw,on chain,join] {};
  \node [draw,on chain,join] {Hallo};
  \node [draw,on chain,join] {Welt};
  \node [draw,on chain=going below,
         join,join=with chain-1 by {blue,<-}] {foo};
\end{tikzpicture}
\end{codeexample}
\end{key}


\subsection{Branches}

Description missing...

\subsection{Examples}

Here is a final example of a chain ``in action.''

\begin{codeexample}[]
\begin{tikzpicture}[
    >=stealth,
    minimum size=6mm,
    terminal/.style={rectangle,rounded corners=3mm,draw,fill=white,thick},
    nonterminal/.style={rectangle,draw,fill=white,thick},
    node distance=3mm]
    
  \begin{scope}[start chain,
                every node/.style={on chain},
                terminal/.append style={join=by {->,shorten >=1pt}},
                nonterminal/.append style={join=by {->,shorten >=1pt}},
                support/.style={coordinate,join}]
    \node [support]             (start)        {};
    \node [nonterminal]                        {unsigned integer};
    \node [support]             (after ui)     {};
    \node [terminal]                           {.};
    \node [support]             (after dot)    {};
    \node [terminal]                           {digit};
    \node [support]             (after digit)  {};
    \node [support]             (skip)         {};    
    \node [support]             (before E)     {};
    \node [terminal]                           {E};
    \node [support]             (after E)      {};
    \node [support,xshift=5mm]  (between)      {};
    \node [support,xshift=5mm]  (before last)  {};
    \node [nonterminal]                        {unsigned integer};
    \node [support]             (after last)   {};
    \node [join=by ->]          (end)          {};
  \end{scope}
  \node (plus)  [terminal,above=of between] {$+$};
  \node (minus) [terminal,below=of between] {$-$};

  \begin{scope}[->,shorten >=1pt,rounded corners]
    \draw (after ui)    -- +(0,.7)  -| (skip);
    \draw (after digit) -- +(0,-.7) -| (after dot);
    \draw (before E)    -- +(0,-1) -| (after last);
    \draw (after E)     |- (plus);
    \draw (plus)        -| (before last);
    \draw (after E)     |- (minus);
    \draw (minus)       -| (before last);
  \end{scope}
\end{tikzpicture}
\end{codeexample}


% Copyright 2008 by Mark Wibrow
%
% This file may be distributed and/or modified
%
% 1. under the LaTeX Project Public License and/or
% 2. under the GNU Free Documentation License.
%
% See the file doc/generic/pgf/licenses/LICENSE for more details.

\section{Decoration Library}
\label{section-library-decorations}

\begin{pgflibrary}{decorations}
  This library package defines a number of decorations. Also, the
  \tikzname\ version of the library is needed in order to use
  decorations in \tikzname\ at all.
\end{pgflibrary}

Parts of this section are still missing....


\subsection{Overview}

The decoration library defines a number of (more or less) useful
decorations that can be applied to paths. The usage of decorations is
not covered in the present section, please consult 
Sections~\ref{section-tikz-snakes-and-decorations}, which explains how
decorations are used in \tikzname, and
\ref{section-base-snakes-and-decorations}, which  explains how new
decorations can be defined. 

\subsubsection{Classification of Decorations}

The present library is structured according to the basic effect the
decorations have on the path:
\begin{enumerate}
\item \emph{Path deforming} decorations deform the path in the sense
  that what used to be a straight  line might afterwards be a squiggly
  line or might have bumps. However, a line is still and a line and
  path deforming decorations do not change the number of subpaths.
\item \emph{Path chopping} decorations deform a path and,
  additionally, may break up a line into numerous subpath. An example
  is a decoration that replaces a line by, say, little triangles.
\item \emph{Path removing} decorations completely remove the
  path. Thus, they have no effect on the main path that is being
  constructed. However, they typically have numerous \emph{side
    effects}. For instance, they might ``write some text'' along the
  (removed) path or they might place nodes along this path. Note that
  for such decorations the path usage command for the main path have
  no influence on how the decoration looks like.
\end{enumerate}

Furthermore, at the end of this section some fractal decorations are
documented separately (even though they are either path deforming or
path chopping).


\subsubsection{Decoration Options}

The decorations are influenced by a number of parameters that can be
set using the |decoration options| option. These parameters are
typically shared between different decorations. In the following, the
general options are documented, special-purpose keys are documented
with the decoration that uses it.

\begin{key}{/pgf/decoration options/amplitude=\meta{dimension} (initially 2.5pt)}
  This key determines the ``desired height'' (or amplitude) of 
  decorations for which this makes sense. For instance, the initial
  value of |2.5pt| means that deforming decorations should deform a
  path by up to 2.5pt away from the original path.
\end{key}

\begin{key}{/pgf/decoration options/meta-amplitude=\meta{dimension} (initially 2.5pt)}
  This key determines the amplitude for a meta-decoration. 
\end{key}

\begin{key}{/pgf/decoration options/segment length=\meta{dimension} (initially 10pt)}
  Many decorations are made up of small segments. This key determines
  the desired length of such segments. 
\end{key}

\begin{key}{/pgf/decoration options/meta-segment length=\meta{dimension} (initially 1cm)}
  This determined the length of the meta-segments from which a
  meta-snake is made up.
\end{key}

\begin{key}{/pgf/decoration options/angle=\meta{degree} (initially 45)}
  The way some decorations look like depends on a configurable angle. For
  instance, a |wave| decoration consists of arcs and the opening angle
  of these arcs is given by the |angle|.
\end{key}

\begin{key}{/pgf/decoration options/aspect=\meta{factor} (initially 0.5)}
  For some decorations there is a natural aspect ratio. For instance,
  for a |brace| decoration the aspect ratio determines where the brace
  point will be.
\end{key}

\begin{key}{/pgf/decoration options/shape width=\meta{dimension}
    (initially \normalfont same as |amplitude|)}
  For decorations that are made up from shapes, this key determines
  the (desired) width of these shapes.
\end{key}

\begin{key}{/pgf/decoration options/text=\meta{text}
    (initially \normalfont empty)}
  For decorations that need text, this option sets the text.
\end{key}

\begin{key}{/pgf/decoration options/text color=\meta{color}
    (initially \normalfont black)}
  For decorations that need colored text, this option sets this color.
\end{key}

\begin{key}{/pgf/decoration options/shape=\meta{shape name} (initially circle)}
  For decorations that use shapes, this is the shape name that is used.
\end{key}

\begin{key}{/pgf/decoration options/anchor=\meta{anchor}    (initially center)}
  For decorations that use shapes and that need to anchor then, this
  is the anchor that is used.
\end{key}



\subsection{Path Deforming Decorations}

A \emph{path deforming decorations} deform the to-be-decorated
path. This means that what used to be a straight line might afterwards
be a snaking curve and have bumps. However, a line is still and a line
and path deforming decorations do not change the number of
subpaths. For instance, if the path used to consist of two circles and
an open arc, the path will after the decoration process still consist
of two closed subpath and one open subpath.


\subsubsection{Deformation by Straight Lines}

The following deformations use only straight lines in order to deform
the paths.


\begin{decoration}{lineto}
  This decoration replaces the path by straight lines. For each curve,
  the path simply goes directly from the start point to the end point.
  In the following example, the arc actually consist of two
  subcurves. 
\begin{codeexample}[]
\begin{tikzpicture}[decoration=lineto]
  \draw [help lines] grid (3,2);
  \draw [decorate] (0,0) -- (3,1) arc (0:180:1.5 and 1);
\end{tikzpicture}
\end{codeexample}
\end{decoration}


\begin{decoration}{line zigzag}
  This (meta-)decoration decorates the path by alternating between 
  |line along| and |zigzag| decorations. It always finishes
  with the |line along| decoration. The following parameters influence
  the decoration:
  \begin{itemize}
  \item |amplitude|
    determines how much the zig-zag lines raises above and falls below
    a straight line to the target point.
  \item |segment length|
    determines the length of a complete ``up-down'' cycle.
  \item	|meta-segment length|
    determines the length of the |line along| and the |zigzag| decorations.
  \end{itemize}

\begin{codeexample}[]
\begin{tikzpicture}[decoration=line zigzag,
                    decoration options={meta-segment length=1.1cm}]
  \draw [help lines] grid (3,2);
  \draw [decorate] (0,0) -- (3,1) arc (0:180:1.5 and 1);
\end{tikzpicture}
\end{codeexample}
\end{decoration}


\begin{decoration}{random steps}
  This snake consists of straight line segments. The line segments
  head towards the target, but each step is randomly shifted a little
  bit. The following parameters influence the snake:
  \begin{itemize}
  \item |segment length|
    determines the basic length of each step.
  \item |amplitude|
    The end of each step is perturbed both in $x$- and in
    $y$-direction by two values drawn uniformly from the interval
    $[-d,d]$, where $d$ is the value of |amplitude|.
  \end{itemize}
\begin{codeexample}[]
\begin{tikzpicture}[decoration=random steps,
                    decoration options={segment length=2mm}]
  \draw [help lines] grid (3,2);
  \draw [decorate] (0,0) -- (3,1) arc (0:180:1.5 and 1);
\end{tikzpicture}
\end{codeexample}
\end{decoration}


\begin{decoration}{saw}
  This decoration looks like the blade of a saw. The following parameters
  influence the decoration:
  \begin{itemize}
  \item |amplitude|
    determines how much each spike raises above the straight line.
  \item |segment length|
    determines the length each spike.
  \end{itemize}
\begin{codeexample}[]
\begin{tikzpicture}[decoration=saw]
  \draw [help lines] grid (3,2);
  \draw [decorate] (0,0) -- (3,1) arc (0:180:1.5 and 1);
\end{tikzpicture}
\end{codeexample}
\end{decoration}


\begin{decoration}{zigzag}
  This decoration looks like a zig-zag line. The following parameters
  influence the decoration:
  \begin{itemize}
  \item |amplitude|
    determines how much the zig-zag lines raises above and falls below
    a straight line to the target point.
  \item |segment length|
    determines the length of a complete ``up-down'' cycle.
  \end{itemize}
\begin{codeexample}[]
\begin{tikzpicture}[decoration=zigzag]
  \draw [help lines] grid (3,2);
  \draw [decorate] (0,0) -- (3,1) arc (0:180:1.5 and 1);
\end{tikzpicture}
\end{codeexample}
\end{decoration}




\subsubsection{Deformation by Curved Lines}


\begin{decoration}{bent}
  This decoration adds a slightly bent line from the start to the
  target. The amplitude of the bent is given |amplitude|
  (an amplitude of zero gives a straight line). 
  \begin{itemize}
  \item |amplitude|
    determines the amplitude of the bent.
  \item |aspect|
    determines how tight the bent is. A good value is around |0.3|. 
  \end{itemize}
  Note that this decoration makes only little sense for curves. You
  should apply it only to straight lines.
\begin{codeexample}[]
\begin{tikzpicture}[decoration=bent]
  \draw [help lines] grid (3,2);
  \draw [decorate] (0,0) -- (3,1) -- (1.5,2) -- (0,1);
\end{tikzpicture}
\end{codeexample}
\begin{codeexample}[]
\begin{tikzpicture}[decoration options={aspect=.3},decoration=bent]
  \node[circle,draw] (A) at (.5,.5) {A};
  \node[circle,draw] (B) at (3,1.5) {B};
  \draw[->,decorate] (A) -- (B);
  \draw[->,decorate] (B) -- (A);

  \draw [decorate] (0,0) rectangle (3.5,2);
\end{tikzpicture}
\end{codeexample}
\end{decoration}


\begin{decoration}{bumps}
  This decoration replaces the path by little half ellipses. The
  following parameters influence itL
  \begin{itemize}
  \item |amplitude|
    determines the height of the half ellipse.
  \item |segment length|
    determines the width of the half ellipse.
  \end{itemize}
\begin{codeexample}[]
\begin{tikzpicture}[decoration=bumps]
  \draw [help lines] grid (3,2);
  \draw [decorate] (0,0) -- (3,1) arc (0:180:1.5 and 1);
\end{tikzpicture}
\end{codeexample}
\end{decoration}


\begin{decoration}{coil}
  This decoration replaces the path by a coiled line. To understand how this works,
  imagine a three-dimensional spring. The spring's axis points along
  the path toward the target. Then, we ``view'' the spring from a
  certain angle. If we look ``straight from the side'' we will see a
  perfect sine curve, if we look ``more from the front'' we will see a
  coil. The following parameters influence the snake:  
  \begin{itemize}
  \item |amplitude|
    determines how much the coil rises above the path and falls below
    it. Thus, this is the radius of the coil.
  \item |segment length|
    determines the distance between two consecutive ``curls.'' Thus,
    when the spring is see ``from the side'' this will be the wave
    length of the sine curve. 
  \item |aspect|
    determines the ``viewing direction.'' A value of |0| means
    ``looking from the side'' and a value of |0.5|, which is the
    default, means ``look more from the front.'' 
  \end{itemize}
\begin{codeexample}[]
\begin{tikzpicture}[decoration=coil]
  \draw [help lines] grid (3,2);
  \draw [decorate] (0,0) -- (3,1) arc (0:180:1.5 and 1);
\end{tikzpicture}
\end{codeexample}
\begin{codeexample}[]
\begin{tikzpicture}[decoration=coil,decoration options=
                    {aspect=0.3,segment length=3mm,amplitude=3mm}]
  \draw [help lines] grid (3,2);
  \draw [decorate] (0,0) -- (3,1) arc (0:180:1.5 and 1);
\end{tikzpicture}
\end{codeexample}
\end{decoration}



\begin{decoration}{line along}
  This decoration simply yields a line following the original
  path. This means that (ideally) it does not change the path. In
  reality, due to the internals of how decorations are implemented,
  this decoration actually replaces the path by numerous small
  straight lines.

  This decoration is useful only in conjunction with
  meta-decorations. 

\begin{codeexample}[]
\begin{tikzpicture}[decoration=line along]
  \draw [help lines] grid (3,2);
  \draw [decorate] (0,0) -- (3,1) arc (0:180:1.5 and 1);
\end{tikzpicture}
\end{codeexample}
\end{decoration}



\begin{decoration}{snake}
  This decoration replaces the path by a line that looks like a snake
  seen from above. More precisely, the snake is a sine wave with a
  ``softened'' start and ending. The following parameters influence
  the snake: 
  \begin{itemize}
  \item |amplitude|
    determines the sine wave's amplitude.
  \item |segment length|
    determines the sine wave's wave length.
  \end{itemize}
\begin{codeexample}[]
\begin{tikzpicture}[decoration=snake]
  \draw [help lines] grid (3,2);
  \draw [decorate] (0,0) -- (3,1) arc (0:180:1.5 and 1);
\end{tikzpicture}
\end{codeexample}
\end{decoration}



  
\subsection{Path Chopping Decorations}

Still missing...

\subsection{Path Removing Decorations}

Still missing...

\subsection{Fractal Decorations}

\begin{decoration}{Koch curve type 1}
  This decoration replaces a straight line by a ``rectangular bump.''
  By repeatedly applying this replacement, different levels of the
  Koch curve fractal can be created. Its Hausdorff dimension is $\log
  5/\log 3$.
\begin{codeexample}[]
\begin{tikzpicture}[decoration=Koch curve type 1]
  \draw decorate{ (0,0) -- (3,0) };
  \draw decorate{ decorate{ (0,-1.5) -- (3,-1.5) }};
  \draw decorate{ decorate{ decorate{ (0,-3) -- (3,-3) }}};
\end{tikzpicture}
\end{codeexample}
\end{decoration}


\begin{decoration}{Koch curve type 2}
  This decoration replaces a straight line by a ``rectangular sine.''
  Its Hausdorff dimension is $3/2$.
\begin{codeexample}[]
\begin{tikzpicture}[decoration=Koch curve type 2]
  \draw decorate{ (0,0) -- (3,0) };
  \draw decorate{ decorate{ (0,-2) -- (3,-2) }};
  \draw decorate{ decorate{ decorate{ (0,-4) -- (3,-4) }}};
\end{tikzpicture}
\end{codeexample}
\end{decoration}

\begin{decoration}{Koch snowflake}
  This decoration replaces a straight line by a ``line with a spike.''
  Koch's snowflake's Hausdorff dimension is $\log 4/\log 3$.
\begin{codeexample}[]
\begin{tikzpicture}[decoration=Koch snowflake]
  \draw decorate{ (0,0) -- (3,0) };
  \draw decorate{ decorate{ (0,-1) -- (3,-1) }};
  \draw decorate{ decorate{ decorate{ (0,-2) -- (3,-2) }}};
  \draw decorate{ decorate{ decorate{ decorate{ (0,-3) -- (3,-3) }}}};
\end{tikzpicture}
\end{codeexample}
\end{decoration}

\begin{decoration}{Cantor set}
  This decoration replaces a straight line by a ``line with a whole in
  the middle.'' The Hausdorff dimension of the Cantor set is $\log
  2/\log 3$. 
\begin{codeexample}[]
\begin{tikzpicture}[decoration=Cantor set,very thick]
  \draw decorate{ (0,0) -- (3,0) };
  \draw decorate{ decorate{ (0,-.5) -- (3,-.5) }};
  \draw decorate{ decorate{ decorate{ (0,-1) -- (3,-1) }}};
  \draw decorate{ decorate{ decorate{ decorate{ (0,-1.5) -- (3,-1.5) }}}};
\end{tikzpicture}
\end{codeexample}
\end{decoration}










\endinput





\begin{decoration}{border}
  This snake adds straight lines the path that are at a specific angle
  to the line toward the target. The idea is to add these little lines
  to indicate the ``border'' or an area. The following parameters
  influence the snake:  
  \begin{itemize}
  \item |/pgf/segment length|
    determines the distance between consecutive ticks.
  \item |/pgf/segment amplitude|
    determines the length of the ticks.
  \item |/pgf/segment angle|
    determines the angle between the ticks and the line toward the
    target. 
  \end{itemize}
\begin{codeexample}[]
\tikz{\draw (0,0) rectangle (3,1)
            [snake=border,segment angle=-45] (0,0) rectangle (3,1);}
\end{codeexample}
\end{decoration}


\begin{decoration}{brace}
  This snake adds a long brace to the path. The left and right end of
  the brace will be exactly on the start and endpoint of the
  snake. The following parameters influence the snake:  
  \begin{itemize}
  \item |/pgf/segment amplitude|
    determines how much the brace rises above the path.
  \item |/pgf/segment aspect|
    determines the fraction of the total length where the ``middle
    part'' of the brace will be.  
  \end{itemize}
\begin{codeexample}[]
\tikz{\draw[snake=brace,segment aspect=0.25] (0,0) -- (3,0);}
\end{codeexample}
\end{decoration}


\begin{decoration}{expanding waves}
  This snake adds arcs to the path that get bigger along the line
  towards the target. The following parameters influence the snake:
  \begin{itemize}
  \item |/pgf/segment length|
    determines the distance between consecutive arcs.
  \item |/pgf/segment angle|
    determines the opening angle below and above the path. Thus, the
    total opening angle is twice this angle.
  \end{itemize}
\begin{codeexample}[]
\tikz{\draw[snake=expanding waves] (0,0) -- (3,0);}
\end{codeexample}
\end{decoration}



\begin{decoration}{ticks}
  This snake adds straight lines  the path that are orthogonal to the
  line toward the target. The following parameters influence the snake: 
  \begin{itemize}
  \item |/pgf/segment length|
    determines the distance between consecutive ticks.
  \item |/pgf/segment amplitude|
    determines half the length of the ticks.
  \end{itemize}
\begin{codeexample}[]
\tikz{\draw[snake=ticks] (0,0) -- (3,0);}
\end{codeexample}
\end{decoration}

\begin{decoration}{triangles}
  This snake adds triangles to the path that point toward the
  target. The following parameters influence the snake: 
  \begin{itemize}
  \item |/pgf/segment length|
    determines the distance between consecutive triangles.
  \item |/pgf/segment amplitude|
    determines half the length of the triangle side that is orthogonal
    to the path.
  \item |/pgf/segment object length|
    determines the height of the triangle.
  \end{itemize}
\begin{codeexample}[]
\tikz{\draw[snake=triangles] (0,0) -- (3,0);}
\end{codeexample}
\end{decoration}

\begin{decoration}{crosses}
  This snake adds (diagonal) crosses to the path. The following
  parameters influence the snake:  
  \begin{itemize}
  \item |/pgf/segment length|
    determines the distance between consecutive crosses.
  \item |/pgf/segment amplitude|
    determines half the hieght of each cross.
  \item |/pgf/segment object length|
    determines width of each cross.
  \end{itemize}
\begin{codeexample}[]
\tikz{\draw[snake=crosses] (0,0) -- (3,0);}
\end{codeexample}
\end{decoration}


\begin{decoration}{waves}
  This snake adds arcs to the path that have a constant size. The
  following parameters influence the snake: 
  \begin{itemize}
  \item |/pgf/segment length|
    determines the distance between consecutive arcs.
  \item |/pgf/segment angle|
    determines the opening angle below and above the path. Thus, the
    total opening angle is twice this angle.
  \item |/pgf/segment amplitude|
    determines the radius of each arc.
  \end{itemize}
\begin{codeexample}[]
\tikz{\draw[snake=waves] (0,0) -- (3,0);}
\end{codeexample}
\end{decoration}


\begin{decoration}{shape snake}
  This snake adds a succession of shapes to the path. The shape must
  have been defined by |\pgfdeclareshape| and must have defined a 
  background path. Please note that the shapes in a snake are not 
  nodes. They cannot have text inside them, be named, or referred to. 
  The snake simply adds the background path of the shape to the ongoing 
  snaked path.

\begin{codeexample}[]
\tikzset{paint/.style={snake=shape snake, draw=#1!50!black, fill=#1!50}}
\begin{tikzpicture}
  \draw [shape snake shape=dart,      paint=red]    (0,1.5) -- (3,1.5);
  \draw [shape snake shape=diamond,   paint=green]  (0,1)   -- (3,1);
  \draw [shape snake shape=rectangle, paint=blue]   (0,0.5) -- (3,0.5);
  \draw [shape snake shape=circle,    paint=yellow] (0,0)   -- (3,0);
\end{tikzpicture}
\end{codeexample}

  All shapes are positioned by their center anchor (as this is the only
  anchor that all shapes must define). A shape is drawn at the start 
  point of the path and, if the distance between the shapes is 
  appropriate, at the end point of the path.
	
\begin{codeexample}[]
\begin{tikzpicture}[snake=shape snake, shape snake shape=regular polygon]
  \draw [help lines] grid (3,2);
  \draw [thick] (0,0) -- (2,2) (1,0) -- (3,0);
  \draw [very thick, red!50, snake, shape snake sep=.5cm]  (1,0) -- (3,0);
  \draw [very thick, blue!50, snake, shape snake sep=.5cm] (0,0) -- (2,2);
\end{tikzpicture}
\end{codeexample}

  Keys for cusomizing specific shapes can be specified (e.g., 
  |star points|, |cloud puffs|, |kite angles|, and so on). However, you
  should be aware that the size of each shape is enforced using a 
  coodinate transformation, which may mean that settings involving 
  angles and distances may not appear entirely accurate. More general
  options such as |inner sep| and |minimum size| will be ignored, 
  but transformations can be applied to each segment as described
  below.
  
\begin{codeexample}[]
\tikzset{
  paint/.style={snake=shape snake, draw=#1!50!black, fill=#1!50},
  my star/.style={shape snake shape=star, star points=#1}
}
\begin{tikzpicture}[shape snake sep=.5cm, shape snake start size=.5cm]
  \draw [my star=9, paint=red]                            (0,1.5) -- (3,1.5);
  \draw [my star=5, paint=blue]                           (0,.75) -- (3,.75);
  \draw [my star=5, paint=yellow, shape border rotate=30] (0,0) -- (3,0);
\end{tikzpicture}
\end{codeexample}

  There are various keys to control the drawing of the shape snake.

\begin{key}{/pgf/shape snake shape=\meta{shape} (initially circle)}
  \keyalias{tikz}
  Set the shape for the snake. If \meta{shape} is defined in a shape
  library which has not been loaded then an error will result.
\end{key}

\begin{key}{/pgf/shape snake sep=\meta{spacing} (initially {.25cm, between centers})}
  \keyalias{tikz}
  Set the spacing between the shapes on the snaked path. This can be
  just a distance on its own, but the additional keywords 
  |between centers|, and |between borders| (which must be preceded by a 
  comma), specify that the distance	is between the center anchors of 
  the shapes or between the edges of the \emph{boundaries} of
  the shape borders.
	
\begin{codeexample}[]
\begin{tikzpicture}[snake=shape snake, shape snake start size=.5cm,
    paint/.style={snake, draw=#1!50!black, fill=#1!50},
    shape snake shape=signal, signal from=west, signal to=east]
  \draw [help lines] grid (3,2);
  \draw [paint=red, shape snake sep=.5cm]                    (0,0) -- (3,0);
  \draw [paint=green, shape snake sep={1cm, between center}] (0,1) -- (3,1);
  \draw [paint=blue, shape snake sep={1cm, between borders}] (0,2) -- (3,2);
\end{tikzpicture}
\end{codeexample}

\end{key}


  
\begin{key}{/pgf/shape snake evenly spread=\meta{number}}
  \keyalias{tikz}
  This key overides the |shape snake sep| key and forces the snake to
  fit \meta{number} shapes evenly across the path. 
  If \meta{number} is less than |1|, then no shapes will be drawn. 
  If \meta{number} equals |1|, then one shape is drawn in the middle 
  of the path. 
  The additional keywords |by centers| (the default, if no keyword is
  specified) and |by borders| can be used (both preceded by a comma), 
  to specify how the distance between shapes is determined. These
  keywords will only have a noticable effect if the snake is scaled.
  
\begin{codeexample}[]
\tikzset{my snake/.style={%
  snake=shape snake, shape snake shape=rectangle, shape snake start size=.5cm,
  draw=#1!50!black, fill=#1!50}
}
\begin{tikzpicture}
  \fill [shape snake evenly spread={5, by borders}, 
         my snake=green, shape snake scaled]           (0,2)   -- (3,2);
  \fill [shape snake evenly spread={5, by centers},
         my snake=blue, shape snake scaled]            (0,1.5) -- (3,1.5);   
  \fill [my snake=red, shape snake evenly spread=5]    (0,1)   -- (3,1);
  \fill [my snake=orange, shape snake evenly spread=4] (0,.5)  -- (3,.5);
  \fill [my snake=gray, shape snake evenly spread=1]   (0,0)   -- (3,0);
\end{tikzpicture}
\end{codeexample}

\end{key}

\begin{key}{/pgf/shape snake sloped=\meta{boolean} (default true)}
  \keyalias{tikz}
  By default, shapes are rotated to the slope of the snaked path. If 
  \meta{boolean} is the value |false|, then this rotation is turned 
  off. Internally this sets the \TeX-if |\ifpgfshapesnakesloped|
  appropriately.

\begin{codeexample}[]
\tikzset{
  shape snake start width=.65cm, shape snake start height=.45cm,
  shape snake shape=isosceles triangle, shape snake sep=.75cm,
  paint/.style={snake, draw=#1!50!black, fill=#1!50}
}
\begin{tikzpicture}[snake=shape snake]
  \draw [help lines] grid (3,2);
  \draw [paint=red] (0,0) -- (2,2);
  \draw [paint=blue, shape snake sloped=false] (1,0) -- (3,2);
\end{tikzpicture}
\end{codeexample}

\end{key}%

It is possible to scale the width and height of the shapes across the
length of the snaked path. The shapes are scaled between the starting
size and the ending size. The following keys customize the way the
snake shapes are scaled:
	
\begin{key}{/pgf/shape snake scaled=\meta{boolean} (default true)}
  \keyalias{tikz}
  Internally this sets the \TeX-if |\ifpgfshapesnakescaled| 
  appropriately.
	
\begin{codeexample}[]
\tikzset{
  bigger/.style={shape snake start size=.125cm, shape snake end size=.5cm},
  smaller/.style={shape snake start size=.5cm, shape snake end size=.125cm},
  shape snake sep={.25cm, between borders}
}
\begin{tikzpicture}[snake=shape snake]
  \draw [help lines] grid (3,2);
  \fill [snake, shape snake scaled, bigger, red!50]   (0,1) -- (3,2);
  \fill [snake, shape snake scaled, smaller, blue!50] (0,0) -- (3,1);
\end{tikzpicture}
\end{codeexample}

\end{key}

\begin{key}{/pgf/shape snake start width=\meta{length} (initially .25cm)}
  \keyalias{tikz}
  The starting width of the shape.
\end{key}%

\begin{key}{/pgf/shape snake start height=\meta{length} (initially .25cm)}
  \keyalias{tikz}
  The starting height of the shape.
\end{key}%

\begin{stylekey}{/pgf/shape snake start size=\meta{length}}
  \keyalias{tikz}
  Set both the the start height and start width simultaneously.
\end{stylekey}%

\begin{key}{/pgf/shape snake end width=\meta{length} (initially .125cm)}
  \keyalias{tikz}
  The recommended ending width of the shape. Note, that this is the
  width that a shape will take only if it is drawn exactly at the end
  of the path.
		
\begin{codeexample}[]
\tikzset{
  bigger/.style={shape snake start size=.25cm, shape snake end size=1cm},
  smaller/.style={shape snake start size=1cm, shape snake end size=.25cm},
  shape snake scaled, snake=shape snake
}
\begin{tikzpicture}
  \draw [help lines]grid(3,2);
  \fill [snake, bigger,  shape snake sep={.25cm, between borders}, blue!50] 
    (0,1.5) -- (3,1.5);
  \fill [snake, smaller, shape snake sep={1cm, between centers},   red!50]  
    (0,.5)  -- (3,.5);
  \draw [gray, dotted] (0,1.625) -- (3,2)    (0,1.375) -- (3,1) 
                       (0,1)     -- (3,.625) (0,0)     -- (3,.375); 
\end{tikzpicture}
\end{codeexample}

\end{key}%

\begin{key}{/pgf/shape snake end height=\meta{length}}
  \keyalias{tikz}
  The recommended ending height of the shape.
\end{key}%

\begin{stylekey}{/pgf/shape snake end size=\meta{length}}
  \keyalias{tikz}
  Set both the the end height and end width simultaneously.
\end{stylekey}

There is an additional TikZ key for the shape snake:

\begin{stylekey}{/tikz/shape snake tranform=\meta{keys}}
  \keyalias{tikz}
  This key parses \meta{keys}, which should be things like |rotate|, or
  |yshift|, and so on. The resulting transformation is applied to each 
  segment as it is drawn. It is analogous to the pgf command
  |\pgfsetsnakesegmenttransformation| (and in fact, uses it internally).

\begin{codeexample}[]
\tikzset{my snake/.style={%
  snake=shape snake, shape snake shape=rectangle, shape snake sep=.5cm,
  very thick, draw=#1!50}
}
\begin{tikzpicture}
  \draw [help lines] grid (3,2);
  \draw [thick] (0,0.5) -- (3,1.5);
  \draw [my snake=red,  shape snake transform={yshift=7.5pt}] 
     (0,0.5) -- (3,1.5);
  \draw [my snake=blue, shape snake transform={yshift=-7.5pt, rotate=45}] 
     (0,0.5) -- (3,1.5);
\end{tikzpicture}
\end{codeexample}
\end{stylekey}

\end{decoration}

\begin{decoration}{triangles}
	This decoration adds triangles to the path that point toward the
  target. The following parameters influence the decoration: 
  \begin{itemize}
  \item |/pgf/decoration segment length|
    determines the distance between consecutive triangles.
  \item |/pgf/decoration segment amplitude|
    determines half the length of the triangle side that is orthogonal
    to the path.
  \item |/pgf/decoration segment object length|
    determines the height of the triangle.
  \end{itemize}
\begin{codeexample}[]
\begin{tikzpicture}[decoration=triangles]
  \draw [help lines] grid (3,2);
  \draw [decorate] (0,0) .. controls (0,2) and (3,0) .. (3,2);
\end{tikzpicture}
\end{codeexample}
\end{decoration}

\begin{decoration}{crosses}
	This decoration adds (diagonal) crosses to the path. The following
  parameters influence the decoration:  
  \begin{itemize}
  \item |/pgf/decoration segment length|
    determines the distance between consecutive crosses.
  \item |/pgf/decoration segment amplitude|
    determines half the hieght of each cross.
  \item |/pgf/decoration segment object length|
    determines width of each cross.
  \end{itemize}
\begin{codeexample}[]
\begin{tikzpicture}[decoration=crosses]
  \draw [help lines] grid (3,2);
  \draw [decorate] (0,0) .. controls (0,2) and (3,0) .. (3,2);
\end{tikzpicture}
\end{codeexample}
\end{decoration}

\begin{decoration}{ticks}
  This decoration adds straight lines  the path that are orthogonal to 
  the line toward the target. The following parameters influence the 
  decoration: 
  \begin{itemize}
  \item |/pgf/decoration segment length|
    determines the distance between consecutive ticks.
  \item |/pgf/decoration segment amplitude|
    determines half the length of the ticks.
  \end{itemize}
\begin{codeexample}[]
\begin{tikzpicture}[decoration=ticks]
  \draw [help lines] grid (3,2);
  \draw [decorate] (0,0) .. controls (0,2) and (3,0) .. (3,2);
\end{tikzpicture}
\end{codeexample}
\end{decoration}

\begin{decoration}{text}
  This decoration decorates the path with text.
	
\begin{codeexample}[]
\catcode`\|12
\begin{tikzpicture}
  \draw [help lines] grid (3,2);
  \draw [red, dashed, postaction={decoration=text, decorate,
    decoration text={Some text along a curve}}] 
    (0,0) .. controls (0,2) and (3,0) .. (3,2);
\end{tikzpicture}
\end{codeexample}

  \pgfname{} ``does its best'' to typeset the text, however you
  should note the following points:
  \begin{itemize}
  \item
    Each character in the text is typeset in a separate |\hbox|. This
    means that if you want fancy things like kerning or ligatures you
    will have to manually annotate the characters in the decoration 
    text within a group, for example, |W{\kern-1ptA}TER|. 
  \item
    Each character is positioned using the center of its baseline. To
    move the text vertcally (relative to the path), the additional
    transform key should be used.
  \item
    No attempt is made to ensure characters do not overlap when
    the angle between segments is considerably less than 180\textdegree{}
    (this is tricky to do in \TeX{} without a huge processing
    overhead). In general this should not be too much of a problem, 
    but, once again, kerning can be used in most cases to overcome 
    any undesirable	effects.
  \item			
    It is only possible to typeset text in math mode under considerable
    restrictions. Math mode is entered and exited using any character	
    of category code 3 (e.g., in plain \TeX{} this is |$|). %$
    Math subscripts and superscripts need to be	contained within braces 
    (e.g., |{^y_i}|) as do commands like |\times| or |\cdot|. 
    However, even modestly complex mathematical	typesetting is unlikely 
    to be sucessful along a path (or even desirable).
  \item
    Some inaccuracies in positioning may be particularly apparent
    at subpath boundaries. This can (unfortunately) only be solved 
    on case by case basis	by individually kerning the offending 
    characters within a group.
  \end{itemize}
  
  The following keys are used by the |text| decoration:
  
  \begin{key}{/pgf/decoration text=\marg{text} (initially \char`\{\char`\})}
    Set the text to typeset along the curve. 
    Consecutive spaces are ignored, so |\ | (or |\space| in \LaTeX) 
    should be used to insert multiple spaces.	It is possible to
    format the text using normal formating commands, such as |\it|, |\bf|
    and |\color|, within customisable delimiters. Initially these
    delimiters are both {\tt\char`\|} (however, care will be needed 
    regarding	the category codes of delimiters --- see below). 

{\catcode`\|12
\begin{codeexample}[]
\catcode`\|12
\begin{tikzpicture}
  \draw [help lines] grid (3,2);
  \path [decorate, decoration=text,	
   decoration text={a big |\color{green}|green|| juicy apple.}] 
    (0,0) .. controls (0,2) and (3,0) .. (3,2);
\end{tikzpicture}
\end{codeexample}
}
  By following the first delimiter
  with |+|, the formatting commands are added to any exisiting 
  formatting.

{\catcode`\|12
\begin{codeexample}[]
\begin{tikzpicture}
  \draw [help lines] grid (3,2);
  \path [decorate, decoration=text,	
     decoration text={a |\large|big |+\bf\color{red}|red|| juicy apple.}] 
    (0,0) .. controls (0,2) and (3,0) .. (3,2);
\end{tikzpicture}
\end{codeexample}
}
	
  Internally, the text is stored in the macro |\pgfdecorationtext|. 
  Any characters that have not been typeset when the end of the 
  path has been reached will be stored in |\pgfdecorationrestoftext|.

\end{key}

{\catcode`\|12
\begin{key}{/pgf/decoration text format delimiters=\marg{before}\marg{after} (initially \char`\{|\char`\}\char`\{\char`\})}

  \catcode`\|13
	
  Set the characters that the text decoration will use to parse 
  formatting commands. 
  If \meta{after} is empty, then \meta{before} will be used for both
  delimiters.
  In general you should stick to characters	whose category codes are 
  |11| or |12|.
  As |+| is used to indicate that the specifed format commands 
  are added	to any exisiting ones, you should avoid using |+| as
  a delimiter. 

\begin{codeexample}[]
\begin{tikzpicture}
  \draw [help lines] grid (3,2);
  \path [decorate, decoration=text, decoration text format delimiters={[}{]}, 
  decoration text={A big [\color{red}]red[] and [\color{green}]green[] apple.}] 
    (0,0) .. controls (0,2) and (3,0) .. (3,2);
\end{tikzpicture}
\end{codeexample}
\end{key}
}

\begin{key}{/pgf/decoration text color=\meta{color} (initially black)}
  Set the color for the text.
\end{key}
\end{decoration}


% Copyright 2006 by Till Tantau
%
% This file may be distributed and/or modified
%
% 1. under the LaTeX Project Public License and/or
% 2. under the GNU Free Documentation License.
%
% See the file doc/generic/pgf/licenses/LICENSE for more details.


\section{Entity-Relationship Diagram Drawing Library}

\begin{tikzlibrary}{er}
  This packages provides styles for drawing entity-relationship
  diagrams. 
\end{tikzlibrary}

This library is intended to help you in creating E/R-diagrams. It defines
only very little new styles, but using these style |entity| instead of
saying |rectangle,draw| makes the code more expressive.


\subsection{Entities}

The package defines a simple style for drawing entities:

\begin{itemize}
  \itemstyle{entity}
  This style is to be used with nodes that represent entity types. It
  causes the node's shape to be set to a rectangle that is drawn and
  whose minimum size and width are set to sensible values.

  Note that this style is called |entity| despite the fact that it is
  to be used for nodes representing entity \emph{types} (the
  difference between an entity and an entity type is the same as the
  difference between an object and a class in object-oriented
  programming). If this bothers you, feel free to define a style
  |entity type| instead.
\begin{codeexample}[]
\begin{tikzpicture}[node distance=2cm]
  \node[entity] (sheep)                   {Sheep};
  \node[entity] (genome) [right of=sheep] {Genome};
\end{tikzpicture}
\end{codeexample}
  
  \itemstyle{every entity}
  This stype is envoked by the style |entity|. To change the
  appearance of entities, you can change this style.
\begin{codeexample}[]
\begin{tikzpicture}
  [node distance=2cm,
   every entity/.style={draw=blue!50,fill=blue!20,thick}]
  \node[entity] (sheep)                   {Sheep};
  \node[entity] (genome) [right of=sheep] {Genome};
\end{tikzpicture}
\end{codeexample}
\end{itemize}



\subsection{Relationships}

Relationships are drawn using styles that are very similar to the
styles for entities.

\begin{itemize}
  \itemstyle{relationship}
  This style works like |entity|, only it is to be used for
  relationships. Again, |relationship|s are actually relationship types. 
\begin{codeexample}[]
\begin{tikzpicture}
  \node[entity] (sheep)  at (0,0)   {Sheep};
  \node[entity] (genome) at (2,0)   {Genome};
  \node[relationship]    at (1,1.5) {has}
    edge (sheep) 
    edge (genome);
\end{tikzpicture}
\end{codeexample}
  \itemstyle{every relationship}
  works like |every entity|
\begin{codeexample}[]
\begin{tikzpicture}
  [every entity/.style={fill=blue!20,draw=blue,thick},
   every relationship/.style={fill=orange!20,draw=orange,thick,aspect=1.5}]
  \node[entity] (sheep)  at (0,0)   {Sheep};
  \node[entity] (genome) at (2,0)   {Genome};
  \node[relationship]    at (1,1.5) {has}
    edge (sheep) 
    edge (genome);
\end{tikzpicture}
\end{codeexample}
\end{itemize}



\subsection{Attributes}

\begin{itemize}
  \itemstyle{attribuate}
  This style is used to indicate that a node is an attribute. To
  connect an attribute to its entity, you can use, for example, the
  |child| command or the |pin| option. 
\begin{codeexample}[]
\begin{tikzpicture}
  \node[entity] (sheep)  {Sheep}
    child {node[attribute] {name}}
    child {node[attribute] {color}};
\end{tikzpicture}
\end{codeexample}
\begin{codeexample}[]
\begin{tikzpicture}[every pin edge/.style=draw]    
  \node[entity,pin={[attribute]60:name},pin={[attribute]120:color}] {Sheep};
\end{tikzpicture}
\end{codeexample}
  \itemstyle{key attribute}
  This style is intended for key attributes. By default, the will
  cause the attribute to be typeset in italics. Typcially, underlining
  is used instead, but that looks ugly and it is difficult to
  implement in \TeX.
  \itemstyle{every attribute}
  This style is used with every (key) attribute.
\begin{codeexample}[]
\begin{tikzpicture}
  [text depth=1pt,
   every attribute/.style={fill=black!20,draw=black},
   every entity/.style={fill=blue!20,draw=blue,thick},
   every relationship/.style={fill=orange!20,draw=orange,thick,aspect=1.5}]

  \node[entity] (sheep)  at (0,0)   {Sheep}
    child {node  [key attribute] {name}};
  \node[entity] (genome) at (2,0)   {Genome};
  \node[relationship]    at (1,1.5) {has}
    edge (sheep) 
    edge (genome);
\end{tikzpicture}
\end{codeexample}
\end{itemize}



%%% Local Variables: 
%%% mode: latex
%%% TeX-master: "pgfmanual-pdftex-version"
%%% End: 

% Copyright 2006 by Till Tantau
%
% This file may be distributed and/or modified
%
% 1. under the LaTeX Project Public License and/or
% 2. under the GNU Free Documentation License.
%
% See the file doc/generic/pgf/licenses/LICENSE for more details.


\section{Fading Library}
\label{section-library-fadings}

\begin{pgflibrary}{fadings}
  The package defines a number of fadings, see
  Section~\ref{section-tikz-transparency} for an introduction.  The
  \tikzname\ version defines special \tikzname\ commands for creating
  fadings. These commands are explained in
  Section~\ref{section-tikz-transparency}.   
\end{pgflibrary}

\newcommand\fadingindex[1]{%
  \index{#1@\protect\texttt{#1} fading}%
  \index{Fadings!#1@\protect\texttt{#1}}%
  \texttt{#1}& 
  \begin{tikzpicture}[baseline=5mm-.5ex]
    \fill [black!20] (0,0) rectangle (1,1);
    \path [pattern=checkerboard,pattern color=black!30] (0,0) rectangle (1,1);

    \fill [path fading=#1,blue] (0,0) rectangle (1,1);
  \end{tikzpicture} \\[4.5mm]
}

\noindent
\begin{tabular}{ll}
  \emph{Fading name} & \emph{Example (solid blue faded on checkerboard)} \\[1mm]
  \fadingindex{west}  
  \fadingindex{east}  
  \fadingindex{north}  
  \fadingindex{south} 
  \fadingindex{circle with fuzzy edge 10 percent} 
  \fadingindex{circle with fuzzy edge 15 percent} 
  \fadingindex{circle with fuzzy edge 20 percent} 
  \fadingindex{fuzzy ring 15 percent} 
\end{tabular}


%%% Local Variables: 
%%% mode: latex
%%% TeX-master: "pgfmanual-pdftex-version"
%%% End: 

% Copyright 2006 by Till Tantau
%
% This file may be distributed and/or modified
%
% 1. under the LaTeX Project Public License and/or
% 2. under the GNU Free Documentation License.
%
% See the file doc/generic/pgf/licenses/LICENSE for more details.


\section{Fitting Library}
\label{section-library-fit}

\begin{tikzlibrary}{fit}
  The library defines (currently only one) option for fitting a node
  so that it contains a set of coordinates.
\end{tikzlibrary}

When you load this library, the following option becomes available:

\begin{key}{/tikz/fit=\meta{coordinates}}
  This option must be given to a |node| path command. The
  \meta{coordinates} should be a sequence of \tikzname\ coordinates,
  one directly after the other without commas (like with the
  |plot coordinates| path operation).  For this sequence of
  coordinates, a minimal bounding box is computed that
  encompasses all the listed \meta{coordinates}. In principle (the
  details will be explained in a moment), things are now setup such
  that the text box of the node will be exactly this bounding box.

  Here is an example: We fit several points in a rectangular node. By
  setting the |inner sep| to zero, we see exactly the text box of the
  node. Then we fit these points again in circular node. Note how
  the circle encompasses exactly the same bounding box.
\begin{codeexample}[]
\begin{tikzpicture}[inner sep=0pt,thick,
                    dot/.style={fill=blue,circle,minimum size=3pt}]
  \draw[help lines] (0,0) grid (3,2);
  \node[dot] (a) at (1,1) {};
  \node[dot] (b) at (2,2) {};
  \node[dot] (c) at (1,2) {};
  \node[dot] (d) at (1.25,0.25) {};
  \node[dot] (e) at (1.75,1.5) {};

  \node[draw=red,   fit=(a) (b) (c) (d) (e)] {box};
  \node[draw,circle,fit=(a) (b) (c) (d) (e)] {};
\end{tikzpicture}  
\end{codeexample}

  Every time the |fit| option is used, the following style is also
  applied to the node:
  \begin{key}{/tikz/every fit (initially \normalfont empty)}
    Set this style to change the appearance of a node that uses the
    |fit| option.
  \end{key}

  The exact effects of the |fit| option are the following:
  \begin{enumerate}
  \item A minimal bounding box containg all coordinates is
    computed. Note that if a coordinate like |(a)| is used, the center
    of the |a| node is used; you will explicitly use a coordinate like
    |(a.south)| if you wish to refer to the border of |a|.
  \item The |text width| option is set to the width of this bounding box.
  \item The |text centered| option is set.
  \item The |anchor| is set to |center|.
  \item The |at| position of the node is set to the center of the
    computed bounding box.
  \item After the node has been typeset, its height and depth are
    adjusted such that they add up to the height of the computed
    bounding box and such that the text of the node is vertically
    centered inside the box.
  \end{enumerate}
  The above means that, generally speaking, if the node contains text
  like |box| in the above example, it will be centered inside the
  box. It will be difficult to put the text elsewhere, in particular,
  changing the |anchor| of the node will not have the desired
  effect. Instead, what you should do is to create a node with the
  |fit| option that does not contain any text, give it a name, and
  then use normal nodes to add text at the desired
  positions. Alternatively, consider using the |label| or |pin|
  options. 

  Suppose, for instance, that in the above example we want the word
  ``box'' to appear inside the box, but at its top. This can be
  achieved as follows: 
\begin{codeexample}[]
\begin{tikzpicture}[inner sep=0pt,thick,
                    dot/.style={fill=blue,circle,minimum size=3pt}]
  \draw[help lines] (0,0) grid (3,2);
  \node[dot] (a) at (1,1) {};
  \node[dot] (b) at (2,2) {};
  \node[dot] (c) at (1,2) {};
  \node[dot] (d) at (1.25,0.25) {};
  \node[dot] (e) at (1.75,1.5) {};

  \node[draw=red,fit=(a) (b) (c) (d) (e)] (fit) {};
  \node[below] at (fit.north) {box};
\end{tikzpicture}  
\end{codeexample}

  Here is a real-life example that uses fitting:

\begin{codeexample}[]
\begin{tikzpicture}
  [vertex/.style={minimum size=2pt,fill,draw,circle},
   open/.style={fill=none},
   sibling distance=1.5cm,level distance=.75cm,
   skip/.style={edge from parent path={}},
   every fit/.style={ellipse,draw,inner sep=-1pt},
   leaf/.style={label={[name=#1]below:$#1$}},auto]

  \node [vertex] (root) {}
  child { node [vertex,open] {}
    child { node [vertex,open] {}
      child { node [vertex] (b's parent) {}
        child { node [vertex] {}
          child { node [vertex,leaf=d] {} }
          child { node [vertex,leaf=e] {} } }
        child { node [vertex,leaf=b] {} } }
      child { node [vertex,leaf=a] {} } }
    child { node [coordinate] {}
      child[skip] 
      child { node [vertex] (f's parent) {}
        child { node [vertex,leaf=c] {} }
        child { node [vertex,leaf=f] {} } } }
    edge from parent node {$\rho$} };
  
  \node [fit=(d.south west) (e.south) (b.east) (b's parent),label=above left:$F^{(b,R)}$] {};
  \node [fit=(c.south west) (f.south east) (f's parent),label=above right:$F^{(c,R)}$]    {};
\end{tikzpicture}
\end{codeexample}


\end{key}


% Copyright 2006 by Till Tantau
%
% This file may be distributed and/or modified
%
% 1. under the LaTeX Project Public License and/or
% 2. under the GNU Free Documentation License.
%
% See the file doc/generic/pgf/licenses/LICENSE for more details.


\section{Matrix Library}

\begin{tikzlibrary}{matrices}
  This library packages defines additional styles and options for
  creating matrices.
\end{tikzlibrary}


\subsection{Matrices of Nodes}

A \emph{matrix of nodes} is a \tikzname\ matrix in which each cell
contains a node. In this case it is bothersome having to write
|\node{| at the beginning of each cell and |};| at the end of each
cell. The following style simplifies typesetting such matrices.

\begin{itemize}
  \itemstyle{matrix of nodes}
  Conceptually, this style adds |\node{| at the beginning and |};| at
  the end of each cell and sets the |anchor| of the node to
  |base|. Furthermore, it adds  the option |name| option to each node,
  where the name is set to  \meta{matrix name}|-|\meta{row
    number}|-|\meta{column number}. For  example, if the matrix has
  the name |my matrix|, then the node in  the upper left cell will get
  the name |my matrix-1-1|. 
\begin{codeexample}[]
\begin{tikzpicture}
  \matrix (magic) [matrix of nodes]
  {
    8 & 1 & 6 \\
    3 & 5 & 7 \\
    4 & 9 & 2 \\
  };

  \draw[thick,red,->] (magic-1-1) |- (magic-2-3);
\end{tikzpicture}
\end{codeexample}

  You may wish to add options to certain nodes in the matrix. This can
  be achieved in three ways.
  \begin{enumerate}
  \item You can modify, say, the
    |row 2 column 5| option to pass special options to this particular
    cell.

\begin{codeexample}[]
\begin{tikzpicture}
  \tikzstyle{row 2 column 3}=[red]
  \matrix [matrix of nodes]
  {
    8 & 1 & 6 \\
    3 & 5 & 7 \\
    4 & 9 & 2 \\
  };
\end{tikzpicture}
\end{codeexample}
    
  \item At the beginning of a cell, you can use a special syntax. If a
    cell starts with a vertical bar, then everything between this bar
    and the next bar is passed on to the |node| command.
{\catcode`\|=12
\begin{codeexample}[]
\begin{tikzpicture}
  \matrix [matrix of nodes]
  {
    8 & 1 &         6 \\
    3 & 5 & |[red]| 7 \\
    4 & 9 &         2 \\
  };
\end{tikzpicture}
\end{codeexample}
}
  You can also use an option like \verb!|[red] (seven)|! to give a
  different name to the node.

  Note that the |&| character also takes an optional argument, which
  is an extra column skip.
{\catcode`\|=12  
\begin{codeexample}[]
\begin{tikzpicture}
  \matrix [matrix of nodes]
  {
    8 &[1cm] 1 &[3mm] |[red]| 6 \\
    3 &      5 &      |[red]| 7 \\
    4 &      9 &              2 \\
  };
\end{tikzpicture}
\end{codeexample}
}
  \item If your cell starts with a |\path| command or any command that
    expands to |\path|, which includes |\draw|, |\node|, |\fill| and
    other, the |\node{| startup code and the |};| code are
    suppressed. This means that for this particular cell you can
    provide a totally different contents.

\begin{codeexample}[]
\begin{tikzpicture}
  \matrix [matrix of nodes]
  {
    8 & 1 & 6 \\
    3 & 5 & \node[red]{7}; \draw(0,0) circle(10pt);\\
    4 & 9 & 2 \\
  };
\end{tikzpicture}
\end{codeexample}
  \end{enumerate}
  \itemstyle{matrix of math nodes}
  This style is almost the same as the previous style, only |$| is %$
  added at the beginning and at the end of each node, so math mode
  will be switched on in all nodes.
{\catcode`\|=12
\begin{codeexample}[]
\begin{tikzpicture}
  \matrix [matrix of math nodes]
  {
    a_8 & a_1 &         a_6 \\
    a_3 & a_5 & |[red]| a_7 \\
    a_4 & a_9 &         a_2 \\
  };
\end{tikzpicture}
\end{codeexample}
}
  \itemoption{nodes in empty cells}\opt{|=|\meta{true or false}}
  When set to |true|, a node (with an empty contents) is put in empty
  cells. Normally, empty cells are just, well, empty. The style can be
  used together with both a |matrix of nodes| and a
  |matrix of math nodes|.
\begin{codeexample}[]
\begin{tikzpicture}
  \matrix [matrix of math nodes,nodes={circle,draw}]
  {
    a_8 &     & a_6 \\
    a_3 &     & a_7 \\
    a_4 & a_9 &     \\
  };
\end{tikzpicture}
\end{codeexample}
\begin{codeexample}[]
\begin{tikzpicture}
  \matrix [matrix of math nodes,nodes={circle,draw},nodes in empty cells]
  {
    a_8 &     & a_6 \\
    a_3 &     & a_7 \\
    a_4 & a_9 &     \\
  };
\end{tikzpicture}
\end{codeexample}
\end{itemize}


\subsection{Delimiters}

Delimiters are parantheses or braces to the left and right of a
formula or a matrix. The matrix library offers options for adding such
delimiters to a matrix. However, delimiters can actually be added to
any node that has the standard anchors |north|, |south|, |north west|
and so on. In particular, you can add delimiters to any |rectangle|
box. They are implemented by ``measuring the height'' of the node and
then adding a delimiter of the correct size to the left or right using
some after node magic.

\begin{itemize}
  \itemoption{left delimiter}|=|\meta{delimiter}
  This option can be given to a any node that has the standard anchors
  |north|, |south| and so on. The \meta{delimiter} can be any
  delimiter that is acceptable to \TeX's |\left| command.
\begin{codeexample}[]
\begin{tikzpicture}
  \matrix [matrix of math nodes,left delimiter=(,right delimiter=\}]
  {
    a_8 & a_1 & a_6 \\
    a_3 & a_5 & a_7 \\
    a_4 & a_9 & a_2 \\
  };
\end{tikzpicture}
\end{codeexample}

\begin{codeexample}[]
\begin{tikzpicture}
  \node [fill=red!20,left delimiter=(,right delimiter=\}]
    {$\displaystyle\int_0^1 x\,dx$};
\end{tikzpicture}
\end{codeexample}

  \itemstyle{every delimiter}
  This style is executed for every delimiter. You can use it to shift
  or color delimiters or do whatever.

  \itemstyle{every left delimiter}
  This style is additionally executed for every left delimiter.
\begin{codeexample}[]
\begin{tikzpicture}
  \tikzstyle{every left delimiter}=[red,xshift=1ex]
  \tikzstyle{every right delimiter}=[xshift=-1ex]
  \matrix [matrix of math nodes,left delimiter=(,right delimiter=\}]
  {
    a_8 & a_1 & a_6 \\
    a_3 & a_5 & a_7 \\
    a_4 & a_9 & a_2 \\
  };
\end{tikzpicture}
\end{codeexample}

  \itemoption{right delimiter}|=|\meta{delimiter}
  Works as above.

  \itemstyle{every right delimiter}
  Works as above.

  \itemoption{above delimiter}|=|\meta{delimiter}
  This option allows you to add a delimiter above the node. It is
  implementing by rotating a left delimiter.
\begin{codeexample}[]
\begin{tikzpicture}
  \matrix [matrix of math nodes,%
           left delimiter=\|,right delimiter=\rmoustache,%
           above delimiter=(,below delimiter=\}]
  {
    a_8 & a_1 & a_6 \\
    a_3 & a_5 & a_7 \\
    a_4 & a_9 & a_2 \\
  };
\end{tikzpicture}
\end{codeexample}

  \itemstyle{every above delimiter}
  Works as above.

  \itemoption{below delimiter}|=|\meta{delimiter}
  works as above.

  \itemstyle{every below delimiter}
  Works as above.
\end{itemize}



%%% Local Variables: 
%%% mode: latex
%%% TeX-master: "pgfmanual-pdftex-version"
%%% End: 

\section{Mindmap Drawing Library}

\begin{package}{pgflibrarytikzmindmap}
  This packages provides styles for drawing mindmap diagrams.
\end{package}

\subsection{Overview}

This library is intended to make the creation of mindmaps easier. A
\emph{mindmap} is a graphical representation of a concept together
with related concepts and annotations. Mindmaps are, essentially,
trees, possibly with a few extra edges added, but they are usually
drawn in a special way: The root concept is placed in the middle of
the page and is drawn as a huge circle, ellipse, or cloud. The related
concepts then ``leave'' this root concept in branch-like tendrils.

The mindmap library of \tikzname\ produces mindmaps that look a bit
different from the standard mindmaps: While the big root concept is
still a circle, related concepts are also depicted as (smaller)
circles. The related concepts are linked to the root concept via
organic-looking connections. The overall effect is visually rather
pleasing, but readers may not immediately think of a mindmap when they
see a picture created with this library.

Although it is not strictly necessary, you will usually create
mindmaps using \tikzname's tree mechanism and some of the styles and
macros of the package work best when used inside trees. However, it is
still possible and sometimes necessary to treat parts of a mindmap as
a graph with arbitrary edges and this is also possible.


\subsection{The Mindmap Style}

Every mindmap should be put in a scope or a picture where the
|mindmap| style is used. This style installs some internal settings.

\begin{itemize}
  \itemstyle{mindmap}
  Use this style with all pictures or at least scopes that contain a
  mindmap. It installs a whole bunch of settings that are useful for
  drawing mindmaps. 
\begin{codeexample}[]
\tikz[mindmap,concept color=red!50]
  \node [concept] {Root concept}
    child[grow=right] {node[concept] {Child concept}};
\end{codeexample}
  The sizes of concepts are predefined in such a way that a
  medium-size mindmap will fit on an A4 page (more or less).  
  \itemstyle{every mindmap}
  This style is included by the |mindmap| style. Change this style to
  add special settings to your mindmaps.
\begin{codeexample}[]
\tikz[large mindmap,concept color=red!50]
  \node [concept] {Root concept}
    child[grow=right] {node[concept] {Child concept}};
\end{codeexample}
  \itemstyle{large mindmap}
  This style includes the |mindmap| style, but additionally changes
  the default size of concepts and of distances so that a medium-sized
  mindmap will fit on an A3 page (A3 pages are twice as large as A4
  pages).
  \itemstyle{huge mindmap}
  This style causes conepts to be even bigger and it is best used with
  A2 paper and above.
\end{itemize}

\subsection{Concepts Nodes}

The basic entities of mindmaps are called \emph{concepts} in
\tikzname. A concept is a node of style |concept| and it must be
circular for some of the connection macros to work.


\subsubsection{Isolated Concepts}

The following styles influence how isolated concepts are rendered:

\begin{itemize}
  \itemstyle{concept}
  This style should be used with all nodes that are concepts, although
  some styles like |extra concept| install this style automatically.

  Bascially, this style makes the concept node circular and installs a
  uniform color called |concept color|, see below. Additionally, the
  style |every concept| is called.
\begin{codeexample}[]
\tikz[mindmap,concept color=red!50] \node [concept] {Some concept};
\end{codeexample}
  \itemstyle{every concept}
  In order to change the appearance of concept nodes, you should
  change this style. Note, however, that the color of a concept should
  be uniform for some of the connection bar stuff to work, so you
  should not change the color or the draw/fill state of concepts using
  this option. It is mostly useful for changing the text color and
  font.
  \itemoption{concept color}|=|\meta{color}
  This option tells \tikzname\ which color should be used for filling
  and stroking concepts. The difference between this option and just
  setting |every concept| to the desired color is that this option
  allows \tikzname\ to keep track of the colors used for
  concepts. This is important when you \emph{change} the color between
  two connected concepts. In this case, \tikzname\ can automatically
  create a shading that provides a smooth transition between the old
  and the new concept color; we will come back to this in the next
  section. 
  \itemstyle{extra concept}
  This style is intended for concepts that are not part of the
  ``mindmap tree,'' but stand beside it. Typically, they will have a
  subdued color are be smaller. In order to have these concepts appear
  in a uniform way and in order to indicate in the code that these
  concepts are extra, you can use this style.
\begin{codeexample}[]
\begin{tikzpicture}[mindmap,concept color=blue!80]
  \node [concept]                 {Root concept};
  \node [extra concept] at (10,0) {extra concept};
\end{tikzpicture}
\end{codeexample}
  \itemstyle{every extra concept}
  Change this style to change the appearance of extra concepts.
\end{itemize}


\subsubsection{Concepts in Trees}

As pointed out earlier, \tikzname\ assumes that your mindmap is build
using the |child| facilities of \tikzname. There are numerous options
that influence how concepts are rendered at the different levels of a
tree. 

\begin{itemize}
  \itemstyle{root concept}
  This style is used for the roots of mindmap trees. More precisely,
  this style is included by the |mindmap| style. Thus, by adding
  something to this, you can change how the root of a mindmap will be
  rendered.
\begin{codeexample}[]
\tikzstyle{root concept}+=[concept color=blue!80,minimum size=3.5cm]    
\tikz[mindmap] \node [concept] {Root concept};
\end{codeexample}

  Note that styles like |large mindmap| redefine these styles, so you
  should add something to this style only inside the picture.
  \itemstyle{level 1 concept}
  The |mindmap| style adds this style to the |level 1| style. This
  means that the first level children of a mindmap tree will use this
  style. 
\begin{codeexample}[]
\tikzstyle{root concept}+=[concept color=blue!80]    
\tikzstyle{level 1 concept}+=[concept color=red!50]    
\tikz[mindmap]
  \node [concept] {Root concept}
    child[grow=30] {node[concept] {child}}
    child[grow=0 ] {node[concept] {child}};
\end{codeexample}
  \itemstyle{level 2 concept}
  works like |level 1 concept|, only for second level children. 
  \itemstyle{level 3 concept}
  works like |level 1 concept|.
  \itemstyle{level 4 concept}
  works like |level 1 concept|. Note that there are not fifth and
  higher level styles, you need to modify |level 5| directly in such
  cases. 
  
  \itemoption{concept color}|=|\meta{color}
  We saw already that this option is used to change the color of
  concepts. We now have a look at its effect when used on child nodes
  of a concept. Normally, this option simply changes the color of the
  children. However, when the option is given as an option to the
  |child| operation (and not to the |node| operation and also not as
  an option to all children via the |level 1| style), \tikzname\ will
  smoothly change the concept color from the parent's color to the
  color of the child concept. 

  Here is an example:
\begin{codeexample}[]
\tikz[mindmap,concept color=blue!80]
  \node [concept] {Root concept}
    child[concept color=red,grow=30] {node[concept] {Child concept}}
    child[concept color=orange,grow=0]  {node[concept] {Child concept}};
\end{codeexample}

  In order to have all children of a certain level have a different
  concept color, a tiny bit of magic is needed:
\begin{codeexample}[]
\tikzstyle{root concept}+=[concept color=blue]    
\tikzstyle{level 1 concept}+=[set style={{every child}=[concept color=blue!50]}]    
\tikz[mindmap,text=white]
  \node [concept] {Root concept}
    child[grow=30] {node[concept] {child}}
    child[grow=0 ] {node[concept] {child}};
\end{codeexample}
\end{itemize}

\subsection{Connecting Concepts}

\subsection{Adding Annotations}

An \emph{annotation} is some text outside a mindmap that, unlike an
extra concept, simply explains something in the mindmap. The following
style is mainly intended to help readers of the code see that a node
in an annotation node.

\begin{itemize}
  \itemstyle{annotation}
  This style indicates that a node is an annotation node. It includes
  |every annotation|, which allows you to change this style in a
  convenient fashion.
\begin{codeexample}[]
\tikzstyle{every annotation}=[fill=red!20]    
\begin{tikzpicture}[mindmap,concept color=blue!80]
  \node [concept] (root)  {Root concept};

  \node [annotation,right] at (root.east)
  {The root concept is, in general, the most important concept.};
\end{tikzpicture}
\end{codeexample}
  \itemstyle{every annotation}
    This style is included by |annotation|.
\end{itemize}



%%% Local Variables: 
%%% mode: latex
%%% TeX-master: "pgfmanual-pdftex-version"
%%% End: 

% Copyright 2006 by Till Tantau
%
% This file may be distributed and/or modified
%
% 1. under the LaTeX Project Public License and/or
% 2. under the GNU Free Documentation License.
%
% See the file doc/generic/pgf/licenses/LICENSE for more details.


\section{Paper Folding Diagrams Library}

\label{section-calender-folding}

\begin{tikzlibrary}{folding}
  This library defines commands for creating paper folding
  diagrams. Currently, it just contains a single command for creating
  a single diagram, but that one is really useful for creating
  calendars for your (real) desktop.
\end{tikzlibrary}

\begin{command}{\tikzfoldingdodecahedron|[|\meta{options}|];|}
  This command draws a folding diagram for a dodecahedron. The syntax
  is intended to remind of the |\path| command, but (currently) you
  must specify the \meta{options} and nothing else may be specified
  between the command name and the closing semicolon.

  The following \meta{options} apply:
  \begin{itemize}
    \itemoption{folding line length}|=|\meta{dimension} sets the length
    of the base line for folding. For the dodecahedron this is the
    length of all the sides of the pentagons.
    \itemoption{face 1}|=|\meta{code}
    The \meta{code} is executed for the first face of the
    dodecahedron. When it is executed, the coordinate system will have
    been shifted and rotated such that it lies at the middle of the
    first face of the dodecahedron.
    \itemoption{face 2}|=|\meta{code}
    Same as |face 1|, but for the second face.
    \itemoption{face 3}|=|\meta{code}
    Same as |face 1|, but for the third face.
    
    There are further similar options, ending with the following:
    \itemoption{face 12}|=|\meta{code}
    Same as |face 1|, but for the last face.
  \end{itemize}

  Here is a simple example:
\begin{codeexample}[]
\begin{tikzpicture}[transform shape]
  \tikzfoldingdodecahedron
  [folding line length=6mm,
  face 1={ \node[red] {1};},
  face 2={ \node      {2};},
  face 3={ \node      {3};},
  face 4={ \node      {4};},
  face 5={ \node      {5};},
  face 6={ \node      {6};},
  face 7={ \node      {7};},
  face 8={ \node      {8};},
  face 9={ \node      {9};},
  face 10={\node      {10};},
  face 11={\node      {11};},
  face 12={\node      {12};}];
\end{tikzpicture}
\end{codeexample}

  The appearance of the cut and folding lines can be influenced using
  the following styles:
  \begin{itemize}
  \itemstyle{every cut} Executed for every line that should be cut using
  scissors. Empty by default.
  \itemstyle{every fold} Executed for every line that should be
  folded. Equal to |help lines| by default.
\begin{codeexample}[]
\begin{tikzpicture}[every cut/.style=red,every fold/.style=dotted]
  \tikzfoldingdodecahedron[folding line length=6mm];
\end{tikzpicture}
\end{codeexample}
  \end{itemize}
\end{command}

Here is a big example that produces a diagram for a calendar:

\begin{codeexample}[leave comments]
\sffamily\scriptsize
\begin{tikzpicture}
  [transform shape,
   every calendar/.style=
   {
     at={(-8ex,4ex)},
     week list,
     month label above centered,
     month text=\bfseries\textcolor{red}{\%mt} \%y0,
     if={(Sunday) [black!50]}
   }]
  \tikzfoldingdodecahedron
  [
    folding line length=2.5cm,
    face 1={ \calendar [dates=\the\year-01-01 to \the\year-01-last];},
    face 2={ \calendar [dates=\the\year-02-01 to \the\year-02-last];},
    face 3={ \calendar [dates=\the\year-03-01 to \the\year-03-last];},
    face 4={ \calendar [dates=\the\year-04-01 to \the\year-04-last];},
    face 5={ \calendar [dates=\the\year-05-01 to \the\year-05-last];},
    face 6={ \calendar [dates=\the\year-06-01 to \the\year-06-last];},
    face 7={ \calendar [dates=\the\year-07-01 to \the\year-07-last];},
    face 8={ \calendar [dates=\the\year-08-01 to \the\year-08-last];},
    face 9={ \calendar [dates=\the\year-09-01 to \the\year-09-last];},
    face 10={\calendar [dates=\the\year-10-01 to \the\year-10-last];},
    face 11={\calendar [dates=\the\year-11-01 to \the\year-11-last];},
    face 12={\calendar [dates=\the\year-12-01 to \the\year-12-last];}
  ];
\end{tikzpicture}
\end{codeexample}




%%% Local Variables: 
%%% mode: latex
%%% TeX-master: "pgfmanual-pdftex-version"
%%% End: 

% Copyright 2006 by Till Tantau
%
% This file may be distributed and/or modified
%
% 1. under the LaTeX Project Public License and/or
% 2. under the GNU Free Documentation License.
%
% See the file doc/generic/pgf/licenses/LICENSE for more details.


\section{Pattern Library}
\label{section-library-patterns}

\begin{pgflibrary}{patterns}
  The package defines patterns for filling areas.
\end{pgflibrary}


\newcommand\patternindex[1]{%
  \index{#1@\protect\texttt{#1} pattern}%
  \index{Patterns!#1@\protect\texttt{#1}}%
  \texttt{#1}& 
  \begin{tikzpicture}[baseline=.5ex]

    % Background
    \pattern [path fading=west,pattern=checkerboard light gray]
      (0,0) rectangle (5cm,2em);
    
    \pattern [pattern=#1,pattern color=black] (0,0) rectangle +(1.5cm,2em);
    \pattern [pattern=#1,pattern color=blue] (1.75,0) rectangle +(1.5cm,2em);
    \pattern [pattern=#1,pattern color=red] (3.5,0) rectangle +(1.5cm,2em);
  \end{tikzpicture} \\[1ex]
}

\newcommand\patternindexinherentlycolored[1]{%
  \index{#1@\protect\texttt{#1} pattern}%
  \index{Patterns!#1@\protect\texttt{#1}}%
  \texttt{#1}& 
  \begin{tikzpicture}[baseline=.5ex]

    % Background
    \pattern [path fading=west,pattern=checkerboard light gray]
      (0,0) rectangle (5cm,2em);
    
    \pattern [pattern=#1,pattern color=blue] (0,0) rectangle +(5cm,2em);
  \end{tikzpicture} \\[1ex]
}

\subsection{Form-Only Patterns}

\begin{tabular}{ll}
  \emph{Pattern name} & \emph{Example (pattern in black, blue, and red
    on faded checkerboard)} \\ 
  \patternindex{horizontal lines} 
  \patternindex{vertical lines} 
  \patternindex{north east lines} 
  \patternindex{north west lines} 
  \patternindex{grid} 
  \patternindex{crosshatch} 
  \patternindex{dots} 
  \patternindex{crosshatch dots} 
  \patternindex{fivepointed stars} 
  \patternindex{sixpointed stars} 
  \patternindex{bricks}
  \patternindex{checkerboard}
\end{tabular}
  
\subsection{Inherently Colored Patterns}


\begin{tabular}{ll}
  \emph{Pattern name} & \emph{Example} \\
  \patternindexinherentlycolored{checkerboard light gray} 
  \patternindexinherentlycolored{horizontal lines light gray} 
  \patternindexinherentlycolored{horizontal lines gray} 
  \patternindexinherentlycolored{horizontal lines dark gray} 
  \patternindexinherentlycolored{horizontal lines light blue} 
  \patternindexinherentlycolored{horizontal lines dark blue} 
  \patternindexinherentlycolored{crosshatch dots gray} 
  \patternindexinherentlycolored{crosshatch dots light steel blue} 
\end{tabular}
  

%%% Local Variables: 
%%% mode: latex
%%% TeX-master: "pgfmanual-pdftex-version"
%%% End: 

% Copyright 2003 by Till Tantau <tantau@cs.tu-berlin.de>.
%
% This program can be redistributed and/or modified under the terms
% of the LaTeX Project Public License Distributed from CTAN
% archives in directory macros/latex/base/lppl.txt.




\section{Petri-Net Drawing Library}

\begin{tikzlibrary}{petri}
  This packages provides shapes and styles for drawing Petri nets. 
\end{tikzlibrary}



\subsection{Places}

The package defines a style for drawing places of Petri nets. 

\begin{itemize}
  \itemstyle{place}
  This style indicates that a node is a place of a Petri net. Usually,
  the text of the node should be empty since places do not contain any
  text. You should use the |label| option to add text outside the node
  like its name or its capacity. You should use the |tokens| options,
  explained in Section~\ref{section-tokens}, to add tokens inside the
  place.
  
\begin{codeexample}[]
\begin{tikzpicture}
  \node[place,label=above:$p_1$,tokens=2]        (p1) {};
  \node[place,label=below:$p_2\ge1$,right of=p1] (p2) {};
\end{tikzpicture}
\end{codeexample}
  
  \itemstyle{every place}
  This stype is envoked by the style |place|. To change the
  appearance of places, you can change this style.
\begin{codeexample}[]
\tikzstyle{every place}=[draw=blue,fill=blue!20,thick,minimum size=9mm]
\begin{tikzpicture}
  \node[place,tokens=7,label=above:$p_1$]  (p1) {};
  \node[place,structured tokens={3,2,9},
        label=below:$p_2\ge1$,right of=p1] (p2) {};
\end{tikzpicture}
\end{codeexample}
\end{itemize}



\subsection{Transitions}

Transitions are also nodes. They should be drawn using the following
style: 

\begin{itemize}
  \itemstyle{transition}
  This style indicates that a node is a transition. As for places, the
  text of a transition should be empty and the |label| option should
  be used for adding labels.

  To connect a transition to places, you can use the |edge| command as
  in the following example:
  
\begin{codeexample}[]
\begin{tikzpicture}[node distance=2cm]
  \node[place,tokens=2,label=above:$p_1$]        (p1) {};
  \node[place,label=above:$p_2\ge1$,right of=p1] (p2) {};

  \node[transition,below right of=p1,label=below:$t_1$] {}
    edge[pre]                 (p1)
    edge[post] node[auto] {2} (p2);
\end{tikzpicture}
\end{codeexample}
  
  \itemstyle{every transition}
  This style is envoked by the style |transition|.

  \itemstyle{pre}
  This style can be used with paths leading \emph{from} a transition
  \emph{to} a place to indicate that the place is in the pre-set of
  the transition. By default, this style is |<-,shorten <=1pt|, but
  feel free to redefine it.

  \itemstyle{post}
  This style is also used with paths leading \emph{from} a transition
  \emph{to} a place, but this time the place is in the post-set of
  the transition. Again, feel free to redefine it.

  \itemstyle{pre and post}
  This style is to be used to indicate that a place is both in the
  pre- and post-set of a transition.
\end{itemize}


\subsection{Tokens}
\label{section-tokens}

Interestingly, the most complicated aspect of drawing Petri nets in
\tikzname\ turns out to be the placement of tokens.

Let us start with a single token. They are also nodes and there is a
simple style for typesetting them.


\begin{itemize}
  \itemstyle{token}
  This style indicates that a node is a token. By default, this causes
  the node to be a small black circle. Unlike places and transitions,
  it \emph{does} make sense to provide text for the token node. Such
  text will be typeset in a tiny font and in white on black
  (naturally, you can easily change this by setting the style
  |every token|).
    
\begin{codeexample}[]
\begin{tikzpicture}
  \node[place,label=above:$p_1$]             (p1) {};
  \node[token] at (p1) {};

  \node[place,label=above:$p_2$,right of=p1] (p2) {};
  \node[token] at (p2) {$y$};
\end{tikzpicture}
\end{codeexample}
\end{itemize}

In the above example, it is bothersome that we need an extra command
for the token node. Worse, when we have \emph{two} tokens on a node,
it is difficult to place both nodes inside the node without overlap.

The Petri net library offers a solution to this problem: The
|children are tokens| style.


\begin{itemize}
  \itemstyle{children are tokens}
  The idea behind this style is to use trees mechanism for placing
  tokens. Every token lying on a place is treated as a child of the
  node. Normally this would have the effect that the tokens are placed
  below the place and they would be connected to the place by an
  edge. The |children are tokens| style, however, redefines the growth
  function of trees such that it places the children next to each
  other inside (or, rather, on top) of the place node. Additionally,
  the edge from the parent node is not drawn.
\begin{codeexample}[]
\begin{tikzpicture}
  \node[place,label=above:$p_1$] {}
  [children are tokens]
  child {node [token] {1}}
  child {node [token] {2}}
  child {node [token] {3}};
\end{tikzpicture}
\end{codeexample}

  In detail, what happens is the following: Tree growth functions tell
  \tikzname\ where it should place the children of nodes. These
  functions get passed the number of children that a node has an the
  number of the child that should be placed. The special tree growth
  function for tokens has a special mapping for each possible number
  of children up to nine children. This mapping decides for each child
  where it should be placed on top of the place. For example, a single
  child is placed directly on top of the place. Two children are
  placed next to each other, separated by the |token distance|. Three
  children are placed in a triangle whose side lengths are
  |token distance|; and so on up to nine tokens. If you wish to place
  more than nice tokens on a place, you will have to write your own
  placement code.
\begin{codeexample}[]
\begin{tikzpicture}
  \node[place,label=above:$p_2$] {}
  [children are tokens]
  child {node [token] {1}}
  child {node [token,fill=red] {2}}
  child {node [token,fill=red] {2}}
  child {node [token] {1}};
\end{tikzpicture}
\end{codeexample}

  \itemoption{token distance}|=|\meta{distance}
  This specifies the distance between the centers of the tokens in the
  arrangements of the option |children are tokens|.
\begin{codeexample}[]
\begin{tikzpicture}
  \node[place,label=above:$p_3$] {}
  [children are tokens,token distance=1.1ex]
  child {node [token] {}}
  child {node [token,red] {}}
  child {node [token,blue] {}}
  child {node [token] {}};
\end{tikzpicture}
\end{codeexample}
\end{itemize}

The |children are tokens| options gives you a lot of flexibility, but
it is a bit cumbersome to use. For this reason there are some options
that help in standard situations. They all use |children are tokens|
internally, so any change to, say, the |every tokens| style will
affect how these options depict tokens.

\begin{itemize}
  \itemoption{tokens}|=|\meta{number}
  This option is given to a |place| node, not to a |token| node. The
  effect of this option is to add \meta{number} many child nodes to
  the place, each having the style |token|. Thus, the following two
  pieces of codes have the same effect:
\begin{codeexample}[]
\tikz
  \node[place] {}
  [children are tokens]
  child {node [token] {}}
  child {node [token] {}}
  child {node [token] {}};
\tikz
  \node[place,tokens=3] {};
\end{codeexample}
  It is legal to say |tokens=0|, no tokens are drawn in this
  case. This option does not handle ten or more tokens correctly. If
  you need this many tokens, you will have to program your own code.
\begin{codeexample}[]
\begin{tikzpicture}
  \tikzstyle{every place}=[minimum size=9mm]

  \foreach \x/\y/\tokennumber in {0/2/1,1/2/2,2/2/3,
                                  0/1/4,1/1/5,2/1/6,
                                  0/0/7,1/0/8,2/0/9}
    \node [place,tokens=\tokennumber] at (\x,\y) {};
\end{tikzpicture}
\end{codeexample}
  \itemoption{colored tokens}|=|\meta{color list}
  This option, which must also be given when a place node is being
  created, gets a list of colors as parameter. It will then add as
  many tokens to the place are in this list, each colored with the
  corresponding color in the list.
\begin{codeexample}[]
\tikz  \node[place,colored tokens={black,black,red,blue}] {};
\end{codeexample}
  \itemoption{structured tokens}|=|\meta{token texts}
  This option, which must again be passed to a place, gets a list
  texts for tokens. For each text, another token will be added to the place.
\begin{codeexample}[]
\tikz  \node[place,structured tokens={$x$,$y$,$z$}] {};
\end{codeexample}
\begin{codeexample}[]
\begin{tikzpicture}
  \tikzstyle{every place}=[minimum size=9mm]

  \foreach \x/\y/\tokennumber in {0/2/1,1/2/2,2/2/3,
                                  0/1/4,1/1/5,2/1/6,
                                  0/0/7,1/0/8,2/0/9}
    \node [place,structured tokens={1,...,\tokennumber}] at (\x,\y) {};
\end{tikzpicture}
\end{codeexample}
  If you use lots of structured tokens, consider redefining the
  |every token| style so that the tokens are larger.
\end{itemize}



%%% Local Variables: 
%%% mode: latex
%%% TeX-master: "pgfmanual-pdftex-version"
%%% End: 

% Copyright 2006 by Till Tantau
%
% This file may be distributed and/or modified
%
% 1. under the LaTeX Project Public License and/or
% 2. under the GNU Free Documentation License.
%
% See the file doc/generic/pgf/licenses/LICENSE for more details.


\section{Plot Handler Library}
\label{section-library-plothandlers}

\begin{pgflibrary}{plothandlers}
  This library packages defines additional plot handlers, see
  Section~\ref{section-plot-handlers} for an introduction to plot
  handlers. The additional handlers are described in the following.

  This library is loaded automatically by \tikzname.
\end{pgflibrary}


\subsection{Curve Plot Handlers}
  
\begin{command}{\pgfplothandlercurveto}
  This handler will issue a |\pgfpathcurveto| command for each point of
  the plot, \emph{except} possibly for the first. As for the line-to
  handler, what happens with the first point can be specified using
  |\pgfsetmovetofirstplotpoint| or |\pgfsetlinetofirstplotpoint|.

  Obviously, the |\pgfpathcurveto| command needs, in addition to the
  points on the path, some control points. These are generated
  automatically using a somewhat ``dumb'' algorithm: Suppose you have
  three points $x$, $y$, and $z$ on the curve such that $y$ is between
  $x$ and $z$:
\begin{codeexample}[]
\begin{tikzpicture}    
  \draw[gray] (0,0) node {x} (1,1) node {y} (2,.5) node {z};
  \pgfplothandlercurveto
  \pgfplotstreamstart
  \pgfplotstreampoint{\pgfpoint{0cm}{0cm}}
  \pgfplotstreampoint{\pgfpoint{1cm}{1cm}}
  \pgfplotstreampoint{\pgfpoint{2cm}{.5cm}}
  \pgfplotstreamend
  \pgfusepath{stroke}
\end{tikzpicture}
\end{codeexample}

  In order to determine the control points of the curve at the point
  $y$, the handler computes the vector $z-x$ and scales it by the
  tension factor (see below). Let us call the resulting vector
  $s$. Then $y+s$ and $y-s$ will be the control points around $y$. The
  first control point at the beginning of the curve will be the
  beginning itself, once more; likewise the last control point is the
  end itself.
\end{command}

\begin{command}{\pgfsetplottension\marg{value}}
  Sets the factor used by the curve plot handlers to determine the
  distance of the control points from the points they control. The
  higher the curvature of the curve points, the higher this value
  should be. A value of $1$ will cause four points at quarter
  positions of a circle to be connected using a circle. The default is
  $0.5$. 

\begin{codeexample}[]
\begin{tikzpicture}    
  \draw[gray] (0,0) node {x} (1,1) node {y} (2,.5) node {z};
  \pgfsetplottension{0.75}
  \pgfplothandlercurveto
  \pgfplotstreamstart
  \pgfplotstreampoint{\pgfpoint{0cm}{0cm}}
  \pgfplotstreampoint{\pgfpoint{1cm}{1cm}}
  \pgfplotstreampoint{\pgfpoint{2cm}{0.5cm}}
  \pgfplotstreamend
  \pgfusepath{stroke}
\end{tikzpicture}
\end{codeexample}
\end{command}


\begin{command}{\pgfplothandlerclosedcurve}
  This handler works like the curve-to plot handler, only it will
  add a new part to the current path that is a closed curve through
  the plot points.
\begin{codeexample}[]
\begin{tikzpicture}    
  \draw[gray] (0,0) node {x} (1,1) node {y} (2,.5) node {z};
  \pgfplothandlerclosedcurve
  \pgfplotstreamstart
  \pgfplotstreampoint{\pgfpoint{0cm}{0cm}}
  \pgfplotstreampoint{\pgfpoint{1cm}{1cm}}
  \pgfplotstreampoint{\pgfpoint{2cm}{0.5cm}}
  \pgfplotstreamend
  \pgfusepath{stroke}
\end{tikzpicture}
\end{codeexample}
\end{command}

\subsection{Constant Plot Handlers}
There are three plot handlers which produce piecewise constant interpolations between successive points.

\begin{command}{\pgfplothandlerconstantlineto}
  This handler works like the line-to plot handler, only it will
  produce a connected, piecewise constant path to connect the points.
\begin{codeexample}[]
\begin{tikzpicture}    
  \draw[gray] (0,0) node {x} (1,1) node {y} (2,.5) node {z};
  \pgfplothandlerconstantlineto
  \pgfplotstreamstart
  \pgfplotstreampoint{\pgfpoint{0cm}{0cm}}
  \pgfplotstreampoint{\pgfpoint{1cm}{1cm}}
  \pgfplotstreampoint{\pgfpoint{2cm}{0.5cm}}
  \pgfplotstreamend
  \pgfusepath{stroke}
\end{tikzpicture}
\end{codeexample}
\end{command}

\begin{command}{\pgfplothandlerconstantlinetomarkright}
A variant of |\pgfplothandlerconstantlineto| which places its mark on
the right line ends.
\begin{codeexample}[]
\begin{tikzpicture}    
  \draw[gray] (0,0) node {x} (1,1) node {y} (2,.5) node {z};
  \pgfplothandlerconstantlinetomarkright
  \pgfplotstreamstart
  \pgfplotstreampoint{\pgfpoint{0cm}{0cm}}
  \pgfplotstreampoint{\pgfpoint{1cm}{1cm}}
  \pgfplotstreampoint{\pgfpoint{2cm}{0.5cm}}
  \pgfplotstreamend
  \pgfusepath{stroke}
\end{tikzpicture}
\end{codeexample}
\end{command}

\begin{command}{\pgfplothandlerjumpmarkleft}
  This handler works like the line-to plot handler, only it will
  produce a non-connected, piecewise constant path to connect the points. If there are any plot marks, they will be placed on the left open pieces.
\begin{codeexample}[]
\begin{tikzpicture}    
  \draw[gray] (0,0) node {x} (1,1) node {y} (2,.5) node {z};
  \pgfplothandlerjumpmarkleft
  \pgfplotstreamstart
  \pgfplotstreampoint{\pgfpoint{0cm}{0cm}}
  \pgfplotstreampoint{\pgfpoint{1cm}{1cm}}
  \pgfplotstreampoint{\pgfpoint{2cm}{0.5cm}}
  \pgfplotstreamend
  \pgfusepath{stroke}
\end{tikzpicture}
\end{codeexample}
\end{command}

\begin{command}{\pgfplothandlerjumpmarkright}
  This handler works like the line-to plot handler, only it will
  produce a non-connected, piecewise constant path to connect the points. If there are any plot marks, they will be placed on the right open pieces.
\begin{codeexample}[]
\begin{tikzpicture}    
  \draw[gray] (0,0) node {x} (1,1) node {y} (2,.5) node {z};
  \pgfplothandlerjumpmarkright
  \pgfplotstreamstart
  \pgfplotstreampoint{\pgfpoint{0cm}{0cm}}
  \pgfplotstreampoint{\pgfpoint{1cm}{1cm}}
  \pgfplotstreampoint{\pgfpoint{2cm}{0.5cm}}
  \pgfplotstreamend
  \pgfusepath{stroke}
\end{tikzpicture}
\end{codeexample}
\end{command}

\subsection{Comb Plot Handlers}

There are three ``comb'' plot handlers. There name stems from the fact
that the plots they produce look like ``combs'' (more or less).

\begin{command}{\pgfplothandlerxcomb}
  This handler converts each point in the plot stream into a line from
  the $y$-axis to the point's coordinate, resulting in a ``horizontal
  comb.''

  
\begin{codeexample}[]
\begin{tikzpicture}    
  \draw[gray] (0,0) node {x} (1,1) node {y} (2,.5) node {z};
  \pgfplothandlerxcomb
  \pgfplotstreamstart
  \pgfplotstreampoint{\pgfpoint{0cm}{0cm}}
  \pgfplotstreampoint{\pgfpoint{1cm}{1cm}}
  \pgfplotstreampoint{\pgfpoint{2cm}{0.5cm}}
  \pgfplotstreamend
  \pgfusepath{stroke}
\end{tikzpicture}
\end{codeexample}
\end{command}


\begin{command}{\pgfplothandlerycomb}
  This handler converts each point in the plot stream into a line from
  the $x$-axis to the point's coordinate, resulting in a ``vertical
  comb.''
  
  This handler is useful for creating ``bar diagrams.''
\begin{codeexample}[]
\begin{tikzpicture}    
  \draw[gray] (0,0) node {x} (1,1) node {y} (2,.5) node {z};
  \pgfplothandlerycomb
  \pgfplotstreamstart
  \pgfplotstreampoint{\pgfpoint{0cm}{0cm}}
  \pgfplotstreampoint{\pgfpoint{1cm}{1cm}}
  \pgfplotstreampoint{\pgfpoint{2cm}{0.5cm}}
  \pgfplotstreamend
  \pgfusepath{stroke}
\end{tikzpicture}
\end{codeexample}
\end{command}

\begin{command}{\pgfplothandlerpolarcomb}
  This handler converts each point in the plot stream into a line from
  the origin to the point's coordinate.
  
\begin{codeexample}[]
\begin{tikzpicture}    
  \draw[gray] (0,0) node {x} (1,1) node {y} (2,.5) node {z};
  \pgfplothandlerpolarcomb
  \pgfplotstreamstart
  \pgfplotstreampoint{\pgfpoint{0cm}{0cm}}
  \pgfplotstreampoint{\pgfpoint{1cm}{1cm}}
  \pgfplotstreampoint{\pgfpoint{2cm}{0.5cm}}
  \pgfplotstreamend
  \pgfusepath{stroke}
\end{tikzpicture}
\end{codeexample}
\end{command}

\subsubsection{Changing the start of comb/bar plots}
\pgfname\ bar or comb plots usually draw something from zero to the current plot's coordinate.

The `zero' offset can be changed using an input stream which returns the desired offset successively for each processed coordinate.

There are two such streams which can be configured independently.
The first one returns `zeros' for coordinate~$x$, the second one
returns `zeros' for coordinate~$y$. They are used as follows.

\begin{codeexample}[code only]
\pgfplotxzerolevelstreamstart
\pgfplotxzerolevelstreamnext % assigns \pgf@x
\pgfplotxzerolevelstreamnext
\pgfplotxzerolevelstreamnext
\pgfplotxzerolevelstreamend
\end{codeexample}
%
\begin{codeexample}[code only]
\pgfplotyzerolevelstreamstart
\pgfplotyzerolevelstreamnext % assigns \pgf@x
\pgfplotyzerolevelstreamend
\end{codeexample}
Different zero level streams can be implemented by overwriting these macros.

\begin{command}{\pgfplotxzerolevelstreamconstant\marg{dimension}}
	This zero level stream always returns \marg{dimension} instead of $x=0$pt.

	It is used for |xcomb| and |xbar|.
\end{command}

\begin{command}{\pgfplotyzerolevelstreamconstant\marg{dimension}}
	This zero level stream always returns \marg{dimension} instead of $y=0$pt.

	It is used for |ycomb| and |ybar|.
\end{command}

\subsection{Bar Plot Handlers}
\label{section-plotlib-bar-handlers}

While comb plot handlers produce a line-to operation to generate combs, bar plot handlers employ rectangular shapes, allowing filled bars (or pattern bars).

\begin{command}{\pgfplothandlerybar}
  This handler converts each point in the plot stream into a rectangle from
  the $x$-axis to the point's coordinate. The rectangle is placed centered at the $x$-axis.
  
\begin{codeexample}[]
\begin{tikzpicture}    
  \draw[gray] (0,0) node {x} (1,1) node {y} (2,.5) node {z};
  \pgfplothandlerybar
  \pgfplotstreamstart
  \pgfplotstreampoint{\pgfpoint{0cm}{0cm}}
  \pgfplotstreampoint{\pgfpoint{1cm}{1cm}}
  \pgfplotstreampoint{\pgfpoint{2cm}{0.5cm}}
  \pgfplotstreamend
  \pgfusepath{stroke}
\end{tikzpicture}
\end{codeexample}
\end{command}

\begin{command}{\pgfplothandlerxbar}
  This handler converts each point in the plot stream into a rectangle from
  the $y$-axis to the point's coordinate. The rectangle is placed centered at the $y$-axis.
  
\begin{codeexample}[]
\begin{tikzpicture}    
  \draw[gray] (0,0) node {x} (1,1) node {y} (2,.5) node {z};
  \pgfplothandlerxbar
  \pgfplotstreamstart
  \pgfplotstreampoint{\pgfpoint{0cm}{0cm}}
  \pgfplotstreampoint{\pgfpoint{1cm}{1cm}}
  \pgfplotstreampoint{\pgfpoint{2cm}{0.5cm}}
  \pgfplotstreamend
  \pgfusepath{stroke}
\end{tikzpicture}
\end{codeexample}
\end{command}

\label{key-bar-width}%
\begin{key}{/pgf/bar width=\marg{dimension} (initially 10pt)}
	\keyalias{tikz}
	Sets the width of |\pgfplothandlerxbar| and |\pgfplothandlerybar| to \marg{dimension}. The argument \marg{dimension} will be evaluated using the math parser.
\end{key}

\label{key-bar-shift}%
\begin{key}{/pgf/bar shift=\marg{dimension} (initially 0pt)}
	\keyalias{tikz}
	Sets a shift used by |\pgfplothandlerxbar| and |\pgfplothandlerybar| to \marg{dimension}. It has the same effect as |xshift|, but it applies only to those bar plots. The argument \marg{dimension} will be evaluated using the math parser.
\end{key}

\begin{command}{\pgfplotbarwidth}
	Expands to the value of |/pgf/bar width|.
\end{command}


\begin{command}{\pgfplothandlerybarinterval}
  This handler is a variant of |\pgfplothandlerybar| which works with intervals instead of points.
  
  Bars are drawn between successive input coordinates and the width is determined relatively to the interval length.
  
\begin{codeexample}[]
\begin{tikzpicture}    
  \draw[gray] (0,2) node {$x_1$} (1,1) node {$x_2$} (2,.5) node {$x_3$} (4,0.7) node {$x_4$};
  \pgfplothandlerybarinterval
  \pgfplotstreamstart
  \pgfplotstreampoint{\pgfpoint{0cm}{2cm}}
  \pgfplotstreampoint{\pgfpoint{1cm}{1cm}}
  \pgfplotstreampoint{\pgfpoint{2cm}{0.5cm}}
  \pgfplotstreampoint{\pgfpoint{4cm}{0.7cm}}
  \pgfplotstreamend
  \pgfusepath{stroke}
\end{tikzpicture}
\end{codeexample}

In more detail, if $(x_i,y_i)$ and $(x_{i+1},y_{i+1})$ denote successive input coordinates, the bar will be placed above the interval $[x_i,x_{i+1}]$, centered at
\[ x_i + \text{\meta{bar interval shift}} \cdot (x_{i+1} - x_i) \]
with width
\[ \text{\meta{bar interval width}} \cdot (x_{i+1} - x_i). \]
Here, \meta{bar interval shift} and \meta{bar interval width} denote the current values of the associated options.

If you have $N+1$ input points, you will get $N$ bars (one for each interval). The $y$~value of the last point will be ignored.
\end{command}

\begin{command}{\pgfplothandlerxbarinterval}
   As |\pgfplothandlerybarinterval|, this handler provides bar plots with relative bar sizes and offsets, one bar for each $y$~coordinate interval.
\end{command}

\label{key-bar-interval-shift}%
\begin{key}{/pgf/bar interval shift=\marg{shift} (initially 0.5)}
	\keyalias{tikz}
	Sets the \emph{relative} shift of |\pgfplothandlerxbarinterval| and |\pgfplothandlerybarinterval| to \marg{shift}. As |/pgf/bar interval width|, the argument is relative to the interval length of the input coordinates.
	
	The argument \marg{scale} will be evaluated using the math parser.
\end{key}

\label{key-bar-interval-width}%
\begin{key}{/pgf/bar interval width=\marg{scale} (initially 1)}
	\keyalias{tikz}
	Sets the \emph{relative} width of |\pgfplothandlerxbarinterval| and |\pgfplothandlerybarinterval| to \marg{scale}. The argument is relative to $(x_{i+1} - x_i)$ for $y$~bar plots and relative to $(y_{i+1}-y_i)$ for $x$~bar plots.
	
	The argument \marg{scale} will be evaluated using the math parser.
\begin{codeexample}[]
\begin{tikzpicture}[bar interval width=0.5]  
  \draw[gray] 
  	(0,3) -- (0,-0.1) 
    (1,3) -- (1,-0.1)
    (2,3) -- (2,-0.1)
    (4,3) -- (4,-0.1);
  \pgfplothandlerybarinterval
  \begin{scope}[bar interval shift=0.25,fill=blue]
  \pgfplotstreamstart
  \pgfplotstreampoint{\pgfpoint{0cm}{2cm}}
  \pgfplotstreampoint{\pgfpoint{1cm}{1cm}}
  \pgfplotstreampoint{\pgfpoint{2cm}{0.5cm}}
  \pgfplotstreampoint{\pgfpoint{4cm}{0.7cm}}
  \pgfplotstreamend
  \pgfusepath{fill}
  \end{scope}
  \begin{scope}[bar interval shift=0.75,fill=red]
  \pgfplotstreamstart
  \pgfplotstreampoint{\pgfpoint{0cm}{3cm}}
  \pgfplotstreampoint{\pgfpoint{1cm}{0.2cm}}
  \pgfplotstreampoint{\pgfpoint{2cm}{0.7cm}}
  \pgfplotstreampoint{\pgfpoint{4cm}{0.2cm}}
  \pgfplotstreamend
  \pgfusepath{fill}
  \end{scope}
\end{tikzpicture}
\end{codeexample}
Please note that bars are always centered, so we have to use shifts $0.25$ and $0.75$ instead of $0$ and $0.5$.
\end{key}

\subsection{Mark Plot Handler}

\label{section-plot-marks}

\begin{command}{\pgfplothandlermark\marg{mark code}}
  This command will execute the \meta{mark code} for some points of the
  plot, but each time the coordinate transformation matrix will be
  setup such that the origin is at the position of the point to be
  plotted. This way, if the \meta{mark code} draws a little circle
  around the origin, little circles will be drawn at some point of the
  plot.

  By default, a mark is drawn at all points of the plot. However, two
  parameters $r$ and $p$ influence this. First, only every $r$th mark
  is drawn. Second, the first mark drawn is the $p$th. These
  parameters can be influenced using the commands below.
  
\begin{codeexample}[]
\begin{tikzpicture}    
  \draw[gray] (0,0) node {x} (1,1) node {y} (2,.5) node {z};
  \pgfplothandlermark{\pgfpathcircle{\pgfpointorigin}{4pt}\pgfusepath{stroke}}
  \pgfplotstreamstart
  \pgfplotstreampoint{\pgfpoint{0cm}{0cm}}
  \pgfplotstreampoint{\pgfpoint{1cm}{1cm}}
  \pgfplotstreampoint{\pgfpoint{2cm}{0.5cm}}
  \pgfplotstreamend
  \pgfusepath{stroke}
\end{tikzpicture}
\end{codeexample}

  Typically, the \meta{code} will be |\pgfuseplotmark{|\meta{plot mark
      name}|}|, where \meta{plot mark name} is the name of a
  predefined plot mark.
\end{command}

\begin{command}{\pgfsetplotmarkrepeat\marg{repeat}}
  Sets the $r$ parameter to \meta{repeat}, that is, only every $r$th
  mark will be drawn.
\end{command}

\begin{command}{\pgfsetplotmarkphase\marg{phase}}
  Sets the $p$ parameter to \meta{phase}, that is, the first mark to
  be drawn is the $p$th, followed by the $(p+r)$th, then the
  $(p+2r)$th, and so on.
\end{command}

\begin{command}{\pgfplothandlermarklisted\marg{mark code}\marg{index list}}
  This command works similar to the previous one. However, marks will
  only be placed at those indices in the given \meta{index list}. The
  syntax for the list is the same as for the |\foreach| statement. For
  example, if you provide the list |1,3,...,25|, a mark will be placed
  only at every second point. Similarly, |1,2,4,8,16,32| yields marks
  only at those points that are powers of two.
  
\begin{codeexample}[]
\begin{tikzpicture}    
  \draw[gray] (0,0) node {x} (1,1) node {y} (2,.5) node {z};
  \pgfplothandlermarklisted
    {\pgfpathcircle{\pgfpointorigin}{4pt}\pgfusepath{stroke}}
    {1,3}
  \pgfplotstreamstart
  \pgfplotstreampoint{\pgfpoint{0cm}{0cm}}
  \pgfplotstreampoint{\pgfpoint{1cm}{1cm}}
  \pgfplotstreampoint{\pgfpoint{2cm}{0.5cm}}
  \pgfplotstreamend
  \pgfusepath{stroke}
\end{tikzpicture}
\end{codeexample}
\end{command}

\begin{command}{\pgfuseplotmark\marg{plot mark name}}
  Draws the given \meta{plot mark name} at the origin. The \meta{plot
    mark name} must previously have been declared using
  |\pgfdeclareplotmark|. 

\begin{codeexample}[]
\begin{tikzpicture}    
  \draw[gray] (0,0) node {x} (1,1) node {y} (2,.5) node {z};
  \pgfplothandlermark{\pgfuseplotmark{pentagon}}
  \pgfplotstreamstart
  \pgfplotstreampoint{\pgfpoint{0cm}{0cm}}
  \pgfplotstreampoint{\pgfpoint{1cm}{1cm}}
  \pgfplotstreampoint{\pgfpoint{2cm}{0.5cm}}
  \pgfplotstreamend
  \pgfusepath{stroke}
\end{tikzpicture}
\end{codeexample}
\end{command}

\begin{command}{\pgfdeclareplotmark\marg{plot mark name}\marg{code}}
  Declares a plot mark for later used with the |\pgfuseplotmark|
  command.

\begin{codeexample}[]
\pgfdeclareplotmark{my plot mark}
  {\pgfpathcircle{\pgfpoint{0cm}{1ex}}{1ex}\pgfusepathqstroke}  
\begin{tikzpicture}    
  \draw[gray] (0,0) node {x} (1,1) node {y} (2,.5) node {z};
  \pgfplothandlermark{\pgfuseplotmark{my plot mark}}
  \pgfplotstreamstart
  \pgfplotstreampoint{\pgfpoint{0cm}{0cm}}
  \pgfplotstreampoint{\pgfpoint{1cm}{1cm}}
  \pgfplotstreampoint{\pgfpoint{2cm}{0.5cm}}
  \pgfplotstreamend
  \pgfusepath{stroke}
\end{tikzpicture}
\end{codeexample}
\end{command}


\begin{command}{\pgfsetplotmarksize\marg{dimension}}
  This command sets the \TeX\ dimension |\pgfplotmarksize| to
  \meta{dimension}. This dimension is a ``recommendation'' for plot
  mark code at which size the plot mark should be drawn; plot mark
  code may choose to ignore this \meta{dimension} altogether. For
  circles, \meta{dimension} should  be the radius, for other shapes it
  should be about half the width/height.

  The predefined plot marks all take this dimension into account.

\begin{codeexample}[]
\begin{tikzpicture}    
  \draw[gray] (0,0) node {x} (1,1) node {y} (2,.5) node {z};
  \pgfsetplotmarksize{1ex}
  \pgfplothandlermark{\pgfuseplotmark{*}}
  \pgfplotstreamstart
  \pgfplotstreampoint{\pgfpoint{0cm}{0cm}}
  \pgfplotstreampoint{\pgfpoint{1cm}{1cm}}
  \pgfplotstreampoint{\pgfpoint{2cm}{0.5cm}}
  \pgfplotstreamend
  \pgfusepath{stroke}
\end{tikzpicture}
\end{codeexample}
\end{command}

\begin{textoken}{\pgfplotmarksize}
  A \TeX\ dimension that is a ``recommendation'' for the size of plot
  marks.
\end{textoken}

The following plot marks are predefined (the filling color has been
set to yellow):

\medskip
\begin{tabular}{lc}
  \plotmarkentry{*}
  \plotmarkentry{x}
  \plotmarkentry{+}
\end{tabular}


%%% Local Variables: 
%%% mode: latex
%%% TeX-master: "pgfmanual-pdftex-version"
%%% End: 

% Copyright 2006 by Till Tantau
%
% This file may be distributed and/or modified
%
% 1. under the LaTeX Project Public License and/or
% 2. under the GNU Free Documentation License.
%
% See the file doc/generic/pgf/licenses/LICENSE for more details.


\section{Plot Mark Library}

\begin{pgflibrary}{plotmarks}
  This library defines a number of plot marks.
\end{pgflibrary}

This library defines the following plot marks in
addition to |*|, |x|, and |+| (the filling color has been set to a
dark yellow):

{
\catcode`\|=12
\medskip
\begin{tabular}{lc}
  \plotmarkentry{-}
  \index{*vbar@\protect\texttt{\protect\myvbar} plot mark}%
  \index{Plot marks!*vbar@\protect\texttt{\protect\myvbar}}
  \texttt{\char`\\pgfuseplotmark\char`\{\declare{|}\char`\}} &
  \tikz\draw[color=black!25] plot[mark=|,mark options={fill=yellow,draw=black}]
  coordinates {(0,0) (.5,0.2) (1,0) (1.5,0.2)};\\
  \plotmarkentry{o}
  \plotmarkentry{asterisk}
  \plotmarkentry{star}
  \plotmarkentry{oplus}
  \plotmarkentry{oplus*}
  \plotmarkentry{otimes}
  \plotmarkentry{otimes*}
  \plotmarkentry{square}
  \plotmarkentry{square*}
  \plotmarkentry{triangle}
  \plotmarkentry{triangle*}
  \plotmarkentry{diamond}
  \plotmarkentry{diamond*}
  \plotmarkentry{pentagon}
  \plotmarkentry{pentagon*}
\end{tabular}
}


%%% Local Variables: 
%%% mode: latex
%%% TeX-master: "pgfmanual-pdftex-version"
%%% End: 

% Copyright 2007 by Till Tantau and Mark Wibrow
%
% This file may be distributed and/or modified
%
% 1. under the LaTeX Project Public License and/or
% 2. under the GNU Free Documentation License.
%
% See the file doc/generic/pgf/licenses/LICENSE for more details.

\section{Shadow Library}
\label{section-libs-shadows}

\begin{pgflibrary}{shadows}
  This library defines styles that help adding a (partly) transparent
  shadow to a path or node.
\end{pgflibrary}


\subsection{Overview}

A \emph{shadow} is usually a black or gray area that is drawn behind a
path or a node, thereby adding visual depth to a picture. The shadows
library defines options that make it easy to add shadows to
paths. Internally, these options are based on using the |preaction|
option to use a path twice: Once for drawing the shadow (slightly
shifted) and once for actually using the path.

Note that you can only add shadows to \emph{paths}, not to whole
scopes.

In addition to the general |shadow| option, there exist special
options like |circular shadow|. These can only (sensibly) be used with
a special kind of path (for |circular shadow|, a circle) and, thus,
there are not as general. The advantage is, however, that they are
more visually pleasing since these shadows blend smoothly with the
background. Note that these special shadows use fadings, which few
printers will support.


\subsection{The General Shadow Option}

The shadows are internally created by using a single option called
|general shadow|. The different options like |drop shadow| or
|copy shadow| only differ in the commands that they preset.

You will not need to use this option directly under normal
circumstances.


\begin{key}{/tikz/general shadow=\meta{shadow options} (default \normalfont empty)}
  This option should be given to a |\path| or a |node|. It has the
  following effect: Before the path is used normally, it is used once
  with the \meta{shadow options} in force. Furthermore, when the path
  is ``preused'' in this way, it is shifted and scaled a little bit.

  In detail, the following happens: A |preaction| is used to
  paint the path in a special manner before it is actually
  painted. This ``special'' manner is as follows: The options in
  \meta{shadow options} are used for painting this path. Typically,
  the \meta{shadow options} will contain options like |fill=black| to
  create, say, a black shadow. Furthermore, after the \meta{shadow
    options} have been setup, the following extra canvas
  transformations are applied to the path: It is scaled by
  |shadow scale| (with the origin of scaling at the path's center) and
  it is shifted by |shadow xshift| and |shadow yshift|. 

  Note that since scaling and shifting is done using canvas
  transformations, shadows are not taken into account when the
  picture's bounding box is computed.
\begin{codeexample}[]
\tikz [even odd rule]
  \draw [general shadow={fill=red}] (0,0) circle (.5) (0.5,0) circle (.5);  
\end{codeexample}
  
  \begin{key}{/tikz/shadow scale=\meta{factor} (initially 1)}
    Shadows are scaled by this amount.
\begin{codeexample}[]
\tikz [even odd rule]
  \draw [general shadow={fill=red,shadow scale=1.25}]
    (0,0) circle (.5) (0.5,0) circle (.5);  
\end{codeexample}
  \end{key}
  \begin{key}{/tikz/shadow xshift=\meta{factor} (initially 0pt)}
    Shadows are shifted horizontally by this amount.
\begin{codeexample}[]
\tikz [even odd rule]
  \draw [general shadow={fill=red,shadow xshift=-5pt}]
    (0,0) circle (.5) (0.5,0) circle (.5);  
\end{codeexample}
  \end{key}
  \begin{key}{/tikz/shadow yshift=\meta{factor} (initially 0pt)}
    Shadows are shifted vertically by this amount.
  \end{key}
\end{key}



\subsection{Shadows for Arbitrary Paths and Shapes}

\subsubsection{Drop Shadows}

\begin{key}{/tikz/drop shadow=\meta{shadow options} (default \normalfont empty)}
  This option adds a drop shadow to a path or node. |\path| or a
  |node|. It uses the |general shadow| and passes the \meta{shadow
    options} to it plus, before them, the following extra options:
\begin{codeexample}[code only]
  shadow scale=1, shadow xshift=.5ex, shadow yshift=-.5ex,
  opacity=.5, fill=black!50, every shadow
\end{codeexample}

\begin{codeexample}[]
\tikz [even odd rule]
  \filldraw [drop shadow,fill=white] (0,0) circle (.5) (0.5,0) circle (.5);  
\end{codeexample}
\begin{codeexample}[]
\begin{tikzpicture}
  \foreach \i in {1,...,4}
    \node[starburst,drop shadow,fill=white,draw] at (0,\i) {Burst \i};
\end{tikzpicture}
\end{codeexample}

\begin{codeexample}[]
\begin{tikzpicture}
  \draw [help lines] (0,0) grid (3,2);
  \filldraw [drop shadow={opacity=0.25},fill=white]
    (1,.5) circle (.5) (1.5,.5) circle (.5);  

  \filldraw [drop shadow={opacity=1},fill=white]
    (1,2)  circle (.5) (1.5,2)  circle (.5);  
\end{tikzpicture}
\end{codeexample}
\end{key}

\begin{stylekey}{/tikz/every shadow (initially \normalfont empty)}
  This style is executed in addition to any \meta{shadow options} for
  each shadow. Use this style to reconfigure the way shadows are
  drawn.
\begin{codeexample}[]
\begin{tikzpicture}[every shadow/.style={opacity=.8,fill=blue!50!black}]    
  \filldraw [drop shadow,fill=white] (0,0) circle (.5) (0.5,0) circle (.5);  
\end{tikzpicture}
\end{codeexample}
\end{stylekey}



\subsubsection{Copy Shadows}

A \emph{copy shadow} is not really a shadow. Rather, it looks like
another copy of the path drawn behind the path and a little bit
offset. This creates the visual impression of having multiple copies
of the path/object present.

\begin{key}{/tikz/copy shadow=\meta{shadow options} (default \normalfont empty)}
  This shadow installs the following default options:
\begin{codeexample}[code only]
  shadow scale=1, shadow xshift=.5ex, shadow yshift=-.5ex, every shadow
\end{codeexample}
  Furthermore, the options |fill=|\meta{fill color} and
  |draw=|\meta{draw color} are also set, where the \meta{fill color}
  and \meta{draw color} are the fill and draw colors used for the main 
  path. 
\begin{codeexample}[]
\begin{tikzpicture}
  \node [copy shadow,fill=blue!20,draw=blue,thick] {Hello World!};

  \node at (0,-1) [copy shadow={shadow xshift=1ex,shadow yshift=1ex},
                   fill=blue!20,draw=blue,thick]
    {Hello World!};

  \node at (0,-2) [copy shadow={opacity=.5},tape,
                   fill=blue!20,draw=blue,thick]
    {Hello World!};

  % We have to repeat the left color since shadings are not
  % automatically applied to shadows
  \node at (0,-3) [copy shadow={left color=blue!50},
                   left color=blue!50,draw=blue,thick]
    {Hello World!};
\end{tikzpicture}
\end{codeexample}
\end{key}

\begin{key}{/tikz/double copy shadow=\meta{shadow options} (default \normalfont empty)}
  This shadow works like a |copy shadow|, only the shadow is added
  twice, the first time with the double |xshift| and |yshift|.
\begin{codeexample}[]
\begin{tikzpicture}
  \node [double copy shadow,fill=blue!20,draw=blue,thick] {Hello World!};

  \node at (0,-1) [double copy shadow={shadow xshift=1ex,shadow yshift=1ex},
                   fill=blue!20,draw=blue,thick]
    {Hello World!};

  \node at (0,-2) [double copy shadow={opacity=.5},tape,
                   fill=blue!20,draw=blue,thick]
    {Hello World!};

  \node at (0,-3) [double copy shadow={left color=blue!50},
                   left color=blue!50,draw=blue,thick]
    {Hello World!};
\end{tikzpicture}
\end{codeexample}
\end{key}


\subsection{Shadows for Special Paths and Nodes}

The shadows in this section should normally be added only to paths
that have a special shape. They will look strange with other shapes.

\begin{key}{/tikz/circular drop shadow=\meta{shadow options}}
  This shadow works like a drop shadow, only it adds a circular
  fading to the shadow. This means that the shadow will fade out at
  the border. The following options are preset for this shadow:
\begin{codeexample}[code only]
  shadow scale=1.1, shadow xshift=.3ex, shadow yshift=-.3ex,
  fill=black, path fading={circle with fuzzy edge 15 percent},
  every shadow,
\end{codeexample}

\begin{codeexample}[]
\begin{tikzpicture}
  \foreach \i in {1,...,8}
    \node[circle,circular drop shadow,draw=blue,fill=blue!20,thick]
      at (\i*45:1) {Circle \i};
\end{tikzpicture}
\end{codeexample}
\end{key}


\begin{key}{/tikz/circular glow=\meta{shadow options}}
  This shadow works much like the |circular shadow|, only it is not
  shifted. This creates a visual effect of a ``glow'' behind the
  circle.  The following options are preset for this shadow:
\begin{codeexample}[code only]
  shadow scale=1.25, shadow xshift=0pt, shadow yshift=0pt,
  fill=black, path fading={circle with fuzzy edge 15 percent},
  every shadow,
\end{codeexample}

\begin{codeexample}[]
\begin{tikzpicture}
  \foreach \i in {1,...,8}
  \node[circle,circular glow,fill=red!20,draw=red,thick]
    at (\i*45:1) {Circle \i};
\end{tikzpicture}
\end{codeexample}
\begin{codeexample}[]
\begin{tikzpicture}
  \foreach \i in {1,...,8}
  \node[circle,circular glow={fill=white},fill=red!20,draw=red,thick]
    at (\i*45:1) {Circle \i};
\end{tikzpicture}
\end{codeexample}
\begin{codeexample}[]
\begin{tikzpicture}
  \foreach \i in {1,...,8}
  \node[circle,circular glow={fill=green},fill=black,text=green!50!black]
    at (\i*45:1) {Circle \i};
\end{tikzpicture}
\end{codeexample}
  An especially interesting effect can be achieved by only using the
  glow and not filling the path:
\begin{codeexample}[]
\begin{tikzpicture}
  \foreach \i in {1,...,8}
  \node[circle,circular glow={fill=red!\i0}]
    at (\i*45:1) {Circle \i};
\end{tikzpicture}
\end{codeexample}
\end{key}


%%% Local Variables: 
%%% mode: latex
%%% TeX-master: "pgfmanual-pdftex-version"
%%% End: 

% Copyright 2006 by Till Tantau and Mark Wibrow
%
% This file may be distributed and/or modified
%
% 1. under the LaTeX Project Public License and/or
% 2. under the GNU Free Documentation License.
%
% See the file doc/generic/pgf/licenses/LICENSE for more details.

\section{Shape Library}
\label{section-libs-shapes}

In addition to the standard shapes |rectangle|, |circle| and
|coordinate|, there exist a number of additional shapes defined in
different shape libraries. Several of these shapes have been 
contributed by Mark Wibrow. In the present section, these shapes are
described.


\begin{pgflibrary}{shapes}
  This library packages just includes all of the libraries defined in
  the following. Note that it includes only those libraries starting
  with |shapes.|, more special-purpose libraries are described in
  dedicated sections.
\end{pgflibrary}

\subsection{Rotating shape borders} \label{section-rotating shape borders}

	Some shapes (but not all), support a special kind of rotation. This 
	rotation affects only the border of a shape and is independent of the 
	node contents, but \emph{in addition} to any other transformations.
	
\begin{codeexample}[]
\begin{tikzpicture}
  [every node={dart,shape border uses incircle,inner sep=1pt,draw}]   
  \foreach \a/\b/\c in {A/0/0, B/45/0, C/0/45, D/45/45}
    \node [shape border rotate=\b, rotate=\c] at (\b/36,-\c/36) {\a};
\end{tikzpicture}
\end{codeexample}

	There are two types of rotation: restricted and unrestricted. Which 
	type of rotation is applied is determined by on how the shape border 
	is	constructed. If the shape border is contructed using an incircle, 
	that is, a circle that tightly fits the node contents (including 
	the |inner sep|), then the rotation can be unrestricted. If, however,
	the border is constructed using the natural dimensions of the node
	contents, the rotation is restricted to integer multiples of 90 
	degrees.
	
	Why should there be two kinds of rotation and border construction?
	Borders constructed using the natural dimensions of the node contents
	provide a much tighter fit to the node contents, but to maintain 
	this tight fit, the border rotation must be restricted to multiples 
	of 90 degrees. By using an incircle, unrestricted rotation is 
	possible, but the border will not make a very tight fit to the node 
	contents.
	
\begin{codeexample}[]
\begin{tikzpicture}[every node={isosceles triangle, draw}]
  \node {abc};
  \node [shape border uses incircle] at (2,0) {abc};
\end{tikzpicture}
\end{codeexample}

	There are \pgfname{} commands and \tikzname{} options to rotate a shape
	border and to determine how the shape border is contructed. It should
	be noted that not all shapes support these commands and options, so 
	reference should be made to the documentation for individual 
	shapes. 
	
	The \pgfname{} commands are as follows:

{\let\ifpgfshapeborderusesincircle\relax%
\begin{command}{\ifpgfshapeborderusesincircle}
   Determines if the border of a shape is constructed using an 
   incircle. 
\end{command}
}

\begin{command}{\pgfsetshapeborderrotate}
   Rotates the border of a shape independently of the node contents,
   but in addition to any other transformations. If 
   |\ifpgfshapeborderusesincircle| is false, the rotation will be
   rounded to the nearest integer multiple of 90 degrees when the
   shape is drawn. 
\end{command}

	The corresponding \tikzname{} options are:
	
\begin{itemize}

	\itemoption{shape border uses incircle}|=|\meta{true}|/|\meta{false}
	\itemoption{shape border rotate}|=|\meta{angle}
	
\end{itemize}

	It should also be noted that if the border of the shape is rotated, 
	the compass point anchors, and `text box' anchors (including 
	|mid east|, |base west|, and so on), \emph{do not rotate}, but the 
	other anchors do:
	
\begin{codeexample}[]
\begin{tikzpicture}
  [every node/.style={trapezium,draw,shape border uses incircle}]
  \node at (0,0)  (A) {\Large A};
  \node [shape border rotate=30] at (1.5,0) (B) {\Large B};
  \foreach \s/\t in 
    {left side/base east, bottom side/north, bottom left corner/base}{
       \fill[red]  (A.\s) circle(1.5pt) (B.\s) circle(1.5pt);
       \fill[blue] (A.\t) circle(1.5pt) (B.\t) circle(1.5pt);
  }
\end{tikzpicture}
\end{codeexample}


\subsection{Predefined Shapes}
\label{section-predefined-shapes}

The three shapes |rectangle|, |circle|, and |coordiante| are always
defined and no library needs to be loaded for them. while the
|coordinate| shape defines only the |center| anchor, the other two
shapes define a standard set of anchors.

\begin{shape}{circle}
  This shape draws a tightly fitting circle around the text. The
  following figure shows the anchors this shape defines; the anchors
  |10| and |130| are example of border anchors. 
\begin{codeexample}[]
\Huge
\begin{tikzpicture}
  \node[name=s,shape=circle,shape example] {Circle\vrule width 1pt height 2cm};
  \foreach \anchor/\placement in
    {north west/above left, north/above, north east/above right, 
     west/left, center/above, east/right, 
     mid west/right, mid/above, mid east/left, 
     base west/left, base/below, base east/right, 
     south west/below left, south/below, south east/below right, 
     text/left, 10/right, 130/above}
     \draw[shift=(s.\anchor)] plot[mark=x] coordinates{(0,0)}
       node[\placement] {\scriptsize\texttt{(s.\anchor)}};
\end{tikzpicture}
\end{codeexample}
\end{shape}

\begin{shape}{rectangle}
  This shape, which is the standard, is a rectangle around the
  text. The inner   and outer separations (see
  Section~\ref{section-shape-seps}) influence the white space around
  the text. The following figure shows the anchors this
  shape defines; the anchors |10| and |130| are example of border anchors.
\begin{codeexample}[]
\Huge
\begin{tikzpicture}
  \node[name=s,shape=rectangle,shape example] {Rectangle\vrule width 1pt height 2cm};
  \foreach \anchor/\placement in
    {north west/above left, north/above, north east/above right, 
     west/left, center/above, east/right, 
     mid west/right, mid/above, mid east/left, 
     base west/left, base/below, base east/right, 
     south west/below left, south/below, south east/below right, 
     text/left, 10/right, 130/above}
     \draw[shift=(s.\anchor)] plot[mark=x] coordinates{(0,0)}
       node[\placement] {\scriptsize\texttt{(s.\anchor)}};
\end{tikzpicture}
\end{codeexample}
\end{shape}


\subsection{Geometric Shapes}

\begin{pgflibrary}{shapes.geometric}
  This library defines different shapes that correspond to basic
  geometric objects like ellipses or polygons.
\end{pgflibrary}



\begin{shape}{diamond}
  This shape is a diamond tightly fitting the text box. The ratio
  between width and height is 1 by default, but can be changed by
  setting the shape aspect ratio (using the following command or the
  |aspect| option of \tikzname). The following figure shows the anchors this
  shape defines; the anchors |10| and |130| are example of border
  anchors.

  \begin{command}{\pgfsetshapeaspect\marg{value}}
    This command sets the macro \declare{|\pgfshapeaspect|} to
    \meta{value}. Furthermore, \declare{|\pgfshapeaspectinverse|} is set
    to the reciprocal of \meta{value}. The aspect is a recommendation
    for the quotient of the width and the height of a shape.
  \end{command}

\begin{codeexample}[]
\Huge
\begin{tikzpicture}
  \node[name=s,shape=diamond,shape example] {Diamond\vrule width 1pt height 2cm};
  \foreach \anchor/\placement in
    {north west/above left, north/above, north east/above right, 
     west/left, center/above, east/right, 
     mid/above, 
     base/below,  
     south west/below left, south/below, south east/below right, 
     text/left, 10/right, 130/above}
     \draw[shift=(s.\anchor)] plot[mark=x] coordinates{(0,0)}
       node[\placement] {\scriptsize\texttt{(s.\anchor)}};
\end{tikzpicture}
\end{codeexample}
\end{shape}

\begin{shape}{ellipse}
  This shape is an ellipse tightly fitting the text box, if no inner
  separation is given. The following figure shows the anchors this
  shape defines; the anchors |10| and |130| are example of border anchors.
\begin{codeexample}[]
\Huge
\begin{tikzpicture}
  \node[name=s,shape=ellipse,shape example] {Ellipse\vrule width 1pt height 2cm};
  \foreach \anchor/\placement in
    {north west/above left, north/above, north east/above right, 
     west/left, center/above, east/right, 
     mid west/right, mid/above, mid east/left, 
     base west/left, base/below, base east/right, 
     south west/below left, south/below, south east/below right, 
     text/left, 10/right, 130/above}
     \draw[shift=(s.\anchor)] plot[mark=x] coordinates{(0,0)}
       node[\placement] {\scriptsize\texttt{(s.\anchor)}};
\end{tikzpicture}
\end{codeexample}
\end{shape}





\begin{shape}{trapezium}
	This shape is a trapezium, that is, a quadrilateral with a single
	pair of parallel lines (this can sometimes be known as a trapezoid).
	The |trapezium| shape supports the rotation of the shape border, as 
	described in Section~\ref{section-rotating shape borders}. 
   
   By default, the the lower corners of the trapezium are extended 
	(i.e., `stick out') by a distance of 1.5ex, but each can be extended 
	independently.	

	
\begin{codeexample}[]
\begin{tikzpicture}[every node/.style={trapezium, draw}]
   \node at (0,1) {A};
   \node[trapezium left extension=10pt, trapezium right extension=20pt]
         at (0,0) {B};
\end{tikzpicture}
\end{codeexample}

   Negative extensions result in the upper points being extended. This
   means that parallelograms can be created (but technically these are
   not trapezia, as they have more than one pair of parallel sides).
   These extensions are in addition to the natural dimensions of the
   node contents.
   
\begin{codeexample}[]
\begin{tikzpicture}[every node/.style={trapezium, draw}]
   \node[trapezium left extension=10pt, trapezium right extension=20pt]
         at (0,2) {A};
   \node[trapezium left extension=-10pt, trapezium right extension=-20pt]
         at (0,1) {B};
   \node[trapezium left extension=-10pt, trapezium right extension=10pt]
         at (0,0) {C};
\end{tikzpicture}
\end{codeexample}

	There are \pgfname{} commands and \tikzname{} options to set the 
  	extensions of the trapezium. The \pgfname{} commands are as follows:
	
	\begin{command}{\pgftrapeziumleftextension\marg{length}}
    Set the distance that the lower left corner `sticks out'. If 
    \meta{length} is negative, the upper left corner is adjusted.
   \end{command}
   
   \begin{command}{\pgftrapeziumrightextension\marg{length}}
    Set the distance that the lower right corner `sticks out'. If 
    \meta{length} is negative, the upper right corner is adjusted.
   \end{command}
   
   The corresponding \tikzname{} options are:

  \begin{itemize}
    \itemoption{trapezium left extension}|=|\meta{length}
    set the extension for the lower left corner (negative values adjust
    the upper left corner).
    
    \itemoption{trapezium right extension}|=|\meta{length}
    set the extension for the lower right corner (negative values adjust
    the upper right corner).
    
    \itemoption{trapezium extension}|=|\meta{length}
    set the extension for both lower corners (negative values adjust
    the upper corners).
    
  \end{itemize}
  
   The anchors for the trapezium are shown below. The anchor |150| is an
	example of a border anchor.

\begin{codeexample}[]
\Huge
\begin{tikzpicture}
  \node[name=s, shape=trapezium, trapezium extension=2cm, shape example, inner sep=1.5cm] 
    {Trapezium\vrule width 1pt height 2cm};
  \foreach \anchor/\placement in
    {bottom left corner/below, top right corner/right, 
     top left corner/left,     bottom right corner/below,
     bottom side/below,        left side/left, 
     right side/right,         top side/above,
     center/above,   text/below,      mid/right,       base/below, 
     mid west/right, base west/below, mid east/left,   base east/below, 
     west/above,     east/above,      north/below,     south/above,
     north west/above, north east/above, 
     south west/below, south east/below, 150/above}    
  \draw[shift=(s.\anchor)] plot[mark=x] coordinates{(0,0)}
    node[\placement] {\scriptsize\texttt{(s.\anchor)}};
\end{tikzpicture}
\end{codeexample}  
   
\end{shape}





\begin{shape}{semicircle}
	
	This shape is a semicircle, which tightly fits the node contents.
	This shape supports the rotation of the shape border, as described in 
	Section~\ref{section-rotating shape borders}.
	The anchors for the |semicircle| shape are shown below. 
	Anchor |30| is an example of a border anchor.
	
\begin{codeexample}[]
\Huge
\begin{tikzpicture}
  \node[name=s,shape=semicircle,shape border rotate=0,shape example, inner sep=1cm] 
  	{Semicircle\vrule width 1pt height 2cm};
  \foreach \anchor/\placement in
    {apex/above,      arc start/below, arc end/below,  chord center/below,
     center/above,    base/below,      mid/right,      text/left,
     base west/below, base east/below, mid west/left, mid east/right, 
     north/below,     south/above,     east/above,     west/above,
     north west/above left, north east/above right,
     south west/below,      south east/below, 30/right}
     \draw[shift=(s.\anchor)] plot[mark=x] coordinates{(0,0)}
       node[\placement] {\scriptsize\texttt{(s.\anchor)}};
\end{tikzpicture}
\end{codeexample}
\end{shape}





\begin{shape}{regular polygon}
  This shape is a regular polygon, which, by default, is drawn so that 
  a side (rather than a corner) is always at the bottom. 
  This shape supports the rotation as described in 
  Section~\ref{section-rotating shape borders}, but the border of the 
  polygon is \emph{always} constructed using the incircle, whose
  radius is calculated to tightly fit the node contents. (including
  any |inner sep|).
  
\begin{codeexample}[]
\begin{tikzpicture}
  \foreach \a in {3,...,7}{
    \draw[gray!50] (\a*2,0)  circle(0.5cm);
    \node[regular polygon, regular polygon sides=\a,
          inner sep=0cm, text=red!50, draw] 
          at (\a*2,0) {\vrule height 0.707cm width 0.707cm};
   }  
\end{tikzpicture}
\end{codeexample}	
	
  If the node is enlarged to any specified minimum size, 
  this is interpreted as the diameter of the the 
  circumcircle, that is, the circle that passes through all the 
  corners of the polygon border.

\begin{codeexample}[]
\begin{tikzpicture}
  \foreach \a in {3,...,7}{
    \draw[gray!50] (\a*2,0)  circle(0.5cm);
    \node[regular polygon, regular polygon sides=\a, minimum size=1cm, draw] at (\a*2,0) {};
   }  
\end{tikzpicture}
\end{codeexample}	

  There is a \pgfname{} command and \tikzname{} option to set the 
  number of sides for the polygon. The \pgfname{} command is as 
  follows:
	
  \begin{command}{\pgfsetpolygonsides\marg{integer}}
  \end{command}
  
  The corresponding \tikzname{} option is:

  \begin{itemize}
    \itemoption{regular polygon sides}|=|\meta{integer}
  \end{itemize}
  
  The anchors for the regular polygon shape are shown below.  
  The anchor |75| is an example of a border anchor.
  
\begin{codeexample}[]
\Huge
\begin{tikzpicture}
  \node[name=s, shape=regular polygon, regular polygon sides=5, shape example, inner sep=.5cm] 
    {Regular Polygon\vrule width 1pt height 2cm};
  \foreach \anchor/\placement in
    {corner 1/above, corner 2/above, corner 3/left, corner 4/right, corner 5/above, 
     side 1/above,   side 2/left,    side 3/below,  side 4/right,   side 5/above,  
     center/above, text/left,  mid/right,   base/below, 75/above,
     west/above,   east/above, north/below, south/above,
     north east/below, south east/above, north west/below, south west/above}
  \draw[shift=(s.\anchor)] plot[mark=x] coordinates{(0,0)}
    node[\placement] {\scriptsize\texttt{(s.\anchor)}};
\end{tikzpicture}
\end{codeexample}

\end{shape}

\begin{shape}{star}
  This shape is a star, which by default (minus any transformations) is
  drawn with the first point pointing upwards.  
  This shape supports the rotation as described in 
  Section~\ref{section-rotating shape borders}, but the border of the 
  star is \emph{always} constructed using the incircle.
  
  A star should be thought of as having an set of ``inner points'' and
  and ``outer points''. 
  The inner points of the border are based on the radius of the circle
  which tightly fits the node contents, and the outer points are based
  on the circumcircle, the circle that passes through every outer
  point.
  Any specified minimum size, width or height, is interpreted as the 
  diameter of the circumcircle.
 
\begin{codeexample}[]
\begin{tikzpicture}
   \draw [help lines]   (0,0) grid (2,2);
   \draw [red, dashed]  (1,1) circle(1cm);
   \draw [blue, dashed] (1,1) circle(.5cm);
   \node [star, star point height=.5cm, minimum size=2cm, draw] 
       at (1,1) {S};
\end{tikzpicture}
\end{codeexample} 
  
  There are \pgfname{} commands and \tikzname{} options to set the 
  number of points, and the height of the star points.
  The \pgfname{} commands are as follows:
  
  \begin{command}{\pgfsetstarpoints\marg{integer}}
    Sets the number of points for the star.
  \end{command}
  
  \begin{command}{\pgfsetstarpointheight\marg{distance}}
    Sets the height of the star points. This is the distance between the
    inner point and outer point radii. If the star is enlarged to some
    specified minimum size, the inner radius is increased to maintain
    the point height.	
  \end{command}
  
  \begin{command}{\pgfsetstarpointratio\marg{number}}
    Sets the ratio between the inner point and outer point radii.		
    If the star is enlarged to some specified minimum size, the
    inner radius is increased to maintain the ratio.	
  \end{command}
  
  The corresponding \tikzname{} options are:
  
  \begin{itemize}
    \itemoption{star points}|=|\meta{integer}
    set the number of points for the star.
    
    \itemoption{star point height}|=|\meta{distance}
    set the height of the points for the star.
    
    \itemoption{star point ratio}|=|\meta{number}
    set the ratio between the outer point radius and the inner point
    radius.
    
  \end{itemize}

	The inner and outer points form the principle anchors for the star,
   as shown below (anchor |75| is an example of a border anchor).
  
  \begin{codeexample}[]
\Huge
\begin{tikzpicture}
  \node[name=s, shape=star, star points=5, star point ratio=1.65, shape example, inner sep=1.5cm] 
    {Star\vrule width 1pt height 2cm};
  \foreach \anchor/\placement in
     {inner point 1/above, inner point 2/above, inner point 3/below, inner point 4/right, 
      inner point 5/above, outer point 1/above, outer point 2/above, outer point 3/left,  
      outer point 4/right, outer point 5/above,
      center/above, text/left,  mid/right,   base/below, 75/above,
     	west/above,   east/above, north/below, south/above,
     	north east/below, south east/above, north west/below, south west/above}
  \draw[shift=(s.\anchor)] plot[mark=x] coordinates{(0,0)}
    node[\placement] {\scriptsize\texttt{(s.\anchor)}};
\end{tikzpicture}
\end{codeexample}
\end{shape}





\begin{shape}{isosceles triangle}
	This shape is an isosceles triangle, which supports the rotation of 
	the shape border, as described in 
	Section~\ref{section-rotating shape borders}.
	
	Minimum size and minimum width requirements ensure the literal width 
	and height of the isosceles triangle, but are applied as if the 
	triangle is rotated to 90 degrees (i.e., with the apex pointing up). 
	In order to keep the apex angle the same, increasing the height will 
	increase the width and vice versa. 
   
\begin{codeexample}[]
\begin{tikzpicture}
  [every node/.style={
    isosceles triangle,
    draw,
    inner sep=0pt, 
    anchor=left corner,
    shape border rotate=90}]
   \draw[help lines] grid(4,2);
   \foreach \a/\c in {1.5/blue, 1/green, 0.5/red}{
      \color{\c}
      \node[minimum height=\a cm] at (0,0) {};
      \node[minimum width=\a cm] at (2,0) {};
   }
\end{tikzpicture}
\end{codeexample}	

	There is a \pgfname{} command and \tikzname{} option to set the 
  	apex angle of the triangle. The \pgfname{} command is as follows:
    
  \begin{command}{\pgfsetisoscelestriangleapexangle\marg{angle}}
    Sets the angle of the apex of the isosceles triangle. The height
    and width of the triangle may be adjusted to maintain this
    angle.
  \end{command}
  
  The \tikzname{} option is:
  
  \begin{itemize}
    \itemoption{isosceles triangle apex angle}|=|\meta{angle}
    set the angle of the apex of the isosceles triangle.
   \end{itemize}
 
	The anchors for the |isosceles triangle| are shown below (the border 
	has been rotated 90 degrees anticlockwise). Anchor |150| is an
	example of a border anchor. 
\begin{codeexample}[]
\Huge
\begin{tikzpicture}
  \node[name=s, shape=isosceles triangle, shape border uses incircle, shape example, 
        shape border rotate=90, inner sep=0cm] 
    {Isosceles Triangle\vrule width 1pt height 2cm};
  \foreach \anchor/\placement in
    {apex/above,      left corner/above right, right corner/above left,
     left side/above, right side/above,        lower side/below,    
     center/above,  text/left,       mid/right,      base/below, 
     mid west/left, base west/below, mid east/right, base east/below,
     west/above,    east/above,      north/below,    south/above,
     north west/left,  north east/right, 
     south west/below, south east/below, 150/above}  
  \draw[shift=(s.\anchor)] plot[mark=x] coordinates{(0,0)}
    node[\placement] {\scriptsize\texttt{(s.\anchor)}};
\end{tikzpicture}
\end{codeexample} 

\end{shape}


\par\leavevmode
\begin{shape}{kite}

	This shape is a kite, which supports the rotation of the shape border, 
	as described in Section~\ref{section-rotating shape borders}. 
	There are \pgfname{} commands and \tikzname{} 
	options to specify the upper and lower	vertex angles of the kite. 
	The \pgfname{} commands are as follows:
	
	\begin{command}{\pgfsetkiteuppervertexangle\marg{angle}}
	Set the upper internal angle of the kite.
	\end{command}
	
	\begin{command}{\pgfsetkitelowervertexangle\marg{angle}}
	Set the lower internal angle of the kite.
	\end{command}
	
	The corresponding \tikzname{} options are as follows:
	
	\begin{itemize}
    \itemoption{kite upper vertex angle}|=|\meta{integer}
    \itemoption{kite lower vertex angle}|=|\meta{angle}
    \itemoption{kite vertex angles}|={|\meta{angle specification}|}|
    
    \meta{angle specification} can be a comma separated pair 
    |{|\meta{upper angle}|,|\meta{lower angle}|}|, or a single angle.
    In this latter case, both the upper and lower vertex angles will 
    be the same (and in addition, the braces |{}| can be omitted).
    
\begin{codeexample}[]
\begin{tikzpicture}[every node/.style={kite, draw}]
  \node[kite upper vertex angle=135, kite lower vertex angle=70] at (0,0) {A};
  \node[kite vertex angles={90,45}] at (1,0) {B};
  \node[kite vertex angles=60]      at (2,0) {C};
\end{tikzpicture}
\end{codeexample}

 \end{itemize}

	The anchors for the |kite| are shown below. Anchor |110| is an 
	example of a border anchor.
	
\begin{codeexample}[]
\Huge
\begin{tikzpicture}
  \node[name=s, shape=kite, shape example, inner sep=1.5cm] 
    {Kite\vrule width 1pt height 2cm};
  \foreach \anchor/\placement in
    {upper vertex/above, left vertex/above,    lower vertex/below, 
     right vertex/above, upper left side/above, upper right side/above,
     lower left side/below, lower right side/below,
     center/above,   text/left,       mid/right,        base/below, 
     mid west/left,  base west/below, mid east/right,   base east/below,
     west/above,     east/above,      north/below,     south/above,
     north west/left, north east/right, 
     south west/above, south east/above, 110/above}  
  \draw[shift=(s.\anchor)] plot[mark=x] coordinates{(0,0)}
    node[\placement] {\scriptsize\texttt{(s.\anchor)}};
\end{tikzpicture}
\end{codeexample}
\end{shape}


\begin{shape}{dart}


	This shape is a dart (which can also be known as an arrowhead or
	concave kite). This shape supports the rotation of the shape border, 
	as described in Section~\ref{section-rotating shape borders}. 
	The angle of the border rotation determines the direction in which 
	the dart points (unless other transformations have been applied).
	
	There are \pgfname{} commands and \tikzname{} options, to set the 
	angle for the `tip' of the dart and the angle between the `tails'
	of the dart. These angles default to 45 degrees and 135 degrees 
	respectively.

\begin{codeexample}[]
\begin{tikzpicture}
   \node[dart, draw, gray, shape border uses incircle, shape border rotate=45] 
       (d) {dart};
   \draw [<->] (d.tip)++(202.5:.5cm) arc(202.5:247.5:.5cm);
   \node [left of=.5cm] at (d.tip) {tip angle};
   \draw [<->] (d.tail center)++(157.5:.5cm) arc(157.5:292.5:.5cm);
   \node [right] at (d.tail center) {tail angle};
\end{tikzpicture}
\end{codeexample}

	The \pgfname{} commands are as follows:
	
	\begin{command}{\pgfsetdarttipangle\marg{angle}}
		Set the angle at the tip of the dart.
	\end{command}
	
	\begin{command}{\pgfsetdarttailangle\marg{angle}}
		Set the angle between the tails of the dart.
	\end{command}
	
	The corresponding \tikzname{} options are:
	
	\begin{itemize}
		\itemoption{dart tip angle}|=|\meta{angle}
		
		\itemoption{dart tail angle}|=|\meta{angle}		
	\end{itemize}
	
	The anchors for the |dart| shape are shown below (note that the 
	shape is rotated 90 degrees anti-clockwise). Anchor |110| is an 
	example of a border anchor.
\begin{codeexample}[]
\Huge
\begin{tikzpicture}
  \node[name=s, shape=dart, shape border rotate=90, shape example, inner sep=1.25cm] 
    {Dart\vrule width 1pt height 2cm};
  \foreach \anchor/\placement in
    {tip/above,       tail center/below, right tail/below, 
     left tail/below, right tail/below,  left side/left,   right side/right,
     center/above,    text/left,         mid/right,        base/below, 
     mid west/left,   base west/below,   mid east/right,   base east/below,
     west/above,      east/above,        north/below,      south/above,
     north west/left, north east/right,  south west/above, south east/above,
     110/above}    
  \draw[shift=(s.\anchor)] plot[mark=x] coordinates{(0,0)}
    node[\placement] {\scriptsize\texttt{(s.\anchor)}};
\end{tikzpicture}
\end{codeexample}
\end{shape}




\begin{shape}{circular sector}

	This shape is a circular sector (which can also be known as a
	wedge).
	This shape supports the rotation of the shape border, 
	as described in Section~\ref{section-rotating shape borders}. 
	The angle of the border rotation determines the direction in which 
	the `apex' of the sector points (unless other transformations have 
	been applied).
	
\begin{codeexample}[]
\begin{tikzpicture}
  [every node/.style={circular sector, shape border uses incircle, draw}]
  \node at (0,0) {A};
  \node [shape border rotate=30] at (1.5,0) {A};
\end{tikzpicture}
\end{codeexample}

	There is a \pgfname{} command and \tikzname{} option to set the 
	central angle of the sector, which is expected to be less than 180
	degrees. This central angle defaults to 60 degrees.
	
	\begin{command}{\pgfsetcircularsectorangle\marg{angle}}
		Set the central angle of the sector. 
	\end{command}
	
	The corresponding \tikzname{} option is:
	
	\begin{itemize}
		\itemoption{circular sector angle}|=|\meta{angle}		
	\end{itemize}
	
	The anchors for the |circular sector| shape are shown below.
	Anchor |30| is an example of a border anchor.
	
\begin{codeexample}[]
\Huge
\begin{tikzpicture}
  \node[name=s,shape=circular sector,  style=shape example, inner sep=1cm] 
  	{Circular Sector\vrule width 1pt height 2cm};
  \foreach \anchor/\placement in
   {sector center/above, arc start/below, arc end/below, arc center/below,
    center/above,        base/below,      mid/right,     text/below,
    north/below,         south/above,     east/below,    west/above,
    north west/above left, north east/above right,
    south west/below,      south east/below, 30/right}
     \draw[shift=(s.\anchor)] plot[mark=x] coordinates{(0,0)}
       node[\placement] {\scriptsize\texttt{(s.\anchor)}};
\end{tikzpicture}
\end{codeexample}
\end{shape}





\subsection{Symbol Shapes}

\begin{pgflibrary}{shapes.symbols}
  This library defines shapes that can be used for drawing symbols
  like a forbidden sign or a cloud.
\end{pgflibrary}



\begin{shape}{forbidden sign}
  This shape places the node inside a circle with a diagonal from the
  lower left to the upper right added. The circle is part of the
  background, the diagonal line part of the foreground path; thus, the
  diagonal line is on top of the text.
  
\begin{codeexample}[]
\begin{tikzpicture}
  \node [forbidden sign,line width=1ex,draw=red,fill=white] {Smoking};
\end{tikzpicture}
\end{codeexample}

  The shape inherits all anchors from the |circle| shape, see also the
  following figure:
\begin{codeexample}[]
\Huge
\begin{tikzpicture}
  \node[name=s,shape=forbidden sign,shape example] {Forbidden\vrule width 1pt height 2cm};
  \foreach \anchor/\placement in
    {north west/above left, north/above, north east/above right, 
     west/left, center/above, east/right, 
     mid west/right, mid/above, mid east/left, 
     base west/left, base/below, base east/right, 
     south west/below left, south/below, south east/below right, 
     text/left, 10/right, 130/above}
     \draw[shift=(s.\anchor)] plot[mark=x] coordinates{(0,0)}
       node[\placement] {\scriptsize\texttt{(s.\anchor)}};
\end{tikzpicture}
\end{codeexample}
\end{shape}



\subsection{Shapes with Multiple Text Parts}

\begin{pgflibrary}{shapes.multipart}
  This library defines general-purpose shapes that are composed of
  multiple (text) parts. 
\end{pgflibrary}


\begin{shape}{circle split}
  This shape is a multi-part shape consisting of a circle with a line
  in the middle. The upper part is the main part (the |text| part),
  the lower part is the |lower| part.
  
\begin{codeexample}[]
\begin{tikzpicture}
  \node [circle split,draw,double,fill=red!20]
  {
    $q_1$
    \nodepart{lower}
    $00$
  };
\end{tikzpicture}
\end{codeexample}

  The shape inherits all anchors from the |circle| shape and defines
  the |lower| anchor in addition. See also the
  following figure:
\begin{codeexample}[]
\Huge
\begin{tikzpicture}
  \node[name=s,shape=circle split,shape example] {text\nodepart{lower}lower};
  \foreach \anchor/\placement in
    {north west/above left, north/above, north east/above right, 
     west/left, center/below, east/right, 
     mid west/right, mid/above, mid east/left, 
     base west/left, base/below, base east/right, 
     south west/below left, south/below, south east/below right, 
     text/left, lower/left, 130/above}
     \draw[shift=(s.\anchor)] plot[mark=x] coordinates{(0,0)}
       node[\placement] {\scriptsize\texttt{(s.\anchor)}};
\end{tikzpicture}
\end{codeexample}
\end{shape}


\subsection{Miscellaneous Shapes}

\begin{pgflibrary}{shapes.misc}
  This library defines general-purpose shapes that do not fit in the
  previous categories.
\end{pgflibrary}



\begin{shape}{cross out}
  This shape ``crosses out'' the node. Its foreground path are simply
  two diagonal lines that between the corners of the node's bounding
  box. Here is an example:

\begin{codeexample}[]
\begin{tikzpicture}
  \draw[help lines] (0,0) grid (3,2);
  \node [cross out,draw=red] at (1.5,1) {cross out};
\end{tikzpicture}
\end{codeexample}

  A useful application is inside text as in the following example:
\begin{codeexample}[]
Cross \tikz[baseline] \node [cross out,draw,anchor=text] {me}; out!  
\end{codeexample}

  This shape inherits all anchors from the |rectangle| shape, see also
  the following figure:
\begin{codeexample}[]
\Huge
\begin{tikzpicture}
  \node[name=s,shape=cross out,shape example] {cross out\vrule width 1pt height 2cm};
  \foreach \anchor/\placement in
    {north west/above left, north/above, north east/above right, 
     west/left, center/above, east/right, 
     mid west/right, mid/above, mid east/left, 
     base west/left, base/below, base east/right, 
     south west/below left, south/below, south east/below right, 
     text/left, 10/right, 130/above}
     \draw[shift=(s.\anchor)] plot[mark=x] coordinates{(0,0)}
       node[\placement] {\scriptsize\texttt{(s.\anchor)}};
\end{tikzpicture}
\end{codeexample}
\end{shape}

\begin{shape}{strike out}
  This shape is idential to the |cross out| shape, only its foreground
  path consists of a single line from the lower left to the upper
  right.
  
\begin{codeexample}[]
Strike \tikz[baseline] \node [strike out,draw,anchor=text] {me}; out!  
\end{codeexample}

  See the |cross out| shape for the anchors.
\end{shape}



%%% Local Variables: 
%%% mode: latex
%%% TeX-master: "pgfmanual-pdftex-version"
%%% End: 

%% Copyright 2006 by Till Tantau
%
% This file may be distributed and/or modified
%
% 1. under the LaTeX Project Public License and/or
% 2. under the GNU Free Documentation License.
%
% See the file doc/generic/pgf/licenses/LICENSE for more details.


\section{Three Dimensional Drawing Library}

\begin{tikzlibrary}{3d}
  This packages, which is still under construction, provides some
  styles and options for drawing three dimensional shapes.

  \textbf{This package is not officially released and commands may
    change or disappear without backward compatibility support.}
\end{tikzlibrary}

This package is not yet documented. Here is, at least, an example of
where this whole thing is supposed to head.

\begin{codeexample}[]
\begin{tikzpicture}[z={(10:10mm)},x={(-45:5mm)}]  
  \def\wave{
    \draw[fill,thick,fill opacity=.2]
     (0,0) sin (1,1) cos (2,0) sin (3,-1) cos (4,0) 
           sin (5,1) cos (6,0) sin (7,-1) cos (8,0)
           sin (9,1) cos (10,0)sin (11,-1)cos (12,0);
    \foreach \shift in {0,4,8}
    {
      \begin{scope}[xshift=\shift cm,thin]
        \draw (.5,0)  -- (0.5,0 |- 45:1cm);
        \draw (1,0)   -- (1,1);
        \draw (1.5,0) -- (1.5,0 |- 45:1cm);
        \draw (2.5,0) -- (2.5,0 |- -45:1cm);
        \draw (3,0)   -- (3,-1);
        \draw (3.5,0) -- (3.5,0 |- -45:1cm);
      \end{scope}
    }
  }
  \begin{scope}[canvas is zy plane at x=0,fill=blue]
    \wave
    \node at (6,-1.5) [transform shape] {magnetic field};
  \end{scope}
  \begin{scope}[canvas is zx plane at y=0,fill=red]
    \draw[help lines] (0,-2) grid (12,2);
    \wave
    \node at (6,1.5) [rotate=180,xscale=-1,transform shape] {electric field};
  \end{scope}
\end{tikzpicture}
\end{codeexample}

\begin{codeexample}[]
\begin{tikzpicture}
  \begin{scope}[canvas is zy plane at x=0]
    \draw (0,0) circle (1cm);
    \draw (-1,0) -- (1,0) (0,-1) -- (0,1);
  \end{scope}

  \begin{scope}[canvas is zx plane at y=0]
    \draw (0,0) circle (1cm);
    \draw (-1,0) -- (1,0) (0,-1) -- (0,1);
  \end{scope}

  \begin{scope}[canvas is xy plane at z=0]
    \draw (0,0) circle (1cm);
    \draw (-1,0) -- (1,0) (0,-1) -- (0,1);
  \end{scope}
\end{tikzpicture}
\end{codeexample}





%%% Local Variables: 
%%% mode: latex
%%% TeX-master: "pgfmanual-pdftex-version"
%%% End: 

% Copyright 2006 by Till Tantau
%
% This file may be distributed and/or modified
%
% 1. under the LaTeX Project Public License and/or
% 2. under the GNU Free Documentation License.
%
% See the file doc/generic/pgf/licenses/LICENSE for more details.

\section{To Path Library}

\label{library-to-paths}

\begin{tikzlibrary}{topaths}
  This library provides predefined to paths for use with the |to|
  path operation. After loading this package, you can say for instance
  |to [loop]| to add a loop to a node.

  This library is loaded automatically by \tikzname, so you do not
  need to load it yourself.
\end{tikzlibrary}


\subsection{Straight Lines}

The following style installs a to path that is simply a straight line
from the start coordinate to the target coordinate.

\begin{itemize}
  \itemstyle{line to}
  causes a straight line to be added to the path upon a |to| or an
  |edge| operation.
\begin{codeexample}[]
\tikz {\draw (0,0) to[line to] (1,0);}
\end{codeexample}
\end{itemize}


\subsection{Curves}

The |curve to| style causes the to path to be set to a curve. The
exact way this curve looks can be influenced via a number of options.

\begin{itemize}
  \itemstyle{curve to}
  Specifies that the |to path| should be a curve. This curve will
  leave the start coordinate at a certain angle, which can be
  specified using the |out| option. It reaches the target coordinate
  also at a certain angle, which is specified using the |in|
  option. The control points of the curve are at a certain distance
  that is computed in different ways, depending on which options are
  set.

  All of the following options implictly cause the |curve to| style to
  be installed.
  \itemoption{out}|=|\meta{angle}
  The angle at which the curve leaves the start coordinate. If the
  start coordinate is a node, the start coordinate is the point on the
  border of the node at the given \meta{angle}. The control point
  will, thus, lie at a certain distance in the direction \meta{angle}
  from the start coordinate.
\begin{codeexample}[]
\begin{tikzpicture}[out=45,in=135]
  \draw (0,0) to (1,0)
        (0,0) to (2,0)
        (0,0) to (3,0);
\end{tikzpicture}
\end{codeexample}
  \itemoption{in}|=|\meta{angle}
  The angle at which the curve reaches the target coordinate.

  \itemoption{relative}\opt{|=|\meta{true or false}}
  This option tells \tikzname\ whether the |in| and |out| angles
  should be considered absolute or relative. Absolute means that an
  |out| angle of 30$^\circ$ means that the curve leaves the start
  coordinate at an angle of 30$^\circ$ relative to the paper (unless,
  of course, further transformations have been installed). A
  \emph{relative} angle is, by comparison, measured relative to a
  straight line from the start coordinate to the target
  coordinate. Thus, a relative angle of 30$^\circ$ means that the
  curve will bend to the left from the line going straight from the
  start to the target. For the target, the relative coordinate is
  measured in the same manner, namely relative to the line going from
  the start to the target. Thus, an angle of 150$^\circ$ means that
  the curve will reach target coming slightly from the left.

\begin{codeexample}[]
\begin{tikzpicture}[out=45,in=135,relative]
  \draw (0,0) to (1,0)
              to (2,1)
              to (2,2);
\end{tikzpicture}
\end{codeexample}

\begin{codeexample}[]
\begin{tikzpicture}[out=90,in=90,relative]
  \node [circle,draw] (a) at (0,0) {a};
  \node [circle,draw] (b) at (1,1) {b};
  \node [circle,draw] (c) at (2,2) {c};

  \path (a) edge (b)
            edge (c);
\end{tikzpicture}
\end{codeexample}

  \itemoption{bend left}\opt{|=|\meta{angle}}
  This option sets |out=|\meta{angle}|,in=|$180-\meta{angle}$|,relative|. If no
  \meta{angle} is given, the last given |bend left| or |bend right|
  angle is used.  
  
\begin{codeexample}[]
\begin{tikzpicture}[shorten >=1pt,node distance=2cm]
  \node[state,initial]  (q_0)                {$q_0$};
  \node[state]          (q_1) [right of=q_0] {$q_1$};
  \node[state,accepting](q_2) [right of=q_1] {$q_2$};

  \path[->] (q_0) edge              node [above]  {0} (q_1)
                  edge [loop above] node          {1} ()
                  edge [bend left]  node [above]  {1} (q_2)
                  edge [bend right] node [below]  {0} (q_2)
            (q_1) edge              node [above]  {1} (q_2);
\end{tikzpicture}
\end{codeexample}

\begin{codeexample}[]
\begin{tikzpicture}
  \foreach \angle in {0,45,...,315}
    \node[rectangle,draw=black!50] (\angle) at (\angle:2) {\angle};

  \foreach \from/\to in {0/45,45/90,90/135,135/180,
                         180/225,225/270,270/315,315/0}
    \path (\from) edge [->,bend right=22,looseness=0.8] (\to)
                  edge [<-,bend left=22,looseness=0.8] (\to);
\end{tikzpicture}
\end{codeexample}

  \itemoption{bend right}\opt{|=|\meta{angle}}
  Works like the |bend left| option, only the bend is to the other side.

  \itemoption{bend angle}|=|\meta{angle}
  Sets the angle to be used by the |bend left| or |bend right|, but
  without actually selecting the |curve to| or the |relative|
  option. This is useful for globally specifying a |bend angle| for a
  whole picture.

  \itemoption{looseness}|=|\meta{number}
  This number specifies how ``loose'' the curve will be. In detail,
  the following happens: \tikzname\ computes the distance between the
  start and the target coordinate (if the start and/or target
  coordinate are nodes, the distance is computed between the points on
  their border). This distance is then multiplied by a fixed factor
  and also by the factor \meta{number}. The resulting distance, let us
  call it $d$, is then used as the distance of the control points from
  the start and target coordinates.

  The fixed factor has been chosen in such a way that if \meta{number}
  is |1|, which is the default, if the |in| and |out| angles differ by
  90$\circ$, then a quarter circle results:
  \begin{codeexample}[]
\tikz \draw (0,0) to [out=0,in=-90]               (1,1);
\tikz \draw (0,0) to [out=0,in=-90,looseness=0.5] (1,1);
  \end{codeexample}

  \itemoption{out looseness}|=|\meta{number}
  specifies the looseness factor for the out distance only. 
  \itemoption{in looseness}|=|\meta{number}
  specifies the looseness factor for the in distance only.
  \itemoption{min distance}|=|\meta{distance}
  If the computed distance for the start and target coordinates are
  below \meta{distance}, then \meta{distance} is used instead.
  \itemoption{max distance}|=|\meta{distance}
  If the computed distance for the start and target coordinates are
  above \meta{distance}, then \meta{distance} is used instead.
  \itemoption{out min distance}|=|\meta{distance}
  The min distance set only for the start coordinate.
  \itemoption{out max distance}|=|\meta{distance}
  The max distance set only for the start coordinate.
  \itemoption{in min distance}|=|\meta{distance}
  The min distance set only for the target coordinate.
  \itemoption{in max distance}|=|\meta{distance}
  The max distance set only for the target coordinate.
  \itemoption{distance}|=|\meta{distance}
  Set the min and max distance to the same value \meta{distance}. Note
  that this causes any computed distance $d$ to be ignored and
  \meta{distance} to be used instead.
\begin{codeexample}[]
\begin{tikzpicture}[out=45,in=135,distance=1cm]
  \draw (0,0) to (1,0)
        (0,0) to (2,0)
        (0,0) to (3,0);
\end{tikzpicture}
\end{codeexample}
  \itemoption{out distance}|=|\meta{distance}
  sets the min and max out distance.
  \itemoption{in distance}|=|\meta{distance}
  sets the min and max in distance.
  \itemoption{out control}|=|\meta{coordinate}
  This option causes the \meta{coordinate} to be used as the start
  control point. All computations of $d$ are ignored. You can use a
  coordinate like |+(1,0)| to specify a point relative to the start
  coordinate. 
  \itemoption{in control}|=|\meta{coordinate}
  This option causes the \meta{coordinate} to be used as the target
  control point.
  \itemoption{controls}|=|\meta{coordinate}| and |\meta{coordinate}
  This option causes the \meta{coordinate}s to be used as control
  points. 
\begin{codeexample}[]
\tikz \draw (0,0) to [controls=+(90:1) and +(90:1)] (3,0);
\end{codeexample}
\end{itemize}


\subsection{Loops}

\begin{itemize}
  \itemstyle{loop}
  This style is similar to the |curve to| style, but differs in the
  following ways: First, the actual target coordinate is ignored and the
  start coordiante is used as the target coordinate. Thus, it is
  allowed not to provide any target coordinate, which can be useful
  with unnamed nodes. Second, the |looseness| is set to |8| and the
  |min distance| to |5mm|. These settings result in rather nice loops
  when the opening angle (difference between |in| and |out|) is
  30$^\circ$.
\begin{codeexample}[]
\begin{tikzpicture}
  \node [circle,draw] {a} edge [in=30,out=60,loop] ();    
\end{tikzpicture}
\end{codeexample}
  \itemstyle{loop above}
  Sets the |loop| style and sets in and out angles such that
  loop is above the node. Furthermore, the |above| option is set,
  which causes a node label to be placed at the correct position. 
\begin{codeexample}[]
\begin{tikzpicture}
  \node [circle,draw] {a} edge [loop above] node {x} ();    
\end{tikzpicture}
\end{codeexample}
  \itemstyle{loop below} works like the previous option.
  \itemstyle{loop left} works like the previous option.
  \itemstyle{loop right} works like the previous option.
  \itemstyle{every loop} This style is installed at the beginning of
  every loop. By default, it is set to |->,shorten >=1pt|, but feel
  free to change this.
\begin{codeexample}[]
\begin{tikzpicture}
  \tikzstyle{every loop}=[]
  \draw (0,0) to [loop above] () to [loop right] ()
              to [loop below] () to [loop left]  ();
\end{tikzpicture}
\end{codeexample}
\end{itemize}



%%% Local Variables: 
%%% mode: latex
%%% TeX-master: "pgfmanual-pdftex-version"
%%% End: 

% Copyright 2006 by Till Tantau
%
% This file may be distributed and/or modified
%
% 1. under the LaTeX Project Public License and/or
% 2. under the GNU Free Documentation License.
%
% See the file doc/generic/pgf/licenses/LICENSE for more details.


\section{Through Library}

\label{section-through-library}


\begin{tikzlibrary}{through}
  This library defines keys for creating shapes that go through given
  points. 
\end{tikzlibrary}


\begin{key}{/tikz/circle through=\meta{coordinate}}

  When this key is given as an option to a node, the following
  happens:
  \begin{enumerate}
  \item The |inner sep| and the |outer sep| are set to zero.
  \item The shape is set to |circle|.
  \item The |minimum size| is set such that the circle around the
    center of the node (which is specified using |at|), goes through
    \meta{coordinate}. 
  \end{enumerate}
\begin{codeexample}[]
\begin{tikzpicture}
  \draw[help lines] (0,0) grid (3,2);
  \node (a) at (2,1.5) {$a$};
  \node [draw] at (1,1) [circle through={(a)}] {$c$};
\end{tikzpicture}
\end{codeexample}
\end{key}

  

%%% Local Variables: 
%%% mode: latex
%%% TeX-master: "pgfmanual-pdftex-version"
%%% End: 

% Copyright 2003 by Till Tantau <tantau@cs.tu-berlin.de>.
%
% This program can be redistributed and/or modified under the terms
% of the LaTeX Project Public License Distributed from CTAN
% archives in directory macros/latex/base/lppl.txt.




\section{Tree Library}

\label{section-tree-library}


\begin{tikzlibrary}{trees}
  This packages defines styles to be used when drawing trees. 
\end{tikzlibrary}

\subsection{Growth Functions}

The package |pgflibrarytikztrees| defines two new growth
functions. They are installed using the following options:

\begin{itemize}
  \itemoption{grow via three points}|=one child at (|\meta{x}%
  |) and two children at (|\meta{y}|) and (|\meta{z}|)|
  This option installs a growth function that works as follows: If a
  parent node has just one child, this child is placed at \meta{x}. If
  the parent node has two children, these are placed at \meta{y} and
  \meta{z}. If the parent node has more than two children, the
  children are placed at points that are linearly extrapolated from
  the three points \meta{x}, \meta{y}, and \meta{z}. In detail, the
  position is $x + \frac{n-1}{2}(y-x) + (c-1)(z-y)$, where $n$ is the
  number of children and $c$ is the number of the current child
  (starting with~$1$).

  The net effect of all this is that if you have a certain ``linear
  arrangement'' in mind and use this option to specify the placement
  of a single child and of two children, then any number of children
  will be placed correctly.

  Here are some arrangements based on this growth function. We start
  with a simple ``above'' arrangement:
\begin{codeexample}[]
\begin{tikzpicture}[grow via three points={%
    one child at (0,1) and two children at (-.5,1) and (.5,1)}]
  \node at (0,0) {one} child;
  \node at (0,-1.5) {two} child child;
  \node at (0,-3) {three} child child child;
  \node at (0,-4.5) {four} child child child child;
\end{tikzpicture}
\end{codeexample}    

  The next arrangement places children above, but ``grows only to the
  right.'' 
\begin{codeexample}[]
\begin{tikzpicture}[grow via three points={%
    one child at (0,1) and two children at (0,1) and (1,1)}]
  \node at (0,0) {one} child;
  \node at (0,-1.5) {two} child child;
  \node at (0,-3) {three} child child child;
  \node at (0,-4.5) {four} child child child child;
\end{tikzpicture}
\end{codeexample}    

  In the final arrangement, the children are placed along a line going
  down and right.
\begin{codeexample}[]
\begin{tikzpicture}[grow via three points={%
    one child at (-1,-.5) and two children at (-1,-.5) and (0,-.75)}]
  \node at (0,0) {one} child;
  \node at (0,-1.5) {two} child child;
  \node at (0,-3) {three} child child child;
  \node at (0,-4.5) {four} child child child child;
\end{tikzpicture}
\end{codeexample}

  These examples should make it clear how you can create new styles to
  arrange your children along a line.

  \itemstyle{grow cyclic}
  This style causes the children to be arranged ``on a circle.'' For
  this, the children are placed at distance |\tikzleveldistance| from
  the parent node, but not on a straight line, but points on a
  circle. Instead of a sibling distance, there is a |sibling angle|
  that denotes the angle between two given children.
  \begin{itemize}
    \itemoption{sibling angle}|=|\meta{angle}
    Sets the angle between siblings in the |grow cyclic| style.
  \end{itemize}
  Note that this function will rotate the coordinate system of the
  children to ensure that the grandchildren will grow in the right
  direction.
\begin{codeexample}[]
\begin{tikzpicture}[grow cyclic]
  \tikzstyle{level 1}=[level distance=8mm,sibling angle=60]
  \tikzstyle{level 2}=[level distance=4mm,sibling angle=45]
  \tikzstyle{level 3}=[level distance=2mm,sibling angle=30]
  \coordinate [rotate=-90] % going down
    child foreach \x in {1,2,3}
      {child foreach \x in {1,2,3}
        {child foreach \x in {1,2,3}}};
\end{tikzpicture}
\end{codeexample}

  \itemoption{clockwise from}|=|\meta{angle}
  This option also cuases children to be arranged on a
  circle. However, the rule for placing children is simpler thatn with
  the |grow cyclic| style: The first child is placed at
  \meta{angle} at a distance of |\tikzleveldistance|. The second child
  is placed at the same distance from the parent, but at angle
  \meta{angle}${}-{}$|\tikzsiblingangle|. The third child is displaced
  by another |\tikzsiblingangle| in a clockwise fashion, and so on. 

  Note that this function will not rotate the coordinate system.
\begin{codeexample}[]
\begin{tikzpicture}
  \node {root}
  [clockwise from=30,sibling angle=30]
  child {node {$30$}}
  child {node {$0$}}
  child {node {$-30$}}
  child {node {$-60$}};
\end{tikzpicture}
\end{codeexample}
  \itemoption{counterclockwise from}|=|\meta{angle}
  Works the same way as |clockwise from|, but sibling angles are added
  instead of subtracted.
\end{itemize}

\subsection{Edges From Parent}

The following styles can be used to modify how the edges from parents
are drawn:

\begin{itemize}
  \itemstyle{edge from parent fork down}
  This style will draw a line from the parent downwards (for half the
  level distance) and then on to the child using only horizontal and
  vertical lines. 
\begin{codeexample}[]
\begin{tikzpicture}
  \node {root}
    [edge from parent fork down]
    child {node {left}}
    child {node {right}
      child[child anchor=north east] {node {child}}
      child {node {child}}
    };
\end{tikzpicture}
\end{codeexample}
  \itemstyle{edge from parent fork right}
  This style behaves similarly, only it will first draw its edge to
  the right.
\begin{codeexample}[]
\begin{tikzpicture}
  \node {root}
    [edge from parent fork right,grow=right]
    child {node {left}}
    child {node {right}
      child {node {child}}
      child {node {child}}
    };
\end{tikzpicture}
\end{codeexample}
  \itemstyle{edge from parent fork left}
  behaves similary. 
  \itemstyle{edge from parent fork up}
  behaves similary. 
\end{itemize}




%%% Local Variables: 
%%% mode: latex
%%% TeX-master: "pgfmanual-pdftex-version"
%%% End: 





\part{Utilities}
\label{part-utilities}

{\Large \emph{by Till Tantau}}


\bigskip
\noindent
The utility packages are not directly involved in creating graphics,
but you may find them useful nonetheless. All of them either directly
depend on \pgfname\ or they are designed to work well together with
\pgfname\ even though they can be used in a stand-alone way.

\vskip2cm
\medskip
\noindent
\begin{codeexample}[graphic=white]
\begin{tikzpicture}[scale=2]
  \shade[top color=blue,bottom color=gray!50] (0,0) parabola (1.5,2.25) |- (0,0);
  \draw (1.05cm,2pt) node[above] {$\displaystyle\int_0^{3/2} \!\!x^2\mathrm{d}x$};
  
  \draw[help lines] (0,0) grid (3.9,3.9)
       [step=0.25cm]      (1,2) grid +(1,1);

  \draw[->] (-0.2,0) -- (4,0) node[right] {$x$};
  \draw[->] (0,-0.2) -- (0,4) node[above] {$f(x)$};

  \foreach \x/\xtext in {1/1, 1.5/1\frac{1}{2}, 2/2, 3/3}
    \draw[shift={(\x,0)}] (0pt,2pt) -- (0pt,-2pt) node[below] {$\xtext$};

  \foreach \y/\ytext in {1/1, 2/2, 2.25/2\frac{1}{4}, 3/3}
    \draw[shift={(0,\y)}] (2pt,0pt) -- (-2pt,0pt) node[left] {$\ytext$};
    
  \draw (-.5,.25) parabola bend (0,0) (2,4) node[below right] {$x^2$};
\end{tikzpicture}
\end{codeexample}

% Copyright 2006 by Till Tantau
%
% This file may be distributed and/or modified
%
% 1. under the LaTeX Project Public License and/or
% 2. under the GNU Free Documentation License.
%
% See the file doc/generic/pgf/licenses/LICENSE for more details.


\section{Key Management}
\label{section-keys}

This section describes the package |pgfkeys|. It is loaded
automatically by both \pgfname\ and \tikzname.

\begin{package}{pgfkeys}
  This package can be used independently of \pgfname. Note that no
  other package of \pgfname\ needs to be loaded (so neither the
  emulation layer nor the system layer is needed). The Con\TeX t
  abbreviation is |pgfkey| if |pgfmod| is not loaded.
\end{package}



\subsection{Introduction}

\subsubsection{Comparison to Other Packages}

The |pgfkeys| package defines a key--value management system that is in
some sense similar to the more light-weight |keyval| system and the
improved |xkeyval| system. However, |pgfkeys| uses a slightly
different philosophy than these systems and it will coexist peacefully
with both of them.

The main differences between |pgfkeys| and |xkeyval| are the
following:

\begin{itemize}
\item |pgfkeys| organizes keys in a tree, while |keyval| and |xkeyval|
  use families. In |pgfkeys| the families correspond to the root
  entries of the key tree.
\item For efficiency reasons, |pgfkeys| does not directly support
  setting keys drawn from multiple families as |xkeyval| does. This
  can be emulated if necessary, but it will be slower than |xkeyval|'s
  native support.
\item |pgfkeys| has no save-stack impact (you will have to read the
  \TeX Book very carefully to appreciate this).
\item |pgfkeys| is slightly slower than |keyval|, but not much.
\item |pgfkeys| supports styles. This means that keys can just stand
  for other keys (which can stand for other keys in turn or which can
  also just execute some code). \tikzname\ uses this mechanism heavily.
\item |pgfkeys| supports multi-argument key code. This can, however,
  be emulated in |keyval|.
\item |pgfkeys| supports handlers. These are call-backs that are
  called when a key is not known. They are very flexible, in fact even
  defining keys in different ways is handled by, well, handlers.
\end{itemize}


\subsubsection{Quick Guide to Using the Key Mechanism}

The following quick guide to \pgfname's key mechanism only treats the
most commonly used features. For an in-depth discussion of what is
going on, please consult the remainder of this section.

Keys are organized in a large tree that is reminiscent of the Unix
file tree. A typical key might be, say, |/tikz/coordinate system/x|
or just |/x|. Again as in Unix, when you specify keys you can provide
the complete path of the key, but you usually just provide the name of
the key (corresponding to the file name without any path) and the path
is added automatically.

Typically (but not necessarily) some code is associated with a key. To
execute this code, you use the |\pgfkeys| command. This command takes
a list of so-called key--value pairs. Each pair is of the form
\meta{key}|=|\meta{value}. For each pair the |\pgfkeys| command will
execute the code stored for the \meta{key} with its parameter set to
\meta{value}.

Here is a typical example of how the |\pgfkeys| command is used:
\begin{codeexample}[code only]
\pgfkeys{/my key=hallo,/your keys/main key=something\strange,
         key name without path=something else}
\end{codeexample}

Now, to set the code that is stored in a key you do not need to learn
a new command. Rather, the |\pgfkeys| command can also be used to set
the code of a key. This is done using so-called \emph{handlers}. They
look like keys whose names look like ``hidden files in Unix'' since
they start with a dot. The handler for setting the code of a key is
appropriately called |.code| and it is used as follows:
\begin{codeexample}[]
\pgfkeys{/my key/.code=The value is '#1'.}
\pgfkeys{/my key=hi!}
\end{codeexample}
As you can see, in the first line we defined the code for the key
|/my key|. In the second line we executed this code with the parameter
set to |hi!|.

There are numerous handlers for defining a key. For instance, we can
also define a key whose value actually consists of more than one
parameter. 
\begin{codeexample}[]
\pgfkeys{/my key/.code 2 args=The values are '#1' and '#2'.}
\pgfkeys{/my key={a1}{a2}}
\end{codeexample}

We often want to have keys where the code is called with some default
value if the user does not provide a value. Not surprisingly, this is
also done using a handler, this time called |.default|.
\begin{codeexample}[]
\pgfkeys{/my key/.code=(#1)}
\pgfkeys{/my key/.default=hello}
\pgfkeys{/my key=hallo,/my key}
\end{codeexample}

The other way round, it is also possible to specify that a value
\emph{must} be specified, using a handler called
|.value required|. Finally, you can also require that no value
\emph{may} be specified using |.value forbidden|.

All keys for a package like, say, \tikzname\ start with the path
|/tikz|. We obviously do not like to write this path down every
time we use a key (so we do not have to write things like
|\draw[/tikz/line width=1cm]|). What we need is to somehow ``change
the default path to a specific location.'' This is done using the
handler |.cd| (for ``change directory''). Once this handler has been
used on a key, all subsequent keys {\itshape in the current call of
  |\pgfkeys| only} are automatically prefixed with this path, if
necessary.

Here is an example:
\begin{codeexample}[code only]
\pgfkeys{/tikz/.cd,line width=1cm,cap=round}
\end{codeexample}
This makes it easy to define commands like |\tikzset|, which could be
defined as follows (the actual definition is a bit faster, but the
effect is the same):
\begin{codeexample}[code only]
\def\tikzset#1{\pgfkeys{/tikz/.cd,#1}}
\end{codeexample}

When a key is handled, instead of executing some code, the key can
also cause further keys to be executed. Such keys will be called
\emph{styles}. A style is, in essence, just a key list that should be
executed whenever the style is executed. Here is an example:
\begin{codeexample}[]
\pgfkeys{/a/.code=(a:#1)}
\pgfkeys{/b/.code=(b:#1)}
\pgfkeys{/my style/.style={/a=foo,/b=bar,/a=#1}}
\pgfkeys{/my style=wow}
\end{codeexample}
As the above example shows, style can also be parametrized, just like
the normal code keys.

As a typical use of styles, suppose we wish to setup the key |/tikz|
so that it will change the default path to |/tikz|. This can be
achieved as follows:
\begin{codeexample}[code only]
\pgfkeys{/tikz/.style=/tikz/.cd}
\pgfkeys{tikz,line width=1cm,draw=red}
\end{codeexample}

Note that when |\pgfkeys| is executed, the default path is set
to~|/|. This means that the first |tikz| will be completed to
|/tikz|. Then |/tikz| is a style and, thus, replaced by |/tikz/.cd|,
which changes the default path to |/tikz|. Thus, the |line width| is
correctly prefixed with |/tikz|.

\subsection{The Key Tree}

The |pgfkeys| package organizes keys in a so-called \emph{key
  tree}. This tree will be familiar to anyone who has used a Unix
operating system: A key is addressed by a path, which consists of
different parts separated by slashes. A typical key might be
|/tikz/line width| or just |/tikz| or something more complicated like
|/tikz/cs/x/.store in|.

Let us fix some further terminology: Given a key like |/a/b/c|, we
call the part leading up the last slash (|/a/b|) the \emph{path} of
the key. We call everything after the last slash (|c|) the \emph{name}
of the key (in a file system this would be the file name). 

We do not always wish to specify keys completely. Instead, we usually
specify only part of a key (typically only the name) and the
\emph{default path} is then added to the key at the front. So, when
the default path is |/tikz| and you 
refer to the (partial) key |line width|, the actual key that is used
is |/tikz/line width|. There is a simple rule for deciding whether a
key is a partial key or a full key: If it starts with a slash, then it
is a full key and it is not modified; if it does not start with
a slash, then the default path is automatically prefixed.

Note that the default path is not the same as a search path. In
particular, the default path is just a single path. When a partial key
is given, only this single default path is prefixed; |pgfkeys| does
not try to lookup the key in different parts of a search path. It is,
however, possible to emulate search paths, but a much more
complicated mechanism must be used.

When you set keys (to be explained in a moment), you can freely mix
partial and full keys and you can change the default path. This makes
it possible to temporarily use keys from another part of the key tree
(this turns out to be a very useful feature).

Each key (may) store some \emph{tokens} and there exist commands,
described below, for setting, getting, and changing the tokens stored
in a key. However, you will only very seldom use these commands
directly. Rather, the standard way of using keys is the |\pgfkeys|
command or some command that uses it internally like, say,
|\tikzset|. So, you may wish to skip the following commands and
continue with the next subsection.

\begin{command}{\pgfkeyssetvalue\marg{full key}\marg{token text}}
  Stores the \meta{token text} in the \meta{full key}. The \meta{full key}
  may not be a partial key, so no default-path-adding is done. The
  \meta{token text} can be arbitrary tokens and may even contain things
  like |#| or unbalanced \TeX-ifs.
\begin{codeexample}[]
\pgfkeyssetvalue{/my family/my key}{Hello, world!}
\pgfkeysinsertvalue{/my family/my key}  
\end{codeexample}

  The setting of a key is always local to the current \TeX\ group.
\end{command}

\begin{command}{\pgfkeyslet\marg{full key}\marg{macro}}
  Performs a |\let| statement so the the \meta{full key} pionts to the
  contents of \meta{macro}.
\begin{codeexample}[]
\def\helloworld{Hello, world!}
\pgfkeyslet{/my family/my key}{\helloworld}
\pgfkeysinsertvalue{/my family/my key}  
\end{codeexample}
  You should never let a key be equal to |\relax|. Such a key may or
  may not be indistinguishable from an undefined key.
\end{command}

\begin{command}{\pgfkeysgetvalue\marg{full key}\marg{macro}}
  Retrieves the tokens stored in the \meta{full key} and lets
  \meta{macro} be equal to these tokens. If the key has
  not been set, the \meta{macro} will be equal to |\relax|. 
\begin{codeexample}[]
\pgfkeyssetvalue{/my family/my key}{Hello, world!}
\pgfkeysgetvalue{/my family/my key}{\helloworld}
\helloworld
\end{codeexample}
\end{command}

\begin{command}{\pgfkeysinserttokens\marg{full key}}
  Inserts the tokens stored in \meta{full key} at the current position
  into the text.

\begin{codeexample}[]
\pgfkeyssetvalue{/my family/my key}{Hello, world!}
\pgfkeysinsertvalue{/my family/my key}
\end{codeexample}
\end{command}


\begin{command}{\pgfkeysifdefined\marg{full key}\marg{if}\marg{else}}
  Checks whether this key was previously set using either
  |\pgfkeyssetvalue| or |\pgfkeyslet|. If so, the code in \meta{if} is
  executed, otherwise the code in \meta{else}.

  This command will use e\TeX's |\ifcsname| command, if available, for
  efficiency. This means, however, that it may behave differently for
  \TeX\ and for e\TeX\ when you set keys to |\relax|. For this reason
  you should not do so. 
\begin{codeexample}[]
\pgfkeyssetvalue{/my family/my key}{Hello, world!}
\pgfkeysifdefined{/my family/my key}{yes}{no}
\end{codeexample}
\end{command}


\subsection{Setting Keys}

Settings keys is done using a powerful command called |\pgfkeys|. This
command takes a list of so-called \emph{key--value pairs}. These are
pairs of the form \meta{key}|=|\meta{value}. The principle idea is the
following: For each pair in the list, some \emph{action} is
taken. This action can be one of the following:

\begin{enumerate}
\item A command is executed whose argument(s) are \meta{value}. This
  command is stored in a special subkey of \meta{key}.
\item The \meta{value} is stored in the \meta{key} itself.
\item If the key's name (the part after the last slahs) is a known
  \emph{handler}, then this handler will take care of the key.
\item If the key is totally unknown, one of several possible
  \emph{unknown key handlers} is called. 
\end{enumerate}

Addtionally, if the \meta{value} is missing, a default value may or
may not be substituted. Before we plunge into all the details,
let us have a quick look at the command itself.

\begin{command}{\pgfkeys\marg{key list}}
  The \meta{key list} should be a list of key--value pairs, separated
  by commas. A key--value pair can have the following two forms:
  \meta{key}|=|\meta{value} or just \meta{key}. Any spaces around the
  \meta{key} or around the \meta{value} are removed. It is permissible
  to surround both the \meta{key} or the \meta{value} in curly braces,
  which are also removed. Especially putting the \meta{value} in curly
  braces needs to be done quite often, namely whenever the \meta{value}
  contains an equal-sign or a comma.

  The key--value pairs in the list are handled in the order they
  appear. How this handling is done, exactly, is described in the rest
  of this section.

  If a \meta{key} is a partial key, the current value of the default
  path is prefixed to the \meta{key} and this ``upgraded'' key is
  then used. The default path is just the root path |/| when the first
  key is handled, but it may change later on. At the end of the
  command, the default path is reset to the value it had before this
  command was executed. 
  
  Calls of this command may be nested. Thus, it is permissible to call
  |\pgfkeys| inside the code that is executed for a key. Since the
  default path is restored after a call of |\pgfkeys|, the default
  path will not change when you call |\pgfkeys| while executing code
  for a key (which is exactly what you want).
\end{command}


\begin{command}{\pgfkeysalso\marg{key list}}
  This command has execatly the same effect as |\pgfkeys|, only the
  default path is not modified before or after the keys are being
  set. This command is mainly intended to be called by the code that
  is being processed for a key.
\end{command}


\subsubsection{Default Arguments}

The arguments of the |\pgfkeys| command can either be of the form
\meta{key}|=|\meta{value} or of the form \meta{key} with the
value-part missing. In the second case, the |\pgfkeys| will try to
provide a \emph{default value} for the \meta{value}. If such a default
value is defined, it will be used as if you had written
\meta{key}|=|\meta{default value}.

In the following, the details of how default values are determined is
described; however, you should normally use the handlers |.default|
and |.value required| as described in
Section~\ref{section-default-handlers} and you can may wish to skip
the following details.

When |\pgfkeys| encounters a \meta{key} without an equal-sign, the
following happens:
\begin{enumerate}
\item The input is replaced by \meta{key}|=\pgfkeysnovalue|. In
  particular, the commands |\pgfkeys{my key}| and
  |\pgfkeys{my key=\pgfkeysnovalue}| have exactly the same effect and
  you can ``simulate'' a missing value by providing the value
  |\pgfkeysnovalue|, which is sometimes useful. 
\item If the \meta{value} is |\pgfkeysnovalue|, then it is checked
  whether the subkey \meta{key}|/.@def| exists. For instance, if you
  write |\pgfkeys{/my key}|, then it is checked whether the key
  |/my key/.@def| exists.
\item If the key \meta{key}|/.@def| exists, then the tokens stored in
  this key are used as \meta{value}.
\item If the key does not exist, then |\pgfkeysnovalue| is used as the
  \meta{value}.
\item At the end, if the \meta{value} is now equal to
  |\pgfkeysvaluerequired|, then the code  (or something fairly equivalent)
  |\pgfkeys{/errors/value required=|\meta{key}|{}}|
  is executed. Thus, by changing this key you can change the error
  message that is printed or you can handle the missing value in some
  other way.
\end{enumerate}



\subsubsection{Keys That Execute Commands}
\label{section-key-code}

After the transformation process described in the previous subsection,
we arrive at a key of the form \meta{key}=\meta{value}, where
\meta{key} is a full key. Different things can now happen, but always
the macro |\pgfkeyscurrentkey| will have been setup to expand to the
text of the \meta{key} that is currently being processed.

The first things that is tested is whether the key \meta{key}|/.@cmd|
exists. If this is the case, then it is assumed that this key stores
the code of a macro and this macro is executed. The argument of this
macro is \meta{value} directly followed by |\pgfeov|, which stands for
``end of value.'' The \meta{value} is not surrounded by braces. After
this code has been executed, |\pgfkeys| continues with the next key in
the \meta{key list}.

It may seem quite peculiar that the macro stored in the key
\meta{key}|/.@cmd| is not simply executed with the argument
|{|\meta{value}|}|. However, the approach taken in the |pgfkeys|
packages allows for more flexibility. For instance, assume that you
have a key that expects a \meta{value} of the form
``\meta{text}|+|\meta{more text}'' and wishes to store \meta{text} and
\meta{more text} in two different macros. This can be achieved as
follows:
\begin{codeexample}[]
\def\mystore#1+#2\pgfeov{\def\a{#1}\def\b{#2}}
\pgfkeyslet{/my key/.@cmd}{\mystore}
\pgfkeys{/my key=hello+world}

|\a| is \a, |\b| is \b.
\end{codeexample}

Naturally, defining the code to be stored in a key in the above manner
is too awkward. The following commands simplify things a bit, but the
usual manner of setting up code for a key is to use one of the
handlers described in Section~\ref{section-code-handlers}.

\begin{command}{\pgfkeysdef\marg{key}\marg{code}}
  This command temporarily defines a \TeX-macro with the argument list
  |#1\pgfeov| and then lets \meta{key}|/.@cmd| be equal to this
  macro. The net effect of all this is that you have then setup code
  for the key \meta{key} so that when you write
  |\pgfkeys{|\meta{key}|=|\meta{value}|}|, then the \meta{code} is
  executed with all occurrences of |#1| in \meta{code} being replaced
  by \meta{value}. (This behaviour is quite similar to the
  |\define@key| command of |keyval| and |xkeyval|).

\begin{codeexample}[]
\pgfkeysdef{/my key}{#1, #1.}
\pgfkeys{/my key=hello}
\end{codeexample}
\end{command}

\begin{command}{\pgfkeysedef\marg{key}\marg{code}}
  This command works like |\pgfkeysdef|, but it uses |\edef| rather
  than |\def| when defining the key macro. If you do not know the
  difference between the two, then you will not need this command;
  and if you know the difference, then you will know when you need this
  command.
\end{command}

\begin{command}{\pgfkeysdeargs\marg{key}\marg{argument pattern}\marg{code}}
  This command works like |\pgfkeysdef|, but it allows you to provide
  an arbitrary \meta{argument pattern} rather than just the simple
  single argument of |\pgfkeysdef|. 

\begin{codeexample}[]
\pgfkeysdefargs{/my key}{#1+#2}{\def\a{#1}\def\b{#2}}
\pgfkeys{/my key=hello+world}

|\a| is \a, |\b| is \b.
\end{codeexample}
\end{command}

\begin{command}{\pgfkeysedefargs\marg{key}\marg{argument pattern}\marg{code}}
  The |\edef| version of |\pgfkeysdefargs|.
\end{command}


\subsubsection{Keys That Store Values}

Let us continue with what happens when |\pgfkeys| processes the
current key and  the subkey \meta{key}|/.@cmd| is not defined. Then
it is checked whether the \meta{key} itself exists (has been
previously assigned a value using, for instance,
|\pgfkeyssetvalue|). In this case, the tokens stored in \meta{key} are
replaced by \meta{value} and |\pgfkeys| proceeds with the next key in
the \meta{key list}. 


\subsubsection{Keys That Are Handled}
\label{section-key-handlers}

If neither the \meta{key} itself nor the subkey \meta{key}|/.@cmd| are
defined, then the \meta{key} cannot be processed ``all by itself.''
Rather, a \meta{handler} is needed for this key. Most of the power of
|pgfkeys| comes from the proper use of such handlers.

Recall that the \meta{key} is always a full key (if it was not
originally, it has already been upgraded at this point to a full
key). It decomposed into  two parts:

\begin{enumerate}
\item The \meta{path} of \meta{key} (everything
  before the last slash) is stored in the macro |\pgfkeyscurrentpath|.
\item The \meta{name} of \meta{key} (everything
  after the last slash) is stored in the macro |\pgfkeyscurrentname|.

  It is recommended (but not necessary) that the name of a handler
  starts with a dot (but not with |.@|), so that they are easy to
  detect for the reader.  
\end{enumerate}

(For efficiency reasons, these two macros are only setup at this point;
so when code is executed for a key in the ``usual'' manner then these
macros are not setup.)

The |\pgfkeys| command now checks whether the key
|/handlers/|\meta{name}|/.@cmd| exists. If so, it should store a command
and this command is executed exactly in the same manner as described
in Section~\ref{section-key-code}.
Thus, this code gets the \meta{value} that was originally intended for
\meta{key} as its argument, followed by |\pgfeov|.
It is the job of the handlers to so something useful with the
\meta{value}.

For an example, let us write a handler that will output the value
stored in a key to the log file. We call this handler
|.print to log|. The idea is that when someone tries to use the key
|/my key/.print to log|, then this key will not be defined and the
handler gets executed. The handler will then have access to the
path-part of the key, which is |/my key|, via the macro
|\pgfkeyscurrentpath|. It can then lookup which value is stored in
this key and print it.

\begin{codeexample}[code only]
\pgfkeysdef{/handlers/.print to log}
{%
  \pgfkeysgetvalue{\pgfkeyscurrentpath}{\temp}
  \writetolog{\temp}
}
\pgfkeyssetvalue{/my key}{Hi!}
...
\pgfkeys{/my key/.print to log}
\end{codeexample}
The above code will print |Hi!| in the log, provided the macro
|\writetolog| is setup appropriately.

For a more interesting handler, let us program a handler that will
setup a key so that when the key is used some code is executed. This
code is given as \meta{value}. All the handler must do is to call
|\pgfkeysdef| for the path of the key (which misses the handler's
name) and assign the parameter value to it.
\begin{codeexample}[]
\pgfkeysdef{/handlers/.my code}{\pgfkeysdef{\pgfkeyscurrentpath}{#1}}
\pgfkeys{/my key/.my code=(#1)}
\pgfkeys{/my key=hallo}
\end{codeexample}


\subsubsection{Keys That Are Unknown}

For some keys, neither the key is defined nor its |.@cmd| subkey nor
is a handler defined for this key. In this case, it is checked whether
the key \meta{current path}|/.unknown/.@cmd| exists. Thus, when you try to
use the key |/tikz/strange|, then it is checked whether
|/tikz/.unknown/.@cmd| exists. If this key exists (which it does), it is
executed. This code can then try to make sense of the key. For
instance, the handler for \tikzname\ will try to interpret the key's
name as a color or as an arrow specification or as a \pgfname\
option.

You can setup unknown key handlers for your own keys by simply setting
the code of the key \meta{my path prefix}|/.unknown|. This also allows
you to setup ``search paths.'' The idea is that you would like keys to
be searched not only in a single default path, but in
several. Suppose, for instance, that you would like keys to be
searched 
for in |/a|, |/b|, and |/b/c|. We setup a key |/my search path| for
this:
\begin{codeexample}[code only]
\pgfkeys{/my search path/.unknown/.code=
  {%
    \let\searchname=\pgfkeyscurrentname%
    \pgfkeysalso{%
      /a/\searchname/.try=#1,
      /b/\searchname/.retry=#1,
      /b/c/\searchname/.retry=#1%
    }%
  }%
}
\pgfkeys{/my search path/.cd,foo,bar}
\end{codeexample}
In the above code, |foo| and |bar| will be searched for in the three
directories  |/a|, |/b|, and |/b/c|. 

If the key \meta{current path}|/.unknown/.@cmd| does not exist, the
handler |/handlers/.unknown| is invoked instead, which is always
defined and which prints an error message by default.

\subsection{Key Handlers}

We now describe which key handlers are defined by default. You can
also define new ones as described in Section~\ref{section-key-handlers}.


\subsubsection{Handlers for Path Management}

\begin{handler}{{.cd}}
  This handler causes the default path to be set to \meta{key}. Note that
  the default path is reset at the beginning of each call to
  |\pgfkeys| to be equal to~|/|.

  \example |\pgfkeys{/tikz/.cd,...}|
\end{handler}

\begin{handler}{{.is family}}
  This handler sets up things such that when \meta{key} is executed, then
  the current path is set to \meta{key}. A typical use is the following:
\begin{codeexample}[code only]
\pgfkeys{/tikz/.is family}
\pgfkeys{tikz,line width=1cm}  
\end{codeexample}
  The effect of this handler is the same as if you had written
  \meta{key}|/.style=|\meta{key}|/.cd|, only the code produced by the
  |.is family| handler is quicker.
\end{handler}


\subsubsection{Setting Defaults}
\label{section-default-handlers}

\begin{handler}{{.default}|=|\meta{value}}
  Sets the default value of \meta{key} to \meta{value}. This means
  that whenever no value is provided in a call to |\pgfkeys|, then
  this \meta{value} will be used instead.

  \example |\pgfkeys{/width/.default=1cm}|
\end{handler}

\begin{handler}{{.value required}}
  This handler causes the error message key |/erros/value required| to
  be issued whenever the \meta{key} is used without a value.

  \example |\pgfkeys{/width/.value required}|
\end{handler}

\begin{handler}{{.value forbidden}}
  This handler causes the error message key |/erros/value forbidden|
  to be issued whenever the \meta{key} is used with a value.

  This handler works be adding code to the code of the key. This means
  that you have to define the key first before you can use this
  handler. 
\begin{codeexample}[code only]
\pgfkeys{/my key/.code=I do not want an argument!}
\pgfkeys{/my key/.value forbidden}

\pgfkeys{/my key}     % Ok
\pgfkeys{/my key=foo} % Error
\end{codeexample}
\end{handler}


\subsubsection{Defining Key Codes}
\label{section-code-handlers}

A number of handlers exist for defining the code of keys.

\begin{handler}{{.code}|=|\meta{code}}
  This handler executes |\pgfkeysdef| with the parameters \meta{key}
  and \meta{code}. This means that, afterwards, whenever the
  \meta{key} is used, the \meta{code} gets executed. More precisely,
  when \meta{key}|=|\meta{value} is encountered in a key list,
  \meta{code} is executed with any occurrence of |#1| replaced by
  \meta{value}. As always, if no \meta{value} is given, the default
  value is used, if defined, or the special value |\pgfkeysnovalue|.

  It is permissible that \meta{code} calls the command |\pgfkeys|. It
  is also permissible the \meta{code} calls the command
  |\pgfkeysalso|, which is useful for styles, see below.

\begin{codeexample}[code only]
\pgfkeys{/par indent/.code={\parindent=#1},/par indent/.default=2em}
\pgfkeys{/par indent=1cm}
...
\pgfkeys{/par indent}
\end{codeexample}
\end{handler}

\begin{handler}{{.ecode}|=|\meta{code}}
  This handler works like |.code|, only the command |\pgfkeysedef| is
  used. 
\end{handler}


\begin{handler}{{.code 2 args}|=|\meta{code}}
  This handler works like |.code|, only two arguments rather than one
  are expected when the \meta{code} is executed. This means that when
  \meta{key}|=|\meta{value} is encountered in a key list, the
  \meta{value} should consist of two arguments. For instance,
  \meta{value} could be |{first}{second}|. Then \meta{code} is
  executed with any occurrence of |#1| replaced |first| and any
  occurrence of |#2| replaced by |second|.

  Because of the special way the \meta{value} is parsed, if you set
  \meta{value} to, for instance, |first| (without any braces), then
  |#1| will be set to |f| and |#2| will be set to |irst|. 

\begin{codeexample}[code only]
\pgfkeys{/page size/.code 2 args={\paperheight=#2\paperwidth=#1}}
\pgfkeys{/page size={30cm}{20cm}}
\end{codeexample}
\end{handler}

\begin{handler}{{.ecode 2 args}|=|\meta{code}}
  This handler works like |.code 2 args|, only an |\edef| is used
  rather than a |\def| to define the macro.
\end{handler}



\begin{handler}{{.code args}|=|\marg{argument pattern}\marg{code}}
  This handler also works like |.code|, but you can now specify an
  arbitrary \meta{argument pattern}. Such a pattern is a usual \TeX\
  macro pattern. For instance, suppose \meta{argument pattern} is
  |(#1/#2)| and \meta{key}|=|\meta{value} is encountered in a
  key list with \meta{value} being |(first/second)|. Then \meta{code}
  is executed with any occurrence of |#1| replaced |first| and any
  occurrence of |#2| replaced by |second|. So, the actual \meta{value}
  is matched against the \meta{argument pattern} in the standard \TeX\
  way. 

\begin{codeexample}[code only]
\pgfkeys{/page size/.code args={#1 and #2}{\paperheight=#2\paperwidth=#1}}
\pgfkeys{/page size=30cm and 20cm}
\end{codeexample}
\end{handler}

\begin{handler}{{.ecode args}|=|\marg{argument pattern}\marg{code}}
  This handler works like |.code args|, only an |\edef| is used
  rather than a |\def| to define the macro.
\end{handler}


There are also handlers for modifying existing keys.

\begin{handler}{{.add code}|=|\marg{prefix code}\marg{append code}}
  This handler adds code to an existing key. The \meta{prefix code} is
  added to the code stored in \meta{key}|/.@cmd| at the beginning, the
  \meta{append code} is added to this code at the end. Either can be
  empty. The argument list of \meta{code} cannot be changed using this
  handler. Note that both \meta{prefix code} and \meta{append code}
  may contain parameters like |#2|. 
  
\begin{codeexample}[code only]
\pgfkeys{/par indent/.code={\parindent=#1}}
\newdimen\myparindent  
\pgfkeys{/par indent/.add code={}{\myparindent=#1}}
...
\pgfkeys{/par indent=1cm} % This will set both \parindent and
                          % \myparindent to 1cm
\end{codeexample}
\end{handler}

\begin{handler}{{.prefix code}|=|\meta{prefix code}}
  This handler is a shortcut for \meta{key}|/.add code={|\meta{prefix
      code}|}{}|. That is, this handler adds the \meta{prefix code} at
  the beginning of the code stored in \meta{key}|/.@cmd|.
\end{handler}

\begin{handler}{{.append code}|=|\meta{append code}}
  This handler is a shortcut for \meta{key}|/.add code={}{|\meta{append
      code}|}{}|.
\end{handler}


\subsubsection{Defining Styles}

The following handlers allow you to define \emph{styles}. A style is a
key list that is processed whenever the style is given as a key in a
key list. Thus, a style ``stands for'' a certain key value
list. Styles can be parametrized just like normal code.

\begin{handler}{{.style}|=|\meta{key list}}
  This handler set things up so that whenever \meta{key}|=|\meta{value} is
  encountered in a key list, then the \meta{key list}, with every
  occurrence of |#1| replaced by \meta{value}, is processed
  instead. As always, if no \meta{value} is given, the default
  value is used, if defined, or the special value |\pgfkeysnovalue|.

  You can achieve the same effect by writing
  \meta{key}|/.code=\pgfkeysalso{|\meta{key list}|}|. This means, in
  particular, that the code of a key could also first execute some
  normal code and only then process some further keys. 

\begin{codeexample}[code only]
\pgfkeys{/par indent/.code={\parindent=#1}}
\pgfkeys{/no indent/.style={/par indent=0pt}}
\pgfkeys{/normal indent/.style={/par indent=2em}}
\pgfkeys{/no indent}
...
\pgfkeys{/normal indent}
\end{codeexample}
  The following example shows a parametrized style ``in action''.
\begin{codeexample}[]
\begin{tikzpicture}[outline/.style={draw=#1,fill=#1!20}]
  \node [outline=red]            {red box};
  \node [outline=blue] at (0,-1) {blue box};
\end{tikzpicture}
\end{codeexample}
\end{handler}

\begin{handler}{{.estyle}|=|\meta{key list}}
  This handler works like |.style|, only the \meta{code} is set using
  |\edef| rather than |\def|. Thus, all macros in the \meta{code} are
  expanded prior to saving the style.
\end{handler}

For styles the corresponding handlers as for normal code exist:

\begin{handler}{{.style 2 args}|=|\meta{key list}}
  This handler works like |.code 2 args|, only for styles. Thus, the
  \meta{key list} may contain occurrences of both |#1| and |#2| and
  when the style is used, two parameters must be given as
  \meta{value}. 
\begin{codeexample}[code only]
\pgfkeys{/paper height/.code={\paperheight=#1},/paper width/.code={\paperwidth=#1}}
\pgfkeys{/page size/.style 2 args={/paper height=#1,/paper width=#2}}
\pgfkeys{/page size={30cm}{20cm}}
\end{codeexample}
\end{handler}

\begin{handler}{{.estyle 2 args}|=|\meta{key list}}
  This handler works like |.style 2 args|, only an |\edef| is used
  rather than a |\def| to define the macro.
\end{handler}

\begin{handler}{{.style args}|=|\marg{argument pattern}\marg{key list}}
  This handler works like |.code args|, only for styles.
\end{handler}

\begin{handler}{{.estyle args}|=|\marg{argument pattern}\marg{code}}
  This handler works like |.ecode args|, only for styles.
\end{handler}

\begin{handler}{{.add style}|=|\marg{prefix key list}\marg{append key list}}
  This handler works like |.add code|, only for styles. However, it is
  permissible to add styles to keys that have previously been set
  using  |.code|. (It is also permissible to add normal \meta{code} to
  a key that has previously been set using |.style|). When you add a
  style to a key that was previously set using |.code|, the following
  happens: When \meta{key} is processed, the \meta{prefix key list}
  will be processed first, then the \meta{code} that was previously
  stored in \meta{key}|/.@cmd|, and then the keys in \meta{append key
    list} are processed.
\begin{codeexample}[code only]
\pgfkeys{/par indent/.code={\parindent=#1}}
\pgfkeys{/par indent/.add style={}{/my key=#1}}
...
\pgfkeys{/par indent=1cm} % This will set \parindent and
                          % then execute /my key=#1
\end{codeexample}
\end{handler}

\begin{handler}{{.prefix style}|=|\meta{prefix key list}}
  Works like |.add style|, but only for the prefix key list.
\end{handler}

\begin{handler}{{.append style}|=|\meta{append key list}}
  Works like |.add style|, but only for the append key list.
\end{handler}


\subsubsection{Defining Value-, Macro-, If- and Choice-Keys}

For some keys, the code that should be executed for them is rather
``specialized.'' For instance, it happens often that the code for a
key just sets a certain \TeX-if to true or false. For these case
predefine handlers make it easier to install the necessary code.

However, we start with some handlers that are used to manage the value
that is directly stored in a key.

\begin{handler}{{.initial}|=|\meta{value}}
  This handler sets the value of \meta{key} to \meta{value}. Note that
  no subkeys are involved. After this handler has been used, by the
  rules governing keys, you can subsequently change the value of the
  \meta{key} by just writing \meta{key}|=|\meta{value}. Thus, this
  handler is used to set the initial value of key.

\begin{codeexample}[code only]
\pgfkeys{/my key/.initial=red}
% "/my key" now stores the value "red"
\pgfkeys{/my key=blue}
% "/my key" now stores the value "blue"
\end{codeexample}

  Note that in the after the example, writing |\pgfkeys{/my key}| will not
  have the effect you might expect (namely that |blue| is inserted
  into the main text). Rather, |/my key| will be promoted to
  |/my key=\pgfkeysnovalue| and, thus, |\pgfkeysnovalue| will be
  stored in |/my key|.

  To retrieve the value stored in a key, the handler |.get| is used.
\end{handler}

\begin{handler}{{.get}|=|\meta{macro}}
  Executes a |\let| command so that \meta{macro} contains the contents
  stored in \meta{key}.  

\begin{codeexample}[]
\pgfkeys{/my key/.initial=red}
\pgfkeys{/my key=blue}
\pgfkeys{/my key/.get=\mymacro}
\mymacro
\end{codeexample}
\end{handler}

\begin{handler}{{.add}|=|\marg{prefix value}\marg{append value}}
  Adds the \meta{prefix value} and the beginning and the \meta{append
    value} at the end of the value stored in \meta{key}.
\end{handler}

The next handler is useful for the common situation where
\meta{key}|=|\meta{value} should cause the \meta{value} to be stored
in some macro. Note that, typically, you could just as well store the
value in the key itself.

\begin{handler}{{.store in}|=|\meta{macro}}
  This handler has the following effect: When you write
  \meta{key}|=|\meta{value}, the code
  |\def|\meta{macro}|{|\meta{value}|}| is executed. Thus, the given
  value is ``stored'' in the \meta{macro}.  
\begin{codeexample}[]
\pgfkeys{/text/.store in=\mytext}
\pgfkeys{/text=Hello world!}
\mytext
\end{codeexample}
\end{handler}

In another common situation a key is used to set a \TeX-if to true or
false. 

\begin{handler}{{.is if}|=|\meta{\TeX-if name}}
  This handler has the following effect: When you write
  \meta{key}|=|\meta{value}, it is first checked that \meta{value} is
  |true| or |false| (the default is |true| if no \meta{value} is
  given). If this is not the case, the error key
  |/errors/boolean expected| is executed. Otherwise, 
  the code |\|\meta{\TeX-if name}\meta{value} is executed, which sets
  the \TeX-if accordingly.
\begin{codeexample}[]
\newif\iftheworldisflat    
\pgfkeys{/flat world/.is if=theworldisflat}
\pgfkeys{/flat world=false}
\iftheworldisflat
  Flat
\else
  Round?
\fi
\end{codeexample}
\end{handler}

The next handler deals with the problem when a
\meta{key}|=|\meta{value} makes sense only for a small set of possible
\meta{value}s. For instance, the line cap can only be |rounded| or
|rect| or |butt|, but nothing else. For this situation the following
handler is useful.

\begin{handler}{{.is choice}}
  This handler set things up so that writing \meta{key}|=|\meta{value}
  will cause the subkey \meta{key}|/|\meta{value} to be executed. So,
  each of the different possible choices should be given by a subkey
  of \meta{key}.
\begin{codeexample}[code only]
\pgfkeys{/line cap/.is choice}
\pgfkeys{/line cap/round/.style={\pgfsetbuttcap}}
\pgfkeys{/line cap/butt/.style={\pgfsetroundcap}}
\pgfkeys{/line cap/rect/.style={\pgfsetrectcap}}
\pgfkeys{/line cap/rectangle/.style={/line cap=rect}}
...
\draw [/line cap=butt] ...
\end{codeexample}
  If the subkey \meta{key}|/|\meta{value} does not exist, the error
  key |/errors/unknown choice value| is executed.
\end{handler}

\subsubsection{Handlers for Testing Keys}

\begin{handler}{{.try}|=|\meta{value}}
  This handler causes the same things to be done as if
  \meta{key}|=|\meta{value} had been written instead. However, if
  neither \meta{key}|/.@cmd| nor the key itself is defined, no
  handlers will be called. Instead, 
  the execution of the key just stops. Thus, this handler will ``try''
  to use the key, but no further action is taken when the key is not
  defined.

  The \TeX-if |\||ifpgfkeyssuccess| will be set according to whether
  the \meta{key} was successfully executed or not. 
\begin{codeexample}[]
\pgfkeys{/a/.code=(a:#1)}
\pgfkeys{/b/.code=(b:#1)}
\pgfkeys{/x/.try=hmm,/a/.try=hallo,/b/.try=welt}
\end{codeexample}
\end{handler}

\begin{handler}{{.retry}|=|\meta{value}}
  This handler works just like |.try|, only it will not do anything if
  |\||ifpgfkeyssuccess| is false. Thus, this handler will only retry
  to set a key if ``the last attempt failed''. 
\begin{codeexample}[]
\pgfkeys{/a/.code=(a:#1)}
\pgfkeys{/b/.code=(b:#1)}
\pgfkeys{/x/.try=hmm,/a/.retry=hallo,/b/.retry=welt}
\end{codeexample}
\end{handler}


\subsubsection{Handlers for Key Inspection}

\begin{handler}{{.show value}}
  This handler executes a |\show| command on the value stored in
  \meta{key}. This is useful mostly for debugging.

  \example |\pgfkeys{/my/obscure key/.show value}|
\end{handler}

\begin{handler}{{.show code}}
  This handler executes a |\show| command on the code stored in
  \meta{key}|/.@cmd|. This is useful mostly for debugging.

  \example |\pgfkeys{/my/obscure key/.show code}|
\end{handler}

The following key is not a handler, but it also commonly used for
inspecting things:
\begin{itemize}
  \itemoption{/utils/exec}|=|\marg{code}
  This key will simply execute the given \meta{code}. 

  \example |\pgfkeys{some key=some value,/utils/exec=\show\hallo,obscure key=obscure}|
\end{itemize}


\subsection{Error Keys}

In certain situations errors can occur, like using an undefined
key. In these situations error keys are executed. They should store a
macro that gets two arguments: The first is the offending key
(possibly only after macro expansion), the second is the value that
was passed as a parameter (also possibly only after macro expansion).

Currently, error keys are simply executed. In the future it might be a
good idea to have different subkeys that are executed depending on the
language currently set so that users get a localized error message.

\begin{itemize}
  \itemoption{/errors/value required} This key is executed whenever a
  key is used without a value when a value is actually required.
  \itemoption{/errors/value forbidden} This key is executed whenever a
  key is used with a value when a value is actually forbidden.
  \itemoption{/errors/boolean expected} This key is executed whenever a
  key setup using |.is if| gets called with a parameter other than
  |true| or |false|.
  \itemoption{/errors/unknown choice value} This key is executed
  whenever a choice is used as a \meta{value} for a key setup using
  the |.is choice| handler that is not defined.
  \itemoption{/errors/unknown key} This key is executed
  whenever a key is unknown and no specific |.unknown| handler is found. 
\end{itemize}
% Copyright 2006 by Till Tantau
%
% This file may be distributed and/or modified
%
% 1. under the LaTeX Project Public License and/or
% 2. under the GNU Free Documentation License.
%
% See the file doc/generic/pgf/licenses/LICENSE for more details.


\section{Repeating Things: The Foreach Statement}
\label{section-foreach}

In this section the package |pgffor.sty| is described. It can be used
independently of \pgfname, but it works particularly well together with
\pgfname\ and \tikzname.

When you say |\usepackage{pgffor}|, two commands are defined:
|\foreach| and |\breakforeach|. Their behaviour is described in the
following:

\begin{command}{\foreach| |\meta{variables}| in |\marg{list}
    \meta{commands}}
  The syntax of this command is a bit complicated, so let us go
  through it step-by-step.

  In the easiest case, \meta{variables} is a single \TeX-command like
  |\x| or |\point|. (If you want to have some fun, you can also use
  active characters. If you do not know what active characters are,
  you are blessed.)

  Still in the easiest case, \meta{list} is a comma-separated list of
  values. Anything can be used as a value, but numbers are most
  likely.

  Finally, in the easiest case, \meta{commands} is some \TeX-text in
  curly braces.

  With all these assumptions, the |\foreach| statement will execute
  the \meta{commands} repeatedly, once for every element of the
  \meta{list}. Each time the \meta{commands} are executed, the
  \meta{variable} will be set to the current value of the list item.

\begin{codeexample}[]
\foreach \x in {1,2,3,0} {[\x]}
\end{codeexample}

  \medskip
  \textbf{Syntax for the commands.}
  Let use move on to a more complicated setting. The first
  complication occurs when the \meta{commands} are not some text in
  curly braces. If the |\foreach| statement does not encounter an
  opening brace, it will instead scan everything up to the next
  semicolon and use this as \meta{commands}. This is most useful in
  situations like the following:

\begin{codeexample}[]
\tikz
  \foreach \x in {0,1,2,3}
    \draw (\x,0) circle (0.2cm);
\end{codeexample}

  However, the ``reading till the next semicolon'' is not the whole
  truth. There is another rule: If a |\foreach| statement is directly
  followed by another |\foreach| statement, this second foreach
  statement is collected as \meta{commands}. This allows you to write
  the following:

\begin{codeexample}[]
\begin{tikzpicture}
  \foreach \x in {0,1,2,3}
    \foreach \y in {0,1,2,3}
      {
        \draw (\x,\y) circle (0.2cm);
        \fill (\x,\y) circle (0.1cm);
      }
\end{tikzpicture}
\end{codeexample}

  \medskip
  \textbf{The dots notation.}
  The second complication concerns the \meta{list}. If this
  \meta{list} contains the list item ``|...|'', this list item is replaced
  by the ``missing values.'' More precisely, the following happens:

  Normally, when a list item |...| is encountered, there should
  already have been \emph{two} list items before it, which where
  numbers. Examples of \emph{numbers} are |1|, |-10|, or
  |-0.24|. Let us call these numbers $x$ and $y$ and let $d := y-x$ be
  their difference. Next, there should also be one number following
  the three dots, let us call this number~$z$.

  In this situation, the part of the list reading
  ``$x$|,|$y$|,...,|$z$'' is replaced by ``$x$, $x+d$, $x+2d$, $x+3d$,
  \dots, $x+md$,'' where the last dots are semantic dots, not
  syntactic dots. The value $m$ is the largest number such that $x +
  md \le z$ if $d$ is positive or such that $x+md \ge z$ if $d$ is
  negative. 

  Perhaps it is best to explain this by some examples:  The following
  \meta{list} have the same effects:

  |\foreach \x in {1,2,...,6} {\x, }| yields \foreach \x in {1,2,...,6} {\x, }

  |\foreach \x in {1,2,3,...,6} {\x, }| yields \foreach \x in {1,2,3,...,6} {\x, }

  |\foreach \x in {1,3,...,11} {\x, }| yields \foreach \x in {1,3,...,11} {\x, }

  |\foreach \x in {1,3,...,10} {\x, }| yields \foreach \x in {1,3,...,10} {\x, }

  |\foreach \x in {0,0.1,...,0.5} {\x, }| yields \foreach \x in {0,0.1,...,0.5} {\x, }

  |\foreach \x in {a,b,9,8,...,1,2,2.125,...,2.5} {\x, }| yields \foreach \x in {a,b,9,8,...,1,2,2.125,...,2.5} {\x, }

  As can be seen, for fractional steps that are not multiples of
  $2^{-n}$ for some small $n$, rounding errors can occur pretty
  easily. Thus, in the second last case, |0.5| should probably be
  replaced by |0.501| for robustness.
  
  There is yet another special case for the |...| statement: If the
  |...| is used right after the first item in the list, that is, if
  there is an $x$, but no $y$, the difference $d$ obviously cannot be
  computed and is set to $1$ if the number $z$ following the dots is
  larger than $x$ and is set to $-1$ if $z$ is smaller:

  |\foreach \x in {1,...,6} {\x, }| yields \foreach \x in {1,...,6} {\x, }

  |\foreach \x in {9,...,3.5} {\x, }| yields \foreach \x in {9,...,3.5} {\x, }

  \medskip
  \textbf{Special handling of pairs.}
  Different list items are separated by commas. However, this causes a
  problem when the list items contain commas themselves as pairs like
  |(0,1)| do. In this case, you should put the items containing commas
  in braces as in |{(0,1)}|. However, since pairs are such a natural
  and useful case, they get a special treatment by the |\foreach|
  statement. When a list item starts with a |(| everything up to the
  next |)| is made part of the item. Thus, we can write things like
  the following:

\begin{codeexample}[]
\tikz
  \foreach \position in {(0,0), (1,1), (2,0), (3,1)}
    \draw \position rectangle +(.25,.5);
\end{codeexample}
  
  \medskip
  \textbf{Using the foreach-statement inside paths.}
  \tikzname\ allows you to use a |\foreach| statement inside a path
  construction. In such a case, the \meta{commands} must be path
  construction commands. Here are two examples:

\begin{codeexample}[]
\tikz
  \draw (0,0)
    \foreach \x in {1,...,3}
      { -- (\x,1) -- (\x,0) }
    ;
\end{codeexample}

\begin{codeexample}[]
\tikz \draw \foreach \p in {1,...,3} {(\p,1)--(\p,3) (1,\p)--(3,\p)};
\end{codeexample}
    
  \medskip
  \textbf{Multiple variables.}
  You will often wish to iterate over two variables at the same
  time. Since you can nest |\foreach| loops, this is normally
  straight-forward. However, you sometimes wish variables to
  iterate ``simultaneously.'' For example, we might be given a list of
  edges that connect two coordinates and might wish to iterate over
  these edges. While doing so, we would like the source and target of
  the edges to be set to two different variables.

  To achieve this, you can use the following syntax: The
  \meta{variables} may not only be a single \TeX-variable. Instead, it
  can also be a list of variables separated by slashes (|/|). In this
  case the list items can also be lists of values separated by
  slashes.

  Assuming that the \meta{variables} and the list items are lists of
  values, each time the \meta{commands} are executed, each of the
  variables in \meta{variables} is set to one part of the list making
  up the current list item. Here is an example to clarify this:

  \example |\foreach \x / \y in {1/2,a/b} {``\x\ and \y''}| yields
  \foreach \x / \y in {1/2,a/b} {``\x\ and \y''}.

  If some entry in the \meta{list} does not have ``enough'' slashes,
  the last entry will be repeated. Here is an example:
\begin{codeexample}[]
\begin{tikzpicture}
  \foreach \x/\xtext in {0,...,3,2.72 / e}
    \draw (\x,0) node{$\xtext$};
\end{tikzpicture}
\end{codeexample}
  
  Here are more useful examples:
\begin{codeexample}[]
\begin{tikzpicture}
  % Define some coordinates:
  \tikzstyle{every node}=[draw,fill]
  \path[shape=circle,fill=examplefill]
    (0,0)    node(a) {a}
    (2,0.55) node(b) {b}
    (1,1.5)  node(c) {c}
    (2,1.75) node(d) {d};

  % Draw some connections:
  \foreach \source/\target in {a/b, b/c, c/a, c/d}
    \draw (\source) .. controls +(.75cm,0pt) and +(-.75cm,0pt)..(\target);  
\end{tikzpicture}
\end{codeexample}

\begin{codeexample}[]
\begin{tikzpicture}
  % Let's draw circles at interesting points:
  \foreach \x / \y / \diameter in {0 / 0 / 2mm, 1 / 1 / 3mm, 2 / 0 / 1mm}
    \draw (\x,\y) circle (\diameter);
    
  % Same effect
  \foreach \center/\diameter in {{(0,0)/2mm}, {(1,1)/3mm}, {(2,0)/1mm}}
    \draw[yshift=2.5cm] \center circle (\diameter);
\end{tikzpicture}
\end{codeexample}

\begin{codeexample}[]
\begin{tikzpicture}[cap=round,line width=3pt]
  \filldraw [fill=examplefill] (0,0) circle (2cm);

  \foreach \angle / \label in
    {0/3, 30/2, 60/1, 90/12, 120/11, 150/10, 180/9,
     210/8, 240/7, 270/6, 300/5, 330/4}
  {
    \draw[line width=1pt] (\angle:1.8cm) -- (\angle:2cm);
    \draw (\angle:1.4cm) node{\textsf{\label}};
  }
  
  \foreach \angle in {0,90,180,270}
    \draw[line width=2pt] (\angle:1.6cm) -- (\angle:2cm);

  \draw (0,0) -- (120:0.8cm); % hour
  \draw (0,0) -- (90:1cm);    % minute
\end{tikzpicture}%
\end{codeexample}

\begin{codeexample}[]
\tikz[shading=ball]
  \foreach \x / \cola in {0/red,1/green,2/blue,3/yellow}
    \foreach \y / \colb in {0/red,1/green,2/blue,3/yellow}
      \shade[ball color=\cola!50!\colb] (\x,\y) circle (0.4cm);
\end{codeexample}
\end{command}


\begin{command}{\breakforeach}
  If this command is given inside a |\foreach| command, no further
  executions of the \meta{commands} will occur. However, the current
  execution of the \meta{commands} is continued normally, so it is
  probably best to use this command only at the end of a |\foreach|
  command. 

\begin{codeexample}[]
\begin{tikzpicture}
  \foreach \x in {1,...,4}
    \foreach \y in {1,...,4}
    {
      \fill[red!50] (\x,\y) ellipse (3pt and 6pt);

      \ifnum \x<\y
        \breakforeach
      \fi
    }      
\end{tikzpicture}
\end{codeexample}
  
\end{command}



% Copyright 2006 by Till Tantau
%
% This file may be distributed and/or modified
%
% 1. under the LaTeX Project Public License and/or
% 2. under the GNU Free Documentation License.
%
% See the file doc/generic/pgf/licenses/LICENSE for more details.


\section{Date and Calendar Utility Macros}
\label{section-calendar}

This section describes the package |pgfcalendar|.

\begin{package}{pgfcalendar}
  This package can be used independently of \pgfname. It has two
  purposes:
  \begin{enumerate}
  \item It provides functions for working with dates. Most noticably,
    it can convert a date in ISO-standard format (like 1975-12-26) to
    a so-called Julian day number, which is defined in Wikipedia as
    follows:  ``The Julian day or Julian day number is the
    (integer) number of days that have elapsed since the initial epoch
    at noon Universal Time (UT) Monday, January 1, 4713 BC in the
    proleptic Julian calendar.'' The package also provides a function
    for converting a Julian day number to an ISO-format date.

    Julian day numbers make it very easy to work with days. For
    example, the date ten days in the future of 2008-02-20 can
    be computed by converting this date to a Julian day number, adding
    10, and then converting it back. Also, the day of week of a given
    date can be computed by taking the Julian day number modulo~7.
  \item It provides a macro for typesetting a calendar. This macro
    is highly configurable and flexible (for example, it can produce
    both plain text calendars and also complicated \tikzname-based
    calendars), but most users will not use the macro directly. It is
    the job of a frontend to provide useful configruations for
    typesetting calendars based on this command.
  \end{enumerate}
\end{package}


\subsection{Handling Dates}

\subsubsection{Conversions Between Date Types}

\begin{command}{\pgfcalendardatetojulian\marg{date}\marg{counter}}
  This macro converts a date in a format to be described in a moment
  to the Julian day number in the Gregorian calendar. The \meta{date}
  should expand to a string of the following form:
  \begin{enumerate}\label{calendar-date-format}
  \item It should start with a number representing the year. Use
    |\year| for the current year, that is, the year the file is being
    typeset.
  \item The year must be followed by a hyphen.
  \item Next should come a number representing the month. Use |\month|
    for the current month. You can, but need not, use leading
    zeros. For example, |02| represents February, just like |2|.
  \item The month must also be followed by a hyphen.
  \item Next you must either provide a day of month (again, a number
    and, again, |\day| yields the current day of month) or the keyword
    |last|. This keyword refers to the last day of the month, which is
    automatically computed (and which is a bit tricky to compute,
    especially for February).
  \item Optionally, you can next provide a plus sign followed by
    positive or negative number. This number of days will be added to
    the computed date.
  \end{enumerate}

  Here are some examples:
  \begin{itemize}
  \item |2006-01-01| refers to the first day of 2006.
  \item |2006-02-last| refers to February 28, 2006.
  \item |\year-\month-\day| refers to today.
  \item |2006-01-01+2| refser to January 3, 2006.
  \item |\year-\month-\day+1| refers to tomorrow.
  \item |\year-\month-\day+-1| refers to yesterday.
  \end{itemize}
  
  The conversion method is taken from the English Wikipedia entry on
  Julian days. 

  \newcount\mycount
  \example |\pgfcalendardatetojulian{2007-01-14}{\mycount}| sets
  |\mycount| to
  \pgfcalendardatetojulian{2007-01-14}{\mycount}\the\mycount.

  
\end{command}

\begin{command}{\pgfcalendarjuliantodate\marg{Julian day}\marg{year
      macro}\marg{month macro}\marg{day macro}}
  This command converts a Julian day number to an ISO-date. The
  \meta{Julian day} must be a number or \TeX\ counter, the \meta{year macro},
  \meta{month macro} and \meta{day macro} must be \TeX\ macro
  names. They will be set to numbers representing the year, month, and
  day of the given Julian day in the Gregorian calendar.

  The \meta{year macro} will be assigned the year without leading
  zeros. Note that this macro will produce year 0 (as opposed to other
  calendars, where year 0 does not exist). However, if you really need
  calendars for before the year 1, it is expected that you know what
  you are doing anyway.

  The \meta{month macro} gets assigned a two-digit number representing
  the month (with a leading zero, if necessary). Thus, the macro is
  set to |01| for January.

  The \meta{day macro} gets assigned a two-digit number representing
  the day of the month (again with a possible leading zero).

  To convert a Julian day number to an ISO-date you use code like the
  following:
\begin{verbatim}
\pgfcalendardatetojulian{2454115}{\myyear}{\mymonth}{\myday}
\edef\isodate{\myyear-\mymonth-\myday}
\end{verbatim}
  The above code sets |\isodate| to
  \pgfcalendarjuliantodate{2454115}{\myyear}{\mymonth}{\myday}%
  \edef\isodate{\myyear-\mymonth-\myday}\texttt{\isodate}.
\end{command}


\begin{command}{\pgfcalendarjuliantoweekday\marg{Julian day}\marg{week day counter}}
  This command converts a Julian day to a week day by computing the
  day modulo 7. The \meta{week day counter} must be a \TeX\
  counter. It will be set to 0 for a Monday, to 1 for a Tuesday, and
  so on.

  \example |\pgfcalendarjuliantoweekday{2454115}{\mycount}| sets
  |\mycount| to
  \pgfcalendarjuliantoweekday{2454115}{\mycount}\the\mycount. 
\end{command}


\subsubsection{Checking Dates}


\begin{command}{\pgfcalendarifdate\marg{date}\marg{tests}\marg{code}\marg{else code}}
  \label{pgfcalendarifdate}
  This command is used to execute code based on properties of
  \meta{date}. The \meta{date} must be a date in ISO-format. For
  this date, the \meta{tests} are checked (to be detailed later)
  and if one of the tests applied, the \meta{code} is
  executed. If none of the tests applies, the \meta{else code} is
  executed.

  \example |\pgfcalendarifdate{2007-02-07}{Wednesday}{Is a Wednesday}{Is not a Wednesday}|
  yields \texttt{\pgfcalendarifdate{2007-02-07}{Wednesday}{Is a
      Wednesday}{Is not a Wednesday}}.

  The \meta{tests} is a comma-separated list of key-value
  pairs. The following are defined by default:
  \begin{itemize}
  \itemcalendaroption{all} This test is passed by all dates.
  \itemcalendaroption{Monday} This test is passed by all dates that
  are Mondays.
  \itemcalendaroption{Tuesday} as above.
  \itemcalendaroption{Wednesday} as above.
  \itemcalendaroption{Thursday} as above.
  \itemcalendaroption{Friday} as above.
  \itemcalendaroption{Saturday} as above.
  \itemcalendaroption{Sunday} as above.
  \itemcalendaroption{workday} Passed by Mondays, Tuesdays,
  Wednesdays, Thursdays, and Fridays.
  \itemcalendaroption{weekend} Passed Saturdays and Sundays. 
  \itemcalendaroption{equals}|=|\meta{reference} The \meta{reference}
  can be in one of two forms: Either, it is a full ISO format date
  like |2007-01-01| or the year may be missing as in |12-31|. In the
  first case, the test is passed if \meta{date} is the same as
  \meta{reference}. In the second case, the test is passed if the
  month and day part of \meta{date} is the same as \meta{reference}.

  For example, the test |equals=2007-01-10| will only be passed by this
  particular date. The test |equals=05-01| will be passed by every first
  of May on any year.
  \itemcalendaroption{at least}|=|\meta{reference} This test works
  similarly to the |equals| test, only it is checked whether
  \meta{date} is equal to \meta{reference} or to any later
  date. Again, the \meta{reference} can be a full date like
  |2007-01-01| or a short version like |07-01|. For example,
  |at least=07-01| is true for every day in the second half of any
  year.
  \itemcalendaroption{at most}|=|\meta{reference} as above.
  \itemcalendaroption{between}|=|\meta{start reference}| and |\meta{end
    refernce} This test checks whether the current date lies between
  the two given reference dates. Both full and short version may be
  given.

  For example |between=2007-01-01 and 2007-02-28| is true for the days
  in January and February of 2007.

  For another example, |between=05-01 and 05-07| is true for the
  days of the first week of May of any year.
  \itemcalendaroption{day of month}|=|\meta{number} Passed by the day
  of month of the \meta{date} is \meta{number}. For example, the test
  |day of month=1| is passed by every first of every month.
  \itemcalendaroption{end of month}\opt{|=|\meta{number}} Passed by
  the day of month of the \meta{date} that is \meta{number} from the
  end of the month. For example, the test |end of month=1| is passed
  by the last day of every month, the test |end of month=2| is passed
  by the second last day of every month. If \meta{number} is omitted,
  it is assumed to be |1|.
  \end{itemize}

  In addition to the above checks, you can also define new checks. To
  do so, you must add a new key to the key-value group |pgfcalendar|
  using |\define@key|. The job of the code of this new key is to
  possibly set the \TeX-if |\ifpgfcalendarmatches| to true (if it is
  already true, no action should be taken) to indicate that the
  \meta{date} passes the test setup by this new key.

  In order to perform the test, the key code needs to know the date
  that should be checked. The date is available through a macro, but a
  whole bunch of additional information about this date is also
  available through the following macros:
  \begin{itemize}
  \item |\pgfcalendarifdatejulian|
    is the Julian day number of the \meta{date} to be checked.
  \item |\pgfcalendarifdateweekday|
    is the weekday of the \meta{date} to be checked.
  \item |\pgfcalendarifdateyear|
    is the year of the \meta{date} to be checked.
  \item |\pgfcalendarifdatemonth|
    is the month of the \meta{date} to be checked.
  \item |\pgfcalendarifdateday|
    is the day of month of the \meta{date} to be checked.
  \end{itemize}

  For example, let us define a new key that checks whether the
  \meta{date} is a Workers day (first of May). This can be done as
  follows:
\begin{verbatim}
\define@key{pgfcalendar}{workers day}[]
{
  \ifnum\pgfcalendarifdatemonth=5\relax
    \ifnum\pgfcalendarifdateday=1\relax
      \pgfcalendarmatchestrue
    \fi
  \fi
}
\end{verbatim}
\end{command}


\subsubsection{Typesetting Dates}




\begin{command}{\pgfcalendarweekdayname\marg{week day number}}
  This command expands to a textual representation of the day of week,
  given by the \meta{week day number}. Thus,
  |\pgfcalendarweekdayname{0}| expands to |Monday| if the current
  language is English and to |Montag| if the current language is
  German, and so on. See Section~\ref{section-calendar-locale} for
  more details on translations.

  \example |\pgfcalendarweekdayname{2}| yields
  \texttt{\pgfcalendarweekdayname{2}}. 
\end{command}


\begin{command}{\pgfcalendarweekdayshortname\marg{week day number}}
  This command works similiarly to the previous command, only an
  abbreviated version of the week day is produced.

  \example |\pgfcalendarweekdayshortname{2}| yields
  \texttt{\pgfcalendarweekdayshortname{2}}. 
\end{command}


\begin{command}{\pgfcalendarmonthname\marg{month number}}
  This command expands to a textual representation of the month, which
  is given by the \meta{month number}.

  \example |\pgfcalendarmonthname{12}| yields
  \texttt{\pgfcalendarmonthname{12}}. 
\end{command}


\begin{command}{\pgfcalendarmonthshortname\marg{month number}}
  As above, only an abbreviated version is produced.

  \example |\pgfcalendarmonthshortname{12}| yields
  \texttt{\pgfcalendarmonthshortname{12}}.   
\end{command}



\subsubsection{Localization}

\label{section-calendar-locale}
All textual representations of week days or months (like ``Monday'' or
``February'') are wrapped with |\translate| commands from the
|translator| package (it this package is not loaded, no translation
takes place). Furthermore, the |pgfcalendar| package will try to load
the |translator-months-dictionary|, if the |translator| package is
loaded.

The net effect of all this is that all dates will be translated to the
current language setup in the |translator| package. See the
documentation of this package for more details.



\subsection{Typesetting Calendars}

\begin{command}{\pgfcalendar\marg{prefix}\marg{start date}\marg{end
      date}\marg{rendering code}}
  This command can be used to typeset a calendar. It is a very general
  command, the actual work has to be done by giving clever
  implementations of \meta{rendering code}. Note that this macro need
  \emph{not} be called inside a |{pgfpicture}| environment (even
  though it typically will be) and you can use it to typeset calendars
  in normal \TeX\ or using packages other than \pgfname.

  \medskip
  \textbf{Basic typesetting process.}
  A calendar is typeset as follows: The \meta{start date} and
  \meta{end date} specify a range of dates. For each date in this 
  range the \meta{rendering code} is executed with certain macros
  setup to yield information about the \emph{current date}
  (the current date in the enumeration of dates of the
  range). Typically, the \meta{rendering code} places nodes inside a
  picture, but it can do other things as well. Note that it is also
  the job of the \meta{rendering code} to position the calendar
  correctly. 

  The different calls of the \meta{rending code} are not
  surrounded by \TeX\ groups (though you can do so yourself, of
  course). This means that settings can accumulate between different
  calls, which is often desirable and useful.

  \medskip
  \textbf{Information about the current date.}
  Inside the \meta{rendering code}, different macros can be access:

  \begin{itemize}
  \item |\pgfcalendarprefix|
    The \meta{prefix} parameter. This prefix is recomended for nodes
    inside the calendar, but you have to use it yourself explicitly.
  \item |\pgfcalendarbeginiso|
    The \meta{start date} of range being typeset in ISO format (like 2006-01-10).
  \item |\pgfcalendarbeginjulian|
    Julian day number of \meta{start date}.
  \item |\pgfcalendarendiso|
    The \meta{end date} of range being typeset in ISO format.
  \item |\pgfcalendarendjulian|
    Julian day number of \meta{end date}.
  \item |\pgfcalendarcurrentjulian| This \TeX\ count holds the 
    Julian day number of day currently begin rendered.
  \item |\pgfcalendarcurrentweekday| The weekday
    (a number with zero representing Monday) of the current date.
  \item |\pgfcalendarcurrentyear| The year of the current date.
  \item |\pgfcalendarcurrentmonth| The month of the current date
    (always two digits with a leading zero, if necessary).
  \item |\pgfcalendarcurrentday| The day of month of the current date
    (alwyas two digits).
  \end{itemize}

  \medskip
  {\bfseries The |\ifdate| command.}
  Inside the |\pgfcalendar| the macro |\ifdate| is available
  locally:
  \begin{command}{\ifdate\marg{tests}\marg{code}\marg{else code}}
    This command has the same effect as calling |\pgfcalendarifdate|
    for the current date.
  \end{command}

  \medskip
  \textbf{Examples.}
  In a first example, let us create a very simple calendar: It just
  lists the dates in a certain range.
\begin{codeexample}[vbox,ignorespaces]
\pgfcalendar{cal}{2007-01-20}{2007-02-10}{\pgfcalendarcurrentday\ }    
\end{codeexample}
  Let us now make this a little more interesting: Let us add a line
  break after each Sunday.
\begin{codeexample}[vbox,ignorespaces]
\pgfcalendar{cal}{2007-01-20}{2007-02-10}
{
  \pgfcalendarcurrentday\
  \ifdate{Sunday}{\par}{}
}    
\end{codeexample}
  We now want to have all Mondays to be aligned on a column. For this,
  different approaches work. Here is one based positioning each day
  horizontally using a skip.
\begin{codeexample}[vbox,ignorespaces]
\pgfcalendar{cal}{2007-01-20}{2007-02-10}
{%
  \leavevmode%
  \hbox to0pt{\hskip\pgfcalendarcurrentweekday cm\pgfcalendarcurrentday\hss}%
  \ifdate{Sunday}{\par}{}%
}    
\end{codeexample}
  Let us now typeset two complete months.
\begin{codeexample}[vbox,ignorespaces]
\pgfcalendar{cal}{2007-01-01}{2007-02-28}{%
  \ifdate{day of month=1}{
    \par\bigskip\hbox to7.5cm{\itshape\hss\pgfcalendarshorthand{J}\hss}\par
  }{}%
  \leavevmode%
  {%
    \ifdate{weekend}{\color{black!50}}{\color{black}}%
    \hbox to0pt{%
      \hskip\pgfcalendarcurrentweekday cm%
      \hbox to1cm{\hss\pgfcalendarshorthand{d}}\hss%
    }%
  }%
  \ifdate{Sunday}{\par}{}%
}    
\end{codeexample}
  For our final example, we use a |{tikzpicture}|. 
\begin{codeexample}[vbox,ignorespaces]
\begin{tikzpicture}    
  \pgfcalendar{cal}{2007-01-20}{2007-02-10}{%
    \ifdate{workday}{\tikzstyle{filling}=[fill=blue!20]}{\tikzstyle{filling}=[fill=red!20]}
    \node (\pgfcalendarsuggestedname) at (\pgfcalendarcurrentweekday,0)
      [anchor=base,circle,filling] {\pgfcalendarcurrentday};
    \ifdate{Sunday}{\pgftransformyshift{-3em}}{}%
  }
  \draw (cal-2007-01-21) -- (cal-2007-02-03);
\end{tikzpicture}
\end{codeexample}
\end{command}


\begin{command}{\pgfcalendarshorthand\marg{kind}\marg{representation}}
  \label{pgfcalendarshorthand}
  This command can be used inside a |\pgfcalendar|, where it will
  expand to a representation of the current day, month, year or day of
  week, depending on whether \meta{kind} is |d|, |m|, |y| or |w|. The
  \meta{representation} can be one of the following: |-|, |=|, |0|,
  |.|, and |t|. They have the following meanings:
  \begin{itemize}
  \item The minus sign selects the shortest numerical representation
    possible (no leading zeros).
  \item The equal sign also selects the shortest numerical
    representation, but a space is added to single digit days and
    months (thereby ensuring that they have the same length as other
    days).
  \item The zero digit selects a two-digit numerical representation
    for days and months. For years it is allowed, but has no effect.
  \item The letter |t| selects a textual representation.
  \item The dot selects an abbreviated textual representation.
  \end{itemize}
  Normally, you should say |\let\%=\pgfcalendarshorthand| locally, so
  that you can write |\%wt| instead of the much more cumbersome
  |\pgfcalendarshorthand{w}{t}|. 

\begin{codeexample}[]
\let\%=\pgfcalendarshorthand
\pgfcalendar{cal}{2007-01-20}{2007-01-20}
{ ISO form: \%y0-\%m0-\%d0, long form: \%wt, \%mt \%d-, \%y0}    
\end{codeexample}
\end{command}


\begin{command}{\pgfcalendarsuggestedname}
  This macro expands to a suggested name for nodes representing days
  in a calendar. In detail, it expands to the \meta{prefix} of the
  calendar, followed by a hyphen, followed by the ISO format version
  of the date. Thus, when the date |2007-01-01| is typeset in a
  calendar for the prefix |mycal|, the macro expands to
  |mycal-2007-01-01|. 
\end{command}



% Copyright 2003 by Till Tantau <tantau@cs.tu-berlin.de>.
%
% This program can be redistributed and/or modified under the terms
% of the LaTeX Project Public License Distributed from CTAN
% archives in directory macros/latex/base/lppl.txt.


\section{Page Management}

This section describes the |pgfpages| packages. Although this package
is not concerned with creating pictures, its implementation relies so
heavily on \pgfname\ that it is documented here. Currently, |pgfpages|
only works with \LaTeX, but if you are adventurous, feel free to hack
the code so that it also works with plain \TeX.

The aim of |pgfpages| is to provide a flexible way of putting multiple
pages on a single page \emph{inside \TeX}. Thus, |pgfpages| is quite
different from useful tools like |psnup| or |pdfnup| insofar as it
creates its output in a single pass. Furthermore, it works uniformly
with both |latex| and |pdflatex|, making it easy to put multiple pages
on a single page without any fuss.

A word of warning: \emph{using |pgfpages| will destroy
  hyperlinks}. Actually, the hyperlinks are not destroyed, only they
will appear at totally wrong positions on the final output. This is
due to a fundamental flaw in the \pdf\ specification: In \pdf\ the
bounding rectangle of a hyperlink is given in ``absolute
page coordinates'' and translations or rotations do not affect
them. Thus, the transformations applied by |pgfpages| to put the pages
where you want them are (cannot, even) be applied to the coordinates
of hyperlinks. It is unlikely that this will change in the foreseeable
future.


\subsection{Basic Usage}

The internals of |pgfpages| are complex since the package can do all
sorts of interesting tricks. For this reason, so-called \emph{layouts}
are predefined that setup all option in appropriate ways.

You use a layout as follows:
\begin{codeexample}[code only]
\documentclass{article}

\usepackage{pgfpages}
\pgfpagesuselayout{2 on 1}[a4paper,landscape,border shrink=5mm]

\begin{document}
This text is shown on the left.
\clearpage
This text is shown on the right.
\end{document}
\end{codeexample}

The layout |2 on 1| puts two pages on a single page. The option
|a4paper| tells |pgfpages| that the \emph{resulting} page (called the
\emph{physical} page in the following) should be |a4paper| and it
should be landscape (which is quite logical since putting two portrait
pages next to each other gives a landscape page). Normally, the
\emph{logical} pages, that is, the pages that \TeX\ ``thinks'' that it
is typesetting, will have the same sizes, but this need not be the
case. |pgfpages| will automatically scale down the logical pages such
that two logical pages fit next to each other inside a DIN A4 page.

The |border shrink| tells |pgfpages| that it should add an additional
5mm to the shrinking such that a 5mm-wide border is shown around the
resulting logical pages.

As a second example, let us put two pages produced by the
\textsc{beamer} class on a single page:

\begin{codeexample}[code only]
\documentclass{beamer}

\usepackage{pgfpages}
\pgfpagesuselayout{2 on 1}[a4paper,border shrink=5mm]

\begin{document}
\begin{frame}
  This text is shown at the top.
\end{frame}
\begin{frame}
  This text is shown at the bottom.
\end{frame}
\end{document}
\end{codeexample}

Note that we do not use the |landscape| option since \textsc{beamer}'s
logical pages are already in landscape mode and putting two landscape
pages on top of each other results in a portrait page. However, if you
had used the |4 on 1| layout, you would have had to add |landscape|
once more, using the |8 on 1| you must not, using |16 on 1| you need
it yet again. And, no, there is no |32 on 1| layout.

Another word of caution: \emph{using |pgfpages| will produce wrong
  page numbers in the |.aux| file}. The reason is that \TeX\
instantiates the page numbers when writing an |.aux| file only when
the physical page is shipped out. Fortunately, this problem is easy to
fix: First, typeset our file normally without using the
|\pgfpagesuselayout| command (just put the comment marker |%| before it)
Then, rerun \TeX\ with the |\pgfpagesuselayout| command included and add
the command |\nofiles|. This command ensures that the |.aux| file is
not modified, which is exactly what you want. So, to typeset the above
example, you should actually first \TeX\ the following file:

\begin{codeexample}[code only]
\documentclass{article}

\usepackage{pgfpages}
%%\pgfpagesuselayout{2 on 1}[a4paper,landscape,border shrink=5mm]
%%\nofiles

\begin{document}
This text is shown on the left.
\clearpage
This text is shown on the right.
\end{document}
\end{codeexample}
and then typeset
\begin{codeexample}[code only]
\documentclass{article}

\usepackage{pgfpages}
\pgfpagesuselayout{2 on 1}[a4paper,landscape,border shrink=5mm]
\nofiles

\begin{document}
This text is shown on the left.
\clearpage
This text is shown on the right.
\end{document}
\end{codeexample}

The final basic example is the |resize to| layout (it works a bit like
a hypothetical |1 on 1| layout). This layout resizes the logical page
such that is fits the specified physical size. Since this does not
change the page numbering, you need not worry about the |.aux| files
with this layout. For example, adding the following lines will ensure
that the physical output will fit on DIN A4 paper:
\begin{codeexample}[code only]
\usepackage{pgfpages}
\pgfpagesuselayout{resize to}[a4paper]
\end{codeexample}

This can be very useful when you have to handle lots of papers that
are typeset for, say, letter paper and you have an A4 printer or the
other way round. For example, the following article will be fit for
printing on letter paper:
\begin{codeexample}[code only]
\documentclass[a4paper]{article}
%% a4 is currently the logical size and also the physical size

\usepackage{pgfpages}
\pgfpagesuselayout{resize to}[letterpaper]
%% a4 is still the logical size, but letter is the physical one

\begin{document}
  \title{My Great Article}
...
\end{document}
\end{codeexample}



\subsection{The Predefined Layouts}

This section explains the predefined layouts in more detail. You
select a layout using the following command:
\begin{command}{\pgfpagesuselayout\marg{layout}\oarg{options}}
  Installs the specified \meta{layout} with the given \meta{options}
  set. The predefined layouts and their permissible options are
  explained below.

  If this function is called multiple times, only the last call
  ``wins.'' You can thereby overwrite any previous settings. In
  particular, layouts \emph{do not} accumulate.

  \example |\pgfpagesuselayout{resize to}[a4paper]|
\end{command}

\begin{pgflayout}{resize to}
  This layout is used to resize every logical page to a specified
  physical size. To determine the target size, the following options
  may be given:
  \begin{itemize}
  \item
    \declare{|physical paper height=|\meta{size}} sets the
    height of the physical pape size to \meta{size}.
  \item
    \declare{|physical paper width=|\meta{size}} sets the
    width of the physical pape size to \meta{size}.
  \item
    \declare{|a0paper|} sets the physical page size to DIN A0 paper.
  \item
    \declare{|a1paper|} sets the physical page size to DIN A1 paper.
  \item
    \declare{|a2paper|} sets the physical page size to DIN A2 paper.
  \item
    \declare{|a3paper|} sets the physical page size to DIN A3 paper.
  \item
    \declare{|a4paper|} sets the physical page size to DIN A4 paper.
  \item
    \declare{|a5paper|} sets the physical page size to DIN A5 paper.
  \item
    \declare{|a6paper|} sets the physical page size to DIN A6 paper.
  \item
    \declare{|letterpaper|} sets the physical page size to the
    American letter paper size.
  \item
    \declare{|legalpaper|} sets the physical page size to the
    American legal paper size.
  \item
    \declare{|executivepaper|} sets the physical page size to the
    American executive paper size.
  \item
    \declare{|landscape|} swaps the height and the width of the
    physical paper.
  \item
    \declare{|border shrink=|\meta{size}} additionally reduces the
    size of the logical page on the physical page by \meta{size}.
  \end{itemize}
\end{pgflayout}

\begin{pgflayout}{2 on 1}
  Puts two logical pages alongside each other on each physical page if
  the logical height is larger than the logical width (logical pages
  are in portrait mode). Otherwise, two
  logical pages are put on top of each other (logical pages are in
  landscape mode). When using this layout, it is advisable to use the
  |\nofiles| command, but this is not done automatically.

  The same \meta{options} as for the |resize to| layout an be used,
  plus the following option:
  \begin{itemize}
  \item
    \declare{|odd numbered pages right|}
    places the first page on the right.
  \end{itemize}
\end{pgflayout}


\begin{pgflayout}{4 on 1}
  Puts four logical pages on a single physical page.
  The same \meta{options} as for the |resize to| layout an be used.
\end{pgflayout}

\begin{pgflayout}{8 on 1}
  Puts eight logical pages on a single physical page. As for |2 on 1|,
  the orientation depends on whether the logical pages are in
  landscape mode or in portrait mode.
\end{pgflayout}

\begin{pgflayout}{16 on 1}
  This is for the \textsc{ceo}.
\end{pgflayout}

\begin{pgflayout}{rounded corners}
  \label{layout-rounded-corners}
  This layout adds ``rounded corners'' to every page, which,
  supposedly, looks nicer during presentations with projectors
  (personally, I doubt this). This is done by (possibly) resizing the
  page to the physical page size. Then four black rectangles are
  drawn in each corner. Next, a clipping region is set up that
  contains all of the logical page except for little rounded
  corners. Finally, the logical page is draw, clipped against the
  clipping region. 

  Note that every logical page should fill its background for this to
  work.

  In addition to the \meta{options} that can be given to |resize to|
  the following options may be given.
  \begin{itemize}
    \item \declare{|corner width=|\meta{size}} specifies the size of
    the corner.
  \end{itemize}

  \begin{codeexample}[code only]
\documentclass{beamer}
\usepackage{pgfpages}
\pgfpagesuselayout{rounded corners}[corner width=5pt]
\begin{document}
...
\end{document}
\end{codeexample}
\end{pgflayout}

\begin{pgflayout}{two screens with lagging second}
  This layout puts two logical pages alongside each other. The second
  page always shows what the main
  page showed on the previous physical page. Thus, the second page
  ``lags behind'' the main page. This can be useful when you have to
  projectors attached to your computer and can show different parts of
  a physical page on different projectors.

  The following \meta{options} may be given:
  \begin{itemize}
  \item \declare{|second right|} puts the second page right of the
    main page. This will make the physical pages twice as wide
    as the logical pages, but it will retain the height.
  \item \declare{|second left|} puts the second page left,
    otherwise it behave the same as |second right|.
  \item \declare{|second bottom|} puts the second page below the main
    page. This make the physical pages twice as high as the logical
    ones.
  \item \declare{|second top|} works like |second bottom|.      
  \end{itemize}
\end{pgflayout}

\begin{pgflayout}{two screens with optional second}
  This layout works similarly to
  |two screens with lagging second|. The difference is that the
  contents of the second screen only changes when one of the commands
  |\pgfshipoutlogicalpage{2}|\marg{box} or
  |\pgfcurrentpagewillbelogicalpage{2}| is called. The first puts the
  given \meta{box} on the second page. The second specifies that the
  current page should be put there, once it is finished.

  The same options as for |two screens with lagging second| may be
  given. 
\end{pgflayout}



You can define your own predefined layouts using the following
command:

\begin{command}{\pgfpagesdeclarelayout\marg{layout}\marg{before
      actions}\marg{after actions}}
  This command predefines a \meta{layout} that can later be installed
  using the |\pgfpagesuselayout| command.

  When |\pgfpagesuselayout|\marg{layout}\oarg{options} is called, the
  following happens: First, the \meta{before actions} are
  executed. They can be used, for example, to setup default values for
  keys. Next, |\setkeys{pgfpagesuselayoutoption}|\marg{options} is
  executed. Finally, the \meta{after actions} are executed.

  Here is an example:
\begin{codeexample}[code only]
\pgfpagesdeclarelayout{resize to}
{
  \def\pgfpageoptionborder{0pt}
}
{
  \pgfpagesphysicalpageoptions
  {%
    logical pages=1,%
    physical height=\pgfpageoptionheight,%
    physical width=\pgfpageoptionwidth%
  }
  \pgfpageslogicalpageoptions{1}
  {%
    resized width=\pgfphysicalwidth,%
    resized height=\pgfphysicalheight,%
    border shrink=\pgfpageoptionborder,%
    center=\pgfpoint{.5\pgfphysicalwidth}{.5\pgfphysicalheight}%
  }%
}
\end{codeexample}
\end{command}




\subsection{Defining a Layout}

If none of the predefined layouts meets your problem or if you wish to
modify them, you can create layouts from scratch. This section
explains how this is done.

Basically, |pgfpages| hooks into \TeX's |\shipout| function. This
function is called whenever \TeX\ has completed typesetting a page and
wishes to send this page to the |.dvi| or |.pdf| file. The |pgfpages|
package redefines this command. Instead of sending the page to the output
file, |pgfpages| stores it in an internal box and then acts as if the
page had been output. When \TeX\ tries to output the next page using
|\shipout|, this call is once more intercepted and the page is stored
in another box. These boxes are called \emph{logical pages}.

At some point, enough logical pages have been accumulated such that a
\emph{physical page} can be output. When this happens, |pgfpages|
possibly scales, rotates, and translates the logical pages (and
possibly even does further modifications) and then puts them at
certain positions of the \emph{physical} page. Once this page is fully
assembled, the ``real'' or ``original'' |\shipout| is called to
send the physical page to the output file.

In reality, things are slightly more complicated. First, once a
physical page has been shipped out, the logical pages are usually
voided, but this need not be the case. Instead, it is possible that
certain logical page just retain their contents after the physical
page has been shipped out and these pages need not be filled once more
before a physical shipout can occur. However, the contents of these
logical pages can still be changed using special commands. It is also
possible that after a shipout certain logical pages are filled with
the contents of \emph{other} logical pages.

A \emph{layout} defines for each logical page where it will go on the
physical page and which further modifications should be done. The
following two commands are used to define the layout:

\begin{command}{\pgfpagesphysicalpageoptions\marg{options}}
  This command sets the characteristic of the ``physical'' page. For
  example, it is used to specify how many logical pages there are and
  how many logical pages must be accumulated before a physical page is
  shipped out. How each individual logical page is typeset is
  specified using the command |\pgfpageslogicalpageoptions|, described
  later.

  \example A layout for putting two portrait pages on a single
  landscape page:
\begin{codeexample}[code only]
\pgfpagesphysicalpageoptions
{%
  logical pages=2,%
  physical height=\paperwidth,%
  physical width=\paperheight,%
}

\pgfpageslogicalpageoptions{1}
{%
  resized width=.5\pgfphysicalwidth,%
  resized height=\pgfphysicalheight,%
  center=\pgfpoint{.25\pgfphysicalwidth}{.5\pgfphysicalheight}%
}%
\pgfpageslogicalpageoptions{2}
{%
  resized width=.5\pgfphysicalwidth,%
  resized height=\pgfphysicalheight,%
  center=\pgfpoint{.75\pgfphysicalwidth}{.5\pgfphysicalheight}%
}%
\end{codeexample}

  The following \meta{options} may be set:
  \begin{itemize}
    \item \declare{|logical pages=|\meta{logical pages}} specified how many
    logical pages there are, in total. These are numbered 1 to
    \meta{logical pages}.
    \item \declare{|first logical shipout=|\meta{first}}. See the the
      next option. By default, \meta{first} is 1.
    \item \declare{|last logical shipout=|\meta{last}}. Together
    with the previous option, these two options define an interval of
    pages inside the range 1 to \meta{logical pages}. Only this range
    is used to store the pages that are shipped out by \TeX. This
    means that after a physical shipout has just occured (or at the
    beginning), the first time \TeX\ wishes to perform a shipout, the
    page to be shipped out is stored in logical page \meta{first}. The
    next time \TeX\ performs a shipout, the page is stored in logical
    page $\meta{first} +1$ and so on, until the logical page
    \meta{last} is also filled. Once this happens, a physical shipout
    occurs and the process starts once more.

    Note that logical pages that lie outside the interval between
    \meta{first} and \meta{last} are filled only indirectly or when
    special commands are used.

    By default, \meta{last} equals \meta{logical pages}.
  \item \declare{|current logical shipout=|\meta{current}} changes
    an internal counter such that \TeX's next logical shipout will be
    stored in logical page \meta{current}.

    This option can be used to ``warp'' the logical page filling
    mechanism to a certain page. You can both skip logical pages and
    overwrite already filled logical pages. After the logical page
    \meta{current} has been filled, the internal counter is
    incremented normally as if the logical page \meta{current} had
    been ``reached'' normally. If you specify a \meta{current} larger
    to \meta{last}, a physical shipout will occur after the logical
    page \meta{current} has been filled.
  \item
    \declare{|physical height=|\meta{height}}
    specifies the height of the physical pages. This height is
    typically different from the normal  |\paperheight|, which is used
    by \TeX\ for its typesetting and page breaking purposes.
  \item
    \declare{|physical width=|\meta{width}}
    specifies the physical width.
  \end{itemize}
\end{command}


\begin{command}{\pgfpageslogicalpageoptions\marg{logical page number}\marg{options}}
  This command is used to specify where the logical page number
  \meta{logical page number} will be placed on the physical page. In
  addition, this command can be used to install additional ``code'' to
  be executed when this page is put on the physical page.

  The number \meta{logical page number} should be between 1 and
  \meta{logical pages}, which has previously been installed using the
  |\pgfpagesphysicalpageoptions| command.

  The following \meta{options} may be given:
  \begin{itemize}
  \item
    \declare{|center=|\meta{pgf point}}
    specifies the center of the logical page inside the physical page
    as a \pgfname-point. The origin of the coordinate system of the
    physical page is at the \emph{lower} left corner.

\begin{codeexample}[code only]
\pgfpageslogicalpageoptions{1}
{% center logical page on middle of left side
  center=\pgfpoint{.25\pgfphysicalwidth}{.5\pgfphysicalheight}%
  resized width=.5\pgfphysicalwidth,%
  resized height=\pgfphysicalheight,%
}
\end{codeexample}

  \item
    \declare{|resized width=|\meta{size}}
    specifies the width that the logical page should have \emph{at
    most} on the physical page. To achieve this width, the pages is
    scaled down appropriately \emph{or more}. The ``or more'' part
    can happen if the |resize height| option is also used. In this
    case, the scaling is chosen such that both the specified height
    and width are met. The aspect ratio of a logical page is not
    modified.
  \item
    \declare{|resized height=|\meta{height}}
    specifies the maximum height of the logical page.
  \item
    \declare{|original width=|\meta{width}}
    specifies the width the \TeX\ ``thinks'' that the logical page
    has. This width is |\paperwidth| at the point of invocation, by
    default. Note that setting this width to something different from
    |\paperwidth| does \emph{not} change the |\pagewidth| during
    \TeX's typesetting. You have to do that yourself.

    You need this option only for special logical pages that have
    a height or width different from the normal one and for which you
    will (later on) set these sizes yourself.
  \item
    \declare{|original height=|\meta{height}}
    works like |original width|.
  \item
    \declare{|scale=|\meta{factor}}
    scales the page by at least the given \meta{factor}. A
    \meta{factor} of |0.5| will half the size of the page, a factor or
    |2| will double the size. ``At least'' means that if options like
    |resize height| are given and if the scaling required to meet that
    option is less than \meta{factor}, that other scaling is used
    instead. 
  \item
    \declare{|xscale=|\meta{factor}}
    scales the logical page along the $x$-axis by the given
    \meta{factor}. This scaling is done independently of any other
    scaling. Mostly, this option is useful for a factor of |-1|, which
    flips the page along the $y$-axis. The aspect ratio is not kept.
  \item
    \declare{|yscale=|\meta{factor}}
    works like |xscale|, only for the $y$-axis.
  \item
    \declare{|rotation=|\meta{degree}}
    rotates the page by \meta{degree} around its center. Use a degree
    of |90| or |-90| to go from portrait to landscape and back. The
    rotation need not be a multiple of |90|.
  \item
    \declare{|copy from=|\meta{logical page number}}.
    Normally, after a physical shipout has occured, all logical pages
    are voided in a loop. However, if this option is given, the
    current logical page is filled with the contents of the old
    logical page number \meta{logical page number}.

    \example Have logical page 2 retain its contents:
\begin{codeexample}[code only]
\pgfpageslogicalpageoptions{2}{copy from=2}
\end{codeexample}

    \example Let logical page 2 show what logical page 1 showed on the
    just-shipped-out physical page:
\begin{codeexample}[code only]
\pgfpageslogicalpageoptions{2}{copy from=1}
\end{codeexample}
  \item
    \declare{|border shrink|=\meta{size}}
    specifies an addition reduction of the size to which the page is
    page is scaled down.
  \item
    \declare{|border code|=\meta{code}}.
    When this option is given, the \meta{code} is executed before the
    page box is inserted with a path preinstalled that is a rectangle
    around the current logical page. Thus, setting \meta{code} to
    |\pgfstroke| draws a rectangle around the logical page. Setting
    \meta{code} to |\pgfsetlinewidth{3pt}\pgfstroke| results in a
    thick (ugly) frame. Adding dashes and filling can result in
    arbitrarily funky and distracting borders.

    You can also call |\pgfdiscardpath| and add your own path
    construction code (for example to paint a rectangle with rounded
    corners). The coordinate system is  setup in such a way that a
    rectangle starting at the origin and having the height and width
    of \TeX-box 0 will result in a rectangle filling exactly the
    logical page currently being put on the physical page. The logical
    page is inserted \emph{after} these commands have been executed.

    \example Add a rectangle around the page:
\begin{codeexample}[code only]
\pgfpageslogicalpageoptions{1}{border code=\pgfstroke}
\end{codeexample}
  \item
    \declare{|corner width|=\meta{size}}
    adds black ``rounded corners'' to the page. See the description of
    the predefined layout |rounded corners| on
    page~\pageref{layout-rounded-corners}. 
  \end{itemize}
\end{command}




\subsection{Creating Logical Pages}

Logical pages are created whenever a \TeX\ thinks that a page is full
and performs a |\shipout| command. This will cause |pgfpages| to store
the box that was supposed to be shipped out internally until enough
logical pages have been collected such that a physical shipout can
occur.

Normally, whenever a logical shipout occurs that current page is
stored in logical page number \meta{current logical page}. This
counter is then incremented, until it is larger than \meta{last
  logical shipout}. You can, however, directly change the value of
\meta{current logical page} by calling |\pgfpagesphysicalpageoptions|.

Another way to set the contents of a logical page is to use the
following command:

\begin{command}{\pgfpagesshipoutlogicalpage\marg{number}\meta{box}}
  This command sets to logical page \meta{number} to \meta{box}. The
  \meta{box} should be the code of a \TeX\ box command. This command
  does not influence the counter \meta{current logical page} and does
  not cause a physical shipout.

\begin{codeexample}[code only]
\pgfpagesshipoutlogicalpage{0}\vbox{Hi!}
\end{codeexample}

  This command can be used to set the contents of logical pages that
  are normally not filled.
\end{command}

The final way of setting a logical page is using the following
command: 

\begin{command}{\pgfpagescurrentpagewillbelogicalpage\marg{number}}
  When the current \TeX\ page has been typeset, it will be become the given
  logical page \meta{number}. This command ``interrupts'' the normal
  order of logical pages, that is, it behaves like the previous
  command and does not update the  \meta{current logical page}
  counter. 

\begin{codeexample}[code only]
\pgfpagesuselayout{two screens with optional second}
...
Text for main page.
\clearpage

\pgfpagescurrentpagewillbelogicalpage{2}
Text that goes to second page
\clearpage

Text for main page.
\end{codeexample}
\end{command}

%%% Local Variables: 
%%% mode: latex
%%% TeX-master: "pgfmanual"
%%% End: 

% Copyright 2006 by Till Tantau
%
% This file may be distributed and/or modified
%
% 1. under the LaTeX Project Public License and/or
% 2. under the GNU Free Documentation License.
%
% See the file doc/generic/pgf/licenses/LICENSE for more details.


\section{Extended Color Support}

This section documents the package \texttt{xxcolor}, which is
currently distributed as part of \pgfname. This package extends the
\texttt{xcolor} package, written by Uwe Kern, which in turn extends
the \texttt{color} package. I hope that the commands in
\texttt{xxcolor} will some day migrate to \texttt{xcolor}, such that
this package becomes superfluous.

The main aim of the \texttt{xxcolor} package is to provide an
environment inside which all colors are ``washed out'' or ``dimmed.''
This is useful in numerous situations and must typically be achieved
in a roundabout manner if such an environment is not available.

\begin{environment}{{colormixin}\marg{mix-in specification}}
  The mix-in specification is applied to all colors inside
  the environment. At the beginning of the environment, the mix-in is
  applied to the current color, i.\,e., the color that was in effect
  before the environment started. A mix-in specification is a number
  between 0 and 100 followed by an exclamation mark and a color
  name. When a |\color| command is 
  encountered inside a mix-in environment, the number states what
  percentage of the desired color should be used. The rest is
  ``filled up'' with the color given in the mix-in
  specification. Thus, a mix-in specification like |90!blue|
  will mix in 10\% of blue into everything, whereas |25!white| will
  make everything nearly white.

\begin{codeexample}[width=4cm]
\begin{minipage}{3.5cm}\raggedright
\color{red}Red text,%
\begin{colormixin}{25!white}
  washed-out red text,
  \color{blue} washed-out blue text,
  \begin{colormixin}{25!black}
    dark washed-out blue text,
    \color{green} dark washed-out green text,%
  \end{colormixin}
  back to washed-out blue text,%
\end{colormixin}
and back to red.
\end{minipage}%
\end{codeexample}
\end{environment}

Note that the environment only changes colors that have been installed
using the standard \LaTeX\ |\color| command. In particular,
the colors in images are not changed. There is, however, some support
offered by the commands |\pgfuseimage| and
|\pgfuseshading|. If the first command is invoked 
inside a |colormixin| environment with the parameter, say,
|50!black| on an image with the name |foo|, the command
will first check whether there is also a defined image with the name
|foo.!50!black|. If so, this image is used instead. This allows
you to provide a different image for this case. If you nest
|colormixin| environments, the different mix-ins are all appended. For
example, inside the inner environment of 
the above example, |\pgfuseimage{foo}| would first check whether
there exists an image named |foo.!50!white!25!black|.

\begin{command}{\colorcurrentmixin}
  Expands to the current accumulated mix-in. Each nesting of a
  |colormixin| adds a mix-in to this list.
\begin{codeexample}[]
\begin{minipage}{\linewidth-6pt}\raggedright
\begin{colormixin}{75!white}
  \colorcurrentmixin\ should be ``!75!white''\par
  \begin{colormixin}{75!black}
    \colorcurrentmixin\ should be ``!75!black!75!white''\par
    \begin{colormixin}{50!white}
      \colorcurrentmixin\ should be ``!50!white!75!black!75!white''\par
    \end{colormixin}
  \end{colormixin}
\end{colormixin}
\end{minipage}
\end{codeexample}
\end{command}







\part{Mathematical Engine}

{\Large \emph{by Mark Wibrow and Till Tantau}}


\bigskip
\noindent
\pgfname\ comes with its own mathematical engine. The job of this
engine is to support mathematical operations like addition,
subtraction, multiplication and division, using both integers and 
non-integers, but also functions such as square-roots, sine, cosine,
and generate pseudo-random numbers.

Mostly, you will use the mathematical facilities of \pgfname\
indirectly, namely when you write a coordinate like |(5cm*3,6cm/4)|,
but the mathematical engine can also be used independently of
\pgfname\ and \tikzname. 

\vskip1cm
\begin{codeexample}[graphic=white]
\pgfmathsetseed{1}
\foreach \col in {black,red,green,blue}
{
  \begin{tikzpicture}[x=10pt,y=10pt,ultra thick,baseline,line cap=round]
    \coordinate (current point) at (0,0);
    \coordinate (old velocity) at (0,0);
    \coordinate (new velocity) at (rand,rand);
    
    \foreach \i in {0,1,...,100}
    {
      \draw[\col!\i] (current point)
      .. controls ++([scale=-1]old velocity) and
                  ++(new velocity) .. ++(rand,rand)
         coordinate (current point);
      \coordinate (old velocity) at (new velocity);
      \coordinate (new velocity) at (rand,rand);
    }
  \end{tikzpicture}
}
\end{codeexample}

% Copyright 2007 by Mark Wibrow and Till Tantau
%
% This file may be distributed and/or modified
%
% 1. under the LaTeX Project Public License and/or
% 2. under the GNU Free Documentation License.
%
% See the file doc/generic/pgf/licenses/LICENSE for more details.



\section{Design Principles}

\pgfname{} needs to perform many computations while typesetting a
picture. For this, \pgfname\ relies on a mathematical engine, which
can also be used independently of \pgfname, but which is distributed
as part of the \pgfname\ package nevertheless. Basically, the engine
provides a parsing mechanism similar to the \calcname{} package so
that expressions like |2*3cm+5cm| can be parsed; but the \pgfname\
engine is more powerful and can be extended and enhanced. 

\pgfname{} provides enhanced functionality, which permits the parsing
of mathematical operations involving integers and non-integers 
with or without units. Furthermore, various functions, including
trigonometric functions and random number generators can also be 
parsed (see Section~\ref{pgfmath-parsing}). 
The \calcname{} macros |\setlength| and friends have \pgfname{} versions 
which can parse these operations and functions 
(see Section~\ref{pgfmath-registers}). Additionally, each operation
and function has an independent \pgfname{} command associated with it
(see Section~\ref{pgfmath-commands}), and can be 
accessed outside the parser.

The mathematical engine of \pgfname\ is implicitly used whenever you
specify a number or dimension in a higher-level macro. For instance,
you can write |\pgfpoint{2cm+4cm/2}{3cm*sin(30)}| or
suchlike. However, the mathematical engine can also be used
independently of the \pgfname\ core, that is, you can also just load
it to get access to a mathematical parser.


\subsection{Loading the Mathematical Engine}

The mathematical engine of \pgfname\ is loaded automatically by
\pgfname, but if you wish to use the mathematical engine but you do
not need \pgfname\ itself, you can load the following package:

\begin{package}{pgfmath}
  This command will load the mathematical engine of \pgfname, but not
  \pgfname\ itself. It defines commands like |\pgfmathparse|.
\end{package}


\subsection{Layers of the Mathematical Engine}

Like \pgfname\ itself, the mathematical engine is also structured into
different layers:

\begin{enumerate}
\item The top layer, which you will typically use directly, provides
  the command |\pgfmathparse|. This command parses a mathematical
  expression and evaluates it.

  Additionally, the top layer also defines some additional functions
  similar to the macros of the |calc| package for setting dimensions
  and counters. These macros are just wrappers around the
  |\pgfmathparse| macro.
\item The calculation layer provides macros for performing one
  specific computation like computing a reciprocal or a
  multiplication. The parser uses these macros for the actual
  computation.
\item The implementation layer provides the actual implementations of
  the computations. These can be changed (and possibly be made more
  efficient) without affecting the higher layers.
\end{enumerate}



\subsection{Efficiency and Accuracy of the Mathematical Engine}

Currently, the mathematical algorithms are all implemented in \TeX.
This poses some intriguing programming challenges as \TeX{} is a
typesetting language not a mathematical one, and as with any 
programming language, there is a trade-off between accuracy and 
efficiency. If you find the level of accuracy insufficient for you
purposes, you will have to replace the algorithms in the
implementation layer.

All the fancy mathematical ``bells-and-whistles'' that the parser 
provides, come with an additional processing cost, and in some
instances, such as simply setting a length to |1cm|, with no other
operations involved, the addition processing time is undesirable. 
To overcome this, the following feature is implemented: when no
mathematical operations are required, the value in \meta{expression}
can be preceded by |+|. This will bypass the parsing process and the 
assignment will be orders of magnitude faster. This feature 
\emph{only} works with the macros for setting registers described in
Section~\ref{pgfmath-registers}.

\begin{codeexample}[code only]
\pgfmathsetlength\mydimen{1cm}  % parsed     : slower.
\pgfmathsetlength\mydimen{+1cm} % not parsed : much faster.
\end{codeexample}


% Copyright 2007 by Mark Wibrow
%
% This file may be distributed and/or modified
%
% 1. under the LaTeX Project Public License and/or
% 2. under the GNU Free Documentation License.
%
% See the file doc/generic/pgf/licenses/LICENSE for more details.
%

\section{Evaluating Mathematical Expressions}

The easiest way of using \pgfname's mathematical engine is to provide
a mathematical expression given in the usual infix notation (such as
|1cm+4*2cm/5.5| or |2*3+3*sin(30)|). This expression can be parsed by
the mathematical engine and the result be placed in a dimension
register, a counter, or a macro. Supported are infix mathematical
operations involving integers and non-integers, with or without
units.

It should be noted that all 
calculations must not exceed $\pm16383.99999$ at \emph{any} point, 
because the underlying algorithms relie on \TeX{} dimensions. This
means that many of the underlying algorithms are necessarily
approximate. It also means that some of the algorithms are not very
fast. \TeX{} is, after all, a typesetting language and not ideally
suited to relatively advanced mathematical operations. However, it is
possible to change the algorithms as described in
Section~\ref{pgfmath-reimplement}. 

In the present section, the high-level macros for parsing an
expression are explained first, then the syntax for expression is
explained.


\subsection{Commands for Parsing Expressions}

\label{pgfmath-registers}

\label{pgfmath-parsing}

The basic command for invoking the parser of \pgfname's mathematical
engine is the following:

\begin{command}{\pgfmathparse\marg{expression}}
  This macro parses \meta{expression} and returns the result without
  units in  the macro |\pgfmathresult|.

  \example |\pgfmathparse{2pt+3.5pt}| will set |\pgfmathresult| to the
  text |5.5|.

  In the following, the special properties of this command are
  explained. The exact syntax of mathematical expressions is explained
  in Section~\ref{pgfmath-syntax}. Note that unlike the rest of the  
  manual, the examples show the result of the calculation (that is,
  the value of the macro |\pgfmathresult|), not what is displayed in
  the document.  

  \begin{itemize}
  \item
    The result stored in the macro |\pgfmathresult| is a decimal
    \emph{without units}. This is true regardless of whether the
    \meta{expression} contains any unit specification. But, any units
    specified will be converted to points first.
\begin{codeexample}[post=\tt\footnotesize\pgfmathresult]
\pgfmathparse{2pt+3.4pt}
\end{codeexample}

\begin{codeexample}[post=\tt\footnotesize\pgfmathresult]
\pgfmathparse{2cm+3.4cm}
\end{codeexample}

  \item If no units are specified \emph{at any point} in the 
    expression, the result will be multiplied by the value in 
    |\pgfmathresultunitscale|, which can be a number or a dimension 
    (which will be converted to points). By default it is set to 1, 
    but can be changed with |\pgfmathsetresultunitscale|. Note that 
    the result will still be a number \emph{without} units.

\begin{codeexample}[post=\tt\footnotesize\pgfmathresult]
\pgfmathparse{2pt+3.4pt}
\end{codeexample}

\begin{codeexample}[post=\tt\footnotesize\pgfmathresult]
\pgfmathsetresultunitscale{1cm}
\pgfmathparse{2+3.4}
\end{codeexample}

    \pgfmathsetresultunitscale{1pt}
    
  \item You can check whether an expression contained a unit using 
    the \TeX-if |\||ifpgfmathunitsdeclared|. After a call
    of |\pgfmathparse| this if will be true exactly if some unit was
    encountered in the expression.
    
  \item The parser handles numbers with or without units regardless
    of the operation.

\begin{codeexample}[post=\tt\footnotesize\pgfmathresult]
\pgfmathparse{54pt/3cm*2.1} 
\end{codeexample}

  \item the parser can cope with \TeX{} registers, including those 
    preceded by |\the|.

    \makeatletter

\begin{codeexample}[post=\tt\footnotesize\pgfmathresult]
\pgf@x=12.34pt
\c@pgf@counta=5
\pgfmathparse{\pgf@x+\c@pgf@counta*6}
\end{codeexample}

\begin{codeexample}[post=\tt\footnotesize\pgfmathresult]
\pgf@x=56.78pt
\pgfmathparse{\pgf@x+\the\pgf@x}
\end{codeexample}

  \item \TeX{} dimension registers can be multiplied without the |*| 
    operator by preceding them with a number (\emph{not} a function),
     or a count register.
	 
\begin{codeexample}[post=\tt\footnotesize\pgfmathresult]
\c@pgf@counta=-4
\pgf@x=10pt
\pgfmathparse{.5\pgf@x-\c@pgf@counta\pgf@x}%
\end{codeexample}

  \item Parenthesis can be used to group operations.

\begin{codeexample}[post=\tt\footnotesize\pgfmathresult]
\pgfmathparse{(4pt+0.5)*3}
\end{codeexample}

  \item functions are recognized, so it is possible to parse
    |sin(.5*pi r)*60|, which means ``the sine of $0.5$ times $\pi$ 
    radians, multiplied by 60''. The argument of most functions can
    be any expression.

\begin{codeexample}[post=\tt\footnotesize\pgfmathresult]
\pgfmathparse{sin(pi/2 r)*60}
\end{codeexample}

  \item Scientific notation in the form |1.234e+4| is recognised (but
  the restriction on the range of values still applies). The exponent
  symbol can be upper or lower case (i.e., |E| or |e|). 
  
\begin{codeexample}[post=\tt\footnotesize\pgfmathresult]
\pgfmathparse{1.234567891e-2}
\end{codeexample}
\begin{codeexample}[post=\tt\footnotesize\pgfmathresult]
\pgfmathparse{1.234567891e4}
\end{codeexample}
  \end{itemize}
\end{command}

\begin{command}{\pgfmathqparse\marg{expression}}
  This macro is similar to |\pgfmathparse|: it parses 
  \meta{expression} and returns the result in the macro 
  |\pgfmathresult|. It differs in two respects. Firstly, 
  |\pgfmathqparse| does not parse functions or scientific
  notation. 
  Secondly, numbers in \meta{expression} \emph{must}
  specify a \TeX{} unit (except in such instances as |0.5\pgf@x|), 
  which greatly simplifies the problem of parsing 
  of non-integers. As a result of these restrictions |\pgfmathqparse| 
  is about twice as fast as |\pgfmathparse|. Note that the result 
  will still be a number \emph{without} units.	
\end{command}

\begin{command}{\pgfmathsetresultunitscale\marg{number or dimension}}
  Sets the value in |\pgfmathresultunitscale|, which scales the result
  of an expression parsed with |\pgfmathparse|, if that expression
  contains no units \emph{at any point}. The argument can be an integer,
  non-integer or a dimension, but the result will still be a number 
  \emph{without} units. Note, that this will affect |\pgfmathsetlength| 
  and friends, but not if the expression starts with |+| (which
  switches parsing off). By default the value in
  |\pgfmathresultunitscale| is 1. 
\end{command}

Instead of the |\pgfmathparse| macro you can also wrapper commands,
whose usage is very similar to their cousins in the \calcname{} 
package. The only difference is that the expressions can be any
expression that is handled by |\pgfmathparse|.

For all of the following commands, if \meta{expression} starts with
|+|, no parsing is done and a simple assignment or increment is done
using normal \TeX\ assignments or increments. This will be orders of
magnitude faster than calling the parser. 

\begin{command}{\pgfmathsetlength\marg{dimension register}\marg{expression}}
  Sets the length of the \TeX{} \meta{dimension register}, to the value
  (in points) specified by \meta{expression}. The \meta{expression}
  will be parsed using |\pgfmathparse|.
\end{command}

\begin{command}{\pgfmathaddtolength\marg{dimension register}\marg{expression}}
  Adds the value (in points) of \meta{expression} to the \TeX{} 
  \meta{dimension register}. 
\end{command}

\begin{command}{\pgfmathsetcount\marg{count register}\marg{expression}}
  Sets the value of the \TeX{} \meta{count register}, to the 
  \emph{truncated} value specified by \meta{expression}. 
\end{command}

\begin{command}{\pgfmathaddtocount\marg{count register}\marg{expression}}
  Adds the \emph{truncated} value  of \meta{expression} to the \TeX{} 
  \meta{count register}.
\end{command}

\begin{command}{\pgfmathsetcounter\marg{counter}\marg{expression}}
  Sets the value of the \meta{counter}, to the \emph{truncated} value 
  specified by \meta{expression}. 
\end{command}

\begin{command}{\pgfmathaddtocounter\marg{counter}\marg{expression}}
  Adds the \emph{truncated} value  of \meta{expression} to 
  \meta{counter}.
\end{command}

% \begin{command}{\pgfmathnewcounter\marg{counter}}
%   This is simply a version of the \LaTeX{} macro |\newcounter|, 
%   implemented to maintain consistency (consistency is good,
%   inconsistency is evil). Considering |\pgfmathnewcounter{foo}|, this
%   creates a new count register |\c@foo|, and a macro |\thefoo|, which
%   returns the value in |\c@foo|.
% \end{command}

\begin{command}{\pgfmathsetmacro\marg{macro}\marg{expression}}
  Defines \meta{macro} as the  value of \meta{expression}. The result
  is a decimal \emph{without} units.
\end{command}



\subsection{Syntax for mathematical expressions}

\label{pgfmath-syntax}

The syntax for the expressions recognized by |\pgfmathparse| and
friends is straightfoward, and the following operations and 
functions are currently recognized:

\begin{math-operator}{\mvar{x}\ +\ \mvar{y}}
	Adds \mvar{y} to \mvar{x}.
	
\begin{codeexample}[post=\tt\footnotesize\pgfmathresult]
\pgfmathparse{4+2pt}
\end{codeexample}
\end{math-operator}

\begin{math-operator}{\mvar{x}\ -\ \mvar{y}}
	Subtracts \mvar{y} from  \mvar{x}.
	
\begin{codeexample}[post=\tt\footnotesize\pgfmathresult]
\pgfmathparse{155.35-4cm}
\end{codeexample}
\end{math-operator}
\begin{math-operator}{\mvar{x}\ *\ \mvar{y}}
	Multiplies \mvar{x} by  \mvar{y}.
	
\begin{codeexample}[post=\tt\footnotesize\pgfmathresult]
\pgfmathparse{3.9pt*4.56}
\end{codeexample}

\end{math-operator}
\begin{math-operator}{\mvar{x}\ /\ \mvar{y}}
	Divides \mvar{x} by  \mvar{y}.
	
\begin{codeexample}[post=\tt\footnotesize\pgfmathresult]
\pgfmathparse{-31.6pt/17}
\end{codeexample}

\end{math-operator}
\begin{math-operator}{\mvar{x}\ {\char94}\ \mvar{y}} 

Raises \mvar{x} to the power \mvar{y}. \mvar{y} should be an integer, but it can be negative.

{
\catcode`\^=7

\begin{codeexample}[post=\tt\footnotesize\pgfmathresult]
\pgfmathparse{2.3^4}
\end{codeexample}

\begin{codeexample}[post=\tt\footnotesize\pgfmathresult]
\pgfmathparse{2^-4}
\end{codeexample}
}
\end{math-operator}

\begin{math-operator}{\mvar{x}\ ==\ \mvar{y}} 

	This evaluates to |1| if \mvar{x} equals \mvar{y}, or |0| if \mvar{x}
	does not equal \mvar{y}. 
	Note that equalities (and inequalities) are evaluated left to right, 
	and are only evaluated when another equality (or inequality) operator 
	is scanned, or the end of the current group or parse is reached. So 
	|5+4==3+2==9| results in |0| because |5+4| does not equal |3+2|, 
	resulting in zero, and the second equality is therefore evaluating 
	|0==9|.

\begin{codeexample}[post=\tt\footnotesize\pgfmathresult]
\pgfmathparse{3*5 == 15}
\end{codeexample}

\end{math-operator}


\begin{math-operator}{\mvar{x}\ >\ \mvar{y}} 

	This evaluates to |1| if \mvar{x} is greater than \mvar{y}, or |0| if 
	\mvar{x} is smaller or equal to \mvar{y}.
	
\begin{codeexample}[post=\tt\footnotesize\pgfmathresult]
\pgfmathparse{17 > 4.2*1.97+4}
\end{codeexample}

\end{math-operator}

\begin{math-operator}{\mvar{x}\ <\ \mvar{y}}

	This evaluates to |1| if \mvar{x} is smaller than \mvar{y}, or |0| if
	\mvar{x} is greater or equal to \mvar{y}.
	
\begin{codeexample}[post=\tt\footnotesize\pgfmathresult]
\pgfmathparse{2 < -5.2/-3.6-2}
\end{codeexample}

\end{math-operator}

\begin{math-function}{mod(\mvar{x},\mvar{y})}
	This evaluates \mvar{x} modulo \mvar{y}. This function cannot be 
	nested inside itself or the functions |max|, |min| or |veclen|.

\begin{codeexample}[post=\tt\footnotesize\pgfmathresult]
\pgfmathparse{mod(20,6)}
\end{codeexample}

\end{math-function}

\begin{math-function}{max(\mvar{x},\mvar{y})}
	This evaluates to the maximum of \mvar{x} or \mvar{y}. This function 
	cannot be nested inside itself or the functions |min|, |mod| or 
	|pow|.

\begin{codeexample}[post=\tt\footnotesize\pgfmathresult]
\pgfmathparse{max(17,23)}
\end{codeexample}

\end{math-function}

\begin{math-function}{min(\mvar{x},\mvar{y})}
	This evaluates to the minimum of \mvar{x} or \mvar{y}. This function 
	cannot be nested inside itself or the functions |max|, |mod| or 
	|pow|.

\begin{codeexample}[post=\tt\footnotesize\pgfmathresult]
\pgfmathparse{min(17,23)}
\end{codeexample}

\end{math-function}

\begin{math-function}{abs(\mvar{x})}

	Evaluates the absolute value of $x$.
	
\begin{codeexample}[post=\tt\footnotesize\pgfmathresult]
\pgfmathparse{abs(-5)}
\end{codeexample}

\begin{codeexample}[post=\tt\footnotesize\pgfmathresult]
\pgfmathparse{-abs(4*-3)}
\end{codeexample}

\end{math-function}

\begin{math-function}{round(\mvar{x})}

	Rounds \mvar{x} to the nearest integer. It uses ``asymmetric half-up'' 
	rounding. So |1.5| is rounded to |2|, but |-1.5| is rounded to |-2| 
	(\emph{not} |0|).

\begin{codeexample}[post=\tt\footnotesize\pgfmathresult]
\pgfmathparse{round(32.5/17)}
\end{codeexample}

\begin{codeexample}[post=\tt\footnotesize\pgfmathresult]
\pgfmathparse{round(398/12)}
\end{codeexample}

\end{math-function}

\begin{math-function}{floor(\mvar{x})}

	Rounds \mvar{x} down to the nearest integer. 
	
\begin{codeexample}[post=\tt\footnotesize\pgfmathresult]
\pgfmathparse{floor(32.5/17)}
\end{codeexample}

\begin{codeexample}[post=\tt\footnotesize\pgfmathresult]
\pgfmathparse{floor(398/12)}
\end{codeexample}

\end{math-function}

\begin{math-function}{ceil(\mvar{x})}

	Rounds \mvar{x} up to the nearest integer. 

\begin{codeexample}[post=\tt\footnotesize\pgfmathresult]
\pgfmathparse{ceil(32.5/17)}
\end{codeexample}

\begin{codeexample}[post=\tt\footnotesize\pgfmathresult]
\pgfmathparse{ceil(398/12)}
\end{codeexample}

\end{math-function}

\begin{math-function}{exp(\mvar{x})}
{
\catcode`\^=7

	Maclaurin series for $e^\textrm{\mvar{x}}$. 
}	
\begin{codeexample}[post=\tt\footnotesize\pgfmathresult]
\pgfmathparse{exp(1)}
\end{codeexample}

\begin{codeexample}[post=\tt\footnotesize\pgfmathresult]
\pgfmathparse{exp(2.34)}
\end{codeexample}

\end{math-function}


\begin{math-function}{ln(\mvar{x})}
{
\catcode`\^=7

	An approximation for for $\ln(\textrm{\mvar{x}})$. 
}	
\begin{codeexample}[post=\tt\footnotesize\pgfmathresult]
\pgfmathparse{ln(10)}
\end{codeexample}

\begin{codeexample}[post=\tt\footnotesize\pgfmathresult]
\pgfmathparse{ln(exp(5))}
\end{codeexample}

\end{math-function}

\begin{math-function}{pow(\mvar{x},\mvar{y})}

 Raises \mvar{x} to the power \mvar{y}. \mvar{y} should be an integer, 
 but it can be negative. 

\begin{codeexample}[post=\tt\footnotesize\pgfmathresult]
\pgfmathparse{pow(2,7)}
\end{codeexample}

\end{math-function}

\begin{math-function}{sqrt(\mvar{x})}

 Calculates $\sqrt{\textrm{\mvar{x}}}$.

\begin{codeexample}[post=\tt\footnotesize\pgfmathresult]
\pgfmathparse{sqrt(10)}
\end{codeexample}

\begin{codeexample}[post=\tt\footnotesize\pgfmathresult]
\pgfmathparse{sqrt(8765.432)}
\end{codeexample}

\end{math-function}

\begin{math-constant}{pi}

	The constant $\pi=3.14159$.
	
\begin{codeexample}[post=\tt\footnotesize\pgfmathresult]
\pgfmathparse{pi}
\end{codeexample}

\begin{codeexample}[post=\tt\footnotesize\pgfmathresult]
\pgfmathparse{pi r}
\end{codeexample}

\end{math-constant}

\begin{math-operator}{\mvar{x}\ r}

	This converts \mvar{x} from radians to degrees. Note that |r| will 
	evaluate any preceding series of multiplication or division 
	\emph{before} conversion, but not other operations. So |3*4/6r| 
	converts 2 radians to degrees, but |3-4+6r|, converts 6 radians to
	degrees and adds the result to |-1|.

\begin{codeexample}[post=\tt\footnotesize\pgfmathresult]
\pgfmathparse{2*pi r-pi r}
\end{codeexample}

\begin{codeexample}[post=\tt\footnotesize\pgfmathresult]
\pgfmathparse{2*pi/8 r}
\end{codeexample}

\begin{codeexample}[post=\tt\footnotesize\pgfmathresult]
\pgfmathparse{sin(3*pi/2r)*60}
\end{codeexample}

\end{math-operator}

\begin{math-function}{rad(\mvar{x})}

	Convert \mvar{x} to radians. \mvar{x} is assumed to be in degrees.
	
\begin{codeexample}[post=\tt\footnotesize\pgfmathresult]
\pgfmathparse{rad(90)}
\end{codeexample}

\end{math-function}

\begin{math-function}{deg(\mvar{x})}

	Convert \mvar{x} to degrees. \mvar{x} is assumed to be in radians.
	
\begin{codeexample}[post=\tt\footnotesize\pgfmathresult]
\pgfmathparse{deg(3*pi/2)}
\end{codeexample}

\end{math-function}

\begin{math-function}{sin(\mvar{x})}

	Sine of \mvar{x}. By employing the |r| operator, \mvar{x} can be in 
	radians.
	
\begin{codeexample}[post=\tt\footnotesize\pgfmathresult]
\pgfmathparse{sin(60)}
\end{codeexample}

\begin{codeexample}[post=\tt\footnotesize\pgfmathresult]
\pgfmathparse{sin(pi/3 r)}
\end{codeexample}

\end{math-function}

\begin{math-function}{cos(\mvar{x})}

	Cosine of \mvar{x}. By employing the |r| operator, \mvar{x} can be in 
	radians.

\begin{codeexample}[post=\tt\footnotesize\pgfmathresult]
\pgfmathparse{cos(60)}
\end{codeexample}

\begin{codeexample}[post=\tt\footnotesize\pgfmathresult]
\pgfmathparse{cos(pi/3 r)}
\end{codeexample}

\end{math-function}

\begin{math-function}{tan(\mvar{x})}

	Tangent of \mvar{x}. By employing the |r| operator, \mvar{x} can be in 
	radians.
	
\begin{codeexample}[post=\tt\footnotesize\pgfmathresult]
\pgfmathparse{tan(45)}
\end{codeexample}

\begin{codeexample}[post=\tt\footnotesize\pgfmathresult]
\pgfmathparse{tan(2*pi/8 r)}
\end{codeexample}

\end{math-function}


\begin{math-function}{sec(\mvar{x})}

	Secant of \mvar{x}. By employing the |r| operator, \mvar{x} can be in 
	radians.

\begin{codeexample}[post=\tt\footnotesize\pgfmathresult]
\pgfmathparse{sec(45)}
\end{codeexample}

\end{math-function}

\begin{math-function}{cosec(\mvar{x})}

	Cosecant of \mvar{x}. By employing the |r| operator, \mvar{x} can be in 
	radians.
	
\begin{codeexample}[post=\tt\footnotesize\pgfmathresult]
\pgfmathparse{cosec(30)}
\end{codeexample}

\end{math-function}

\begin{math-function}{cot(\mvar{x})}

	Cotangent of \mvar{x}. By employing the |r| operator, \mvar{x} can be in 
	radians.
	
\begin{codeexample}[post=\tt\footnotesize\pgfmathresult]
\pgfmathparse{cot(15)}
\end{codeexample}

\end{math-function}

\begin{math-function}{asin(\mvar{x})}

	Arcsine of \mvar{x}. The result is in degrees.

\begin{codeexample}[post=\tt\footnotesize\pgfmathresult]
\pgfmathparse{asin(0.7071)}
\end{codeexample}

\end{math-function}

\begin{math-function}{acos(\mvar{x})}

	Arccosine of \mvar{x} in degrees.  

\begin{codeexample}[post=\tt\footnotesize\pgfmathresult]
\pgfmathparse{acos(0.5)}
\end{codeexample}

\end{math-function}

\begin{math-function}{atan(\mvar{x})}

	Arctangent of $x$ in degrees. 

\begin{codeexample}[post=\tt\footnotesize\pgfmathresult]
\pgfmathparse{atan(1)}
\end{codeexample}

\end{math-function}

\begin{math-function}{rnd}

	Generates a pseudo-random number between 0 and 1.

\begin{codeexample}[post=\tt\footnotesize\pgfmathresult]
\pgfmathparse{rnd}
\end{codeexample}

\begin{codeexample}[post=\tt\footnotesize\pgfmathresult]
\pgfmathparse{2*rnd}
\end{codeexample}

\begin{codeexample}[post=\tt\footnotesize\pgfmathresult]
\pgfmathparse{-rnd+5}
\end{codeexample}

\end{math-function}

\begin{math-function}{rand}

	Generates a pseudo-random number between -1 and 1.

\begin{codeexample}[post=\tt\footnotesize\pgfmathresult]
\pgfmathparse{rand}
\end{codeexample}

\begin{codeexample}[post=\tt\footnotesize\pgfmathresult]
\pgfmathparse{rand*15}
\end{codeexample}

\end{math-function}

% Copyright 2007 by Mark Wibrow
%
% This file may be distributed and/or modified
%
% 1. under the LaTeX Project Public License and/or
% 2. under the GNU Free Documentation License.
%
% See the file doc/generic/pgf/licenses/LICENSE for more details.

\section{Evaluating Mathematical Operations}

\label{pgfmath-commands}

Instead of parsing and evaluating complex expressions, you can also
use the mathematical engine to evaluate a single mathematical
operation. The macros used for these computations are described in the
following. 


\subsection{Basic Operations and Functions}

\label{pgfmath-operations}

\begin{command}{\pgfmathadd\marg{x}\marg{y}}  
	Defines |\pgfmathresult| as $\meta{x}+\meta{y}$.
\end{command}

\begin{command}{\pgfmathsubtract\marg{x}\marg{y}}      
	Defines |\pgfmathresult| as $\meta{x}-\meta{y}$.                                       
\end{command}

\begin{command}{\pgfmathmultiply\marg{x}\marg{y}}      
	Defines |\pgfmathresult| as $\meta{x}\times\meta{y}$.                                
\end{command}

\begin{command}{\pgfmathdivide\marg{x}\marg{y}}        
	Defines |\pgfmathresult| as $\meta{x}\div\meta{y}$. An error will
	result if \meta{y} is	|0|, or if the result of the division is
	too big for the mathematical engine.
	Please remember	when using this command that accurate (and reasonably 
	quick) division of non-integers is particularly tricky in \TeX{}. 	
	There are three different forms of division used in this command:
	\begin{itemize}
		\item 
		If \meta{y} is an integer then the native |\divide| operation of 
		\TeX{} is used.
		\item
		If \vrule\meta{y}\vrule$<1$, then |\pgfmathreciprocal| is employed.
		\item
		For all other values of \meta{y} an optimised long division 
		algorithm is used. In theory this should be accurate
		to any finite precision, but in practice it is constrained by the
		limits of \TeX{}'s native mathematics.
	\end{itemize}
	                             
\end{command}

\begin{command}{\pgfmathreciprocal\marg{x}}         
	Defines |\pgfmathresult| as $1\div\meta{x}$.                            
\end{command}

\begin{command}{\pgfmathgreaterthan\marg{x}\marg{y}}   
	Defines |\pgfmathresult| as 1.0 if \meta{x} $>$ \meta{y}, but 0.0 otherwise.                 
\end{command}

\begin{command}{\pgfmathlessthan\marg{x}\marg{y}} 
	Defines |\pgfmathresult| as 1.0 if \meta{x} $<$ \meta{y}, but 0.0 otherwise.             
\end{command}
	
\begin{command}{\pgfmathequalto\marg{x}\marg{y}}       
	Defines |\pgfmathresult| 1.0 if \meta{x} $=$ \meta{y}, but 0.0 otherwise.                    
\end{command}

\begin{command}{\pgfmathround\marg{x}}              
	Defines |\pgfmathresult| as $\left\lfloor\textrm{\meta{x}}\right\rceil$.	
	This uses asymmetric	half-up rounding.                          
\end{command}

\begin{command}{\pgfmathfloor\marg{x}}              
	Defines |\pgfmathresult| as $\left\lfloor\textrm{\meta{x}}\right\rfloor$.
\end{command}

\begin{command}{\pgfmathceil\marg{x}}               
	Defines |\pgfmathresult| as $\left\lceil\textrm{\meta{x}}\right\rceil$.                           
\end{command}
	
\begin{command}{\pgfmathpow\marg{x}\marg{y}}         
	Defines |\pgfmathresult| as $\meta{x}^{\meta{y}}$.  For greatest 
	accuracy \mvar{y} should be an integer. If \mvar{y} is not an integer 
	the actual calculation will be an approximation of $e^{y\ln(x)}$.
\end{command}

\begin{command}{\pgfmathmod\marg{x}\marg{y}}           
	Defines |\pgfmathresult| as \meta{x} modulo \meta{y}.                       
\end{command}

\begin{command}{\pgfmathmax\marg{x}\marg{y}}           
	Defines |\pgfmathresult| as the maximum of \meta{x} or \meta{y}.                       
\end{command}

\begin{command}{\pgfmathmin\marg{x}\marg{y}}           
	Defines |\pgfmathresult| as the minimum \meta{x} or \meta{y}.                       
\end{command}
	
\begin{command}{\pgfmathabs\marg{x}}                
	Defines |\pgfmathresult| as  absolute value of \meta{x}.                                 
\end{command}
	
\begin{command}{\pgfmathexp\marg{x}}                
	Defines |\pgfmathresult| as $e^{\meta{x}}$. Here, \meta{x} can be a 
	non-integer. The algorithm	uses a Maclaurin series.               
\end{command}

\begin{command}{\pgfmathln\marg{x}}                
	Defines |\pgfmathresult| as $\ln{\meta{x}}$. This uses an algorithm
	due to Rouben Rostamian, and coefficients suggested by
	Alain Matthes.             
\end{command}
	
\begin{command}{\pgfmathsqrt\marg{x}} 
	Defines |\pgfmathresult| as $\sqrt{\meta{x}}$. 
\end{command}
	
\begin{command}{\pgfmathveclen\marg{x}\marg{y}}        
	Defines |\pgfmathresult| as $\sqrt{\meta{x}^2+\meta{y}^2}$. This uses
	a polynomial approximation, based on ideas due to Rouben Rostamian.                                    
\end{command}

\subsection{Trignometric Functions}

\label{pgfmath-trigonmetry}

\begin{command}{\pgfmathpi}
  	Defines |\pgfmathresult| as $3.14159$.
\end{command}
   
\begin{command}{\pgfmathdeg{\marg{x}}} 
	Defines |\pgfmathresult| as \meta{x} (given in radians) converted to 
	degrees. 
\end{command}

\begin{command}{\pgfmathrad{\marg{x}}} 
	Defines |\pgfmathresult| as \meta{x} (given in degrees) converted to 
	radians. 
\end{command}

\begin{command}{\pgfmathsin{\marg{x}}}  
	Defines |\pgfmathresult| as the sine of \meta{x}.  
\end{command}

\begin{command}{\pgfmathcos{\marg{x}}}
	Defines |\pgfmathresult| as the cosine of \meta{x}.
\end{command}

\begin{command}{\pgfmathtan{\marg{x}}}  
	Defines |\pgfmathresult| as the tangant of \meta{x}.  
\end{command}

\begin{command}{\pgfmathsec{\marg{x}}}
	Defines |\pgfmathresult| as the secant of \meta{x}.
\end{command}

\begin{command}{\pgfmathcosec{\marg{x}}}  
	Defines |\pgfmathresult| as the cosecant of \meta{x}.  
\end{command}

\begin{command}{\pgfmathcot{\marg{x}}}  
	Defines |\pgfmathresult| as the cotangant of \meta{x}.  
\end{command}

\begin{command}{\pgfmathasin{\marg{x}}}
	the arcsine of \meta{x}.
\end{command}

\begin{command}{\pgfmathacos{\marg{x}}}
	Defines |\pgfmathresult| as the arccosine of \meta{x}.
\end{command}

\begin{command}{\pgfmathatan{\marg{x}}}
 	Defines |\pgfmathresult| as the arctangent of \meta{x}.
\end{command}



\subsection{Pseudo-Random Numbers}

\label{pgfmath-random}


\begin{command}{\pgfmathgeneratepseudorandomnumber}
	Defines |\pgfmathresult| as a pseudo-random integer between 1 and 
	$2^{31}-1$. This uses a linear congruency generator, based on ideas
	due to Erich Janka.
\end{command}

\begin{command}{\pgfmathrnd}
	Defines |\pgfmathresult| as a pseudo-random number between 0.0 and 1.0
\end{command}

\begin{command}{\pgfmathrand}
	Defines |\pgfmathresult| as a pseudo-random number between -1.0 and 1.0
\end{command}

\begin{command}{\pgfmathrandominteger\marg{macro}\marg{maximum}\marg{minimum}}
	This defines \meta{macro} as a pseudo-randomly generated integer from 
	the range \meta{maximum} to \meta{minimum} (inclusive).
	
\begin{codeexample}[]
\begin{pgfpicture}
   \foreach \x in {1,...,50}{
      \pgfmathrandominteger{\a}{1}{50}
      \pgfmathrandominteger{\b}{1}{50}
      \pgfpathcircle{\pgfpoint{+\a pt}{+\b pt}}{+2pt}
      \color{blue!40!white}
      \pgfsetstrokecolor{blue!80!black}
      \pgfusepath{stroke, fill}
   }	  
\end{pgfpicture}
\end{codeexample}
\end{command}

\begin{command}{\pgfmathdeclarerandomlist\marg{list name}\{\marg{item-1}\marg{item 2}...\}}
	This creates a list of items with the name \meta{list name}.
\end{command}

\begin{command}{\pgfmathrandomitem\marg{macro}\marg{list name}}
	Select an item from a random list \meta{list name}. The
	selected item is placed in \meta{macro}.
\end{command}

\begin{codeexample}[]
\begin{pgfpicture}
   \pgfmathdeclarerandomlist{color}{{red}{blue}{green}{yellow}{white}}
   \foreach \a in {1,...,50}{
      \pgfmathrandominteger{\x}{1}{85}
      \pgfmathrandominteger{\y}{1}{85}
      \pgfmathrandominteger{\r}{5}{10}
      \pgfmathrandomitem{\c}{color}
      \pgfpathcircle{\pgfpoint{+\x pt}{+\y pt}}{+\r pt}
      \color{\c!40!white}
      \pgfsetstrokecolor{\c!80!black}
      \pgfusepath{stroke, fill}
   }	  
\end{pgfpicture}
\end{codeexample}

\begin{command}{\pgfmathsetseed\marg{integer}}
  Explicitly set seed for the pseudo-random number generator. By
  default it is set to the value of |\time|$\times$|\year|.
\end{command}


      
\subsection{Conversion Between Bases}
	
\label{pgfmath-bases}

\pgfname{} provides limited support for conversion between 
\emph{representations} of numbers. Currently the numbers must be
positive integers in the range $0$ to $2^{31}-1$, and the bases in the
range $2$ to $36$. All digits representing numbers greater than 9 (in
base ten), are alphabetic, but may be upper or lower case. 

\begin{command}{\pgfmathbasetodec\marg{macro}\marg{number}\marg{base}}
	Defines \meta{macro} as the result of converting \meta{number} from
	base \meta{base} to base 10. Alphabetic digits can be upper or lower
	case.

\medskip{\def\medskip{}

\begin{codeexample}[]
\pgfmathbasetodec\mynumber{107f}{16} \mynumber
\end{codeexample}


\begin{codeexample}[]
\pgfmathbasetodec\mynumber{33FC}{20} \mynumber
\end{codeexample}

}\medskip

\end{command}

\begin{command}{\pgfmathdectobase\marg{macro}\marg{number}\marg{base}}
	Defines \meta{macro} as the result of converting \meta{number} from
	base 10 to base \meta{base}. Any resulting alphabetic digits are in
	\emph{lower case}.
	
\begin{codeexample}[]
\pgfmathdectobase\mynumber{65535}{16} \mynumber
\end{codeexample}

\end{command}

\begin{command}{\pgfmathdectoBase\marg{macro}\marg{number}\marg{base}}
	Defines \meta{macro} as the result of converting \meta{number} from
	base 10 to base \meta{base}. Any resulting alphabetic digits are in
	\emph{upper case}.
	
\begin{codeexample}[]
\pgfmathdectoBase\mynumber{65535}{16} \mynumber
\end{codeexample}

\end{command}

\begin{command}{\pgfmathbasetobase\marg{macro}\marg{number}\marg{base-1}\marg{base-2}}
	Defines \meta{macro} as the result of converting \meta{number} from
	base \meta{base-1} to base \meta{base-2}. Alphabetic digits in 
	\meta{number} can be upper or lower case, but any resulting 
	alphabetic digits are in \emph{lower case}.
	
\begin{codeexample}[]
\pgfmathbasetobase\mynumber{11011011}{2}{16} \mynumber
\end{codeexample}

\end{command}

\begin{command}{\pgfmathbasetoBase\marg{macro}\marg{number}\marg{base-1}\marg{base-2}}
	Defines \meta{macro} as the result of converting \meta{number} from
	base \meta{base-1} to base \meta{base-2}. Alphabetic digits in 
	\meta{number} can be upper or lower case, but any resulting 
	alphabetic digits are in \emph{upper case}.
	
\begin{codeexample}[]
\pgfmathbasetoBase\mynumber{121212}{3}{12} \mynumber
\end{codeexample}

\end{command}


\begin{command}{\pgfmathsetbasenumberlength\marg{integer}}
	Set the number of digits in the result of a base conversion to 
	\meta{integer}. If the result of a conversion has less digits
	than this number it is prefixed with zeros.

\begin{codeexample}[]
\pgfmathsetbasenumberlength{8}
\pgfmathdectobase\mynumber{15}{2} \mynumber
\end{codeexample}

\end{command}
% Copyright 2007 by Mark Wibrow
%
% This file may be distributed and/or modified
%
% 1. under the LaTeX Project Public License and/or
% 2. under the GNU Free Documentation License.
%
% See the file doc/generic/pgf/licenses/LICENSE for more details.


\section{Customising the Mathematical Engine}

\label{pgfmath-reimplement}


Perhaps you have a desire for some function that \pgfname\ does not 
provide. Perhaps you are not happy with the accuracy or efficiency of 
some of the algorithms that are implemented in \pgfname. In these 
cases you will want to add a function to the parser or replace the 
current implementations of the algorithms with your own code.

The mathematical engine was designed with such customisation in mind.
It is possible to add new functions, or modify the code for for 
existing functions. Note, however, that whilst adding new operators 
is possible, it can be a rather tricky business and is only 
recommended for adventurous users.

To add a new function to the math engine the following command can be
used:

\begin{command}{\pgfmathdeclarefunction\marg{function name}\marg{number of arguments}\marg{code}}

  This will set up the parser to recognise a function called 
  \meta{name}. The name of the function can consist of, uppercase or
  lower case letters, numbers or the underscore |_|. In line with 
  many programming languages, a function name cannot begin with a 
  number or contain any spaces.
  
  The \meta{number of arguments} can be any positive integer, zero,
  or the value |...|, which indicates a variable number of 
  arguments. \pgfname{} treats constants, such as |pi| and |e|, as
  functions with zero arguments. Functions with more than nine
  arguments or with variable arguments are a ``bit special'' and
  are discussed below.
  
  The effect of \meta{code} should be to set the macro 
  |\pgfmathresult| to the correct value (namely to the result of the 
  computation without units).  Furthermore, the function should have 
  no other side effects, that is, it should not change any global 
  values. As an example, consider the creation of a new function
  |double|, which takes one argument, and returns the value of that
  argument times two.
  
\begin{codeexample}[]
\makeatletter
\pgfmathdeclarefunction{double}{1}{
  \begingroup
    \pgf@x=#1pt\relax
    \multiply\pgf@x by2\relax
    \pgfmathreturn\pgf@x
  \endgroup
}
\makeatother
\pgfmathparse{double(44.3)}\pgfmathresult
\end{codeexample}

  The macro |\pgfmathreturn|\meta{tokens} must be 
  directly followed by an |\endgroup| and will save the result of the 
  computation, by defining |\pgfmathresult| as the expansion of 
  \meta{tokens} (without units) outside the group, so \meta{tokens}
  must be somthing that can be assigned to a dimension register.
  
  Alternatively, the |\pgfmathsmuggle|\meta{macro} can be used. This
  must also be directly followed by an |\endgroup| and will simply
  ``smuggle'' the definition of \meta{macro} outside the \TeX-group.
  
	By performing computations within a \TeX-group, \pgfname{}
	registers such as |\pgf@x|, |\pgf@y| and |\c@pgf@counta|, 
	|\c@pgfcountb|, and so forth, can be used at will.
  
  Beyond setting up the parser, this command also defines two macros
  which provide access to the function independently of the parser:  
    
  \begin{itemize}
  \item
  |\pgfmath|\meta{function name}
  
  This macro will provide ``public'' interface for the function 
  \meta{function name} allowing the function to be called 
  independently of the parser. All arguments passed to this macro are
  evaluated using |\pgfmathparse| and then passed on to the following 
  macro:
  
  \item
  |\pgfmath|\meta{function name}|@|
  
  This macro is the ``private'' implementation of the functions 
  algorithm (but note that, for speed, the parser calls this macro 
  rather than the ``public'' one). Arguments passed to this macro 
  are expected to be numbers without units. It is defined using
  \meta{code}, but need not be self contained.
  
  \end{itemize}
  
  For functions that are declared with less than ten arguments,
  the public macro is defined in the same way as normal \TeX{} 
  macros using, for example, |\def\pgfmathNoArgs{|\meta{code}|}| 
  for a function with no arguments, or
   |\def\pgfmathThreeArgs#1#2#3{|\meta{code}|}| for a function with
   three arguments.
  The private macro is defined in the same way, and each argument 
  can therefore be accessed in \meta{code}
  using |#1|, |#2| and so on.
  
  For functions with more than nine arguments, or functions with
  a variable number of arguments, these macros are only
  defined as taking \emph{one} argument. The public macro
  expects its arguments to be comma separated, for example,
  |\pgfmathVariableArgs{1.1,3.5,-1.5,2.6}|. Each
  argument is parsed and 
  passed on to the private macro as follows:
  |\pgfmathVariableArgs@{{1.1}{3.5}{-1.5}{2.6}}|.
  This means that some ``extra work'' will be required to access
  each argument (although it is a fairly simple task).
  
  Note, that there are, two execptions to this arrangement:
  the public versions of the |min| and |max| functions still
  take two arguments for compatibility with older versions, but
  each of these arguments can take several comma separated values.
  


  
\end{command}

  To redefine a function use the following command: 

\begin{command}{\pgfmathredeclarefunction\marg{function name}\marg{algorithm code}}
  
  This command redefines the |\pgfmath|\meta{function name}|@| macro 
  with the new \meta{algorithm code}. See the description of the
  |\pgfmathdeclarefunction| for details. You cannot change the number
  of arguments for an existing function.

\end{command}

  \pgfname{} uses the last known definition of a function within the
  prevailing scope, so it is possible for a function to be redefined 
  locally. You should also remember that any |.sty| or |.tex| file
  contatining any re-implementions should be loaded after pgf-Math.
  


\part{The Basic Layer}

{\Large \emph{by Till Tantau}}


\bigskip
\noindent
\vskip1cm
\begin{codeexample}[graphic=white]
\begin{tikzpicture}
  \draw[gray,very thin] (-1.9,-1.9) grid (2.9,3.9)
          [step=0.25cm] (-1,-1) grid (1,1);
  \draw[blue] (1,-2.1) -- (1,4.1); % asymptote
                
  \draw[->] (-2,0) -- (3,0) node[right] {$x(t)$};
  \draw[->] (0,-2) -- (0,4) node[above] {$y(t)$};

  \foreach \pos in {-1,2}
    \draw[shift={(\pos,0)}] (0pt,2pt) -- (0pt,-2pt) node[below] {$\pos$};

  \foreach \pos in {-1,1,2,3}
    \draw[shift={(0,\pos)}] (2pt,0pt) -- (-2pt,0pt) node[left] {$\pos$};

  \fill (0,0) circle (0.064cm);
  \draw[thick,parametric,domain=0.4:1.5,samples=200]
    % The plot is reparameterised such that there are more samples
    % near the center.
    plot[id=asymptotic-example] function{(t*t*t)*sin(1/(t*t*t)),(t*t*t)*cos(1/(t*t*t))}
    node[right] {$\bigl(x(t),y(t)\bigr) = (t\sin \frac{1}{t}, t\cos \frac{1}{t})$};

  \fill[red] (0.63662,0) circle (2pt)
    node [below right,fill=white,yshift=-4pt] {$(\frac{2}{\pi},0)$};
\end{tikzpicture}
\end{codeexample}


% Copyright 2006 by Till Tantau
%
% This file may be distributed and/or modified
%
% 1. under the LaTeX Project Public License and/or
% 2. under the GNU Free Documentation License.
%
% See the file doc/generic/pgf/licenses/LICENSE for more details.


\section{Design Principles}

This section describes the basic layer of \pgfname. This layer is
build on top of the system layer. Whereas the system layer just
provides the absolute minimum for drawing graphics, the basic
layer provides numerous commands that make it possible to create
sophisticated graphics easily and also quickly.

The basic layer does not provide a convenient syntax for describing
graphics, which is left to frontends like \tikzname. For this reason, the
basic layer is typically used only by ``other programs.'' For example,
the \textsc{beamer} package uses the basic layer extensively, but does
not need a convenient input syntax. Rather, speed and flexibility are
needed when \textsc{beamer} creates graphics.

The following basic design principles underlie the basic layer:
\begin{enumerate}
\item Structuring into a core and several optional packages.
\item Consistently named \TeX\ macros for all graphics commands.
\item Path-centered description of graphics.
\item Coordinate transformation system.
\end{enumerate}



\subsection{Core and Optional Packages}

The basic layer consists of a \emph{core package}, called |pgfcore|,
which provides the most basic commands, and several optional package
like |pgfbaseshade| that offer more special-purpose commands.

\begin{package}{pgfbaseplot}
  provides commands for plotting functions
\end{package}

\begin{package}{pgfbaseshapes}
  provides commands for drawing shapes and nodes
\end{package}

\begin{package}{pgfbasepatterns}
  provides commands for declaring and using tiling
  patterns
\end{package}

\begin{package}{pgfbaseimage}
  This package provides commands for including external
  images. For \LaTeX-users the |graphicx| package does a better job at
  this than the |pgfbaseimage| package does, so you should normally use
  |\includegraphics| and not |\pgfimage|. However, in some situations
  (like when masking is needed or when plain \TeX\ is used) this
  package is needed.
\end{package}

\begin{package}{pgfbaselayers}
  This package provides commands for creating layered
  graphics. Using layers you can later on say that a certain path
  should be behind a path that was specified earlier.
\end{package}

\begin{package}{pgfbasesnakes}
  This package provides commands for adding snaked lines to the
  path. Such lines are not straight but rather wind in some specific
  fashion.
\end{package}


\begin{package}{pgfbasematrix}
  This package provides the |\pgfmatrix| command.
\end{package}

If you say |\usepackage{pgf}| or |\input pgf.tex| or
|\usemodule[pgf]|, all of the optional packages are loaded (as well as
the core and the system layer). 



\subsection{Communicating with the Basic Layer via Macros}

In order to ``communicate'' with the basic layer you use long
sequences of commands that start with |\pgf|. You are only allowed to
give these commands inside a |{pgfpicture}| environment. (Note that
|{tikzpicture}| opens a |{pgfpicture}| internally, so you can freely
mix \pgfname\ commands and \tikzname\ commands inside a
|{tikzpicture}|.) It is possible to ``do other things'' between the
commands. For example, you might use one command to move to a certain
point, then have a complicated computation of the next point, and then
move there. 

\begin{codeexample}[]
\newdimen\myypos
\begin{pgfpicture}
  \pgfpathmoveto{\pgfpoint{0cm}{\myypos}}
  \pgfpathlineto{\pgfpoint{1cm}{\myypos}}
  \advance \myypos by 1cm
  \pgfpathlineto{\pgfpoint{1cm}{\myypos}}
  \pgfpathclose
  \pgfusepath{stroke}
\end{pgfpicture}
\end{codeexample}

The following naming conventions are used in the basic layer:

\begin{enumerate}
\item
  All commands and environments start with |pgf|.
\item
  All commands that specify a point (a coordinate) start with |\pgfpoint|.
\item
  All commands that extend the current path start with |\pgfpath|.
\item
  All commands that set/change a graphics parameter start with |\pgfset|.
\item
  All commands that use a previously declared object (like a path,
  image or shading) start with |\pgfuse|.
\item
  All commands having to do with coordinate transformations start with
  |\pgftransform|. 
\item
  All commands having to do with arrow tips start with |\pgfarrows|.
\item
  All commands for ``quickly'' extending or drawing a path start with
  |\pgfpathq| or |\pgfusepathq|.
\item
  All commands having to do with matrices start with |\pgfmatrix|.
\end{enumerate}


\subsection{Path-Centered Approach}

In \pgfname\ the most important entity is the \emph{path}. All
graphics are composed of numerous paths that can be stroked,
filled, shaded, or clipped against. Paths can be closed or open, they
can self-intersect and consist of unconnected parts.

Paths are first \emph{constructed} and then \emph{used}. In order to
construct a path, you can use commands starting with |\pgfpath|. Each
time such a command is called, the current path is extended in some
way.

Once a path has been completely constructed, you can use it using the
command |\pgfusepath|. Depending on the parameters given to this
command, the path will be stroked (drawn) or filled or subsequent
drawings will be clipped against this path.




\subsection{Coordinate Versus Canvas Transformations}

\label{section-design-transformations}

\pgfname\ provides two transformation systems: \pgfname's own
\emph{coordinate} transformation matrix and \pdf's or PostScript's
\emph{canvas} transformation matrix. These two systems are quite
different. Whereas a scaling by a factor of, say, $2$ of the canvas
causes \emph{everything} to be scaled by this factor (including
the thickness of lines and text), a scaling of two in the coordinate 
system causes only the \emph{coordinates} to be scaled, but not the
line width nor text.

By default, all transformations only apply to the coordinate
transformation system. However, using the command |\pgflowlevel|
it is possible to apply a transformation to the canvas.

Coordinate transformations are often preferable over canvas
transformations. Text and lines that are transformed using canvas 
transformations suffer from differing sizes and lines whose thickness 
differs depending on whether the line is horizontal or vertical. To
appreciate the difference, consider the following two ``circles'' both
of which have been scaled in the $x$-direction by a factor of $3$ and
by a factor of $0.5$ in the $y$-direction. The left circle uses a
canvas transformation, the right uses \pgfname's coordinate
transformation (some viewers will render the left graphic incorrectly
since they do no apply the low-level transformation the way they
should):  

\begin{tikzpicture}[line width=5pt]
  \useasboundingbox (-1.75,-1) rectangle (14,1);
  
  \begin{scope}
    \pgflowlevel{\pgftransformxscale{3}}
    \pgflowlevel{\pgftransformyscale{.5}}

    \draw (0,0) circle (0.5cm);
    \draw (.55cm,0pt) node[right] {canvas};
  \end{scope}
  \begin{scope}[xshift=9cm,xscale=3,yscale=.5]
    \draw (0,0) circle (0.5cm);
    \draw (.55cm,0pt) node[right] {coordinate};
  \end{scope}
\end{tikzpicture}


% Copyright 2003 by Till Tantau <tantau@cs.tu-berlin.de>.
%
% This program can be redistributed and/or modified under the terms
% of the LaTeX Project Public License Distributed from CTAN
% archives in directory macros/latex/base/lppl.txt.


\section[Hierarchical Structures: Package, Environments, Scopes, and Text]
{Hierarchical Structures:\\
  Package, Environments, Scopes, and Text}


\subsection{Overview}

\pgfname\ uses two kinds of hierarchical structuring: First, the
package itself is structured hierarchically, consisting of different
packages that are built on top of each other. Second, \pgfname\ allows you
to structure your graphics hierarchically using environments and scopes.

\subsubsection{The  Hierarchical Structure of the Package}

The \pgfname\ system consists of several layers:

\begin{description}
\item[System layer.]
  The lowest layer is called the \emph{system layer}, though it might
  also be called ``driver layer'' or perhaps ``backend layer.'' Its
  job is to provide an abstraction of the details of which driver
  is used to transform the |.dvi| file. The system layer is
  implemented by the package |pgfsys|, which will load appropriate
  driver files as needed.

  The system layer is documented in Part~\ref{part-system}.
\item[Basic layer.]
  The basic layer is loaded by the package |pgf|. Some
  applications do not need all of the functionality of the basic
  layer, so it is possible to load only the |pgfcore| and some other
  packages starting with |pgfbase|.

  The basic layer is documented in the present part.
\item[Frontend layer.]
  The frontend layer is not loaded by a single packages. Rather,
  different packages, like \tikzname\ or \textsc{pgfpict2e}, are
  different frontends to the basic layer.

  The \tikzname\ frontend is documented in Part~\ref{part-tikz}.
\end{description}

Each layer will automatically load the necessary files of the layers below
it. 

In addition to the packages of these layers, there are also some
library packages. These packages provide additional definitions of
things like new arrow tips or new plot handlers.

The library packages are documented in Part~\ref{part-libraries}.




\subsubsection{The Hierarchical Structure of Graphics}

Graphics in \pgfname\ are typically structured
hierarchically. Hierarchical structuring can be used to identify
groups of graphical elements that are to be treated ``in the same
way.'' For example, you might group together a number of paths, all of
which are to be drawn in red. Then, when you decide later on that you
like them to be drawn in, say, blue, all you have to do is to change
the color once.

The general mechanism underlying hierarchical structuring is known as
\emph{scoping} in computer science. The idea is that all changes to
the general ``state'' of the graphic that are done inside a scope are
local to that scope. So, if you change the color inside a scope, this
does not affect the color used outside the scope. Likewise, when you
change the line width in a scope, the line width outside is not
changed, and so on.

There are different ways of starting and ending scopes of graphic
parameters. Unfortunately, these scopes are sometimes ``in conflict''
with each other and it is sometimes not immediately clear which scopes
apply. In essence, the following scoping mechanisms are available:

\begin{enumerate}
\item
  The ``outermost'' scope supported by \pgfname\ is the |{pgfpicture}|
  environment. All changes to the graphic state done inside a
  |{pgfpicture}| are local to that picture.

  In general, it is \emph{not} possible to set graphic parameters
  globally outside any |{pgfpicture}| environments. Thus, you can
  \emph{not} say |\pgfsetlinewidth{1pt}| at the beginning of your
  document to have a default line width of one point. Rather, you have
  to (re)set all graphic parameters inside each |{pgfpicture}|. (If
  this is too bothersome, try defining some macro that does the job
  for you.)
\item
  Inside a |{pgfpicture}| you can use a |{pgfscope}| environment to
  keep changes of the graphic state local to that environment.

  The effect of commands that change the graphic state are local to
  the current |{pgfscope}| but not always to the current \TeX\
  group. Thus, if you open a \TeX\ group (some text in curly braces)
  inside a |{pgfscope}|, and if you change, for example, the dash
  pattern, the effect of this changed dash pattern will persist till
  the end of the |{pgfscope}|.

  Unfortunately, this is not always the case. \emph{Some} graphic
  parameters only persist till the end of the current \TeX\ group. For
  example, when you use |\pgfsetarrows| to set the arrow tip 
  inside a \TeX\ group, the effect lasts only till the end of the
  current \TeX\ group.
\item
  Some graphic parameters are not scoped by |{pgfscope}| but
  ``already'' by \TeX\ groups. For example, the effect of coordinate
  transformation commands is always local to the current \TeX\
  group.

  Since every |{pgfscope}| automatically creates a \TeX\ group, all
  graphic parameters that are local to the current \TeX\ group are
  also local to the current |{pgfscope}|.
\item
  Some graphic parameters can only be scoped using \TeX\ groups, since
  in some situations it is not possible to introduce a
  |{pgfscope}|. For example, a path always has to be completely
  constructed and used in the same |{pgfscope}|. However, we might
  wish to have different coordinate transformations apply to different
  points on the path. In this case, we can use \TeX\ groups to keep
  the effect local, but we could not use |{pgfscope}|.  
\item
  The |\pgftext| command can be used to create a scope in which \TeX\
  ``escapes back'' to normal \TeX\ mode. The text passed to the
  |\pgftext| is ``heavily guarded'' against having any effect on the
  scope in which it is used. For example, it is possibly to use
  another  |{pgfpicture}| environment inside the argument of
  |\pgftext|. 
\end{enumerate}


Most of the complications can be avoided if you stick to the following
rules:

\begin{itemize}
\item
  Give graphic commands only inside |{pgfpicture}| environments.
\item
  Use |{pgfscope}| to structure graphics.
\item
  Do not use \TeX\ groups inside graphics, \emph{except} for keeping
  the effect of coordinate transformations local.
\end{itemize}



\subsection{The Hierarchical Structure of the Package}

Before we come to the structuring commands provided by \pgfname\ to
structure your graphics, let us first have a look at the structure of
the package itself.

\subsubsection{The Main Package}

To use \pgfname, include the following package:

\begin{package}{pgf}
  This package loads the complete ``basic layer'' of \pgfname. That
  is, it will load all of the commands described in the current part
  of this manual, but it will not load frontends like \tikzname.

  In detail, this package will load the following packages, each of
  which can also be loaded individually:
  \begin{itemize}
  \item
    |pgfsys|, which is the lowest layer of \pgfname\ and which is
    always needed. This file will read |pgf.cfg| to discern which
    driver is to be used. See Section~\ref{section-pgfsys} for
    details. 
  \item
    |pgfcore|, which is the central core of \pgfname\ and which is
    always needed unless you intend to write a new basic layer from
    scratch.
  \item
    |pgfbaseimage|, which provides commands for declaring and
    using images. An example is |\pgfuseimage|.
  \item
    |pgfbaseshapes|, which provides commands for declaring and using
    shapes. An example is |\pgfdeclareshape|.
  \item
    |pgfbaseplot|, which provides commands for plotting functions.    
  \end{itemize}

  Including any of the last three packages will automatically load the
  first two.
\end{package}

In \LaTeX, the package takes two options:
\begin{packageoption}{draft}
  When this option is set, all images will be replaced by empty
  rectangles. This can speedup compilation.
\end{packageoption}
 
\begin{packageoption}{version=\meta{version}}
  Indicates that the commands of version \meta{version} need to be
  defined. If you set \meta{version} to |0.65|, then a large bunch of
  ``compatibility commands'' are loaded. If you set \meta{version} to
  |0.96|, then these compatibility commands will not be loaded.

  If this option is not given at all, then the commands of all
  versions are defined.
\end{packageoption}


\subsubsection{The Core Package}

\begin{package}{pgfcore}
  This package defines all of the basic layer's commands, except for
  the commands defined in the additional packages like
  |pgfbaseplot|. Typically commands defined by the core include
  |\pgfusepath| or   |\pgfpoint|. The core is internally structured
  into several subpackages, but the subpackages cannot be loaded
  individually since they are all ``interrelated.''
\end{package}


\subsubsection{The Optional Basic Layer Packages}

The |pgf| package automatically loads the following packages, but you
can also load them individually (all of them automatically include the
core):

\begin{itemize}
  \item |pgfbaseshapes|
  This package provides commands for drawing nodes and shapes. These
  commands are explained in Section~\ref{section-shapes}.

  \item |pgfbaseplot|
  This package provides commands for plotting function. The
  commands are explained in Section~\ref{section-plots}.

  \item |pgfbaseimage|
  This package provides commands for including (external) images. The 
  commands are explained in Section~\ref{section-images}.
\end{itemize}





\subsection{The Hierarchical Structure of the Graphics}

\subsubsection{The Main Environment}


Most, but not all, commands of the \pgfname\ package must be given
within a |{pgfpicture}| environment. The only commands that (must) be
given outside are commands having to do with including images (like
|\pgfuseimage|) and with inserting complete shadings (like
|\pgfuseshading|). However, just to keep life entertaining, the
|\pgfshadepath| command must be given \emph{inside} a |{pgfpicture}|
environment.

\begin{environment}{{pgfpicture}}
  This environment will insert a \TeX\ box containing the graphic drawn by
  the \meta{environment contents} at the current position. 

  \medskip
  \textbf{The size of the bounding box.}
  The size of the box is determined in the following
  manner: While \pgfname\ parses the \meta{environment contents}, it
  keeps track of a bounding box for the graphic. Essentially, this
  bounding box is the smallest box that contains all coordinates
  mentioned in the graphics. Some coordinates may be ``mentioned'' by
  \pgfname\ itself; for example, when you add circle to the current
  path, the support points of the curve making up the circle are also
  ``mentioned'' despite the fact that you will not ``see'' them in
  your code.

  Once the \meta{environment contents} has been parsed completely, a
  \TeX\ box is created whose size is the size of the computed bounding
  box and this box is inserted at the current position.

\begin{codeexample}[]
Hello \begin{pgfpicture}
  \pgfpathrectangle{\pgfpointorigin}{\pgfpoint{2ex}{1ex}}
  \pgfusepath{stroke}
\end{pgfpicture} World!
\end{codeexample}

  Sometimes, you may need more fine-grained control over the size of
  the bounding box. For example, the computed bounding box may be too
  large or you intensionally wish the box to be ``too small.'' In
  these cases, you can use the command
  |\pgfusepath{use as bounding box}|, as described in
  Section~\ref{section-using-bb}.


  \medskip
  \textbf{The baseline of the bounding box.}
  When the box containing the graphic is inserted into the normal
  text, the baseline of the graphic is normally at the bottom of the
  graphic. For this reason, the following two sets of code lines have
  the same effect, despite the fact that the second graphic uses
  ``higher'' coordinates than the first:
 
\begin{codeexample}[]
Rectangles \begin{pgfpicture}
  \pgfpathrectangle{\pgfpointorigin}{\pgfpoint{2ex}{1ex}}
  \pgfusepath{stroke}
\end{pgfpicture} and \begin{pgfpicture}
  \pgfpathrectangle{\pgfpoint{0ex}{1ex}}{\pgfpoint{2ex}{1ex}}
  \pgfusepath{stroke}
\end{pgfpicture}.
\end{codeexample}

  You can change the baseline using the |\pgfsetbaseline| command, see
  below. 

\begin{codeexample}[]
Rectangles \begin{pgfpicture}
  \pgfpathrectangle{\pgfpointorigin}{\pgfpoint{2ex}{1ex}}
  \pgfusepath{stroke}
  \pgfsetbaseline{0pt}
\end{pgfpicture} and \begin{pgfpicture}
  \pgfpathrectangle{\pgfpoint{0ex}{1ex}}{\pgfpoint{2ex}{1ex}}
  \pgfusepath{stroke}
  \pgfsetbaseline{0pt}
\end{pgfpicture}.
\end{codeexample}

  \medskip
  \textbf{Including text and images in a picture.}
  You cannot directly include text and images in a picture. Thus, you
  should \emph{not} simply write some text in a |{pgfpicture}| or use
  a command like |\includegraphics| or even |\pgfimage|. In all these
  cases, you need to place the text inside a |\pgftext| command. This
  will ``escape back'' to normal \TeX\ mode, see
  Section~\ref{section-text-command} for details.

  \medskip
  \textbf{Remembering a picture position for later reference.}
  After a picture has been typset, its position on the page is
  normally forgotten by \pgfname\ and also by \TeX. This means that is
  not possible to reference a node in this picture later on. In
  particular, it is normally impossible to draw lines between nodes in
  different pictures automatically.

  In order to make \pgfname\ ``remember'' a picture, the \TeX-\if
  |\||ifpgfrememberpicturepositiononpage| should be set to |true|. It
  is only important that this \TeX-if is |true| at the end of the
  |{pgfpicture}|-environment, so you can switch it on inside the
  environment. However, you can also just switch it on globally, then
  the positions of all pictures are remembered.

  There are several reasons why the remembering is not switched on by
  default. First, it does not work for all backend drivers (currently, it
  works only for pdf\TeX). Second, it requires two passes of \TeX\
  over the file; on the first pass all positions will be wrong. Third,
  for every remembered picture a line is added to the |.aux|-file,
  which may result in a large number of extra lines.

  Despite all these ``problems,'' for documents that are processed
  with pdf\TeX\ and in which there is only a small number of pictures
  (less than a hundred or so), you can switch on this option globally,
  it will not cause any significant slowing of \TeX.
\end{environment}

\begin{plainenvironment}{{pgfpicture}}
  The plain \TeX\ version of the environment. Note that in this
  version, also, a \TeX\ group is created around the environment.
\end{plainenvironment}

\begin{contextenvironment}{{pgfpicture}}
  This is the Con\TeX t version of the environment.
\end{contextenvironment}


{\let\ifpgfrememberpicturepositiononpage=\relax
\begin{command}{\ifpgfrememberpicturepositiononpage}
  Determines whether the position of pictures on the page should be
  recorded. The value of this \TeX-if at the end of a |{pgfpicture}|
  environment is important, not the value at the beginning.

  If this option is set to true of a picture, \pgfname\ will attempt
  to record the position of the picture on the page. (This attempt
  will fail with most drivers and when it works it typically requires
  two runs of \TeX.) The position is not directly accessible. Rather,
  the nodes mechanism will use this position if you access a node from
  another picture. See Sections~\ref{section-cross-pictures-pgf}
  and~\ref{section-cross-picture-tikz} for more details. 
\end{command}
}


\makeatletter
\begin{command}{\pgfsetbaseline\marg{dimension}}
  This command specifies a $y$-coordinate of the picture that should
  be used as the baseline of the whole picture. When a \pgfname\
  picture has been typeset completely, \pgfname\ must decide at which
  height the baseline of the picture should lie. Normally, the
  baseline is set to the $y$-coordinate of the bottom of the picture,
  but it is often desirable to use another height.

\begin{codeexample}[]
Text \tikz{\pgfpathcircle{\pgfpointorigin}{1ex}\pgfusepath{stroke}},
     \tikz{\pgfsetbaseline{0pt}
          \pgfpathcircle{\pgfpointorigin}{1ex}\pgfusepath{stroke}},
     \tikz{\pgfsetbaseline{.5ex}
          \pgfpathcircle{\pgfpointorigin}{1ex}\pgfusepath{stroke}},
     \tikz{\pgfsetbaseline{-1ex}
          \pgfpathcircle{\pgfpointorigin}{1ex}\pgfusepath{stroke}}.
\end{codeexample}
\end{command}

\begin{command}{\pgfsetbaselinepointnow\marg{point}}
  This command specifies the baseline indirectly, namely as the
  $y$-coordinate that the given \meta{point} has when the command is
  called.
\end{command}

\begin{command}{\pgfsetbaselinepointlater\marg{point}}
  This command also specifies the baseline indirectly, but the
  $y$-coordinate of the given \meta{point} is only computed at the end
  of the picture.

\begin{codeexample}[]
Hello
\tikz{
  \pgfsetbaselinepointlater{\pgfpointanchor{X}{base}}
  % Note: no shape X, yet
  \node [cross out,draw] (X) {world.};
}
\end{codeexample}
\end{command}



\subsubsection{Graphic Scope Environments}

Inside a |{pgfpicture}| environment you can substructure your picture
using the following environment:

\begin{environment}{{pgfscope}}
  All changes to the graphic state done inside this environment are
  local to the environment. The graphic state includes the following:
  \begin{itemize}
  \item
    The line width.
  \item
    The stroke and fill colors.
  \item
    The dash pattern.
  \item
    The line join and cap.
  \item
    The miter limit.
  \item
    The canvas transformation matrix.
  \item
    The clipping path.
  \end{itemize}
  Other parameters may also influence how graphics are rendered, but they
  are \emph{not} part of the graphic state. For example, the arrow tip
  kind is not part of the graphic state and the effect of commands
  setting the arrow tip kind are local to the current \TeX\ group, not
  to the current |{pgfscope}|. However, since |{pgfscope}| starts and
  ends a \TeX\ group automatically, a |{pgfscope}| can be used to
  limit the effect of, say, commands that set the arrow tip kind.

\begin{codeexample}[]
\begin{pgfpicture}
  \begin{pgfscope}
    {
      \pgfsetlinewidth{2pt}
      \pgfpathrectangle{\pgfpointorigin}{\pgfpoint{2ex}{2ex}}
      \pgfusepath{stroke}
    }
    \pgfpathrectangle{\pgfpoint{3ex}{0ex}}{\pgfpoint{2ex}{2ex}}
    \pgfusepath{stroke}
  \end{pgfscope}
  \pgfpathrectangle{\pgfpoint{6ex}{0ex}}{\pgfpoint{2ex}{2ex}}
  \pgfusepath{stroke}
\end{pgfpicture}
\end{codeexample}
  
\begin{codeexample}[]
\begin{pgfpicture}
  \begin{pgfscope}
    {
      \pgfsetarrows{-to}
      \pgfpathmoveto{\pgfpointorigin}\pgfpathlineto{\pgfpoint{2ex}{2ex}}
      \pgfusepath{stroke}
    }
    \pgfpathmoveto{\pgfpoint{3ex}{0ex}}\pgfpathlineto{\pgfpoint{5ex}{2ex}}
    \pgfusepath{stroke}
  \end{pgfscope}
  \pgfpathmoveto{\pgfpoint{6ex}{0ex}}\pgfpathlineto{\pgfpoint{8ex}{2ex}}
  \pgfusepath{stroke}
\end{pgfpicture}
\end{codeexample}

  At the start of the scope, the current path must be empty, that is,
  you cannot open a scope while constructing a path.

  It is usually a good idea \emph{not} to introduce \TeX\ groups
  inside a |{pgfscope}| environment.
\end{environment}

\begin{plainenvironment}{{pgfscope}}
  Plain \TeX\ version of the |{pgfscope}| environment.
\end{plainenvironment}

\begin{contextenvironment}{{pgfscope}}
  This is the Con\TeX t version of the environment.
\end{contextenvironment}


The following scopes also encapsulate certain properties of the
graphic state. However, they are typically not used directly by the
user.

\begin{environment}{{pgfinterruptpath}}
  This environment can be used to temporarily interrupt the
  construction of the current path. The effect will be that the path
  currently under construction will be ``stored away'' and restored at
  the end of the environment. Inside the environment you can construct
  a new path and do something with it.

  An example application of this environment is the arrow tip
  caching. Suppose you ask \pgfname\ to use a specific arrow tip
  kind. When the arrow tip needs to be rendered for the first time,
  \pgfname\ will ``cache'' the path that makes up the arrow tip. To do
  so, it interrupts the current path construction and then protocols
  the path of the arrow tip. The |{pgfinterruptpath}| environment is
  used to ensure that this does not interfere with the path to which
  the arrow tips should be attached.

  This command does \emph{not} install a |{pgfscope}|. In particular,
  it does not call any |\pgfsys@| commands at all, which would,
  indeed, be dangerous in the middle of a path construction.
\end{environment}

\begin{plainenvironment}{{pgfinterruptpath}}
  Plain \TeX\ version of the environment.
\end{plainenvironment}

\begin{contextenvironment}{{pgfinterruptpath}}
  Con\TeX t version of the environment.
\end{contextenvironment}


\begin{environment}{{pgfinterruptpicture}}
  This environment can be used to temporarily interrupt a
  |{pgfpicture}|. However, the environment is intended only to be used
  at the beginning and end of a box that is (later) inserted into a
  |{pgfpicture}| using |\pgfqbox|. You cannot use this environment
  directly inside a |{pgfpicture}|.

\begin{codeexample}[]
\begin{pgfpicture}
  \pgfpathmoveto{\pgfpoint{0cm}{0cm}} % In the middle of path, now
  \newbox\mybox
  \setbox\mybox=\hbox{
    \begin{pgfinterruptpicture}
      Sub-\begin{pgfpicture} % a subpicture
        \pgfpathmoveto{\pgfpoint{1cm}{0cm}}
        \pgfpathlineto{\pgfpoint{1cm}{1cm}}
        \pgfusepath{stroke}
      \end{pgfpicture}-picture.
    \end{pgfinterruptpicture}
  }
  \pgfqbox{\mybox}%
  \pgfpathlineto{\pgfpoint{0cm}{1cm}}
  \pgfusepath{stroke}
\end{pgfpicture}\hskip3.9cm
\end{codeexample}
\end{environment}

\begin{plainenvironment}{{pgfinterruptpicture}}
  Plain \TeX\ version of the environment.
\end{plainenvironment}

\begin{contextenvironment}{{pgfinterruptpicture}}
  Con\TeX t version of the environment.
\end{contextenvironment}


\subsubsection{Inserting Text and Images}

\label{section-text-command}

Often, you may wish to add normal \TeX\ text at a certain point inside
a |{pgfpicture}|. You cannot do so ``directly,'' that is, you cannot
simply write this text inside the |{pgfpicture}| environment. Rather,
you must pass the text as an argument to the |\pgftext| command.

You must \emph{also} use the |\pgftext| command to insert an image or
a shading into a |{pgfpicture}|.

\begin{command}{\pgftext\opt{\oarg{options}}\marg{text}}
  This command will typeset \meta{text} in normal \TeX\ mode and
  insert the resulting box into the |{pgfpicture}|. The bounding box
  of the graphic will be updated so that all of the text box is
  inside. Be default, the text box is centered at the origin, but this
  can be changed either by giving appropriate \meta{options} or by
  applying an appropriate coordinate transformation beforehand.

  The \meta{text} may contain verbatim text. (In other words, the
  \meta{text} ``argument'' is not a normal argument, but is put in a
  box and some |\aftergroup| hackery is used to find the end of the
  box.)

  \pgfname's current (high-level) coordinate transformation is
  synchronized with the canvas transformation matrix temporarily
  when the text box is inserted. The effect is that if there is
  currently a high-level rotation of, say, 30 degrees, the \meta{text}
  will also be rotated by thirty degrees. If you do not want this
  effect, you have to (possibly temporarily) reset the high-level
  transformation matrix.

  The following \meta{options} may be given as conveniences:
  \begin{itemize}
    \itemoption{left}
    causes the text box to be placed such that its left border is on the origin.
\begin{codeexample}[]
\tikz{\draw[help lines] (-1,-.5) grid (1,.5);
     \pgftext[left] {lovely}}
\end{codeexample}
    \itemoption{right}
    causes the text box to be placed such that its right border is on the origin.
\begin{codeexample}[]
\tikz{\draw[help lines] (-1,-.5) grid (1,.5);
     \pgftext[right] {lovely}}
\end{codeexample}
    \itemoption{top}
    causes the text box to be placed such that its top is on the
    origin. This option can be used together with the |left| or
    |right| option.
\begin{codeexample}[]
\tikz{\draw[help lines] (-1,-.5) grid (1,.5);
     \pgftext[top] {lovely}}
\end{codeexample}
\begin{codeexample}[]
\tikz{\draw[help lines] (-1,-.5) grid (1,.5);
     \pgftext[top,right] {lovely}}
\end{codeexample}
    \itemoption{bottom}
    causes the text box to be placed such that its bottom is on the
    origin.
\begin{codeexample}[]
\tikz{\draw[help lines] (-1,-.5) grid (1,.5);
     \pgftext[bottom] {lovely}}
\end{codeexample}
\begin{codeexample}[]
\tikz{\draw[help lines] (-1,-.5) grid (1,.5);
     \pgftext[bottom,right] {lovely}}
\end{codeexample}
    \itemoption{base}
    causes the text box to be placed such that its baseline is on the
    origin.
\begin{codeexample}[]
\tikz{\draw[help lines] (-1,-.5) grid (1,.5);
     \pgftext[base] {lovely}}
\end{codeexample}
\begin{codeexample}[]
\tikz{\draw[help lines] (-1,-.5) grid (1,.5);
     \pgftext[base,right] {lovely}}
\end{codeexample}
    \itemoption{at}|=|\meta{point}
      Translates the origin (that is, the point where the text is
      shown) to \meta{point}. 
\begin{codeexample}[]
\tikz{\draw[help lines] (-1,-.5) grid (1,.5);
     \pgftext[base,at={\pgfpoint{1cm}{0cm}}] {lovely}}
\end{codeexample}
    \itemoption{x}|=|\meta{dimension}
      Translates the origin by \meta{dimension} along the $x$-axis.
\begin{codeexample}[]
\tikz{\draw[help lines] (-1,-.5) grid (1,.5);
     \pgftext[base,x=1cm,y=-0.5cm] {lovely}}
\end{codeexample}
  \itemoption{y}|=|\meta{dimension}
    works like the |x| option.
  \itemoption{rotate}|=|\meta{degree}
    Rotates the coordinate system by \meta{degree}. This will also
    rotate the text box.
\begin{codeexample}[]
\tikz{\draw[help lines] (-1,-.5) grid (1,.5);
     \pgftext[base,x=1cm,y=-0.5cm,rotate=30] {lovely}}
\end{codeexample}
  \end{itemize}
  
\end{command}



% Copyright 2006 by Till Tantau
%
% This file may be distributed and/or modified
%
% 1. under the LaTeX Project Public License and/or
% 2. under the GNU Free Documentation License.
%
% See the file doc/generic/pgf/licenses/LICENSE for more details.


\section{Specifying Coordinates}

\label{section-points}

\subsection{Overview}

Most \pgfname\ commands expect you to provide the coordinates of a
\emph{point} (also called \emph{coordinate}) inside your
picture. Points are always ``local'' to your picture, that is, they
never refer to an absolute position on the page, but to a position
inside the current |{pgfpicture}| environment. To specify a coordinate
you can use commands that start with |\pgfpoint|.


\subsection{Basic Coordinate Commands}

The following commands are the most basic  for specifying a
coordinate.

\begin{command}{\pgfpoint\marg{x coordinate}\marg{y coordinate}}
  Yields a point location. The coordinates are given as \TeX\
  dimensions.

\begin{codeexample}[]
\begin{tikzpicture}
  \draw[help lines] (0,0) grid (3,2);
  \pgfpathcircle{\pgfpoint{1cm}{1cm}} {2pt}
  \pgfpathcircle{\pgfpoint{2cm}{5pt}} {2pt}
  \pgfpathcircle{\pgfpoint{0pt}{.5in}}{2pt}
  \pgfusepath{fill}
\end{tikzpicture}   
\end{codeexample}
\end{command}

\begin{command}{\pgfpointorigin}
  Yields the origin. Same as |\pgfpoint{0pt}{0pt}|.
\end{command}

\begin{command}{\pgfpointpolar\marg{degree}{\ttfamily\char`\{}\meta{radius}\opt{|/|\meta{y-radius}}{\ttfamily\char`\}}}
  Yields a point location given in polar coordinates. You can specify
  the angle only in degrees, radians are not supported, currently.

  If the optional \meta{y-radius} is given, the polar coordinate is
  actually a coordinate on an ellipse whose $x$-radius is given by
  \meta{radius} and whose $y$-radius is given by \meta{y-radius}.
\begin{codeexample}[]
\begin{tikzpicture}
  \draw[help lines] (0,0) grid (3,2);

  \foreach \angle in {0,10,...,90}
    {\pgfpathcircle{\pgfpointpolar{\angle}{1cm}}{2pt}}
  \pgfusepath{fill}
\end{tikzpicture}
\end{codeexample}
\begin{codeexample}[]
\begin{tikzpicture}
  \draw[help lines] (0,0) grid (3,2);

  \foreach \angle in {0,10,...,90}
    {\pgfpathcircle{\pgfpointpolar{\angle}{1cm/2cm}}{2pt}}
  \pgfusepath{fill}
\end{tikzpicture}   
\end{codeexample}
\end{command}



\subsection{Coordinates in the XY-Coordinate System}

Coordinates can also be specified as multiples of an $x$-vector and a
$y$-vector. Normally, the $x$-vector points one centimeter in the
$x$-direction and the $y$-vector points one centimeter in the
$y$-direction, but using the commands |\pgfsetxvec| and
|\pgfsetyvec| they can be changed. Note that the $x$- and
$y$-vector do not necessarily point ``horizontally'' and
``vertically.''

\begin{command}{\pgfpointxy\marg{$s_x$}\marg{$s_y$}}
  Yields a point that is situated at $s_x$ times the
  $x$-vector plus $s_y$ times the $y$-vector.
\begin{codeexample}[]
\begin{tikzpicture}
  \draw[help lines] (0,0) grid (3,2);
  \pgfpathmoveto{\pgfpointxy{1}{0}}
  \pgfpathlineto{\pgfpointxy{2}{2}}
  \pgfusepath{stroke}
\end{tikzpicture}   
\end{codeexample}
\end{command}


\begin{command}{\pgfsetxvec\marg{point}}
  Sets that current $x$-vector for usage in the $xyz$-coordinate
  system. 
  \example
\begin{codeexample}[]
\begin{tikzpicture}
  \draw[help lines] (0,0) grid (3,2);
  
  \pgfpathmoveto{\pgfpointxy{1}{0}}
  \pgfpathlineto{\pgfpointxy{2}{2}}
  \pgfusepath{stroke}

  \color{red}
  \pgfsetxvec{\pgfpoint{0.75cm}{0cm}}
  \pgfpathmoveto{\pgfpointxy{1}{0}}
  \pgfpathlineto{\pgfpointxy{2}{2}}
  \pgfusepath{stroke}
\end{tikzpicture}   
\end{codeexample}
\end{command}

\begin{command}{\pgfsetyvec\marg{point}}
  Works like |\pgfsetyvec|.
\end{command}



\begin{command}{\pgfpointpolarxy\marg{degree}{\ttfamily\char`\{}\meta{radius}\opt{|/|\meta{y-radius}}{\ttfamily\char`\}}}
  This command is similar to the |\pgfpointpolar| command, but the
  \meta{radius} is now a factor to be interpreted in the
  $xy$-coordinate system. This means that a degree of |0| is the same
  as the $x$-vector of the $xy$-coordinate  system times \meta{radius}
  and a degree of |90| is the $y$-vecotr times \meta{radius}. As for
  |\pgfpointpolar|, a \meta{radius} can also be a pair separated by a
  slash. In this case, the $x$- and $y$-vectors are multiplied by
  different factors.
\begin{codeexample}[]
\begin{tikzpicture}
  \draw[help lines] (0,0) grid (3,2);

  \begin{scope}[x={(1cm,-5mm)},y=1.5cm]
    \foreach \angle in {0,10,...,90}
      {\pgfpathcircle{\pgfpointpolarxy{\angle}{1}}{2pt}}
    \pgfusepath{fill}
  \end{scope}
\end{tikzpicture}
\end{codeexample}
\end{command}



\subsection{Three Dimensional Coordinates}

It is also possible to specify a point as a multiple of three vectors,
the $x$-, $y$-, and $z$-vector. This is useful for creating simple
three dimensional graphics.

\begin{command}{\pgfpointxyz\marg{$s_x$}\marg{$s_y$}\marg{$s_z$}}
  Yields a point that is situated at $s_x$ times the
  $x$-vector plus $s_y$ times the $y$-vector plus  $s_z$ times the
  $z$-vector.
\begin{codeexample}[]
\begin{pgfpicture}
  \pgfsetarrowsend{to}
  
  \pgfpathmoveto{\pgfpointorigin}
  \pgfpathlineto{\pgfpointxyz{0}{0}{1}}
  \pgfusepath{stroke}
  \pgfpathmoveto{\pgfpointorigin}
  \pgfpathlineto{\pgfpointxyz{0}{1}{0}}
  \pgfusepath{stroke}
  \pgfpathmoveto{\pgfpointorigin}
  \pgfpathlineto{\pgfpointxyz{1}{0}{0}}
  \pgfusepath{stroke}
\end{pgfpicture}
\end{codeexample}
\end{command}

\begin{command}{\pgfsetzvec\marg{point}}
  Works like |\pgfsetzvec|.
\end{command}

Inside the $xyz$-coordinate system, you can also specify points
using spherical and cylindrical coordinates.


\begin{command}{\pgfpointcylindrical\marg{degree}\marg{radius}\marg{height}}
  This command yields the same as
\begin{verbatim}
\pgfpointadd{\pgfpointpolarxy{degree}{radius}}{\pgfpointxyz{0}{0}{height}}
\end{verbatim}
\begin{codeexample}[]
\begin{tikzpicture}
  \draw [->] (0,0) -- (1,0,0) node [right] {$x$};
  \draw [->] (0,0) -- (0,1,0) node [above] {$y$};
  \draw [->] (0,0) -- (0,0,1) node [below left] {$z$};

  \pgfpathcircle{\pgfpointcylindrical{80}{1}{.5}}{2pt}
  \pgfusepath{fill}

  \draw[red] (0,0) -- (0,0,.5) -- +(80:1);
\end{tikzpicture}
\end{codeexample}
\end{command}

\begin{command}{\pgfpointspherical\marg{longitude}\marg{latitude}\marg{radius}}
  This command yields a point ``on the surface of the earth''
  specified by the \meta{longitude} and the \marg{latitude}. The
  radius of the earth is given by \meta{radius}. The equator lies in
  the $xy$-plane.
\begin{codeexample}[]
\begin{tikzpicture}
  \pgfsetfillcolor{lightgray}

  \foreach \latitude in {-90,-75,...,30}
  {  
    \foreach \longitude in {0,20,...,360}
    {
      \pgfpathmoveto{\pgfpointspherical{\longitude}{\latitude}{1}}
      \pgfpathlineto{\pgfpointspherical{\longitude+20}{\latitude}{1}}
      \pgfpathlineto{\pgfpointspherical{\longitude+20}{\latitude+15}{1}}
      \pgfpathlineto{\pgfpointspherical{\longitude}{\latitude+15}{1}}
      \pgfpathclose
    }
    \pgfusepath{fill,stroke}
  }
\end{tikzpicture}
\end{codeexample}
\end{command}



\subsection{Building Coordinates From Other Coordinates}

Many commands allow you to construct a coordinate in terms of other
coordinates.


\subsubsection{Basic Manipulations of Coordinates}

\begin{command}{\pgfpointadd\marg{$v_1$}\marg{$v_2$}}
  Returns the sum vector $\meta{$v_1$} + \meta{$v_2$}$.
\begin{codeexample}[]
\begin{tikzpicture}
  \draw[help lines] (0,0) grid (3,2);
  \pgfpathcircle{\pgfpointadd{\pgfpoint{1cm}{0cm}}{\pgfpoint{1cm}{1cm}}}{2pt}
  \pgfusepath{fill} 
\end{tikzpicture}
\end{codeexample}
\end{command}

\begin{command}{\pgfpointscale\marg{factor}\marg{coordinate}}
  Returns the vector $\meta{factor}\meta{coordinate}$.
\begin{codeexample}[]
\begin{tikzpicture}
  \draw[help lines] (0,0) grid (3,2);
  \pgfpathcircle{\pgfpointscale{1.5}{\pgfpoint{1cm}{0cm}}}{2pt}
  \pgfusepath{fill} 
\end{tikzpicture}
\end{codeexample}
\end{command}

\begin{command}{\pgfpointdiff\marg{start}\marg{end}}
  Returns the difference vector $\meta{end} - \meta{start}$.
\begin{codeexample}[]
\begin{tikzpicture}
  \draw[help lines] (0,0) grid (3,2);
  \pgfpathcircle{\pgfpointdiff{\pgfpoint{1cm}{0cm}}{\pgfpoint{1cm}{1cm}}}{2pt}
  \pgfusepath{fill} 
\end{tikzpicture}
\end{codeexample}
\end{command}


\begin{command}{\pgfpointnormalised\marg{point}}
  This command returns a normalized version of \meta{point}, that is,
  a vector of length 1pt pointing in the direction of \meta{point}. If
  \meta{point} is the $0$-vector or extremely short, a vector of
  length 1pt pointing upwards is returned.

  This command is \emph{not} implemented by calculating the length of
  the vector, but rather by calculating the angle of the vector and
  then using (something equivalent to) the |\pgfpointpolar|
  command. This ensures that the point will really have length 1pt,
  but it is not guaranteed that the vector will \emph{precisely} point
  in the direction of \meta{point} due to the fact that the polar
  tables are accurate only up to one degree. Normally, this is not a
  problem.
\begin{codeexample}[]
\begin{tikzpicture}
  \draw[help lines] (0,0) grid (3,2);
  \pgfpathcircle{\pgfpoint{2cm}{1cm}}{2pt}
  \pgfpathcircle{\pgfpointscale{20}
    {\pgfpointnormalised{\pgfpoint{2cm}{1cm}}}}{2pt}
  \pgfusepath{fill} 
\end{tikzpicture}
\end{codeexample}  
\end{command}


\subsubsection{Points Traveling along Lines and Curves}

\label{section-pointsattime}

The commands in this section allow you to specify points on a line or
a curve. Imaging a point ``traveling'' along a curve from some point
$p$ to another point $q$. At time $t=0$ the point is at $p$ and at
time $t=1$ it is at $q$ and at time, say, $t=1/2$ it is ``somewhere in
the middle.'' The exact location at time $t=1/2$ will not necessarily
be the ``halfway point,'' that is, the point whose distance on the
curve from $p$ and $q$ is equal. Rather, the exact location will
depend on the ``speed'' at which the point is traveling, which in
turn depends on the lengths of the support vectors in a complicated
manner. If you are interested in the details, please see a good book
on B�zier curves.



\begin{command}{\pgfpointlineattime\marg{time $t$}\marg{point $p$}\marg{point $q$}}
  Yields a point that is the $t$th fraction between $p$
  and~$q$, that is, $p + t(q-p)$. For $t=1/2$ this is the middle of
  $p$ and $q$.

\begin{codeexample}[]
\begin{tikzpicture}
  \draw[help lines] (0,0) grid (3,2);
  \pgfpathmoveto{\pgfpointorigin}
  \pgfpathlineto{\pgfpoint{2cm}{2cm}}
  \pgfusepath{stroke}
  \foreach \t in {0,0.25,...,1.25}
    {\pgftext[at=
      \pgfpointlineattime{\t}{\pgfpointorigin}{\pgfpoint{2cm}{2cm}}]{\t}}
\end{tikzpicture}    
\end{codeexample}
\end{command}

\begin{command}{\pgfpointlineatdistance\marg{distance}\marg{start point}\marg{end point}}
  Yields a point that is located \meta{distance} many units removed
  from the start point in the direction of the end point. In other
  words, this is the point that results if we travel \meta{distance}
  steps from \meta{start point} towards \meta{end point}.
  \example
\begin{codeexample}[]
\begin{tikzpicture}
  \draw[help lines] (0,0) grid (3,2);
  \pgfpathmoveto{\pgfpointorigin}
  \pgfpathlineto{\pgfpoint{3cm}{2cm}}
  \pgfusepath{stroke}
  \foreach \d in {0pt,20pt,40pt,70pt}
    {\pgftext[at=
      \pgfpointlineatdistance{\d}{\pgfpointorigin}{\pgfpoint{3cm}{2cm}}]{\d}}
\end{tikzpicture}    
\end{codeexample}
\end{command}

\begin{command}{\pgfpointcurveattime\marg{time $t$}\marg{point
      $p$}\marg{point $s_1$}\marg{point $s_2$}\marg{point $q$}} 
  Yields a point that is on the B�zier curve from $p$ to $q$ with the
  support points $s_1$ and $s_2$. The time $t$ is used to determine
  the location, where $t=0$ yields $p$ and $t=1$ yields $q$.

\begin{codeexample}[]
\begin{tikzpicture}
  \draw[help lines] (0,0) grid (3,2);
  \pgfpathmoveto{\pgfpointorigin}
  \pgfpathcurveto
    {\pgfpoint{0cm}{2cm}}{\pgfpoint{0cm}{2cm}}{\pgfpoint{3cm}{2cm}}
  \pgfusepath{stroke}
  \foreach \t in {0,0.25,0.5,0.75,1}
    {\pgftext[at=\pgfpointcurveattime{\t}{\pgfpointorigin}
                                         {\pgfpoint{0cm}{2cm}}
                                         {\pgfpoint{0cm}{2cm}}
                                         {\pgfpoint{3cm}{2cm}}]{\t}}
\end{tikzpicture}    
\end{codeexample}
\end{command}

\subsubsection{Points on Borders of Objects}

The following commands are useful for specifying a point that lies on
the border of special shapes. They are used, for example, by the shape
mechanism to determine border points of shapes.

\begin{command}{\pgfpointborderrectangle\marg{direction point}\marg{corner}}
  This command returns a point that lies on the intersection of a line
  starting at the origin and going towards the point \meta{direction
    point} and a rectangle whose center is in the origin and whose
  upper right corner is at \meta{corner}.

  The \meta{direction point} should have length ``about 1pt,'' but it
  will be normalized automatically. Nevertheless, the ``nearer'' the
  length is to 1pt, the less rounding errors.

\begin{codeexample}[]
\begin{tikzpicture}
  \draw[help lines] (0,0) grid (2,1.5);
  \pgfpathrectanglecorners{\pgfpoint{-1cm}{-1.25cm}}{\pgfpoint{1cm}{1.25cm}}
  \pgfusepath{stroke}

  \pgfpathcircle{\pgfpoint{5pt}{5pt}}{2pt}
  \pgfpathcircle{\pgfpoint{-10pt}{5pt}}{2pt}
  \pgfusepath{fill}
  \color{red}
  \pgfpathcircle{\pgfpointborderrectangle
    {\pgfpoint{5pt}{5pt}}{\pgfpoint{1cm}{1.25cm}}}{2pt}
  \pgfpathcircle{\pgfpointborderrectangle
    {\pgfpoint{-10pt}{5pt}}{\pgfpoint{1cm}{1.25cm}}}{2pt}
  \pgfusepath{fill}
\end{tikzpicture}    
\end{codeexample}
\end{command}


\begin{command}{\pgfpointborderellipse\marg{direction point}\marg{corner}}
  This command works like the corresponding command for rectangles,
  only this time the \meta{corner} is the corner of the bounding
  rectangle of an ellipse.

\begin{codeexample}[]
\begin{tikzpicture}
  \draw[help lines] (0,0) grid (2,1.5);
  \pgfpathellipse{\pgfpointorigin}{\pgfpoint{1cm}{0cm}}{\pgfpoint{0cm}{1.25cm}}
  \pgfusepath{stroke}

  \pgfpathcircle{\pgfpoint{5pt}{5pt}}{2pt}
  \pgfpathcircle{\pgfpoint{-10pt}{5pt}}{2pt}
  \pgfusepath{fill}
  \color{red}
  \pgfpathcircle{\pgfpointborderellipse
    {\pgfpoint{5pt}{5pt}}{\pgfpoint{1cm}{1.25cm}}}{2pt}
  \pgfpathcircle{\pgfpointborderellipse
    {\pgfpoint{-10pt}{5pt}}{\pgfpoint{1cm}{1.25cm}}}{2pt}
  \pgfusepath{fill}
\end{tikzpicture}    
\end{codeexample}
\end{command}


\subsubsection{Points on the Intersection of Lines}


\begin{command}{\pgfpointintersectionoflines\marg{$p$}\marg{$q$}\marg{$s$}\marg{$t$}}
  This command returns the intersection of a line going through $p$
  and $q$ and a line going through $s$ and $t$. If the lines do not
  intersection, an arithmetic overflow will occur.

\begin{codeexample}[]
\begin{tikzpicture}
  \draw[help lines] (0,0) grid (2,2);
  \draw (.5,0) -- (2,2);
  \draw (1,2) -- (2,0);
  \pgfpathcircle{%
    \pgfpointintersectionoflines
      {\pgfpointxy{.5}{0}}{\pgfpointxy{2}{2}}
      {\pgfpointxy{1}{2}}{\pgfpointxy{2}{0}}}
    {2pt}
  \pgfusepath{stroke}
\end{tikzpicture}    
\end{codeexample}
\end{command}

\subsection{Extracting Coordinates}

There are two commands that can be used to ``extract'' the $x$- or
$y$-coordinate of a coordinate. 

\begin{command}{\pgfextractx\marg{dimension}\marg{point}}
  Sets the \TeX-\meta{dimension} to the $x$-coordinate of the point.

\begin{codeexample}[code only]
\newdimen\mydim
\pgfextractx{\mydim}{\pgfpoint{2cm}{4pt}}
%% \mydim is now 2cm
\end{codeexample}
\end{command}

\begin{command}{\pgfextracty\marg{dimension}\marg{point}}
  Like |\pgfextractx|, except for the $y$-coordinate.
\end{command}




\subsection{Internals of How Point Commands Work}

As a normal user of \pgfname\ you do not need to read this section. It
is relevant only if you need to understand how the point commands work
internally. 

When a command like |\pgfpoint{1cm}{2pt}| is called, all that happens
is that the two \TeX-dimension variables |\pgf@x| and |\pgf@y| are set
to |1cm| and |2pt|, respectively. A command like |\pgfpathmoveto| that
takes a coordinate as parameter will just execute this parameter and
then use the values of |\pgf@x| and |\pgf@y| as the coordinates to
which it will move the pen on the current path.

since commands like |\pgfpointnormalised| modify other variables
besides |\pgf@x| and |\pgf@y| during the computation of the final values of
|\pgf@x| and |\pgf@y|, it is a good idea to enclose a call of a
command like |\pgfpoint| in a \TeX-scope and then make the changes of
|\pgf@x| and |\pgf@y| global as in the following example:
\begin{codeexample}[code only]
...
{ % open scope
  \pgfpointnormalised{\pgfpoint{1cm}{1cm}}
  \global\pgf@x=\pgf@x % make the change of \pgf@x persist past the scope
  \global\pgf@y=\pgf@y % make the change of \pgf@y persist past the scope
}
% \pgf@x and \pgf@y are now set correctly, all other variables are
% unchanged 
\end{codeexample}

\makeatletter
Since this situation arises very often, the macro |\pgf@process| can
be used to perform the above code:
\begin{command}{\pgf@process\marg{code}}
  Executes the \meta{code} in a scope and then makes |\pgf@x| and
  |\pgf@y| global.
\end{command}

Note that this macro is used often internally. For this reason, it is
not a good idea to keep anything important in the variables |\pgf@x|
and |\pgf@y| since they will be overwritten and changed
frequently. Instead, intermediate values can ge stored in the
\TeX-dimensions |\pgf@xa|, |\pgf@xb|, |\pgf@xc| and their
|y|-counterparts |\pgf@ya|, |\pgf@yb|, |pgf@yc|. For example, here is
the code of the command |\pgfpointadd|:
\begin{codeexample}[code only]
\def\pgfpointadd#1#2{%
  \pgf@process{#1}%
  \pgf@xa=\pgf@x%
  \pgf@ya=\pgf@y%
  \pgf@process{#2}%
  \advance\pgf@x by\pgf@xa%
  \advance\pgf@y by\pgf@ya}
\end{codeexample}



%%% Local Variables: 
%%% mode: latex
%%% TeX-master: "pgfmanual"
%%% End: 

% Copyright 2006 by Till Tantau
%
% This file may be distributed and/or modified
%
% 1. under the LaTeX Project Public License and/or
% 2. under the GNU Free Documentation License.
%
% See the file doc/generic/pgf/licenses/LICENSE for more details.


\section{Constructing Paths}

\subsection{Overview}

The ``basic entity of drawing'' in \pgfname\ is the \emph{path}. A
path consists of several parts, each of which is either a closed or
open curve. An open curve has a starting point and an end point and,
in between, consists of several \emph{segments}, each of which is
either a straight line or a B�zier curve. Here is an example of a
path (in red) consisting of two parts, one open, one closed:

\begin{codeexample}[]
\begin{tikzpicture}[scale=2]
  \draw[thick,red]
       (0,0) coordinate (a)
    -- coordinate (ab) (1,.5) coordinate (b)
    .. coordinate (bc) controls +(up:1cm) and +(left:1cm) .. (3,1)  coordinate (c)
       (0,1) -- (2,1) -- coordinate (x) (1,2) -- cycle;

  \draw (a)  node[below] {start part 1}
        (ab) node[below right] {straight segment}
        (b)  node[right] {end first segment}
        (c)  node[right] {end part 1}
        (x)  node[above right]  {part 2 (closed)};        
\end{tikzpicture}
\end{codeexample}

A path, by itself, has no ``effect,'' that is, it does not leave any
marks on the page. It is just a set of points on the plane. However,
you can \emph{use} a path in different ways. The most natural actions
are \emph{stroking} (also known as \emph{drawing}) and
\emph{filling}. Stroking can be imagined as picking up a pen of a
certain diameter and ``moving it along the path.'' Filling means that
everything ``inside'' the path is filled with a uniform
color. Naturally, the open parts of a path must first be closed before
a path can be filled.

In \pgfname, there are numerous commands for constructing paths, all
of which start with |\pgfpath|. There are also commands for
\emph{using} paths, though most operations can be performed by calling
|\pgfusepath| with an appropriate parameter.

As a side-effect, the path construction commands keep track of two
bounding boxes. One is the bounding box for the current path, the
other is a bounding box for all paths in the current picture. See
Section~\ref{section-bb} for more details.

Each path construction command extends the current path in some
way. The ``current path'' is a global entity that persists across
\TeX\ groups. Thus, between calls to the path construction commands
you can perform arbitrary computations and even open and closed \TeX\
groups. The current path only gets ``flushed'' when the |\pgfusepath|
command is called (or when the soft-path subsystem is used directly,
see Section~\ref{section-soft-paths}).

\subsection{The Move-To Path Operation}

The most basic operation is the move-to operation. It must be given at
the beginning of paths, though some path construction command (like
|\pgfpathrectangle|) generate move-tos implicitly. A move-to operation
can also be used to start a new part of a path. 

\begin{command}{\pgfpathmoveto\marg{coordinate}}
  This command expects a \pgfname-coordinate like |\pgfpointorigin| as
  its parameter. When the current path is empty, this operation will
  start the path at the given \meta{coordinate}. If a path has already
  been partly constructed, this command will end the current part of
  the path and start a new one.
\begin{codeexample}[]
\begin{pgfpicture}
  \pgfpathmoveto{\pgfpointorigin}
  \pgfpathlineto{\pgfpoint{1cm}{1cm}}
  \pgfpathlineto{\pgfpoint{2cm}{1cm}}
  \pgfpathlineto{\pgfpoint{3cm}{0.5cm}}
  \pgfpathlineto{\pgfpoint{3cm}{0cm}}
  \pgfsetfillcolor{examplefill}
  \pgfusepath{fill,stroke}
\end{pgfpicture}
\end{codeexample}
\begin{codeexample}[]
\begin{pgfpicture}
  \pgfpathmoveto{\pgfpointorigin}
  \pgfpathlineto{\pgfpoint{1cm}{1cm}}
  \pgfpathlineto{\pgfpoint{2cm}{1cm}}
  \pgfpathmoveto{\pgfpoint{2cm}{1cm}} % New part
  \pgfpathlineto{\pgfpoint{3cm}{0.5cm}}
  \pgfpathlineto{\pgfpoint{3cm}{0cm}}
  \pgfsetfillcolor{examplefill}
  \pgfusepath{fill,stroke}
\end{pgfpicture}
\end{codeexample}
  The command will apply the current coordinate transformation matrix
  to \meta{coordinate} before using it.

  The command will update the bounding box of the current path and
  picture, if necessary. 
\end{command}


\subsection{The Line-To Path Operation}

\begin{command}{\pgfpathlineto\marg{coordinate}}
  This command extends the current path in a straight line to the
  given \meta{coordinate}. If this command is given at the beginning
  of path without any other path construction command given before (in
  particular without a move-to operation), the \TeX\ file may compile
  without an error message, but a viewer application may display an
  error message when trying to render the picture. 
\begin{codeexample}[]
\begin{pgfpicture}
  \pgfpathmoveto{\pgfpointorigin}
  \pgfpathlineto{\pgfpoint{1cm}{1cm}}
  \pgfpathlineto{\pgfpoint{2cm}{1cm}}
  \pgfsetfillcolor{examplefill}
  \pgfusepath{fill,stroke}
\end{pgfpicture}
\end{codeexample}
  The command will apply the current coordinate transformation matrix
  to \meta{coordinate} before using it.

  The command will update the bounding box of the current path and
  picture, if necessary. 
\end{command}


\subsection{The Curve-To Path Operations}

\begin{command}{\pgfpathcurveto\marg{support 1}\marg{support 2}\marg{coordinate}}
  This command extends the current path with a B�zier curve from the
  last point of the path to  \meta{coordinate}. The \meta{support 1}
  and \meta{support 2} are the first and second support point of the
  B�zier curve. For more information on B�zier curve, please consult a
  standard textbook on computer graphics.

  Like the line-to command, this command may not be the first path
  construction command in a path.
\begin{codeexample}[]
\begin{pgfpicture}
  \pgfpathmoveto{\pgfpointorigin}
  \pgfpathcurveto
    {\pgfpoint{1cm}{1cm}}{\pgfpoint{2cm}{1cm}}{\pgfpoint{3cm}{0cm}}
  \pgfsetfillcolor{examplefill}
  \pgfusepath{fill,stroke}
\end{pgfpicture}
\end{codeexample}
  The command will apply the current coordinate transformation matrix
  to \meta{coordinate} before using it.

  The command will update the bounding box of the current path and
  picture, if necessary. However, the bounding box is simply made
  large enough such that it encompasses all of the support points and
  the \meta{coordinate}. This will guarantee that the curve is
  completely inside the bounding box, but the bounding box will
  typically be quite a bit too large. It is not clear (to me) how this 
  can be avoided without resorting to ``some serious math'' in order
  to calculate a precise bounding box. 
\end{command}

\begin{command}{\pgfpathquadraticcurveto\marg{support}\marg{coordinate}}
  This command works like |\pgfpathcurveto|, only it uses a quadratic
  B�zier curve rather than a cubic one. This means that only one
  support point is needed. 
\begin{codeexample}[]
\begin{pgfpicture}
  \pgfpathmoveto{\pgfpointorigin}
  \pgfpathquadraticcurveto
    {\pgfpoint{1cm}{1cm}}{\pgfpoint{2cm}{0cm}}
  \pgfsetfillcolor{examplefill}
  \pgfusepath{fill,stroke}
\end{pgfpicture}
\end{codeexample}
  Internally, the quadratic curve is converted into a cubic
  curve. The only noticeable effect of this is that the points used for
  computing the bounding box are the control points of the converted
  curve rather than \meta{support}. The main effect of this is that
  the bounding box will be a bit tighter than might be expected. In
  particular, \meta{support} will not always be part of the bounding
  box. 
\end{command}


There exist two commands to draw only part of a cubic B�zier curve:

\begin{command}{\pgfpathcurvebetweentime\marg{time $t_1$}\marg{time $t_2$}\marg{point p}\marg{point $s_1$}\marg{point $s_2$}\marg{point q}}

  This command draws the part of the curve described by $p$, $s_1$, 
  $s_2$ and $q$ between the times $t_1$ and $t_2$. A time value of 0 
  indicates the point $p$ and a time vaue of 1 indicates point $q$. 
  This command includes a moveto operation to the first point.

\begin{codeexample}[]
\begin{tikzpicture}
  \draw [thin] (0,0) .. controls (0,2) and (3,0) .. (3,2);
  \pgfpathcurvebetweentime{0.25}{0.9}{\pgfpointxy{0}{0}}{\pgfpointxy{0}{2}}
    {\pgfpointxy{3}{0}}{\pgfpointxy{3}{2}}
  \pgfsetstrokecolor{red}
  \pgfsetstrokeopacity{0.5}
  \pgfsetlinewidth{2pt}
  \pgfusepath{stroke}
\end{tikzpicture}
\end{codeexample}
\end{command}

\begin{command}{\pgfpathcurvebetweentimecontinue\marg{time $t_1$}\marg{time $t_2$}\marg{point p}\marg{point $s_1$}\marg{point $s_2$}\marg{point q}}
  This command works like |\pgfpathcurvebetweentime|, except that a 
  moveto operation is \emph{not} made to the first point.
\end{command}


\subsection{The Close Path Operation}

\begin{command}{\pgfpathclose}
  This command closes the current part of the path by appending a
  straight line to the start point of the current part. Note that there
  \emph{is} a difference between closing a path and using the line-to
  operation to add a straight line to the start of the current
  path. The difference is demonstrated by the upper corners of the triangles
  in the following example: 
\begin{codeexample}[]
\begin{tikzpicture}
  \draw[help lines] (0,0) grid (3,2);
  \pgfsetlinewidth{5pt}
  \pgfpathmoveto{\pgfpoint{1cm}{1cm}}
  \pgfpathlineto{\pgfpoint{0cm}{-1cm}}
  \pgfpathlineto{\pgfpoint{1cm}{-1cm}}
  \pgfpathclose
  \pgfpathmoveto{\pgfpoint{2.5cm}{1cm}}
  \pgfpathlineto{\pgfpoint{1.5cm}{-1cm}}
  \pgfpathlineto{\pgfpoint{2.5cm}{-1cm}}
  \pgfpathlineto{\pgfpoint{2.5cm}{1cm}}
  \pgfusepath{stroke}
\end{tikzpicture}
\end{codeexample}
\end{command}


\subsection{Arc, Ellipse and Circle Path Operations}

The path construction commands that we have discussed up to now are
sufficient to create all paths that can be created ``at all.''
However, it is useful to have special commands to create certain
shapes, like circles, that arise often in practice.

In the following, the commands for adding (parts of) (transformed)
circles to a path are described.

\begin{command}{\pgfpatharc\marg{start angle}\marg{end
      angle}{\ttfamily\char`\{}\meta{radius}\opt{| and |\meta{y-radius}}{\ttfamily\char`\}}}
  This command appends a part of a circle (or an ellipse) to the current
  path. Imaging the curve between \meta{start angle} and \meta{end
    angle} on a circle of radius \meta{radius} (if $\meta{start angle}
  < \meta{end angle}$, the curve goes around the circle
  counterclockwise, otherwise clockwise). This curve is now moved such
  that the point where the curve starts is the previous last point of the
  path. Note that this command will \emph{not} start a new part of the
  path, which is important for example for filling purposes. 

\begin{codeexample}[]
\begin{tikzpicture}
  \draw[help lines] (0,0) grid (3,2);
  \pgfpathmoveto{\pgfpointorigin}
  \pgfpathlineto{\pgfpoint{0cm}{1cm}}
  \pgfpatharc{180}{90}{.5cm}
  \pgfpathlineto{\pgfpoint{3cm}{1.5cm}}
  \pgfpatharc{90}{-45}{.5cm}
  \pgfusepath{fill}
\end{tikzpicture}
\end{codeexample}

  Saying |\pgfpatharc{0}{360}{1cm}| ``nearly'' gives you a full
  circle. The ``nearly'' refers to the fact that the circle will not
  be closed. You can close it using |\pgfpathclose|.

  If the optional \meta{y-radius} is given, the \meta{radius} is the
  $x$-radius and the \meta{y-radius} the $y$-radius of the ellipse
  from which the curve is taken:

\begin{codeexample}[]
\begin{tikzpicture}
  \draw[help lines] (0,0) grid (3,2);
  \pgfpathmoveto{\pgfpointorigin}
  \pgfpatharc{180}{45}{2cm and 1cm}
  \pgfusepath{draw}
\end{tikzpicture}
\end{codeexample}

  The axes of the circle or ellipse from which the arc is ``taken''
  always point up and right. However, the current coordinate
  transformation matrix will have an effect on the arc. This can be
  used to, say, rotate an arc:

\begin{codeexample}[]
\begin{tikzpicture}
  \draw[help lines] (0,0) grid (3,2);
  \pgftransformrotate{30}
  \pgfpathmoveto{\pgfpointorigin}
  \pgfpatharc{180}{45}{2cm and 1cm}
  \pgfusepath{draw}
\end{tikzpicture}
\end{codeexample}

  The command will update the bounding box of the current path and
  picture, if necessary. Unless rotation or shearing transformations
  are applied, the bounding box will be tight.
\end{command}

\begin{command}{\pgfpatharcaxes\marg{start angle}\marg{end
      angle}\marg{first axis}\marg{second axis}}
  This command is similar to |\pgfpatharc|. The main difference is how
  the ellipse or circle is specified from which the arc is taken. The
  two parameters \meta{first axis} and \meta{second axis} are the
  $0^\circ$-axis and the $90^\circ$-axis of the ellipse from which the
  path is taken. Thus, |\pgfpatharc{0}{90}{1cm and 2cm}| has the same effect
  as
\begin{verbatim}
\pgfpatharcaxes{0}{90}{\pgfpoint{1cm}{0cm}}{\pgfpoint{0cm}{2cm}}
\end{verbatim}
\begin{codeexample}[]
\begin{tikzpicture}
  \draw[help lines] (0,0) grid (3,2);
  \draw (0,0) -- (2cm,5mm) (0,0) -- (0cm,1cm);
  
  \pgfpathmoveto{\pgfpoint{2cm}{5mm}}
  \pgfpatharcaxes{0}{90}{\pgfpoint{2cm}{5mm}}{\pgfpoint{0cm}{1cm}}
  \pgfusepath{draw}
\end{tikzpicture}
\end{codeexample}
\end{command}  


\begin{command}{\pgfpatharcto\marg{x-radius}\marg{y-radius}\marg{rotation}
    \marg{large arc flag}\marg{counterclockwise flag}\\\marg{target point}} 
  This command (which directly corresponds to the arc-path command of
  \textsc{svg}) is used to add an arc to the path that starts at the
  current point and ends at \meta{target point}. This arc is part of
  an ellipse that is determined in the following way: Imagine an
  ellipse with radii \meta{x-radius} and \meta{y-radius} that is
  rotated around its center by \meta{rotation} degrees. When you move
  this ellipse around in the plane, there will be exactly two
  positions such that the two current point and the target point lie
  on the border of the ellipse (excluding pathological cases). The
  flags \meta{large arc flag} and \meta{clockwise flag} are then used to
  decide which of these ellipses should be picked and which arc on the
  picked ellipsis should be used.
\begin{codeexample}[]
\begin{tikzpicture}
  \draw[help lines] (0,0) grid (3,2);
  
  \pgfpathmoveto{\pgfpoint{0mm}{20mm}}
  \pgfpatharcto{3cm}{1cm}{0}{0}{0}{\pgfpoint{3cm}{1cm}}
  \pgfusepath{draw}
\end{tikzpicture}
\end{codeexample}
  Both flags are considered to be false exactly if they evaluate to
  |0|, otherwise they are true. If the \meta{large arc flag} is true,
  then the angle spanned by the arc will be greater than $180^\circ$,
  otherwise it will be less than $180^\circ$. The \meta{clockwise
    flag} is used to determine which of the two ellipses should be
  used: if the flag is true, then the arc goes from the current point
  to the target point in a counterclockwise direction, otherwise in a
  clockwise fashion.
\begin{codeexample}[]
\begin{tikzpicture}
  \pgfsetlinewidth{2pt}
  % Flags 0 0: red
  \pgfsetstrokecolor{red}
  \pgfpathmoveto{\pgfpointorigin}
  \pgfpatharcto{20pt}{10pt}{0}{0}{0}{\pgfpoint{20pt}{10pt}}
  \pgfusepath{stroke}
  % Flags 0 1: blue
  \pgfsetstrokecolor{blue}
  \pgfpathmoveto{\pgfpointorigin}
  \pgfpatharcto{20pt}{10pt}{0}{0}{1}{\pgfpoint{20pt}{10pt}}
  \pgfusepath{stroke}
  % Flags 1 0: orange
  \pgfsetstrokecolor{orange}
  \pgfpathmoveto{\pgfpointorigin}
  \pgfpatharcto{20pt}{10pt}{0}{1}{0}{\pgfpoint{20pt}{10pt}}
  \pgfusepath{stroke}
  % Flags 1 1: black
  \pgfsetstrokecolor{black}
  \pgfpathmoveto{\pgfpointorigin}
  \pgfpatharcto{20pt}{10pt}{0}{1}{1}{\pgfpoint{20pt}{10pt}}
  \pgfusepath{stroke}
\end{tikzpicture}
\end{codeexample}
  \emph{Warning:} The internal computations necessary for this command
  are numerically very unstable. In particular, the arc will not
  always really end at the \meta{target coordinate}, but may be off by
  up to several points. A more precise positioning is currently
  infeasible due to \TeX's numerical weaknesses. The only case that
  works quite nicely is when the resulting angle is a multiple
  of~$90^\circ$.  
\end{command}

\begin{command}{\pgfpathellipse\marg{center}\marg{first
      axis}\marg{second axis}}
  The effect of this command is to append an ellipse to the current
  path (if the path is not empty, a new part is started). The
  ellipse's center will be \meta{center} and \meta{first axis} and
  \meta{second axis} are the axis \emph{vectors}. The same effect as
  this command can also be achieved using an appropriate sequence of
  move-to, arc, and close operations, but this command is easier and
  faster. 

\begin{codeexample}[]
\begin{tikzpicture}
  \draw[help lines] (0,0) grid (3,2);
  \pgfpathellipse{\pgfpoint{1cm}{0cm}}
                 {\pgfpoint{1.5cm}{0cm}}
                 {\pgfpoint{0cm}{1cm}}
  \pgfusepath{draw}
  \color{red}               
  \pgfpathellipse{\pgfpoint{1cm}{0cm}}
                 {\pgfpoint{1cm}{1cm}}
                 {\pgfpoint{-0.5cm}{0.5cm}}
  \pgfusepath{draw}
\end{tikzpicture}
\end{codeexample}

  The command will apply coordinate transformations to all coordinates
  of the ellipse. However, the coordinate transformations are applied
  only after the ellipse is ``finished conceptually.'' Thus, a
  transformation of 1cm to the right will simply shift the ellipse one
  centimeter to the right; it will not add 1cm to the $x$-coordinates
  of the two axis vectors.

  The command will update the bounding box of the current path and
  picture, if necessary. 
\end{command}

\begin{command}{\pgfpathcircle\marg{center}\marg{radius}}
  A shorthand for |\pgfpathellipse| applied to \meta{center} and the
  two axis vectors $(\meta{radius},0)$ and $(0,\meta{radius})$. 
\end{command}


\subsection{Rectangle Path Operations}

Another shape that arises frequently is the rectangle. Two commands
can be used to add a rectangle to the current path. Both commands will
start a new part of the path.


\begin{command}{\pgfpathrectangle\marg{corner}\marg{diagonal vector}}
  Adds a rectangle to the path whose one corner is \meta{corner} and
  whose opposite corner is given by $\meta{corner} + \meta{diagonal
    vector}$.

\begin{codeexample}[]
\begin{tikzpicture}
  \draw[help lines] (0,0) grid (3,2);
  \pgfpathrectangle{\pgfpoint{1cm}{0cm}}{\pgfpoint{1.5cm}{1cm}}
  \pgfpathrectangle{\pgfpoint{1.5cm}{0.25cm}}{\pgfpoint{1.5cm}{1cm}}
  \pgfpathrectangle{\pgfpoint{2cm}{0.5cm}}{\pgfpoint{1.5cm}{1cm}}
  \pgfusepath{draw}
\end{tikzpicture}
\end{codeexample}
  The command will apply coordinate transformations and update the
  bounding boxes tightly.
\end{command}


\begin{command}{\pgfpathrectanglecorners\marg{corner}\marg{opposite corner}}
  Adds a rectangle to the path whose two opposing corners are
  \meta{corner} and \meta{opposite corner}.
\begin{codeexample}[]
\begin{tikzpicture}
  \draw[help lines] (0,0) grid (3,2);
  \pgfpathrectanglecorners{\pgfpoint{1cm}{0cm}}{\pgfpoint{1.5cm}{1cm}}
  \pgfusepath{draw}
\end{tikzpicture}
\end{codeexample}
  The command will apply coordinate transformations and update the
  bounding boxes tightly.
\end{command}



\subsection{The Grid Path Operation}

\begin{command}{\pgfpathgrid\oarg{options}\marg{lower left}\marg{upper right}}
  Appends a grid to the current path. That is, a (possibly large)
  number of parts are added to the path, each part consisting of a
  single horizontal or vertical straight line segment.

  Conceptually, the origin is part of the grid and the grid is clipped 
  to the rectangle specified by the \meta{lower left} and
  the \meta{upper right} corner. However, no clipping occurs (this
  command just adds parts to the current path). Rather, the points
  where the lines enter and leave the ``clipping area'' are computed
  and used to add simple lines to the current path.

  The following keys influence the grid:
  \begin{key}{/pgf/stepx=\meta{dimension} (initially 1cm)}
    The horizontal stepping.
  \end{key}
  \begin{key}{/pgf/stepy=\meta{dimension} (initially 1cm)}
    The vertical stepping.
  \end{key}
  \begin{key}{/pgf/step=\meta{vector}}
    Sets the horizontal stepping to the $x$-coordinate of
    \meta{vector} and the vertical stepping to its $y$-coordinate.
  \end{key}
\begin{codeexample}[]
\begin{pgfpicture}
  \pgfsetlinewidth{0.8pt}
  \pgfpathgrid[step={\pgfpoint{1cm}{1cm}}]
    {\pgfpoint{-3mm}{-3mm}}{\pgfpoint{33mm}{23mm}}
  \pgfusepath{stroke}
  \pgfsetlinewidth{0.4pt}
  \pgfpathgrid[stepx=1mm,stepy=1mm]
    {\pgfpoint{-1.5mm}{-1.5mm}}{\pgfpoint{31.5mm}{21.5mm}}
  \pgfusepath{stroke}
\end{pgfpicture}
\end{codeexample}
  The command will apply coordinate transformations and update the
  bounding boxes tightly. As for ellipses, the transformations are
  applied to the ``conceptually finished'' grid. 
\begin{codeexample}[]
\begin{pgfpicture}
  \pgftransformrotate{10}
  \pgfpathgrid[stepx=1mm,stepy=2mm]{\pgfpoint{0mm}{0mm}}{\pgfpoint{30mm}{30mm}}
  \pgfusepath{stroke}
\end{pgfpicture}
\end{codeexample}
\end{command}


\subsection{The Parabola Path Operation}

\begin{command}{\pgfpathparabola\marg{bend vector}\marg{end vector}}
  This command appends two half-parabolas to the  current path. The
  first starts at the current point and ends at the current point plus
  \meta{bend vector}. At his point, it has its bend. The second half
  parabola starts at that bend point and end at point that is given by
  the bend plus \meta{end vector}.

  If you set \meta{end vector} to the null vector, you append only a
  half parabola that goes from the current point to the bend; by
  setting \meta{bend vector} to the null vector, you append only a
  half parabola that goes to current point plus \meta{end vector} and
  has its bend at the current point.

  It is not possible to use this command to draw a part of a parabola
  that does not contain the bend.

\begin{codeexample}[]
\begin{pgfpicture}
  % Half-parabola going ``up and right''
  \pgfpathmoveto{\pgfpointorigin}
  \pgfpathparabola{\pgfpointorigin}{\pgfpoint{2cm}{4cm}}
  \color{red}
  \pgfusepath{stroke}

  % Half-parabola going ``down and right''
  \pgfpathmoveto{\pgfpointorigin}
  \pgfpathparabola{\pgfpoint{-2cm}{4cm}}{\pgfpointorigin}
  \color{blue}
  \pgfusepath{stroke}

  % Full parabola
  \pgfpathmoveto{\pgfpoint{-2cm}{2cm}}
  \pgfpathparabola{\pgfpoint{1cm}{-1cm}}{\pgfpoint{2cm}{4cm}}
  \color{orange}
  \pgfusepath{stroke}
\end{pgfpicture}
\end{codeexample}
  The command will apply coordinate transformations and update the
  bounding boxes.
\end{command}


\subsection{Sine and Cosine Path Operations}

Sine and cosine curves often need to be drawn and the following commands
may help with this. However, they only allow you to append sine and
cosine curves in intervals that are multiples of $\pi/2$.

\begin{command}{\pgfpathsine\marg{vector}}
  This command appends a sine curve in the interval $[0,\pi/2]$ to the
  current path. The sine curve is squeezed or stretched such that the
  curve starts at the current point and ends at the current point plus
  \meta{vector}.
\begin{codeexample}[]
\begin{tikzpicture}
  \draw[help lines] (0,0) grid (3,1);
  \pgfpathmoveto{\pgfpoint{1cm}{0cm}}
  \pgfpathsine{\pgfpoint{1cm}{1cm}}
  \pgfusepath{stroke}

  \color{red}
  \pgfpathmoveto{\pgfpoint{1cm}{0cm}}
  \pgfpathsine{\pgfpoint{-2cm}{-2cm}}
  \pgfusepath{stroke}
\end{tikzpicture}
\end{codeexample}
  The command will apply coordinate transformations and update the
  bounding boxes.  
\end{command}

\begin{command}{\pgfpathcosine\marg{vector}}
  This command appends a cosine curve in the interval $[0,\pi/2]$ to the
  current path. The curve is squeezed or stretched such that the
  curve starts at the current point and ends at the current point plus
  \meta{vector}. Using several sine and cosine operations in sequence
  allows you to produce a complete sine or cosine curve
\begin{codeexample}[]
\begin{pgfpicture}
  \pgfpathmoveto{\pgfpoint{0cm}{0cm}}
  \pgfpathsine{\pgfpoint{1cm}{1cm}}
  \pgfpathcosine{\pgfpoint{1cm}{-1cm}}
  \pgfpathsine{\pgfpoint{1cm}{-1cm}}
  \pgfpathcosine{\pgfpoint{1cm}{1cm}}
  \pgfsetfillcolor{examplefill}
  \pgfusepath{fill,stroke}
\end{pgfpicture}
\end{codeexample}
  The command will apply coordinate transformations and update the
  bounding boxes.  
\end{command}



\subsection{Plot Path Operations}

There exist several commands for appending
plots to a path. These
commands are available through the module |plot|. They are
documented in Section~\ref{section-plots}.


\subsection{Rounded Corners}

Normally, when you connect two straight line segments or when you
connect two curves that end and start ``at different angles'' you get
``sharp corners'' between the lines or curves. In some cases it is
desirable to produce ``rounded corners'' instead. Thus, the lines
or curves should be shortened a bit and then connected by arcs.

\pgfname\ offers an easy way to achieve this effect, by calling the
following two commands.

\begin{command}{\pgfsetcornersarced\marg{point}}
  This command causes all subsequent corners to be replaced by little
  arcs. The effect of this command lasts till the end of the current
  \TeX\ scope.

  The \meta{point} dictates how large the corner arc will be. Consider
  a corner made by two lines $l$ and~$r$ and assume that the line $l$
  comes first on the path. The $x$-dimension of the \meta{point}
  decides by how much the line~$l$ will be shortened, the
  $y$-dimension of \meta{point} decides by how much the line $r$ will
  be shortened. Then, the shortened lines are connected by an arc.

\begin{codeexample}[]
\begin{tikzpicture}
  \draw[help lines] (0,0) grid (3,2);

  \pgfsetcornersarced{\pgfpoint{5mm}{5mm}}
  \pgfpathrectanglecorners{\pgfpointorigin}{\pgfpoint{3cm}{2cm}}
  \pgfusepath{stroke}
\end{tikzpicture}
\end{codeexample}

\begin{codeexample}[]
\begin{tikzpicture}
  \draw[help lines] (0,0) grid (3,2);

  \pgfsetcornersarced{\pgfpoint{10mm}{5mm}}
  % 10mm entering,
  % 5mm leaving.
  \pgfpathmoveto{\pgfpointorigin}
  \pgfpathlineto{\pgfpoint{0cm}{2cm}}
  \pgfpathlineto{\pgfpoint{3cm}{2cm}}
  \pgfpathcurveto
    {\pgfpoint{3cm}{0cm}}
    {\pgfpoint{2cm}{0cm}}
    {\pgfpoint{1cm}{0cm}}
  \pgfusepath{stroke}
\end{tikzpicture}
\end{codeexample}

  If the $x$- and $y$-coordinates of \meta{point} are the same and the
  corner is a right angle, you will get a perfect quarter circle
  (well, not quite perfect, but perfect up to six decimals). When the
  angle is not $90^\circ$, you only get a fair approximation.

  More or less ``all'' corners will be rounded, even the corner
  generated by a |\pgfpathclose| command. (The author is a bit proud
  of this feature.)
  
\begin{codeexample}[]
\begin{pgfpicture}
  \pgfsetcornersarced{\pgfpoint{4pt}{4pt}}
  \pgfpathmoveto{\pgfpointpolar{0}{1cm}}
  \pgfpathlineto{\pgfpointpolar{72}{1cm}}
  \pgfpathlineto{\pgfpointpolar{144}{1cm}}
  \pgfpathlineto{\pgfpointpolar{216}{1cm}}
  \pgfpathlineto{\pgfpointpolar{288}{1cm}}
  \pgfpathclose
  \pgfusepath{stroke}
\end{pgfpicture}
\end{codeexample}

  To return to normal (unrounded) corners, use
  |\pgfsetcornersarced{\pgfpointorigin}|.

  Note that the rounding will produce strange and undesirable effects
  if the lines at the corners are too short. In this case the
  shortening may cause the lines to ``suddenly extend over the other
  end'' which is rarely desirable. 
\end{command}




\subsection{Internal Tracking of Bounding Boxes for Paths and Pictures}

\label{section-bb}

\makeatletter

The path construction commands keep track of two bounding boxes: One
for the current path, which is reset whenever the path is used and
thereby flushed, and a bounding box for the current |{pgfpicture}|. 

\begin{command}{\pgfresetboundingbox}
	Resets the picture's bounding box. The picture will simply forget any previous bounding box updates and start collecting from scratch. 
	
	You can use this together with |\pgfusepath{use as bounding box}| to replace the bounding box by the one of a particular path (ignoring subsequent paths).
\end{command}

The bounding boxes are not accessible by ``normal'' macros. Rather,
two sets of four dimension variables are used for this, all of which
contain the letter~|@|.

\begin{textoken}{\pgf@pathminx}
  The minimum $x$-coordinate ``mentioned'' in the current
  path. Initially, this is set to $16000$pt.
\end{textoken}

\begin{textoken}{\pgf@pathmaxx}
  The maximum $x$-coordinate ``mentioned'' in the current
  path. Initially, this is set to $-16000$pt.
\end{textoken}

\begin{textoken}{\pgf@pathminy}
  The minimum $y$-coordinate ``mentioned'' in the current
  path. Initially, this is set to $16000$pt.
\end{textoken}

\begin{textoken}{\pgf@pathmaxy}
  The maximum $y$-coordinate ``mentioned'' in the current
  path. Initially, this is set to $-16000$pt.
\end{textoken}

\begin{textoken}{\pgf@picminx}
  The minimum $x$-coordinate ``mentioned'' in the current
  picture. Initially, this is set to $16000$pt.
\end{textoken}

\begin{textoken}{\pgf@picmaxx}
  The maximum $x$-coordinate ``mentioned'' in the current
  picture. Initially, this is set to $-16000$pt.
\end{textoken}

\begin{textoken}{\pgf@picminy}
  The minimum $y$-coordinate ``mentioned'' in the current
  picture. Initially, this is set to $16000$pt.
\end{textoken}

\begin{textoken}{\pgf@picmaxy}
  The maximum $y$-coordinate ``mentioned'' in the current
  picture. Initially, this is set to $-16000$pt.
\end{textoken}


Each time a path construction command is called, the above variables
are (globally) updated. To facilitate this, you can use the following
command:

\begin{command}{\pgf@protocolsizes\marg{x-dimension}\marg{y-dimension}}
  Updates all of the above dimension in such a way that the point
  specified by the two arguments is inside both bounding boxes. For
  the picture's bounding box this updating occurs only if
  |\ifpgf@relevantforpicturesize| is true, see below.
\end{command}

For the bounding box of the picture it is not always desirable that
every path construction command affects this bounding box. For
example, if you have just used a clip command, you do not want anything
outside the clipping area to affect the bounding box. For this reason,
there exists a special ``\TeX\ if'' that (locally) decides whether
updating should be applied to the picture's bounding box. Clipping
will set this if to false, as will certain other commands.

\begin{command}{\pgf@relevantforpicturesizefalse}
  Suppresses updating of the picture's bounding box.
\end{command}

\begin{command}{\pgf@relevantforpicturesizetrue}
  Causes updating of the picture's bounding box.
\end{command}


 % Copyright 2008 by Till Tantau and Mark Wibrow
%
% This file may be distributed and/or modified
%
% 1. under the LaTeX Project Public License and/or
% 2. under the GNU Free Documentation License.
%
% See the file doc/generic/pgf/licenses/LICENSE for more details.

\section{Decorations}
\label{section-base-decorations}


\begin{pgfmodule}{decorations}
  The commands for creating decorations are defined in this
  module, so you need to load this module to use decorations. This
  module is automatically loaded by the different decoration
  libraries. 
\end{pgfmodule}


\subsection{Overview}

Decorations are a general way of creating graphics by ``moving along''
a path and, while doing so, either drawing something or constructing a
new path. This could be as simple as extending a path with a
``zigzagged'' line\ldots 

\begin{codeexample}[]
\tikz \draw decorate[decoration=zigzag] {(0,0) -- (3,0)};
\end{codeexample}
\ldots but could also be as complex as typesetting text along a path:
{\catcode`\|12
\begin{codeexample}[]
\tikz \path decorate [decoration={text along path,
     text={Some text along a path}}]
   { (0,2) .. controls (2,2) and (1,0) .. (3,0) };
\end{codeexample}
}

The workflow for using decorations is the following:
\begin{enumerate}
\item You define a decoration using the |\pgfdeclaredecoration|
  command. Different useful decorations are already declared in
  libraries like |decorations.shapes|.
\item You use normal path construction commands like |\pgfpathlineto|
  to construct a path. Let us call this path the
  \emph{to-be-decorated} path.
\item You place the path construction commands inside the environment
  |{pgfdecoration}|. This environment takes the name of a previously
  declared decoration as a parameter. It will then starting ``walking
  along'' the to-be-decorated path. As it does this, a special finite
  automaton called a \emph{decoration automaton} produces as its
  output new path construction commands (or even other outputs). These
  outputs replace the to-be-decorated path; indeed, after the
  to-be-decorated path has been fully walked along it is thrown away,
  only the output of the automaton persists.
\end{enumerate}

In the present section the process of how decoration automata work is
explained first. Then the command(s) for declaring decoration automata
and for using them are covered.



\subsection{Decoration Automata}

Decoration automata (and the closely related meta-decoration automata)
are a general concept for creating graphics ``along paths.'' For
straight lines, this idea was first proposed by Till Tantau in an
earlier version of \pgfname, the idea to extend this to arbitrary path
was proposed and implemented by Mark Wibrow. Further versitility is
provided by ``meta-decorations''. These are automata that decorate a
path with decorations. 

In the present subsection the different ideas underlying decoration
automata are presented.



\subsubsection{The Different Paths}

In order to prevent confusion with different types of path, such
as those that are extended, those that are decorated and those that 
are created, the following conventions will be used:

\begin{itemize}
\item 
  The \emph{preexisting} path refers to the current path in existence 
  before a decoration environment. (Possibly this path has been
  created by another decoration used earlier, but we will still call
  this path the preexisting path also in this case.)
\item
  The \emph{input} path refers to the to-be-decorated path that the
  decoration automaton moves along. The input path may consist of many
  line and curve subpaths (for example, a circle or an ellipse
  consists of four curves). It is specified inside the decoration
  environment. 
\item
  The \emph{output} path refers to the path that the decoration 
  creates. Depending on the decoration, this path may or may not be
  empty (a decoration can also choose to use side-effects instead of
  producing an output path). The input path is always consumed by the
  decoration automaton, that is, it is no longer available in any way
  after the decoration automaton has finished. 
\end{itemize}

The effect of a decoration environment is the following: The input
path, which is specified inside the environment, is constructed and
stored. This process does not alter the preexisting path in any
way. Then the decoration automaton is started (as described later) and
it produces an output path (possibly empty). Whenever part of the
output path is produced, it is concatenated with the preexisting
path. After the environment, the current path will equal the original
preexisting path followed by the output path.

It is permissible that a decoration issues a |\pgfusepath|
command. As usual, this causes the current path to be
filled or stroked or some other action to be taken and the current
path is set to the empty path. As described above, when the decoration
automaton starts the current path is the preexisting path and as the
automaton progresses, the current path is constantly being extend by
the output path. This means that first time e |\pgfusepath| command is
used on a decoration, the preexisting path is part of the path this
command operates on; in subsequent calls only the part of the output
path constructed since the last |\pgfusepath| command will be used. 

You can use this mechanism to stroke or fill different part of the
output path in different colors, line widths, fills and shades; all 
within the same decoration. Alternatively, a decoration can choose to
produce no output path at all: the |text| decoration simply typesets
text along a path.


\subsubsection{Segments and States}

The most common use a decoration is to ``repeat something along a
path'' (for example, the |zigzag| decoration  
repeats \tikz\draw decorate[decoration=zigzag]
{(0,0)--(\pgfdecorationsegmentlength,0)}; along a path). However, it
not necessarily the case that only one thing be repeated: a decoration
can consist of different parts, or \emph{segments}, repeated in a
particular order. 

When you declare a decoration, you provide a description 
of how their different segments will be rendered. The description of
each segment should be given in a way as if the ``x-axis'' of the
segment is the tangent to the path at a particular point,
and that point is the origin of the segment.
Thus, for example, the segment of the |zigzag| decoration might be
defined using the following code: 
\begin{codeexample}[code only]
\pgfpathlineto{\pgfpoint{5pt}{5pt}}
\pgfpathlineto{\pgfpoint{15pt}{-5pt}}
\pgfpathlineto{\pgfpoint{20pt}{0pt}}
\end{codeexample}

\pgfname\ will ensure that an appropriate coordinate transformation
is in place when the segment is rendered such that
the segment actually points in the right direction. Also
subsequent segments will be transformed such that they are
``further along the path'' toward the end of the path.
All transformations are setup automatically.

Note that we did not use a |\pgfpathmoveto{\pgfpointorigin}| at the
beginning of the segment code. Doing so would subdivide the path into
numerous subpaths. Rather, we assume that the previous segment caused
the current point to be at the origin.

The width of a segment can (and must) be specified
explicitly. \pgfname\ will use this width to find out the start point
of the next segment and the correct rotation. The width the you
provide need not be the ``real'' width of the segment, which allows
decoration segments to overlap or to be spaced far apart. 

The |zigzag| decoration only has one segment that is repeated again and
again. However, we might also like to have \emph{different} segments
and use rules to describe which segment should be used where. For
example, we might have special segments at the start and at the end.

Decorations use a mechanism known in theoretical in computer science
as \emph{finite state automata} to describe which segment is used at a
particular point. The idea is the following: For the first segment we 
start in a special \emph{state} called the \emph{initial state}. In
this state, and also in all other state later, \pgfname\ first
computes how much space is left on the input path. That is, \pgfname\ keeps
track of the distance to the end of the input path. Attached to each state 
there is a set of rules of the following form: ``If the remaining 
distance on the input path is less than $x$, switch to state~$q$.''
\pgfname\ checks for each of these rules whether it applies and, if
so, immediately switches to state~$q$.

Only if none of the rules tell us to switch to another
state, \pgfname\ will execute the state's code. This code will
(typically) add a segment to the output path. In addition to the rules
there is also width parameter attached to each state. \pgfname\ then
translates the coordinate system by this width and reduces the
remaining distance on the input path. Then, \pgfname\ either stays in
the current state or switches to another state, depending on yet
another property attached of the state.

The whole process stops when a special state called |final| is
reached. The segment of this state is immediately added to the output
path (it is often empty, though) and the process ends.




\subsection{Declaring Decorations}

The following command is used to declare a decoration. Essentially,
this command describes the decoration automaton.


\begin{command}{\pgfdeclaredecoration\marg{name}\marg{initial
      state}\marg{states}}
  This command declares a new decoration called \meta{name}. The
  \meta{states} argument contains a description of the decoration
  automaton's states and the transitions between them. The
  \meta{initial state} is the state in which the automaton starts.

  When the automaton is later applied to an input path, it keeps track
  of a certain position on the input path. This current point
  will ``travel along the path,'' each time being moved along by a
  certain distance. This will also work if the path is not a straight
  line. That is, it is permissible that the path curves are veers at a
  sharp angle.  It is also permissible that while travelling along the
  input path the current subpath ends and a new subpath starts. In this
  case, the remaining distance on the first subpath is subtracted
  from the \meta{dimension} and then we travelled along the second
  subpath for the remaining distance. This subpath may also end
  early, in which case we travel along the next subpath, and so
  on. Note that it cannot happen that we travel past the end of the
  input path since this would have caused an immediate switch to
  the |final| state.

  Note note that the computation of the path lengths has only a low
  accuracy because of \TeX's small math capabilities. Do not
  expect high accuracy alignments when using decorations (unless the
  input path consists only of horizontal and vertical lines).

  The \meta{states} argument should consist of |\state| commands, one
  for each state of the decoration automaton. The |\state| command is
  defined only when the \meta{states} argument is executed.

  \begin{command}{\state\marg{name}\oarg{options}\marg{code}}
    This command declares a new state inside the current decoration
    automaton. The state is named \meta{name}.
    
    When the decoration automaton is in state \meta{name}, the following things
    happen:
    \begin{enumerate}
    \item
      The \meta{options} are parsed. This may lead, see below, to a 
      state switch. When this happens, the following steps are not
      executed. The \meta{options} are executed one after the other in
      the given order. If an option causes a state switch, the switch
      is immediate, even if later options might cause a different
      state switch.
    \item
      The \meta{code} is executed in a \TeX-group with the current
      transformation matrix setup in such a way that the origin is on
      the input path at the current point (the point at the distance
      travelled up to now) and the coordinate system is rotated in
      such a way that the positive $x$-axis points in the direction of
      the tangent to the input path at the current point, while the
      positive $y$-axis points to the left of this tangent.
      
      As described earlier, the \meta{code} can have two different
      effects: If it just contains path construction commands, the
      decoration will produce an output path, which is extends the
      preexisting path. Here is an example:

\begin{codeexample}[]
\pgfdeclaredecoration{example}{initial}
{
  \state{initial}[width=10pt]
  {
    \pgfpathlineto{\pgfpoint{0pt}{5pt}}
    \pgfpathlineto{\pgfpoint{5pt}{5pt}}
    \pgfpathlineto{\pgfpoint{5pt}{-5pt}}
    \pgfpathlineto{\pgfpoint{10pt}{-5pt}}
    \pgfpathlineto{\pgfpoint{10pt}{0pt}}
  }
  \state{final}
  {
    \pgfpathlineto{\pgfpointdecoratedpathlast}
  }
}
\tikz[decoration=example]
{
  \draw [decorate]     (0,0) -- (3,0);
  \draw [red,decorate] (0,0) to [out=45,in=135] (3,0);
}
\end{codeexample}

    Alternatively, the \meta{code} can also contain the
    |\pgfusepath| command. This will use the path in usual manner,
    where ``the path'' is the preexisting path plus a part of the
    output path for the first invocation and the different parts of
    the rest of the output path for the following invocation. Here is
    an example:
            
\begin{codeexample}[]
\pgfdeclaredecoration{stars}{initial}{
  \state{initial}[width=15pt]
  {
    \pgfmathparse{round(rnd*100)}
    \pgfsetfillcolor{yellow!\pgfmathresult!orange}
    \pgfsetstrokecolor{yellow!\pgfmathresult!red}
    \pgfnode{star}{center}{}{}{\pgfusepath{stroke,fill}}
  }
  \state{final}
  {
    \pgfpathmoveto{\pgfpointdecoratedpathlast}
  }
}
\tikz\path[decorate, decoration=stars, star point ratio=2, star points=5,
           inner sep=0, minimum size=rnd*10pt+2pt]            
  (0,0) .. controls (0,2)  and (3,2)  .. (3,0)
        .. controls (3,-3) and (0,0)  .. (0,-3)
        .. controls (0,-5) and (3,-5) .. (3,-3);
\end{codeexample}

    \item
      After the \meta{code} has been executed (possibly more than
      once, if the |repeat state| option is used), the state switches to
      whatever state has been specified inside the \meta{options}
      using the |next state| option. If no |next state| has been
      specified, the state stays the same.
    \end{enumerate}

    The \meta{options} are executed with the key path set to
    |/pgf/decoration automaton|. The following keys are defined:
    \begin{key}{/pgf/decoration automaton/switch if less than=\meta{dimension}| to |\meta{new state}}
      When this key is encountered, \pgfname\ checks whether the
      remaining distance to the end of the input path is less than
      \meta{dimension}. If so, an immediate state switch to \meta{new
        state} occurs.
    \end{key}
    \begin{key}{/pgf/decoration automaton/switch if subpath less than=\meta{dimension}| to |\meta{new state}}
      When this key is encountered, \pgfname\ checks whether the
      remaining distance to the end of the current subpath of the
      input path is less than \meta{dimension}. If so, an immediate
      state switch to \meta{new state} occurs. A new subpath of the
      input path is started (exactly) with each |\pgfpathmoveto|
      operation on the input path.
    \end{key}
    \begin{key}{/pgf/decoration automaton/width=\meta{dimension}}
      First, this option causes an immediate switch to the
      state |final| if the remaining distance on the input path is
      less than \meta{dimension}. The effect is the same as if you had
      said |switch if less than=|\meta{dimension}| to final| just
      before the |width| option.

      If no switch occurs, this option tells \pgfname\ the width of
      the segment. The current point will travel along the input path
      (as described earlier)   by this distance.
    \end{key}
    \begin{key}{/pgf/decoration automaton/repeat state=\meta{repetitions} (initially 0)}
      Tells \pgfname\ how long the automaton stays ``normally'' in the
      current state. This count is reset to \meta{repetitions} each
      time one of the |switch if| keys causes a state switch. If no
      state switches occur, the \meta{code} is executed and the
      repetition counter is decreased. Then, there is once more a
      chance of a state change caused by any of the \meta{options}. If
      no repetition occurs, the \meta{code} is executed 
      once more and the counter is decreased once more. When the
      counter reaches zero, the \meta{code} is executed once more,
      but, then, a different state is entered, as specified by the
      |next state| option.

      Note, that the maximum number of times the state will be executed 
      is $\meta{repetitions}+1$.
    \end{key}
    \begin{key}{/pgf/decoration automaton/next state=\meta{new state}}
      After the \meta{code} for state has been executed for the last
      time, a state switch to \meta{new state} is performed. If this
      option is not given, the next state is the same as the current state.
    \end{key}

    You may sometimes wish to do computations outside the
    transformational \TeX-group of the current segment,
    so that these results of these computations are available in the
    next state. For this, the following two options are useful:
    
    \begin{key}{/pgf/decoration automaton/persisting precomputation=\meta{precode}}
      If the \meta{code} of state is executed, the \meta{precode} is
      executed first and it executed outside the \TeX-group of the
      \meta{code}. Note that when the \meta{precode} is executed, the
      transformation matrix is not setup.
    \end{key}

    \begin{key}{/pgf/decoration automaton/persisting postcomputation=\meta{postcode}}
      Works like the |persisting precomputation| option, only the
      \meta{postcode} is executed after (and also outside) the
      \TeX-group of the main \meta{code}.
    \end{key}
    
    There are a number of macros and dimensions which may be useful
    inside a decoration automaton. The following macros are available:
    
    \begin{command}{\pgfdecoratedpathlength}
      The length of the input path. If the input path consists of
      several subpaths, this number is the sum of the lengths of the
      subpaths. 
    \end{command}
    
    \begin{command}{\pgfdecoratedsubpathlength}
      The length of the current subpath of the input path. ``Current
      subpath''  refers to the subpath on which the current point
      lies. 
    \end{command}
		
    \begin{command}{\pgfpointdecoratedpathlast}
      The final point of the input path.
    \end{command}
    
    \begin{command}{\pgfpointdecoratedsubpathlast}
      The final point of the current subpath of the input path.
    \end{command}
		
    \begin{command}{\pgfdecoratedangle}
      The angle of the tangent to the decorated path at the \emph{origin}
      of the current segment. The transformation matrix applied at 
      the beginning of a state includes a rotation equivalent to 
      this angle.
    \end{command}
		
    The following \TeX\ dimension registers are also available inside the 
    automaton:
    
    \begin{command}{\pgfdecoratedremainingdistance}
      The remaining distance on the input path.
    \end{command}
		
    \begin{command}{\pgfdecoratedcompleteddistance}
      The completed distance on the input path.
    \end{command}
    
    \begin{command}{\pgfdecoratedsubpathremainingdistance}
      The remaining distance on the current subpath of the input path.
    \end{command}
    
    \begin{command}{\pgfdecoratedsubpathcompleteddistance}
      The completed distance on the current subpath of the input path.
    \end{command}

    Further keys and macros are defined and used by the decoration
    libraries, see Section~\ref{section-library-decorations}.
    
    The following example shows how these options can be used:
\begin{codeexample}[]
\pgfdeclaredecoration{complicated example decoration}{initial}
{
  \state{initial}[width=5pt,next state=up]
  { \pgfpathlineto{\pgfpoint{5pt}{0pt}} }
  
  \state{up}[width=5pt,next state=down]
  {
    \ifdim\pgfdecoratedremainingdistance>\pgfdecoratedcompleteddistance
      % Growing
      \pgfpathlineto{\pgfpoint{0pt}{\pgfdecoratedcompleteddistance}}
      \pgfpathlineto{\pgfpoint{5pt}{\pgfdecoratedcompleteddistance}}
      \pgfpathlineto{\pgfpoint{5pt}{0pt}}
    \else
      % Shrinking
      \pgfpathlineto{\pgfpoint{0pt}{\pgfdecoratedremainingdistance}}
      \pgfpathlineto{\pgfpoint{5pt}{\pgfdecoratedremainingdistance}}
      \pgfpathlineto{\pgfpoint{5pt}{0pt}}
    \fi%      
  }
  \state{down}[width=5pt,next state=up]
  {
    \ifdim\pgfdecoratedremainingdistance>\pgfdecoratedcompleteddistance
      % Growing
      \pgfpathlineto{\pgfpoint{0pt}{-\pgfdecoratedcompleteddistance}}
      \pgfpathlineto{\pgfpoint{5pt}{-\pgfdecoratedcompleteddistance}}
      \pgfpathlineto{\pgfpoint{5pt}{0pt}}
    \else
      % Shrinking
      \pgfpathlineto{\pgfpoint{0pt}{-\pgfdecoratedremainingdistance}}
      \pgfpathlineto{\pgfpoint{5pt}{-\pgfdecoratedremainingdistance}}
      \pgfpathlineto{\pgfpoint{5pt}{0pt}}
    \fi%      
  }
  \state{final}
  {
    \pgfpathlineto{\pgfpointdecoratedpathlast}
  }
}
\begin{tikzpicture}[decoration=complicated example decoration]
  \draw decorate{ (0,0) -- (3,0)};
  \fill [red!50,rounded corners=2pt]
    decorate {(.5,-2) -- ++(2.5,-2.5)} -- (3,-5) -| (0,-2) -- cycle;
\end{tikzpicture}
\end{codeexample}
  \end{command}
\end{command}



\subsubsection{Predefined Decorations}

The three decorations |moveto|, |lineto|, and |curveto| are predefined
and ``always available.'' They are mostly useful in conjunction with
meta-decorations. They are documented in
Section~\ref{section-library-decorations} alongside the other
decorations.



\subsection{Using Decorations}

Once a decoration has been declared, it can be used. 

\begin{environment}{{pgfdecoration}\marg{decoration list}}
  The \meta{environment contents} should contain commands for creating
  an path. This path is the basis for the \emph{input paths}
  for the decorations in the \meta{decoration list}. In detail, the
  following happens: 
  \begin{enumerate}
  \item
    The preexisting unused path is saved.
  \item 
    The path commands specified in \meta{environment contents} are
    executed and this resulting path is saved. The path is then
    divided into different \emph{input paths} as follows:
    The format for each item in \marg{decoration list} is 
    \begin{quote}
      \marg{decoration}\marg{length}\opt{\marg{before code}\marg{after code}}
    \end{quote}
    The \meta{before code} and the \meta{after code} are optional. The
    input path is divided into input paths as follows: The first input
    path consists of the first lines of the path specified in the
    \meta{environment contents} until the \meta{length}  of the first
    element of the \meta{decoration list} is reached. If this length
    is reached in the middle of a line, the line is broken up at this
    exact position. Then the second input path has the \meta{length}
    of the second element in the \meta{decoration list} and consists
    of the lines making up the following \meta{length} part of the
    path in the \meta{environment contents}, and so on.

    If the lengths in the \meta{decoration list}
    do not add up to the total length of the path in the
    \meta{environment contents}, either some  decorations are dropped
    (if their lengths add up to more than the length of the
    \meta{environment contents}) or
    the input path is not fully used (if their lengths  add up to less).
  \item
    The preexisting path is reinstalled.
  \item
    The decoration automata move along the input paths, thus creating
    (and  possibly using) the output paths. These output paths extend
    (unless they are used) the current path.
  \end{enumerate}
	 
  Some important points should be noted regarding the use of this
  environment:
  
  \begin{itemize}
  \item
    If \meta{environment contents} does not begin with 
    |\pgfpathmoveto|,	he last known point on the preexisting path is 
    assumed as the starting point.
  \item
    All except the last of any sequence of consecutive move-to commands 
    in \meta{environment contents} are discarded.
  \item
    Any move-to commands at end of \meta{environment contents} are 
    ignored.
  \item
    Any close-path commands on the input path are interpreted as 
    straight lines.
    Internally somthing a little more complicated is going on,
    however, a closed path on the input path has no effect on the 
    output path, other than causing the automaton to travel in a 
    straight line towards the location of the last move-to command on 
    the input path.
  \item
    Although tangent computations for the input path work with the
    last point on the preexisting path, no automatic move-to
    operations are issued for the output path. 
    If an output path commences with a line-to or curve-to when the 
    existing path is empty, an appropriate move-to command should be 
    inserted before the decoration commences.
  \item
    If a decoration uses its own path, the first time this happens the
    preexisting path is part of the path that is used at this point.
  \end{itemize}

  When the decoration automata ``work on'' their respective input
  paths, before the automaton starts, \meta{before code} is
  executed. After the decoration automaton has finished, \meta{after
    code} is executed. 
        
\begin{codeexample}[]
\begin{tikzpicture}[decoration={segment length=5pt}]
  \draw [help lines] grid (3,2);
  \begin{pgfdecoration}{{curveto}{1cm},{zigzag}{2cm},{curveto}{1cm}}
    \pgfpathmoveto{\pgfpointorigin}
    \pgfpathcurveto
      {\pgfpoint{0cm}{2cm}}{\pgfpoint{3cm}{2cm}}{\pgfpoint{3cm}{0cm}}
  \end{pgfdecoration}
\pgfusepath{stroke}
\end{tikzpicture}
\end{codeexample}

  When the lengths are evaluated, the dimension
  |\pgfdecoratedremainingdistance| holds the remaining distance on
  the entire decorated path, and |\pgfdecoratedpathlength| holds the
  total length of the path. Thus, it is possible to specify lengths
  like |\pgfdecoratedpathlength/3|.

\begin{codeexample}[]
\begin{tikzpicture}[decoration={segment length=5pt}]
  \draw [help lines] grid (3,2);
  \begin{pgfdecoration}{
      {curveto}{\pgfdecoratedpathlength/3},
      {zigzag}{\pgfdecoratedpathlength/3},
      {curveto}{\pgfdecoratedremainingdistance}
    }
    \pgfpathmoveto{\pgfpointorigin}
    \pgfpathcurveto
      {\pgfpoint{0cm}{2cm}}{\pgfpoint{3cm}{2cm}}{\pgfpoint{3cm}{0cm}}
  \end{pgfdecoration}
  \pgfusepath{stroke}
\end{tikzpicture}
\end{codeexample}

  When \meta{before code} is executed, the following macro is useful:
  \begin{command}{\pgfpointdecoratedpathfirst}
    Returns the point corresponding to the start of the current
    input path.
  \end{command}
  When \meta{after code} is executed, the following macro can be used:
  \begin{command}{\pgfpointdecoratedpathlast}
    Returns the point corresponding to the end of the current input
    path.
  \end{command}
  This means that if decorations do not use their own path, it is 
  possible to do something with them and and continue from the
  correct place. 
	
\begin{codeexample}[]
\begin{tikzpicture}
  \draw [help lines] grid (3,2);
  \begin{pgfdecoration}{
      {curveto}{\pgfdecoratedpathlength/3}
      {}
      {
        \pgfusepath{stroke}
      },
      {zigzag}{\pgfdecoratedpathlength/3}
      {
        \pgfpathmoveto{\pgfpointdecoratedpathfirst}
        \pgfdecorationsegmentlength=5pt
      }
      {
        \pgfsetstrokecolor{red}
        \pgfusepath{stroke}
        \pgfpathmoveto{\pgfpointdecoratedpathlast}
        \pgfsetstrokecolor{black}
      },
      {curveto}{\pgfdecoratedremainingdistance}
    }
    \pgfpathmoveto{\pgfpointorigin}
    \pgfpathcurveto
      {\pgfpoint{0cm}{2cm}}{\pgfpoint{3cm}{2cm}}{\pgfpoint{3cm}{0cm}}
  \end{pgfdecoration}
  \pgfusepath{stroke}
\end{tikzpicture}
\end{codeexample}
	
  After the |{decoration}| environment has finished, the following 
  macros are available:
	
  \begin{command}{\pgfdecorateexistingpath}
    The preexisting path before the environment was entered.
  \end{command}
	
  \begin{command}{\pgfdecoratedpath}
    The (total) input path (that is, the path created by the environment contents).
  \end{command}
	
  \begin{command}{\pgfdecorationpath}
    The output path. If the path is used, this macro contains only the
    last unused part of the output path.
  \end{command}
	
  \begin{command}{\pgfpointdecoratedpathlast}
    The final point of the input path.
  \end{command}
  
  \begin{command}{\pgfpointdecorationpathlast}
    The final point of the output path.
  \end{command}
\end{environment}

\begin{plainenvironment}{{pgfdecoration}\marg{name}}
  The plain-\TeX{} version of the |{pgfdecorate}| environment.
\end{plainenvironment}

\begin{contextenvironment}{{pgfdecoration}\marg{name}}
  The Con\TeX t version of the |{pgfdecoration}| environment.
\end{contextenvironment}

For convenience, the following macros provide a ``shorthand''
for decorations (internally, they all use the |{pgfdecoration}|
environment).

\begin{command}{\pgfdecoratepath\marg{name}\marg{path commands}}
  Decorate the path described by \meta{path commands} with the
  decoration \meta{name}. This is equivalent to
\begin{codeexample}[code only]
\pgfdecorate{{name}{\pgfdecoratedpathlength}
             {\pgfdecoratebeforecode}{\pgfdecorateaftercode}}
  // the path commands.
\endpgfdecorate    
\end{codeexample}
\end{command}

\begin{command}{\pgfdecoratecurrentpath\marg{name}}
  Decorate the preexisting path with the decoration \meta{name}.
\end{command}

Both the above commands use the current definitons of the following
macros:

\begin{command}{\pgfdecoratebeforecode}
  Code executed as \meta{before code}, see the description of
  |\pgfdecorate|. 
\end{command}

\begin{command}{\pgfdecorateaftercode}
  Code executed as \meta{after code}, see the description of
  |\pgfdecorate|. 
\end{command}

It may sometimes be useful to add an additional transformation
for each segment of a decoration. The following command allows 
you to define such a ``last minute transformation.''

\begin{command}{\pgfsetdecorationsegmenttransformation\marg{code}}
  The \meta{code} will be executed at the very beginning of each
  segment. Note when applying multiple decorations, this will
  be reset between decorations, so it needs to be specified for
  each segment.

\begin{codeexample}[]
\begin{tikzpicture}
  \draw [help lines] grid (3,2);
  \begin{pgfdecoration}{
      {curveto}{\pgfdecoratedpathlength/3},
      {zigzag}{\pgfdecoratedpathlength/3}
      {
        \pgfdecorationsegmentlength=5pt
        \pgfsetdecorationsegmenttransformation{\pgftransformyshift{.5cm}}
      },
      {curveto}{\pgfdecoratedremainingdistance}
    }
    \pgfpathmoveto{\pgfpointorigin}
    \pgfpathcurveto
      {\pgfpoint{0cm}{2cm}}{\pgfpoint{3cm}{2cm}}{\pgfpoint{3cm}{0cm}}
  \end{pgfdecoration}
  \pgfusepath{stroke}
\end{tikzpicture}
\end{codeexample}
\end{command}




\subsection{Meta-Decorations}

\label{section-base-meta-decorations}

A meta-decoration provides an alternative way to decorate a path with 
mutiple decorations. It is, in essence, an automaton that decorates
an input path with decoration automatons. In general, however, the end
effect is still that a path is decorated with other paths, and the input 
path should be thought of as being divided into sub-input-paths, each with 
their own decoration. Like ordinary decorations, before a
meta-decoration can be used it must be declared.

\subsubsection{Declaring Meta-Decorations}

\begin{command}{\pgfdeclaremetadecorate\marg{name}\marg{initial state}\marg{states}}

  This command declares a new meta-decoration called \meta{name}. The
  \meta{states} argument contains a description of the meta-decoration
  automaton's states and the transitions between them. The
  \meta{initial state} is the state in which the automaton starts.
  
  The |\state| command is similar to the one found in 
  decoration declarations, and takes the same form:
  
  \begin{command}{\state\marg{name}\oarg{options}\marg{code}}
    Declares the state \meta{name} inside the current meta-decoration
    automaton. Unlike decorations, states in meta-decorations are not
    executed within a group, which makes the persisting computation
    options superfluous. Consider using an initial state with
    |width=0pt| to do precalculations that could speed the execution
    of the meta-decoration. 
    
    The \meta{options} are executed with the key path set to
    |/pgf/meta-decorations automaton/|, and the following keys are defined for 
    this path: 
    
    \begin{key}{/pgf/meta-decoration automaton/switch if less than=\meta{dimension}| to |\meta{new state}}
      This causes \pgfname\ to check whether the
      remaining distance to the end of the input path is less than
      \meta{dimension}, and, if so, to immediately switch to the state 
      \meta{new state}. When this key is evaluated, the macro 
      |\pgfmetadecoratedpathlength| will be defined as the total length of 
      the decoration path, allowing for values such as
      |\pgfmetadecoratedpathlength/8|.
    \end{key}
    
    \begin{key}{/pgf/meta-decoration automaton/width=\meta{dimension}}
      As always, this option will cause an immediate switch to the
      state |final| if the remaining distance on the input path is less than
      \meta{dimension}. 

      Otherwise, this option tells \pgfname\ the width of the
      ``meta-segment'', that is, the length of the sub-input-path
      which the decoration automaton specified  in \meta{code} will decorate.
    \end{key}
    
    \begin{key}{/pgf/meta-decoration automaton/next state=\meta{new state}}
      After the code for a state has been executed, a state switch to
      \meta{new state} is performed. If this option is not given, the
      next state is the same as the current state.
    \end{key}
    
    The code in \meta{code} is quite different from the code in a 
    decoration state. In almost all cases only the following three
    macros will be required: 
    
    \begin{command}{\decoration\marg{name}}
      This sets the decoration for the current state to \meta{name}.
      If this command is omitted, the |moveto| decoration will be
      used.
    \end{command}
    
    \begin{command}{\beforedecoration\marg{before code}}
      Defines \marg{before code} as (typically) \pgfname{} commands to be
      executed before the decoration is applied to the current segment.
      This command can be omitted.
      If you wish to set up some decoration specific parameters 
      such as segment length, or segment amplitude, then they
      can be set in \meta{before code}.
    \end{command}
   
    \begin{command}{\afterdecoration\marg{after code}}
      Defines \marg{after code} as commands to be executed afer the 
      decoration has been applied to the current segment.
      This command can be omitted.
    \end{command}
    
    There are some macros that may be usedful when creating 
    meta-decorations (note that they are all macros):
    
    \begin{command}{\pgfpointmetadecoratedpathfirst}
      When the \meta{before code} is executed,
      this macro stores the first point on the current
      sub-input-path. 
    \end{command}

    \begin{command}{\pgfpointmetadecoratedpathlast}
      When the \meta{after code} is executed,
      this macro stores the last point on the current
      sub-input-path. 
    \end{command}
    
    \begin{command}{\pgfmetadecoratedpathlength}
      The entire length of the entire input path.
    \end{command}    
    
    \begin{command}{\pgfmetadecoratedcompleteddistance}
      The completed distance on the entire input path.
    \end{command}
    
    \begin{command}{\pgfmetadecoratedremainingdistance}
      The remaining distance on the entire input path.
    \end{command}
    
    \begin{command}{\pgfmetadecoratedsubpathcompleteddistance}
      The completed distance on the current subpath of the entire
      input path.
    \end{command}
    
    \begin{command}{\pgfmetadecoratedsubpathremainingdistance}
      The remaining distance on the current subpath of the entire
      input path.
    \end{command}
  \end{command}

  Here is a complete example of a meta-decoration:

\begin{codeexample}[]
\pgfdeclaremetadecoration{arrows}{initial}{
  \state{initial}[width=0pt, next state=arrow]
  {	
    \pgfmathdivide{100}{\pgfmetadecoratedpathlength}
    \let\factor\pgfmathresult
    \pgfsetlinewidth{1pt}
    \pgfset{/pgf/decoration/segment length=4pt}
  }
  \state{arrow}[
    switch if less than=\pgfmetadecorationsegmentlength to final,
    width=\pgfmetadecorationsegmentlength/3, 
    next state=zigzag]
  {
    \decoration{curveto}
    \beforedecoration
    {   
      \pgfmathparse{\pgfmetadecoratedcompleteddistance*\factor}
      \pgfsetcolor{red!\pgfmathresult!yellow}
      \pgfpathmoveto{\pgfpointmetadecoratedpathfirst}
    }	
  }
  \state{zigzag}[width=\pgfmetadecorationsegmentlength/3, next state=end arrow]
  {
  	\decoration{zigzag}
  }
  \state{end arrow}[width=\pgfmetadecorationsegmentlength/3, next state=move]
  {
    \decoration{curveto}
    \beforedecoration{\pgfpathmoveto{\pgfpointmetadecoratedpathfirst}}
    \afterdecoration
    {	
      \pgfsetarrowsend{to}
      \pgfusepath{stroke}	
    }
  }  
  \state{move}[width=\pgfmetadecorationsegmentlength/2, next state=arrow]{}
  \state{final}{}
}

\tikz\draw[decorate,decoration={arrows,meta-segment length=2cm}]
  (0,0) .. controls (0,2)   and (3,2)   .. (3,0)
        .. controls (3,-2)  and (0,-2)  .. (0,-4)
        .. controls (0,-6)  and (3,-6)  .. (3,-8)
        .. controls (3,-10) and (0,-10) .. (0,-8);
\end{codeexample}

\end{command}


\subsubsection{Predefined Meta-decorations}

There are no predefined meta-decorations loaded with \pgfname{}.


\subsubsection{Using Meta-Decorations}

Using meta-decorations is ``simpler'' than using decorations, because
you can only use one meta-decoration per path.

\begin{environment}{{pgfmetadecoration}\marg{name}}
  This environment decorates the input path described in 
  \meta{environment contents}, with the	meta-decoration \meta{name}.
\end{environment}

\begin{plainenvironment}{{pgfmetadecoration}\marg{name}}
  The plain \TeX{} version of the |{pgfmetadecoration}| environment.
\end{plainenvironment}

\begin{contextenvironment}{{pgfmetadecoration}\marg{name}}
  The Con\TeX t version of the |{pgfmetadecoration}| environment.
\end{contextenvironment}

% Copyright 2003 by Till Tantau <tantau@cs.tu-berlin.de>.
%
% This program can be redistributed and/or modified under the terms
% of the LaTeX Project Public License Distributed from CTAN
% archives in directory macros/latex/base/lppl.txt.


\section{Using Paths}

\subsection{Overview}

Once a path has been constructed, it can be \emph{used} in different
ways. For example, you can draw the path or fill it or use it for
clipping.

Numerous graph parameters influence how a path will be rendered. For
example, when you draw a path, the line width is important as well as
the dashing pattern. The options that govern how paths are rendered
can all be set with commands starting with |\pgfset|. \emph{All
  options that influence how a path is rendered always influence the
  complete path.} Thus, it is not possible to draw part of a path
using, say, a red color and drawing another part using a green
color. To achieve such an effect, you must use two paths.

In detail, paths can be used in the following ways:

\begin{enumerate}
\item
  You can \emph{stroke} (also known as \emph{draw}) a path.
\item
  You can \emph{fill} a path with a uniform color.
\item
  You can \emph{clip} subsequent renderings against the path.
\item
  You can \emph{shade} a path.
\item
  You can \emph{use the path as bounding box} for the whole picture.
\end{enumerate}
You can also perform any combination of the above, though it makes no
sense to fill and shade a path at the same time.

To perform (a combination of) the first three actions, you can use the
following command:
\begin{command}{\pgfusepath\marg{actions}}
  Applies the given \meta{actions} to the current path. Afterwards,
  the current path is (globally) empty. The following actions are
  possible:
  \begin{itemize}
  \item \declare{|fill|}
    fills the path. See Section~\ref{section-fill} for further details.
\begin{codeexample}[]
\begin{pgfpicture}
  \pgfpathmoveto{\pgfpointorigin}
  \pgfpathlineto{\pgfpoint{1cm}{1cm}}
  \pgfpathlineto{\pgfpoint{1cm}{0cm}}
  \pgfusepath{fill}
\end{pgfpicture}
\end{codeexample}
  \item \declare{|stroke|}
    strokes the path. See Section~\ref{section-stroke} for further details.
\begin{codeexample}[]
\begin{pgfpicture}
  \pgfpathmoveto{\pgfpointorigin}
  \pgfpathlineto{\pgfpoint{1cm}{1cm}}
  \pgfpathlineto{\pgfpoint{1cm}{0cm}}
  \pgfusepath{stroke}
\end{pgfpicture}
\end{codeexample}
  \item \declare{|clip|}
    clips all subsequent drawings against the path. See
    Section~\ref{section-clip} for further details.
\begin{codeexample}[]
\begin{pgfpicture}
  \pgfpathmoveto{\pgfpointorigin}
  \pgfpathlineto{\pgfpoint{1cm}{1cm}}
  \pgfpathlineto{\pgfpoint{1cm}{0cm}}
  \pgfusepath{stroke,clip}
  \pgfpathcircle{\pgfpoint{1cm}{1cm}}{0.5cm}
  \pgfusepath{fill}
\end{pgfpicture}
\end{codeexample}
  \item \declare{|discard|}
    discards the path, that is, it is not used at all. Giving this
    option (alone) has the same effect as giving an empty options
    list.
  \end{itemize}
  When more than one of the first three actions are given, they are
  applied in the above ordering, regardless of their ordering in
  \meta{actions}. Thus, |{stroke,fill}| and |{fill,stroke}| have the
  same effect. 
\end{command}

To shade a path, use the |\pgfshadepath| command, which is explained
in Section~\ref{section-shadings}.



\subsection{Stroking a Path}
\label{section-stroke}

When you use |\pgfusepath{stroke}| to stroke a path, several graphic
parameters influence how the path is drawn. The commands for setting
these parameters are explained in the following.

Note that all graphic parameters apply to the path as a whole, never
only to a part of it.

All graphic parameters are local to the current |{pgfscope}|, but they
persists past \TeX\ groups, \emph{except} for the interior rule
(even-odd or nonzero) and the arrow tip kinds. The latter graphic
parameters only persist till the end of the current \TeX\ group, but 
this may change in the future, so do not count on this.

\subsubsection{Graphic Parameter: Line Width}

\begin{command}{\pgfsetlinewidth\marg{line width}}
  This command sets the line width for subsequent strokes (in the
  current |pgfscope|). The line width is given as a normal \TeX\
  dimension like |0.4pt| or |1mm|.

\begin{codeexample}[]
\begin{pgfpicture}
  \pgfsetlinewidth{1mm}
  \pgfpathmoveto{\pgfpoint{0mm}{0mm}}
  \pgfpathlineto{\pgfpoint{2cm}{0mm}}
  \pgfusepath{stroke}
  \pgfsetlinewidth{2\pgflinewidth} % double in size
  \pgfpathmoveto{\pgfpoint{0mm}{5mm}}
  \pgfpathlineto{\pgfpoint{2cm}{5mm}}
  \pgfusepath{stroke}
\end{pgfpicture}
\end{codeexample}
\end{command}

\begin{textoken}{\pgflinewidth}
  You can access the current line width via the \TeX\ dimension
  |\pgflinewidth|. It will be set to the correct line width, that is,
  even when a \TeX\ group closed, the value will be correct since it
  is set globally, but when a |{pgfscope}| closes, the value is set to
  the correct value it had before the scope.
\end{textoken}


\subsubsection{Graphic Parameter: Caps and Joins}

\begin{command}{\pgfsetbuttcap}
  Sets the line cap to a butt cap. See Section~\ref{section-cap-joins}
  for an explanation of what this is.
\end{command}
\begin{command}{\pgfsetroundcap}
  Sets the line cap to a round cap. See again
  Section~\ref{section-cap-joins}.
\end{command}
\begin{command}{\pgfsetrectcap}
  Sets the line cap to a square cap. See again
  Section~\ref{section-cap-joins}. 
\end{command}
\begin{command}{\pgfsetroundjoin}
  Sets the line join to a round join. See again
  Section~\ref{section-cap-joins}. 
\end{command}
\begin{command}{\pgfsetbeveljoin}
  Sets the line join to a bevel join. See again
  Section~\ref{section-cap-joins}. 
\end{command}
\begin{command}{\pgfsetmiterjoin}
  Sets the line join to a miter join. See again
  Section~\ref{section-cap-joins}. 
\end{command}
\begin{command}{\pgfsetmiterlimit\marg{miter limit factor}}
  Sets the miter limit to  \meta{miter limit factor}. See again 
  Section~\ref{section-cap-joins}. 
\end{command}

\subsubsection{Graphic Parameter: Dashing}

\begin{command}{\pgfsetdash\marg{list of even length of dimensions}\marg{phase}}
  Sets the dashing of a line. The first entry in the list specifies
  the length of the first solid part of the list. The second entry
  specifies the length of the following gap. Then comes the length of
  the second solid part, following by the length of the second gap,
  and so on. The \meta{phase} specifies where the first solid part
  starts relative to the beginning of the line.

\begin{codeexample}[]
\begin{pgfpicture}
  \pgfsetdash{{0.5cm}{0.5cm}{0.1cm}{0.2cm}}{0cm}
  \pgfpathmoveto{\pgfpoint{0mm}{0mm}}
  \pgfpathlineto{\pgfpoint{2cm}{0mm}}
  \pgfusepath{stroke}
  \pgfsetdash{{0.5cm}{0.5cm}{0.1cm}{0.2cm}}{0.1cm}
  \pgfpathmoveto{\pgfpoint{0mm}{1mm}}
  \pgfpathlineto{\pgfpoint{2cm}{1mm}}
  \pgfusepath{stroke}
  \pgfsetdash{{0.5cm}{0.5cm}{0.1cm}{0.2cm}}{0.2cm}
  \pgfpathmoveto{\pgfpoint{0mm}{2mm}}
  \pgfpathlineto{\pgfpoint{2cm}{2mm}}
  \pgfusepath{stroke}
\end{pgfpicture}
\end{codeexample}

  Use |\pgfsetdash{}{0pt}| to get a solid dashing.
\end{command}

\subsubsection{Graphic Parameter: Stroke Color}

\begin{command}{\pgfsetstrokecolor\marg{color}}
  Sets the color used for stroking lines to \meta{color}, where
  \meta{color} is a \LaTeX\ color like |red| or |black!20!red|. Unlike
  the |\color| command, the effect of this command lasts till the end
  of the current |{pgfscope}| and not till the end of the current
  \TeX\ group.

  The color used for stroking may be different from the color used for
  filling. However, a |\color| command will always ``immediately
  override'' any special settings for the stroke and fill colors.

  In plain \TeX, this command will also work, but the problem of
  \emph{defining} a color arises. After all, plain \TeX\ does not
  provide \LaTeX\ colors. For this reason, \pgfname\ implements a
  minimalistic ``emulation'' of the |\definecolor|, |\colorlet|, and
  |\color| commands. Only gray-scale and rgb colors are supported. For
  most cases this turns out to be enough.

\begin{codeexample}[]
\begin{pgfpicture}
  \pgfsetlinewidth{1pt}
  \color{red}
  \pgfpathcircle{\pgfpoint{0cm}{0cm}}{3mm} \pgfusepath{fill,stroke}
  \pgfsetstrokecolor{black}
  \pgfpathcircle{\pgfpoint{1cm}{0cm}}{3mm} \pgfusepath{fill,stroke}
  \color{red}
  \pgfpathcircle{\pgfpoint{2cm}{0cm}}{3mm} \pgfusepath{fill,stroke}
\end{pgfpicture}
\end{codeexample}
\end{command}

\begin{command}{\pgfsetcolor\marg{color}}
  Sets both the stroke and fill color. The difference to the normal
  |\color| command is that the effect lasts till the end of the
  current |{pgfscope}|, not only till the end of the current \TeX\
  group. 
\end{command}


\subsubsection{Graphic Parameter: Arrows}

After a path has been drawn, \pgfname\ can add arrow tips at the
ends. Currently, it will only add arrows correctly at the end of paths
that consist of a single open part. For other paths, like closed paths
or path consisting of multiple parts, the result is not defined.

\begin{command}{\pgfsetarrowsstart\marg{arrow kind}}
  Sets the arrow tip kind used at the start of a (possibly curved)
  path. When this option is used, the line will often be slightly
  shortened to ensure that the tip of the arrow will exactly ``touch''
  the ``real'' start of the line.

  To ``clear'' the start arrow, say |\pgfsetarrowsstart{}|.
\begin{codeexample}[]
\begin{pgfpicture}
  \pgfsetarrowsstart{latex}
  \pgfpathmoveto{\pgfpointorigin}
  \pgfpathlineto{\pgfpoint{1cm}{0cm}}
  \pgfusepath{stroke}
  \pgfsetarrowsstart{to}
  \pgfpathmoveto{\pgfpoint{0cm}{2mm}}
  \pgfpathlineto{\pgfpoint{1cm}{2mm}}
  \pgfusepath{stroke}
\end{pgfpicture}
\end{codeexample}

  The effect of this command persists only till the end of the current
  \TeX\ scope.

  The different possible arrow kinds are explained in
  Section~\ref{section-arrows}.  
\end{command}

\begin{command}{\pgfsetarrowsend\marg{arrow kind}}
  Sets the arrow tip kind used at the end of a path.
\begin{codeexample}[]
\begin{pgfpicture}
  \pgfsetarrowsstart{latex}
  \pgfsetarrowsend{to}
  \pgfpathmoveto{\pgfpointorigin}
  \pgfpathlineto{\pgfpoint{1cm}{0cm}}
  \pgfusepath{stroke}
\end{pgfpicture}
\end{codeexample}
\end{command}

\begin{command}{\pgfsetarrows{\texttt{\char`\{}}\meta{start kind}|-|\meta{end kind}{\texttt{\char`\}}}}
  Sets the start arrow kind to \meta{start kind} and the end kind to
  \meta{end kind}.
\begin{codeexample}[]
\begin{pgfpicture}
  \pgfsetarrows{latex-to}
  \pgfpathmoveto{\pgfpointorigin}
  \pgfpathlineto{\pgfpoint{1cm}{0cm}}
  \pgfusepath{stroke}
\end{pgfpicture}
\end{codeexample}
\end{command}

\begin{command}{\pgfsetshortenstart\marg{dimension}}
  This command will shortened the start of every stroked path by the
  given dimension. This shortening is done in addition to automatic
  shortening done by a start arrow, but it can be used even if no
  start arrow is given.

  This command is useful if you wish arrows or lines to ``stop shortly
  before'' a given point.
\begin{codeexample}[]
\begin{pgfpicture}
  \pgfpathcircle{\pgfpointorigin}{5mm}
  \pgfusepath{stroke}
  \pgfsetarrows{latex-}
  \pgfsetshortenstart{4pt}
  \pgfpathmoveto{\pgfpoint{5mm}{0cm}} % would be on the circle
  \pgfpathlineto{\pgfpoint{2cm}{0cm}}
  \pgfusepath{stroke}
\end{pgfpicture}
\end{codeexample}
\end{command}
  
\begin{command}{\pgfsetshortenend\marg{dimension}}
  Works like |\pgfsetshortenstart|.
\end{command}



\subsection{Filling a Path}
\label{section-fill}

Filling a path means coloring every interior point of the path with
the current fill color. It is not always obvious whether a point is
``inside'' a  path when the path is self-intersecting and/or consists
or multiple parts. In this case either the nonzero winding number rule
or the even-odd crossing number rule is used to decide, which points
lie ``inside.'' These rules are explained in
Section~\ref{section-rules}. 

\subsubsection{Graphic Parameter: Interior Rule}

You can set which rule is used using the following commands:

\begin{command}{\pgfseteorule}
  Dictates that the even-odd rule is used in subsequent fillings in
  the current \emph{\TeX\ scope}. Thus, for once, the effect of this
  command does not persist past the current \TeX\ scope.

\begin{codeexample}[]
\begin{pgfpicture}
  \pgfseteorule
  \pgfpathcircle{\pgfpoint{0mm}{0cm}}{7mm}
  \pgfpathcircle{\pgfpoint{5mm}{0cm}}{7mm}
  \pgfusepath{fill}
\end{pgfpicture}
\end{codeexample}
\end{command}

\begin{command}{\pgfsetnonzerorule}
  Dictates that the nonzero winding number rule is used in subsequent
  fillings in the current \TeX\ scope. This is the default.

\begin{codeexample}[]
\begin{pgfpicture}
  \pgfsetnonzerorule
  \pgfpathcircle{\pgfpoint{0mm}{0cm}}{7mm}
  \pgfpathcircle{\pgfpoint{5mm}{0cm}}{7mm}
  \pgfusepath{fill}
\end{pgfpicture}
\end{codeexample}
\end{command}

\subsubsection{Graphic Parameter: Filling Color}

\begin{command}{\pgfsetfillcolor\marg{color}}
  Sets the color used for filling paths to \meta{color}. Like the
  stroke color, the effect lasts only till the next use of |\color|. 
\end{command}


\subsection{Clipping a Path}
\label{section-clip}

When you add the |clip| option, the current path is used for
clipping subsequent drawings. The same rule as for filling is used to
decide whether a point is inside or outside the path, that is, either
the even-odd rule or the nonzero rule.

Clipping never enlarges the clipping area. Thus, when you clip against
a certain path and then clip again against another path, you clip
against the intersection of both.

The only way to enlarge the clipping path is to end the |{pgfscope}|
in which the clipping was done. At the end of a |{pgfscope}| the
clipping path that was in force at the beginning of the scope is
reinstalled. 

\subsection{Using a Path as a Bounding Box}
\label{section-using-bb}

When you add the |use as bounding box| option, the bounding box of the
picture will be enlarged such that the path in encompassed, but any
\emph{subsequent} paths of the current \TeX\ scope will not have any
effect on the size of the bounding box. Typically, you use this
command at the very beginning of a |{pgfpicture}| environment.

\begin{codeexample}[]
Left
\begin{pgfpicture}
  \pgfpathrectangle{\pgfpointorigin}{\pgfpoint{2ex}{1ex}}
  \pgfusepath{use as bounding box} % draws nothing

  \pgfpathcircle{\pgfpointorigin}{2ex}
  \pgfusepath{stroke}
\end{pgfpicture}
right.
\end{codeexample}


% Copyright 2006 by Till Tantau
%
% This file may be distributed and/or modified
%
% 1. under the LaTeX Project Public License and/or
% 2. under the GNU Free Documentation License.
%
% See the file doc/generic/pgf/licenses/LICENSE for more details.


\section{Arrow Tips}
\label{section-arrows}


\subsection{Overview}

\subsubsection{When Does PGF Draw Arrow Tips?}

\pgfname\ offers an interface for placing \emph{arrow tips} at the end
of lines. The interface works as follows:

\begin{enumerate}
\item
  You (or someone else) assigns a name to a certain kind of arrow
  tips. For example, the 
  arrow tip |latex| is the arrow tip used by the standard \LaTeX\
  picture environment; the arrow tip |to| looks like the tip of the
  arrow in \TeX's |\to| command; and so on.

  This is done once at the beginning of the document.
\item
  Inside some picture, at some point you specify that in the current
  scope from now on you would like tips of, say, kind |to| to be added
  at the end and/or beginning of all paths.

  When an arrow kind has been installed and when \pgfname\ is about to
  stroke a path, the following things happen:
  \begin{enumerate}
  \item
    The beginning and/or end of the path is shortened appropriately.
  \item
    The path is stroked.
  \item
    The arrow tip is drawn at the beginning and/or end of the path,
    appropriately rotated and appropriately resized.
  \end{enumerate}
\end{enumerate}

In the above description, there are a number of ``appropriately.''
The exact details are not quite trivial and described later on.

\subsubsection{Meta-Arrow Tips}

In \pgfname, arrows are ``meta-arrows'' in the same way that fonts in
\TeX\ are ``meta-fonts.'' When a meta-arrow is resized, it is not
simply scaled, but a possibly complicated transformation is applied to
the size.

A meta-font is not one particular font at a specific size with a
specific stroke width (and with a large number of other parameters
being fixed). Rather, it is a ``blueprint'' (actually, more like a
program) for generating such a font at a particular size and
width. This allows the designer of a meta-font to make sure that, say,
the font is somewhat thicker and wider at very small sizes. To
appreciate the difference: Compare the following texts: ``Berlin'' and
``\tikz{\node [scale=2,inner sep=0pt,outer sep=0pt]{\tiny
    Berlin};}''. The first is a ``normal'' text, the second is the tiny
version scaled by a factor of two. Obviously, the first look
better. Now, compare  ``\tikz{\node [scale=.5,inner sep=0pt,outer
  sep=0pt]{Berlin};}'' and ``{\tiny Berlin}''. This time, the normal
text was scaled down, while the second text is a ``normal'' tiny
text. The second text is easier to read. 

\pgfname's meta-arrows work in a similar fashion: The shape of an
arrow tip can vary according to the line width of the arrow tip is
used. Thus, an arrow tip drawn at a line width of 5pt will typically
\emph{not} be five times as large as an arrow tip of line width
1pt. Instead, the size of the arrow will get bigger only slowly as the
line width increases.

To appreciate the difference, here are the |latex| and |to| arrows, as
drawn by \pgfname\ at four different sizes:

\medskip
\begin{tikzpicture}
  \draw[-latex,line width=0.1pt] (0pt,0ex) -- +(3,0)  node[thin,right] {line width is 0.1pt};
  \draw[-latex,line width=0.4pt] (0pt,-2em) -- +(3,0) node[thin,right] {line width is 0.4pt};
  \draw[-latex,line width=1.2pt] (0pt,-4em) -- +(3,0) node[thin,right] {line width is 1.2pt};
  \draw[-latex,line width=5pt]   (0pt,-6em) -- +(3,0) node[thin,right] {line width is 5pt};

  \draw[-to,line width=0.1pt] (6cm,0ex) -- +(3,0)  node[thin,right] {line width is 0.1pt};
  \draw[-to,line width=0.4pt] (6cm,-2em) -- +(3,0) node[thin,right] {line width is 0.4pt};
  \draw[-to,line width=1.2pt] (6cm,-4em) -- +(3,0) node[thin,right] {line width is 1.2pt};
  \draw[-to,line width=5pt]   (6cm,-6em) -- +(3,0) node[thin,right] {line width is 5pt};
\end{tikzpicture}

\medskip
Here, by comparison, is the same arrow when it is simply ``resized''
(as done by most programs):

\pgfarrowsdeclare{bad latex}{bad latex}
{
  \pgfarrowsleftextend{-1\pgflinewidth}
  \pgfarrowsrightextend{9\pgflinewidth}
}
{
  \pgfpathmoveto{\pgfpoint{9\pgflinewidth}{0pt}}
  \pgfpathcurveto
  {\pgfpoint{6.3333\pgflinewidth}{.5\pgflinewidth}}
  {\pgfpoint{2\pgflinewidth}{2\pgflinewidth}}
  {\pgfpoint{-1\pgflinewidth}{3.75\pgflinewidth}}
  \pgfpathlineto{\pgfpoint{-1\pgflinewidth}{-3.75\pgflinewidth}}
  \pgfpathcurveto
  {\pgfpoint{2\pgflinewidth}{-2\pgflinewidth}}
  {\pgfpoint{6.3333\pgflinewidth}{-.5\pgflinewidth}}
  {\pgfpoint{9\pgflinewidth}{0pt}}
  \pgfusepathqfill
}

\pgfarrowsdeclare{bad to}{bad to}
{
  \pgfarrowsleftextend{-2\pgflinewidth}
  \pgfarrowsrightextend{\pgflinewidth}
}
{
  \pgfsetlinewidth{0.8\pgflinewidth}
  \pgfsetdash{}{0pt}
  \pgfsetroundcap
  \pgfsetroundjoin
  \pgfpathmoveto{\pgfpoint{-3\pgflinewidth}{4\pgflinewidth}}
  \pgfpathcurveto
  {\pgfpoint{-2.75\pgflinewidth}{2.5\pgflinewidth}}
  {\pgfpoint{0pt}{0.25\pgflinewidth}}
  {\pgfpoint{0.75\pgflinewidth}{0pt}}
  \pgfpathcurveto
  {\pgfpoint{0pt}{-0.25\pgflinewidth}}
  {\pgfpoint{-2.75\pgflinewidth}{-2.5\pgflinewidth}}
  {\pgfpoint{-3\pgflinewidth}{-4\pgflinewidth}}
  \pgfusepathqstroke
}

\medskip
\begin{tikzpicture}
  \draw[-bad latex,line width=0.1pt] (0pt,0ex) -- +(3,0)  node[thin,right] {line width is 0.1pt};
  \draw[-bad latex,line width=0.4pt] (0pt,-2em) -- +(3,0) node[thin,right] {line width is 0.4pt};
  \draw[-bad latex,line width=1.2pt] (0pt,-4em) -- +(3,0) node[thin,right] {line width is 1.2pt};
  \draw[-bad latex,line width=5pt]   (0pt,-6em) -- +(3,0) node[thin,right] {line width is 5pt};

  \draw[-bad to,line width=0.1pt] (6cm,0ex) -- +(3,0)  node[thin,right] {line width is 0.1pt};
  \draw[-bad to,line width=0.4pt] (6cm,-2em) -- +(3,0) node[thin,right] {line width is 0.4pt};
  \draw[-bad to,line width=1.2pt] (6cm,-4em) -- +(3,0) node[thin,right] {line width is 1.2pt};
  \draw[-bad to,line width=5pt]   (6cm,-6em) -- +(3,0) node[thin,right] {line width is 5pt};
\end{tikzpicture}

\bigskip
As can be seen, simple scaling produces arrow tips that are way too
large at larger sizes and way too small at smaller sizes.



\subsection{Declaring an Arrow Tip Kind}

To declare an arrow kind ``from scratch,'' the following command is
used:

\begin{command}{\pgfarrowsdeclare\marg{start name}\marg{end
      name}\marg{extend code}\marg{arrow tip code}}
  This command declares a new arrow kind. An arrow kind has two names,
  which will typically be the same. When the arrow tip needs to be
  drawn, the \meta{arrow tip code} will be invoked, but the canvas
  transformation is setup beforehand to a rotation such that when an
  arrow tip pointing right is specified, the arrow tip that is
  actually drawn points in the direction of the line.

  \medskip
  \textbf{Naming the arrow kind.}
  The \meta{start name} is the name
  used for the arrow tip when it is at the start of a path, the \meta{end
    name} is the name used at the end of a path. For example, the
  arrow kind that looks like a parenthesis has the \meta{start
    name} |(| and the \meta{end name} |)| so that you can say
  |\pgfsetarrows{(-)}| to specify that you want parenthesis arrows and
  both ends.

  The \meta{end name} and \meta{start name} can be quite arbitrary and
  may contain spaces.

  \medskip
  \textbf{Basics of the arrow tip code.}
  Let us next have a look at the \meta{arrow tip code}. This code will
  be used to draw the arrow tip when \pgfname\ thinks this is
  necessary. The code should draw an arrow that ``points right,''
  which means that is should draw an arrow at the end of a line coming
  from the left and ending at the origin.

  As an example, suppose we wanted to declare an arrow tip consisting
  of two arcs, that is, we want the arrow tip to look more or less
  like the red part of the following picture:
\begin{codeexample}[]
\begin{tikzpicture}[line width=3pt]
  \draw (-2,0) -- (0,0);
  \draw[red,join=round,cap=round]
        (-10pt,10pt) arc (180:270:10pt) arc (90:180:10pt);
\end{tikzpicture}
\end{codeexample}

  We could use the following as \meta{arrow tip code} for this:
\begin{codeexample}[code only]
\pgfarrowsdeclare{arcs}{arcs}{...}
{
  \pgfsetdash{}{0pt} % do not dash
  \pgfsetroundjoin   % fix join
  \pgfsetroundcap    % fix cap
  \pgfpathmoveto{\pgfpoint{-10pt}{10pt}}
  \pgfpatharc{180}{270}{10pt}
  \pgfpatharc{90}{180}{10pt}
  \pgfusepathqstroke
}
\end{codeexample}

  Indeed, when the |...| is set appropriately (in a moment), we can
  write the following:
\pgfarrowsdeclare{arcs}{arcs}{\pgfarrowsleftextend{0pt}\pgfarrowsrightextend{0pt}}
{
  \pgfsetdash{}{0pt} % do not dash
  \pgfsetroundjoin   % fix join
  \pgfsetroundcap    % fix cap
  \pgfpathmoveto{\pgfpoint{-10pt}{10pt}}
  \pgfpatharc{180}{270}{10pt}
  \pgfpatharc{90}{180}{10pt}
  \pgfusepathqstroke
}
\begin{codeexample}[]
\begin{tikzpicture}
  \draw[-arcs,line width=3pt] (-2,0)  -- (0,0);
  \draw[arcs-arcs,line width=1pt] (-2,-1.5) -- (0,-1);
  \useasboundingbox (-2,-2) rectangle (0,0.75);
\end{tikzpicture}
\end{codeexample}

  As can be seen in the second example, the arrow tip is automatically
  rotated as needed when the arrow is drawn. This is achieved by a
  canvas rotation.

  \medskip
  \textbf{Special considerations about the arrow tip code.}
  There are several things you need to be aware of when designing
  arrow tip code:
  \begin{itemize}
  \item
    Inside the code, you may not use the |\pgfusepath|
    command. The reason is that this command internally calls arrow
    construction commands, which is something you obviously do not want
    to happen.

    Instead of |\pgfusepath|, use the quick versions. Typically, you
    will use |\pgfusepathqstroke|, |\pgfusepathqfill|, or
    |\pgfusepathqfillstroke|.
  \item
    The code will be executed only once, namely the first time the
    arrow tip needs to be drawn. The resulting low-level driver
    commands are protocoled and stored away. In all subsequent 
    uses of the arrow tip, the protocoled code is directly inserted.
  \item
    However, the code will be executed anew for each line width. Thus,
    an arrow of line width 2pt may result in a different protocol than
    the same arrow for a line width of 0.4pt.
  \item
    If you stroke the path that you construct, you should first set
    the dashing to solid and setup fixed joins and caps, as
    needed. This will ensure that the arrow tip will always look the
    same.
  \item
    When the arrow tip code is executed, it is automatically put
    inside a low-level scope, so nothing will ``leak out'' from the
    scope.
  \item
    The high-level coordinate transformation matrix will be set to the
    identity matrix when the code is executed for the first time.
  \end{itemize}

  \medskip
  \textbf{Designing meta-arrows.}
  The \meta{arrow tip code} should adjust the size of the arrow in
  accordance with the line width. For a small line width, the arrow
  tip should be small, for a large line width, it should be
  larger. However, the size of the arrow typically \emph{should not}
  grow in direct proportion to the line width. On the other hand, the
  size of the arrow head typically \emph{should} grow ``a bit'' with
  the line width. 

  For these reasons, \pgfname\ will not simply executed your arrow
  code within a scaled scope, where the scaling depends on the line
  width. Instead, your \meta{arrow tip code} is reexecuted again for
  each different line width.

  In our example, we could use the following code for the new arrow
  tip kind |arc'| (note the prime):
\begin{codeexample}[code only]
\newdimen\arrowsize    
\pgfarrowsdeclare{arcs'}{arcs'}{...}
{
  \arrowsize=0.2pt
  \advance\arrowsize by .5\pgflinewidth
  \pgfsetdash{}{0pt} % do not dash
  \pgfsetroundjoin   % fix join
  \pgfsetroundcap    % fix cap
  \pgfpathmoveto{\pgfpoint{-4\arrowsize}{4\arrowsize}}
  \pgfpatharc{180}{270}{4\arrowsize}
  \pgfpatharc{90}{180}{4\arrowsize}
  \pgfusepathqstroke
}
\end{codeexample}
\newdimen\arrowsize    
\pgfarrowsdeclare{arcs'}{arcs'}{\pgfarrowsleftextend{0pt}\pgfarrowsrightextend{0pt}}
{
  \arrowsize=0.2pt
  \advance\arrowsize by .5\pgflinewidth
  \pgfsetdash{}{0pt} % do not dash
  \pgfsetroundjoin   % fix join
  \pgfsetroundcap    % fix cap
  \pgfpathmoveto{\pgfpoint{-4\arrowsize}{4\arrowsize}}
  \pgfpatharc{180}{270}{4\arrowsize}
  \pgfusepathqstroke
  \pgfpathmoveto{\pgfpointorigin}
  \pgfpatharc{90}{180}{4\arrowsize}
  \pgfusepathqstroke
}
\begin{codeexample}[]
\begin{tikzpicture}
  \draw[-arcs',line width=3pt] (-2,0)  -- (0,0);
  \draw[arcs'-arcs',line width=1pt] (-2,-1.5) -- (0,-1);
  \useasboundingbox (-2,-1.75) rectangle (0,0.5);
\end{tikzpicture}
\end{codeexample}
  
  However, sometimes, it can also be useful to have arrows that do not
  resize at all when the line width changes. This can be achieved by
  giving absolute size coordinates in the code, as done for |arc|. On
  the other hand, you can also have the arrow resize linearly with the
  line width by specifying all coordinates as multiples of
  |\pgflinewidth|.

  \textbf{The left and right extend.}
  Let us have another look at the exact left and right ``ends'' of our
  arrow tip. Let us draw the arrow tip |arc'| at a very large size:

\begin{codeexample}[]
\begin{tikzpicture}
  \draw[help lines] (-2,-1) grid (1,1);
  \draw[line width=10pt,-arcs'] (-2,0) -- (0,0);
  \draw[line width=2pt,white] (-2,0) -- (0,0);
\end{tikzpicture}
\end{codeexample}

  As one can see, the arrow tip does not ``touch'' the origin as it
  should, but protrudes a little over the origin. One remedy to this
  undesirable effect is to change the code of the arrow tip such that
  everything is shifted half an |\arrowsize| to the left. While this
  will cause the arrow tip to touch the origin, the line itself will
  then interfere with the arrow: The arrow tip will be partly
  ``hidden'' by the line itself.

  \pgfname\ uses a different approach to solving the problem: The
  \meta{extend code} argument can be used to ``tell'' \pgfname\ how
  much the arrow protrudes over the origin. The argument is also used
  to tell \pgfname\ where the ``left'' end of the arrow is. However,
  this number is important only when the arrow is being reversed or
  composed with other arrow tips.

  Once \pgfname\ knows the right extend of an arrow kind, it can
  \emph{shorten} lines by this amount when drawing arrows.

  Here is a picture that shows what the visualizes the extends. The
  arrow tip itself is shown in red once more:

  \medskip
  \begin{tikzpicture}
    \draw[line width=1cm,-arcs',red] (-6,0) -- (0,0);
    \draw[line width=1cm,black]      (-6,0) -- (0,0);
    \draw[help lines] (-6,0) -- (2,0)     (0,-3) -- (0,3) coordinate (a);
    \draw[help lines,xshift=0.5cm]        (0,-3) -- (0,3) coordinate (b);
    \draw[help lines,xshift=-2.5cm-0.8pt] (0,-3) -- (0,3) coordinate (c);

    \coordinate (xline 1) at (0,1.5);
    \coordinate (xline 2) at (0,2.8);
    
    \draw[|->|] (xline 1 -| a) -- node[above=2pt] {right extend} (xline 1 -| b);    
    \draw[|<-|] (xline 2 -| c) -- node[above=2pt] {left extend}  (xline 2 -| a);

    \draw (0,0) -- (1,-1) node[below right] {origin};
   \end{tikzpicture}
  

  The \meta{extend code} is normal \TeX\ code that is executed
  whenever \pgfname\ wants to know how far the arrow tip will protrude
  to the right and left. The code should call the following two
  commands: \declare{|\pgfarrowsrightextend|} and
  \declare{|\pgfarrowsleftextend|}. Both arguments take one argument
  that specifies the size. Here is the final code for the |arc''| arrow
  tip: 
\begin{codeexample}[]
\pgfarrowsdeclare{arcs''}{arcs''}
{
  \arrowsize=0.2pt
  \advance\arrowsize by .5\pgflinewidth
  \pgfarrowsleftextend{-4\arrowsize-.5\pgflinewidth}
  \pgfarrowsrightextend{.5\pgflinewidth}
}
{
  \arrowsize=0.2pt
  \advance\arrowsize by .5\pgflinewidth
  \pgfsetdash{}{0pt} % do not dash
  \pgfsetroundjoin   % fix join
  \pgfsetroundcap    % fix cap
  \pgfpathmoveto{\pgfpoint{-4\arrowsize}{4\arrowsize}}
  \pgfpatharc{180}{270}{4\arrowsize}
  \pgfusepathqstroke
  \pgfpathmoveto{\pgfpointorigin}
  \pgfpatharc{90}{180}{4\arrowsize}
  \pgfusepathqstroke
}
\begin{tikzpicture}
  \draw[help lines] (-2,-1) grid (1,1);
  \draw[line width=10pt,-arcs''] (-2,0) -- (0,0);
  \draw[line width=2pt,white] (-2,0) -- (0,0);
\end{tikzpicture}
\end{codeexample}
\end{command}

\pgfarrowsdeclare{arcs''}{arcs''}
{
  \arrowsize=0.2pt
  \advance\arrowsize by .5\pgflinewidth
  \pgfarrowsleftextend{-4\arrowsize-.5\pgflinewidth}
  \pgfarrowsrightextend{.5\pgflinewidth}
}
{
  \arrowsize=0.2pt
  \advance\arrowsize by .5\pgflinewidth
  \pgfsetdash{}{0pt} % do not dash
  \pgfsetroundjoin   % fix join
  \pgfsetroundcap    % fix cap
  \pgfpathmoveto{\pgfpoint{-4\arrowsize}{4\arrowsize}}
  \pgfpatharc{180}{270}{4\arrowsize}
  \pgfusepathqstroke
  \pgfpathmoveto{\pgfpointorigin}
  \pgfpatharc{90}{180}{4\arrowsize}
  \pgfusepathqstroke
}


\subsection{Declaring a Derived Arrow Tip Kind}

It is possible to declare arrow kinds in terms of existing ones. For
these command to work correctly, the left and right extends must be
set correctly.

\begin{command}{\pgfarrowsdeclarealias\marg{start name}\marg{end
      name}\marg{old start name}\marg{old end name}}
  This command can be used to create an alias (another name) for an
  existing arrow kind.

\begin{codeexample}[]
\pgfarrowsdeclarealias{<}{>}{arcs''}{arcs''}%
\begin{tikzpicture}
  \pgfsetarrows{<->}
  \pgfsetlinewidth{1ex}
  \pgfpathmoveto{\pgfpointorigin}
  \pgfpathlineto{\pgfpoint{3.5cm}{2cm}}
  \pgfusepath{stroke}
  \useasboundingbox (-0.25,-0.25) rectangle (3.75,2.25);
\end{tikzpicture}
\end{codeexample}
\end{command}


\begin{command}{\pgfarrowsdeclarereversed\marg{start name}\marg{end
      name}\marg{old start name}\marg{old end name}}
  This command creates a new arrow kind that is the ``reverse'' of an
  existing arrow kind. The (automatically cerated) code of the new
  arrow kind will contain a flip of the canvas and the meanings of the
  left and right extend will be reversed. 

\begin{codeexample}[]
\pgfarrowsdeclarereversed{arcs reversed}{arcs reversed}{arcs''}{arcs''}%
\begin{tikzpicture}
  \pgfsetarrows{arcs reversed-arcs reversed}
  \pgfsetlinewidth{1ex}
  \pgfpathmoveto{\pgfpointorigin}
  \pgfpathlineto{\pgfpoint{3.5cm}{2cm}}
  \pgfusepath{stroke}
  \useasboundingbox (-0.25,-0.25) rectangle (3.75,2.25);
\end{tikzpicture}
\end{codeexample}
\end{command}



\begin{command}{\pgfarrowsdeclarecombine\opt{|*|}\opt{\oarg{offset}}\marg{start
      name}\marg{end name}\marg{first start name}\marg{first end
      name}\penalty0\marg{second start name}\marg{second end name}}
  This command creates a new arrow kind that combines two existing
  arrow kinds. The first arrow kind is the ``innermost'' arrow kind,
  the second arrow kind is the ``outermost.''

  The code for the combined arrow kind will install a canvas
  translation before the innermost arrow kind in drawn. This
  translation is calculated such that the right tip of the innermost
  arrow touches the right  end of the outermost arrow. The optional
  \meta{offset} can be used to increase (or decrease) the distance
  between the inner and outermost arrow.

\begin{codeexample}[]
\pgfarrowsdeclarecombine[\pgflinewidth]
  {combined}{combined}{arcs''}{arcs''}{latex}{latex}%
\begin{tikzpicture}
  \pgfsetarrows{combined-combined}
  \pgfsetlinewidth{1ex}
  \pgfpathmoveto{\pgfpointorigin}
  \pgfpathlineto{\pgfpoint{3.5cm}{2cm}}
  \pgfusepath{stroke}
  \useasboundingbox (-0.25,-0.25) rectangle (3.75,2.25);
\end{tikzpicture}
\end{codeexample}

  In the star variant, the end of the line is not in the outermost
  arrow, but inside the innermost arrow.

\begin{codeexample}[]
\pgfarrowsdeclarecombine*[\pgflinewidth]
  {combined'}{combined'}{arcs''}{arcs''}{latex}{latex}%
\begin{tikzpicture}
  \pgfsetarrows{combined'-combined'}
  \pgfsetlinewidth{1ex}
  \pgfpathmoveto{\pgfpointorigin}
  \pgfpathlineto{\pgfpoint{3.5cm}{2cm}}
  \pgfusepath{stroke}
  \useasboundingbox (-0.25,-0.25) rectangle (3.75,2.25);
\end{tikzpicture}
\end{codeexample}
\end{command}


\begin{command}{\pgfarrowsdeclaredouble\opt{\oarg{offset}}\marg{start
      name}\marg{end name}\marg{old start name}\marg{old end
      name}}
  This command is a shortcut for combining an arrow kind with itself.

\begin{codeexample}[]
\pgfarrowsdeclaredouble{<<}{>>}{arcs''}{arcs''}%
\begin{tikzpicture}
  \pgfsetarrows{<<->>}
  \pgfsetlinewidth{1ex}
  \pgfpathmoveto{\pgfpointorigin}
  \pgfpathlineto{\pgfpoint{3.5cm}{2cm}}
  \pgfusepath{stroke}
  \useasboundingbox (-0.25,-0.25) rectangle (3.75,2.25);
\end{tikzpicture}
\end{codeexample} 
\end{command}


\begin{command}{\pgfarrowsdeclaretriple\opt{\oarg{offset}}\marg{start
      name}\marg{end name}\marg{old start name}\marg{old end
      name}}
  This command is a shortcut for combining an arrow kind with itself
  and then again.

\begin{codeexample}[]
\pgfarrowsdeclaretriple{<<<}{>>>}{arcs''}{arcs''}%
\begin{tikzpicture}
  \pgfsetarrows{<<<->>>}
  \pgfsetlinewidth{1ex}
  \pgfpathmoveto{\pgfpointorigin}
  \pgfpathlineto{\pgfpoint{3.5cm}{2cm}}
  \pgfusepath{stroke}
  \useasboundingbox (-0.25,-0.25) rectangle (3.75,2.25);
\end{tikzpicture}
\end{codeexample} 
\end{command}





\subsection{Using an Arrow Tip Kind}

The following commands install the arrow kind that will be used when
stroking is done.

\begin{command}{\pgfsetarrowsstart\marg{start arrow kind}}
  Installs the given \meta{start arrow kind} for all subsequent
  strokes in the in the current \TeX-group. If \meta{start arrow kind}
  is empty, no arrow tips will be drawn at the start of the last
  segment of paths.
\begin{codeexample}[]
\begin{tikzpicture}
  \pgfsetarrowsstart{latex}
  \pgfsetlinewidth{1ex}
  \pgfpathmoveto{\pgfpointorigin}
  \pgfpathlineto{\pgfpoint{3.5cm}{2cm}}
  \pgfusepath{stroke}
  \useasboundingbox (-0.25,-0.25) rectangle (3.75,2.25);
\end{tikzpicture}
\end{codeexample} 
\end{command}

\begin{command}{\pgfsetarrowsend\marg{start arrow kind}}
  Like |\pgfsetarrowsstart|, only for the end of the arrow.
\begin{codeexample}[]
\begin{tikzpicture}
  \pgfsetarrowsend{latex}
  \pgfsetlinewidth{1ex}
  \pgfpathmoveto{\pgfpointorigin}
  \pgfpathlineto{\pgfpoint{3.5cm}{2cm}}
  \pgfusepath{stroke}
  \useasboundingbox (-0.25,-0.25) rectangle (3.75,2.25);
\end{tikzpicture}
\end{codeexample} 
\end{command}

\emph{Warning:} If the compatibility mode is active (which is the
default), there also exist old commands called |\pgfsetstartarrow| and 
|\pgfsetendarrow|, which are incompatible with the meta-arrow
management.


\begin{command}{\pgfsetarrows\texttt{\char`\{}\meta{start kind}|-|\meta{end kind}\texttt{\char`\}}}
  Calls |\pgfsetarrowsstart| for \meta{start kind} and
  |\pgfsetarrowsend| for \meta{end kind}.
\begin{codeexample}[]
\begin{tikzpicture}
  \pgfsetarrows{latex-to}
  \pgfsetlinewidth{1ex}
  \pgfpathmoveto{\pgfpointorigin}
  \pgfpathlineto{\pgfpoint{3.5cm}{2cm}}
  \pgfusepath{stroke}
  \useasboundingbox (-0.25,-0.25) rectangle (3.75,2.25);
\end{tikzpicture}
\end{codeexample} 
\end{command}


\subsection{Predefined Arrow Tip Kinds}

\label{standard-arrows}

The following arrow tip kinds are always defined:

{
\bigskip
\catcode`\|=12
\begin{tabular}{ll}
  \sarrow{stealth}{stealth} \\
  \sarrow{stealth reversed}{stealth reversed}  \\
  \sarrow{to}{to} \\
  \sarrow{to reversed}{to reversed}  \\
  \sarrow{latex}{latex} \\
  \sarrow{latex reversed}{latex reversed}  \\
  \index{*vbar@\protect\texttt{\protect\myvbar} arrow tip}%
  \index{Arrow tips!*vbar@\protect\texttt{\protect\myvbar}}
  \texttt{|-|}& yields thick  
  \begin{tikzpicture}[arrows={|-|},thick]
    \useasboundingbox (0pt,-0.5ex) rectangle (1cm,2ex);
    \draw (0,0) -- (1,0);
  \end{tikzpicture} and thin
  \begin{tikzpicture}[arrows={|-|},thin]
    \useasboundingbox (0pt,-0.5ex) rectangle (1cm,2ex);
    \draw (0,0) -- (1,0);
  \end{tikzpicture}
\end{tabular}
}

For further arrow tips, see page~\pageref{section-library-arrows}.

%%% Local Variables: 
%%% mode: latex
%%% TeX-master: "pgfmanual"
%%% End: 

% Copyright 2006 by Till Tantau
%
% This file may be distributed and/or modified
%
% 1. under the LaTeX Project Public License and/or
% 2. under the GNU Free Documentation License.
%
% See the file doc/generic/pgf/licenses/LICENSE for more details.


\section{Nodes and Shapes}

\label{section-shapes}

This section describes the |shapes| module.

\begin{pgfmodule}{shapes}
  This module defines commands both for creating nodes and for
  creating shapes. The package is loaded automatically by |pgf|, but
  you can load it manually if you have  only included |pgfcore|.  
\end{pgfmodule}


\subsection{Overview}

\pgfname\ comes with a sophisticated set of commands for creating
\emph{nodes} and \emph{shapes}. A \emph{node} is a graphical object
that consists (typically) of (one or more) text labels and some
additional stroked or filled paths. Each node has a certain
\emph{shape}, which may be something simple like a |rectangle| or a
|circle|, but it may also be something complicated like a
|uml class diagram| (this shape is currently not implemented,
though). Different nodes that have the same shape may look quite
different, however, since shapes (need not) specify whether the shape
path is stroked or filled.


\subsubsection{Creating and Referencing Nodes}

You create a node by calling the macro |\pgfnode| or the more general
|\pgfmultipartnode|. These macro takes several parameters and draws
the requested shape at a certain position. In addition, it will
``remember'' the node's position within the current
|{pgfpicture}|. You can then, later on, refer to the 
node's position. Coordinate transformations are ``fully supported,''
which means that if you used coordinate transformations to shift or
rotate the shape of a node, the node's position will still be correctly
determined by \pgfname. This is \emph{not} the case if you use canvas
transformations, instead.


\subsubsection{Anchors}

An important property of a node or a shape in general are its
\emph{anchors}. Anchors are ``important'' positions in a shape. For
example, the |center| anchor lies at the center of a shape, the
|north| anchor is usually ``at the top, in the middle'' of a shape,
the |text| anchor is the lower left corner of the shape's text label
(if present), and so on.

Anchors are important both when you create a node and when you
reference it. When you create a node, you specify the node's
``position'' by asking \pgfname\ to place the shape in such a way that
a certain anchor lies at a certain point. For example, you might ask
that the node is placed such that the |north| anchor is at the
origin. This will effectively cause the node to be placed below the
origin.

When you reference a node, you always reference an anchor of the
node. For example, when you request the ``|north| anchor of the node
just placed'' you will get the origin. However, you can also request
the ``|south| anchor of this node,'' which will give you a point
somewhere below the origin. When a coordinate transformation was in
force at the time of creation of a node, all anchors are also
transformed accordingly.

\subsubsection{Layers of a Shape}

The simplest shape, the |coordinate|, has just one anchor, namely the
|center|, and a label (which is usually empty). More complicated
shapes like the |rectangle| shape also have a \emph{background
  path}. This is a \pgfname-path that is defined by the shape. The
shape does not prescribe what should happen with the path: When a node
is created this path may be stroked (resulting in a frame around the
label), filled (resulting in a background color for the text), or just
discarded.

Although most shapes consist just of a background path plus some label
text, when a shape is drawn, up to seven different layers are drawn:

\begin{enumerate}
\item
  The ``behind the background layer.'' Unlike the background path,
  which be used in different ways by different nodes, the graphic
  commands given for this layer will always stroke or
  always fill the path they construct. They might also insert some
  text that is ``behind everything.''
\item
  The background path layer. How this path is used depends on how the
  arguments of the |\pgfnode| command.
\item
  The ``before the background path layer.'' This layer works like the
  first one, only the commands of this layer are executed after the
  background path has been used (in whatever way the creator of the
  node chose).
\item
  The label layer. This layer inserts the node's text box(es).
\item
  The ``behind the foreground layer.'' This layer, like the
  first layer, once more contains graphic commands that are ``simply
  executed.''
\item
  The foreground path layer. This path is treated in the same way as the
  background path, only it is drawn only after the label text has been
  drawn.
\item
  The ``before the foreground layer.''
\end{enumerate}

Which of these layers are actually used depends on the shape.



\subsubsection{Node Parts}

A shape typically does not consist only of different background and
foreground paths, but it may also have text labels. Indeed, for many
shapes the text labels are the more important part of the shape.

Most shapes will have only one text label. In this case, this text
label is simply passed as a parameter to the |\pgfnode| command. When
the node is drawn, the text label is shifted around such that its
lower left corner is at the |text| anchor of the node.

More complicated shapes may have more than one text label. Nodes of
such shapes are called \emph{multipart nodes}. The different
\emph{node parts} are simply the different text labels. For example, a
|uml class| shape might have a |class name| part, a |method| part and
an |attributes| part. Indeed, single part nodes are a special case of
multipart nodes: They only have one part named |text|.

When a shape is declared, you must specify the node parts. There is a
simple command called |\nodeparts| that takes a list of the part names
as input. When you create a node of a multipart shape, for each part
of the node you must have setup a \TeX-box containing the text of the
part. For a part named |XYZ| you must setup the box
|\pgfnodepartXYZbox|. The box will be placed at the anchor |XYZ|. See
the description of |\pgfmultipartnode| for more details.


\subsection{Creating Nodes}

You create a node using on of the following commands:

\begin{command}{\pgfnode\marg{shape}\marg{anchor}\marg{label
      text}\marg{name}\marg{path usage command}} 
  This command creates a new node. The \meta{shape} of the node must
  have been declared previously using |\pgfdeclareshape|.

  The shape is shifted such that the \meta{anchor} is at the
  origin. In order to place the shape somewhere else, use the
  coordinate transformation prior to calling this command.

  The \meta{name} is a name for later reference. If no name is given,
  nothing will be ``saved'' for the node, it will just be drawn.

  The \meta{path usage command} is executed for the background and the
  foreground path (if the shape defines them).

\begin{codeexample}[]
\begin{tikzpicture}
  \draw[help lines] (0,0) grid (4,3);
  {
    \pgftransformshift{\pgfpoint{1cm}{1cm}}
    \pgfnode{rectangle}{north}{Hello World}{hellonode}{\pgfusepath{stroke}}
  }
  {
    \color{red!20}
    \pgftransformrotate{10}
    \pgftransformshift{\pgfpoint{3cm}{1cm}}
    \pgfnode{rectangle}{center}
      {\color{black}Hello World}{hellonode}{\pgfusepath{fill}}
  }
\end{tikzpicture}
\end{codeexample}

  As can be seen, all coordinate transformations are also applied to
  the text of the shape. Sometimes, it is desirable that the
  transformations are applied to the point where the shape will be
  anchored, but you do not wish the shape itself to the
  transformed. In this case, you should call
  |\pgftransformresetnontranslations| prior to calling the |\pgfnode|
  command. 

\begin{codeexample}[]
\begin{tikzpicture}
  \draw[help lines] (0,0) grid (4,3);
  {
    \color{red!20}
    \pgftransformrotate{10}
    \pgftransformshift{\pgfpoint{3cm}{1cm}}
    \pgftransformresetnontranslations
    \pgfnode{rectangle}{center}
      {\color{black}Hello World}{hellonode}{\pgfusepath{fill}}
  }
\end{tikzpicture}
\end{codeexample}

  The \meta{label text} is typeset inside the \TeX-box
  |\pgfnodeparttextbox|. This box is shown at the |text| anchor of the
  node, if the node has a |text| part. See the description of
  |\pgfmultipartnode| for details.
\end{command}

\begin{command}{\pgfmultipartnode\marg{shape}\marg{anchor}\marg{name}\marg{path
      usage command}}
  This command is the more general (and less user-friendly) version of
  the |\pgfnode| command. While the |\pgfnode| command can only be
  used for shapes that have a single part (which is the case for most
  shapes), this command can also be used with multi-part nodes.

  When this command is called, for each node part of the node you must
  have setup one \TeX-box. Suppose the shape has two parts: The |text|
  part and the |lower| part. Then, prior to calling
  |\pgfmultipartnode|, you must have setup the boxes
  |\pgfnodeparttextbox| and |\pgfnodepartlowerbox|. These boxes may 
  contain any \TeX-text. The shape code will then compute the
  positions of the shape's anchors based on the sizes of the these
  shapes. Finally, when the node is drawn, the boxes are placed at the
  anchor positions |text| and |lower|.

\begin{codeexample}[]
\setbox\pgfnodeparttextbox=\hbox{$q_1$}
\setbox\pgfnodepartlowerbox=\hbox{01}
\begin{pgfpicture}
  \pgfmultipartnode{circle split}{center}{my state}{\pgfusepath{stroke}}
\end{pgfpicture}
\end{codeexample}

  \emph{Note:\/} Be careful when using the |\setbox| command inside a
  |{pgfpicture}| command. You will have to use |\pgfinterruptpath| at
  the beginning of the box and |\endpgfinterruptpath| at the end of
  the box to make sure that the box is typeset correctly. In the above
  example this problem was sidestepped by moving the box construction
  outside the environment.

  \emph{Note:\/} It is not necessary to use |\newbox| for every node
  part name. Although you need a different box for each part of a
  single shape, two different shapes may very well use the same box
  even when the names of the parts are different. Suppose you have a
  |circle split| shape that has an |lower| part and you have a
  |uml class| shape that has a |methods| part. Then, in order to avoid
  exhausting \TeX's limited number of box registers, you can say
\begin{codeexample}[code only]
\newbox\pgfnodepartlowerbox
\let\pgfnodepartmethodsbox=\pgfnodepartlowerbox  
\end{codeexample}
  Also, when you have a node part name with spaces like |class name|,
  it may be useful to create an alias:
\begin{codeexample}[code only]
\newbox\mybox
\expandafter\let\csname pgfnodepartclass namebox\endcsname=\mybox
\end{codeexample}
\end{command}

There are a number of values that have an influence on the size of a
node. These values are stored in the following keys.

\begin{key}{/pgf/minimum width=\meta{dimension} (initially 1pt)}
  \keyalias{tikz}
  This key stores the \emph{recommended} minimum width of a
  shape. Thus, when a shape is drawn and when the shape's width would
  be smaller than \meta{dimension}, the shape's width is enlarged by
  adding some empty space. 

  Note that this value is just a recommendation. A shape may choose to
  ignore this key.
  
\begin{codeexample}[]
\begin{tikzpicture}
  \draw[help lines] (-2,0) grid (2,1);

  \pgfset{minimum width=3cm}
  \pgfnode{rectangle}{center}{Hello World}{}{\pgfusepath{stroke}}
\end{tikzpicture}
\end{codeexample}
\end{key}

\begin{key}{/pgf/minimum height=\meta{dimension} (initially 1pt)}
  \keyalias{tikz}
  Works like |/pgf/minimum width|.
\end{key}

\begin{key}{/pgf/minimum size=\meta{dimension}}
  \keyalias{tikz}
  This sytle both |/pgf/minimum width| and |/pgf/minimum height| to \meta{dimension}.
\end{key}


\begin{key}{/pgf/inner xsep=\meta{dimension} (initially 0.3333em)}
  \keyalias{tikz}
  This key stores the \emph{recommended} horizontal
  inner separation between the label text and the background path. As
  before, this value is just a recommendation and a shape may choose
  to ignore this key.
  
\begin{codeexample}[]
\begin{tikzpicture}
  \draw[help lines] (-2,0) grid (2,1);

  \pgfset{inner xsep=1cm}
  \pgfnode{rectangle}{center}{Hello World}{}{\pgfusepath{stroke}}
\end{tikzpicture}
\end{codeexample}
\end{key}

\begin{key}{/pgf/inner ysep=\meta{dimension} (initially 0.3333em)}
  \keyalias{tikz}
  Works like |/pgf/inner xsep|.
\end{key}

\begin{key}{/pgf/inner sep=\meta{dimension}}
  \keyalias{tikz}
  This style sets both |/pgf/inner xsep| and |/pgf/inner ysep| to \meta{dimension}.
\end{key}



\begin{key}{/pgf/outer xsep=\meta{dimension} (initially .5\string\pgflinewidth)}
  \keyalias{tikz}
  This key stores the recommended horizontal separation between the
  background path and the ``outer anchors.'' For example, if
  \meta{dimension} is |1cm| then the |east| anchor will be 1cm to the
  right of the right border of the background path. 
  As before, this value is just a recommendation.
  
\begin{codeexample}[]
\begin{tikzpicture}
  \draw[help lines] (-2,0) grid (2,1);

  \pgfset{outer xsep=.5cm}
  \pgfnode{rectangle}{center}{Hello World}{x}{\pgfusepath{stroke}}

  \pgfpathcircle{\pgfpointanchor{x}{north}}{2pt}
  \pgfpathcircle{\pgfpointanchor{x}{south}}{2pt}
  \pgfpathcircle{\pgfpointanchor{x}{east}}{2pt}
  \pgfpathcircle{\pgfpointanchor{x}{west}}{2pt}
  \pgfpathcircle{\pgfpointanchor{x}{north east}}{2pt}
  \pgfusepath{fill}
\end{tikzpicture}
\end{codeexample}
\end{key}

\begin{key}{/pgf/outer ysep=\meta{dimension} (initially .5\string\pgflinewidth)}
  \keyalias{tikz}
  Works like |/pgf/outer xsep|.
\end{key}

\begin{key}{/pgf/outer sep=\meta{dimension}}
  \keyalias{tikz}
  This style sets both |/pgf/outer xsep| and |/pgf/outer ysep| to \meta{dimension}.
\end{key}



\subsection{Using Anchors}

Each shape defines a set of anchors. We saw already that the anchors
are used when the shape is drawn: the shape is placed in such a way
that the given anchor is at the origin (which in turn is typically
translated somewhere else).

One has to look up the set of anchors of each shape, there is no
``default'' set of anchors, except for the |center| anchor, which
should always be present. Also, most shapes will declare anchors like
|north| or |east|, but this is not guaranteed.


\subsubsection{Referencing Anchors of Nodes in the Same Picture}

Once a node has been defined, you can refer to its anchors using the
following commands:

\begin{command}{\pgfpointanchor\marg{node}\marg{anchor}}
  This command is another ``point command'' like the commands
  described in Section~\ref{section-points}. It returns the coordinate
  of the given \meta{anchor} in the given \meta{node}. The command can
  be used in commands like |\pgfpathmoveto|.

\begin{codeexample}[]
\begin{pgfpicture}
  \pgftransformrotate{30}
  \pgfnode{rectangle}{center}{Hello World!}{x}{\pgfusepath{stroke}}

  \pgfpathcircle{\pgfpointanchor{x}{north}}{2pt}
  \pgfpathcircle{\pgfpointanchor{x}{south}}{2pt}
  \pgfpathcircle{\pgfpointanchor{x}{east}}{2pt}
  \pgfpathcircle{\pgfpointanchor{x}{west}}{2pt}
  \pgfpathcircle{\pgfpointanchor{x}{north east}}{2pt}
  \pgfusepath{fill}
\end{pgfpicture}
\end{codeexample}

  In the above example, you may have noticed something curious: The
  rotation transformation is still in force when the anchors are
  invoked, but it does not seem to have an effect. You might expect
  that the rotation should apply to the already rotated points once
  more.

  However, |\pgfpointanchor| returns a point that takes the current
  transformation matrix into account: \emph{The inverse transformation
    to the current coordinate transformation is applied to an anchor
    point before returning it.}

  This behavior may seem a bit strange, but you will find it very
  natural in most cases. If you really want to apply a transformation
  to an anchor point (for example, to ``shift it away'' a little bit),
  you have to invoke |\pgfpointanchor| without any transformations in
  force. Here is an example:

\makeatletter
\begin{codeexample}[]
\begin{pgfpicture}
  \pgftransformrotate{30}
  \pgfnode{rectangle}{center}{Hello World!}{x}{\pgfusepath{stroke}}

  {
    \pgftransformreset
    \pgfpointanchor{x}{east}
    \xdef\mycoordinate{\noexpand\pgfpoint{\the\pgf@x}{\the\pgf@y}}
  }
    
  \pgfpathcircle{\mycoordinate}{2pt}
  \pgfusepath{fill}
\end{pgfpicture}
\end{codeexample}

  A special situation arises when the \meta{node} lies in a picture
  different from the current picture. In this case, if you have not
  told \pgfname\ that the picture should be ``remembered,'' the
  \meta{node} will be treated as if it lied in the current
  picture. For example, if the \meta{node} was at position $(3,2)$ in
  the original picture, it is treated as if it lied at position
  $(3,2)$ in the current picture. However, if you have told \pgfname\
  to remember the picture position of the node's picture and also of
  the current picture,
  then |\pgfpointanchor| will return a coordinate that corresponds to
  the position of the node's anchor on the page, transformed into the
  current coordinate system. For examples and more details see
  Section~\ref{section-cross-pictures-pgf}. 
\end{command}

\begin{command}{\pgfpointshapeborder\marg{node}\marg{point}}
  This command returns the point on the border of the shape that lies
  on a straight line from the center of the node to \meta{point}. For
  complex shapes it is not guaranteed that this point will actually
  lie on the border, it may be on the border of a ``simplified''
  version of the shape.

\begin{codeexample}[]
\begin{pgfpicture}
  \begin{pgfscope}
    \pgftransformrotate{30}
    \pgfnode{rectangle}{center}{Hello World!}{x}{\pgfusepath{stroke}}
  \end{pgfscope}
  \pgfpathcircle{\pgfpointshapeborder{x}{\pgfpoint{2cm}{1cm}}}{2pt}
  \pgfpathcircle{\pgfpoint{2cm}{1cm}}{2pt}
  \pgfpathcircle{\pgfpointshapeborder{x}{\pgfpoint{-1cm}{1cm}}}{2pt}
  \pgfpathcircle{\pgfpoint{-1cm}{1cm}}{2pt}
  \pgfusepath{fill}
\end{pgfpicture}
\end{codeexample}
\end{command}


\subsubsection{Referencing Anchors of Nodes in Different Pictures}
\label{section-cross-pictures-pgf}

As a picture is typeset, \pgfname\ keeps track of the positions of all
nodes inside the picture. What \pgfname\ does not remember is the
position of the picture \emph{itself} on the page. Thus, if you define
a node in one picture and then try to reference this node while
another picture is typeset, \pgfname\ will only know the position of
the nodes that you try to typeset inside the original picture, but it
will not know where this picture lies. What is missing is the relative
positioning of the two pictures.

To overcome this problem, you need to tell \pgfname\ that it should
remember the position of pictures on a page. If these positions are
remembered, then \pgfname\ can compute the offset between the pictures
and make nodes in different pictures accessible.

Determining the positions of pictures on the page is, alas,
not-so-easy. Because of this, \pgfname\ does not do so
automatically. Rather, you have to proceed as follows:
\begin{enumerate}
\item You have to use a backend driver that supports position
  tracking. pdf\TeX\ is one such drivers, |dvips| currently is not.
\item You have to say |\pgfrememberpicturepositiononpagetrue|
  somewhere before or inside every picture
  \begin{itemize}
  \item in which you wish to reference a node and
  \item from which you wish to reference a node in another picture.
  \end{itemize}
  The second item is important since \pgfname\ does not only need to
  know the position of the picture in which the node you wish to
  reference lies, but it also needs to know where the current picture
  lies.
\item You typically have to run \TeX\ twice (depending on the backend
  driver) since the position information typically gets written into
  an external file on the first run and is available only on the
  second run.
\item You have to switch off automatic bounding bound
  computations. The reason is that the node in the other picture
  should not influence the size of the bouding box of the current
  picture. You should say |\pgfusepath{use as bounding box}| before
  using a coordinate in another picture.
\end{enumerate}



\subsection{Predefined Nodes}

There are several special nodes that are always defined and which you
should not attempt to redefine.

\begin{predefinednode}{current bounding box}
  This node is of shape |rectangle|. Unlike normal nodes, its size
  changes constantly and always reflects the size of the bounding box
  of the current picture. This means that, for instance, that
\begin{codeexample}[code only]
\pgfpointanchor{current bounding box}{south east}
\end{codeexample}
  returns the lower left corner of the bounding box of the current
  picture. 
\end{predefinednode}

\begin{predefinednode}{current path bounding box}
  This node is also of shape |rectangle|. Its size is the size of the
  bounding box of the current path.
\end{predefinednode}

\begin{predefinednode}{current page}
  This node is inside a virtual remembered picture. The size of this
  node is the size of the current page. This means that if you create
  a remembered picture and inside this picture you reference an anchor
  of this node, you reference an absolute position on the page. To
  demonstrate the effect, the following code puts some text in the
  lower left corner of the current page. Note that this works only if
  the backend driver supports it, otherwise the text is inserted right
  here.%
{%
\pgfrememberpicturepositiononpagetrue%
\begin{pgfpicture}
  \pgfusepath{use as bounding box}
  \pgftransformshift{\pgfpointanchor{current page}{south west}}
  \pgftransformshift{\pgfpoint{1cm}{1cm}}
  \pgftext[left,base]{
    \textcolor{red}{
      Text absolutely positioned in
      the lower left corner.}
  }
\end{pgfpicture}
}%
\begin{codeexample}[code only]
\pgfrememberpicturepositiononpagetrue
\begin{pgfpicture}
  \pgfusepath{use as bounding box}
  \pgftransformshift{\pgfpointanchor{current page}{south west}}
  \pgftransformshift{\pgfpoint{1cm}{1cm}}
  \pgftext[left,base]{
    \textcolor{red}{
      Text absolutely positioned in
      the lower left corner.}
  }
\end{pgfpicture}  
\end{codeexample}
\end{predefinednode}




\subsection{Declaring New Shapes}

There are only three predefined shapes, see
Section~\ref{section-predefined-shapes}, so there must be some way of
defining new shapes. Defining a shape is, unfortunately, a
not-quite-trivial process. The reason is that shapes need to be both
very flexible (their size will vary greatly according to
circumstances) and they need to be constructed reasonably ``fast.''
\pgfname\ must be able to handle pictures with several hundreds of
nodes and documents with thousands of nodes in total. It would not do
if \pgfname\ had to compute and store, say, dozens of anchor positions
for every node. 


\subsubsection{What Must Be Defined For a Shape?}

In order to define a new shape, you must provide:
\begin{itemize}
\item
  a \emph{shape name},
\item
  code for computing the  \emph{saved anchors} and \emph{saved
    dimensions}, 
\item
  code for computing \emph{anchor} positions in terms of the saved anchors,
\item
  optionally code for the \emph{background path} and \emph{foreground path},
\item
  optionally code for \emph{things to be drawn before or behind} the
  background and foreground paths.
\item
  optionally a list of node parts.
\end{itemize}


\subsubsection{Normal Anchors Versus Saved Anchors}

Anchors  are special places in shape. For example, the |north east|
anchor, which is a normal anchor, lies at the upper right corner of
the  |rectangle| shape, as does |\northeast|, which is a saved
anchor. The difference is the following: \emph{saved anchors are 
  computed and stored for each node, anchors are only computed as
  needed.} The user only has access to the normal anchors, but a
normal anchor can just ``copy'' or ``pass through'' the location of a
saved anchor. 

The idea behind all this is that a shape can declare a very large
number of normal anchors, but when a node of this shape is created,
these anchors are not actually computed. However, this causes a
problem: When we wish to reference an anchor of a node at some later
time, we must still able to compute the position of the anchor. For 
this, we may need a lot of information: What was the transformation
matrix that was in force when the node was created? What was the size
of the text box? What were the values of the different separation
dimensions? And so on. 

To solve this problem, \pgfname\ will always compute the locations of
all \emph{saved anchors} and store these positions. Then, when an
normal anchor position is requested later on, the anchor position can
be given just from knowing where the locations of the saved anchors.

As an example, consider the |rectangle| shape. For this shape two
anchors are saved: The |\northeast| corner and the |\southwest|
corner. A normal anchor like |north west| can now easily be expressed
in terms of these coordinates: Take the $x$-position of the
|\southwest| point  and the $y$-position of the |\northeast| point. 
The |rectangle| shape currently defines 13 normal anchors, but needs
only two saved anchors. Adding new anchors like a  |south south east|
anchor would not increase the memory and computation requirements of
pictures. 

All anchors (both saved and normal) are specified in a local
\emph{shape coordinate space}. This is also true for the background
and foreground paths. The |\pgfnode| macro will automatically apply
appropriate transformations to the coordinates so that the shape is
shifted to the right anchor or otherwise transformed. 


\subsubsection{Command for Declaring New Shapes}

The following command declares a new shape:
\begin{command}{\pgfdeclareshape\marg{shape name}\marg{shape
      specification}}
  This command declares a new shape named \meta{shape name}. The shape
  name can later be used in commands like |\pgfnode|.

  The \meta{shape specification} is some \TeX\ code containing calls
  to special commands that are only defined inside the \meta{shape
    specification} (similarly to commands like |\draw| that are only
  available inside the |{tikzpicture}| environment).

  \example Here is the code of the |coordinate| shape:
\begin{codeexample}[code only]
\pgfdeclareshape{coordinate}
{
  \savedanchor\centerpoint{%
    \pgf@x=.5\wd\pgfnodeparttextbox%
    \pgf@y=.5\ht\pgfnodeparttextbox%
    \advance\pgf@y by -.5\dp\pgfnodeparttextbox%
  }
  \anchor{center}{\centerpoint}
  \anchorborder{\centerpoint}
}
\end{codeexample}

  The special commands are explained next. In the examples given for
  the special commands a new shape will be constructed, which we might
  call |simple rectangle|. It should behave like the normal rectangle
  shape, only without bothering about the fine details like inner and
  outer separations. The skeleton for the shape is the following.
\begin{codeexample}[code only]
\pgfdeclareshape{simple rectangle}{
  ...
}
\end{codeexample}

  \begin{command}{\nodeparts\marg{list of node parts}}
    This command declares which parts make up nodes of this shape. A
    \emph{node part} is a (possibly empty) text label that is drawn
    when a node of the shape is created.

    By default, a shape has just one node part called |text|. However,
    there can be several node parts. For example, the
    |circle split| shape has two parts: the |text| part, which
    shows that upper text, and a |lower| part, which shows the
    lower text. For the |circle split| shape the |\nodeparts| command
    was called with the argument |{text,lower}|.

    When a multipart node is created, the text labels are drawn in the
    sequences listed in the \meta{list of node parts}. For each node
    part there you must have declared one anchor and the \TeX-box of
    the part is placed at this anchor. For a node part called |XYZ|
    the \TeX-box |\pgfnodepartXYZbox| is placed at anchor |XYZ|.
  \end{command}

  \begin{command}{\savedanchor\marg{command}\marg{code}}
    This command declares a saved anchor. The argument \meta{command}
    should be a \TeX\ macro name like |\centerpoint|.

    The \meta{code} will be executed each time |\pgfnode| (or
    |\pgfmultipartnode|) is called to  create a node of the shape
    \meta{shape name}. When the \meta{code} 
    is executed, the \TeX-boxes of the node parts will contain the
    text labels of the node. Possibly, these box are void. For
    example, if there is just a |text| part, the node
    |\pgfnodeparttextbox| will be setup when the \meta{code} is
    executed. 

    The \meta{code} can use the width, height, and depth of the
    box(es) to compute the location of the saved anchor. In addition,
    the \meta{code} can take into account the values of dimensions like
    |\pgfshapeminwidth| or |\pgfshapeinnerxsep|. Furthermore, the
    \meta{code} can take into consideration the values of any further
    shape-specific variables that are set at the moment when
    |\pgfnode| is called.

    The net effect of the \meta{code} should be to set the two \TeX\
    dimensions |\pgf@x| and |\pgf@y|. One way to achieve this is to
    say |\pgfpoint{|\meta{x value}|}{|\meta{y value}|}| at the end of
    the \meta{code}, but you can also just set these variables.
    The values that |\pgf@x| and |\pgf@y| have after the code has been
    executed, let us call them $x$ and $y$, will be recorded and
    stored together with the node that is created by the command
    |\pgfnode|.

    The macro \meta{command} is defined to be
    |\pgfpoint{|$x$|}{|$y$|}|. However, the \meta{command} is only
    locally defined while anchor positions are being computed. Thus,
    it is possible to use very simple names for \meta{command}, like
    |\center| or |\a|, without causing a name-clash. (To be precise,
    very simple \meta{command} names will clash with existing names,
    but only locally inside the computation of anchor positions; and
    we do not need the normal |\center| command during these
    computations.)

    For our |simple rectangle| shape, we will need only one saved
    anchor: The upper right corner. The lower left corner could either
    be the origin or the ``mirrored'' upper right corner, depending on
    whether we want the text label to have its lower left corner at
    the origin or whether the text label should be centered on the
    origin. Either will be fine, for the final shape this will make no
    difference since the shape will be shifted anyway. So, let us
    assume that the text label is centered on the origin (this will be
    specified later on using the |text| anchor). We get 
    the following code for the upper right corner:
\begin{codeexample}[code only]
\savedanchor{\upperrightcorner}{
  \pgf@y=.5\ht\pgfnodeparttextbox % height of the box, ignoring the depth
  \pgf@x=.5\wd\pgfnodeparttextbox % width of the box
}
\end{codeexample}

    If we wanted to take, say, the |\pgfshapeminwidth| into account,
    we could use the following code:
    
\begin{codeexample}[code only]
\savedanchor{\upperrightcorner}{
  \pgf@y=.\ht\pgfnodeparttextbox % height of the box
  \pgf@x=.\wd\pgfnodeparttextbox % width of the box
  \setlength{\pgf@xa}{\pgfshapeminwidth}
  \ifdim\pgf@x<.5\pgf@xa
    \pgf@x=.5\pgf@xa
  \fi
}
\end{codeexample}
    Note that we could not have written |.5\pgfshapeminwidth| since
    the minium width is stored in a ``plain text macro,'' not as a
    real dimension. So if |\pgfshapeminwidth| depth were 
    2cm, writing |.5\pgfshapeminwidth| would yield the same as |.52cm|.

    In the ``real'' |rectangle| shape the code is somewhat more
    complex, but you get the basic idea.
  \end{command}  
  \begin{command}{\saveddimen\marg{command}\marg{code}}
    This command is similar to |\savedanchor|, only instead of setting
    \meta{command} to |\pgfpoint{|$x$|}{|$y$|}|, the \meta{command} is
    set just to (the value of) $x$.

    In the |simple rectangle| shape we might use a saved dimension to
    store the depth of the shape box.
  
\begin{codeexample}[code only]
\saveddimen{\depth}{
  \pgf@x=\dp\pgfnodeparttextbox 
}
\end{codeexample}
  \end{command}  
  \begin{command}{\savedmacro\marg{command}\marg{code}}
    This command is similar to |\saveddimen|, only at some point
    in \meta{code}, \meta{command} should be defined appropriately,
    (this could be a value, or some text).

    In the |regular polygon| shape, a saved macro is used to
    store the number of sides of the polygon.
  
\begin{codeexample}[code only]
\savedmacro{\sides}{\let\sides\pgfpolygonsides}
\end{codeexample}
  \end{command}  
  \begin{command}{\anchor\marg{name}\marg{code}}
    This command declares an anchor named \meta{name}. Unlike for saved
    anchors, the \meta{code} will not be executed each time a node is
    declared. Rather, the \meta{code} is only executed when the anchor
    is specifically requested; either for anchoring the node during
    its creation or as a  position in the shape referenced later on.

    The \meta{name} is a quite arbitrary string that is not ``passed
    down'' to the system level. Thus, names like |south| or |1| or
    |::| would all be fine.

    A saved anchor is not automatically also a normal anchor. If you
    wish to give the users access to a saved anchor you must declare a
    normal anchor that just returns the position of the saved anchor.

    When the \meta{code} is executed, all saved anchor macros will be
    defined. Thus, you can reference them in your \meta{code}. The
    effect of the \meta{code} should be to set the values of |\pgf@x|
    and |\pgf@y| to the coordinates of the anchor.

    Let us consider some example for the |simple rectangle|
    shape. First, we would like to make the upper right corner
    publicly available, for example as |north east|:
    
\begin{codeexample}[code only]
\anchor{north east}{\upperrightcorner}
\end{codeexample}

    The |\upperrightcorner| macro will set |\pgf@x| and |\pgf@y| to
    the coordinates of the upper right corner. Thus, |\pgf@x| and
    |\pgf@y| will have exactly the right values at the end of the
    anchor's code.

    Next, let us define a |north west| anchor. For this anchor, we can
    negate the |\pgf@x| variable:
   
\begin{codeexample}[code only]
\anchor{north west}{
  \upperrightcorner
  \pgf@x=-\pgf@x
}
\end{codeexample}

    Finally, it is a good idea to always define a |center| anchor,
    which will be the default location for a shape.

\begin{codeexample}[code only]
\anchor{center}{\pgfpointorigin}
\end{codeexample}

    You might wonder whether we should not take into consideration
    that the node is not placed at the origin, but has been shifted
    somewhere. However, the anchor positions are always specified in
    the shape's ``private'' coordinate system. The ``outer''
    transformation that has been applied to the shape upon its
    creation is applied automatically to the coordinates returned by
    the anchor's \meta{code}.

    Out |simple rectangle| only has one text label (node
    part) called |text|. This is the default situation, so we need not
    do anything. For the |text| node part we must setup a |text|
    anchor. This   anchor is used upon creation of a node to determine
    the lower left  corner of the text label (within the private
    coordinate system of the shape). By default, the |text| anchor is
    at the origin, but you may change this. For example, we would say
\begin{codeexample}[code only]
\anchor{text}{%
  \upperrightcorner%
  \pgf@x=-\pgf@x%
  \pgf@y=-\pgf@y%
}
\end{codeexample}
    to center the text label on the origin in the shape coordinate
    space. Note that we could \emph{not} have written the following:
    
\begin{codeexample}[code only]
\anchor{text}{\pgfpoint{-.5\wd\pgfnodeparttextbox}{-.5\ht\pgfnodeparttextbox}}
\end{codeexample}
    Do you see why this is wrong? The problem is that the box
    |\pgfnodeparttextbox| will most likely not have the correct size
    when the anchor is computed. After all, the anchor position might
    be recomputed at a time when several other nodes have been created. 

    If a shape has several node parts, we would have to define an
    anchor for each part.    
  \end{command}  
  \begin{command}{\anchorborder\marg{code}}
    A \emph{border anchor} is an anchor point on the border of the
    shape. What exactly is considered as the ``border'' of the shape
    depends on the shape.

    When the user request a point on the border of the shape using the
    |\pgfpointshapeborder| command, the \meta{code} will be executed
    to discern this point. When the execution of  the \meta{code}
    starts, the dimensions |\pgf@x| and |\pgf@y| will have been set to
    a location $p$ in the shape's coordinate system. It is now the job of
    the \meta{code} to setup |\pgf@x| and |\pgf@y| such that they
    specify the point on the shape's border that lies on a straight
    line from the shape's center to the point $p$. Usually, this is a
    somewhat complicated computation, involving many case distinctions
    and some basic math.

    For our |simple rectangle| we must compute a point on the border
    of a rectangle whose one corner is the origin (ignoring the depth
    for simplicity) and whose other corner is |\upperrightcorner|. The
    following code might be used:
\begin{codeexample}[code only]
\anchorborder{%
  % Call a function that computes a border point. Since this
  % function will modify dimensions like \pgf@x, we must move them to
  % other dimensions.
  \@tempdima=\pgf@x
  \@tempdimb=\pgf@y
  \pgfpointborderrectangle{\pgfpoint{\@tempdima}{\@tempdimb}}{\upperrightcorner}
}
\end{codeexample}
  \end{command}  
  \begin{command}{\backgroundpath\marg{code}}
    This command specifies the path that ``makes up'' the background
    of the shape. Note that the shape cannot prescribe what is going
    to happen with the path: It might be drawn, shaded, filled, or
    even thrown away. If you want to specify that something should
    ``always'' happen when this shape is drawn (for example, if the
    shape is a stop-sign, we \emph{always} want it to be filled with a
    red color), you can use commands like |\beforebackgroundpath|,
    explained below.

    When the \meta{code} is executed, all saved anchors will be in
    effect. The \meta{code} should contain path construction
    commands.

    For our |simple rectangle|, the following code might be used:
\begin{codeexample}[code only]
\backgroundpath{
  \pgfpathrectanglecorners
    {\upperrightcorner}
    {\pgfpointscale{-1}{\upperrightcorner}}
}  
\end{codeexample}
    As the name suggests, the background path is used ``behind'' the
    text labels. Thus, this path is used first, then the text labels are
    drawn, possibly obscuring part of the path.
  \end{command}  
  \begin{command}{\foregroundpath\marg{code}}
    This command works like |\backgroundpath|, only it is invoked
    after the text labels have been drawn. This means that this path can
    possibly obscure (part of) the text labels.
  \end{command}  
  \begin{command}{\behindbackgroundpath\marg{code}}
    Unlike the previous two commands, \meta{code} should not only
    construct a path, it should also use this path in whatever way is
    appropriate. For example, the \meta{code} might fill some area
    with a uniform color.

    Whatever the \meta{code} does, it does it first. This means that
    any drawing done by \meta{code} will be even behind the background
    path.

    Note that the \meta{code} is protected with a |{pgfscope}|.
  \end{command}  
  \begin{command}{\beforebackgroundpath\marg{code}}
    This command works like |\behindbackgroundpath|, only the
    \meta{code} is executed after the background path has been used,
    but before the texts label are drawn.
  \end{command}  
  \begin{command}{\behindforegroundpath\marg{code}}
    The \meta{code} is executed after the text labels have been drawn,
    but before the foreground path is used.
  \end{command}  
  \begin{command}{\beforeforegroundpath\marg{code}}
    This \meta{code} is executed at the very end.
  \end{command}  
  \begin{command}{\inheritsavedanchors|[from=|\marg{another shape name}|]|}
    This command allows you to inherit the code for saved anchors from
    \meta{another shape name}. The idea is that if you wish to create
    a new shape that is just a small modification of a another shape,
    you can recycle the code used for \meta{another shape name}.

    The effect of this command is the same as if you had called
    |\savedanchor| and |\saveddimen| for each saved anchor or saved
    dimension declared in \meta{another shape name}. Thus, it is not
    possible to ``selectively'' inherit only some saved anchors, you
    always have to inherit all saved anchors from another
    shape. However, you can inherit the saved anchors of more than one
    shape by calling this command several times.
  \end{command}  
  \begin{command}{\inheritbehindbackgroundpath|[from=|\marg{another shape name}|]|}
    This command can be used to inherit the code used for the
    drawings behind the background path from \meta{another shape name}. 
  \end{command}  
  \begin{command}{\inheritbackgroundpath|[from=|\marg{another shape name}|]|}
    Inherits the background path code from \meta{another shape name}.
  \end{command}  
  \begin{command}{\inheritbeforebackgroundpath|[from=|\marg{another shape name}|]|}
    Inherits the before background path code from \meta{another shape name}.
  \end{command}  
  \begin{command}{\inheritbehindforegroundpath|[from=|\marg{another shape name}|]|}
    Inherits the behind foreground path code from \meta{another shape name}.
  \end{command}  
  \begin{command}{\inheritforegroundpath|[from=|\marg{another shape name}|]|}
    Inherits the foreground path code from \meta{another shape name}.
  \end{command}  
  \begin{command}{\inheritbeforeforegroundpath|[from=|\marg{another shape name}|]|}
    Inherits the before foreground path code from \meta{another shape name}.
  \end{command}  
  \begin{command}{\inheritanchor|[from=|\marg{another shape name}|]|\marg{name}}
    Inherits the code of one specific anchor named \meta{name} from
    \meta{another shape name}. Thus, unlike saved anchors, which must
    be inherited collectively, normal anchors can and must be
    inherited individually.
  \end{command}  
  \begin{command}{\inheritanchorborder|[from=|\marg{another shape name}|]|}
    Inherits the border anchor code from \meta{another shape name}.
  \end{command}

  The following example shows how a shape can be defined that relies
  heavily on inheritance:
\makeatletter
\begin{codeexample}[]
\pgfdeclareshape{document}{
  \inheritsavedanchors[from=rectangle] % this is nearly a rectangle
  \inheritanchorborder[from=rectangle]
  \inheritanchor[from=rectangle]{center}
  \inheritanchor[from=rectangle]{north}
  \inheritanchor[from=rectangle]{south}
  \inheritanchor[from=rectangle]{west}
  \inheritanchor[from=rectangle]{east}
  % ... and possibly more
  \backgroundpath{% this is new
    % store lower right in xa/ya and upper right in xb/yb
    \southwest \pgf@xa=\pgf@x \pgf@ya=\pgf@y
    \northeast \pgf@xb=\pgf@x \pgf@yb=\pgf@y
    % compute corner of ``flipped page''
    \pgf@xc=\pgf@xb \advance\pgf@xc by-5pt % this should be a parameter
    \pgf@yc=\pgf@yb \advance\pgf@yc by-5pt
    % construct main path
    \pgfpathmoveto{\pgfpoint{\pgf@xa}{\pgf@ya}}
    \pgfpathlineto{\pgfpoint{\pgf@xa}{\pgf@yb}}
    \pgfpathlineto{\pgfpoint{\pgf@xc}{\pgf@yb}}
    \pgfpathlineto{\pgfpoint{\pgf@xb}{\pgf@yc}}
    \pgfpathlineto{\pgfpoint{\pgf@xb}{\pgf@ya}}
    \pgfpathclose
    % add little corner
    \pgfpathmoveto{\pgfpoint{\pgf@xc}{\pgf@yb}}
    \pgfpathlineto{\pgfpoint{\pgf@xc}{\pgf@yc}}
    \pgfpathlineto{\pgfpoint{\pgf@xb}{\pgf@yc}}
    \pgfpathlineto{\pgfpoint{\pgf@xc}{\pgf@yc}}
 }
}\hskip-1.2cm
\begin{tikzpicture}
  \node[shade,draw,shape=document,inner sep=2ex] (x) {Remark};
  \node[fill=examplefill,draw,ellipse,double]
    at ([shift=(-80:3cm)]x) (y) {Use Case};

  \draw[dashed] (x) -- (y);  
\end{tikzpicture}
\end{codeexample}
  
\end{command}




%%% Local Variables: 
%%% mode: latex
%%% TeX-master: "pgfmanual"
%%% End: 

% Copyright 2006 by Till Tantau
%
% This file may be distributed and/or modified
%
% 1. under the LaTeX Project Public License and/or
% 2. under the GNU Free Documentation License.
%
% See the file doc/generic/pgf/licenses/LICENSE for more details.


\section{Matrices}

\label{section-base-matrices}

\subsection{Overview}

Matrices are a mechanism for aligning several so-called cell pictures 
horizontally and vertically. The resulting alignment is placed in a
normal node and the command for creating matrices, |\pgfmatrix|, takes
options very similar to the |\pgfnode| command.

In the following, the basic idea behind the alignment mechanism is
explained first. Then the command |\pgfmatrix| is explained. At the
end of the section additional ways of modifying the width of columns
and rows is discussed.


\subsection{Cell Pictures and Their Alignment}

A matrix consists of rows of \emph{cells}. Cells are separated using
the special command |\pgfmatrixnextcell|, rows are ended using the
command |\\|. Each cell contains a \emph{cell picture}, although cell
pictures are not complete pictures as they lack layers. However, each
cell picture has its own bouding box like a normal picture does. These
bounding boxes are important for the alignment as explained
in the following.

Each cell picture will have an origin somewhere in the picture (or
even outside the picture). The position of these origins is important
for the alignment: On each row the origins will be on the same
horizontal line and for each column the origins will also be on the
same vertical line. These two requirements mean that the cell pictures
may need to be shifted around so that the origins wind up on the same
lines. The top of a row is given by the top of the cell picture whose
bounding box's maximum $y$-position is largest. Similarly, the bottom
of a row is given by the bottom of the cell picture whose bounding
box's minimum $y$-position is the most negative. Similarly, the left
end of a row is given by the left end of the cell whose bounding box's
$x$-position is the most negative; and similarly for the right end of
a row.

\begin{codeexample}[]
\begin{tikzpicture}[x=3mm,y=3mm,fill=blue!50]
  \def\atorig#1{\node[black] at (0,0) {\tiny #1};}

  \pgfmatrix{rectangle}{center}{mymatrix}
    {\pgfusepath{}}{\pgfpointorigin}{}
    {
      \fill (0,-3)  rectangle (1,1);\atorig1 \pgfmatrixnextcell
      \fill (-1,0)  rectangle (1,1);\atorig2 \pgfmatrixnextcell
      \fill (-1,-2) rectangle (0,0);\atorig3 \pgfmatrixnextcell
      \fill (-1,-1) rectangle (0,3);\atorig4 \\
      \fill (-1,0)  rectangle (4,1);\atorig5 \pgfmatrixnextcell
      \fill (0,-1)  rectangle (1,1);\atorig6 \pgfmatrixnextcell
      \fill (0,0)   rectangle (1,4);\atorig7 \pgfmatrixnextcell
      \fill (-1,-1) rectangle (0,0);\atorig8 \\
    }
\end{tikzpicture}
\end{codeexample}


\subsection{The Matrix Command}

All matrices are typeset using the following command:

\begin{command}{\pgfmatrix\marg{shape}\marg{anchor}\marg{name}%
    \marg{usage}\marg{shift}\marg{pre-code}\marg{matrix cells}}

  This command creates a node that contains a matrix. The name of the
  node is \meta{name}, its shape is \meta{shape} and the node is
  anchored at \meta{anchor}.

  The \meta{matrix cell} parameter contains the cells of the
  matrix. In each cell drawing commands may be given, which create a
  so-called cell picture. For each cell picture a bounding box is
  computed and the cells are aligned according to the rules outlined
  in the previous section.

  The resulting matrix is used as the |text| box of the node. As for a
  normal node, the \meta{usage} commands are applied, so that the
  path(s) of the resulting node are stroked or filled or whatever.

  \medskip
  \textbf{Specifiying the cells and rows.\ }
  Even though this command uses |\halign| internally, there are two
  special rules for indicating cells:
  \begin{enumerate}
  \item Cells in the same row must be separated using the macro
    |\pgfmatrixnextcell| rather than |&|. Using |&| will result in an
    error message.

    However, you can make |&| an active character and have it expand
    to |\pgfmatrixnextcell|. This way, it will ``look'' as if |&| is
    used.
  \item Every row \emph{including the last row} must be ended using
    the command |\\|. 
  \end{enumerate}

  Both |\pgfmatrixnextcell| and |\\| take an optional argument as
  explained in the Section~\ref{section-matrix-spacing}

\begin{codeexample}[]
\begin{tikzpicture}
  \pgfmatrix{rectangle}{center}{mymatrix}
    {\pgfusepath{}}{\pgfpointorigin}{}
    {
      \node {a}; \pgfmatrixnextcell \node {b}; \\
      \node {c}; \pgfmatrixnextcell \node {d}; \\
    }
\end{tikzpicture}
\end{codeexample}

  \medskip
  \textbf{Anchoring matrices at nodes inside the matrix.\ }
  The parameter \meta{shift} is an additional negative shift for the
  node. Normally, such a shift could be given beforehand (that is, the
  shift could be preapplied to the current transformation
  matrix). However, when \meta{shift} is evaluated, you can refer to
  \emph{temporary} positions of nodes inside the matrix. In detail,
  the following happens: When the matrix has been typeset, all nodes
  in the matrix temporarily get assigned their positions in the matrix
  box. The origin of this coordinate system is at the left baseline
  end of the matrix box, which corresponds to the |text| anchor. The
  position \meta{shift} is then interpreted inside this coordinate
  system and then used for shifting. 

  This allows you to use the parameter \meta{shift} in the following
  way: If you use |text| as the \meta{anchor} and specify
  |\pgfpointanchor{inner node}{some anchor}| for the parameter
  \meta{shift}, where |inner node| is a node that 
  is created in the matrix, then the whole matrix will be shifted such
  that |inner node.some anchor| lies at the origin of the whole
  picture. 

  \medskip
  \textbf{Rotations and scaling.\ }
  The matrix node is never rotated or shifted, because the current
  coordinate transformation matrix is reset (except for the
  translational part) at the beginning of |\pgfmatrix|. This is
  intentional and will not change in the future. If you need to rotate
  the matrix, you must install an appropriate canvas transformation
  yourself.

  However, nodes and stuff inside the cell pictures can be rotated and
  scaled normally.

  \medskip
  \textbf{Callbacks.\ }
  At the beginning and at the end of each cell the special macros 
  |\pgfmatrixbegincode|, |\pgfmatrixendcode| and possibly
  |\pgfmatrixemptycode| are called. The effect is explained in
  Section~\ref{section-matrix-callbacks}. 

  \medskip
  \textbf{Executing extra code.\ }
  The parameter \meta{pre-code} is executed at the beginning of the
  outermost \TeX-group enclosing the matrix node. It is inside this
  \TeX-group, but outside the matrix itself. It can be used
  for different purposes:
  \begin{enumerate}
  \item It can be used to simplify the next cell macro. For example,
    saying |\let\&=\pgfmatrixnextcell| allows you to use |\&| instead
    of |\pgfmatrixnextcell|. You can also set the catcode of |&| to 
    active.
  \item It can be used to issue an |\aftergroup| command. This allows
    you to regain control after the |\pgfmatrix| command. (If you do
    not know the |\aftergroup| command, you are probably blessed with
    a simple and happy life.)
  \end{enumerate}
  
  \medskip
  \textbf{Special considerations concerning macro expansion.\ }
  As said before, the matrix is typeset using |\halign|
  internally. This command does a lot of strange and magic things like
  expanding the first macro of every cell in a most unusual
  manner. Here are some effects you may wish to be aware of:
  \begin{itemize}
  \item It is not necessary to actually mention |\pgfmatrixnextcell|
    or |\\| inside the \meta{matrix cells}. It suffices that the
    macros inside \meta{matrix cells} expand to these macros sooner or
    later.
  \item In particular, you can define clever macros that insert
    columns and rows as needed for special effects. 
  \end{itemize}
\end{command}


\subsection{Row and Column Spacing}
\label{section-matrix-spacing}

It is possible to control the space between columns and rows rather
detailedly. Two commands are important for the row spacing and two
commands for the column spacing.

\begin{command}{\pgfsetmatrixcolumnsep\marg{sep list}}
  This macro sets the default separation list for columns. The details of the
  format of this list are explained in the description of the next command.
\end{command}

\begin{command}{\pgfmatrixnextcell\opt{\oarg{additional sep list}}}
  This command has two puposes: First, it is used to separate
  cells. Second, by providing the optional argument \meta{additional
    sep list} you can modify the spacing between the columns that are
  separated by this command.

  The optional \meta{additional sep list} may only be provided when
  the |\pgfmatrixnextcell| command starts a new column. Normally, this
  will only be the case in the first row, but sometimes a later row
  has more elements than the first row. In this case, the
  |\pgfmatrixnextcell| commands that start the new columns in the
  later row may also have the optional argument. Once a column has
  been started, subsequent uses of this optional argument for the
  column have no effect.

  To determine the space between the two columns the are separated by
  |\pgfmatrixnextcell|, the following algorithm is executed:
  \begin{enumerate}
  \item Both the default separation list (as setup by
    |\pgfsetmatrixcolumnsep|) and the \meta{additional sep list} are
    processed, in this order. If the \meta{additional sep list}
    argument is missing, only the default separation list is
    processed.
  \item Both lists may contain dimensions, separated by commas, as
    well as occurences of the keywords |between origins| and
    |between borders|.
  \item All dimensions occuring in either list are added together to
    arrive at an dimension $d$.
  \item The last occurence of either of the keywords is located. If
    neither keyword is present, we proceed as if |between borders|
    were present.
  \end{enumerate}
  At the end of the algorithm, a dimension $d$ has been computed and
  one of the two \emph{modes} |between borders| and |between origins|
  has been determined. Depending on which mode has been determined,
  the following happens:
  \begin{itemize}
  \item For the |between borders| mode, an additional horizontal space
    of $d$ is added between the two columns. Note that $d$ may be
    negative.
  \item For the |between origins| mode, the spacing between the two
    columns is computed differently: Recall that the origins of the
    cell pictures in both pictures lie on two vertical lines. The
    spacing between the two columns is setup such that the horizontal
    distance between these two lines is exactly $d$.

    This mode may only be used between columns \emph{already
      introduced in the first row}. 
  \end{itemize}
  All of the above rules boil down to the following effects:
  \begin{itemize}
  \item A default spacing between columns should be setup using
    |\pgfsetmatrixcolumnsep|. For example, you might say
    |\pgfsetmatrixcolumnsep{5pt}| to have columns be spaced apart by
    |5pt|. You could say
\begin{verbatim}
\pgfsetmatrixcolumnsep{1cm,between origins}
\end{verbatim}
    to specify that horizontal space between the origins of cell
    pictures in adjacent columns should be 1cm by default --
    regardless of the actual size of the cell pictures.
  \item You can now use the optional argument of |\pgfmatrixnextcell|
    to locally overrule the spacing between two columns. By saying
    |\pgfmatrixnextcell[5pt]| you \emph{add} 5pt to the space between
    of the two columns, regadless of the mode.

    You can also (locally) change the spacing mode for these two
    columns. For example, even if the normal spacing mode is
    |between origins|, you can say
\begin{verbatim}
\pgfmatrixnextcell[5pt,between borders]
\end{verbatim}
    to locally change the mode for these columns to
    |between borders|.
  \end{itemize}

    \begin{codeexample}[]
\begin{tikzpicture}
  \tikzstyle{every node}=[draw]
  \pgfsetmatrixcolumnsep{1mm}
  \pgfmatrix{rectangle}{center}{mymatrix}
    {\pgfusepath{}}{\pgfpointorigin}{\let\&=\pgfmatrixnextcell}
  {
    \node {8}; \&[2mm] \node{1}; \&[-1mm] \node {6}; \\
    \node {3}; \&      \node{5}; \&       \node {7}; \\
    \node {4}; \&      \node{9}; \&       \node {2}; \\
  }
\end{tikzpicture}
    \end{codeexample}
    \begin{codeexample}[]
\begin{tikzpicture}
  \tikzstyle{every node}=[draw]
  \pgfsetmatrixcolumnsep{1mm}
  \pgfmatrix{rectangle}{center}{mymatrix}
    {\pgfusepath{}}{\pgfpointorigin}{\let\&=\pgfmatrixnextcell}
  {
    \node {8}; \&[2mm] \node(a){1}; \&[1cm,between origins] \node(b){6}; \\
    \node {3}; \&      \node   {5}; \&                      \node   {7}; \\
    \node {4}; \&      \node   {9}; \&                      \node   {2}; \\
  }
  \tikzstyle{every node}=[]
  \draw [<->,red,thick] (a.center) -- (b.center) node [above,midway] {11mm};
\end{tikzpicture}
    \end{codeexample}
    \begin{codeexample}[]
\begin{tikzpicture}
  \tikzstyle{every node}=[draw]
  \pgfsetmatrixcolumnsep{1cm,between origins}
  \pgfmatrix{rectangle}{center}{mymatrix}
    {\pgfusepath{}}{\pgfpointorigin}{\let\&=\pgfmatrixnextcell}
  {
    \node (a) {8}; \& \node (b) {1}; \&[between borders] \node (c) {6}; \\
    \node     {3}; \& \node     {5}; \&                  \node     {7}; \\
    \node     {4}; \& \node     {9}; \&                  \node     {2}; \\
  }
  \tikzstyle{every node}=[]
  \draw [<->,red,thick] (a.center) -- (b.center) node [above,midway] {10mm};
  \draw [<->,red,thick] (b.east) -- (c.west) node [above,midway] {10mm};
\end{tikzpicture}
    \end{codeexample}
\end{command}

The mechanism for the between-row-spacing is the same, only the
commands are called differently.


\begin{command}{\pgfsetmatrixrowsep\marg{sep list}}
  This macro sets the default separation list for rows.
\end{command}



\begin{pgfmanualentry}
  \pgfmanualentryheadline{\declare{\doublebs}\opt{\oarg{additional sep list}}}
  \index{*bs@\protect\doublebs}
  \pgfmanualbody
  This command ends a line. The optional \meta{additional sep list} is
  used to determine the spacing between the row being ended and the
  next row. The modes and the computation of $d$ is done in the same
  way as for columns. For the last row the optional argument has no
  effect.

  This command is defined in this special way only inside matrices.
\end{pgfmanualentry}

\subsection{Callbacks}
\label{section-matrix-callbacks}

There are three macros that get called at the beginning and end of
cells. By redefining these macros, which are empty by default, you can
change the appearance of cells in a very general manner.

\begin{command}{\pgfmatrixemptycode}
  This macro is executed for empty cells. This means that \pgfname\
  uses some macro magic to determine whether a cell is empty (it
  immediatly ends with |\pgfmatrixemptycode| or |\\|) and, if so, put
  this macro inside the cell.
  \begin{codeexample}[]
\begin{tikzpicture}
  \def\pgfmatrixemptycode{\node{empty};}
  \pgfmatrix{rectangle}{center}{mymatrix}
    {\pgfusepath{}}{\pgfpointorigin}{\let\&=\pgfmatrixnextcell}
  {
    \node {a}; \&           \& \node {b}; \\
               \& \node{c}; \& \node {d}; \& \\
  }
\end{tikzpicture}
  \end{codeexample}
  As can be seen, the macro is not executed for empty cells at the end
  of row when columns are added only later on.
\end{command}


\begin{command}{\pgfmatrixbegincode}
  This macro is executed at the beginning of non-empty
  cells. Correspondingly, |\pgfmatrixendcode| is added at
  the end of every non-empty cell.
  \begin{codeexample}[]
\begin{tikzpicture}
  \def\pgfmatrixbegincode{\node[draw]\bgroup}
  \def\pgfmatrixendcode{\egroup;}
  \pgfmatrix{rectangle}{center}{mymatrix}
    {\pgfusepath{}}{\pgfpointorigin}{\let\&=\pgfmatrixnextcell}
  {
    a \& b \& c \\
    d \&   \& e \\
  }
\end{tikzpicture}
  \end{codeexample}
  Note that between |\pgfmatrixbegincode| and |\pgfmatrixendcode|
  there will \emph{not} only be the contents of the cell. Rather,
  \pgfname\ will add some (invisible) commands for book-keeping
  purposes that involve |\let| and |\gdef|. In particular, it is not a
  good idea to have |\pgfmatrixbegincode| end with |\csname| and
  |\pgfmatrixendcode| start with |\endcsname|.
\end{command}

\begin{command}{\pgfmatrixendcode}
  See the explanation above.
\end{command}


The following two counters allow you to access the current row and
current column in a callback:

\begin{command}{\pgfmatrixcurrentrow}
  This counter stores the current row of the current cell of the
  matrix. Do not even think of changing this counter.  
\end{command}

\begin{command}{\pgfmatrixcurrentcolumn}
  This counter stores the current column of the current cell of the
  matrix. 
\end{command}

%%% Local Variables: 
%%% mode: latex
%%% TeX-master: "pgfmanual"
%%% End: 

% Copyright 2003 by Till Tantau <tantau@cs.tu-berlin.de>.
%
% This program can be redistributed and/or modified under the terms
% of the LaTeX Project Public License Distributed from CTAN
% archives in directory macros/latex/base/lppl.txt.


\section{Coordinate and Canvas Transformations}

\subsection{Overview}

\pgfname\ offers two different ways of scaling, shifting, and rotating
(these operations are generally known as \emph{transformations})
graphics: You can apply \emph{coordinate transformations} to all
coordinates and you can apply \emph{canvas transformations} to the
canvas on which you draw. (The names ``coordinate'' and ``canvas''
transformations are not standard, I introduce them only for the
purposes of this manual.) 

The difference is the following:

\begin{itemize}
\item
  As the name ``coordinate transformation'' suggests, coordinate
  transformations apply only to coordinates. For example, when you
  specify a coordinate like |\pgfpoint{1cm}{2cm}| and you wish to
  ``use'' this coordinate---for example as an argument to a
  |\pgfpathmoveto| command---then the coordinate transformation matrix
  is applied to the coordinate, resulting in a new
  coordinate. Continuing the example, if the current coordinate
  transformation is ``scale by a factor of two,'' the coordinate
  |\pgfpoint{1cm}{2cm}| actually designates the point
  $(2\mathrm{cm},4\mathrm{cm})$. 

  Note that coordinate transformations apply \emph{only} to
  coordinates. They do not apply to, say, line width or shadings or
  text.
\item
  The effect of a ``canvas transformation'' like ``scale by a factor
  of two'' can be imagined as follows: You first draw your picture on
  a ``rubber canvas'' normally. Then, once you are done, the whole
  canvas is transformed, in this case stretched by a factor of
  two. In the resulting image \emph{everything} will be larger: Text,
  lines, coordinates, and shadings.
\end{itemize}

In many cases, it is preferable that you use coordinate
transformations and not canvas transformations. When canvas
transformations are used, \pgfname\ looses track of the coordinates of
nodes and shapes. Also, canvas transformations often cause undesirable
effects like changing text size. For these reasons, \pgfname\ makes it
easy to setup the coordinate transformation, but a bit harder to
change the canvas transformation.


\subsection{Coordinate Transformations}

\subsubsection{How PGF Keeps Track of the Coordinate Transformation
  Matrix}

\pgfname\ has an internal coordinate transformation matrix. This
matrix is applied to coordinates ``in certain situations.'' This means
that the matrix is not always applied to every coordinate ``no matter
what.'' Rather, \pgfname\ tries to be reasonably smart at when and how
this matrix should be applied. The most prominent examples are the
path construction commands, which apply the coordinate transformation
matrix to their inputs.

The coordinate transformation matrix consists of four numbers $a$,
$b$, $c$, and $d$, and two dimensions $s$ and $t$. When the coordinate
transformation matrix is applied to a coordinate $(x,y)$ the new
coordinate $(ax+by+s,cx+dy+t)$ results. For more details on how
transformation matrices work in general, please see, for example, the
\textsc{pdf} or PostScript reference or a textbook on computer
graphics.

The coordinate transformation matrix is equal to the identity matrix
at the beginning. More precisely, $a=1$, $b=0$, $c=0$, $d=1$,
$s=0\mathrm{pt}$, and $t=0\mathrm{pt}$.

The different coordinate transformation commands will modify the
matrix by concatenating it with another transformation matrix. This
way the effect of applying several transformation commands will
\emph{accumulate}.

The coordinate transformation matrix is local to the current \TeX\
group (unlike the canvas transformation matrix, which is local to the
current |{pgfscope}|). Thus, the effect of adding a coordinate
transformation to the coordinate transformation matrix will last only
till the end of the current \TeX\ group.




\subsubsection{Commands for Relative Coordinate Transformations}

The following commands add a basic coordinate transformation to the
current coordinate transformation matrix. For all commands, the
transformation is applied \emph{in addition} to any previous
coordinate transformations.

\begin{command}{\pgftransformshift\marg{point}}
  Shifts coordinates by \meta{point}.
\begin{codeexample}[]
\begin{tikzpicture}
  \draw[help lines] (0,0) grid (3,2);
  \draw      (0,0) -- (2,1) -- (1,0);
  \pgftransformshift{\pgfpoint{1cm}{1cm}}
  \draw[red] (0,0) -- (2,1) -- (1,0);
\end{tikzpicture}
\end{codeexample}
\end{command}

\begin{command}{\pgftransformxshift\marg{dimensions}}
  Shifts coordinates by \meta{dimension} along the $x$-axis.
\begin{codeexample}[]
\begin{tikzpicture}
  \draw[help lines] (0,0) grid (3,2);
  \draw      (0,0) -- (2,1) -- (1,0);
  \pgftransformxshift{.5cm}
  \draw[red] (0,0) -- (2,1) -- (1,0);
\end{tikzpicture}
\end{codeexample}
\end{command}

\begin{command}{\pgftransformyshift\marg{dimensions}}
  Like |\pgftransformxshift|, only for the $y$-axis.
\end{command}

\begin{command}{\pgftransformscale\marg{factor}}
  Scales coordinates by \meta{factor}.
\begin{codeexample}[]
\begin{tikzpicture}
  \draw[help lines] (0,0) grid (3,2);
  \draw      (0,0) -- (2,1) -- (1,0);
  \pgftransformscale{.75}
  \draw[red] (0,0) -- (2,1) -- (1,0);
\end{tikzpicture}
\end{codeexample}
\end{command}

\begin{command}{\pgftransformxscale\marg{factor}}
  Scales coordinates by \meta{factor} in the $x$-direction.
\begin{codeexample}[]
\begin{tikzpicture}
  \draw[help lines] (0,0) grid (3,2);
  \draw      (0,0) -- (2,1) -- (1,0);
  \pgftransformxscale{.75}
  \draw[red] (0,0) -- (2,1) -- (1,0);
\end{tikzpicture}
\end{codeexample}
\end{command}


\begin{command}{\pgftransformyscale\marg{factor}}
  Like |\pgftransformxscale|, only for the $y$-axis.
\end{command}


\begin{command}{\pgftransformxslant\marg{factor}}
  Slants coordinates by \meta{factor} in the $x$-direction. Here, a
  factor of |1| means $45^\circ$.
\begin{codeexample}[]
\begin{tikzpicture}
  \draw[help lines] (0,0) grid (3,2);
  \draw      (0,0) -- (2,1) -- (1,0);
  \pgftransformxslant{.5}
  \draw[red] (0,0) -- (2,1) -- (1,0);
\end{tikzpicture}
\end{codeexample}
\end{command}


\begin{command}{\pgftransformyslant\marg{factor}}
  Slants coordinates by \meta{factor} in the $y$-direction.
\begin{codeexample}[]
\begin{tikzpicture}
  \draw[help lines] (0,0) grid (3,2);
  \draw      (0,0) -- (2,1) -- (1,0);
  \pgftransformyslant{-1}
  \draw[red] (0,0) -- (2,1) -- (1,0);
\end{tikzpicture}
\end{codeexample}
\end{command}

  

\begin{command}{\pgftransformrotate\marg{degrees}}
  Rotates coordinates counterclockwise by \meta{degrees}.
\begin{codeexample}[]
\begin{tikzpicture}
  \draw[help lines] (0,0) grid (3,2);
  \draw      (0,0) -- (2,1) -- (1,0);
  \pgftransformrotate{30}
  \draw[red] (0,0) -- (2,1) -- (1,0);
\end{tikzpicture}
\end{codeexample}
\end{command}

  

\begin{command}{\pgftransformtriangle\marg{a}\marg{b}\marg{c}}
  This command transforms the coordinate system in such a way that the
  triangle given by the points \meta{a}, \meta{b} and \meta{c} lies at
  the coordinates $(0,0)$, $(1\mathrm{pt},0\mathrm{pt})$ and
  $(0\mathrm{pt},1\mathrm{pt})$. 
\begin{codeexample}[]
\begin{tikzpicture}
  \draw[help lines] (0,0) grid (3,2);
  \pgftransformtriangle
  {\pgfpoint{1cm}{0cm}}
  {\pgfpoint{0cm}{2cm}}
  {\pgfpoint{3cm}{1cm}}
  
  \draw (0,0) -- (1pt,0pt) -- (0pt,1pt) -- cycle;
\end{tikzpicture}
\end{codeexample}
\end{command}

  
\begin{command}{\pgftransformcm\marg{a}\marg{b}\marg{c}\marg{d}\marg{point}}
  Applies the transformation matrix given by $a$, $b$, $c$, and $d$
  and the shift \meta{point} to coordinates (in addition to any
  previous transformations already in force).
\begin{codeexample}[]
\begin{tikzpicture}
  \draw[help lines] (0,0) grid (3,2);
  \draw      (0,0) -- (2,1) -- (1,0);
  \pgftransformcm{1}{1}{0}{1}{\pgfpoint{.25cm}{.25cm}}
  \draw[red] (0,0) -- (2,1) -- (1,0);
\end{tikzpicture}
\end{codeexample}
\end{command}

  
\begin{command}{\pgftransformarrow\marg{start}\marg{end}}
  Shift coordinates to the end of the line going from \meta{start} 
  to \meta{end} with the correct rotation. 
\begin{codeexample}[]
\begin{tikzpicture}
  \draw[help lines] (0,0) grid (3,2);
  \draw      (0,0) -- (3,1);
  \pgftransformarrow{\pgfpointorigin}{\pgfpoint{3cm}{1cm}}
  \pgftext{tip}
\end{tikzpicture}
\end{codeexample}
\end{command}

  
\begin{command}{\pgftransformlineattime\marg{time}\marg{start}\marg{end}}
  Shifts coordinates by a specific point on a line at a specific
  time. The point by which the coordinate is shifted is calculated by
  calling |\pgfpointlineattime|, see
  Section~\ref{section-pointsattime}.

  In addition to shifting the coordinate, a rotation \emph{may} also
  be applied. Whether this is the case depends on whether the \TeX\ if
  |\ifpgfslopedattime| is set to true or not.
\begin{codeexample}[]
\begin{tikzpicture}
  \draw[help lines] (0,0) grid (3,2);
  \draw      (0,0) -- (2,1);
  \pgftransformlineattime{.25}{\pgfpointorigin}{\pgfpoint{2cm}{1cm}}
  \pgftext{Hi!}
\end{tikzpicture}
\end{codeexample}
\begin{codeexample}[]
\begin{tikzpicture}
  \draw[help lines] (0,0) grid (3,2);
  \draw      (0,0) -- (2,1);
  \pgfslopedattimetrue
  \pgftransformlineattime{.25}{\pgfpointorigin}{\pgfpoint{2cm}{1cm}}
  \pgftext{Hi!}
\end{tikzpicture}
\end{codeexample}

  There is another \TeX\ if that influences this command. If you set
  |\ifpgfresetnontranslationattime| to true, then, between
  shifting the coordinate and (possibly) rotating/sloping the
  coordinate, the command |\pgftransformresetnontranslations| is
  called. See the description of this command for details.
\begin{codeexample}[]
\begin{tikzpicture}
  \draw[help lines] (0,0) grid (3,2);
  \pgftransformscale{1.5}
  \draw      (0,0) -- (2,1);
  \pgfslopedattimetrue
  \pgfresetnontranslationattimefalse
  \pgftransformlineattime{.25}{\pgfpointorigin}{\pgfpoint{2cm}{1cm}}
  \pgftext{Hi!}
\end{tikzpicture}
\end{codeexample}
\begin{codeexample}[]
\begin{tikzpicture}
  \draw[help lines] (0,0) grid (3,2);
  \pgftransformscale{1.5}
  \draw      (0,0) -- (2,1);
  \pgfslopedattimetrue
  \pgfresetnontranslationattimetrue
  \pgftransformlineattime{.25}{\pgfpointorigin}{\pgfpoint{2cm}{1cm}}
  \pgftext{Hi!}
\end{tikzpicture}
\end{codeexample}
\end{command}


\begin{command}{\pgftransformcurveattime\marg{time}\marg{start}\marg{first
      support}\marg{second support}\marg{end}}
  Shifts coordinates by a specific point on a curve at a specific
  time, see  Section~\ref{section-pointsattime} once more.

  As for the line-at-time transformation command, |\ifpgfslopedattime|
  decides whether an additional rotation should be applied.
\begin{codeexample}[]
\begin{tikzpicture}
  \draw[help lines] (0,0) grid (3,2);
  \draw      (0,0) .. controls (0,2) and (1,2) .. (2,1);
  \pgftransformcurveattime{.25}{\pgfpointorigin}
    {\pgfpoint{0cm}{2cm}}{\pgfpoint{1cm}{2cm}}{\pgfpoint{2cm}{1cm}}
  \pgftext{Hi!}
\end{tikzpicture}
\end{codeexample}
\begin{codeexample}[]
\begin{tikzpicture}
  \draw[help lines] (0,0) grid (3,2);
  \draw      (0,0) .. controls (0,2) and (1,2) .. (2,1);
  \pgfslopedattimetrue
  \pgftransformcurveattime{.25}{\pgfpointorigin}
    {\pgfpoint{0cm}{2cm}}{\pgfpoint{1cm}{2cm}}{\pgfpoint{2cm}{1cm}}
  \pgftext{Hi!}
\end{tikzpicture}
\end{codeexample}
  The value of |\ifpgfresetnontranslationsattime| is also taken into account.
\end{command}


{
  \let\ifpgfslopedattime=\relax
  \begin{textoken}{\ifpgfslopedattime}
    Decides whether the ``at time'' transformation commands also
    rotate coordinates or not.
  \end{textoken}
}
{
  \let\ifpgfresetnontranslationsattime=\relax
  \begin{textoken}{\ifpgfresetnontranslationsattime}
    Decides whether the ``at time'' transformation commands should
    reset the non-translations between shifting and rotating.
  \end{textoken}
}


\subsubsection{Commands for Absolute Coordinate Transformations}

The coordinate transformation commands introduced up to now are always
applied in addition to any previous transformations. In contrast, the
commands presented in the following can be used to change the
transformation matrix ``absolutely.'' Note that this is, in general,
dangerous and will often produce unexpected effects. You should use
these commands only if you really know what you are doing.

\begin{command}{\pgftransformreset}
  Resets the coordinate transformation matrix to the identity
  matrix. Thus, once this command is given no transformations are
  applied till the end of the scope.
\begin{codeexample}[]
\begin{tikzpicture}
  \draw[help lines] (0,0) grid (3,2);
  \pgftransformrotate{30}
  \draw      (0,0) -- (2,1) -- (1,0);
  \pgftransformreset
  \draw[red] (0,0) -- (2,1) -- (1,0);
\end{tikzpicture}
\end{codeexample}
\end{command}


\begin{command}{\pgftransformresetnontranslations}
  This command sets the $a$, $b$, $c$, and $d$ part of the coordinate
  transformation matrix to $a=1$, $b=0$, $c=0$, and $d=1$. However,
  the current shifting of the matrix is not modified.

  The effect of this command is that any rotation/scaling/slanting is
  undone in the current \TeX\ group, but the origin is not ``moved
  back.''

  This command is mostly useful directly before a |\pgftext| command
  to ensure that the text is not scaled or rotated.
\begin{codeexample}[]
\begin{tikzpicture}
  \draw[help lines] (0,0) grid (3,2);
  \pgftransformscale{2}
  \pgftransformrotate{30}
  \pgftransformxshift{1cm}
  {\color{red}\pgftext{rotated}}
  \pgftransformresetnontranslations
  \pgftext{shifted only}
\end{tikzpicture}
\end{codeexample}
\end{command}


\begin{command}{\pgftransforminvert}
  Replaces the coordinate transformation matrix by a coordinate
  transformation matrix that ``exactly undoes the original
  transformation.'' For example, if the original transformation was
  ``scale by 2 and then shift right by 1cm'' the new one is ``shift
  left by 1cm and then scale by $1/2$.''

  This command will produce an error if the determinant of
  the matrix is too small, that is, if the matrix is near-singular.
\begin{codeexample}[]
\begin{tikzpicture}
  \draw[help lines] (0,0) grid (3,2);
  \pgftransformrotate{30}
  \draw      (0,0) -- (2,1) -- (1,0);
  \pgftransforminvert
  \draw[red] (0,0) -- (2,1) -- (1,0);
\end{tikzpicture}
\end{codeexample}
\end{command}

\subsubsection{Saving and Restoring the Coordinate Transformation
  Matrix}

There are two commands for saving and restoring coordinate
transformation matrices.

\begin{command}{\pgfgettransform\marg{macro}}
  This command will (locally) define \meta{macro} to a representation
  of the current coordinate transformation matrix. This matrix can
  later on be reinstalled using |\pgfsettransform|.
\end{command}


\begin{command}{\pgfsettransform\marg{macro}}
  Reinstalls a coordinate transformation matrix that was previously
  saved using |\pgfgettransform|.
\end{command}



\subsection{Canvas Transformations}

The canvas transformation matrix is not managed by \pgfname, but by
the output format like \pdf\ or PostScript. All the \pgfname\ does is
to call appropriate low-level |\pgfsys@| commands to change the canvas
transformation matrix.

Unlike coordinate transformations, canvas transformations apply to
``everything,'' including images, text, shadings, line thickness, and
so on. The idea is that a canvas transformation really stretches and
deforms the canvas after the graphic is finished.

Unlike coordinate transformations, canvas transformations are local to
the current |{pgfscope}|, not to the current \TeX\ group. This is due
to the fact that they are managed by the backend driver, not by \TeX\
or \pgfname.

Unlike the coordinate transformation matrix, it is not possible to
``reset'' the canvas transformation matrix. The only way to change it
is to concatenate it with another canvas transformation matrix or to
end the current |{pgfscope}|.

Unlike coordinate transformations, \pgfname\ does not ``keep track''
of canvas transformations. In particular, it will not be able to
correctly save the coordinates of shapes or nodes when a canvas
transformation is used.

\pgfname\ does not offer a whole set of special commands for modifying
the canvas transformation matrix. Instead, different commands allow
you to concatenate the canvas transformation matrix with a coordinate
transformation matrix (and there are numerous commands for specifying
a coordinate transformation, see the previous section).

\begin{command}{\pgflowlevelsynccm}
  This command concatenates the canvas transformation matrix with the
  current coordinate transformation matrix. Afterward, the coordinate
  transformation matrix is reset.

  The effect of this command is to ``synchronize'' the coordinate
  transformation matrix and the canvas transformation matrix. All
  transformations that were previously applied by the coordinate
  transformations matrix are now applied by the canvas transformation
  matrix.

\begin{codeexample}[]
\begin{tikzpicture}
  \draw[help lines] (0,0) grid (3,2);
  \pgfsetlinewidth{1pt}
  \pgftransformscale{5}
  \draw      (0,0) -- (0.4,.2);
  \pgftransformxshift{0.2cm}
  \pgflowlevelsynccm
  \draw[red] (0,0) -- (0.4,.2);
\end{tikzpicture}
\end{codeexample}
\end{command}


\begin{command}{\pgflowlevel\marg{transformation code}}
  This command concatenates the canvas transformation matrix with the
  coordinate transformation specified by \meta{transformation code}.

\begin{codeexample}[]
\begin{tikzpicture}
  \draw[help lines] (0,0) grid (3,2);
  \pgfsetlinewidth{1pt}
  \pgflowlevel{\pgftransformscale{5}}
  \draw      (0,0) -- (0.4,.2);
\end{tikzpicture}
\end{codeexample}
\end{command}


\begin{command}{\pgflowlevelobj\marg{transformation code}\marg{code}}
  This command creates a local |{pgfscope}|. Inside this scope,
  |\pgflowlevel| is first called with the argument
  \meta{transformation code}, then the \meta{code} is inserted. 

\begin{codeexample}[]
\begin{tikzpicture}
  \draw[help lines] (0,0) grid (3,2);
  \pgfsetlinewidth{1pt}
  \pgflowlevelobj{\pgftransformscale{5}}    {\draw (0,0) -- (0.4,.2);}
  \pgflowlevelobj{\pgftransformxshift{-1cm}}{\draw (0,0) -- (0.4,.2);}
\end{tikzpicture}
\end{codeexample}
\end{command}


\begin{environment}{{pgflowlevelscope}\marg{transformation code}}
  This environment first surrounds the \meta{environment contents} by
  a |{pgfscope}|. Then it calls |\pgflowlevel| with the argument
  \meta{transformation code}.

\begin{codeexample}[]
\begin{tikzpicture}
  \draw[help lines] (0,0) grid (3,2);
  \pgfsetlinewidth{1pt}
  \begin{pgflowlevelscope}{\pgftransformscale{5}}
    \draw (0,0) -- (0.4,.2);
  \end{pgflowlevelscope}
  \begin{pgflowlevelscope}{\pgftransformxshift{-1cm}}
    \draw (0,0) -- (0.4,.2);
  \end{pgflowlevelscope}
\end{tikzpicture}
\end{codeexample}
\end{environment}


\begin{plainenvironment}{{pgflowlevelscope}\marg{transformation code}}
  Plain \TeX\ version of the environment.
\end{plainenvironment}

\begin{contextenvironment}{{pgflowlevelscope}\marg{transformation code}}
  Con\TeX t version of the environment.
\end{contextenvironment}




%%% Local Variables: 
%%% mode: latex
%%% TeX-master: "pgfmanual"
%%% End: 

% Copyright 2003 by Till Tantau <tantau@cs.tu-berlin.de>.
%
% This program can be redistributed and/or modified under the terms
% of the LaTeX Project Public License Distributed from CTAN
% archives in directory macros/latex/base/lppl.txt.


\section{Patterns}

\label{section-patterns}

\begin{package}{pgfbasepattterns}
  This package provides commands for declaring and using patterns. The
  package is loaded automatically by 
  |pgf|, but you can load it manually if you have only included
  |pgfcore|.   
\end{package}



\subsection{Overview}

Not documented yet.





%%% Local Variables: 
%%% mode: latex
%%% TeX-master: "pgfmanual"
%%% End: 

%% Copyright 2006 by Till Tantau
%
% This file may be distributed and/or modified
%
% 1. under the LaTeX Project Public License and/or
% 2. under the GNU Free Documentation License.
%
% See the file doc/generic/pgf/licenses/LICENSE for more details.


\section{Declaring and Using Images}
\label{section-images}


This section describes the commands for creating images.


\subsection{Overview}

To be quite frank, \LaTeX's |\includegraphics| is designed better than
\pgfname's image mechanism. For this reason, \emph{I recommend that you use the
  standard image inclusion mechanism of your format}. Thus, \LaTeX\
users are encouraged to use |\includegraphics| to include images.

However, there are reasons why you might need to use the image
inclusion facilities of \pgfname:
\begin{itemize}
\item
  There is no standard image inclusion mechanism in your format. For
  example, plain \TeX\ does not have one, so \pgfname's inclusion
  mechanism is ``better than nothing.''

  However, this applies only to the |pdftex| backend. For all other
  backends, \pgfname\ currently maps its commands back to the |graphicx|
  package. Thus, in plain \TeX, this does not really help. It might be
  a good idea to fix this in the future such that \pgfname\ becomes
  independent of \LaTeX, thereby providing a uniform image abstraction
  for all formats. 
\item
  You wish to use masking. This is a feature that is only supported by
  \pgfname, though I hope that someone will implement this also for
  the graphics package in \LaTeX\ in the future.
\end{itemize}

Whatever your choice, you can still use the usual image inclusion
facilities of the |graphics| package.

The general approach taken by \pgfname\ to including an image is the
following: First, |\pgfdeclareimage| declares the
image. This must be done prior to the first use of the image. Once you
have declared an image, you can insert it into the text using
|\pgfuseimage|. The advantage of this two-phase approach is that, at
least for \textsc{pdf}, the image data will only be included once in the
file. This can drastically reduce the file size if you use an image
repeatedly, for example in an overlay. However, there is also a
command called |\pgfimage| that declares and then immediately uses the
image.

To speedup the compilation, you may wish to use the following class
option:
\begin{packageoption}{draft}
  In draft mode boxes showing the image name replace the
  images. It is checked whether the image files exist, but they are
  not read. If either height or width is not given, 1cm is used
  instead. 
\end{packageoption}

\subsection{Declaring an Image}

\begin{command}{\pgfdeclareimage\oarg{options}\marg{image
      name}\marg{filename}}
  Declares an image, but does not paint anything. To draw the image,
  use |\pgfuseimage{|\meta{image name}|}|. The \meta{filename} may not
  have an extension.  For \textsc{pdf}, the extensions |.pdf|, |.jpg|,
  and |.png| will automatically tried. For PostScript, the extensions
  |.eps|, |.epsi|, and |.ps| will be tried. 

  The following options are possible:
  \begin{itemize}
  \item
    \declare{|height=|\meta{dimension}} sets the height of the
    image. If the width is not specified simultaneously, the aspect
    ratio of the image is kept.
  \item
    \declare{|width=|\meta{dimension}} sets the width of the
    image. If the height is not specified simultaneously, the aspect
    ratio of the image is kept.
  \item
    \declare{|page=|\meta{page number}} selects a given page number
    from a multipage document. Specifying this option will have the
    following effect: first, \pgfname\ tries to find a file named
    \begin{quote}
      \meta{filename}|.page|\meta{page number}|.|\meta{extension}
    \end{quote}
    If such a file is found, it will be used instead of the originally
    specified filename. If not, \pgfname\ inserts the image stored in
    \meta{filename}|.|\meta{extension} and if a recent version of
    |pdflatex| is used, only the selected page is inserted. For older
    versions of |pdflatex| and for |dvips| the complete document is
    inserted and a warning is printed.    
  \item
    \declare{|interpolate=|\meta{true or false}} selects whether the
    image should ``smoothed'' when zoomed. False by default.
  \item
    \declare{|mask=|\meta{mask name}} selects a transparency mask. The
    mask must previously be declared using |\pgfdeclaremask| (see
    below). This option only has an effect for |pdf|. Not all viewers
    support masking. 
  \end{itemize}

\begin{codeexample}[code only]
\pgfdeclareimage[interpolate=true,height=1cm]{image1}{brave-gnu-world-logo}
\pgfdeclareimage[interpolate=true,width=1cm,height=1cm]{image2}{brave-gnu-world-logo}
\pgfdeclareimage[interpolate=true,height=1cm]{image3}{brave-gnu-world-logo}
\end{codeexample}
\end{command}


\begin{command}{\pgfaliasimage\marg{new image name}\marg{existing image name}}
  The \marg{existing image name} is ``cloned'' and the \marg{new image
    name} can now be used whenever original image is used. This
  command is useful for creating aliases for alternate extensions
  and for accessing the last image inserted using |\pgfimage|.

  \example |\pgfaliasimage{image.!30!white}{image.!25!white}|
\end{command}


\subsection{Using an Image}

\begin{command}{\pgfuseimage\marg{image name}}
  Inserts a previously declared image into the \emph{normal text}. If
  you wish to use it in a |{pgfpicture}| environment, you must put a
  |\pgftext| around it.

  If the macro |\pgfalternateextension| expands to some nonempty
  \meta{alternate extension}, \pgfname\ will first try to use the image
  names \meta{image name}|.|\meta{alternate extension}. If this
  image is not defined, \pgfname\ will next check whether \meta{alternate
    extension} contains a |!| character. If so, everything up to this
  exclamation mark and including it is deleted from \meta{alternate
    extension} and the \pgfname\ again tries to use the image \meta{image
    name}|.|\meta{alternate extension}. This is repeated until
  \meta{alternate extension} no longer contains a~|!|. Then the
  original image is used.

  The |xxcolor| package sets the alternate extension to the current
  color mixin. 

\begin{codeexample}[]
\pgfdeclareimage[interpolate=true,width=1cm,height=1cm]
  {image1}{brave-gnu-world-logo}
\pgfdeclareimage[interpolate=true,width=1cm]{image2}{brave-gnu-world-logo}
\pgfdeclareimage[interpolate=true,height=1cm]{image3}{brave-gnu-world-logo}
\begin{pgfpicture}
  \pgftext[at=\pgfpoint{1cm}{5cm},left,base]{\pgfuseimage{image1}}
  \pgftext[at=\pgfpoint{1cm}{3cm},left,base]{\pgfuseimage{image2}}
  \pgftext[at=\pgfpoint{1cm}{1cm},left,base]{\pgfuseimage{image3}}

  \pgfpathrectangle{\pgfpoint{1cm}{5cm}}{\pgfpoint{1cm}{1cm}}
  \pgfpathrectangle{\pgfpoint{1cm}{3cm}}{\pgfpoint{1cm}{1cm}}
  \pgfpathrectangle{\pgfpoint{1cm}{1cm}}{\pgfpoint{1cm}{1cm}}
  \pgfusepath{stroke}
\end{pgfpicture}
\end{codeexample}

  The following example demonstrates the effect of using
  |\pgfuseimage| inside a color mixin environment.

\begin{codeexample}[]
\pgfdeclareimage[interpolate=true,width=1cm,height=1cm]
  {image1.!25!white}{brave-gnu-world-logo.25}
\pgfdeclareimage[interpolate=true,width=1cm]
  {image2.25!white}{brave-gnu-world-logo.25}
\pgfdeclareimage[interpolate=true,height=1cm]
  {image3.white}{brave-gnu-world-logo.25}
\begin{colormixin}{25!white}
\begin{pgfpicture}
  \pgftext[at=\pgfpoint{1cm}{5cm},left,base]{\pgfuseimage{image1}}
  \pgftext[at=\pgfpoint{1cm}{3cm},left,base]{\pgfuseimage{image2}}
  \pgftext[at=\pgfpoint{1cm}{1cm},left,base]{\pgfuseimage{image3}}

  \pgfpathrectangle{\pgfpoint{1cm}{5cm}}{\pgfpoint{1cm}{1cm}}
  \pgfpathrectangle{\pgfpoint{1cm}{3cm}}{\pgfpoint{1cm}{1cm}}
  \pgfpathrectangle{\pgfpoint{1cm}{1cm}}{\pgfpoint{1cm}{1cm}}
  \pgfusepath{stroke}
\end{pgfpicture}
\end{colormixin}
\end{codeexample}
\end{command}

\begin{command}{\pgfalternateextension}
  You should redefine this command to install a different alternate
  extension.

  \example |\def\pgfalternateextension{!25!white}|
\end{command}


\begin{command}{\pgfimage\oarg{options}\marg{filename}}
  Declares the image under the name |pgflastimage| and
  immediately uses it. You can ``save'' the image for later usage by
  invoking |\pgfaliasimage| on |pgflastimage|.
  
\begin{codeexample}[]
\begin{colormixin}{25!white}
\begin{pgfpicture}
  \pgftext[at=\pgfpoint{1cm}{5cm},left,base]
    {\pgfimage[interpolate=true,width=1cm,height=1cm]{brave-gnu-world-logo}}
  \pgftext[at=\pgfpoint{1cm}{3cm},left,base]
    {\pgfimage[interpolate=true,width=1cm]{brave-gnu-world-logo}}
  \pgftext[at=\pgfpoint{1cm}{1cm},left,base]
    {\pgfimage[interpolate=true,height=1cm]{brave-gnu-world-logo}}

  \pgfpathrectangle{\pgfpoint{1cm}{5cm}}{\pgfpoint{1cm}{1cm}}
  \pgfpathrectangle{\pgfpoint{1cm}{3cm}}{\pgfpoint{1cm}{1cm}}
  \pgfpathrectangle{\pgfpoint{1cm}{1cm}}{\pgfpoint{1cm}{1cm}}
  \pgfusepath{stroke}
\end{pgfpicture}
\end{colormixin}
\end{codeexample}
\end{command}



\subsection{Masking an Image}


\begin{command}{\pgfdeclaremask\oarg{options}\marg{mask  name}\marg{filename}}
  Declares a transparency mask named \meta{mask name} (called a
  \emph{soft mask} in the \textsc{pdf} specification). This mask is
  read from the file \meta{filename}. This file should contain a
  grayscale image that is as large as the actual image. A white
  pixel in the mask will correspond to ``transparent,'' a black pixel
  to ``solid,'' and gray values correspond to intermediate values. The
  mask must have a single ``color channel.'' This means that the
  mask must be a ``real'' grayscale image, not an \textsc{rgb}-image
  in which all \textsc{rgb}-triples happen to have the same
  components.

  You can only mask images the are in a ``pixel format.'' These are
  |.jpg| and |.png|.  You cannot mask |.pdf| images in this way. Also,
  again, the mask file and the image file must have the same size.

  The following options may be given:
  \begin{itemize}
  \item |matte=|\marg{color components} sets the so-called
    \emph{matte} of the actual image (strangely, this has to be
    specified together with the mask, not with the image itself). The
    matte is the color that has been used to preblend the image. For
    example, if the image has been preblended with a red background,
    then \meta{color components} should be set to |{1 0 0}|. The
    default is |{1 1 1}|, which is white in the rgb model.

    The matte is specified in terms of the parent's image color
    space. Thus, if the parent is a grayscale image, the matte has to
    be set to |{1}|.
  \end{itemize}
  \example
\begin{codeexample}[]
%% Draw a large colorful background
\pgfdeclarehorizontalshading{colorful}{5cm}{color(0cm)=(red);
color(2cm)=(green); color(4cm)=(blue); color(6cm)=(red);
color(8cm)=(green); color(10cm)=(blue); color(12cm)=(red);
color(14cm)=(green)}
\hbox{\pgfuseshading{colorful}\hskip-14cm\hskip1cm
\pgfimage[height=4cm]{brave-gnu-world-logo}\hskip1cm
\pgfimage[height=4cm]{brave-gnu-world-logo-mask}\hskip1cm
\pgfdeclaremask{mymask}{brave-gnu-world-logo-mask}
\pgfimage[mask=mymask,height=4cm,interpolate=true]{brave-gnu-world-logo}}
\end{codeexample}
\end{command}

%%% Local Variables: 
%%% mode: latex
%%% TeX-master: "pgfmanual"
%%% End: 

% Copyright 2007 by Till Tantau
%
% This file may be distributed and/or modified
%
% 1. under the LaTeX Project Public License and/or
% 2. under the GNU Free Documentation License.
%
% See the file doc/generic/pgf/licenses/LICENSE for more details.


\section{Externalizing Graphics}
\label{section-external}


\subsection{Overview}

There are two fundamentally different ways of inserting graphics into
a \TeX-document. First, you can create a graphic using some external
program like |xfig| or |InDesign| and then include this graphic in
your text. This is done using commands like |\includegraphics| or
|\pgfimage|. In this case, the graphic file contains all the low-level
graphic commands that describe the picture. When such a file is
included, all \TeX\ has to worry about is the size of the picture; the
internals of the picture are unknown to \TeX\ and it does not care
about them.

The second method of creating graphics is to use a special package
that transforms \TeX-commands like |\draw| or |\psline| into
appropriate low-level graphic commands. In this case, \TeX\ has to do
all the hard work of ``typesetting'' the picture and if a picture has
a complicated internal structure this may take a lot of time.

While \pgfname\ was created to facilitate the second method of
creating pictures, there are two main reasons why you may need to
employ the first method of image-inclusion, nevertheless:

\begin{enumerate}
\item Typesetting a picture using \TeX\ can be a very time-consuming
  process. If \TeX\ needs a minute to typeset a picture, you do not
  want to wait this minute when you re\TeX\ your document after having
  changed a single comma.
\item Some users, especially journal editors, may not be able to
  process files that contain \pgfname\ commands -- for the simple
  reason that the systems of many publishing houses do not have
  \pgfname\ installed. 
\end{enumerate}

In both cases, the solution is to ``extract'' or ``externalize''
pictures that would normally be typeset every time a document is \TeX
ed. Once the pictures have been extracted into separate graphics
files, these graphic files can be reinserted into the text using the
first method.

Extracting a graphic from a file is not as easy as it may sound at
first since \TeX\ cannot write parts of its output into different
files. A bit of trickery is needed and the |pgfbaseimage| package
provides macros that simplify the following workflow:

\begin{enumerate}
\item You have to tell \pgfname\ which files will be used for which
  pictures. To do so, you enclose each picture that you wish to be
  ``externalized'' in a pair of |\beginpgfgraphicnamed| and
  |\endpgfgraphicnamed| macros.
\item The next step is to generate the extracted graphics. For this
  you run \TeX\ with the |\jobname| set to the graphic file's
  name. This will cause |\pgfname| to behave in a very special way:
  All of your document will simply be thrown away, \emph{except} for
  the single graphic having the same name as the current jobname.
\item After you have run \TeX\ once for each graphic that your wish to
  externalize, you can rerun \TeX\ on your document normally. This
  will have the following effect: Each time a |\beginpgfgraphicnamed|
  is encountered, \pgfname\ checks whether a graphic file of the given
  name exists (if you did step 2, it will). If this graphic file
  exists, it will be input and the text till the corresponding
  |\endpgfgraphicnamed| will be ignored.
\end{enumerate}

In the rest of this section, the above workflow is explained in more
detail.


\subsection{Workflow Step 1: Naming Graphics}

In order to put each graphic in an external file, you first need to
tell \pgfname\ the names of these files. For this, you use a pair of
commands that are declared in the package |pgfbaseimage|.

\begin{command}{\beginpgfgraphicnamed\marg{file name prefix}}
  This command indicates that everything up to
  the next call of |\endpgfgraphicnamed| is part of a graphic that
  should be placed in a file named \meta{file name
    prefix}|.|\meta{suffix}, where the \meta{suffix} depends on your
  backend driver. Typically, \meta{suffix} will be |dvi| or |pdf|.

  Here is a typical example of how this command is used:
\begin{codeexample}[code only]
% In file main.tex:
...
As we see in Figure~\ref{fig1}, the world is flat.
\begin{figure}
  \beginpgfgraphicnamed{graphic-of-flat-world}
  \begin{tikzpicture}
    \fill (0,0) circle (1cm);
  \end{tikzpicture}
  \endpgfgraphicnamed
  \caption{The flat world.}
  \label{fig1}
\end{figure}
\end{codeexample}

  Each graphic that is be externalized should have a unique name. Note
  that this name will be used as the name of a file in the file
  system, so it should not contain any funny characters.

  This command can have three different effects:
  \begin{enumerate}
  \item The easiest situation arises if there does not yet exist a
    graphic file called \meta{file name  prefix}|.|\meta{suffix},
    where the \meta{suffix} is one of the suffixes understood by your
    current backend driver (so |pdf| or |jpg| if you use |pdftex|,
    |eps| if you use |dvips|, and so on). In this case, both this
    command and the |\endpgfgraphicnamed| command simply have no
    effect. 
  \item A more complex situation arises when a graphic file named
    \meta{file name  prefix}|.|\meta{suffix} \emph{does} exist. In
    this case, this graphic file is included using the
    |\includegraphics| command. Furthermore, the text between
    |\beginpgfgraphicnamed| and |\endpgfgraphicnamed| is ignored.

    When the text is ``ignored,'' what actually happens is that all
    text up to the next occurrence  of |\endpgfgraphicnamed| is thrown
    away without any macro expansion. This means, in particular, that
    (a) you cannot put |\endpgfgraphicnamed| inside a macro and (b)
    the macros used in the graphics need not be defined at all when
    the graphic file is included.
  \item The most complex behaviour arises when current the |\jobname|
    equals the \meta{file name prefix} and, furthermore, the
    a \emph{real job name} has been declared. The behaviour for this
    case is explained later.
  \end{enumerate}

  Note that the |\beginpgfgraphicnamed| does not really have any
  effect until you have generated the graphic files named. Till then,
  this command is simply ignored. Also, if you delete the graphics
  file later on, the graphics are typeset normally once more.
\end{command}

\begin{command}{\endpgfgraphicnamed}
  This command just marks the end of the graphic that should be
  externalized.
\end{command}


\subsection{Workflow Step 2: Generating the External Graphics}

We have now indicated all the graphics for which we would like graphic
files to be generated. In order to generate the files, you now need to
modify the |\jobname| appropriately. This is done in two steps:

\begin{enumerate}
\item You use the following command to tell \pgfname\ the real name of
  your |.tex| file:
  \begin{command}{\pgfrealjobname\marg{name}}
    Tells \pgfname\ the real name of your job. For instance, if you
    have a file called |survey.tex| that contains two graphics that
    you wish to be called |survey-graphic1| and |survey-graphic2|,
    then you should write the following.
\begin{codeexample}[code only]
% This is file survey.tex
\documentclass{article}
...
\usepackage{tikz}
\pgfrealjobname{survey}
\end{codeexample}
  \end{command}
\item  You run \TeX\ with the |\jobname| set to the name of
the graphic for which you need an external graphic to be generated.
To set the |\jobname|, you use the |--jobname=| option of \TeX:

\begin{codeexample}[code only]
bash> latex --jobname=survey-graphic1 survey.tex
\end{codeexample}
\end{enumerate}

The following things will now happen:
\begin{enumerate}
\item |\pgfrealjobname| notices that the |\jobname|
  is not the ``real'' jobname and, thus, must be the name of a graphic
  that is to be put in an external file.
\item At the beginning of the document, \pgfname\ changes the
  definition of \TeX's internal |\shipout| macro. The new shipout
  macro simply throws away the output. This means that the document is
  typeset normally, but no output is produced.
\item When the |\beginpgfgraphicnamed{|\meta{name}|}| command is
  encountered where the \meta{name} is the same as the current
  |\jobname|, then a \TeX-box is started and \meta{everything} up to the
  following |\endpgfgraphicnamed| command is stored inside this box.

  Note that, typically, \meta{everything} will contain just a single
  |{tikzpicture}| or |{pgfpicture}| environment. However, this need
  not be the case, you use, say, a |{pspicture}| environment as
  \meta{everything} or even just some normal \TeX-text.  
\item At the |\endpgfgraphicnamed|, the box \emph{is} shipped out
  using the original |\shipout| command. Thus, unlike everything else,
  the contents of the graphic is made part of the output.
\item When the box containing the graphic is shipped out, the paper
  size is modified such that it exactly equal to the height and width
  of the box. 
\end{enumerate}

The net effect of everything described above is that the two
commands
\begin{codeexample}[code only]
bash> latex --jobname=survey-graphic1 survey.tex
bash> dvips survey-graphic1
\end{codeexample}
\noindent produce a file called |survey-graphic1.ps| that consists of a single
page that contains exactly the graphic produced by the code between
|\beginpgfgraphicnamed{survey-graphic1}| and
|\endpgfgraphicnamed|. Furthermore, the size of this single page is
exactly the size of the graphic.

If you use pdf\TeX, producing the graphic is even simpler:
\begin{codeexample}[code only]
bash> pdflatex --jobname=survey-graphic1 survey.tex
\end{codeexample}
\noindent produces the single-page |pdf|-file |survey-graphic1.pdf|.

\subsection{Workflow Step 3: Including the External Graphics}

Once you have produced all the pictures in the text, including them
into the main document is easy: Simply run \TeX\ again without any
modification of the |\jobname|. In this case the
|\pgfrealjobname| command will notice that the main file is, indeed,
the main file. The main file will then be typeset normally and the
|\beginpgfgraphicnamed| commands also behave normally, which means
that they will try to include the generated graphic files -- which is
exactly what you want.

Suppose that you wish to send your survey to a journal that does not
have \pgfname\ installed. In this case, you now have all the necessary
external graphics, but you still need \pgfname\ to automatically
include them instead of the executing the picture code! One way to
solve this problem is to simply delete all of the \pgfname\ or
\tikzname\ code from your |survey.tex| and instead insert appropriate
|\includegraphics| commands ``by hand.'' However, there is a better
way: You input the file |pgfexternal.tex|.

\begin{filedescription}{pgfexternal.tex}
  This file defines the command |\beginpgfgraphicnamed| and causes it
  to have the following effect: It includes the graphic file given as
  a parameter to it and then gobbles everything up to
  |\endpgfgraphicnamed|.

  Since |\beginpgfgraphicnamed| does not do macro expansion as it
  searches for |\endpgfgraphicnamed|, it is not necessary to actually
  include the packages necessary for \emph{creating} the graphics. 
  So the idea is that you comment out things like |\usepackage{tikz}|
  and instead say |\input pgfexternal.tex|.

  Indeed, the contents of this file is simply the following line:
\begin{codeexample}[code only]
\long\def\beginpgfgraphicnamed#1#2\endpgfgraphicnamed{\includegraphics{#1}}
\end{codeexample}

  Instead of |\input pgfexternal.tex| you could also include this line
  in your main file. 
\end{filedescription}

As a final remark, note that the |baseline| option does not work
together with pictures written to an external graphic file. The simple
reason is that there is no way to store this baseline information is
an external graphic file. 


\subsection{A Complete Example}

Let us now have a look at a simple, but complete example. We start out
with a normal file called |survey.tex| that has the following
contents:
\begin{codeexample}[code only]
% This is the file survey.tex
\documentclass{article}

\usepackage{graphics}
\usepackage{tikz}

\begin{document}
In the following figure, we see a circle:
\begin{tikzpicture}
  \fill (0,0) circle (10pt);
\end{tikzpicture}

By comparison, in this figure we see a rectangle:
\begin{tikzpicture}
  \fill (0,0) rectangle (10pt,10pt);
\end{tikzpicture}
\end{document}
\end{codeexample}

Now our editor tells us that the publisher will need all figures to be
provided in separate PostScript or |.pdf|-files. For this, we 
enclose all figures in |...graphicnamed|-pairs and we add a call to
the |\pgfrealjobname| macro:
\begin{codeexample}[code only]
% This is the file survey.tex
\documentclass{article}

\usepackage{graphics}
\usepackage{tikz}
\pgfrealjobname{survey}

\begin{document}
In the following figure, we see a circle:
\beginpgfgraphicnamed{survey-f1}
\begin{tikzpicture}
  \fill (0,0) circle (10pt);
\end{tikzpicture}
\endpgfgraphicnamed

By comparison, in this figure we see a rectangle:
\beginpgfgraphicnamed{survey-f2}
\begin{tikzpicture}
  \fill (0,0) rectangle (10pt,10pt);
\end{tikzpicture}
\endpgfgraphicnamed
\end{document}
\end{codeexample}

After these changes, typesetting the file will still yield the same
output as it did before -- after all, we have not yet created any
external graphics.

To create the external graphics, we run |pdflatex| twice, once for
each graphic:
\begin{codeexample}[code only]
bash> pdflatex --jobname=survey-f1 survey.tex
This is pdfTeX, Version 3.141592-1.40.3 (Web2C 7.5.6)
entering extended mode
(./survey.tex
LaTeX2e <2005/12/01>
...
) [1] (./survey-f1.aux) )
Output written on survey-f1.pdf (1 page, 1016 bytes).
Transcript written on survey-f1.log.
\end{codeexample}

\begin{codeexample}[code only]
bash> pdflatex --jobname=survey-f2 survey.tex
This is pdfTeX, Version 3.141592-1.40.3 (Web2C 7.5.6)
entering extended mode
(./survey.tex
LaTeX2e <2005/12/01>
...
(./survey-f2.aux) )
Output written on survey-f2.pdf (1 page, 1002 bytes).
Transcript written on survey-f2.log.
\end{codeexample}

We can now send the two generated graphics (|survey-f1.pdf| and
|survey-f2.pdf|) to the editor. However, the publisher cannot use our
|survey.tex| file, yet. The reason is that it contains the command
|\usepackage{tikz}| and they do not have \pgfname\ installed.

Thus, we modify the main file |survey.tex| as follows:
\begin{codeexample}[code only]
% This is the file survey.tex
\documentclass{article}

\usepackage{graphics}
\input pgfexternal.tex
% \usepackage{tikz}
% \pgfrealjobname{survey}

\begin{document}
In the following figure, we see a circle:
\beginpgfgraphicnamed{survey-f1}
\begin{tikzpicture}
  \fill (0,0) circle (10pt);
\end{tikzpicture}
\endpgfgraphicnamed

By comparison, in this figure we see a rectangle:
\beginpgfgraphicnamed{survey-f2}
\begin{tikzpicture}
  \fill (0,0) rectangle (10pt,10pt);
\end{tikzpicture}
\endpgfgraphicnamed
\end{document}
\end{codeexample}
If we now run pdf\LaTeX, then, indeed, \pgfname\ is no longer needed:
\begin{codeexample}[code only]
bash> pdflatex survey.tex
This is pdfTeX, Version 3.141592-1.40.3 (Web2C 7.5.6)
entering extended mode
(./survey.tex
LaTeX2e <2005/12/01>
Babel <v3.8h> and hyphenation patterns for english, ..., loaded.
(/usr/local/gwTeX/texmf.texlive/tex/latex/base/article.cls
Document Class: article 2005/09/16 v1.4f Standard LaTeX document class
(/usr/local/gwTeX/texmf.texlive/tex/latex/base/size10.clo))
(/usr/local/gwTeX/texmf.texlive/tex/latex/graphics/graphics.sty
(/usr/local/gwTeX/texmf.texlive/tex/latex/graphics/trig.sty)
(/usr/local/gwTeX/texmf.texlive/tex/latex/config/graphics.cfg)
(/usr/local/gwTeX/texmf.texlive/tex/latex/pdftex-def/pdftex.def))
(/Users/tantau/Library/texmf/tex/generic/pgf/generic/pgf/utilities/pgfexternal.
tex) (./survey.aux)
(/usr/local/gwTeX/texmf.texlive/tex/context/base/supp-pdf.tex
[Loading MPS to PDF converter (version 2006.09.02).]
) <survey-f1.pdf, id=1, 23.33318pt x 19.99973pt> <use survey-f1.pdf>
<survey-f2.pdf, id=2, 13.33382pt x 10.00037pt> <use survey-f2.pdf> [1{/Users/ta
ntau/Library/texmf/fonts/map/pdftex/updmap/pdftex.map} <./survey-f1.pdf> <./sur
vey-f2.pdf>] (./survey.aux) )</usr/local/gwTeX/texmf.texlive/fonts/type1/bluesk
y/cm/cmr10.pfb>
Output written on survey.pdf (1 page, 10006 bytes).
Transcript written on survey.log.
\end{codeexample}

To our editor, we send the following files:
\begin{itemize}
\item The last |survey.tex| shown above.
\item The graphic file |survey-f1.pdf|.
\item The graphic file |survey-f2.pdf|.
\item The file |pgfexternal.tex|, whose contents is simply
\begin{codeexample}[code only]
\long\def\beginpgfgraphicnamed#1#2\endpgfgraphicnamed{\includegraphics{#1}}
\end{codeexample}
  (Alternatively, we can also directly add this line to our
  |survey.tex| file).
\end{itemize}

%%% Local Variables: 
%%% mode: latex
%%% TeX-master: "pgfmanual"
%%% End: 

% Copyright 2006 by Till Tantau
%
% This file may be distributed and/or modified
%
% 1. under the LaTeX Project Public License and/or
% 2. under the GNU Free Documentation License.
%
% See the file doc/generic/pgf/licenses/LICENSE for more details.


\section{Creating Plots}

\label{section-plots}

This section describes the |plot| module.

\begin{pgfmodule}{plot}
  This module provides a set of commands that are intended to make it
  reasonably easy to plot functions using \pgfname. It is loaded
  automatically by |pgf|, but you can load it manually if you have
  only included |pgfcore|.  
\end{pgfmodule}


\subsection{Overview}

There are different reasons for using \pgfname\ for creating plots
rather than some more powerful program such as \textsc{gnuplot} or
\textsc{mathematica}, as discussed in
Section~\ref{section-why-pgname-for-plots}. So, let us assume that --
for whatever reason -- you wish to use \pgfname\ for generating a plot.

\pgfname\ (conceptually) uses a two-stage process for generating
plots. First, a \emph{plot stream} must be produced. This stream
consists (more or less) of a large number of coordinates. Second a 
\emph{plot handler} is applied to the stream. A plot handler ``does
something'' with the stream. The standard handler will issue
line-to operations to the coordinates in the stream. However, a
handler might also try to issue appropriate curve-to operations in
order to smooth the curve. A handler may even do something else
entirely, like writing each coordinate to another stream, thereby
duplicating the original stream.

Both for the creation of streams and the handling of streams different
sets of commands exist. The commands for creating streams start with
|\pgfplotstream|, the commands for setting the handler start with
|\pgfplothandler|.



\subsection{Generating Plot Streams}

\subsubsection{Basic Building Blocks of Plot Streams}
A \emph{plot stream} is a (long) sequence of the following three
commands:
\begin{enumerate}
\item
  |\pgfplotstreamstart|,
\item
  |\pgfplotstreampoint|, and
\item
  |\pgfplotstreamend|.
\end{enumerate}
Between calls of these commands arbitrary other code may be
called. Obviously, the stream should start with the first command and
end with the last command. Here is an example of a plot stream:
\begin{codeexample}[code only]
\pgfplotstreamstart
\pgfplotstreampoint{\pgfpoint{1cm}{1cm}}
\newdimen\mydim
\mydim=2cm
\pgfplotstreampoint{\pgfpoint{\mydim}{2cm}}
\advance \mydim by 3cm
\pgfplotstreampoint{\pgfpoint{\mydim}{2cm}}
\pgfplotstreamend
\end{codeexample}

\begin{command}{\pgfplotstreamstart}
  This command signals that a plot stream starts. The effect of this
  command is to call the internal command |\pgf@plotstreamstart|,
  which is set by the current plot handler to do whatever needs to be
  done at the beginning of the plot.
\end{command}

\begin{command}{\pgfplotstreampoint\marg{point}}
  This command adds a \meta{point} to the current plot stream. The
  effect of this command is to call the internal command |\pgf@plotstreampoint|,
  which is also set by the current plot handler. This command should
  now ``handle'' the point in some sensible way. For example, a
  line-to command might be issued for the point.
\end{command}

\begin{command}{\pgfplotstreamend}
  This command signals that a plot stream ends. It calls
  |\pgf@plotstreamend|, which should now do any necessary ``cleanup.''
\end{command}

Note that plot streams are not buffered, that is, the different points
are handled immediately. However, using the recording handler, it is
possible to record a stream.

\subsubsection{Commands That Generate Plot Streams}

Plot streams can be created ``by hand'' as in the earlier
example. However, most of the time the coordinates will be produced
internally by some command. For example, the |\pgfplotxyfile| reads a
file and converts it into a plot stream.

\begin{command}{\pgfplotxyfile\marg{filename}}
  This command will try to open the file \meta{filename}. If this
  succeeds, it will convert the file contents into a plot stream as
  follows: A |\pgfplotstreamstart| is issued. Then, each nonempty line
  of the file should start with two numbers separated by a space, such
  as |0.1 1| or |100 -.3|. Anything following the numbers is ignored.

  Each pair \meta{x} and \meta{y} of numbers is converted into one
  plot stream point in the xy-coordinate system. Thus, a line like
\begin{codeexample}[code only]
2 -5 some text
\end{codeexample}
  is turned into 
\begin{codeexample}[code only]
\pgfplotstreampoint{\pgfpointxy{2}{-5}}
\end{codeexample}

  The two characters |%| and |#| are also allowed in a file and they
  are both treated as comment characters. Thus, a line starting with
  either of them is empty and, hence, ignored.

  When the file has been read completely, |\pgfplotstreamend| is
  called. 
\end{command}


\begin{command}{\pgfplotxyzfile\marg{filename}}
  This command works like |\pgfplotxyfile|, only \emph{three} numbers
  are expected on each non-empty line. They are converted into points
  in the xyz-coordinate system. Consider, the following file:
\begin{codeexample}[code only]
% Some comments
# more comments
2 -5  1 first entry
2 -.2 2 second entry
2 -5  2 third entry
\end{codeexample}
  It is turned into the following stream:
\begin{codeexample}[code only]
\pgfplotstreamstart
\pgfplotstreampoint{\pgfpointxyz{2}{-5}{1}}
\pgfplotstreampoint{\pgfpointxyz{2}{-.2}{2}}
\pgfplotstreampoint{\pgfpointxyz{2}{-5}{2}}
\pgfplotstreamend
\end{codeexample}
\end{command}


Currently, there is no command that can decide automatically whether
the xy-coordinate system should be used or whether the xyz-system
should be used. However, it would not be terribly difficult to write a
``smart file reader'' that parses coordinate files a bit more
intelligently. 


\begin{command}{\pgfplotfunction\marg{variable}\marg{sample list}\marg{point}} 
  This command will produce coordinates by iterating the
  \meta{variable} over all values in \meta{sample list}, which should
  be a list in the |\foreach| syntax. For each value of
  \meta{variable}, the \meta{point} is evaluated and the resulting
  coordinate is inserted into the plot stream.

\begin{codeexample}[]
\begin{tikzpicture}[x=3.8cm/360]
  \pgfplothandlerlineto
  \pgfplotfunction{\x}{0,5,...,360}{\pgfpointxy{\x}{sin(\x)+sin(3*\x)}}
  \pgfusepath{stroke}  
\end{tikzpicture}
\end{codeexample}

\begin{codeexample}[]
\begin{tikzpicture}[y=3cm/360]
  \pgfplothandlerlineto
  \pgfplotfunction{\y}{0,5,...,360}{\pgfpointxyz{sin(2*\y)}{\y}{cos(2*\y)}}
  \pgfusepath{stroke}  
\end{tikzpicture}
\end{codeexample}

  Be warnded that if the expressions that need to evaluated for each
  point are complex, then this command can be very slow.
\end{command}



\begin{command}{\pgfplotgnuplot\oarg{prefix}\marg{function}}
  This command will ``try'' to call the \textsc{gnuplot} program to
  generate the coordinates of the \meta{function}. In detail, the
  following happens:

  This command works with two files: \meta{prefix}|.gnuplot| and
  \meta{prefix}|.table|.  If the optional argument \meta{prefix} is
  not given, it is set to |\jobname|.

  Let us start with the situation where none of these files
  exists. Then \pgfname\ will first generate the file
  \meta{prefix}|.gnuplot|. In this file it writes
\begin{codeexample}[code only]
set terminal table; set output "#1.table"; set format "%.5f"
\end{codeexample}
  where |#1| is replaced by \meta{prefix}. Then, in a second line, it
  writes the text \meta{function}.

  Next, \pgfname\ will try to invoke the program |gnuplot| with the
  argument \meta{prefix}|.gnuplot|. This call may or may not succeed,
  depending on whether the |\write18| mechanism (also known as
  shell escape) is switched on and whether the |gnuplot| program is
  available.

  Assuming that the call succeeded, the next step is to invoke
  |\pgfplotxyfile| on the file \meta{prefix}|.table|; which is exactly
  the file that has just been created by |gnuplot|.
  
\begin{codeexample}[]
\begin{tikzpicture}
  \draw[help lines] (0,-1) grid (4,1);
  \pgfplothandlerlineto
  \pgfplotgnuplot[plots/pgfplotgnuplot-example]{plot [x=0:3.5] x*sin(x)}
  \pgfusepath{stroke}
\end{tikzpicture}
\end{codeexample}

  The more difficult situation arises when the |.gnuplot| file exists,
  which will be the case on the second run of \TeX\ on the \TeX\
  file. In this case \pgfname\ will read this file and check whether
  it contains exactly what \pgfname\ ``would have written'' into
  this file. If this is not the case, the file contents is overwritten
  with what ``should be there'' and, as above, |gnuplot| is invoked to
  generate a new |.table| file. However, if the file contents is ``as
  expected,'' the external |gnuplot| program is \emph{not}
  called. Instead, the \meta{prefix}|.table| file is immediately
  read.

  As explained in Section~\ref{section-tikz-gnuplot}, the net effect
  of the above mechanism is that |gnuplot| is called as little as
  possible and that when you pass along the |.gnuplot| and |.table|
  files with your |.tex| file to someone else, that person can
  \TeX\ the |.tex| file without having |gnuplot| installed and without
  having the |\write18| mechanism switched on.
\end{command}



\subsection{Plot Handlers}

\label{section-plot-handlers}

A \emph{plot handler}  prescribes what ``should be done'' with a
plot stream. You must set the plot handler before the stream starts.
The following commands install the most basic plot handlers; more plot
handlers are defined in the file |pgflibraryplothandlers|, which is
documented in Section~\ref{section-library-plothandlers}.

All plot handlers work by setting redefining the following three
macros: |\pgf@plotstreamstart|, |\pgf@plotstreampoint|, and
|\pgf@plotstreamend|.

\begin{command}{\pgfplothandlerlineto}
  This handler will issue a |\pgfpathlineto| command for each point of
  the plot, \emph{except} possibly for the first. What happens with
  the first point can be specified using the two commands described
  below.

\begin{codeexample}[]
\begin{pgfpicture}
  \pgfpathmoveto{\pgfpointorigin}
  \pgfplothandlerlineto
  \pgfplotstreamstart
  \pgfplotstreampoint{\pgfpoint{1cm}{0cm}}
  \pgfplotstreampoint{\pgfpoint{2cm}{1cm}}
  \pgfplotstreampoint{\pgfpoint{3cm}{2cm}}
  \pgfplotstreampoint{\pgfpoint{1cm}{2cm}}
  \pgfplotstreamend
  \pgfusepath{stroke}
\end{pgfpicture}
\end{codeexample}
\end{command}

\begin{command}{\pgfsetmovetofirstplotpoint}
  Specifies that the line-to plot handler (and also some other plot 
  handlers) should issue a move-to command for the
  first point of the plot instead of a line-to. This will start a new
  part of the current path, which is not always, but often,
  desirable. This is the default.
\end{command}

\begin{command}{\pgfsetlinetofirstplotpoint}
  Specifies that  plot handlers should issue a line-to command for the
  first point of the plot.

\begin{codeexample}[]
\begin{pgfpicture}
  \pgfpathmoveto{\pgfpointorigin}
  \pgfsetlinetofirstplotpoint
  \pgfplothandlerlineto
  \pgfplotstreamstart
  \pgfplotstreampoint{\pgfpoint{1cm}{0cm}}
  \pgfplotstreampoint{\pgfpoint{2cm}{1cm}}
  \pgfplotstreampoint{\pgfpoint{3cm}{2cm}}
  \pgfplotstreampoint{\pgfpoint{1cm}{2cm}}
  \pgfplotstreamend
  \pgfusepath{stroke}
\end{pgfpicture}
\end{codeexample}
\end{command}

\begin{command}{\pgfplothandlerdiscard}
  This handler will simply throw away the stream.
\end{command}

\begin{command}{\pgfplothandlerrecord\marg{macro}}
  When this handler is installed, each time a plot stream command is
  called, this command will be appended to \meta{macros}. Thus, at
  the end of the stream, \meta{macro} will contain all the
  commands that were issued on the stream. You can then install
  another handler and invoke \meta{macro} to ``replay'' the stream
  (possibly many times).
 
\begin{codeexample}[]
\begin{pgfpicture}
  \pgfplothandlerrecord{\mystream}
  \pgfplotstreamstart
  \pgfplotstreampoint{\pgfpoint{1cm}{0cm}}
  \pgfplotstreampoint{\pgfpoint{2cm}{1cm}}
  \pgfplotstreampoint{\pgfpoint{3cm}{1cm}}
  \pgfplotstreampoint{\pgfpoint{1cm}{2cm}}
  \pgfplotstreamend
  \pgfplothandlerlineto
  \mystream
  \pgfplothandlerclosedcurve
  \mystream
  \pgfusepath{stroke}
\end{pgfpicture}
\end{codeexample} 
\end{command}

%%% Local Variables: 
%%% mode: latex
%%% TeX-master: "pgfmanual"
%%% End: 

% Copyright 2006 by Till Tantau
%
% This file may be distributed and/or modified
%
% 1. under the LaTeX Project Public License and/or
% 2. under the GNU Free Documentation License.
%
% See the file doc/generic/pgf/licenses/LICENSE for more details.


\section{Layered Graphics}

\label{section-layers}

\subsection{Overview}

\pgfname\ provides a layering mechanism for composing graphics from
multiple layers. (This mechanism is not be confused with the
conceptual ``software layers'' the \pgfname\ system is composed of.)
Layers are often used in graphic programs. The idea is that you can
draw on the different layers in any order. So you might start drawing
something on the ``background'' layer, then something on the
``foreground'' layer, then something on the ``middle'' layer, and then
something on the background layer once more, and so on. At the end, no
matter in which ordering you drew on the different layers, the layers
are ``stacked on top of each other'' in a fixed ordering to produce
the final picture. Thus, anything drawn on the middle layer would come
on top of everything of the background layer.

Normally, you do not need to use different layers since you will have
little trouble ``ordering'' your graphic commands in such a way that
layers are superfluous. However, in certain situations you only
``know'' what you should draw behind something else after the
``something else'' has been drawn.

For example, suppose you wish to draw a yellow background behind your
picture. The background should be as large as the bounding box of the
picture, plus a little border. If you know the size of the bounding box
of the picture at its beginning, this is easy to accomplish. However,
in general this is not the case and you need to create a
``background'' layer in addition to the standard ``main'' layer. Then,
at the end of the picture, when the bounding box has been established,
you can add a rectangle of the appropriate size to the picture.



\subsection{Declaring Layers}

In \pgfname\ layers are referenced using names. The standard layer,
which is a bit special in certain ways, is called |main|. If nothing
else is specified, all graphic commands are added to the |main|
layer. You can declare a new layer using the following command:

\begin{command}{\pgfdeclarelayer\marg{name}}
  This command declares a layer named \meta{name} for later
  use. Mainly, this will setup some internal bookkeeping.
\end{command}

The next step toward using a layer is to tell \pgfname\ which layers
will be part of the actual picture and which will be their
ordering. Thus, it is possible to have more layers declared than are
actually used.

\begin{command}{\pgfsetlayers\marg{layer list}}
  This command, which should be used \emph{outside} a |{pgfpicture}|
  environment, tells \pgfname\ which layers will be used in
  pictures. They are stacked on top of each other in the order
  given. The layer |main| should always be part of the list. Here is
  an example:
\begin{codeexample}[code only]
\pgfdeclarelayer{background}
\pgfdeclarelayer{foreground}  
\pgfsetlayers{background,main,foreground}
\end{codeexample}
\end{command}


\subsection{Using Layers}

Once the layers of your picture have been declared, you can start to
``fill'' them. As said before, all graphics commands are normally
added to the |main| layer. Using the |{pgfonlayer}| environment, you
can tell \pgfname\ that certain commands should, instead, be added to
the given layer.

\begin{environment}{{pgfonlayer}\marg{layer name}}
  The whole \meta{environment contents} is added to the layer with the
  name \meta{layer name}. This environment can be used anywhere inside
  a picture. Thus, even if it is used inside a |{pgfscope}| or a \TeX\
  group, the contents will still be added to the ``whole'' picture.
  Using this environment multiple times inside the same picture will
  cause the \meta{environment contents} to accumulate.

  \emph{Note:} You can \emph{not} add anything to the |main| layer
  using this environment. The only way to add anything to the main
  layer is to give graphic commands outside all |{pgfonlayer}|
  environments. 

\begin{codeexample}[]
\pgfdeclarelayer{background layer}
\pgfdeclarelayer{foreground layer}
\pgfsetlayers{background layer,main,foreground layer}
\begin{tikzpicture}
  % On main layer:
  \fill[blue] (0,0) circle (1cm);
  
  \begin{pgfonlayer}{background layer}
    \fill[yellow] (-1,-1) rectangle (1,1);
  \end{pgfonlayer}
  
  \begin{pgfonlayer}{foreground layer}
    \node[white] {foreground};
  \end{pgfonlayer}
  
  \begin{pgfonlayer}{background layer}
    \fill[black] (-.8,-.8) rectangle (.8,.8);
  \end{pgfonlayer}

  % On main layer again:
  \fill[blue!50] (-.5,-1) rectangle (.5,1);
\end{tikzpicture}
\end{codeexample}
\end{environment}

\begin{plainenvironment}{{pgfonlayer}\marg{layer name}}
  This is the plain \TeX\ version of the environment.
\end{plainenvironment}

\begin{contextenvironment}{{pgfonlayer}\marg{layer name}}
  This is the Con\TeX t version of the environment.
\end{contextenvironment}






%%% Local Variables: 
%%% mode: latex
%%% TeX-master: "pgfmanual"
%%% End: 

% Copyright 2003 by Till Tantau <tantau@cs.tu-berlin.de>.
%
% This program can be redistributed and/or modified under the terms
% of the LaTeX Project Public License Distributed from CTAN
% archives in directory macros/latex/base/lppl.txt.


\section{Declaring and Using Shadings}

\label{section-shadings}

\subsection{Overview}

A shading is an area in which the color changes smoothly between different
colors. Similarly to an image, a shading must first be declared before
it can be used. Also similarly to an image, a shading is put into a
\TeX-box. Hence, in order to include a shading in a |{pgfpicture}|,
you have to use |\pgftext| around it.

There are three kinds of shadings: horizontal, vertical, and radial
shadings. However, you can rotate and clip shadings like any other
graphics object, which allows you to create more complicated
shadings. Horizontal shadings could be created by rotating a vertical
shading by 90 degrees, but explicit commands for creating both
horizontal and vertical shadings are included for convenience.

Once you have declared a shading, you can insert it into text using
the command |\pgfuseshading|. This command cannot be used directly in
a |{pgfpicture}|, you have to put a |\pgftext| around it. The second
command for using shadings, |\pgfshadepath|, on the other hand, can
only be used  inside |{pgfpicture}| environments. It will ``fill'' the
current path with the shading.

A horizontal shading is a horizontal bar of a certain height whose
color changes smoothly. You must at least specify the colors at the
left and at the right end of the bar, but you can also add color
specifications for points in between. For example, suppose you
which to create a bar that is red at the left end, green in the
middle, and blue at the end. Suppose you would like the bar to be 4cm
long. This could be specified as follows:
\begin{codeexample}[code only]
rgb(0cm)=(1,0,0); rgb(2cm)=(0,1,0); rgb(4cm)=(0,0,1)
\end{codeexample}
This line means that at 0cm (the left end) of the bar, the color
should be red, which has red-green-blue (rgb) components (1,0,0). At
2cm, the bar should be green, and at 4cm it should be blue.
Instead of |rgb|, you can currently also specify |gray| as
color model, in which case only one value is needed, or |color|,
in which case you must provide the name of a color in parentheses. In
a color specification the individual specifications must 
be separated using a semicolon, which may be followed by a whitespace
(like a space or a newline). Individual specifications must be given
in increasing order. 

\subsection{Declaring Shadings}

\begin{command}{\pgfdeclarehorizontalshading\oarg{color list}\marg{shading
      name}\marg{shading height}\marg{color specification}}
  Declares a horizontal shading named \meta{shading name} of the specified
  \meta{height} with the specified colors. The length of the bar is
  deduced automatically from the maximum dimension in the specification.

\begin{codeexample}[]
\pgfdeclarehorizontalshading{myshading}
  {1cm}{rgb(0cm)=(1,0,0); color(2cm)=(green); color(4cm)=(blue)}
\pgfuseshading{myshading}
\end{codeexample}

  The effect of the \meta{color list}, which is a
  comma-separated list of colors, is the following: Normally, when
  this list is empty, once a shading has been declared, it becomes
  ``frozen.'' This means that even if you change a color that was used
  in the declaration of the shading later on, the shading will not
  change. By specifying a \meta{color list} you can specify
  that the shading should be recalculated whenever one of the colors
  listed in the list changes (this includes effects like color
  mixins). Thus, when you specify a \meta{color list},
  whenever the shading is used, \pgfname\ first converts the colors in the
  list to \textsc{rgb} triples using the current values of the
  colors and taking any mixins and blends into account. If the
  resulting \textsc{rgb} triples have not yet been   used, a new
  shading is internally created and used. Note that if the 
  option \meta{color list} is used, then no shading is created until
  the first use of |\pgfuseshading|. In particular, the colors
  mentioned in the shading need not be defined when the declaration is
  given.

  When a shading is recalculated because of a change in the
  colors mentioned in \meta{color list}, the complete shading
  is recalculated. Thus even colors not mentioned in the list will be
  used with their current values, not with the values they had upon
  declaration.
  
\begin{codeexample}[]
\pgfdeclarehorizontalshading[mycolor]{myshading}
  {1cm}{rgb(0cm)=(1,0,0); color(2cm)=(mycolor)}
\colorlet{mycolor}{green}
\pgfuseshading{myshading}
\colorlet{mycolor}{blue}
\pgfuseshading{myshading}
\end{codeexample}
\end{command}


\begin{command}{\pgfdeclareverticalshading\oarg{color list}\marg{shading
      name}\marg{shading width}\marg{color specification}}
   Declares a vertical shading named \meta{shading name} of the
   specified \meta{width}. The height of the bar is deduced
   automatically. The effect of \meta{color list} is the same as for
   horizontal shadings.

\begin{codeexample}[]
\pgfdeclareverticalshading{myshading2}
  {4cm}{rgb(0cm)=(1,0,0); rgb(1.5cm)=(0,1,0); rgb(2cm)=(0,0,1)}
\pgfuseshading{myshading2}
\end{codeexample}
\end{command}


\begin{command}{\pgfdeclareradialshading\oarg{color list}\marg{shading
      name}\marg{center point}\marg{color specification}}
  Declares an radial shading. A radial shading is a circle whose inner
  color changes as specified by the color specification. Assuming that
  the center of the shading is at the origin, the color of the center
  will be the color specified for 0cm and the color of the border of
  the circle will be the color for the maximum dimension given in
  the \meta{color specified}. This maximum will also be the radius of
  the circle. If the \meta{center point} is not at the 
  origin, the whole shading inside the circle (whose size remains
  exactly the same) will be distorted such that the given center now
  has the color specified for 0cm. The effect of \meta{color list} is
  the same as for horizontal shadings. 

\begin{codeexample}[]  
\pgfdeclareradialshading{sphere}{\pgfpoint{0.5cm}{0.5cm}}%
  {rgb(0cm)=(0.9,0,0);
   rgb(0.7cm)=(0.7,0,0);
   rgb(1cm)=(0.5,0,0);
   rgb(1.05cm)=(1,1,1)}
\pgfuseshading{sphere}
\end{codeexample}
\end{command}


\subsection{Using Shadings}
\label{section-shading-a-path}

\begin{command}{\pgfuseshading\marg{shading name}}
  Inserts a previously declared shading into the text. If you wish to
  use it in a |pgfpicture| environment, you should put a |\pgfbox|
  around it.
  
\begin{codeexample}[]
\begin{pgfpicture}
  \pgfdeclareverticalshading{shading}
    {20pt}{color(0pt)=(red); color(20pt)=(blue)}
  \pgftext[at=\pgfpoint{1cm}{0cm}]  {\pgfuseshading{shading}}
  \pgftext[at=\pgfpoint{2cm}{0.5cm}]{\pgfuseshading{shading}}
\end{pgfpicture}
\end{codeexample}
\end{command}

\begin{command}{\pgfshadepath\marg{shading name}\marg{angle}}
  This command must be used inside a |{pgfpicture}| environment. The
  effect is a bit complex, so let us go over it step by step.

  First, \pgfname\ will setup a local scope.

  Second, it uses the current path to clip everything inside this
  scope. However, the current path is once more available after the
  scope, so it can be used, for example, to stroke it.

  Now, the \meta{shading name} should be a shading whose width and
  height are 100\,bp, that is, 100 big points. \pgfname\ has a look at
  the bounding box of the current path. This bounding box is computed
  automatically when a path is computed; however, it can sometimes be
  (quite a bit) too large, especially when complicated curves are
  involved. 

  Inside the scope, the low-level transformation matrix is modified.
  The center of the shading is translated (moved) such that it lies on
  the center of the bounding box of the path. The low-level coordinate
  system is also scaled such that the shading ``covers'' the shading (the 
  details are a bit more complex, see below). Finally, the coordinate
  system is rotated by \meta{angle}.

  After everything has been set up, the shading is inserted. Due to
  the transformations and clippings, the effect will be that  the
  shading seems to ``fill'' the path.

  If both the path and the shadings were always rectangles and if
  rotation were never involved, it would be easy to scale shadings
  such they always cover the path. However, when a vertical shading is
  rotated, it must obviously be ``magnified'' so that it
  still covers the path. Things get worse when the path is not a
  rectangle itself.

  For these reasons, things work slightly differently ``in reality.''
  The shading is scaled and translated such that the
  the point $(50\mathrm{bp},50\mathrm{bp})$, which is the middle of
  the shading, is at the middle of the path and such that the the
  point $(25\mathrm{bp},25\mathrm{bp})$ is at the lower left corner of
  the path and that  $(75\mathrm{bp},75\mathrm{bp})$  is at upper
  right corner.

  In other words, only the center quarter of the shading will actually
  ``survive the clipping'' if the path is a rectangle. If the path is
  not a rectangle, but, say, a circle, even less is seen of the
  shading. Here is an example that demonstrates this effect:

\begin{codeexample}[]
\pgfdeclareverticalshading{shading}{100bp}    
 {color(0bp)=(red); color(25bp)=(green);  color(75bp)=(blue);  color(100bp)=(black)}
\pgfuseshading{shading}
\hskip 1cm
\begin{pgfpicture}
  \pgfpathrectangle{\pgfpointorigin}{\pgfpoint{2cm}{1cm}}
  \pgfshadepath{shading}{0}
  \pgfusepath{stroke}
  \pgfpathrectangle{\pgfpoint{3cm}{0cm}}{\pgfpoint{1cm}{2cm}}
  \pgfshadepath{shading}{0}
  \pgfusepath{stroke}
  \pgfpathrectangle{\pgfpoint{5cm}{0cm}}{\pgfpoint{2cm}{2cm}}
  \pgfshadepath{shading}{45}
  \pgfusepath{stroke}
  \pgfpathcircle{\pgfpoint{9cm}{1cm}}{1cm}
  \pgfshadepath{shading}{45}
  \pgfusepath{stroke}
\end{pgfpicture}
\end{codeexample}

  As can be seen above in the last case, the ``hidden'' part of the
  shading actually \emph{can} become visible if the shading is
  rotated. The reason is that it is scaled as if no rotation took
  place, then the rotation is done.

  The following graphics show which part of the shading are actually
  shown: 

\begin{codeexample}[]
\begin{tikzpicture}
  \draw (50bp,50bp) node {\pgfuseshading{shading}};
  \draw[white,thick] (25bp,25bp) rectangle (75bp,75bp);
  \draw (50bp,0bp) node[below] {first two applications};

  \begin{scope}[xshift=5cm]
    \draw (50bp,50bp) node{\pgfuseshading{shading}};
    \draw[rotate around={45:(50bp,50bp)},white,thick] (25bp,25bp) rectangle (75bp,75bp);
    \draw (50bp,0bp) node[below] {third application};
  \end{scope}

  \begin{scope}[xshift=10cm]
    \draw (50bp,50bp) node{\pgfuseshading{shading}};
    \draw[white,thick] (50bp,50bp) circle (25bp);
    \draw (50bp,0bp) node[below] {fourth application};
  \end{scope}
\end{tikzpicture}
\end{codeexample}
  
  An advantage of this approach is that when you rotate a radial
  shading, no distortion is introduced:

\begin{codeexample}[]
\pgfdeclareradialshading{ballshading}{\pgfpoint{-10bp}{10bp}}
 {color(0bp)=(red!15!white); color(9bp)=(red!75!white);
 color(18bp)=(red!70!black); color(25bp)=(red!50!black); color(50bp)=(black)}
\pgfuseshading{ballshading}
\hskip 1cm
\begin{pgfpicture}
  \pgfpathrectangle{\pgfpointorigin}{\pgfpoint{1cm}{1cm}}
  \pgfshadepath{ballshading}{0}
  \pgfusepath{}
  \pgfpathcircle{\pgfpoint{3cm}{0cm}}{1cm}
  \pgfshadepath{ballshading}{0}
  \pgfusepath{}
  \pgfpathcircle{\pgfpoint{6cm}{0cm}}{1cm}
  \pgfshadepath{ballshading}{45}
  \pgfusepath{}
\end{pgfpicture}
\end{codeexample}

  If you specify a rotation of $90^\circ$
  and if the path is not a square, but an elongated rectangle,  the
  ``desired'' effect results: The shading will exactly vary between
  the colors at the 25bp and 75bp boundaries. Here is an example:
  
\begin{codeexample}[]
\begin{pgfpicture}
  \pgfpathrectangle{\pgfpointorigin}{\pgfpoint{2cm}{1cm}}
  \pgfshadepath{shading}{0}
  \pgfusepath{stroke}
  \pgfpathrectangle{\pgfpoint{3cm}{0cm}}{\pgfpoint{2cm}{1cm}}
  \pgfshadepath{shading}{90}
  \pgfusepath{stroke}
  \pgfpathrectangle{\pgfpoint{6cm}{0cm}}{\pgfpoint{2cm}{1cm}}
  \pgfshadepath{shading}{45}
  \pgfusepath{stroke}
\end{pgfpicture}
\end{codeexample}


  As a final example, let us define a ``rainbow spectrum'' shading for
  use with \tikzname.
\begin{codeexample}[]
\pgfdeclareverticalshading{rainbow}{100bp}
 {color(0bp)=(red); color(25bp)=(red); color(35bp)=(yellow);
  color(45bp)=(green); color(55bp)=(cyan); color(65bp)=(blue);
  color(75bp)=(violet); color(100bp)=(violet)}
\begin{tikzpicture}[shading=rainbow]
  \shade (0,0) rectangle node[white] {\textsc{pride}} (2,1);
  \shade[shading angle=90] (3,0) rectangle +(1,2);
\end{tikzpicture}
\end{codeexample}

  Note that rainbow shadings are \emph{way} to colorful in almost all
  applications. 
\end{command}

%%% Local Variables: 
%%% mode: latex
%%% TeX-master: "pgfmanual"
%%% End: 

% Copyright 2006 by Till Tantau
%
% This file may be distributed and/or modified
%
% 1. under the LaTeX Project Public License and/or
% 2. under the GNU Free Documentation License.
%
% See the file doc/generic/pgf/licenses/LICENSE for more details.


\section{Transparency}

\label{section-transparency}


For an introduction to the notion of transparency, fadings, and
transparency groups, please consult Section~\ref{section-tikz-transparency}. 


\subsection{Specifying a Uniform Opacity}

Specifying a stroke and/or fill opacity is quite easy.

\begin{command}{\pgfsetstrokeopacity\marg{value}}
  Sets the opacity of stroking operations. The \meta{value} should be
  a number between |0| and |1|, where |1| means ``fully opaque'' and
  |0| means ``fully transparent.'' A value like |0.5| will cause paths
  to be stroked in a semitransparent way.
  
\begin{codeexample}[]
\begin{pgfpicture}
  \pgfsetlinewidth{5mm}
  \color{red}
  \pgfpathcircle{\pgfpoint{0cm}{0cm}}{10mm} \pgfusepath{stroke}
  \color{black}
  \pgfsetstrokeopacity{0.5}
  \pgfpathcircle{\pgfpoint{1cm}{0cm}}{10mm} \pgfusepath{stroke}
\end{pgfpicture}
\end{codeexample}
\end{command}


\begin{command}{\pgfsetfillopacity\marg{value}}
  Sets the opacity of filling operations. As for stroking, the
  \meta{value} should be a number between |0| and~|1|.

  The ``filling transparency'' will also be used for text and images.  
  
\begin{codeexample}[]
\begin{tikzpicture}
  \pgfsetfillopacity{0.5}
  \fill[red]   (90:1cm)  circle (11mm);
  \fill[green] (210:1cm) circle (11mm);
  \fill[blue]  (-30:1cm) circle (11mm);
\end{tikzpicture}
\end{codeexample}
\end{command}

Note the following effect: If you setup a certain opacity for stroking
or filling and you stroke or fill the same area twice, the effect
accumulates:

\begin{codeexample}[]
\begin{tikzpicture}
  \pgfsetfillopacity{0.5}
  \fill[red] (0,0) circle (1);
  \fill[red] (1,0) circle (1);
\end{tikzpicture}
\end{codeexample}

Often, this is exactly what you intend, but not always. You can use
transparency groups, see the end of this section, to change this.


\subsection{Specifying a Fading}

The method used by \pgfname\ for specifying fadings is quite
general: You ``paint'' the fading using any of the standard graphics
commands. In more detail: You create a normal picture, which may even
contain text, image, and shadings. Then, you create a fading based on
this picture. For this, the \emph{luminosity} of each pixel of the
picture is analysed (the brighter the pixel, the higher the luminosity
-- a black pixel has luminosity $0$, a white pixel has luminosity $1$,
a gray pixel has some intermediate value as does a red pixel). Then,
when the fading is used, the luminosity of the pixel determines the
opacity of the fading at that position. Positions in the fading where
the picture was black will be completely transparent, positions where
the picture was white will be completely opaque. Positions that have
not been painted at all in the picture are always completely
transparent.


\begin{command}{\pgfdeclarefading\marg{name}\marg{contents}}
  This command declare a fading named \meta{name} for later use. The
  ``picture'' on which the fading is based is given by the
  \meta{contents}. This \meta{contents} is normally typeset in a \TeX\
  box. The resulting box is then used as the ``picture.'' In
  particular, inside the \meta{contents} you must explicitly open a
  |{pgfpicture}| environment if you wish to use \pgfname\ commands.

  Let's start with an easy example. Our first fading picture is just
  some text:
\begin{codeexample}[]
\pgfdeclarefading{fading1}{\color{white}Ti\emph{k}Z}    
\begin{tikzpicture}
  \fill [black!20] (0,0) rectangle (2,2);
  \fill [black!30] (0,0) arc (180:0:1);
  \pgfsetfading{fading1}{\pgftransformshift{\pgfpoint{1cm}{1cm}}}
  \fill [red] (0,0) rectangle (2,2);
\end{tikzpicture}
\end{codeexample}
  What's happening here? The ``fading picture'' is mostly transparent,
  except for the pixels that are part of the word Ti\emph{k}Z. Now,
  these pixels are \emph{white} and, thus, have a high
  luminosity. This in turn means that these pixels of the fading will
  be highly opaque. For this reason, only those pixels of the big red
  rectangle ``shine through'' that are at the positions of these
  opaque pixels.

  It is somewhat counter-intuitive that the white pixels in a fading
  picture are opaque in a fading. For this reason, the color
  |pgftransparent| is defined to be the same as |black|. This allows
  one to write |pgftransparent| for completely transparent parts of a
  fading picture and |pgftransparent!0| for the opaque parts and
  things liek |pgftransparent!20| for parts that are 20\%
  transparent.

  Furthermore, the color |pgftransparent!0| (which is the same as
  white and which corresponds to completely opaque) is installed at
  the beginning of a fading picture. Thus, in the above example the
  |\color{white}| was not really necessary.

  Next, let us create a fading that gets more and more transparent as
  we go from left to right. For this, we put a shading inside the
  fading picture that has the color |pgftransparent!0| at the
  left-hand side and the color |pgftransparent!100| at the right-hand
  side. 
\begin{codeexample}[]
\pgfdeclarefading{fading2}
{\tikz \shade[left color=pgftransparent!0,
              right color=pgftransparent!100] (0,0) rectangle (2,2);}    
\begin{tikzpicture}
  \fill [black!20] (0,0) rectangle (2,2);
  \fill [black!30] (0,0) arc (180:0:1);
  \pgfsetfading{fading2}{\pgftransformshift{\pgfpoint{1cm}{1cm}}}
  \fill [red] (0,0) rectangle (2,2);
\end{tikzpicture}
\end{codeexample}

  In our final example, we create a fading that is based on a radial
  shading.
\begin{codeexample}[]
\pgfdeclareradialshading{myshading}{\pgfpointorigin}
{
  color(0mm)=(pgftransparent!0);
  color(5mm)=(pgftransparent!0);
  color(8mm)=(pgftransparent!100);
  color(15mm)=(pgftransparent!100)
}
\pgfdeclarefading{fading3}{\pgfuseshading{myshading}}
\begin{tikzpicture}
  \fill [black!20] (0,0) rectangle (2,2);
  \fill [black!30] (0,0) arc (180:0:1);
  \pgfsetfading{fading3}{\pgftransformshift{\pgfpoint{1cm}{1cm}}}
  \fill [red] (0,0) rectangle (2,2);
\end{tikzpicture}
\end{codeexample}
\end{command}

After having declared a fading, we can use it. As for shadings, there
are two different commands for using fadings:

\begin{command}{\pgfsetfading\marg{name}\marg{transformations}}
  This command sets the graphic state parameter ``fading'' to a
  previously defined fading \meta{name}. This graphic state works like
  other graphic states, that is, is persists till the end of the
  current scope or until a different transparency setting is chosen.

  When the fading is installed, it will be centered on the origin with
  its natural size. Anything outside the fading pictures's original
  bounding box will be transparent and, thus, the fading effectively
  clips against this bounding box.

  The \meta{transformations} are applied to the fading before it is
  used. They contain normal \pgfname\ transformation commands like
  |\pgftransformshift|. You can also scale the fading using this
  command. Note, however, that the transformation needs to be inverted
  internally, which may result in inaccuracies and the following
  graphics may be slightly distorted if you use a strong
  \meta{transformation}.
\begin{codeexample}[]
\pgfdeclarefading{fading2}
{\tikz \shade[left color=pgftransparent!0,
              right color=pgftransparent!100] (0,0) rectangle (2,2);}    
\begin{tikzpicture}
  \fill [black!20] (0,0) rectangle (2,2);
  \fill [black!30] (0,0) arc (180:0:1);
  \pgfsetfading{fading2}{}
  \fill [red] (0,0) rectangle (2,2);
\end{tikzpicture}
\end{codeexample}
\begin{codeexample}[]
\begin{tikzpicture}
  \fill [black!20] (0,0) rectangle (2,2);
  \fill [black!30] (0,0) arc (180:0:1);
  \pgfsetfading{fading2}{\pgftransformshift{\pgfpoint{1cm}{1cm}}
                         \pgftransformrotate{20}}
  \fill [red] (0,0) rectangle (2,2);
\end{tikzpicture}
\end{codeexample}
\end{command}

\begin{command}{\pgfsetfadingforcurrentpath\marg{name}\marg{transformations}}
  This command works like |\pgfsetfading|, but the fading is scaled
  are transformed according to the following rules:
  \begin{enumerate}
  \item
    If the current path is empty, the command has the same effect as
    |\pgfsetfading|. 
  \item
    Otherwise it is assumed that the fading has a size of 100bp times
    100bp. 
  \item
    The fading is resized and shiften (using appropriate
    transformations) such that the position
    $(25\mathrm{bp},25\mathrm{bp})$ lies at the lower-left corner of
    the current path and the position $(75\mathrm{bp},75\mathrm{bp})$
    lies at the upper-right corner of the current path.
  \end{enumerate}
  Note that these rules are the same as the ones used in
  |\pgfshadepath| for shadings. After these transformations, the
  \meta{transformations} are executed (typically a rotation).
\begin{codeexample}[]
\pgfdeclarehorizontalshading{shading}{100bp}
{ color(0pt)=(transparent!0);    color(25bp)=(transparent!0);
  color(75bp)=(transparent!100); color(100bp)=(transparent!100)}

\pgfdeclarefading{fading}{\pgfuseshading{shading}}

\begin{tikzpicture}
  \fill [black!20] (0,0) rectangle (2,2);
  \fill [black!30] (0,0) arc (180:0:1);

  \pgfpathrectangle{\pgfpointorigin}{\pgfpoint{2cm}{1cm}}
  \pgfsetfadingforcurrentpath{fading}{}
  \pgfusepath{discard}
  
  \fill [red] (0,0) rectangle (2,1);

  \pgfpathrectangle{\pgfpoint{0cm}{1cm}}{\pgfpoint{2cm}{1cm}}
  \pgfsetfadingforcurrentpath{fading}{\pgftransformrotate{90}}
  \pgfusepath{discard}

  \fill [red] (0,1) rectangle (2,2);
\end{tikzpicture}
\end{codeexample}

\end{command}

\subsection{Transparency Groups}

Transparency groups are declared using the following commands.

\begin{environment}{{pgftransparencygroup}}
  This environment should only be used inside a |{pgfpicture}|. It has
  the following effect:
  \begin{enumerate}
  \item The \meta{environment contents} is stroked/filled
    ``ignoring any outside transparency.'' This means, all previous
    transparency settings are ignored (you can still set transparency
    inside the group, but never mind). This means that if in the
    \meta{environment contents} you stroke a pixel three times in
    black, it is just black. Stroking it white afterwards yields a
    white pixel, and so on.
  \item When the group is finished, it is painted as a whole. The 
    \emph{fill} transparency settings are now applied to the resulting
    picutre. For instance, the pixel that has been painted three times
    in black and once in white is just white at the end, so this white
    color will be blended with whatever is ``behind'' the group on the
    page.
  \end{enumerate}

  Note that, depending on the driver, \pgfname\ may have to guess the
  size of the contents of the transparency group (because such a group
  is put in an XForm in \textsc{pdf} and a bounding box must be
  supplied). \pgfname\ will use normally use the size of the picture's
  bounding box at the end of the transparency group plus a safety
  margin of 1cm. Under normal circumstances, this will work nicely
  since the picture's bounding box contains everything
  anyway. However, if you have switched off the picture size tracking
  or if you are using canvas transformations, you may have to make
  sure that the bounding box is big enough. The trick is to locallly
  create a picture that is ``large enough'' and then insert this
  picture into the main picture while ignoring the size. The following
  example shows how this is done:

  
\begin{codeexample}[]
\begin{tikzpicture}
  \draw [help lines] (0,0) grid (2,2);

  % Stuff outside the picture, but still in a transparency group.
  \node [left,overlay] at (0,1) {
    \begin{tikzpicture}
      \pgfsetfillopacity{0.5}
      \pgftransparencygroup
      \node at (2,0) [forbidden sign,line width=2ex,draw=red,fill=white]
        {Smoking};
      \endpgftransparencygroup
    \end{tikzpicture}  
  };
\end{tikzpicture}
\end{codeexample}


\begin{plainenvironment}{{pgftransparencygroup}}
  Plain \TeX\ version of the |{pgftransparencygroup}| environment.
\end{plainenvironment}

\begin{contextenvironment}{{pgftransparencygroup}}
  This is the Con\TeX t version of the environment.
\end{contextenvironment}

\end{environment}



%%% Local Variables: 
%%% mode: latex
%%% TeX-master: "pgfmanual"
%%% End: 

% Copyright 2006 by Till Tantau
%
% This file may be distributed and/or modified
%
% 1. under the LaTeX Project Public License and/or
% 2. under the GNU Free Documentation License.
%
% See the file doc/generic/pgf/licenses/LICENSE for more details.


\section{Quick Commands}

This section explains the ``quick'' commands of \pgfname. These
commands are executed more quickly than the normal commands of
\pgfname, but offer less functionality. You should use these commands
only if you either have a very large number of commands that need to
be processed or if you expect your commands to be executed very often.


\subsection{Quick Coordiante Commands}

\begin{command}{\pgfqpoint\marg{x}\marg{y}}
  This command does the same as |\pgfpoint|, but \meta{x} and \meta{y}
  must be simple dimensions like |1pt| or |1cm|. Things like |2ex| or
  |2cm+1pt| are not allowed.
\end{command}

\subsection{Quick Path Construction Commands}

The difference between the quick and the normal path commands is that
the quick path commands
\begin{itemize}
\item
  do not keep track of the bounding boxes,
\item
  do not allow you to arc corners,
\item
  do not apply coordinate transformations.
\end{itemize}

However, they do use the soft-path subsystem (see
Section~\ref{section-soft-paths} for details), which allows you to mix
quick and normal path commands arbitrarily.

All quick path construction commands start with |\pgfpathq|.

\begin{command}{\pgfpathqmoveto\marg{x dimension}\marg{y dimension}}
  Either starts a path or starts a new part of a path at the coordinate
  $(\meta{x dimension},\meta{y dimension})$. The coordinate is
  \emph{not} transformed by the current coordinate transformation
  matrix. However, any low-level transformations apply.

\begin{codeexample}[]
\begin{tikzpicture}
  \draw[help lines] (0,0) grid (3,2);
  \pgftransformxshift{1cm}
  \pgfpathqmoveto{0pt}{0pt} % not transformed
  \pgfpathqlineto{1cm}{1cm} % not transformed
  \pgfpathlineto{\pgfpoint{2cm}{0cm}}
  \pgfusepath{stroke}
\end{tikzpicture}
\end{codeexample}
\end{command}

\begin{command}{\pgfpathqlineto\marg{x dimension}\marg{y dimension}}
  The quick version of the line-to operation.
\end{command}

\begin{command}{\pgfpathqcurveto\marg{$s^1_x$}\marg{$s^1_y$}\marg{$s^2_x$}\marg{$s^2_y$}\marg{$t_x$}\marg{$t_y$}}
  The quick version of the curve-to operation. The first support point
  is $(s^1_x,s^1_y)$, the second support point is  $(s^2_x,s^2_y)$,
  and the target is $(t_x,t_y)$.
 
\begin{codeexample}[]
\begin{tikzpicture}
  \draw[help lines] (0,0) grid (3,2);
  \pgfpathqmoveto{0pt}{0pt}
  \pgfpathqcurveto{1cm}{1cm}{2cm}{1cm}{3cm}{0cm}
  \pgfusepath{stroke}
\end{tikzpicture}
\end{codeexample}
\end{command}

\begin{command}{\pgfpathqcircle\marg{radius}}
  Adds a radius around the origin of the given \meta{radius}. This
  command is orders of magnitude faster than
  |\pgfcircle{\pgfpointorigin}{|\meta{radius}|}|. 
 
\begin{codeexample}[]
\begin{tikzpicture}
  \draw[help lines] (0,0) grid (1,1);
  \pgfpathqcircle{10pt}
  \pgfsetfillcolor{examplefill}
  \pgfusepath{stroke,fill}
\end{tikzpicture}
\end{codeexample}
\end{command}



\subsection{Quick Path Usage Commands}

The quick path usage commands perform similar tasks as |\pgfusepath|,
but they
\begin{itemize}
\item
  do not add arrows,
\item
  do not modify the path in any way, in particular,
\item
  ends are not shortened,
\item
  corners are not replaced by arcs.
\end{itemize}

Note that you \emph{have to} use the quick versions in the code of
arrow tip definitions since, inside these definition, you obviously do
not want arrows to be drawn.

\begin{command}{\pgfusepathqstroke}
  Strokes the path without further ado. No arrows are drawn, no
  corners are arced.

\begin{codeexample}[]
\begin{pgfpicture}
  \pgfpathqcircle{5pt}
  \pgfusepathqstroke
\end{pgfpicture}
\end{codeexample}
\end{command}

\begin{command}{\pgfusepathqfill}
  Fills the path without further ado.
\end{command}

\begin{command}{\pgfusepathqfillstroke}
  Fills and then strokes the path without further ado.
\end{command}

\begin{command}{\pgfusepathqclip}
  Clips all subsequent drawings against the current path. The path is
  not processed.
\end{command}


\subsection{Quick Text Box Commands}

\begin{command}{\pgfqbox\marg{box number}}
  This command inserts a \TeX\ box into a |{pgfpicture}| by
  ``escaping'' to \TeX, inserting the box number \meta{box number} at
  the origin, and then returning to the typesetting the picture.
\end{command}

\begin{command}{\pgfqboxsynced\marg{box number}}
  This command works similarly to the |\pgfqbox| command. However,
  before inserting the text in \meta{box number}, the current
  coordinate transformation matrix is applied to the current canvas
  transformation matrix (is it ``synced'' with this matrix, hence the
  name).

  Thus, this command basically has the same effect as if you first
  called |\pgflowlevelsynccm| followed by |\pgfqbox|. However, this
  command will use |\hskip| and |\raise| commands for the
  ``translational part'' of the coordinate transformation matrix,
  instead of adding the translational part to the current
  canvas transformation matrix directly. Both methods have the same
  effect (box \meta{box number} is translated where it should), but
  the method used by |\pgfqboxsynced| ensures that hyperlinks are
  placed correctly. Note that scaling and rotation will not (cannot,
  even) apply to hyperlinks.
\end{command}

%%% Local Variables: 
%%% mode: latex
%%% TeX-master: "pgfmanual"
%%% End: 





\part{The System Layer}
\label{part-system}

{\Large \emph{by Till Tantau}}


\bigskip
\noindent
This part describes the low-level interface of \pgfname, called the
\emph{system layer}. This interface provides a complete abstraction of
the internals of the underlying drivers. 

Unless you intend to port \pgfname\ to another driver or unless you intend
to write your own optimized frontend, you need not read this part.

In the following it is assumed that you are familiar with the basic
workings of the |graphics| package and that you know what
\TeX-drivers are and how they work.

\vskip1cm
\begin{codeexample}[graphic=white]
\begin{tikzpicture}
  [shorten >=1pt,->,
   vertex/.style={circle,fill=black!25,minimum size=17pt,inner sep=0pt}]
  
  \foreach \name/\x in {s/1, 2/2, 3/3, 4/4, 15/11, 16/12, 17/13, 18/14, 19/15, t/16}
    \node[vertex] (G-\name) at (\x,0) {$\name$};

  \foreach \name/\angle/\text in {P-1/234/5, P-2/162/6, P-3/90/7, P-4/18/8, P-5/-54/9}
    \node[vertex,xshift=6cm,yshift=.5cm] (\name) at (\angle:1cm) {$\text$};
  
  \foreach \name/\angle/\text in {Q-1/234/10, Q-2/162/11, Q-3/90/12, Q-4/18/13, Q-5/-54/14}
    \node[vertex,xshift=9cm,yshift=.5cm] (\name) at (\angle:1cm) {$\text$};

  \foreach \from/\to in {s/2,2/3,3/4,3/4,15/16,16/17,17/18,18/19,19/t}
    \draw (G-\from) -- (G-\to);  

  \foreach \from/\to in {1/2,2/3,3/4,4/5,5/1,1/3,2/4,3/5,4/1,5/2}
    { \draw (P-\from) -- (P-\to); \draw (Q-\from) -- (Q-\to); }

  \draw (G-3) .. controls +(-30:2cm) and +(-150:1cm) .. (Q-1);
  \draw (Q-5) -- (G-15);
\end{tikzpicture}
\end{codeexample}

% Copyright 2006 by Till Tantau
%
% This file may be distributed and/or modified
%
% 1. under the LaTeX Project Public License and/or
% 2. under the GNU Free Documentation License.
%
% See the file doc/generic/pgf/licenses/LICENSE for more details.



\section{Design of the System Layer}

\makeatletter


\subsection{Driver Files}
\label{section-pgfsys}

The \pgfname\ system layer mainly consists of a large number of
commands starting with |\pgfsys@|. These commands will be called
\emph{system commands} in the following. The higher layers
``interface'' with the system layer by calling these commands. The
higher layers should never use |\special| commands directly or even
check whether |\pdfoutput| is defined. Instead, all drawing requests
should be ``channeled'' through the system commands. 

The system layer is loaded and setup by the following package:

\begin{package}{pgfsys}
  This file provides ``default implementations'' of all system
  commands, but most simply produce a warning that they are not
  implemented. The actual implementations of the system commands for a
  particular driver like, say, |pdftex| reside in files called
  |pgfsys-xxxx.sty|, where |xxxx| is the driver name. These will be
  called \emph{driver files} in the following.

  When |pgfsys.sty| is loaded, it will try to determine which driver
  is used by loading |pgf.cfg|. This file should setup the macro
  |\pgfsysdriver| appropriately. The, |pgfsys.sty| will input the
  appropriate |pgfsys-|\meta{drivername}|.sty|. 
\end{package}

\begin{command}{\pgfsysdriver}
  This macro should expand to the name of the driver to be used by
  |pgfsys|. The default from |pgf.cfg| is |pgfsys-\Gin@driver|. This
  is very likely to be correct if you are using \LaTeX. For plain
  \TeX, the macro will be set to |pgfsys-pdftex.def| if |pdftex| is
  used and to |pgfsys-dvips.def| otherwise.
\end{command}

\begin{filedescription}{pgf.cfg}
  This file should setup the command |\pgfsysdriver| correctly. If
  |\pgfsysdriver| is already set to some value, the driver normally
  should not change it. Otherwise, it should make a ``good guess'' at
  which driver will be appropriate.
\end{filedescription}


The currently supported backend drivers are discussed in
Section~\ref{section-drivers}. 


\subsection{Common Definition Files}

Some drivers share many |\pgfsys@| commands. For the reason, files
defining these ``common'' commands are available. These files are
\emph{not} usable alone.

\begin{filedescription}{pgfsys-common-postscript}
  This file defines some |\pgfsys@| commands so that they produce
  appropriate PostScript code.
\end{filedescription}

\begin{filedescription}{pgfsys-common-pdf}
  This file defines some |\pgfsys@| commands so that they produce
  appropriate \textsc{pdf} code.
\end{filedescription}


%%% Local Variables: 
%%% mode: latex
%%% TeX-master: "pgfmanual"
%%% End: 

% Copyright 2003 by Till Tantau <tantau@cs.tu-berlin.de>.
%
% This program can be redistributed and/or modified under the terms
% of the LaTeX Project Public License Distributed from CTAN
% archives in directory macros/latex/base/lppl.txt.

\section{Commands of the System Layer}

\makeatletter

\subsection{Beginning and Ending a Stream of System Commands}

A ``user'' of the \pgfname\ system layer (like the basic layer or a
frontend) will interface with the system layer by calling a stream of
commands starting with |\pgfsys@|. From the system layer's point of
view, these commands form a long stream. Between calls to the system
layer, control goes back to the user.

The driver files implement system layer commands by inserting
|\special| commands that implement the desired operation. For example,
|\pgfsys@stroke| will be mapped to |\special{pdf: S}| by the driver
file for |pdftex|.

For many drivers, when such a stream of specials starts, it is
necessary to install an appropriate transformation and perhaps perform
some more bureaucratic tasks. For this reason, every stream will start
with a |\pgfsys@beginpicture| and will end with a corresponding ending
command.

\begin{command}{\pgfsys@beginpicture}
  Called at the beginning of a |{pgfpicture}|. This command should
  ``setup things.''

  Most drivers will need to implement this command.
\end{command}

\begin{command}{\pgfsys@endpicture}
  Called at the end of a pgfpicture. 

  Most drivers will need to implement this command.
\end{command}

\begin{command}{\pgfsys@typesetpicturebox\marg{box}}
  Called \emph{after} a |{pgfpicture}| has been typeset. The picture
  will have been put in box \meta{box}. This command should insert the
  box into the normal text. The box \meta{box} will still be a ``raw''
  box that contains only the |\special|'s that make up the description
  of the picture. The  job of this command is to resize and shift
  \meta{box} according to the  baseline shift and the size of the
  box. 

  This command has a default implementation and need not be
  implemented by a driver file.
\end{command}

\begin{command}{\pgfsys@beginpurepicture}
  This version of the |\pgfsys@beginpicture| picture command can be
  used for pictures that are guaranteed not to contain any escaped
  boxes (see below). In this case, a driver might provide a more
  compact version of the command. 
  
  This command has a default implementation and need not be
  implemented by a driver file.
\end{command}

\begin{command}{\pgfsys@endpurepicture}
  Called at the end of a ``pure'' |{pgfpicture}|.
  
  This command has a default implementation and need not be
  implemented by a driver file.
\end{command}

Inside a stream it is sometimes necessary to ``escape'' back into
normal typesetting mode; for example to insert some normal text, but
with all of the current transformations and clippings being in
force. For this escaping, the following command is used:

\begin{command}{\pgfsys@hbox\marg{box number}}
  Called to insert a (horizontal) TeX box inside a
  |{pgfpicture}|.
  
  Most drivers will need to (re-)implement this command.
\end{command}

\begin{command}{\pgfsys@hboxsynced\marg{box number}}
  Called to insert a (horizontal) TeX box inside a
  |{pgfpicture}|, but with the current coordiante transformation
  matrix synced with the canvas transformation matrix. 

  In essence, this command does the same as if you first said
  |\pgflowlevelsynccm| and then |\pgfsys@hbox|. However, the default
  implementation of this command will use a ``TeX-translation'' for
  the translation part of the transformation matrix. This will ensure
  that hyperlinks ``survive'' at least translations. On the other
  hand, a driver may choose to revert to a simpler
  implementation. This is done, for example, for the \textsc{svg}
  implementation, where a \TeX-translation makes no sense.
\end{command}



\subsection{Path Construction System Commands}

\begin{command}{\pgfsys@moveto\marg{x}\marg{y}}
  This command is used to start a path at a specific point
  $(x,y)$ or to move the current point of the current path to  $(x,y)$
  without drawing anything upon stroking (the current path is
  ``interrupted'').

  Both \meta{x} and \meta{y} are given as \TeX\ dimensions. It is the
  driver's job to transform these to the coordinate system of the
  backend. Typically, this means converting the \TeX\ dimension into a
  dimensionless multiple of $\frac{1}{72}\mathrm{in}$. The function
  |\pgf@sys@bp| helps with this conversion.

  \example Draw a line from $(10\mathrm{pt},10\mathrm{pt})$ to the
  origin of the picture. 
\begin{codeexample}[code only]
\pgfsys@moveto{10pt}{10pt}
\pgfsys@lineto{0pt}{0pt}
\pgfsys@stroke
\end{codeexample}

  This command is protocoled, see Section~\ref{section-protocols}.
\end{command}


\begin{command}{\pgfsys@lineto\marg{x}\marg{y}}
  Continue the current path to $(x,y)$ with
  a straight line.

  This command is protocoled, see Section~\ref{section-protocols}.
\end{command}


\begin{command}{\pgfsys@curveto\marg{$x_1$}\marg{$y_1$}\marg{$x_2$}\marg{$y_2$}\marg{$x_3$}\marg{$y_3$}}
  Continue the current path to $(x_3,y_3)$
  with a B�zier curve that has the two control points  $(x_1,y_1)$ and  $(x_2,y_2)$.

  \example Draw a good approximation of a quarter circle:
\begin{codeexample}[code only]
\pgfsys@moveto{10pt}{0pt}
\pgfsys@curveto{10pt}{5.55pt}{5.55pt}{10pt}{0pt}{10pt}
\pgfsys@stroke
\end{codeexample}

  This command is protocoled, see Section~\ref{section-protocols}.
\end{command}


\begin{command}{\pgfsys@rect\marg{x}\marg{y}\marg{width}\marg{height}}
  Append a rectangle to the current path whose lower left corner is
  at $(x,y)$ and whose width and height in
  big points are  given by \meta{width} and \meta{height}.

  This command can be ``mapped back'' to |\pgfsys@moveto| and
  |\pgfsys@lineto| commands, but it is included since \pdf\ has a
  special, quick version of this command. 

  This command is protocoled, see Section~\ref{section-protocols}.
\end{command}


\begin{command}{\pgfsys@closepath}
  Close the current path. This results in joining the current point of
  the path with the point specified by the last |\pgfsys@moveto|
  operation. Typically, this is preferable over using |\pgfsys@lineto|
  to the last point specified by a |\pgfsys@moveto|, since the line
  starting at this point and the line ending at this point will be
  smoothly joined by |\pgfsys@closepath|.

  \example Consider
\begin{codeexample}[code only]
\pgfsys@moveto{0pt}{0pt}
\pgfsys@lineto{10bp}{10bp}
\pgfsys@lineto{0bp}{10bp}
\pgfsys@closepath
\pgfsys@stroke
\end{codeexample}
  and
\begin{codeexample}[code only]
\pgfsys@moveto{0bp}{0bp}
\pgfsys@lineto{10bp}{10bp}
\pgfsys@lineto{0bp}{10bp}
\pgfsys@lineto{0bp}{0bp}
\pgfsys@stroke
\end{codeexample}
  
  The difference between the above will be that in the second triangle
  the corner at the origin will be wrong; it will just be the overlay
  of two lines going in different directions, not a sharp pointed
  corner.

  This command is protocoled, see Section~\ref{section-protocols}.
\end{command}




\subsection{Canvas Transformation System Commands}

\begin{command}{\pgfsys@transformcm\marg{a}\marg{b}\marg{c}\marg{d}\marg{e}\marg{f}}
  Perform a concatenation of the canvas transformation matrix with the
  matrix given by the values \meta{a} to \meta{f}, see the \pdf\ or
  PostScript manual for details. The values \meta{a} to \meta{d} are
  dimensionless factors, \meta{e} and \meta{f} are \TeX\ dimensions 

  \example |\pgfsys@transformcm{1}{0}{0}{1}{1cm}{1cm}|.

  This command is protocoled, see Section~\ref{section-protocols}.  
\end{command}


\begin{command}{\pgfsys@transformshift\marg{x displacement}\marg{y displacement}}
  This command will change the origin of the canvas to $(x,y)$.

  This command has a default implementation and need not be
  implemented by a driver file.

  This command is protocoled, see Section~\ref{section-protocols}.
\end{command}

\begin{command}{\pgfsys@transformxyscale\marg{x scale}\marg{y scale}}
  This command will scale the canvas (and  everything that is drawn)
  by a factor of \meta{x scale} in the $x$-direction and \meta{y
    scale} in the  $y$-direction. Note that this applies to
  everything, including  lines. So a scaled line will have a different
  width and may even have a different width when going along the
  $x$-axis and when going along the $y$-axis, if the scaling is
  different in these directions. Usually, you do not want this.

  This command has a default implementation and need not be
  implemented by a driver file.

  This command is protocoled, see Section~\ref{section-protocols}.
\end{command}


\subsection{Stroking, Filling, and Clipping System Commands}

\begin{command}{\pgfsys@stroke}
  Stroke the current path (as if it were drawn with a pen). A number
  of graphic state parameters influence this, which can be
  set using appropriate system commands described later.

  \begin{description}
  \item[Line width]
    The ``thickness'' of the line. A width of 0 is the thinnest width
    renderable on the device. On a high-resolution printer this may
    become invisible and should be avoided. A good choice is 0.4pt,
    which is the default.

  \item[Stroke color]
    This special color is used for stroking. If it is not set, the
    current color is used.
 
  \item[Cap]
    The cap describes how the endings of lines are drawn. A round cap
    adds a little half circle to these endings. A butt cap ends the
    lines exactly at the end (or start) point without anything
    added. A rectangular cap ends the lines like the butt cap, but the
    lines protrude over the endpoint by the line thickness. (See also
    the \pdf\ manual.) If the path has been closed, no cap
    is drawn.
 
  \item[Join]
    This describes how a bend (a join) in a path is rendered. A round
    join draws bends using small arcs. A bevel join just draws the two
    lines and then fills the join minimally so that it becomes
    convex. A miter join extends the lines so that they form a single
    sharp corner, but only up to a certain miter limit. (See the \pdf\
    manual once more.)
 
  \item[Dash]
    The line may be dashed according to a dashing pattern.
 
  \item[Clipping area]
    If a clipping area is established, only those parts of the path
    that are inside the clipping area will be drawn.
  \end{description}
  
  In addition to stroking a path, the path may also be used for
  clipping after it has been stroked. This will happen if the
  |\pgfsys@clipnext| is used prior to this command, see there for
  details.

  This command is protocoled, see Section~\ref{section-protocols}.
\end{command}


\begin{command}{\pgfsys@closestroke}
  This command should have the same effect as first closing the path
  and then stroking it.

  This command has a default implementation and need not be
  implemented by a driver file.

  This command is protocoled, see Section~\ref{section-protocols}.
\end{command}


\begin{command}{\pgfsys@fill}
  This command fills the area surrounded by the current path. If the
  path has not yet been closed, it is closed prior to filling. The
  path itself is not stroked. For self-intersecting paths or paths
  consisting of multiple parts, the nonzero winding number rule is
  used to determine whether a point is inside or outside the
  path, except if |\ifpgfsys@eorule| holds -- in which case the
  even-odd rule should be used. (See the \pdf\ or PostScript manual
  for details.)  
 
  The following graphic state parameters influence the filling:
 
  \begin{description}
  \item[Interior rule]
    If |\ifpgfsys@eorule| is set, the even-odd rule is used, otherwise
    the non-zero winding number rule.
 
  \item[Fill color]
    If the fill color is not especially set, the current color is
    used. 
 
  \item[Clipping area]
    If a clipping area is established, only those parts of the filling
    area that are inside the clipping area will be drawn.
  \end{description}

  In addition to filling the path, the path will also be used for
  clipping if |\pgfsys@clipnext| is used prior to this command.

  This command is protocoled, see Section~\ref{section-protocols}.
\end{command}

\begin{command}{\pgfsys@fillstroke}
  First, the path is filled, then the path is stroked. If the fill and
  stroke colors are the same (or if they are not specified and the
  current color is used), this yields almost the same as a
  |\pgfsys@fill|. However, due to the line thickness of the stroked
  path, the fill-stroked area will be slightly larger.

  In addition to stroking and filling the path, the path will also be
  used for clipping if |\pgfsys@clipnext| is used prior to this command.

  This command is protocoled, see Section~\ref{section-protocols}.
\end{command}


\begin{command}{\pgfsys@discardpath}
 Normally, this command should ``throw away'' the current path.
 However, after |\pgfsys@clipnext| has been called, the current path
 should subsequently be used for clipping. See |\pgfsys@clipnext| for 
 details. 

  This command is protocoled, see Section~\ref{section-protocols}.
\end{command}


\begin{command}{\pgfsys@clipnext}
  This command should be issued after a path has been constructed, but
  before it has been stroked and/or filled or discarded. When the
  command is used, the next stroking/filling/discarding command will
  first be executed normally. Then, afterwards, the just-used path
  will be used for subsequent clipping. If there has already been a
  clipping region, this region is intersected with the new clipping
  path (the clipping cannot get bigger). The nonzero winding number
  rule is used to determine whether a point is inside or outside the
  clipping area or the even-odd rule, depending on whether
  |\ifpgfsys@eorule| holds.
\end{command}




\subsection{Graphic State Option System Commands}

\begin{command}{\pgfsys@setlinewidth\marg{width}}
  Sets the width of lines, when stroked, to \meta{width}, which must
  be a \TeX\ dimension.

  This command is protocoled, see Section~\ref{section-protocols}.
\end{command}

\begin{command}{\pgfsys@buttcap}
  Sets the cap to a butt cap. See |\pgfsys@stroke|.

  This command is protocoled, see Section~\ref{section-protocols}.
\end{command}

\begin{command}{\pgfsys@roundcap}
  Sets the cap to a round cap. See |\pgfsys@stroke|.

  This command is protocoled, see Section~\ref{section-protocols}.
\end{command}

\begin{command}{\pgfsys@rectcap}
  Sets the cap to a rectangular cap. See |\pgfsys@stroke|.

  This command is protocoled, see Section~\ref{section-protocols}.
\end{command}

\begin{command}{\pgfsys@miterjoin}
  Sets the join to a miter join. See |\pgfsys@stroke|.

  This command is protocoled, see Section~\ref{section-protocols}.
\end{command}

\begin{command}{\pgfsys@setmiterlimit\marg{factor}}
  Sets the miter limit of lines to \meta{factor}. See
  the \pdf\ or PostScript for details on what the miter limit is.

  This command is protocoled, see Section~\ref{section-protocols}.
\end{command}

\begin{command}{\pgfsys@roundjoin}
  Sets the join to a round join. See |\pgfsys@stroke|.

  This command is protocoled, see Section~\ref{section-protocols}.
\end{command}

\begin{command}{\pgfsys@beveljoin}
  Sets the join to a bevel join. See |\pgfsys@stroke|.

  This command is protocoled, see Section~\ref{section-protocols}.
\end{command}

\begin{command}{\pgfsys@setdash\marg{pattern}\marg{phase}}
  Sets the dashing patter. \meta{pattern} should be a list of \TeX\
  dimensions lengths separated by commas. \meta{phase} should be a
  single dimension.

  \example |\pgfsys@setdash{3pt,3pt}{0pt}|
 
  The list of values in \meta{pattern} is used to determine the
  lengths of the ``on'' phases of the dashing and of the ``off''
  phases. For example, if \meta{pattern} is |3bp,4bp|, then the dashing
  pattern is ``3bp on followed by 4bp off, followed by 3bp on,
  followed by 4bp off, and so on.'' A pattern of |.5pt,4pt,3pt,1.5pt| means
  ``.5pt on, 4pt off, 3pt on, 1.5pt off, .5pt on, \dots'' If the
  number of entries is odd, the last one is used twice, so |3pt| means
  ``3pt on, 3pt off, 3pt on, 3pt off, \dots'' An empty list 
  means  ``always on.''
 
  The second argument determines the ``phase'' of the pattern. For
  example, for a pattern of |3bp,4bp| and a phase of |1bp|, the pattern
  would start: ``2bp on, 4bp off, 3bp on, 4bp off, 3bp on, 4bp off,
  \dots''

  This command is protocoled, see Section~\ref{section-protocols}.
\end{command}

{\let\ifpgfsys@eorule=\relax
\begin{command}{\ifpgfsys@eorule}
  Determines whether the even odd rule is used for filling and
  clipping or not.
\end{command}
}

\begin{command}{\pgfsys@stroke@opacity\marg{value}}
  Sets the opacity of stroking operations.
\end{command}

\begin{command}{\pgfsys@fill@opacity\marg{value}}
  Sets the opacity of filling operations.
\end{command}


\subsection{Color System Commands}

The \pgfname\ system layer provides a number of system commands for
setting colors. These command coexist with commands from the |color|
and |xcolor| package, which perform similar functions. However, the
|color| package does not support having two different colors for
stroking and filling, which is a useful feature that is supported by
\pgfname. For this reason, the \pgfname\ system layer offers commands for
setting these colors separatedly. Also, plain \TeX\ profits from the
fact that \pgfname\ can set colors.

For \pdf, implementing these color commands is easy since \pdf\ 
supports different stroking and filling colors directly. For
PostScript, a more complicated approach is needed in which the colors
need to be stored in special PostScript variables that are set
whenever a stroking or a filling operation is done.

\begin{command}{\pgfsys@color@rgb\marg{red}\marg{green}\marg{blue}}
  Sets the color used for stroking and filling operations to the given 
  red/green/blue tuple (numbers between 0 and 1).

  This command is protocoled, see Section~\ref{section-protocols}.
\end{command}

\begin{command}{\pgfsys@color@rgb@stroke\marg{red}\marg{green}\marg{blue}}
  Sets the color used for stroking operations to the given
  red/green/blue tuple (numbers between 0 and 1). 

  \example Make stroked text dark red: |\pgfsys@color@rgb@stroke{0.5}{0}{0}|

  The special stroking color is only used if the stroking color has
  been set since the last |\color| or |\pgfsys@color@xxx|
  command. Thus, each |\color| command will reset both the stroking
  and filling colors by calling |\pgfsys@color@reset|. 

  This command is protocoled, see Section~\ref{section-protocols}.
\end{command}

\begin{command}{\pgfsys@color@rgb@fill\marg{red}\marg{green}\marg{blue}}
  Sets the color used for filling operations to the given
  red/green/blue tuple (numbers between 0 and 1). This color may be
  different from the stroking color.

  This command is protocoled, see Section~\ref{section-protocols}.
\end{command}

\begin{command}{\pgfsys@color@cmyk\marg{cyan}\marg{magenta}\marg{yellow}\marg{black}}
  Sets the color used for stroking and filling operations to the given
  cymk tuple (numbers between 0 and 1). 

  This command is protocoled, see Section~\ref{section-protocols}.
\end{command}

\begin{command}{\pgfsys@color@cmyk@stroke\marg{cyan}\marg{magenta}\marg{yellow}\marg{black}}
  Sets the color used for stroking operations to the given cymk tuple
  (numbers between 0 and 1). 

  This command is protocoled, see Section~\ref{section-protocols}.
\end{command}

\begin{command}{\pgfsys@color@cmyk@fill\marg{cyan}\marg{magenta}\marg{yellow}\marg{black}}
  Sets the color used for filling operations to the given cymk tuple
  (numbers between 0 and 1). 

  This command is protocoled, see Section~\ref{section-protocols}.
\end{command}

\begin{command}{\pgfsys@color@cmy\marg{cyan}\marg{magenta}\marg{yellow}}
  Sets the color used for stroking and filling operations to the given
  cym tuple (numbers between 0 and 1). 

  This command is protocoled, see Section~\ref{section-protocols}.
\end{command}

\begin{command}{\pgfsys@color@cmy@stroke\marg{cyan}\marg{magenta}\marg{yellow}}
  Sets the color used for stroking operations to the given cym tuple
  (numbers between 0 and 1). 

  This command is protocoled, see Section~\ref{section-protocols}.
\end{command}

\begin{command}{\pgfsys@color@cmy@fill\marg{cyan}\marg{magenta}\marg{yellow}}
  Sets the color used for filling operations to the given cym tuple
  (numbers between 0 and 1). 

  This command is protocoled, see Section~\ref{section-protocols}.
\end{command}

\begin{command}{\pgfsys@color@gray\marg{black}}
  Sets the color used for stroking and filling operations to the given
  black value, where 0 means black and 1 means white.

  This command is protocoled, see Section~\ref{section-protocols}.
\end{command}

\begin{command}{\pgfsys@color@gray@stroke\marg{black}}
  Sets the color used for stroking operations to the given black value,
  where 0 means black and 1 means white.

  This command is protocoled, see Section~\ref{section-protocols}.
\end{command}

\begin{command}{\pgfsys@color@gray@fill\marg{black}}
  Sets the color used for filling operations to the given black value,
  where 0 means black and 1 means white.

  This command is protocoled, see Section~\ref{section-protocols}.
\end{command}

\begin{command}{\pgfsys@color@reset}
  This command will be called when the |\color| command is used. It
  should purge any internal settings of stroking and filling
  color. After this call, till the next use of a command like
  |\pgfsys@color@rgb@fill|, the current color installed by the
  |\color| command should be used.

  If the \TeX-if |\pgfsys@color@reset@inorder| is set to true, this
  command may ``assume'' that any call to a color command that sets
  the fill or stroke color came ``before'' the call to this command
  and may try to optimize the output accordingly.

  An example of an incorrect ``out of order'' call would be using
  |\pgfsys@color@reset| at the beginning of a box that is constructed
  using |\setbox|. Then, when the box is constructed, no special fill
  or stroke color might be in force. However, when the box is later on
  inserted at some point, a special fill color might already have been
  set. In this case, this command is not guaranteed to reset the color
  correctly. 
\end{command}

\begin{command}{\pgfsys@color@reset@inordertrue}
  Sets the optimized ``in order'' version of the color resetting. This
  is the default.
\end{command}

\begin{command}{\pgfsys@color@reset@inorderfalse}
  Switches off the optimized color resetting. 
\end{command}

\begin{command}{\pgfsys@color@unstacked\marg{\LaTeX\ color}}
  This slightly obscure command causes the color stack to be
  tricked. When called, this command should set the current color to
  \meta{\LaTeX\ color} without causing any change in the color stack.

  \example |\pgfsys@color@unstacked{red}|
\end{command}


\subsection{Scoping System Commands}

The scoping commands are used to keep changes of the graphics state
local.

\begin{command}{\pgfsys@beginscope}
  Saves the current graphic state on a graphic state stack. All
  changes to the graphic state parameters mentioned for |\pgfsys@stroke|
  and |\pgfsys@fill| will be local to the current graphic state and 
  the old values will be restored after |\pgfsys@endscope| is used.
 
  \emph{Warning:} \pdf\ and PostScript differ with respect to the
  question of whether the current path is part of the graphic state or
  not. For this reason, you should never use this command unless the
  path is currently empty. For example, it might be a good idea to use 
  |\pgfsys@discardpath| prior to calling this command. 

  This command is protocoled, see Section~\ref{section-protocols}.
\end{command}

\begin{command}{\pgfsys@endscope}
  Restores the last saved graphic state.

  This command is protocoled, see Section~\ref{section-protocols}.
\end{command}







\subsection{Image System Commands}

The system layer provides some commands for image inclusion.

\begin{command}{\pgfsys@imagesuffixlist}
  This macro should expand to a list of suffixes, separated by `:',
  that will be tried when searching for an image.

  \example |\def\pgfsys@imagesuffixlist{eps:epsi:ps}|
\end{command}


\begin{command}{\pgfsys@defineimage}
  Called, when an image should be defined. 
 
  This command does not take any parameters. Instead, certain macros
  will be preinstalled with appropriate values when this command is
  invoked. These are:
 
  \begin{itemize}
  \item\declare{|\pgf@filename|}
    File name of the image to be defined.

  \item\declare{|\pgf@imagewidth|}
    Will be set to the desired (scaled) width of the image.

  \item\declare{|\pgf@imageheight|}
    Will be set to the desired (scaled) height of the image.
 
    If this macro and also the height macro are empty, the image
    should have its ``natural'' size.
 
    If exactly only of them is specified, the undefined value the
    image is scaled so that the aspect ratio is kept.
 
    If both are set, the image is scaled in both directions
    independently, possibly changing the aspect ratio.
  \end{itemize}
 
  The following macros presumable mostly make sense for drivers that
  can handle \pdf: 

  \begin{itemize}
  \item \declare{|\pgf@imagepage|}
    The desired page number to be extracted from a multi-page
    ``image.''

  \item\declare{|\pgf@imagemask|}
    If set, it will be set to |/SMask x 0 R| where |x| is the \pdf\ 
    object number of a soft mask to be applied to the image.

  \item\declare{|\pgf@imageinterpolate|}
    If set, it will be set to |/Interpolate true| or
    |/Interpolate false|, indicating whether the image should be
    interpolated in \pdf. 
  \end{itemize}
 
  The command should now setup the macro |\pgf@image| such that calling
  this macro will result in typesetting the image. Thus, |\pgf@image| is
  the ``return value'' of the command.

  This command has a default implementation and need not be
  implemented by a driver file.
\end{command}


\begin{command}{\pgfsys@definemask}
  This command declares a mask for usage with images. It works similar
  to |\pgfsys@defineimage|: Certain macros are set when the command is
  called. The result should be to set the macro |\pgf@mask| to a pdf
  object count that can subsequently be used as a soft mask. The
  following macros will be set when this command is invoked:
 
  \begin{itemize}
  \item \declare{|\pgf@filename|}
    File name of the mask to be defined.

  \item \declare{|\pgf@maskmatte|}
    The so-called matte of the mask (see the \pdf\ documentation for
    details). The matte is a color specification consisting of 1, 3 or
    4 numbers between 0 and 1. The number of numbers depends on the
    number of color channels in the image (not in the mask!). It will
    be assumed that the image has been preblended with this color.
  \end{itemize}
\end{command}


\subsection{Shading System Commands}


\begin{command}{\pgfsys@horishading\marg{name}\marg{height}\marg{specification}}
  Declares a horizontal shading for later use. The effect of this
  command should be the definition of a macro called |\@pgfshading|\meta{name}|!|
  (or |\csname @pdfshading|\meta{name}|!\endcsname|, to be
  precise). When invoked, this new macro should insert a shading at
  the current position. 
 
  \meta{name} is the name of the shading, which is also used in the
  output macro name. \meta{height} is the height of the shading and
  must be given as a TeX dimension like |2cm| or
  |10pt|. \meta{specification} is a shading color 
  specification as specified in Section~\ref{section-shadings}. The
  shading specification implicitly fixes the width of the shading. 
 
  When |\@pgfshading|\meta{name}|!| is invoked, it should insert a box
  of height \meta{height} and the width implicit in the shading
  declaration. 
\end{command}


\begin{command}{\pgfsys@vertshading\marg{name}\marg{width}\marg{specification}}
  Like the horizontal version, only for vertical shadings. This time,
  the height of the shading is implicit in \meta{specification} and
  the width is given as \meta{width}.
\end{command}

\begin{command}{\pgfsys@radialshading\marg{name}\marg{starting point}\marg{specification}}
  Declares a radial shading. Like the previous macros, this command
  should setup the macro |\@pgfshading|\meta{name}|!|, which upon
  invocation should insert a radial shading whose size is implicit in
  \meta{specification}.

  The parameter \meta{starting point} is a \pgfname\ point
  specifying the inner starting point of the shading.
\end{command}


\subsection{Reusable Objects System Commands}

\begin{command}{\pgfsys@invoke\marg{literals}}
  This command gets protocoled literals and should insert them into
  the |.pdf| or |.dvi| file using an appropriate |\special|.
\end{command}

\begin{command}{\pgfsys@defobject\marg{name}\marg{lower
      left}\marg{upper right}\marg{code}}
  Declares an object for later use. The idea is that the object can be
  precached in some way and then be rendered more quickly when used
  several times. For example, an arrow head might be defined and
  prerendered in this way.
 
  The parameter \meta{name} is the name for later use. \meta{lower
  left} and \meta{upper right} are \pgfname\ points specifying a bounding
  box for the object. \meta{code} is the code for the object. The code
  should not be too fancy.

  This command has a default implementation and need not be
  implemented by a driver file.
\end{command}

\begin{command}{\pgfsys@useobject\marg{name}\marg{extra code}}
  Renders a previously declared object. The first parameter is the
  name of the the object. The second parameter is extra code that
  should be executed right \emph{before} the object is
  rendered. Typically, this will be some transformation code.

  This command has a default implementation and need not be
  implemented by a driver file.
\end{command}


\subsection{Invisibility System Commands}

All drawing or stroking or text rendering between calls of the
following commands should be suppressed. A similar effect can be
achieved by clipping against an empty region, but the following
commands do not open a graphics scope and can be opened and closed
``orthogonally'' to other scopes.

\begin{command}{\pgfsys@begininvisible}
  Between this command and the closing |\pgfsys@endinvisible| all
  output should be suppressed. Nothing should be drawn at all, which
  includes all paths, images and shadings. However, no groups (neither
  \TeX\ groups nor graphic state groups) should be opened by this
  command. 

  This command has a default implementation and need not be
  implemented by a driver file.

  This command is protocoled, see Section~\ref{section-protocols}.
\end{command}
  
\begin{command}{\pgfsys@endinvisible}
  Ends the invisibility section, unless invisibility blocks have been
  nested. In this case, only the ``last'' one restores visibility.

  This command has a default implementation and need not be
  implemented by a driver file.

  This command is protocoled, see Section~\ref{section-protocols}.
\end{command}




\subsection{Internal Conversion Commands}

The system commands take \TeX\ dimensions as input, but the dimensions
that have to be inserted into \pdf\ and PostScript files need to be
dimensionless values that are interpreted as multiples of
$\frac{1}{72}\mathrm{in}$. For example, the \TeX\ dimension $2bp$
should be inserted as |2| into a \pdf\ file and the \TeX\ dimension
$10\mathrm{pt}$ as |9.9626401|. To make this conversion easier, the following
command may be useful:

\begin{command}{\pgf@sys@bp\marg{dimension}}
  Inserts how many multiples of $\frac{1}{72}\mathrm{in}$ the
  \meta{dimension} is into the current protocol stream (buffered).

  \example |\pgf@sys@bp{\pgf@x}| or |\pgf@sys@bp{1cm}|.
\end{command}

Note that this command is \emph{not} a system command that can/needs
to be overwritten by a driver. 

%%% Local Variables: 
%%% mode: latex
%%% TeX-master: "pgfmanual"
%%% End: 

% Copyright 2003 by Till Tantau <tantau@cs.tu-berlin.de>.
%
% This program can be redistributed and/or modified under the terms
% of the LaTeX Project Public License Distributed from CTAN
% archives in directory macros/latex/base/lppl.txt.

\section{The Soft Path Subsystem}

\label{section-soft-paths}

\makeatletter


This section describes a set of commands for creating \emph{soft
  paths} as opposed to the commands of the previous section, which
created \emph{hard paths}. A soft path is a path that can still be
``changed'' or ``molded.'' Once you (or the \pgfname\ system) is
satisfied with a soft path, it is turned into a hard path, which can
be inserted into the resulting |.pdf| or |.ps| file.

Note that the commands described in this section are ``high-level'' in
the sense that they are not implemented in driver files, but rather
directly by the \pgfname-system layer. For this reason, the commands for
creating soft paths do not start with |\pgfsys@|, but rather with
|\pgfsyssoftpath@|. On the other hand, as a user you will never use
these commands directly, so they are described as part of the
low-level interface. 



\subsection{Path Creation Process}

When the user writes a command like |\draw (0bp,0bp) -- (10bp,0bp);|
quite a lot happens behind the scenes:
\begin{enumerate}
\item
  The frontend command is translated by \tikzname\ into commands
  of the basic layer. In essence, the command is translated to
  something like
\begin{codeexample}[code only]
\pgfpathmoveto{\pgfpoint{0bp}{0bp}}
\pgfpathlineto{\pgfpoint{10bp}{0bp}}
\pgfusepath{stroke}
\end{codeexample}
\item
  The |\pgfpathxxxx| command do \emph{not} directly call ``hard''
  commands like |\pgfsys@xxxx|. Instead, the command |\pgfpathmoveto|
  invokes a special command called |\pgfsyssoftpath@moveto| and
  |\pgfpathlineto| invokes |\pgfsyssoftpath@lineto|. 

  The |\pgfsyssoftpath@xxxx| commands, which are described below,
  construct a soft path. Each time such a command is used, special
  tokens are added to the end of an internal macro that stores the
  soft path currently being constructed. 
\item
  When the |\pgfusepath| is encountered, the soft path stored in
  the internal macro is ``invoked.'' Only now does a special macro
  iterate over the soft path. For each line-to or move-to
  operation on this path it calls an appropriate |\pgfsys@moveto| or
  |\pgfsys@lineto| in order to, finally, create the desired hard path,
  namely, the string of literals in the |.pdf| or |.ps| file.
\item
  After the path has been invoked, |\pgfsys@stroke| is called to
  insert the literal for stroking the path.
\end{enumerate}

Why such a complicated process? Why not have |\pgfpathlineto| directly
call |\pgfsys@lineto| and be done with it? There are two reasons:
\begin{enumerate}
\item
  The \pdf\ specification requires that a path is not interrupted by
  any non-path-construction commands. Thus, the following code will
  result in a corrupted |.pdf|:
\begin{codeexample}[code only]
\pgfsys@moveto{0}{0}
\pgfsys@setlinewidth{1}
\pgfsys@lineto{10}{0}
\pgfsys@stroke
\end{codeexample}
  Such corrupt code is \emph{tolerated} by most viewers, but not
  always. It is much better to create only (reasonably) legal code.
\item
  A soft path can still be changed, while a hard path is fixed. For
  example, one can still change the starting and end points of a soft
  path or do optimizations on it. Such transformations are not possible
  on hard paths.
\end{enumerate}


\subsection{Starting and Ending a Soft Path}

No special action must be taken in order to start the creation of a
soft path. Rather, each time a command like |\pgfsyssoftpath@lineto|
is called, a special token is added to the (global) current soft path
being constructed.

However, you can access and change the current soft path. In this way,
it is possible to store a soft path, to manipulate it, or to invoke
it.

\begin{command}{\pgfsyssoftpath@getcurrentpath\marg{macro name}}
  This command will store the current soft path in \meta{macro name}.
\end{command}

\begin{command}{\pgfsyssoftpath@setcurrentpath\marg{macro name}}
  This command will set the current soft path to be the path stored in
  \meta{macro name}. This macro should store a path that has
  previously been extracted using the |\pgfsyssoftpath@getcurrentpath|
  command and has possibly been modified subsequently.
\end{command}

\begin{command}{\pgfsyssoftpath@invokecurrentpath}
  This command will turn the current soft path in a ``hard'' path. To
  do so, it iterates over the soft path and calls an appropriate
  |\pgfsys@xxxx| command for each element of the path. Note that the
  current soft path is \emph{not changed} by this command. Thus, in
  order to start a new soft path after the old one has been invoked
  and is no longer needed, you need to set the current soft path to be
  empty. This may seems strange, but it is often useful to immediately
  use the last soft path again.
\end{command}

\begin{command}{\pgfsyssoftpath@flushcurrentpath}
  This command will invoke the current soft path and then set it to be
  empty. 
\end{command}



\subsection{Soft Path Creation Commands}

\begin{command}{\pgfsyssoftpath@moveto\marg{x}\marg{y}}
  This command appends a ``move-to'' segment to the current soft
  path. The coordinates \meta{x} and \meta{y} are given as normal
  \TeX\ dimensions.

  \example One way to draw a line:
\begin{codeexample}[code only]
\pgfsyssoftpath@moveto{0pt}{0pt}
\pgfsyssoftpath@lineto{10pt}{10pt}
\pgfsyssoftpath@flushcurrentpath
\pgfsys@stroke
\end{codeexample}
\end{command}

\begin{command}{\pgfsyssoftpath@lineto\marg{x}\marg{y}}
  Appends a ``line-to'' segment to the current soft path. 
\end{command}

\begin{command}{\pgfsyssoftpath@curveto\marg{a}\marg{b}\marg{c}\marg{d}\marg{x}\marg{y}}
  Appends a ``curve-to'' segment to the current soft path with controls
  $(a,b)$ and $(c,d)$.
\end{command}

\begin{command}{\pgfsyssoftpath@rect\marg{lower left x}\marg{lower left y}\marg{width}\marg{height}}
  Appends a rectangle segment to the current soft path. 
\end{command}

\begin{command}{\pgfsyssoftpath@closepath}
  Appends a ``close-path'' segment to the current soft path. 
\end{command}




\subsection{The Soft Path Data Structure}

A soft path is stored in a standardized way, which makes it possible to
modify it before it becomes ``hard.'' Basically, a soft path is a long
sequence of triples. Each triple starts with a \emph{token} that
identifies what is going on. This token is followed by two dimensions in
braces. For example, the following is a soft path that means ``the
path starts at $(0\mathrm{bp}, 0\mathrm{bp})$ and then
continues in a straight line to $(10\mathrm{bp},
0\mathrm{bp})$.''

\begin{codeexample}[code only]
\pgfsyssoftpath@movetotoken{0bp}{0bp}\pgfsyssoftpath@linetotoken{10bp}{0bp}
\end{codeexample}

A curve-to is hard to express in this way since we need six numbers to
express it, not two. For this reasons, a curve-to is expressed using
three triples as follows: The command
\begin{codeexample}[code only]
\pgfsyssoftpath@curveto{1bp}{2bp}{3bp}{4bp}{5bp}{6bp}
\end{codeexample}
\noindent
results in the following three triples:
\begin{codeexample}[code only]
\pgfsyssoftpath@curvetosupportatoken{1bp}{2bp}
\pgfsyssoftpath@curvetosupportbtoken{3bp}{4bp}
\pgfsyssoftpath@curvetotoken{5bp}{6bp}
\end{codeexample}

These three triples must always ``remain together.'' Thus, a lonely
|supportbtoken| is forbidden.

In details, the following tokens exist:
\begin{itemize}
\item
  \declare{|\pgfsyssoftpath@movetotoken|} indicates a move-to
  operation. The two following numbers indicate the position to which
  the current point should be moved.
\item
  \declare{|\pgfsyssoftpath@linetotoken|} indicates a line-to
  operation. 
\item
  \declare{|\pgfsyssoftpath@curvetosupportatoken|} indicates the first
  control point of a curve-to operation. The triple must be followed
  by a |\pgfsyssoftpath@curvetosupportbtoken|.
\item
  \declare{|\pgfsyssoftpath@curvetosupportbtoken|} indicates the second
  control point of a curve-to operation. The triple must be followed
  by a |\pgfsyssoftpath@curvetotoken|.
\item
  \declare{|\pgfsyssoftpath@curvetotoken|} indicates the target
  of a curve-to operation.
\item
  \declare{|\pgfsyssoftpath@rectcornertoken|} indicates the corner of
  a rectangle on the soft path. The triple must be followed
  by a |\pgfsyssoftpath@rectsizetoken|.
\item
  \declare{|\pgfsyssoftpath@rectsizetoken|} indicates the size of
  a rectangle on the soft path.
\item
  \declare{|\pgfsyssoftpath@closepath|} indicates that the subpath
  begun with the last move-to operation should be closed. The parameter
  numbers are currently not important, but if set to anything
  different from |{0pt}{0pt}|, they should be set to the coordinate of
  the original move-to operation to which the path ``returns'' now.
\end{itemize}





% Copyright 2003 by Till Tantau <tantau@cs.tu-berlin.de>.
%
% This program can be redistributed and/or modified under the terms
% of the LaTeX Project Public License Distributed from CTAN
% archives in directory macros/latex/base/lppl.txt.

\section{The Protocol Subsystem}

\label{section-protocols}

\makeatletter

This section describes commands for \emph{protocolling} literal text
created by \pgfname. The idea is that some literal text, like the string
of commands used to draw an arrow head, will be used over and over
again in a picture. It is then much more efficient to compute the
necessary literal text just once and to quickly insert it ``in a
single sweep.''

When protocolling is ``switched on,'' there is a ``current protocol''
to which literal text gets appended. Once all commands that needed to
be protocoled have been issued, the protocol can be obtained and
stored using |\pgfsysprotocol@getcurrentprotocol|. At any point, the
current protocol can be changed using a corresponding setting
command. Finally, |\pgfsysprotocol@invokecurrentprotocol| is used to
insert the protocoled commands into the |.pdf| or |.dvi| file.

Only those |\pgfsys@| commands can be protocolled that use the
command |\pgfsysprotocol@literal| interally. For example, the
definition of |\pgfsys@moveto| in |pgfsys-common-pdf.def| is
\begin{codeexample}[code only]
\def\pgfsys@moveto#1#2{\pgfsysprotocol@literal{#1 #2 m}}
\end{codeexample}
All ``normal'' system-level commands can be protocolled. However,
commands for creating or invoking shadings, images, or whole pictures
require special |\special|'s and cannot be protocolled.

\begin{command}{\pgfsysprotocol@literalbuffered\marg{literal text}}
  Adds the \meta{literal text} to the current protocol, after it has
  been ``|\edef|ed.'' This command will always protocol.
\end{command}

\begin{command}{\pgfsysprotocol@literal\marg{literal text}}
  First calls |\pgfsysprotocol@literalbuffered| on \meta{literal
    text}. Then, if protocolling is currently switched off, the
  \meta{literal text} is passed on to |\pgfsys@invoke|.
\end{command}

\begin{command}{\pgfsysprotocol@bufferedtrue}
  Turns on protocolling. All subsequent calls of
  |\pgfsysprotocol@literal| will append their argument to the current
  protocol. 
\end{command}

\begin{command}{\pgfsysprotocol@bufferedfalse}
  Turns off protocolling. Subsequent calls of
  |\pgfsysprotocol@literal| directly insert their argument into the
  current |.pdf| or |.ps|.

  Note that if the current protocol is not empty when protocolling is
  switched off, the next call to |\pgfsysprotocol@literal| will first
  flush the current protocol, that is, insert it into the file.
\end{command}

\begin{command}{\pgfsysprotocol@getcurrentprotocol\marg{macro name}}
  Stores the current protocol in \meta{macro name} for later use.
\end{command}

\begin{command}{\pgfsysprotocol@setcurrentprotocol\marg{macro name}}
  Sets the current protocol to \meta{macro name}.
\end{command}

\begin{command}{\pgfsysprotocol@invokecurrentprotocol}
  Inserts the text stored in the current protocol into the |.pdf| or
  |.dvi| file. This does \emph{not} change the current protocol.
\end{command}

\begin{command}{\pgfsysprotocol@flushcurrentprotocol}
  First inserts the current protocol, then sets the current protocol
  to the empty string.
\end{command}


%%% Local Variables: 
%%% mode: latex
%%% TeX-master: "pgfmanual"
%%% End: 




\part{References and Index}

\vskip1cm
\begin{codeexample}[graphic=white]
\begin{tikzpicture}
  \draw[line width=0.3cm,color=red!30,line cap=round,line join=round] (0,0)--(2,0)--(2,5);
  \draw[help lines] (-2.5,-2.5) grid (5.5,7.5);
  \draw[very thick] (1,-1)--(-1,-1)--(-1,1)--(0,1)--(0,0)--
    (1,0)--(1,-1)--(3,-1)--(3,2)--(2,2)--(2,3)--(3,3)--
    (3,5)--(1,5)--(1,4)--(0,4)--(0,6)--(1,6)--(1,5)
    (3,3)--(4,3)--(4,5)--(3,5)--(3,6)
    (3,-1)--(4,-1);
  \draw[below left] (0,0) node(s){$s$};
  \draw[below left] (2,5) node(t){$t$};
  \fill (0,0) circle (0.06cm) (2,5) circle (0.06cm);
  \draw[->,rounded corners=0.2cm,shorten >=2pt]
    (1.5,0.5)-- ++(0,-1)-- ++(1,0)-- ++(0,2)-- ++(-1,0)-- ++(0,2)-- ++(1,0)--
    ++(0,1)-- ++(-1,0)-- ++(0,-1)-- ++(-2,0)-- ++(0,3)-- ++(2,0)-- ++(0,-1)--
    ++(1,0)-- ++(0,1)-- ++(1,0)-- ++(0,-1)-- ++(1,0)-- ++(0,-3)-- ++(-2,0)--
    ++(1,0)-- ++(0,-3)-- ++(1,0)-- ++(0,-1)-- ++(-6,0)-- ++(0,3)-- ++(2,0)--
    ++(0,-1)-- ++(1,0);
\end{tikzpicture}
\end{codeexample}

\printindex

\end{document}



%%% Local Variables: 
%%% mode: latex
%%% TeX-master: "~/texmf/tex/generic/pgf/doc/pgf/version-for-pdftex/en/pgfmanual"
%%% End: 
