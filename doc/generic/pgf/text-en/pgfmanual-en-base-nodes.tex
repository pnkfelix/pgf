% Copyright 2006 by Till Tantau
%
% This file may be distributed and/or modified
%
% 1. under the LaTeX Project Public License and/or
% 2. under the GNU Free Documentation License.
%
% See the file doc/generic/pgf/licenses/LICENSE for more details.


\section{Nodes and Shapes}

\label{section-shapes}

This section describes the |shapes| module.

\begin{pgfmodule}{shapes}
  This module defines commands both for creating nodes and for
  creating shapes. The package is loaded automatically by |pgf|, but
  you can load it manually if you have  only included |pgfcore|.  
\end{pgfmodule}


\subsection{Overview}

\pgfname\ comes with a sophisticated set of commands for creating
\emph{nodes} and \emph{shapes}. A \emph{node} is a graphical object
that consists (typically) of (one or more) text labels and some
additional stroked or filled paths. Each node has a certain
\emph{shape}, which may be something simple like a |rectangle| or a
|circle|, but it may also be something complicated like a
|uml class diagram| (this shape is currently not implemented,
though). Different nodes that have the same shape may look quite
different, however, since shapes (need not) specify whether the shape
path is stroked or filled.


\subsubsection{Creating and Referencing Nodes}

You create a node by calling the macro |\pgfnode| or the more general
|\pgfmultipartnode|. These macro takes several parameters and draws
the requested shape at a certain position. In addition, it will
``remember'' the node's position within the current
|{pgfpicture}|. You can then, later on, refer to the 
node's position. Coordinate transformations are ``fully supported,''
which means that if you used coordinate transformations to shift or
rotate the shape of a node, the node's position will still be correctly
determined by \pgfname. This is \emph{not} the case if you use canvas
transformations, instead.


\subsubsection{Anchors}

An important property of a node or a shape in general are its
\emph{anchors}. Anchors are ``important'' positions in a shape. For
example, the |center| anchor lies at the center of a shape, the
|north| anchor is usually ``at the top, in the middle'' of a shape,
the |text| anchor is the lower left corner of the shape's text label
(if present), and so on.

Anchors are important both when you create a node and when you
reference it. When you create a node, you specify the node's
``position'' by asking \pgfname\ to place the shape in such a way that
a certain anchor lies at a certain point. For example, you might ask
that the node is placed such that the |north| anchor is at the
origin. This will effectively cause the node to be placed below the
origin.

When you reference a node, you always reference an anchor of the
node. For example, when you request the ``|north| anchor of the node
just placed'' you will get the origin. However, you can also request
the ``|south| anchor of this node,'' which will give you a point
somewhere below the origin. When a coordinate transformation was in
force at the time of creation of a node, all anchors are also
transformed accordingly.

\subsubsection{Layers of a Shape}

The simplest shape, the |coordinate|, has just one anchor, namely the
|center|, and a label (which is usually empty). More complicated
shapes like the |rectangle| shape also have a \emph{background
  path}. This is a \pgfname-path that is defined by the shape. The
shape does not prescribe what should happen with the path: When a node
is created this path may be stroked (resulting in a frame around the
label), filled (resulting in a background color for the text), or just
discarded.

Although most shapes consist just of a background path plus some label
text, when a shape is drawn, up to seven different layers are drawn:

\begin{enumerate}
\item
  The ``behind the background layer.'' Unlike the background path,
  which be used in different ways by different nodes, the graphic
  commands given for this layer will always stroke or
  always fill the path they construct. They might also insert some
  text that is ``behind everything.''
\item
  The background path layer. How this path is used depends on how the
  arguments of the |\pgfnode| command.
\item
  The ``before the background path layer.'' This layer works like the
  first one, only the commands of this layer are executed after the
  background path has been used (in whatever way the creator of the
  node chose).
\item
  The label layer. This layer inserts the node's text box(es).
\item
  The ``behind the foreground layer.'' This layer, like the
  first layer, once more contains graphic commands that are ``simply
  executed.''
\item
  The foreground path layer. This path is treated in the same way as the
  background path, only it is drawn only after the label text has been
  drawn.
\item
  The ``before the foreground layer.''
\end{enumerate}

Which of these layers are actually used depends on the shape.



\subsubsection{Node Parts}

A shape typically does not consist only of different background and
foreground paths, but it may also have text labels. Indeed, for many
shapes the text labels are the more important part of the shape.

Most shapes will have only one text label. In this case, this text
label is simply passed as a parameter to the |\pgfnode| command. When
the node is drawn, the text label is shifted around such that its
lower left corner is at the |text| anchor of the node.

More complicated shapes may have more than one text label. Nodes of
such shapes are called \emph{multipart nodes}. The different
\emph{node parts} are simply the different text labels. For example, a
|uml class| shape might have a |class name| part, a |method| part and
an |attributes| part. Indeed, single part nodes are a special case of
multipart nodes: They only have one part named |text|.

When a shape is declared, you must specify the node parts. There is a
simple command called |\nodeparts| that takes a list of the part names
as input. When you create a node of a multipart shape, for each part
of the node you must have setup a \TeX-box containing the text of the
part. For a part named |XYZ| you must setup the box
|\pgfnodepartXYZbox|. The box will be placed at the anchor |XYZ|. See
the description of |\pgfmultipartnode| for more details.


\subsection{Creating Nodes}


\subsubsection{Creating Simple Nodes}

\begin{command}{\pgfnode\marg{shape}\marg{anchor}\marg{label
      text}\marg{name}\marg{path usage command}} 
  This command creates a new node. The \meta{shape} of the node must
  have been declared previously using |\pgfdeclareshape|.

  The shape is shifted such that the \meta{anchor} is at the
  origin. In order to place the shape somewhere else, use the
  coordinate transformation prior to calling this command.

  The \meta{name} is a name for later reference. If no name is given,
  nothing will be ``saved'' for the node, it will just be drawn.

  The \meta{path usage command} is executed for the background and the
  foreground path (if the shape defines them).

\begin{codeexample}[]
\begin{tikzpicture}
  \draw[help lines] (0,0) grid (4,3);
  {
    \pgftransformshift{\pgfpoint{1.5cm}{1cm}}
    \pgfnode{rectangle}{north}{Hello World}{hellonode}{\pgfusepath{stroke}}
  }
  {
    \color{red!20}
    \pgftransformrotate{10}
    \pgftransformshift{\pgfpoint{3cm}{1cm}}
    \pgfnode{rectangle}{center}
      {\color{black}Hello World}{hellonode}{\pgfusepath{fill}}
  }
\end{tikzpicture}
\end{codeexample}

  As can be seen, all coordinate transformations are also applied to
  the text of the shape. Sometimes, it is desirable that the
  transformations are applied to the point where the shape will be
  anchored, but you do not wish the shape itself to the
  transformed. In this case, you should call
  |\pgftransformresetnontranslations| prior to calling the |\pgfnode|
  command. 

\begin{codeexample}[]
\begin{tikzpicture}
  \draw[help lines] (0,0) grid (4,3);
  {
    \color{red!20}
    \pgftransformrotate{10}
    \pgftransformshift{\pgfpoint{3cm}{1cm}}
    \pgftransformresetnontranslations
    \pgfnode{rectangle}{center}
      {\color{black}Hello World}{hellonode}{\pgfusepath{fill}}
  }
\end{tikzpicture}
\end{codeexample}

  The \meta{label text} is typeset inside the \TeX-box
  |\pgfnodeparttextbox|. This box is shown at the |text| anchor of the
  node, if the node has a |text| part. See the description of
  |\pgfmultipartnode| for details.
\end{command}



\subsubsection{Creating Multi-Part Nodes}

\begin{command}{\pgfmultipartnode\marg{shape}\marg{anchor}\marg{name}\marg{path
      usage command}}
  This command is the more general (and less user-friendly) version of
  the |\pgfnode| command. While the |\pgfnode| command can only be
  used for shapes that have a single part (which is the case for most
  shapes), this command can also be used with multi-part nodes.

  When this command is called, for each node part of the node you must
  have setup one \TeX-box. Suppose the shape has two parts: The |text|
  part and the |lower| part. Then, prior to calling
  |\pgfmultipartnode|, you must have setup the boxes
  |\pgfnodeparttextbox| and |\pgfnodepartlowerbox|. These boxes may 
  contain any \TeX-text. The shape code will then compute the
  positions of the shape's anchors based on the sizes of the these
  shapes. Finally, when the node is drawn, the boxes are placed at the
  anchor positions |text| and |lower|.

\begin{codeexample}[]
\setbox\pgfnodeparttextbox=\hbox{$q_1$}
\setbox\pgfnodepartlowerbox=\hbox{01}
\begin{pgfpicture}
  \pgfmultipartnode{circle split}{center}{my state}{\pgfusepath{stroke}}
\end{pgfpicture}
\end{codeexample}

  \emph{Note:\/} Be careful when using the |\setbox| command inside a
  |{pgfpicture}| command. You will have to use |\pgfinterruptpath| at
  the beginning of the box and |\endpgfinterruptpath| at the end of
  the box to make sure that the box is typeset correctly. In the above
  example this problem was sidestepped by moving the box construction
  outside the environment.

  \emph{Note:\/} It is not necessary to use |\newbox| for every node
  part name. Although you need a different box for each part of a
  single shape, two different shapes may very well use the same box
  even when the names of the parts are different. Suppose you have a
  |circle split| shape that has an |lower| part and you have a
  |uml class| shape that has a |methods| part. Then, in order to avoid
  exhausting \TeX's limited number of box registers, you can say
\begin{codeexample}[code only]
\newbox\pgfnodepartlowerbox
\let\pgfnodepartmethodsbox=\pgfnodepartlowerbox  
\end{codeexample}
  Also, when you have a node part name with spaces like |class name|,
  it may be useful to create an alias:
\begin{codeexample}[code only]
\newbox\mybox
\expandafter\let\csname pgfnodepartclass namebox\endcsname=\mybox
\end{codeexample}
\end{command}

\begin{command}{\pgfcoordinate\marg{name}\marg{coordinate}}
  This command creates a node of shape |coordinate| at the given
  \meta{coordinate}. Exactly the same effect can be achieved using
  first a shift of the coordinate system to \meta{coordinate},
  followed by creating a node of shape |coordinate| named
  \meta{name}. However, this command is easier and more natural to use
  and, more importantly, it is much faster. 
\end{command}

\begin{command}{\pgfnodealias\marg{new name}\marg{existing node}}
  This command does not actually create a new node. Rather, it allows
  you to subsequently access the node \meta{existing node} using the
  name \meta{new name}.
\end{command}


\begin{command}{\pgfnoderename\marg{new name}\marg{existing node}}
  This command renames an existing node. 
\end{command}


There are a number of values that have an influence on the size of a
node. These values are stored in the following keys.

\begin{key}{/pgf/minimum width=\meta{dimension} (initially 1pt)}
  \keyalias{tikz}
  This key stores the \emph{recommended} minimum width of a
  shape. Thus, when a shape is drawn and when the shape's width would
  be smaller than \meta{dimension}, the shape's width is enlarged by
  adding some empty space. 

  Note that this value is just a recommendation. A shape may choose to
  ignore this key.
  
\begin{codeexample}[]
\begin{tikzpicture}
  \draw[help lines] (-2,0) grid (2,1);

  \pgfset{minimum width=3cm}
  \pgfnode{rectangle}{center}{Hello World}{}{\pgfusepath{stroke}}
\end{tikzpicture}
\end{codeexample}
\end{key}

\begin{key}{/pgf/minimum height=\meta{dimension} (initially 1pt)}
  \keyalias{tikz}
  Works like |/pgf/minimum width|.
\end{key}

\begin{key}{/pgf/minimum size=\meta{dimension}}
  \keyalias{tikz}
  This sytle both |/pgf/minimum width| and |/pgf/minimum height| to \meta{dimension}.
\end{key}


\begin{key}{/pgf/inner xsep=\meta{dimension} (initially 0.3333em)}
  \keyalias{tikz}
  This key stores the \emph{recommended} horizontal
  inner separation between the label text and the background path. As
  before, this value is just a recommendation and a shape may choose
  to ignore this key.
  
\begin{codeexample}[]
\begin{tikzpicture}
  \draw[help lines] (-2,0) grid (2,1);

  \pgfset{inner xsep=1cm}
  \pgfnode{rectangle}{center}{Hello World}{}{\pgfusepath{stroke}}
\end{tikzpicture}
\end{codeexample}
\end{key}

\begin{key}{/pgf/inner ysep=\meta{dimension} (initially 0.3333em)}
  \keyalias{tikz}
  Works like |/pgf/inner xsep|.
\end{key}

\begin{key}{/pgf/inner sep=\meta{dimension}}
  \keyalias{tikz}
  This style sets both |/pgf/inner xsep| and |/pgf/inner ysep| to \meta{dimension}.
\end{key}



\begin{key}{/pgf/outer xsep=\meta{dimension} (initially .5\string\pgflinewidth)}
  \keyalias{tikz}
  This key stores the recommended horizontal separation between the
  background path and the ``outer anchors.'' For example, if
  \meta{dimension} is |1cm| then the |east| anchor will be 1cm to the
  right of the right border of the background path. 
  As before, this value is just a recommendation.
  
\begin{codeexample}[]
\begin{tikzpicture}
  \draw[help lines] (-2,0) grid (2,1);

  \pgfset{outer xsep=.5cm}
  \pgfnode{rectangle}{center}{Hello World}{x}{\pgfusepath{stroke}}

  \pgfpathcircle{\pgfpointanchor{x}{north}}{2pt}
  \pgfpathcircle{\pgfpointanchor{x}{south}}{2pt}
  \pgfpathcircle{\pgfpointanchor{x}{east}}{2pt}
  \pgfpathcircle{\pgfpointanchor{x}{west}}{2pt}
  \pgfpathcircle{\pgfpointanchor{x}{north east}}{2pt}
  \pgfusepath{fill}
\end{tikzpicture}
\end{codeexample}
\end{key}

\begin{key}{/pgf/outer ysep=\meta{dimension} (initially .5\string\pgflinewidth)}
  \keyalias{tikz}
  Works like |/pgf/outer xsep|.
\end{key}

\begin{key}{/pgf/outer sep=\meta{dimension}}
  \keyalias{tikz}
  This style sets both |/pgf/outer xsep| and |/pgf/outer ysep| to \meta{dimension}.
\end{key}


\subsubsection{Deferred Node Positioning}

Normally, when a node is created using a command like |\pgfnode|, the
node is immediately inserted into the current picture. In particular,
you have no chance to change the position of a created node after it
has been create. Using |\pgfpositionnodelater| in concert with
|\pgfpositionnodenow|, you can create a node whose position is
determined only at some later time.

\begin{command}{\pgfpositionnodelater\marg{macro name}}
  This command is not a replacement for |\pgfnode|. Rather, when this
  command is used in a scope, all subsequent node creations in this
  scope will be affected in the following way:  When a node is
  created, it is not inserted into the current picture. Instead, it
  is stored in the box |\pgfpositionnodelaterbox|. Furthermore, the
  node is not relevant for the picture's bounding box, but a bounding
  box for the node is computed and stored in the macros
  |\pgfpositionnodelaterminx| to |\pgfpositionnodelatermaxy|. Then,
  the \meta{macro name} is called with the following macros setup:
  
  \begin{command}{\pgfpositionnodelaterbox}
    A box register number (|0| currently) that stores the node's paths
    and texts. You should move the contents of this box to a box of
    your choice inside \meta{macro name}.
  \end{command}

  \begin{command}{\pgfpositionnodelatername}
    The name of the just-created-node. This name will be the
    originally "desired" name of the box plus the fixed prefix
    |not yet positioned@|. The idea is to ensure that the original
    name is not inadvertently used before the node is actually
    positioned. When |\pgfpositionnodenow| is called, it will change
    the name to the original name.
  \end{command}
  
  \begin{command}{\pgfpositionnodelaterminx}
    The minimal $x$-position of a bounding box of the node. This
    bounding box refers to the node when it is positioned with the
    anchor at the origin.
  \end{command}
  \begin{command}{\pgfpositionnodelaterminy}
  \end{command}
  \begin{command}{\pgfpositionnodelatermaxx}
  \end{command}
  \begin{command}{\pgfpositionnodelatermaxy}
  \end{command}
  
  Once a late node has been created, you can add arbitrary code in the  
  same picture. Then, at some later point, you call
  |\pgfpositionnodenow| to finally position the node at a given
  position.  At this point, the above macros must have the exact same
  values they had when \meta{macro name} was called. Note that the
  above macros are local to a scope that ends right after the call to
  \meta{macro name}, so it is your job to copy the values to safety
  inside \meta{macro name}.
  
  By setting \meta{macro name} to |\relax| (which is the default), you
  can switch off the whole mechanism. When a picture is interrupted,
  this is done automatically.
\end{command}


\begin{command}{\pgfpositionnodenow\marg{coordinate}}
  This command is used to position a node that has previously been
  created with |\pgfpositionnodelater|. When |\pgfpositionnodenow| is
  called, the macros and boxes mentioned in the description of
  |\pgfpositionnodenow| must be set to the value they had when the
  \meta{macro name} was called. Provided this is the case, this
  command will insert the box into the current picture, shifted by
  \meta{coordinate}. Subsequently, you can refer to the node with its
  original name as if it had just been created.

  \begin{codeexample}[]
\newbox\mybox

\def\mysaver{
  \global\setbox\mybox=\box\pgfpositionnodelaterbox
  \global\let\myname=\pgfpositionnodelatername
  \global\let\myminx=\pgfpositionnodelaterminx
  \global\let\myminy=\pgfpositionnodelaterminy
  \global\let\mymaxx=\pgfpositionnodelatermaxx
  \global\let\mymaxy=\pgfpositionnodelatermaxy
}

\begin{tikzpicture}
  {
    \pgfpositionnodelater{\mysaver}
    \node [fill=blue!20,below,rotate=30] (hi) {Hello world};
  }
  \draw [help lines] (0,0) grid (3,2);
  
  \let\pgfpositionnodelatername=\myname
  \let\pgfpositionnodelaterminx=\myminx
  \let\pgfpositionnodelaterminy=\myminy
  \let\pgfpositionnodelatermaxx=\mymaxx
  \let\pgfpositionnodelatermaxy=\mymaxy
  \setbox\pgfpositionnodelaterbox=\box\mybox
  \pgfpositionnodenow{\pgfqpoint{2cm}{2cm}}
  
  \draw (hi) -- (0,0);
\end{tikzpicture}
  \end{codeexample}
\end{command}


\subsection{Using Anchors}

Each shape defines a set of anchors. We saw already that the anchors
are used when the shape is drawn: the shape is placed in such a way
that the given anchor is at the origin (which in turn is typically
translated somewhere else).

One has to look up the set of anchors of each shape, there is no
``default'' set of anchors, except for the |center| anchor, which
should always be present. Also, most shapes will declare anchors like
|north| or |east|, but this is not guaranteed.


\subsubsection{Referencing Anchors of Nodes in the Same Picture}

Once a node has been defined, you can refer to its anchors using the
following commands:

\begin{command}{\pgfpointanchor\marg{node}\marg{anchor}}
  This command is another ``point command'' like the commands
  described in Section~\ref{section-points}. It returns the coordinate
  of the given \meta{anchor} in the given \meta{node}. The command can
  be used in commands like |\pgfpathmoveto|.

\begin{codeexample}[]
\begin{pgfpicture}
  \pgftransformrotate{30}
  \pgfnode{rectangle}{center}{Hello World!}{x}{\pgfusepath{stroke}}

  \pgfpathcircle{\pgfpointanchor{x}{north}}{2pt}
  \pgfpathcircle{\pgfpointanchor{x}{south}}{2pt}
  \pgfpathcircle{\pgfpointanchor{x}{east}}{2pt}
  \pgfpathcircle{\pgfpointanchor{x}{west}}{2pt}
  \pgfpathcircle{\pgfpointanchor{x}{north east}}{2pt}
  \pgfusepath{fill}
\end{pgfpicture}
\end{codeexample}

  In the above example, you may have noticed something curious: The
  rotation transformation is still in force when the anchors are
  invoked, but it does not seem to have an effect. You might expect
  that the rotation should apply to the already rotated points once
  more.

  However, |\pgfpointanchor| returns a point that takes the current
  transformation matrix into account: \emph{The inverse transformation
    to the current coordinate transformation is applied to an anchor
    point before returning it.}

  This behavior may seem a bit strange, but you will find it very
  natural in most cases. If you really want to apply a transformation
  to an anchor point (for example, to ``shift it away'' a little bit),
  you have to invoke |\pgfpointanchor| without any transformations in
  force. Here is an example:

\makeatletter
\begin{codeexample}[]
\begin{pgfpicture}
  \pgftransformrotate{30}
  \pgfnode{rectangle}{center}{Hello World!}{x}{\pgfusepath{stroke}}

  {
    \pgftransformreset
    \pgfpointanchor{x}{east}
    \xdef\mycoordinate{\noexpand\pgfpoint{\the\pgf@x}{\the\pgf@y}}
  }
    
  \pgfpathcircle{\mycoordinate}{2pt}
  \pgfusepath{fill}
\end{pgfpicture}
\end{codeexample}

  A special situation arises when the \meta{node} lies in a picture
  different from the current picture. In this case, if you have not
  told \pgfname\ that the picture should be ``remembered,'' the
  \meta{node} will be treated as if it lied in the current
  picture. For example, if the \meta{node} was at position $(3,2)$ in
  the original picture, it is treated as if it lied at position
  $(3,2)$ in the current picture. However, if you have told \pgfname\
  to remember the picture position of the node's picture and also of
  the current picture,
  then |\pgfpointanchor| will return a coordinate that corresponds to
  the position of the node's anchor on the page, transformed into the
  current coordinate system. For examples and more details see
  Section~\ref{section-cross-pictures-pgf}. 
\end{command}

\begin{command}{\pgfpointshapeborder\marg{node}\marg{point}}
  This command returns the point on the border of the shape that lies
  on a straight line from the center of the node to \meta{point}. For
  complex shapes it is not guaranteed that this point will actually
  lie on the border, it may be on the border of a ``simplified''
  version of the shape.

\begin{codeexample}[]
\begin{pgfpicture}
  \begin{pgfscope}
    \pgftransformrotate{30}
    \pgfnode{rectangle}{center}{Hello World!}{x}{\pgfusepath{stroke}}
  \end{pgfscope}
  \pgfpathcircle{\pgfpointshapeborder{x}{\pgfpoint{2cm}{1cm}}}{2pt}
  \pgfpathcircle{\pgfpoint{2cm}{1cm}}{2pt}
  \pgfpathcircle{\pgfpointshapeborder{x}{\pgfpoint{-1cm}{1cm}}}{2pt}
  \pgfpathcircle{\pgfpoint{-1cm}{1cm}}{2pt}
  \pgfusepath{fill}
\end{pgfpicture}
\end{codeexample}
\end{command}


\subsubsection{Referencing Anchors of Nodes in Different Pictures}
\label{section-cross-pictures-pgf}

As a picture is typeset, \pgfname\ keeps track of the positions of all
nodes inside the picture. What \pgfname\ does not remember is the
position of the picture \emph{itself} on the page. Thus, if you define
a node in one picture and then try to reference this node while
another picture is typeset, \pgfname\ will only know the position of
the nodes that you try to typeset inside the original picture, but it
will not know where this picture lies. What is missing is the relative
positioning of the two pictures.

To overcome this problem, you need to tell \pgfname\ that it should
remember the position of pictures on a page. If these positions are
remembered, then \pgfname\ can compute the offset between the pictures
and make nodes in different pictures accessible.

Determining the positions of pictures on the page is, alas,
not-so-easy. Because of this, \pgfname\ does not do so
automatically. Rather, you have to proceed as follows:
\begin{enumerate}
\item You have to use a backend driver that supports position
  tracking. pdf\TeX\ is one such drivers, |dvips| currently is not.
\item You have to say |\pgfrememberpicturepositiononpagetrue|
  somewhere before or inside every picture
  \begin{itemize}
  \item in which you wish to reference a node and
  \item from which you wish to reference a node in another picture.
  \end{itemize}
  The second item is important since \pgfname\ does not only need to
  know the position of the picture in which the node you wish to
  reference lies, but it also needs to know where the current picture
  lies.
\item You typically have to run \TeX\ twice (depending on the backend
  driver) since the position information typically gets written into
  an external file on the first run and is available only on the
  second run.
\item You have to switch off automatic bounding bound
  computations. The reason is that the node in the other picture
  should not influence the size of the bouding box of the current
  picture. You should say |\pgfusepath{use as bounding box}| before
  using a coordinate in another picture.
\end{enumerate}



\subsection{Special Nodes}

There are several special nodes that are always defined and which you
should not attempt to redefine.

\begin{predefinednode}{current bounding box}
  This node is of shape |rectangle|. Unlike normal nodes, its size
  changes constantly and always reflects the size of the bounding box
  of the current picture. This means that, for instance, that
\begin{codeexample}[code only]
\pgfpointanchor{current bounding box}{south east}
\end{codeexample}
  returns the lower left corner of the bounding box of the current
  picture. 
\end{predefinednode}

\begin{predefinednode}{current path bounding box}
  This node is also of shape |rectangle|. Its size is the size of the
  bounding box of the current path.
\end{predefinednode}

\begin{predefinednode}{current page}
  This node is inside a virtual remembered picture. The size of this
  node is the size of the current page. This means that if you create
  a remembered picture and inside this picture you reference an anchor
  of this node, you reference an absolute position on the page. To
  demonstrate the effect, the following code puts some text in the
  lower left corner of the current page. Note that this works only if
  the backend driver supports it, otherwise the text is inserted right
  here.%
{%
\pgfrememberpicturepositiononpagetrue%
\begin{pgfpicture}
  \pgfusepath{use as bounding box}
  \pgftransformshift{\pgfpointanchor{current page}{south west}}
  \pgftransformshift{\pgfpoint{1cm}{1cm}}
  \pgftext[left,base]{
    \textcolor{red}{
      Text absolutely positioned in
      the lower left corner.}
  }
\end{pgfpicture}
}%
\begin{codeexample}[code only]
\pgfrememberpicturepositiononpagetrue
\begin{pgfpicture}
  \pgfusepath{use as bounding box}
  \pgftransformshift{\pgfpointanchor{current page}{south west}}
  \pgftransformshift{\pgfpoint{1cm}{1cm}}
  \pgftext[left,base]{
    \textcolor{red}{
      Text absolutely positioned in
      the lower left corner.}
  }
\end{pgfpicture}  
\end{codeexample}
\end{predefinednode}


There is also an option that allows you to create new special nodes
quite similar to the above:
\begin{key}{/pgf/local bounding box=\meta{node name}}
  \keyalias{tikz}
  This defines a new node \meta{node name} whose size is the bouding
  box around all objects in the current scope starting at the position
  where this option was given. After the end of the scope, the \meta{node
    name} is still available. You can use this option to keep track of
  the size of a certain area. Note that excessive use of this option
  (keeping track of dozens of bounding boxes at the same time)
  will slow things down.
\begin{codeexample}[]
\begin{tikzpicture}
  \draw [help lines] (0,0) grid (3,2);
  { [local bounding box=outer box]
    \draw (1,1) circle (.5) [local bounding box=inner box] (2,2) circle (.5);
  }
  \draw      (outer box.south west) rectangle (outer box.north east);
  \draw[red] (inner box.south west) rectangle (inner box.north east);
\end{tikzpicture}
\end{codeexample}
\end{key}


\subsection{Declaring New Shapes}

There are only three predefined shapes, see
Section~\ref{section-predefined-shapes}, so there must be some way of
defining new shapes. Defining a shape is, unfortunately, a
not-quite-trivial process. The reason is that shapes need to be both
very flexible (their size will vary greatly according to
circumstances) and they need to be constructed reasonably ``fast.''
\pgfname\ must be able to handle pictures with several hundreds of
nodes and documents with thousands of nodes in total. It would not do
if \pgfname\ had to compute and store, say, dozens of anchor positions
for every node. 


\subsubsection{What Must Be Defined For a Shape?}

In order to define a new shape, you must provide:
\begin{itemize}
\item
  a \emph{shape name},
\item
  code for computing the  \emph{saved anchors} and \emph{saved
    dimensions}, 
\item
  code for computing \emph{anchor} positions in terms of the saved anchors,
\item
  optionally code for the \emph{background path} and \emph{foreground path},
\item
  optionally code for \emph{things to be drawn before or behind} the
  background and foreground paths.
\item
  optionally a list of node parts.
\end{itemize}


\subsubsection{Normal Anchors Versus Saved Anchors}

Anchors  are special places in shape. For example, the |north east|
anchor, which is a normal anchor, lies at the upper right corner of
the  |rectangle| shape, as does |\northeast|, which is a saved
anchor. The difference is the following: \emph{saved anchors are 
  computed and stored for each node, anchors are only computed as
  needed.} The user only has access to the normal anchors, but a
normal anchor can just ``copy'' or ``pass through'' the location of a
saved anchor. 

The idea behind all this is that a shape can declare a very large
number of normal anchors, but when a node of this shape is created,
these anchors are not actually computed. However, this causes a
problem: When we wish to reference an anchor of a node at some later
time, we must still able to compute the position of the anchor. For 
this, we may need a lot of information: What was the transformation
matrix that was in force when the node was created? What was the size
of the text box? What were the values of the different separation
dimensions? And so on. 

To solve this problem, \pgfname\ will always compute the locations of
all \emph{saved anchors} and store these positions. Then, when an
normal anchor position is requested later on, the anchor position can
be given just from knowing where the locations of the saved anchors.

As an example, consider the |rectangle| shape. For this shape two
anchors are saved: The |\northeast| corner and the |\southwest|
corner. A normal anchor like |north west| can now easily be expressed
in terms of these coordinates: Take the $x$-position of the
|\southwest| point  and the $y$-position of the |\northeast| point. 
The |rectangle| shape currently defines 13 normal anchors, but needs
only two saved anchors. Adding new anchors like a  |south south east|
anchor would not increase the memory and computation requirements of
pictures. 

All anchors (both saved and normal) are specified in a local
\emph{shape coordinate space}. This is also true for the background
and foreground paths. The |\pgfnode| macro will automatically apply
appropriate transformations to the coordinates so that the shape is
shifted to the right anchor or otherwise transformed. 


\subsubsection{Command for Declaring New Shapes}

The following command declares a new shape:
\begin{command}{\pgfdeclareshape\marg{shape name}\marg{shape
      specification}}
  This command declares a new shape named \meta{shape name}. The shape
  name can later be used in commands like |\pgfnode|.

  The \meta{shape specification} is some \TeX\ code containing calls
  to special commands that are only defined inside the \meta{shape
    specification} (similarly to commands like |\draw| that are only
  available inside the |{tikzpicture}| environment).

  \example Here is the code of the |coordinate| shape:
\begin{codeexample}[code only]
\pgfdeclareshape{coordinate}
{
  \savedanchor\centerpoint{%
    \pgf@x=.5\wd\pgfnodeparttextbox%
    \pgf@y=.5\ht\pgfnodeparttextbox%
    \advance\pgf@y by -.5\dp\pgfnodeparttextbox%
  }
  \anchor{center}{\centerpoint}
  \anchorborder{\centerpoint}
}
\end{codeexample}

  The special commands are explained next. In the examples given for
  the special commands a new shape will be constructed, which we might
  call |simple rectangle|. It should behave like the normal rectangle
  shape, only without bothering about the fine details like inner and
  outer separations. The skeleton for the shape is the following.
\begin{codeexample}[code only]
\pgfdeclareshape{simple rectangle}{
  ...
}
\end{codeexample}

  \begin{command}{\nodeparts\marg{list of node parts}}
    This command declares which parts make up nodes of this shape. A
    \emph{node part} is a (possibly empty) text label that is drawn
    when a node of the shape is created.

    By default, a shape has just one node part called |text|. However,
    there can be several node parts. For example, the
    |circle split| shape has two parts: the |text| part, which
    shows that upper text, and a |lower| part, which shows the
    lower text. For the |circle split| shape the |\nodeparts| command
    was called with the argument |{text,lower}|.

    When a multipart node is created, the text labels are drawn in the
    sequences listed in the \meta{list of node parts}. For each node
    part there you must have declared one anchor and the \TeX-box of
    the part is placed at this anchor. For a node part called |XYZ|
    the \TeX-box |\pgfnodepartXYZbox| is placed at anchor |XYZ|.
  \end{command}

  \begin{command}{\savedanchor\marg{command}\marg{code}}
    This command declares a saved anchor. The argument \meta{command}
    should be a \TeX\ macro name like |\centerpoint|.

    The \meta{code} will be executed each time |\pgfnode| (or
    |\pgfmultipartnode|) is called to  create a node of the shape
    \meta{shape name}. When the \meta{code} 
    is executed, the \TeX-boxes of the node parts will contain the
    text labels of the node. Possibly, these box are void. For
    example, if there is just a |text| part, the node
    |\pgfnodeparttextbox| will be setup when the \meta{code} is
    executed. 

    The \meta{code} can use the width, height, and depth of the
    box(es) to compute the location of the saved anchor. In addition,
    the \meta{code} can take into account the values of dimensions like
    |\pgfshapeminwidth| or |\pgfshapeinnerxsep|. Furthermore, the
    \meta{code} can take into consideration the values of any further
    shape-specific variables that are set at the moment when
    |\pgfnode| is called.

    The net effect of the \meta{code} should be to set the two \TeX\
    dimensions |\pgf@x| and |\pgf@y|. One way to achieve this is to
    say |\pgfpoint{|\meta{x value}|}{|\meta{y value}|}| at the end of
    the \meta{code}, but you can also just set these variables.
    The values that |\pgf@x| and |\pgf@y| have after the code has been
    executed, let us call them $x$ and $y$, will be recorded and
    stored together with the node that is created by the command
    |\pgfnode|.

    The macro \meta{command} is defined to be
    |\pgfpoint{|$x$|}{|$y$|}|. However, the \meta{command} is only
    locally defined while anchor positions are being computed. Thus,
    it is possible to use very simple names for \meta{command}, like
    |\center| or |\a|, without causing a name-clash. (To be precise,
    very simple \meta{command} names will clash with existing names,
    but only locally inside the computation of anchor positions; and
    we do not need the normal |\center| command during these
    computations.)

    For our |simple rectangle| shape, we will need only one saved
    anchor: The upper right corner. The lower left corner could either
    be the origin or the ``mirrored'' upper right corner, depending on
    whether we want the text label to have its lower left corner at
    the origin or whether the text label should be centered on the
    origin. Either will be fine, for the final shape this will make no
    difference since the shape will be shifted anyway. So, let us
    assume that the text label is centered on the origin (this will be
    specified later on using the |text| anchor). We get 
    the following code for the upper right corner:
\begin{codeexample}[code only]
\savedanchor{\upperrightcorner}{
  \pgf@y=.5\ht\pgfnodeparttextbox % height of the box, ignoring the depth
  \pgf@x=.5\wd\pgfnodeparttextbox % width of the box
}
\end{codeexample}

    If we wanted to take, say, the |\pgfshapeminwidth| into account,
    we could use the following code:
    
\begin{codeexample}[code only]
\savedanchor{\upperrightcorner}{
  \pgf@y=.\ht\pgfnodeparttextbox % height of the box
  \pgf@x=.\wd\pgfnodeparttextbox % width of the box
  \setlength{\pgf@xa}{\pgfshapeminwidth}
  \ifdim\pgf@x<.5\pgf@xa
    \pgf@x=.5\pgf@xa
  \fi
}
\end{codeexample}
    Note that we could not have written |.5\pgfshapeminwidth| since
    the minium width is stored in a ``plain text macro,'' not as a
    real dimension. So if |\pgfshapeminwidth| depth were 
    2cm, writing |.5\pgfshapeminwidth| would yield the same as |.52cm|.

    In the ``real'' |rectangle| shape the code is somewhat more
    complex, but you get the basic idea.
  \end{command}  
  \begin{command}{\saveddimen\marg{command}\marg{code}}
    This command is similar to |\savedanchor|, only instead of setting
    \meta{command} to |\pgfpoint{|$x$|}{|$y$|}|, the \meta{command} is
    set just to (the value of) $x$.

    In the |simple rectangle| shape we might use a saved dimension to
    store the depth of the shape box.
  
\begin{codeexample}[code only]
\saveddimen{\depth}{
  \pgf@x=\dp\pgfnodeparttextbox 
}
\end{codeexample}
  \end{command}  
  \begin{command}{\savedmacro\marg{command}\marg{code}}
    This command is similar to |\saveddimen|, only at some point
    in \meta{code}, \meta{command} should be defined appropriately,
    (this could be a value, or some text).

    In the |regular polygon| shape, a saved macro is used to
    store the number of sides of the polygon.
  
\begin{codeexample}[code only]
\savedmacro{\sides}{\let\sides\pgfpolygonsides}
\end{codeexample}
  \end{command}  
  \begin{command}{\anchor\marg{name}\marg{code}}
    This command declares an anchor named \meta{name}. Unlike for saved
    anchors, the \meta{code} will not be executed each time a node is
    declared. Rather, the \meta{code} is only executed when the anchor
    is specifically requested; either for anchoring the node during
    its creation or as a  position in the shape referenced later on.

    The \meta{name} is a quite arbitrary string that is not ``passed
    down'' to the system level. Thus, names like |south| or |1| or
    |::| would all be fine.

    A saved anchor is not automatically also a normal anchor. If you
    wish to give the users access to a saved anchor you must declare a
    normal anchor that just returns the position of the saved anchor.

    When the \meta{code} is executed, all saved anchor macros will be
    defined. Thus, you can reference them in your \meta{code}. The
    effect of the \meta{code} should be to set the values of |\pgf@x|
    and |\pgf@y| to the coordinates of the anchor.

    Let us consider some example for the |simple rectangle|
    shape. First, we would like to make the upper right corner
    publicly available, for example as |north east|:
    
\begin{codeexample}[code only]
\anchor{north east}{\upperrightcorner}
\end{codeexample}

    The |\upperrightcorner| macro will set |\pgf@x| and |\pgf@y| to
    the coordinates of the upper right corner. Thus, |\pgf@x| and
    |\pgf@y| will have exactly the right values at the end of the
    anchor's code.

    Next, let us define a |north west| anchor. For this anchor, we can
    negate the |\pgf@x| variable:
   
\begin{codeexample}[code only]
\anchor{north west}{
  \upperrightcorner
  \pgf@x=-\pgf@x
}
\end{codeexample}

    Finally, it is a good idea to always define a |center| anchor,
    which will be the default location for a shape.

\begin{codeexample}[code only]
\anchor{center}{\pgfpointorigin}
\end{codeexample}

    You might wonder whether we should not take into consideration
    that the node is not placed at the origin, but has been shifted
    somewhere. However, the anchor positions are always specified in
    the shape's ``private'' coordinate system. The ``outer''
    transformation that has been applied to the shape upon its
    creation is applied automatically to the coordinates returned by
    the anchor's \meta{code}.

    Out |simple rectangle| only has one text label (node
    part) called |text|. This is the default situation, so we need not
    do anything. For the |text| node part we must setup a |text|
    anchor. This   anchor is used upon creation of a node to determine
    the lower left  corner of the text label (within the private
    coordinate system of the shape). By default, the |text| anchor is
    at the origin, but you may change this. For example, we would say
\begin{codeexample}[code only]
\anchor{text}{%
  \upperrightcorner%
  \pgf@x=-\pgf@x%
  \pgf@y=-\pgf@y%
}
\end{codeexample}
    to center the text label on the origin in the shape coordinate
    space. Note that we could \emph{not} have written the following:
    
\begin{codeexample}[code only]
\anchor{text}{\pgfpoint{-.5\wd\pgfnodeparttextbox}{-.5\ht\pgfnodeparttextbox}}
\end{codeexample}
    Do you see why this is wrong? The problem is that the box
    |\pgfnodeparttextbox| will most likely not have the correct size
    when the anchor is computed. After all, the anchor position might
    be recomputed at a time when several other nodes have been created. 

    If a shape has several node parts, we would have to define an
    anchor for each part.    
  \end{command}  

  \begin{command}{\deferedanchor\marg{name}\marg{code}}
    This command declares an anchor named \meta{name}. It works like
    |\anchor|. However, unlike for
    anchors declared by |\anchor|, \marg{name} will \emph{not} be
    expanded during the shape declaration (i.e.\ not during
    |\pgfdeclareshape|). Rather, the \marg{name} is expanded when the
    \emph{node} is actually used (with |\pgfnode| or more likely with
    |\node|). This may be useful if the anchor name is context
    dependent (depending, for example, on the value of a key).

\begin{codeexample}[code only]
\makeatletter
\def\foo{foo}
\pgfdeclareshape{simple shape}{%
  \savedanchor{\center}{%
    \pgfpointorigin}
  \anchor{center}{\center}
  \savedanchor{\anchorfoo}{%
    \pgf@x=1cm
    \pgf@y=0cm}
  \deferedanchor{anchor \foo}{\anchorfoo}}

\begin{tikzpicture}
  \node[simple shape] (Test1) at (0,0) {};
  \node at (Test1.anchor foo) {Test};

  \def\foo{bar}
  \node[simple shape] (Test2) at (4,4) {};
  \node at (Test2.anchor bar) {Test};
\end{tikzpicture}
\end{codeexample}
  \end{command}
  \begin{command}{\anchorborder\marg{code}}
    A \emph{border anchor} is an anchor point on the border of the
    shape. What exactly is considered as the ``border'' of the shape
    depends on the shape.

    When the user request a point on the border of the shape using the
    |\pgfpointshapeborder| command, the \meta{code} will be executed
    to discern this point. When the execution of  the \meta{code}
    starts, the dimensions |\pgf@x| and |\pgf@y| will have been set to
    a location $p$ in the shape's coordinate system. It is now the job of
    the \meta{code} to setup |\pgf@x| and |\pgf@y| such that they
    specify the point on the shape's border that lies on a straight
    line from the shape's center to the point $p$. Usually, this is a
    somewhat complicated computation, involving many case distinctions
    and some basic math.

    For our |simple rectangle| we must compute a point on the border
    of a rectangle whose one corner is the origin (ignoring the depth
    for simplicity) and whose other corner is |\upperrightcorner|. The
    following code might be used:
\begin{codeexample}[code only]
\anchorborder{%
  % Call a function that computes a border point. Since this
  % function will modify dimensions like \pgf@x, we must move them to
  % other dimensions.
  \@tempdima=\pgf@x
  \@tempdimb=\pgf@y
  \pgfpointborderrectangle{\pgfpoint{\@tempdima}{\@tempdimb}}{\upperrightcorner}
}
\end{codeexample}
  \end{command}  
  \begin{command}{\backgroundpath\marg{code}}
    This command specifies the path that ``makes up'' the background
    of the shape. Note that the shape cannot prescribe what is going
    to happen with the path: It might be drawn, shaded, filled, or
    even thrown away. If you want to specify that something should
    ``always'' happen when this shape is drawn (for example, if the
    shape is a stop-sign, we \emph{always} want it to be filled with a
    red color), you can use commands like |\beforebackgroundpath|,
    explained below.

    When the \meta{code} is executed, all saved anchors will be in
    effect. The \meta{code} should contain path construction
    commands.

    For our |simple rectangle|, the following code might be used:
\begin{codeexample}[code only]
\backgroundpath{
  \pgfpathrectanglecorners
    {\upperrightcorner}
    {\pgfpointscale{-1}{\upperrightcorner}}
}  
\end{codeexample}
    As the name suggests, the background path is used ``behind'' the
    text labels. Thus, this path is used first, then the text labels are
    drawn, possibly obscuring part of the path.
  \end{command}  
  \begin{command}{\foregroundpath\marg{code}}
    This command works like |\backgroundpath|, only it is invoked
    after the text labels have been drawn. This means that this path can
    possibly obscure (part of) the text labels.
  \end{command}  
  \begin{command}{\behindbackgroundpath\marg{code}}
    Unlike the previous two commands, \meta{code} should not only
    construct a path, it should also use this path in whatever way is
    appropriate. For example, the \meta{code} might fill some area
    with a uniform color.

    Whatever the \meta{code} does, it does it first. This means that
    any drawing done by \meta{code} will be even behind the background
    path.

    Note that the \meta{code} is protected with a |{pgfscope}|.
  \end{command}  
  \begin{command}{\beforebackgroundpath\marg{code}}
    This command works like |\behindbackgroundpath|, only the
    \meta{code} is executed after the background path has been used,
    but before the texts label are drawn.
  \end{command}  
  \begin{command}{\behindforegroundpath\marg{code}}
    The \meta{code} is executed after the text labels have been drawn,
    but before the foreground path is used.
  \end{command}  
  \begin{command}{\beforeforegroundpath\marg{code}}
    This \meta{code} is executed at the very end.
  \end{command}  
  \begin{command}{\inheritsavedanchors|[from=|\marg{another shape name}|]|}
    This command allows you to inherit the code for saved anchors from
    \meta{another shape name}. The idea is that if you wish to create
    a new shape that is just a small modification of a another shape,
    you can recycle the code used for \meta{another shape name}.

    The effect of this command is the same as if you had called
    |\savedanchor| and |\saveddimen| for each saved anchor or saved
    dimension declared in \meta{another shape name}. Thus, it is not
    possible to ``selectively'' inherit only some saved anchors, you
    always have to inherit all saved anchors from another
    shape. However, you can inherit the saved anchors of more than one
    shape by calling this command several times.
  \end{command}  
  \begin{command}{\inheritbehindbackgroundpath|[from=|\marg{another shape name}|]|}
    This command can be used to inherit the code used for the
    drawings behind the background path from \meta{another shape name}. 
  \end{command}  
  \begin{command}{\inheritbackgroundpath|[from=|\marg{another shape name}|]|}
    Inherits the background path code from \meta{another shape name}.
  \end{command}  
  \begin{command}{\inheritbeforebackgroundpath|[from=|\marg{another shape name}|]|}
    Inherits the before background path code from \meta{another shape name}.
  \end{command}  
  \begin{command}{\inheritbehindforegroundpath|[from=|\marg{another shape name}|]|}
    Inherits the behind foreground path code from \meta{another shape name}.
  \end{command}  
  \begin{command}{\inheritforegroundpath|[from=|\marg{another shape name}|]|}
    Inherits the foreground path code from \meta{another shape name}.
  \end{command}  
  \begin{command}{\inheritbeforeforegroundpath|[from=|\marg{another shape name}|]|}
    Inherits the before foreground path code from \meta{another shape name}.
  \end{command}  
  \begin{command}{\inheritanchor|[from=|\marg{another shape name}|]|\marg{name}}
    Inherits the code of one specific anchor named \meta{name} from
    \meta{another shape name}. Thus, unlike saved anchors, which must
    be inherited collectively, normal anchors can and must be
    inherited individually.
  \end{command}  
  \begin{command}{\inheritanchorborder|[from=|\marg{another shape name}|]|}
    Inherits the border anchor code from \meta{another shape name}.
  \end{command}

  The following example shows how a shape can be defined that relies
  heavily on inheritance:
\makeatletter
\begin{codeexample}[]
\pgfdeclareshape{document}{
  \inheritsavedanchors[from=rectangle] % this is nearly a rectangle
  \inheritanchorborder[from=rectangle]
  \inheritanchor[from=rectangle]{center}
  \inheritanchor[from=rectangle]{north}
  \inheritanchor[from=rectangle]{south}
  \inheritanchor[from=rectangle]{west}
  \inheritanchor[from=rectangle]{east}
  % ... and possibly more
  \backgroundpath{% this is new
    % store lower right in xa/ya and upper right in xb/yb
    \southwest \pgf@xa=\pgf@x \pgf@ya=\pgf@y
    \northeast \pgf@xb=\pgf@x \pgf@yb=\pgf@y
    % compute corner of ``flipped page''
    \pgf@xc=\pgf@xb \advance\pgf@xc by-5pt % this should be a parameter
    \pgf@yc=\pgf@yb \advance\pgf@yc by-5pt
    % construct main path
    \pgfpathmoveto{\pgfpoint{\pgf@xa}{\pgf@ya}}
    \pgfpathlineto{\pgfpoint{\pgf@xa}{\pgf@yb}}
    \pgfpathlineto{\pgfpoint{\pgf@xc}{\pgf@yb}}
    \pgfpathlineto{\pgfpoint{\pgf@xb}{\pgf@yc}}
    \pgfpathlineto{\pgfpoint{\pgf@xb}{\pgf@ya}}
    \pgfpathclose
    % add little corner
    \pgfpathmoveto{\pgfpoint{\pgf@xc}{\pgf@yb}}
    \pgfpathlineto{\pgfpoint{\pgf@xc}{\pgf@yc}}
    \pgfpathlineto{\pgfpoint{\pgf@xb}{\pgf@yc}}
    \pgfpathlineto{\pgfpoint{\pgf@xc}{\pgf@yc}}
 }
}\hskip-1.2cm
\begin{tikzpicture}
  \node[shade,draw,shape=document,inner sep=2ex] (x) {Remark};
  \node[fill=examplefill,draw,ellipse,double]
    at ([shift=(-80:3cm)]x) (y) {Use Case};

  \draw[dashed] (x) -- (y);  
\end{tikzpicture}
\end{codeexample}
  
\end{command}




%%% Local Variables: 
%%% mode: latex
%%% TeX-master: "pgfmanual"
%%% End: 
