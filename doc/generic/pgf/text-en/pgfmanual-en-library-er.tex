% Copyright 2003 by Till Tantau <tantau@cs.tu-berlin.de>.
%
% This program can be redistributed and/or modified under the terms
% of the LaTeX Project Public License Distributed from CTAN
% archives in directory macros/latex/base/lppl.txt.



\section{Entity-Relationship Diagram Drawing Library}

\begin{package}{pgflibrarytikzer}
  This packages provides styles for drawing entity-relationship
  diagrams. 
\end{package}

This library is intended to help you creating E/R-diagrams. It defines
only very little new styles, but using these style |entity| instead of
saying |rectangle,draw| makes the code more expressive.


\subsection{Entities}

The package defines a simple style for drawing entities:

\begin{itemize}
  \itemstyle{entity}
  This style is to be used with nodes that represent entity types. It
  causes the node's shape to be set to a rectangle that is drawn and
  whose minimum size and width are set to sensible values.

  Note that this style is called |entity| despite the fact that it is
  to be used for nodes representing entity \emph{types} (the
  difference between an entity and an entity type is the same as the
  difference between an object and a class in object-oriented
  programming). If this bothers you, feel free to define a style
  |entity type| instead.
\begin{codeexample}[]
\begin{tikzpicture}[node distance=2cm]
  \node[entity] (sheep)                   {Sheep};
  \node[entity] (genome) [right of=sheep] {Genome};
\end{tikzpicture}
\end{codeexample}
  
  \itemstyle{every entity}
  This stype is envoked by the style |entity|. To change the
  appearance of entities, you can change this style.
\begin{codeexample}[]
\begin{tikzpicture}[node distance=2cm]
  \tikzstyle{every entity}=[draw=blue!50,fill=blue!20,thick]
  \node[entity] (sheep)                   {Sheep};
  \node[entity] (genome) [right of=sheep] {Genome};
\end{tikzpicture}
\end{codeexample}
\end{itemize}



\subsection{Relationships}

Relationships are drawn using styles that are very similar to the
styles for entities.

\begin{itemize}
  \itemstyle{relationship}
  This style works like |entity|, only it is to be used for
  relationships. Again, |relationship|s are actually relationship types. 
\begin{codeexample}[]
\begin{tikzpicture}
  \node[entity] (sheep)  at (0,0)   {Sheep};
  \node[entity] (genome) at (2,0)   {Genome};
  \node[relationship]    at (1,1.5) {has}
    edge (sheep) 
    edge (genome);
\end{tikzpicture}
\end{codeexample}
  \itemstyle{every relationship}
  works like |every entity|
\begin{codeexample}[]
\begin{tikzpicture}
  \tikzstyle{every entity}=[fill=blue!20,draw=blue,thick]
  \tikzstyle{every relationship}=[fill=orange!20,draw=orange,thick,aspect=1.5]
  \node[entity] (sheep)  at (0,0)   {Sheep};
  \node[entity] (genome) at (2,0)   {Genome};
  \node[relationship]    at (1,1.5) {has}
    edge (sheep) 
    edge (genome);
\end{tikzpicture}
\end{codeexample}
\end{itemize}



\subsection{Attributes}

\begin{itemize}
  \itemstyle{attribuate}
  This style is used to indicate that a node is an attribute. To
  connect an attribute to its entity, you can use, for example, the
  |child| command or the |pin| option. 
\begin{codeexample}[]
\begin{tikzpicture}
  \node[entity] (sheep)  {Sheep}
    child {node[attribute] {name}}
    child {node[attribute] {color}};
\end{tikzpicture}
\end{codeexample}
\begin{codeexample}[]
\begin{tikzpicture}
  \tikzstyle{every pin edge}=[draw]    
  \node[entity,pin={[attribute]60:name},pin={[attribute]120:color}] {Sheep};
\end{tikzpicture}
\end{codeexample}
  \itemstyle{key attribute}
  This style is intended for key attributes. By default, the will
  cause the attribute to be typeset in italics. Typcially, underlining
  is used instead, but that looks ugly and it is difficult to
  implement in \TeX.
  \itemstyle{every attribute}
  This style is used with every (key) attribute.
\begin{codeexample}[]
\begin{tikzpicture}[text depth=1pt]
  \tikzstyle{every attribute}=[fill=black!20,draw=black]
  \tikzstyle{every entity}=[fill=blue!20,draw=blue,thick]
  \tikzstyle{every relationship}=[fill=orange!20,draw=orange,thick,aspect=1.5]
  \node[entity] (sheep)  at (0,0)   {Sheep}
    child {node  [key attribute] {name}};
  \node[entity] (genome) at (2,0)   {Genome};
  \node[relationship]    at (1,1.5) {has}
    edge (sheep) 
    edge (genome);
\end{tikzpicture}
\end{codeexample}
\end{itemize}



%%% Local Variables: 
%%% mode: latex
%%% TeX-master: "pgfmanual-pdftex-version"
%%% End: 
