% Copyright 2006 by Till Tantau
%
% This file may be distributed and/or modified
%
% 1. under the LaTeX Project Public License and/or
% 2. under the GNU Free Documentation License.
%
% See the file doc/generic/pgf/licenses/LICENSE for more details.


\section{Fitting Library}
\label{section-library-fit}

\begin{tikzlibrary}{fit}
  The library defines (currently only two) options for fitting a node
  so that it contains a set of coordinates.
\end{tikzlibrary}

When you load this library, the following options become available:

\begin{key}{/tikz/fit=\meta{coordinates or nodes}}
  This option must be given to a |node| path command. The
  \meta{coordinates or nodes} should be a sequence of \tikzname\
  coordinates or node names,  one directly after the other without
  commas (like with the |plot coordinates| path operation). Examples
  as |(1,0) (2,2)| or |(a) (1,0) (b)|, where |a| and |b| are nodes.

  For this sequence of coordinates, a minimal bounding box is computed
  that encompasses all the listed \meta{coordinates or nodes}. For
  coordinates in the list, the bounding box is guaranteed to contain
  this coordinate, for nodes it is guaranteed to contain the |east|,
  |west|, |north| and |south| anchors of the node. In principle (the
  details will be explained in a moment), things are now setup such
  that the text box of the node will be exactly this bounding box.

  Here is an example: We fit several points in a rectangular node. By
  setting the |inner sep| to zero, we see exactly the text box of the
  node. Then we fit these points again in circular node. Note how
  the circle encompasses exactly the same bounding box.
\begin{codeexample}[]
\begin{tikzpicture}[inner sep=0pt,thick,
                    dot/.style={fill=blue,circle,minimum size=3pt}]
  \draw[help lines] (0,0) grid (3,2);
  \node[dot] (a) at (1,1) {};
  \node[dot] (b) at (2,2) {};
  \node[dot] (c) at (1,2) {};
  \node[dot] (d) at (1.25,0.25) {};
  \node[dot] (e) at (1.75,1.5) {};

  \node[draw=red,   fit=(a) (b) (c) (d) (e)] {box};
  \node[draw,circle,fit=(a) (b) (c) (d) (e)] {};
\end{tikzpicture}  
\end{codeexample}

  Every time the |fit| option is used, the following style is also
  applied to the node:
  \begin{stylekey}{/tikz/every fit (initially \normalfont empty)}
    Set this style to change the appearance of a node that uses the
    |fit| option.
  \end{stylekey}

  The exact effects of the |fit| option are the following:
  \begin{enumerate}
  \item A minimal bounding box containg all coordinates is
    computed. Note that if a coordinate like |(a)| is used that
    contain a node name, this has the same effect as explicitly
    providing the |(a.north)| and |(a.south)| and |(a.west)| and
    |(a.east)|. If you wish to refer only to the center of the |a|
    node, use  |(a.center)| instead.
  \item The |text width| option is set to the width of this bounding box.
  \item The |text centered| option is set.
  \item The |anchor| is set to |center|.
  \item The |at| position of the node is set to the center of the
    computed bounding box.
  \item After the node has been typeset, its height and depth are
    adjusted such that they add up to the height of the computed
    bounding box and such that the text of the node is vertically
    centered inside the box.
  \end{enumerate}
  The above means that, generally speaking, if the node contains text
  like |box| in the above example, it will be centered inside the
  box. It will be difficult to put the text elsewhere, in particular,
  changing the |anchor| of the node will not have the desired
  effect. Instead, what you should do is to create a node with the
  |fit| option that does not contain any text, give it a name, and
  then use normal nodes to add text at the desired
  positions. Alternatively, consider using the |label| or |pin|
  options. 

  Suppose, for instance, that in the above example we want the word
  ``box'' to appear inside the box, but at its top. This can be
  achieved as follows: 
\begin{codeexample}[]
\begin{tikzpicture}[inner sep=0pt,thick,
                    dot/.style={fill=blue,circle,minimum size=3pt}]
  \draw[help lines] (0,0) grid (3,2);
  \node[dot] (a) at (1,1) {};
  \node[dot] (b) at (2,2) {};
  \node[dot] (c) at (1,2) {};
  \node[dot] (d) at (1.25,0.25) {};
  \node[dot] (e) at (1.75,1.5) {};

  \node[draw=red,fit=(a) (b) (c) (d) (e)] (fit) {};
  \node[below] at (fit.north) {box};
\end{tikzpicture}  
\end{codeexample}

 Here is a real-life example that uses fitting:

\begin{codeexample}[]
\begin{tikzpicture}
  [vertex/.style={minimum size=2pt,fill,draw,circle},
   open/.style={fill=none},
   sibling distance=1.5cm,level distance=.75cm,
   every fit/.style={ellipse,draw,inner sep=-2pt},
   leaf/.style={label={[name=#1]below:$#1$}},auto]

  \node [vertex] (root) {}
  child { node [vertex,open] {}
    child { node [vertex,open] {}
      child { node [vertex] (b's parent) {}
        child { node [vertex] {}
          child { node [vertex,leaf=d] {} }
          child { node [vertex,leaf=e] {} } }
        child { node [vertex,leaf=b] {} } }
      child { node [vertex,leaf=a] {} } }
    child { node [coordinate] {}
      child[missing] 
      child { node [vertex] (f's parent) {}
        child { node [vertex,leaf=c] {} }
        child { node [vertex,leaf=f] {} } } }
    edge from parent node {$\rho$} };
  
  \node [fit=(d) (e) (b) (b's parent),label=above left:$F^{(b,R)}$] {};
  \node [fit=(c) (f) (f's parent),label=above right:$F^{(c,R)}$]    {};
\end{tikzpicture}
\end{codeexample}

\end{key}

\begin{key}{/tikz/rotate fit=\meta{angle} (initially 0)}
  This key fits \meta{coordinates or nodes} inside a node that is
  rotated by \meta{angle}. As a side effect, it also sets the
  |/tikz/rotate| key.

\begin{codeexample}[]  
\begin{tikzpicture}[inner sep=0pt,thick,
  dot/.style={fill=blue,circle,minimum size=3pt}]
  \draw[help lines] (0,0) grid (3,2);
  \node[dot] (a) at (1,1) {};
  \node[dot] (b) at (2,2) {};
  \node[dot] (c) at (1,2) {};
  \node[dot] (d) at (1.25,0.25) {};
  \node[dot] (e) at (1.75,1.5) {};
  \node[draw, fit=(a) (b) (c) (d) (e)] {};
  \node[draw=red, rotate fit=30, fit=(a) (b) (c) (d) (e)] {};
\end{tikzpicture}
\end{codeexample}

\end{key}




