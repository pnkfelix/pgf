% Copyright 2006 by Till Tantau
%
% This file may be distributed and/or modified
%
% 1. under the LaTeX Project Public License and/or
% 2. under the GNU Free Documentation License.
%
% See the file doc/generic/pgf/licenses/LICENSE for more details.


\section{Introduction to Data Visualization}



% The following is just testing.

\usepgflibrary{datavisualization,plotmarks,plothandlers}


\def\daxline#1 #2 #3 #4 {
  \pgfkeyssetvalue{/data point/dax/low}{#1}
  \pgfkeyssetvalue{/data point/dax/high}{#2}
  \pgfkeyssetvalue{/data point/dax/entry}{#3}
  \pgfkeyssetvalue{/data point/dax/exit}{#4}
  \pgfdatapoint
}

\begin{tikzpicture}
  \pgfnewdatavisualization

  \pgfoonew new count up(day)
  
  \pgfoonew new line transformer(day,\pgfqpoint{3mm}{0mm})
  \pgfoonew new line transformer(y,\pgfqpoint{1cm}{1cm})

  \pgfoonew new attribute locator(dax,0,100,y,0,1,)
  \pgfoonew new candle stick visualizer(dax)

  \pgfoonew \dayinterval=new interval(,)
  \dayinterval.set min adjust relative(-.1)
  \dayinterval.set max adjust relative(.1)
  \pgfoonew \daybounder=new attribute bounder(day,\dayinterval)

  \pgfoonew \daxinterval=new interval(,)
  \daxinterval.set min adjust relative(-.1)
  \daxinterval.set max adjust relative(.1)
  \pgfoonew \daxbounder=new attribute bounder(dax,\daxinterval)
  
  \pgfoonew \dayaxis=new straight axis(day,\dayinterval)
  \pgfoonew \daxaxis=new straight axis(dax,\daxinterval)

  \begin{pgfdataset}
    \daxline 2000 2300 2100 2200 
    \daxline 2000 2350 2200 2250 
    \daxline 2200 2300 2250 2260 
    \daxline 1800 2260 2260 1900 
    \daxline 2000 2300 2100 2200 
    \daxline 2000 2350 2200 2250 
    \daxline 2200 2300 2250 2260 
    \daxline 1800 2260 2260 1900 
  \end{pgfdataset}


\end{tikzpicture}


\def\dataline#1 #2 #3 #4 #5 #6 #7 #8 {%
  \pgfkeyssetvalue{/data point/a}{#1}
  \pgfkeyssetvalue{/data point/b}{#2}
  \pgfkeyssetvalue{/data point/c}{#3}
  \pgfkeyssetvalue{/data point/d}{#4}
  \pgfkeyssetvalue{/data point/e}{#5}
  \pgfkeyssetvalue{/data point/f}{#6}
  \pgfkeyssetvalue{/data point/g}{#7}
  \pgfkeyssetvalue{/data point/h}{#8}
  \pgfdatapoint
}

\begin{tikzpicture}
  \pgfnewdatavisualization

  %\pgfoonew new attribute locator(a,0,1,x,0,1,)
  %\pgfoonew new attribute locator(b,0,1,y,0,1,)
  \pgfoonew new count up(y)

  \pgfoonew new accumulator(a,x)
  
  \pgfoonew \transformer=new line transformer(x,\pgfpointxy{.1}{0})
  \pgfoonew \transformer=new line transformer(y,\pgfpointxy{0}{1})
  \pgfoonew \visualizer=new plot handler visualizer(\pgfplothandlerlineto)
  \visualizer.set use path(\color{blue}\pgfusepath{stroke})
  \pgfoonew \visualizer=new plot mark visualizer(*)

  \pgfoonew \interval=new interval(0,)
  \interval.set min adjust relative(-.1)
  \interval.set max adjust relative(.1)
  \pgfoonew \abounder=new attribute bounder(x,\interval)
  
  \pgfoonew \yinterval=new interval(,)
  \yinterval.set min adjust relative(-.1)
  \yinterval.set max adjust relative(.1)
  
  \pgfoonew \bbounder=new attribute bounder(y,\yinterval)

  \pgfoonew \pinterval=new interval(,)
  \pgfoonew \pbounder=new attribute bounder(angle,\pinterval)

  \pgfoonew \rinterval=new interval(,)
  \pgfoonew \rbounder=new attribute bounder(b,\rinterval)

  \pgfoonew \axis=new straight axis(x,\interval)
  \pgfoonew \bxis=new straight axis(y,\yinterval)

  \begin{pgfdataset}
    \dataline 10.0	8.04	10.0	9.14	10.0	7.46	8.0	6.58 
    \dataline 8.0	6.95	8.0	8.14	8.0	6.77	8.0	5.76 
    \dataline 13.0	7.58	13.0	8.74	13.0	12.74	8.0	7.71 
    \dataline 9.0	8.81	9.0	8.77	9.0	7.11	8.0	8.84 
    \dataline 11.0	8.33	11.0	9.26	11.0	7.81	8.0	8.47 
    \dataline 14.0	9.96	14.0	8.10	14.0	8.84	8.0	7.04 
    \dataline 6.0	7.24	6.0	6.13	6.0	6.08	8.0	5.25 
    \dataline 4.0	4.26	4.0	3.10	4.0	5.39	19.0	12.50 
    \dataline 12.0	10.84	12.0	9.13	12.0	8.15	8.0	5.56 
    \dataline 7.0	4.82	7.0	7.26	7.0	6.42	8.0	7.91 
    \dataline 5.0	5.68	5.0	4.74	5.0	5.73	8.0    6.89 
  \end{pgfdataset}

  \begin{scope}[->,thick]
     {
       \bbounder.goto min()
       \axis.visualize axis()
       % Grid lines
       \begin{scope}[-,thin,black!50]
         \bbounder.goto min()
         \pgfkeysgetvalue{/data point/y}{\mymin}
         \bbounder.goto max()
         \pgfkeysgetvalue{/data point/y}{\mymax}
         \foreach \yval in {\mymin,...,\mymax}
         {
           \pgfkeyssetvalue{/data point/y}{\yval}
           \axis.visualize axis()
         }
       \end{scope}
     }
    {
      \abounder.goto min()
      \bxis.visualize axis()
      % Grid lines
      \begin{scope}[-,thin,black!50]
        \abounder.goto min()
        \pgfkeysgetvalue{/data point/x}{\mymin}
        \abounder.goto max()
        \pgfkeysgetvalue{/data point/x}{\mymax}
        \foreach \xval in {0,10,...,\mymax}
        {
          \pgfkeyssetvalue{/data point/x}{\xval}
          \bxis.visualize axis()
        }
      \end{scope}
    }
  \end{scope}
  

\end{tikzpicture}



%%% Local Variables: 
%%% mode: latex
%%% TeX-master: "pgfmanual"
%%% End: 
