% Copyright 2007 by Mark Wibrow
%
% This file may be distributed and/or modified
%
% 1. under the LaTeX Project Public License and/or
% 2. under the GNU Free Documentation License.
%
% See the file doc/generic/pgf/licenses/LICENSE for more details.


\section{Customising the Mathematical Engine}

\label{pgfmath-reimplement}


Perhaps you have a desire for some function that \pgfname\ does not 
provide. Perhaps you are not happy with the accuracy or efficiency of 
some of the algorithms that are implemented in \pgfname. In these 
cases you will want to add a function to the parser or replace the 
current implementations of the algorithms with your own code.

The mathematical engine was designed with such customisation in mind.
It is possible to add new functions, or modify the code for for 
existing functions. Note, however, that whilst adding new operators 
is possible, it can be a rather tricky business and is only 
recommended for adventurous users.

To add a new function to the math engine the following command can be
used:

\begin{command}{\pgfmathdeclarefunction\marg{function name}\marg{number of arguments}\marg{code}}

  This will set up the parser to recognise a function called 
  \meta{name}. The name of the function can consist of, uppercase or
  lower case letters, numbers or the underscore |_|. In line with 
  many programming languages, a function name cannot begin with a 
  number or contain any spaces.
  
  The \meta{number of arguments} can be any positive integer, zero,
  or the value |...|, which indicates a variable number of 
  arguments. \pgfname{} treats constants, such as |pi| and |e|, as
  functions with zero arguments. Functions with more than nine
  arguments or with variable arguments are a ``bit special'' and
  are discussed below.
  
  The effect of \meta{code} should be to set the macro 
  |\pgfmathresult| to the correct value (namely to the result of the 
  computation without units).  Furthermore, the function should have 
  no other side effects, that is, it should not change any global 
  values. As an example, consider the creation of a new function
  |double|, which takes one argument, and returns the value of that
  argument times two.
  
\begin{codeexample}[]
\makeatletter
\pgfmathdeclarefunction{double}{1}{
  \begingroup
    \pgf@x=#1pt\relax
    \multiply\pgf@x by2\relax
    \pgfmathreturn\pgf@x
  \endgroup
}
\makeatother
\pgfmathparse{double(44.3)}\pgfmathresult
\end{codeexample}

  The macro |\pgfmathreturn|\meta{tokens} must be 
  directly followed by an |\endgroup| and will save the result of the 
  computation, by defining |\pgfmathresult| as the expansion of 
  \meta{tokens} (without units) outside the group, so \meta{tokens}
  must be somthing that can be assigned to a dimension register.
  
  Alternatively, the |\pgfmathsmuggle|\meta{macro} can be used. This
  must also be directly followed by an |\endgroup| and will simply
  ``smuggle'' the definition of \meta{macro} outside the \TeX-group.
  
	By performing computations within a \TeX-group, \pgfname{}
	registers such as |\pgf@x|, |\pgf@y| and |\c@pgf@counta|, 
	|\c@pgfcountb|, and so forth, can be used at will.
  
  Beyond setting up the parser, this command also defines two macros
  which provide access to the function independently of the parser:  
    
  \begin{itemize}
  \item
  |\pgfmath|\meta{function name}
  
  This macro will provide ``public'' interface for the function 
  \meta{function name} allowing the function to be called 
  independently of the parser. All arguments passed to this macro are
  evaluated using |\pgfmathparse| and then passed on to the following 
  macro:
  
  \item
  |\pgfmath|\meta{function name}|@|
  
  This macro is the ``private'' implementation of the functions 
  algorithm (but note that, for speed, the parser calls this macro 
  rather than the ``public'' one). Arguments passed to this macro 
  are expected to be numbers without units. It is defined using
  \meta{code}, but need not be self contained.
  
  \end{itemize}
  
  For functions that are declared with less than ten arguments,
  the public macro is defined in the same way as normal \TeX{} 
  macros using, for example, |\def\pgfmathNoArgs{|\meta{code}|}| 
  for a function with no arguments, or
   |\def\pgfmathThreeArgs#1#2#3{|\meta{code}|}| for a function with
   three arguments.
  The private macro is defined in the same way, and each argument 
  can therefore be accessed in \meta{code}
  using |#1|, |#2| and so on.
  
  For functions with more than nine arguments, or functions with
  a variable number of arguments, these macros are only
  defined as taking \emph{one} argument. The public macro
  expects its arguments to be comma separated, for example,
  |\pgfmathVariableArgs{1.1,3.5,-1.5,2.6}|. Each
  argument is parsed and 
  passed on to the private macro as follows:
  |\pgfmathVariableArgs@{{1.1}{3.5}{-1.5}{2.6}}|.
  This means that some ``extra work'' will be required to access
  each argument (although it is a fairly simple task).
  
  Note, that there are, two execptions to this arrangement:
  the public versions of the |min| and |max| functions still
  take two arguments for compatibility with older versions, but
  each of these arguments can take several comma separated values.
  


  
\end{command}

  To redefine a function use the following command: 

\begin{command}{\pgfmathredeclarefunction\marg{function name}\marg{algorithm code}}
  
  This command redefines the |\pgfmath|\meta{function name}|@| macro 
  with the new \meta{algorithm code}. See the description of the
  |\pgfmathdeclarefunction| for details. You cannot change the number
  of arguments for an existing function.

\end{command}

  \pgfname{} uses the last known definition of a function within the
  prevailing scope, so it is possible for a function to be redefined 
  locally. You should also remember that any |.sty| or |.tex| file
  contatining any re-implementions should be loaded after pgf-Math.
  