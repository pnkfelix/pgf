\section{Tutorial: A Petri-Net for Hagen}

In this second tutorial we explore the node mechanism of
\tikzname\ and \pgfname.

Hagen must give a talk tomorrow about his favorite formalism for
distributed systems: Petri nets! Hagen used to give his talks using a
blackboard and everyone seemed to be perfectly concent with
this. Unfortunately, his audience has been spoiled recently with fancy
projector-based presentations and there seems to be a certain amount
of peer pressure that this Petri nets should also be drawn using a
graphic program. One of the professors at his institutes recommends
\tikzname\ for this and Hagen decides to give it a try.


\subsection{Problem Statement}

For his talk, Hagen wishes to create a graphic that demonstrates how a
net with place capacities can be simulated by a net without
capacities. The graphic should look like this, ideally:

\begin{quote}
\begin{tikzpicture}[node distance=1.3cm,>=stealth',bend angle=45]

  \tikzstyle{place}=[circle,thick,draw=blue!75,fill=blue!20,minimum size=6mm]
  \tikzstyle{red place}=[place,draw=red!75,fill=red!20]
  \tikzstyle{transition}=[rectangle,thick,draw=black!75,fill=black!20,minimum size=4mm]
  \tikzstyle{pre}=[<-,shorten <=1pt,semithick,black!80]
  \tikzstyle{post}=[->,shorten >=1pt,semithick,black!80]

  \tikzstyle{mark}=[circle,fill,minimum size=1.5mm,inner sep=0pt]

  \begin{scope}
    % First net
    \node [place] (w1)               {};
    \node [place] (c1) [below of=w1] {};
    \node [place] (s)  [below of=c1] {}; \node [above,red!75!black] at (s.north) {$s\le 3$};
    \node [place] (c2) [below of=s]  {};
    \node [place] (w2) [below of=c2] {};
    
    \node [transition] (e1) [left of=c1] {}
      edge [pre,bend left]   (w1)
      edge [post,bend right] (s)
      edge [post]            (c1);

    \node [transition] (e2) [left of=c2] {}
      edge [pre,bend right]  (w2)
      edge [post,bend left]  (s)
      edge [post]            (c2);
      
    \node [transition] (l1) [right of=c1] {}
      edge [pre]             (c1)
      edge [pre,bend left]   (s)
      edge [post,bend right] (w1);

    \node [transition] (l2) [right of=c2] {}
      edge [pre]             (c2)
      edge [pre,bend right]  (s)
      edge [post,bend left]  (w2);

    \node [mark] at (w1) {};
    \node [mark] at (w2) {};
  \end{scope}
  
  \begin{scope}[xshift=6cm]
    % Second net
    \node [place]     (w1')                            {};
    \node [place]     (c1') [below of=w1']             {};
    \node [red place] (s1') [below of=c1',xshift=-5mm] {}; \node [left,red!75!black]  at (s1'.west) {$s$};
    \node [red place] (s2') [below of=c1',xshift=5mm]  {}; \node [right,red!75!black] at (s2'.east) {$\bar s$};
    \node [place]     (c2') [below of=s1',xshift=5mm]  {};
    \node [place]     (w2') [below of=c2']             {};
    
    \node [transition] (e1') [left of=c1'] {}
      edge [pre,bend left]   (w1')
      edge [post]            (s1')
      edge [pre]             (s2')
      edge [post]            (c1');

    \node [transition] (e2') [left of=c2'] {}
      edge [pre,bend right]  (w2')
      edge [post]            (s1')
      edge [pre]             (s2')
      edge [post]            (c2');
      
    \node [transition] (l1') [right of=c1'] {}
      edge [pre]             (c1')
      edge [pre]             (s1')
      edge [post]            (s2')
      edge [post,bend right] (w1');

    \node [transition] (l2') [right of=c2'] {}
      edge [pre]             (c2')
      edge [pre]             (s1')
      edge [post]            (s2')
      edge [post,bend left]  (w2');

    \node [mark] at (w1') {};
    \node [mark] at (w2') {};
    \node [mark] at ([shift={(90:1.1mm)}]s2') {};
    \node [mark] at ([shift={(210:1.1mm)}]s2') {};
    \node [mark] at ([shift={(330:1.1mm)}]s2') {};
  \end{scope}

  \draw [-to,thick,snake=snake,segment amplitude=.4mm,segment length=2mm,line after snake=1mm]
    ([xshift=5mm]s -| l1) -- ([xshift=-5mm]s1' -| e1')
    node [above=1mm,midway,text width=3cm,text centered]
      {replacement of the \textcolor{red}{capacity} by \textcolor{red}{two places}};

  \begin{pgfonlayer}{background}
    \filldraw [line width=4mm,join=round,black!10]
      (w1.north  -| l1.east)  rectangle (w2.south  -| e1.west)
      (w1'.north -| l1'.east) rectangle (w2'.south -| e1'.west);
  \end{pgfonlayer}
\end{tikzpicture}
\end{quote}


\subsection{Setting up the Environment}

For the picture Hagen will need to load the \tikzname\ package as did
Karl in the previous tutorial. However, Hagen will also need to load
some additional  \emph{library packages} that Karl did not need. These
library packages contain additional definitions like extra arrow tips
that are typically not needed in a picture and that need to be
loaded explicitly.

Hagen will need to load four libraries: The arrow tip library for the
special arrow tip used in the graphic, the snake library with the
``snaking line'' in the middle, the background library for the two
rectangular areas that are behind the two main parts of the picture,
and the to-path library for easily creating the curved lines. 


\subsubsection{Setting up the Environment in \LaTeX}

When using \LaTeX\ use:

\begin{codeexample}[code only]
\documentclass{article} % say

\usepackage{tikz}
\usepackage{pgflibraryarrows}
\usepackage{pgflibrarysnakes}
\usepackage{pgflibrarytikzbackgrounds}
\usepackage{pgflibrarytikztopaths}

\begin{document}
\begin{tikzpicture}
  \draw (0,0) -- (1,1);
\end{tikzpicture}
\end{document}
\end{codeexample}


\subsubsection{Setting up the Environment in Plain \TeX}

When using plain \TeX\ use:

\begin{codeexample}[code only]
%% Plain TeX file
\input tikz.tex
\input pgflibraryarrows.tex
\input pgflibrarysnakes.tex
\input pgflibrarytikzbackgrounds.tex
\input pgflibrarytikztopaths.tex
\baselineskip=12pt
\hsize=6.3truein
\vsize=8.7truein
\tikzpicture
  \draw (0,0) -- (1,1);
\endtikzpicture
\bye
\end{codeexample}


\subsubsection{Setting up the Environment in Con\TeX t}

When using Con\TeX\ use:
\begin{codeexample}[code only]
%% ConTeXt file
\usemodule[tikz]
\usemodule[pgflibraryarrows]
\usemodule[pgflibrarysnakes]
\usemodule[pgflibrarytikzbackgrounds]
\usemodule[pgflibrarytikztopaths]

\starttikzpicture
  \draw (0,0) -- (1,1);
\stoptikzpicture
\end{codeexample}



\subsection{Introduction to Nodes}

In principle, we already know how to create the graphics that Hagen
desires (except perhaps for the snaked line, we will come to that): We
start with big light gray rectangle and then add lots of circles and
small rectangle, plus some arrows.

However, this approach has numerous disadvantages: First, it is hard
to change anything at a later stage. For example, if we decide to add
more places to the Petri nets (the circles are called places in Petri
net theory), all of the coordinates change and we need to recalculate
everything. Second, it is hard to read the code for the Petri net as
it just a long and complicated list of coordinates and drawing
commands -- the underlying structure of the Petri net is lost.

Fortunately, \tikzname\ offers a powerful mechanism for avoiding the
above problems: nodes. We already came across nodes in the previous
tutorial, where we used them to add labels to Karl's graphic. In the
present tutorial we will see that nodes are much more powerful.

A node is a small part of a picture. When a node is created, you
provide a position where the node should be drawn and a
\emph{shape}. A node of shape |circle| will be drawn as a circle, a
node of shape |rectangle| as a rectangle, and so on. A node may also
contain same text, which is why Karl used nodes to show text. Finally,
a node can get a \emph{name} for later reference.

In Hagen's picture we will use nodes for the places and for the
transitions of the Petri net (the places are the circles, the
transitions are the rectangles). Let us start with the upper half of
the left Petir net. In this upper half we have three places and two
transitions. Instead of drawing three circles and two rectangles, we
use three nodes of shape |circle| and two nodes of shape
|rectangle|.

\begin{codeexample}[]
\begin{tikzpicture}
  \path ( 0,2) node [shape=circle,draw] {}    
        ( 0,1) node [shape=circle,draw] {}    
        ( 0,0) node [shape=circle,draw] {}    
        ( 1,1) node [shape=rectangle,draw] {}
        (-1,1) node [shape=rectangle,draw] {};    
\end{tikzpicture}
\end{codeexample}

Hagen notes that this does not quite look like the final picture, but
it seems like a good first step.

Let us have a more detailed look at the code. The whole picture
consists of a single path. Ignoring the |node| operations there is not
much going on in this path: It is just a sequence of coordinates with
nothing ``happening'' between them. Indeed, even if something were to
happen like a line-to or a curve-to, the |\path| command would not
``do'' anything with the resulting path. So, all the magic must be in
the |node| commands.

In the previous tutorial we learned that a |node| will add a piece of
text at the last coordinate. Thus, each of the five nodes is added at
a different position. In the above code, this text is empty
(because of the empty |{}|). So, why do we see anything at all? The
answer is the |draw| option for the |node| operation: It causes the
``shape around the text'' to be drawn.

So, the code |(0,2) node [shape=circle,draw] {}| means the following:
``In the main path, add a move-to to the coordinate |(0,2)|. Then,
temporarily suspend the construction of the main path while the node
is build. This node will be a |circle| around an empty text. This
circle is to be |draw|n, but not filled or otherwise used. Once this
whole node is constructed, it is saved until after the 
main path is finished. Then, it is drawn.'' Then following
|(0,1) node [shape=circle,draw] {}| then has the following effect:
``Continue the main path with a move-to to |(0,1)|. Then construct a
node at this position also. This node is also shown after the main
path is finished.'' And so on.



\subsection{Placing Nodes I}

Hagen now understands how the |node| operation adds nodes to the path,
but it seems a bit silly to create a path using the |\path| operation,
consisting of numerous superfluous move-to operations, only to place
nodes. He is pleased to learn that there are ways to add nodes in a
more sensible manner.

First, the |node| operation allows one to add
|at (|\meta{coordinate}|)| in order to directly specify where the node
should be placed, sidestepping the rule that nodes are placed on the
last coordinate. Hagen can then write the following:

\begin{codeexample}[]
\begin{tikzpicture}
  \path node at ( 0,2) [shape=circle,draw] {}    
        node at ( 0,1) [shape=circle,draw] {}    
        node at ( 0,0) [shape=circle,draw] {}    
        node at ( 1,1) [shape=rectangle,draw] {}
        node at (-1,1) [shape=rectangle,draw] {};    
\end{tikzpicture}
\end{codeexample}

Now Hagen is still left with a single empty path, but at least the
path no longer contains strange move-tos. It turns out that this can
be improved further: The |\node| command is an abbreviation for
|\path node|, which allows Hagen to write:

\begin{codeexample}[]
\begin{tikzpicture}
  \node at ( 0,2) [circle,draw] {};
  \node at ( 0,1) [circle,draw] {};   
  \node at ( 0,0) [circle,draw] {};   
  \node at ( 1,1) [rectangle,draw] {};
  \node at (-1,1) [rectangle,draw] {};    
\end{tikzpicture}
\end{codeexample}

Hagen likes this syntax much better than the previous one. Note that
Hagen has also omitted the |shape=| since, like |color=|, \tikzname\ 
allows you to omit the |shape=| if there is no confusion.



\subsection{Using Styles}

Feeling adventurous, Hagen tries to make the nodes look nicer. In the
final picture, the circles and rectangle should be filled with
different colors, resulting in the following code:

\begin{codeexample}[]
\begin{tikzpicture}[thick]
  \node at ( 0,2) [circle,draw=blue!50,fill=blue!20] {};
  \node at ( 0,1) [circle,draw=blue!50,fill=blue!20] {};   
  \node at ( 0,0) [circle,draw=blue!50,fill=blue!20] {};   
  \node at ( 1,1) [rectangle,draw=black!50,fill=black!20] {};
  \node at (-1,1) [rectangle,draw=black!50,fill=black!20] {};    
\end{tikzpicture}
\end{codeexample}

While this looks nicer in the picture, the code starts to get a bit
ugly. Ideally, we would like our code to transport the message ``there
are three places and two transitions'' and not so much which
filling colors should be used.

To solve this problem, Hagen uses styles. He defines a style for
places and another style for transitions:

\begin{codeexample}[]
\tikzstyle{place}=[circle,draw=blue!50,fill=blue!20,thick]
\tikzstyle{transition}=[rectangle,draw=black!50,fill=black!20,thick]
\begin{tikzpicture}
  \node at ( 0,2) [place] {};
  \node at ( 0,1) [place] {};   
  \node at ( 0,0) [place] {};   
  \node at ( 1,1) [transition] {};
  \node at (-1,1) [transition] {};    
\end{tikzpicture}
\end{codeexample}


\subsection{Node Size}

Before Hagen starts naming and connecting the nodes, let us first
make sure that the nodes get their final appearance. They are still
too small. Indeed, Hagen wonders why they have any size at all, after
all, the text is empty. The reason is than \tikzname\ automatically
adds some space around the text. The amount is set using the option
|inner sep|. So, to increase the size of the nodes, Hagen could write:

\begin{codeexample}[]
\tikzstyle{place}=[circle,draw=blue!50,fill=blue!20,thick]
\tikzstyle{transition}=[rectangle,draw=black!50,fill=black!20,thick]
\begin{tikzpicture}[inner sep=2mm]
  \node at ( 0,2) [place] {};
  \node at ( 0,1) [place] {};   
  \node at ( 0,0) [place] {};   
  \node at ( 1,1) [transition] {};
  \node at (-1,1) [transition] {};    
\end{tikzpicture}
\end{codeexample}

However, this is not really the best way to achieve the desired
effect. It is much better to use the |minimum size| option
instead. This option allows Hagen to specify a minimum size that the
node should have. If the nodes actually needs to be bigger because of
a longer text, it will be larger, but if the text is empty, then the
node will have |minimum size|. This option is also useful to ensure
that several nodes containing different amounts of text have the same
size. The options |minimum height| and |minimum width| allow you to
specify the minimum height and width independently. 

So, what Hagen needs to do is to provide |minimum size| for the
nodes. To be on the safe side, he also sets |inner sep=0pt|. This
ensures that the nodes will really have size |minimum size| and not,
for very small minimum sizes, the minimal size necessary to encompass
the automatically added space.

\begin{codeexample}[]
\tikzstyle{place}=[circle,draw=blue!50,fill=blue!20,thick,
                   inner sep=0pt,minimum size=6mm]
\tikzstyle{transition}=[rectangle,draw=black!50,fill=black!20,thick,
                        inner sep=0pt,minimum size=4mm]
\begin{tikzpicture}
  \node at ( 0,2) [place] {};
  \node at ( 0,1) [place] {};   
  \node at ( 0,0) [place] {};   
  \node at ( 1,1) [transition] {};
  \node at (-1,1) [transition] {};    
\end{tikzpicture}
\end{codeexample}




\subsection{Naming Nodes}

Hagen's next aim is to connect the nodes using arrows. This seems like
a tricky business since the arrows should not start in the middle of
the nodes, but somewhere on the border and Hagen would very much like
to avoid computing these positions by hand.

Fortunately, \pgfname\ will perform all the necessary calculations for
him. However, he first has to assign names to the nodes so that he can
reference them later on.

There are two ways to name a node. The first is the use the |name=|
option. The second method is to write the desired name in parentheses
after the |node| operation. Hagen thinks that this second method seems
strange, but he will soon change his opinion.

{
\tikzstyle{place}=[circle,draw=blue!50,fill=blue!20,thick,
                   inner sep=0pt,minimum size=6mm]
\tikzstyle{transition}=[rectangle,draw=black!50,fill=black!20,thick,
                        inner sep=0pt,minimum size=4mm]
\begin{codeexample}[]
% ... setup styles
\begin{tikzpicture}
  \node (waiting 1)  at ( 0,2)     [place] {};
  \node (critical 1) at ( 0,1)     [place] {};   
  \node (semaphore)  at ( 0,0)     [place] {};   
  \node (leave critical) at ( 1,1) [transition] {};
  \node (enter critical) at (-1,1) [transition] {};    
\end{tikzpicture}
\end{codeexample}
}

Hagen is pleased to note that the names help in understanding the
code. Names for nodes can be pretty arbitrary, but they should not
contain commas, periods, parentheses, colons, and some other special
characters. However, they can contain underscores and hyphens. 

The syntax for the |node| operation is quite liberal with respect to
the order in which node names, the |at| specifier, and the options
must come. Indeed, you can even have multiple option blocks between
the |node| and the text in curly braces, they accumulate. You can
rearrange them arbitrarily and perhaps the following might be preferable:

{
\tikzstyle{place}=[circle,draw=blue!50,fill=blue!20,thick,
                   inner sep=0pt,minimum size=6mm]
\tikzstyle{transition}=[rectangle,draw=black!50,fill=black!20,thick,
                        inner sep=0pt,minimum size=4mm]
\begin{codeexample}[]
\begin{tikzpicture}
  \node[place]      (waiting 1)      at ( 0,2) {};
  \node[place]      (critical 1)     at ( 0,1) {};   
  \node[place]      (semaphore)      at ( 0,0) {};   
  \node[transition] (leave critical) at ( 1,1) {};
  \node[transition] (enter critical) at (-1,1) {};    
\end{tikzpicture}
\end{codeexample}
}



\subsection{Placing Nodes II}

Although Hagen still wishes to connect the nodes, he first wishes to
address another problem again: The placement of the nodes. Although he
likes the |at| syntax, in this particular case he would prefer placing
the nodes ``relative to each other.'' So, Hagen would like to say that
the |critical 1| node should be below the |waiting 1| node, wherever
the |waiting 1| node might be. There are different ways of achieving
this, but the nicest one in Hagen's case is the |below of| option:

{
\tikzstyle{place}=[circle,draw=blue!50,fill=blue!20,thick,
                   inner sep=0pt,minimum size=6mm]
\tikzstyle{transition}=[rectangle,draw=black!50,fill=black!20,thick,
                        inner sep=0pt,minimum size=4mm]
\begin{codeexample}[]
\begin{tikzpicture}
  \node[place]      (waiting)                            {};
  \node[place]      (critical)       [below of=waiting]  {};   
  \node[place]      (semaphore)      [below of=critical] {};   
  \node[transition] (leave critical) [right of=critical] {};
  \node[transition] (enter critical) [left of=critical]  {};    
\end{tikzpicture}
\end{codeexample}
}

The |below of| and similar options setup the position of the node in
such a manner that it is placed at the distance |node distance| in the
specified direction of the given direction. The |node distance| is the
distance between the centers of the nodes, not between the borders.

Even though the above code has the same effect the earlier code, Hagen
can pass it to his colleagues how will be able to just read and
understand it, perhaps without even having to see the picture.


\subsection{Connecting Nodes}

It is now high time to connect the nodes. Let us start with something
simple, namely with the straight line from |enter critical| to
|critical|. We want this line to start at the right side of
|enter critical| and to end at the left side of |critical|. For
this, we can use the \emph{anchors} of the nodes. Every node defines a
whole bunch of anchors that lie on its border or inside it. For
example, the |center| anchor is at the center of the node, the |west|
anchor is on the left of the node, and so on. To access the coordinate
of a node, we use a coordinate that contains the node's name followed
by a dot, followed by the anchor's name:

{
\tikzstyle{place}=[circle,draw=blue!50,fill=blue!20,thick,
                   inner sep=0pt,minimum size=6mm]
\tikzstyle{transition}=[rectangle,draw=black!50,fill=black!20,thick,
                        inner sep=0pt,minimum size=4mm]
\begin{codeexample}[]
\begin{tikzpicture}
  \node[place]      (waiting)                            {};
  \node[place]      (critical)       [below of=waiting]  {};   
  \node[place]      (semaphore)      [below of=critical] {};   
  \node[transition] (leave critical) [right of=critical] {};
  \node[transition] (enter critical) [left of=critical]  {};    
  \draw [->] (critical.west) -- (enter critical.east);
\end{tikzpicture}
\end{codeexample}
}

Next, let us tackle the curve from |waiting| to |enter critical|. This
can be specified using curves and controls:

{
\tikzstyle{place}=[circle,draw=blue!50,fill=blue!20,thick,
                   inner sep=0pt,minimum size=6mm]
\tikzstyle{transition}=[rectangle,draw=black!50,fill=black!20,thick,
                        inner sep=0pt,minimum size=4mm]
\begin{codeexample}[]
\begin{tikzpicture}
  \node[place]      (waiting)                            {};
  \node[place]      (critical)       [below of=waiting]  {};   
  \node[place]      (semaphore)      [below of=critical] {};   
  \node[transition] (leave critical) [right of=critical] {};
  \node[transition] (enter critical) [left of=critical]  {};    
  \draw [->] (enter critical.east) -- (critical.west);
  \draw [->] (waiting.west) .. controls +(left:5mm) and +(up:5mm)
                            .. (enter critical.north);
\end{tikzpicture}
\end{codeexample}
}

Hagen sees how he can now add all his edges, but the whole process
seems a but awkward and not very flexible. Again, the code seems to
obscure the structure of the graphic rather than showing it.

So, let us start improving the code for the edges. First, Hagen can
leave out the anchors:

{
\tikzstyle{place}=[circle,draw=blue!50,fill=blue!20,thick,
                   inner sep=0pt,minimum size=6mm]
\tikzstyle{transition}=[rectangle,draw=black!50,fill=black!20,thick,
                        inner sep=0pt,minimum size=4mm]
\begin{codeexample}[]
\begin{tikzpicture}
  \node[place]      (waiting)                            {};
  \node[place]      (critical)       [below of=waiting]  {};   
  \node[place]      (semaphore)      [below of=critical] {};   
  \node[transition] (leave critical) [right of=critical] {};
  \node[transition] (enter critical) [left of=critical]  {};    
  \draw [->] (enter critical) -- (critical);
  \draw [->] (waiting) .. controls +(left:8mm) and +(up:8mm)
                       .. (enter critical);
\end{tikzpicture}
\end{codeexample}
}

Hagen is a bit surprised that this works. After all, how did
\tikzname\ know that the line from |enter critical| to |critical|
should actually start on the borders? Whenever \tikzname\ encounters a
whole node name as a ``coordinate,'' it tries to ``be smart'' about
the anchor that it should choose for this node. Depending on what
happens next, \tikzname\ will choose an anchor that lies on the border
of the node on a line to the next coordinate or control point. The
exact rules are a bit complex, but the chosen point will usually be
correct -- and when it is not, Hagen can still specify the desired
anchor by hand.

Hagen would now like to simplify the curve operation somehow. It turns
out that this can be accomplished using a special path operation: the
|to| operation. This operation takes many options (you can even define
new ones yourself). One pair of option is useful for Hagen: The pair
|in| and |out|. These options take angles at which a curve should
leave or reach the start or target coordinates. Without these options,
a straight line is drawn:

{
\tikzstyle{place}=[circle,draw=blue!50,fill=blue!20,thick,
                   inner sep=0pt,minimum size=6mm]
\tikzstyle{transition}=[rectangle,draw=black!50,fill=black!20,thick,
                        inner sep=0pt,minimum size=4mm]
\begin{codeexample}[]
\begin{tikzpicture}
  \node[place]      (waiting)                            {};
  \node[place]      (critical)       [below of=waiting]  {};   
  \node[place]      (semaphore)      [below of=critical] {};   
  \node[transition] (leave critical) [right of=critical] {};
  \node[transition] (enter critical) [left of=critical]  {};    
  \draw [->] (enter critical) to                 (critical);
  \draw [->] (waiting)        to [out=180,in=90] (enter critical);
\end{tikzpicture}
\end{codeexample}
}

There is another option for the |to| operation, that is even better
suited to Hagen's problem: The |bend right| option. This option also
takes an angle, but this angle only specifies the angle by which the
curve is bend to the right:

{
\tikzstyle{place}=[circle,draw=blue!50,fill=blue!20,thick,
                   inner sep=0pt,minimum size=6mm]
\tikzstyle{transition}=[rectangle,draw=black!50,fill=black!20,thick,
                        inner sep=0pt,minimum size=4mm]
\begin{codeexample}[]
\begin{tikzpicture}
  \node[place]      (waiting)                            {};
  \node[place]      (critical)       [below of=waiting]  {};   
  \node[place]      (semaphore)      [below of=critical] {};   
  \node[transition] (leave critical) [right of=critical] {};
  \node[transition] (enter critical) [left of=critical]  {};    
  \draw [->] (enter critical) to                 (critical);
  \draw [->] (waiting)        to [bend right=45] (enter critical);
  \draw [->] (enter critical) to [bend right=45] (semaphore);
\end{tikzpicture}
\end{codeexample}
}

It is now time for Hagen to learn about yet another way of specifying
edges: Using the |edge| path operation. This operation is very similar
to the |to| operation, but there is one important difference: Like a
node the edge generated by the |edge| operation is not part of the
main path, but is added only later. This may not seem very important,
but is has some nice consequences. For example, every edge can have
its own arrow tips and its own color and so one and, still, all the
edges can be given on the same path. This allows Hagen to write the
following: 


{
\tikzstyle{place}=[circle,draw=blue!50,fill=blue!20,thick,
                   inner sep=0pt,minimum size=6mm]
\tikzstyle{transition}=[rectangle,draw=black!50,fill=black!20,thick,
                        inner sep=0pt,minimum size=4mm]
\begin{codeexample}[]
\begin{tikzpicture}
  \node[place]      (waiting)                            {};
  \node[place]      (critical)       [below of=waiting]  {};   
  \node[place]      (semaphore)      [below of=critical] {};   
  \node[transition] (leave critical) [right of=critical] {};
  \node[transition] (enter critical) [left of=critical]  {}
    edge [->]               (critical)
    edge [<-,bend left=45]  (waiting)
    edge [->,bend right=45] (semaphore);
\end{tikzpicture}
\end{codeexample}
}

Each |edge| caused a new path to be constructed, consisting of a |to|
between the node |enter critical| and the node following the |edge|
command.

The finishing touch is to introduce two styles |pre| and |post| and to
use the |bend angle=45| option to set the bend angle once and for all:

{
\tikzstyle{place}=[circle,draw=blue!50,fill=blue!20,thick,
                   inner sep=0pt,minimum size=6mm]
\tikzstyle{transition}=[rectangle,draw=black!50,fill=black!20,thick,
                        inner sep=0pt,minimum size=4mm]
\begin{codeexample}[]
% Styles place and transition as before
\tikzstyle{pre}=[<-,shorten <=1pt,>=stealth',semithick]  
\tikzstyle{post}=[->,shorten >=1pt,>=stealth',semithick]  
\begin{tikzpicture}[bend angle=45]
  \node[place]      (waiting)                            {};
  \node[place]      (critical)       [below of=waiting]  {};   
  \node[place]      (semaphore)      [below of=critical] {};   

  \node[transition] (leave critical) [right of=critical] {}
    edge [pre]             (critical)
    edge [post,bend right] (waiting)
    edge [pre, bend left]  (semaphore);
  \node[transition] (enter critical) [left of=critical]  {}
    edge [post]            (critical)
    edge [pre, bend left]  (waiting)
    edge [post,bend right] (semaphore);
\end{tikzpicture}
\end{codeexample}
}


\subsection{Placing Nodes III}

