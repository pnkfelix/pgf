% $Header: /cvsroot/pgf/pgf/doc/pgf/macros/Attic/pgfmanual-macros.tex,v 1.6 2005/07/05 21:39:05 tantau Exp $

% Copyright 2003, 2004 by Till Tantau <tantau@users.sourceforge.net>.
%
% This program can be redistributed and/or modified under the terms
% of the GNU Public License, version 2.




\def\Class#1{\hbox{\small#1}}
\def\bs{$\backslash$}

\def\Environment#1{\par\bigskip\noindent\textbf{Environment \texttt{#1}}\par}
\def\Command#1{\par\bigskip\noindent\textbf{Command \texttt{#1}}\par}
\long\def\Parameters#1{\medskip\noindent Parameters:
  \begin{enumerate}\itemsep=0pt\parskip=0pt
    #1
  \end{enumerate}}
\long\def\Description#1{\unskip\medskip\noindent Description: #1}
\def\Example{\par\medskip\noindent Example: }


%\renewcommand*\descriptionlabel[1]{\hspace\labelsep\normalfont #1}


\def\pgflayout#1{\list{}{\leftmargin=2em\itemindent-\leftmargin\def\makelabel##1{\hss##1}}%
\item\strut\texttt{\string\pgfpagesuselayout\char`\{\declare{#1}\char`\}}\oarg{options}\par\topsep=0pt%
  \index{#1@\protect\texttt{#1} layout}%
  \index{Page layouts!#1@\protect\texttt{#1}}%
}
\def\endpgflayout{\endlist}
  


\newcommand\opt[1]{{\color{black!50!green}#1}}
\newcommand\ooarg[1]{{\ttfamily[}\meta{#1}{\ttfamily]}}

\def\opt{\afterassignment\pgfmanualopt\let\next=}
\def\pgfmanualopt{\ifx\next\bgroup\bgroup\color{black!50!green}\else{\color{black!50!green}\next}\fi}



\def\beamer{\textsc{beamer}}
\def\pdf{\textsc{pdf}}
\def\pgfname{\textsc{pgf}}
\def\tikzname{Ti\emph{k}Z}
\def\pstricks{\textsc{pstricks}}
\def\prosper{\textsc{prosper}}
\def\seminar{\textsc{seminar}}
\def\texpower{\textsc{texpower}}
\def\foils{\textsc{foils}}

{
  \makeatletter
  \global\let\myempty=\@empty
  \global\let\mygobble=\@gobble
  \catcode`\@=12
  \gdef\getridofats#1@#2\relax{%
    \def\getridtest{#2}%
    \ifx\getridtest\myempty%
      \expandafter\def\expandafter\strippedat\expandafter{\strippedat#1}
    \else%
      \expandafter\def\expandafter\strippedat\expandafter{\strippedat#1\protect\printanat}
      \getridofats#2\relax%
    \fi%
  }

  \gdef\removeats#1{%
    \let\strippedat\myempty%
    \edef\strippedtext{\stripcommand#1}%
    \expandafter\getridofats\strippedtext @\relax%
  }
  
  \gdef\stripcommand#1{\expandafter\mygobble\string#1}
}

\def\printanat{\char`\@}

\def\declare{\afterassignment\pgfmanualdeclare\let\next=}
\def\pgfmanualdeclare{\ifx\next\bgroup\bgroup\color{red!75!black}\else{\color{red!75!black}\next}\fi}

\def\command#1{\list{}{\leftmargin=2em\itemindent-\leftmargin\def\makelabel##1{\hss##1}}%
\item\extractcommand#1\@@\par\topsep=0pt}
\def\endcommand{\endlist}
\def\extractcommand#1#2\@@{\strut\declare{\texttt{\string#1}}#2%
  \removeats{#1}%
  \index{\strippedat @\protect\myprintocmmand{\strippedat}}}

\let\textoken=\command
\let\endtextoken=\endcommand

\def\myprintocmmand#1{\texttt{\char`\\#1}}

\def\example{\par\smallskip\noindent\textit{Example: }}
\def\themeauthor{\par\smallskip\noindent\textit{Theme author: }}

\def\environment#1{\list{}{\leftmargin=2em\itemindent-\leftmargin\def\makelabel##1{\hss##1}}%
\extractenvironement#1\@@\par\topsep=0pt}
\def\endenvironment{\endlist}
\def\extractenvironement#1#2\@@{%
\item{{\ttfamily\char`\\begin\char`\{\declare{#1}\char`\}}#2}%
  {\itemsep=0pt\parskip=0pt\item{\meta{environment contents}}%
  \item{\ttfamily\char`\\end\char`\{\declare{#1}\char`\}}}%
  \index{#1@\protect\texttt{#1} environment}%
  \index{Environments!#1@\protect\texttt{#1}}}

\def\plainenvironment#1{\list{}{\leftmargin=2em\itemindent-\leftmargin\def\makelabel##1{\hss##1}}%
\extractplainenvironment#1\@@\par\topsep=0pt}
\def\endplainenvironment{\endlist}
\def\extractplainenvironment#1#2\@@{%
\item{{\ttfamily\declare{\char`\\#1}}#2}%
  {\itemsep=0pt\parskip=0pt\item{\meta{environment contents}}%
  \item{\ttfamily\declare{\char`\\end#1}}}%
  \index{#1@\protect\texttt{#1} environment}%
  \index{Environments!#1@\protect\texttt{#1}}}

\def\shape#1{\list{}{\leftmargin=2em\itemindent-\leftmargin\def\makelabel##1{\hss##1}}%
\extractshape#1\@@\par\topsep=0pt}
\def\endshape{\endlist}
\def\extractshape#1\@@{%
\item{Shape {\ttfamily\declare{#1}}}%
  \index{#1@\protect\texttt{#1} shape}%
  \index{Shapes!#1@\protect\texttt{#1}}}

\def\package#1{\list{}{\leftmargin=2em\itemindent-\leftmargin\def\makelabel##1{\hss##1}}%
\item{{\ttfamily\char`\\usepackage\char`\{\declare{#1}\char`\}\space\space \char`\%\space\space  \LaTeX}}
  \index{#1@\protect\texttt{#1} package}%
  \index{Packages and files!#1@\protect\texttt{#1}}%
  \par\topsep=0pt\itemsep=0pt
\item{{\ttfamily\char`\\input \declare{#1}.tex\space\space\space \char`\%\space\space  plain \TeX}}
  \par\topsep=0pt\itemsep=0pt
\item{{\ttfamily\char`\\input \declare{#1}.tex\space\space\space \char`\%\space\space  Con\TeX t}}
  \par\topsep=0pt
}
\def\endpackage{\endlist}

\def\filedescription#1{\list{}{\leftmargin=2em\itemindent-\leftmargin\def\makelabel##1{\hss##1}}%
\item{File {\ttfamily\declare{#1}}}
  \index{#1@\protect\texttt{#1} file}%
  \index{Packages and files!#1@\protect\texttt{#1}}%
  \par\topsep=0pt
}
\def\endfiledescription{\endlist}


\def\packageoption#1{\list{}{\leftmargin=2em\itemindent-\leftmargin\def\makelabel##1{\hss##1}}%
\item{{\ttfamily\char`\\usepackage[\declare{#1}]\char`\{pgf\char`\}}}
  \index{#1@\protect\texttt{#1} package option}%
  \index{Package options for \textsc{pgf}!#1@\protect\texttt{#1}}%
  \par\topsep=0pt}
\def\endpackageoption{\endlist}

\def\itemoption#1{\item \declare{\texttt{#1}}%
  \indexoption{#1}%
}

\def\indexoption#1{%
  \index{#1@\protect\texttt{#1} option}%
  \index{Graphic options!#1@\protect\texttt{#1}}%
}

\def\itemstyle#1{\item \texttt{style=}\declare{\texttt{#1}}%
  \index{#1@\protect\texttt{#1} style}%
  \index{Styles!#1@\protect\texttt{#1}}%
}



\def\class#1{\list{}{\leftmargin=2em\itemindent-\leftmargin\def\makelabel##1{\hss##1}}%
\extractclass#1@\par\topsep=0pt}
\def\endclass{\endlist}
\def\extractclass#1#2@{%
\item{{{\ttfamily\char`\\documentclass}#2{\ttfamily\char`\{\declare{#1}\char`\}}}}%
  \index{#1@\protect\texttt{#1} class}%
  \index{Classes!#1@\protect\texttt{#1}}}

\def\partname{Part}

\makeatletter
\def\index@prologue{\section*{Index}\addcontentsline{toc}{section}{Index}
  This index only contains automatically generated entries. A good
  index should also contain carefully selected keywords. This index is
  not a good index.
  \bigskip
}
\c@IndexColumns=2
  \def\theindex{\@restonecoltrue
    \columnseprule \z@  \columnsep 35\p@
    \twocolumn[\index@prologue]%
       \parindent -30pt
       \columnsep 15pt
       \parskip 0pt plus 1pt
       \leftskip 30pt
       \rightskip 0pt plus 2cm
       \small
       \def\@idxitem{\par}%
    \let\item\@idxitem \ignorespaces}
  \def\endtheindex{\onecolumn}
\def\noindexing{\let\index=\@gobble}


\newcommand\symarrow[1]{
  \index{#1@\protect\texttt{#1} arrow tip}%
  \index{Arrow tips!#1@\protect\texttt{#1}}
  \texttt{#1}& yields thick  
  \begin{tikzpicture}[arrows={#1-#1},thick]
    \useasboundingbox (0pt,-0.5ex) rectangle (1cm,2ex);
    \draw (0,0) -- (1,0);
  \end{tikzpicture} and thin
  \begin{tikzpicture}[arrows={#1-#1},thin]
    \useasboundingbox (0pt,-0.5ex) rectangle (1cm,2ex);
    \draw (0,0) -- (1,0);
  \end{tikzpicture}
}

\newcommand\sarrow[2]{
  \index{#1@\protect\texttt{#1} arrow tip}%
  \index{Arrow tips!#1@\protect\texttt{#1}}
  \index{#2@\protect\texttt{#2} arrow tip}%
  \index{Arrow tips!#2@\protect\texttt{#2}}
  \texttt{#1-#2}& yields thick  
  \begin{tikzpicture}[arrows={#1-#2},thick]
    \useasboundingbox (0pt,-0.5ex) rectangle (1cm,2ex);
    \draw (0,0) -- (1,0);
  \end{tikzpicture} and thin
  \begin{tikzpicture}[arrows={#1-#2},thin]
    \useasboundingbox (0pt,-0.5ex) rectangle (1cm,2ex);
    \draw (0,0) -- (1,0);
  \end{tikzpicture}
}

\newcommand\carrow[1]{
  \index{#1@\protect\texttt{#1} arrow tip}%
  \index{Arrow tips!#1@\protect\texttt{#1}}
  \texttt{#1}& yields for line width 1ex
  \begin{tikzpicture}[arrows={#1-#1},line width=1ex]
    \useasboundingbox (0pt,-0.5ex) rectangle (1.5cm,2ex);
    \draw (0,0) -- (1.5,0);
  \end{tikzpicture}
}
\def\myvbar{\char`\|}
\newcommand\plotmarkentry[1]{%
  \index{#1@\protect\texttt{#1} plot mark}%
  \index{Plot marks!#1@\protect\texttt{#1}}
  \texttt{\char`\\pgfuseplotmark\char`\{\declare{#1}\char`\}} &
  \tikz\draw[color=black!25] plot[mark=#1,fill=yellow,draw=black] coordinates{(0,0) (.5,0.2) (1,0) (1.5,0.2)};\\
}
\newcommand\plotmarkentrytikz[1]{%
  \index{#1@\protect\texttt{#1} plot mark}%
  \index{Plot marks!#1@\protect\texttt{#1}}
  \texttt{mark=\declare{#1}} & \tikz\draw[color=black!25]
  plot[mark=#1,fill=yellow,draw=black] 
    coordinates {(0,0) (.5,0.2) (1,0) (1.5,0.2)};\\
}


\colorlet{graphicbackground}{yellow!80!black!20}
\colorlet{codebackground}{blue!20}


\ifx\scantokens\@undefined
  \PackageError{pgfmanual-macros}{You need to use extended latex
    (elatex) or (pdfelatex) to process this document}{}
\fi

\begingroup
\catcode`|=0
\catcode`[= 1
\catcode`]=2
\catcode`\{=12
\catcode `\}=12
\catcode`\\=12 |gdef|find@example#1\end{codeexample}[|endofcodeexample[#1]]
|endgroup

\begingroup
\catcode`\^=7
\catcode`\^^M=13
\catcode`\ =13%
\gdef\returntospace{\catcode`\ =13\def {\space}\catcode`\^^M=13\def^^M{}}%
\endgroup

\begingroup
\catcode`\%=13
\catcode`\^^M=13
\gdef\commenthandler{\catcode`\%=13\def%{\@gobble@till@return}}
\gdef\@gobble@till@return#1^^M{}
\gdef\typesetcomment{\catcode`\%=13\def%{\@typeset@till@return}}
\gdef\@typeset@till@return#1^^M{{\def%{\char`\%}\textsl{\char`\%#1}}\par}
\endgroup

\define@key{codeexample}{width}{\setlength\codeexamplewidth{#1}}
\define@key{codeexample}{graphic}{\colorlet{graphicbackground}{#1}}
\define@key{codeexample}{code}{\colorlet{codebackground}{#1}}
\define@key{codeexample}{execute code}{\csname code@execute#1\endcsname}
\define@key{codeexample}{code only}[]{\code@executefalse}

\newdimen\codeexamplewidth
\setlength\codeexamplewidth{4cm+7pt}
\newif\ifcode@execute
\newbox\codeexamplebox
\def\codeexample[#1]{%
  \code@executetrue
  \setkeys{codeexample}{#1}%
  \parindent0pt
  \begingroup%
  \par%
  \medskip%
  \let\do\@makeother%
  \dospecials%
  \obeylines%
  \@vobeyspaces%
  \catcode`\%=13%
  \catcode`\^^M=13%
  \find@example}
\def\endofcodeexample#1{%
  \endgroup%
  \ifcode@execute%
    \setbox\codeexamplebox=\hbox{%
      {%
        {%
          \returntospace%
          \commenthandler%
          \xdef\code@temp{#1}% removes returns and comments
        }%
        \colorbox{graphicbackground}{\ignorespaces%
          \expandafter\scantokens\expandafter{\code@temp\ignorespaces}\ignorespaces}%
      }%
    }%
    \ifdim\wd\codeexamplebox>\codeexamplewidth%
      \def\code@start{\par}%
      \def\code@flushstart{}\def\code@flushend{}%
      \def\code@mid{\parskip2pt\par\noindent}%
      \def\code@width{\linewidth-6pt}%
      \def\code@end{}%
    \else%
      \def\code@start{%
        \linewidth=\textwidth%
        \parshape \@ne 0pt \linewidth
        \leavevmode%
        \hbox\bgroup}%
      \def\code@flushstart{\hfill}%
      \def\code@flushend{\hbox{}}%
      \def\code@mid{\hskip6pt}%
      \def\code@width{\linewidth-12pt-\codeexamplewidth}%
      \def\code@end{\egroup}%
    \fi%
    \code@start%
    \noindent%
    \begin{minipage}[t]{\codeexamplewidth}\raggedright
      \hrule width0pt%
      \footnotesize\vskip-1em%
      \code@flushstart\box\codeexamplebox\code@flushend%
      \vskip-1ex
      \leavevmode%
    \end{minipage}%
  \else%
    \def\code@mid{\par}
    \def\code@width{\linewidth-6pt}
    \def\code@end{}
  \fi%
  \code@mid%  
  \colorbox{codebackground}{%
    \begin{minipage}[t]{\code@width}%
      {%
        \let\do\@makeother
        \dospecials
        \frenchspacing\@vobeyspaces
        \normalfont\ttfamily\footnotesize
        \typesetcomment%
        \@tempswafalse
        \def\par{%
          \if@tempswa
          \leavevmode \null \@@par\penalty\interlinepenalty
          \else
          \@tempswatrue
          \ifhmode\@@par\penalty\interlinepenalty\fi
          \fi}%
        \obeylines
        \everypar \expandafter{\the\everypar \unpenalty}%
        #1}
    \end{minipage}}%
  \code@end%
  \par%
  \medskip
  \endgroup
}



\makeatother


%%% Local Variables: 
%%% mode: latex
%%% TeX-master: "beameruserguide"
%%% End: 
