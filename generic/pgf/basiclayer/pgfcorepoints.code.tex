% Copyright 2006 by Till Tantau
%
% This file may be distributed and/or modified
%
% 1. under the LaTeX Project Public License and/or
% 2. under the GNU Public License.
%
% See the file doc/generic/pgf/licenses/LICENSE for more details.

\ProvidesFileRCS $Header: /cvsroot/pgf/pgf/generic/pgf/basiclayer/pgfcorepoints.code.tex,v 1.11 2007/08/20 12:44:06 vibrovski Exp $

\newdimen\pgf@picminx
\newdimen\pgf@picmaxx
\newdimen\pgf@picminy
\newdimen\pgf@picmaxy

\newdimen\pgf@pathminx
\newdimen\pgf@pathmaxx
\newdimen\pgf@pathminy
\newdimen\pgf@pathmaxy

\newif\ifpgf@relevantforpicturesize

\def\pgf@process#1{{#1\global\pgf@x=\pgf@x\global\pgf@y=\pgf@y}}

% Save a point.
%
% #1 = macro for storing point.
% #2 = code for point (should define x and y)
%
% Example:
%
% \pgfsavepgf@process\mypoint{\pgf@x=10pt \pgf@y10pt}
% \pgfsavepgf@process\myarcpoint{\pgfpointpolar{30}{5cm and 2cm}}

\def\pgfsavepgf@process#1#2{%
	\pgf@process{#2}%
	\edef#1{\noexpand\pgf@x\the\pgf@x\noexpand\pgf@y\the\pgf@y}%
}


% Return a point
%  
% #1 = x-coordinate of the point 
% #2 = y-coordinate of the point 
%  
% x = #1
% y = #2
%
% Example:
%
% \pgfpathmoveto{\pgfpoint{2pt+3cm}{3cm}}

\def\pgfpoint#1#2{%
  \pgfmathsetlength\pgf@x{#1}%
  \pgfmathsetlength\pgf@y{#2}\ignorespaces}


% Quickly a point
%  
% #1 = x-coordinate of the point (no calculations done)
% #2 = y-coordinate of the point (no calculations done)
%  
% x = #1
% y = #2
%
% Example:
%
% \pgfpathmoveto{\pgfqpoint{2pt}{3cm}}

\def\pgfqpoint#1#2{\pgf@x=#1\pgf@y=#2}



% Return the origin.
%  
% x = 0
% y = 0
%
% Example:
%
% \pgfpathmoveto{\pgfpointorigin}

\def\pgfpointorigin{\pgf@x=0pt\pgf@y=\pgf@x\ignorespaces}



% Return a transformed point
%  
% #1 = a point
%
% Description:
%
% This command applies pgf's current transformation matrix to the
% given point. Normally, this is done automatically by commands like
% lineto or moveto, but sometimes you may wish to access a transformed
% point yourself. In the below example, this command is used for a low level
% coordinate system shift.
%
% In addition to applying the transformation matrix to the given
% point, this function also applies the pre- and post-morphing
% macros. 
%
% Example:
%
% \begin{pgflowleveltransformshiftscope}{\pgfpointtransformed{\pgfpointorigin}}
%   \pgfbox[center,center]{Hi!}
% \end{pgflowleveltransformshiftscope}

\def\pgfpointtransformed#1{%
  \pgf@process{%
    #1%
    \pgfmorph@prelist%
    \pgf@pos@transform{\pgf@x}{\pgf@y}%
    \pgfmorph@postlist%
  }%
}


% Return the difference vector of two points.
%  
% #1 = start of vector  
% #2 = end of vector  
%  
% x = x-component of difference 
% y = y-component of difference 
%
% Example:
%
% \pgfpathmoveto{\pgfpointdiff{\pgfpointxy{1}{1}}{\pgfpointxy{2}{3}}}

\def\pgfpointdiff#1#2{%
  \pgf@process{#1}%
  \pgf@xa=\pgf@x%
  \pgf@ya=\pgf@y%
  \pgf@process{#2}%
  \advance\pgf@x by-\pgf@xa\relax%
  \advance\pgf@y by-\pgf@ya\relax\ignorespaces}

% Add two vectors. 
%  
% #1 = first vector  
% #2 = second vector  
%  
% x = x-component of addition
% y = y-component of addition
%
% Example:
%
% \pgfpathmoveto{\pgfpointadd{\pgfpointxy{0}{1}}{\pgfpointxy{2}{3}}}

\def\pgfpointadd#1#2{%
  \pgf@process{#1}%
  \pgf@xa=\pgf@x%
  \pgf@ya=\pgf@y%
  \pgf@process{#2}%
  \advance\pgf@x by\pgf@xa%
  \advance\pgf@y by\pgf@ya}



% Multiply a vector by a factor.
%  
% #1 = factor  
% #2 = vector  
%
% Example:
%
% \pgfpointscale{2}{\pgfpointxy{0}{1}}

\def\pgfpointscale#1#2{%
  \pgf@process{#2}%
  \pgf@x=#1\pgf@x%
  \pgf@y=#1\pgf@y%
}


% The intersection of two lines
%  
% #1 = point on first line
% #2 = another point on first line
% #3 = point on second line
% #4 = another point on second line
%
% Returns the intersection of the two lines. If there is no
% intersection or if the points #1 and #2 or the points #3 and #4 are
% identical, the behaviour is not specified.
%
% Example:
%
% \pgfpointintersectionoflines{\pgfpointxy{0}{1}}{\pgfpointxy{1}{0}}{\pgfpointxy{2}{2}}{\pgfpointxy{3}{4}}

\def\pgfpointintersectionoflines#1#2#3#4{%
  {
    % 
    % Compute orthogonal vector to #1--#2
    % 
    \pgf@process{#2}%
    \pgf@xa=\pgf@x%
    \pgf@ya=\pgf@y%
    \pgf@process{#1}%
    \advance\pgf@xa by-\pgf@x%
    \advance\pgf@ya by-\pgf@y%
    \pgf@ya=-\pgf@ya%
    % Normalise a bit
    \c@pgf@counta=\pgf@xa%
    \ifnum\c@pgf@counta<0\relax%
      \c@pgf@counta=-\c@pgf@counta\relax%
    \fi%
    \c@pgf@countb=\pgf@ya%
    \ifnum\c@pgf@countb<0\relax%
      \c@pgf@countb=-\c@pgf@countb\relax%
    \fi%
    \advance\c@pgf@counta by\c@pgf@countb\relax%
    \divide\c@pgf@counta by 65536\relax%
    \ifnum\c@pgf@counta>0\relax%
      \divide\pgf@xa by\c@pgf@counta\relax%
      \divide\pgf@ya by\c@pgf@counta\relax%
    \fi%
    %
    % Compute projection
    %
    \pgf@xc=\pgf@sys@tonumber{\pgf@ya}\pgf@x%
    \advance\pgf@xc by\pgf@sys@tonumber{\pgf@xa}\pgf@y%
    % 
    % The orthogonal vector is (\pgf@ya,\pgf@xa)
    % 
    % 
    % Compute orthogonal vector to #3--#4
    % 
    \pgf@process{#4}%
    \pgf@xb=\pgf@x%
    \pgf@yb=\pgf@y%
    \pgf@process{#3}%
    \advance\pgf@xb by-\pgf@x%
    \advance\pgf@yb by-\pgf@y%
    \pgf@yb=-\pgf@yb%
    % Normalise a bit
    \c@pgf@counta=\pgf@xb%
    \ifnum\c@pgf@counta<0\relax%
      \c@pgf@counta=-\c@pgf@counta\relax%
    \fi%
    \c@pgf@countb=\pgf@yb%
    \ifnum\c@pgf@countb<0\relax%
      \c@pgf@countb=-\c@pgf@countb\relax%
    \fi%
    \advance\c@pgf@counta by\c@pgf@countb\relax%
    \divide\c@pgf@counta by 65536\relax%
    \ifnum\c@pgf@counta>0\relax%
      \divide\pgf@xb by\c@pgf@counta\relax%
      \divide\pgf@yb by\c@pgf@counta\relax%
    \fi%
    %
    % Compute projection
    %
    \pgf@yc=\pgf@sys@tonumber{\pgf@yb}\pgf@x%
    \advance\pgf@yc by\pgf@sys@tonumber{\pgf@xb}\pgf@y%
    % 
    % The orthogonal vector is (\pgf@yb,\pgf@xb)
    % 
    % Setup transformation matrx (this is just to use the matrix
    % inversion)
    % 
    \pgfsettransform{{\pgf@sys@tonumber\pgf@ya}{\pgf@sys@tonumber\pgf@yb}{\pgf@sys@tonumber\pgf@xa}{\pgf@sys@tonumber\pgf@xb}{0pt}{0pt}}%
    \pgftransforminvert%
    \pgf@process{\pgfpointtransformed{\pgfpoint{\pgf@xc}{\pgf@yc}}}%
  }%
}


% Returns point on a line from #2 to #3 at time #1.
%  
% #1 = a time, where 0 is the start and 1 is the end
% #2 = start point
% #3 = end point 
%  
% x = x-component of #1*start + (1-#1)*end
% y = y-component of #1*start + (1-#1)*end 
%
% Example:
%
% % Middle of (1,1) and (2,3)
% \pgfpathmoveto{\pgfpointlineattime{0.5}{\pgfpointxy{0}{1}}{\pgfpointxy{2}{3}}}

\def\pgfpointlineattime#1#2#3{%
  \pgf@process{#3}%
  \pgf@xa=\pgf@x%
  \pgf@ya=\pgf@y%
  \pgf@process{#2}%
  \advance\pgf@xa by-\pgf@x\relax%
  \advance\pgf@ya by-\pgf@y\relax%
  \advance\pgf@x by #1\pgf@xa\relax%
  \advance\pgf@y by #1\pgf@ya\relax%  
  \ignorespaces}


% Move point #2 #1 many units in the direction of #3.
%  
% #1 = a distance
% #2 = start point
% #3 = end point 
%  
% x = x-component of start + #1*(normalise(end-start))
% y = y-component of start + #1*(normalise(end-start))
% xa = #1*(normalise(end-start))
% ya = #1*(normalise(end-start))
%
% Example:
%
%
% \pgfpathmoveto{\pgfpointlineatdistance{2pt}{\pgfpointxy{0}{1}}{\pgfpointxy{2}{3}}}
% \pgfpathlineto{\pgfpointlineatdistance{3pt}{\pgfpointxy{2}{3}}{\pgfpointxy{0}{1}}}

\def\pgfpointlineatdistance#1#2#3{%
  \pgf@process{#2}%
  \pgf@xb=\pgf@x\relax% xb/yb = start point
  \pgf@yb=\pgf@y\relax%
  \pgf@process{#3}%
  \advance\pgf@x by-\pgf@xb\relax%
  \advance\pgf@y by-\pgf@yb\relax%
  \pgf@process{\pgfpointnormalised{}}% x/y = normalised vector
  \pgfmathsetlength\pgf@xa{#1}%
  \pgf@ya=\pgf@xa\relax%
  \pgf@xa=\pgf@sys@tonumber{\pgf@x}\pgf@xa%
  \pgf@ya=\pgf@sys@tonumber{\pgf@y}\pgf@ya%
  \pgf@x=\pgf@xb\relax%
  \pgf@y=\pgf@yb\relax%
  \advance\pgf@x by\pgf@xa\relax%
  \advance\pgf@y by\pgf@ya\relax%
  \ignorespaces}


% Returns point on a curve from #2 to #5 with controls #3 and #4 at time #1.
%  
% #1 = a time
% #2 = start point
% #3 = first control point
% #4 = second control point
% #5 = end point 
%  
% x = x-component of place on the curve at time t
% y = y-component of place on the curve at time t
%
% Additionally, (\pgf@xa,\pgf@ya) and (\pgf@xb,\pgf@yb) will be on a
% tangent to the point on the curve (this can be useful for computing
% a label rotation).
%
% Example:
%
% % Middle of (1,1) and (2,3)
% \pgfpathmoveto{\pgfpointcurveattime{0.5}{\pgfpointxy{0}{1}}{\pgfpointxy{1}{1}}{\pgfpointxy{1}{1}}{\pgfpointxy{2}{3}}}

\def\pgfpointcurveattime#1#2#3#4#5{%
  \pgfmathparse{#1}%
  \let\pgf@time@s=\pgfmathresult%
  \pgf@x=\pgfmathresult pt%
  \pgf@x=-\pgf@x%
  \advance\pgf@x by 1pt%
  \edef\pgf@time@t{\pgf@sys@tonumber{\pgf@x}}%
  \pgf@process{#5}%
  \pgf@xc=\pgf@x%
  \pgf@yc=\pgf@y%
  \pgf@process{#4}%
  \pgf@xb=\pgf@x%
  \pgf@yb=\pgf@y%
  \pgf@process{#3}%
  \pgf@xa=\pgf@x%
  \pgf@ya=\pgf@y%
  \pgf@process{#2}%
  % First iteration:
  \pgf@x=\pgf@time@t\pgf@x\advance\pgf@x by\pgf@time@s\pgf@xa%
  \pgf@y=\pgf@time@t\pgf@y\advance\pgf@y by\pgf@time@s\pgf@ya%
  \pgf@xa=\pgf@time@t\pgf@xa\advance\pgf@xa by\pgf@time@s\pgf@xb%
  \pgf@ya=\pgf@time@t\pgf@ya\advance\pgf@ya by\pgf@time@s\pgf@yb%
  \pgf@xb=\pgf@time@t\pgf@xb\advance\pgf@xb by\pgf@time@s\pgf@xc%
  \pgf@yb=\pgf@time@t\pgf@yb\advance\pgf@yb by\pgf@time@s\pgf@yc%
  % Second iteration:
  \pgf@x=\pgf@time@t\pgf@x\advance\pgf@x by\pgf@time@s\pgf@xa%
  \pgf@y=\pgf@time@t\pgf@y\advance\pgf@y by\pgf@time@s\pgf@ya%
  \pgf@xa=\pgf@time@t\pgf@xa\advance\pgf@xa by\pgf@time@s\pgf@xb%
  \pgf@ya=\pgf@time@t\pgf@ya\advance\pgf@ya by\pgf@time@s\pgf@yb%
  % Save x/y
  \pgf@xb=\pgf@x%
  \pgf@yb=\pgf@y%
  % Third iteration:
  \pgf@x=\pgf@time@t\pgf@x\advance\pgf@x by\pgf@time@s\pgf@xa%
  \pgf@y=\pgf@time@t\pgf@y\advance\pgf@y by\pgf@time@s\pgf@ya%
}






% Internal registers
\newdimen\pgf@xx
\newdimen\pgf@xy
\newdimen\pgf@yx
\newdimen\pgf@yy
\newdimen\pgf@zx
\newdimen\pgf@zy



% A polar coordinate
%
% #1 = a degree
% #2 = a radius -- either a dimension or two dimensions separated by 
%      " and ". 
%
% x = (first dimension in #2) * cos(#1)
% y = (second dimension in #2) * sin(#2)
% 
% Example:
%
% \pgfpathmoveto{\pgfpointpolar{30}{1cm}}
% \pgfpathlineto{\pgfpointpolar{30}{1cm and 2cm}}

\def\pgfpointpolar#1#2{%
  \pgfutil@in@{and }{#2}%
  \ifpgfutil@in@%
    \pgf@polar@#2\@@%
  \else%
    \pgf@polar@#2 and #2\@@%
  \fi%  
  \pgfmathparse{#1}%
  \let\pgfpoint@angle=\pgfmathresult%
  \pgfmathcos@{\pgfpoint@angle}%
  \pgf@x=\pgfmathresult\pgf@x%
  \pgfmathsin@{\pgfpoint@angle}%
  \pgf@y=\pgfmathresult\pgf@y%
}

\def\pgf@polar@#1and #2\@@{%
  \pgfmathsetlength{\pgf@y}{#2}%
  \pgfmathsetlength{\pgf@x}{#1}%  
}

% Quick version of the polar coordinate method

\def\pgfqpointpolar#1#2{%
  \pgf@x=#2%
  \pgf@y=\pgf@x%
  \pgfmathcos@{#1}%
  \pgf@x=\pgfmathresult\pgf@x%
  \pgfmathsin@{#1}%
  \pgf@y=\pgfmathresult\pgf@y\relax%
}




% A polar coordinate in the xy plane.
%
% #1 = a degree
% #2 = a radius given as a number or two radi
%
% result = (first dim in #2) * x-vector * cos(#1) +
%          (second dim in #2) * y-vector * sin(#1)
% 
% Example:
%
% \pgfpathmoveto{\pgfpointpolarxy{30}{2}}

\def\pgfpointpolarxy#1#2{%
  \pgfutil@in@{and }{#2}%
  \ifpgfutil@in@%
    \pgf@polarxy@#2\@@%
  \else%
    \pgf@polarxy@#2and #2\@@%
  \fi%  
  \pgfmathparse{#1}%
  \let\pgfpoint@angle=\pgfmathresult%
  \pgfmathcos@{\pgfpoint@angle}%
  \pgf@xa=\pgfmathresult\pgf@xa%
  \pgfmathsin@{\pgfpoint@angle}%
  \pgf@ya=\pgfmathresult\pgf@ya%
  \pgf@x=\pgf@sys@tonumber{\pgf@xa}\pgf@xx%
  \advance\pgf@x by \pgf@sys@tonumber{\pgf@ya}\pgf@yx%
  \pgf@y=\pgf@sys@tonumber{\pgf@xa}\pgf@xy%
  \advance\pgf@y by \pgf@sys@tonumber{\pgf@ya}\pgf@yy}

\def\pgf@polarxy@#1and #2\@@{%
  \pgfmathsetlength{\pgf@xa}{#1}%
  \pgfmathsetlength{\pgf@ya}{#2}%
}



% A cylindrical coordinate.
%
% #1 = a degree
% #2 = a radius given as a number
% #3 = a height given as a number
%
% result = #2*(x-vector * cos(#1) + y-vector * sin(#1)) + #3*z-vector
% 
% Example:
%
% \pgfpathmoveto{\pgfpointcylindrical{30}{2}{1}}

\def\pgfpointcylindrical#1#2#3{%
  \pgfpointpolarxy{#1}{#2}%
  \pgfmathparse{#3}%
  \advance\pgf@x by \pgfmathresult\pgf@zx%
  \advance\pgf@y by \pgfmathresult\pgf@zy}


% A spherical coordinate.
%
% #1 = a longitude
% #2 = a latitude
% #3 = a radius
%
% result = #3*(cos(#2)*(x-vector * cos(#1) + y-vector * sin(#1)) + sin(#2)*z-vector)
% 
% Example:
%
% \pgfpathmoveto{\pgfpointspherical{30}{30}{2}}

\def\pgfpointspherical#1#2#3{%
  \pgfmathparse{#1}%
  \let\pgfpoint@angle=\pgfmathresult%
  \pgfmathsin@{\pgfpoint@angle}%
  \pgf@xb=\pgfmathresult\pgf@xx%
  \pgf@yb=\pgfmathresult\pgf@xy%
  \pgfmathcos@{\pgfpoint@angle}%
  \advance\pgf@xb by \pgfmathresult\pgf@yx%
  \advance\pgf@yb by \pgfmathresult\pgf@yy%
  %
  \pgfmathparse{#2}%
  \let\pgfpoint@angle=\pgfmathresult%
  \pgfmathcos@{\pgfpoint@angle}%
  \pgf@xc=\pgfmathresult\pgf@xb%
  \pgf@yc=\pgfmathresult\pgf@yb%
  \pgfmathsin@{\pgfpoint@angle}%
  \advance\pgf@xc by \pgfmathresult\pgf@zx%
  \advance\pgf@yc by \pgfmathresult\pgf@zy%
  \pgfmathparse{#3}%
  \pgf@x=\pgfmathresult\pgf@xc%
  \pgf@y=\pgfmathresult\pgf@yc\relax%
}


% Store the vector #1 * x-vec + #2 * y-vec
%
% #1 = a factor for the x-vector
% #2 = a factor fot the y-vector
%
% x = x-component of result vector
% y = y-component of result vector
%
% Description:
% 
% This command can be used to create a new coordinate system
% without using the rotate/translate/scale commands. This
% may be useful, if you do not want arrows and line width to
% be scaled/transformed together with the coordinate system.
% 
% Example:
%
% % Create a slanted rectangle
%
% \pgfsetxvec{\pgfpoint{1cm}{1cm}}
% \pgfsetyvec{\pgfpoint{0cm}{1cm}}
% 
% \pgfpathmoveto{\pgfpointxy{0}{0}}
% \pgfpathlineto{\pgfpointxy{1}{0}}
% \pgfpathlineto{\pgfpointxy{1}{1}}
% \pgfpathlineto{\pgfpointxy{0}{1}}
% \pgfclosestroke

\def\pgfpointxy#1#2{%
  \pgfmathparse{#1}%
  \let\pgftemp@x=\pgfmathresult%
  \pgfmathparse{#2}%
  \let\pgftemp@y=\pgfmathresult%
  \pgf@x=\pgftemp@x\pgf@xx%
  \advance\pgf@x by \pgftemp@y\pgf@yx%
  \pgf@y=\pgftemp@x\pgf@xy%
  \advance\pgf@y by \pgftemp@y\pgf@yy}


% Store the vector #1 * x-vec + #2 * y-vec + #3 * z-vec
%
% #1 = a factor for the x-vector
% #2 = a factor fot the y-vector
% #3 = a factor fot the z-vector
%
% x = x-component of result vector
% y = y-component of result vector
%
%
% Description:
% 
% This command allows you to use a 3d coordinate system.
% 
%
% Example:
%
% % Draw a cubus
% 
% \pgfline{\pgfpointxyz{0}{0}{0}}{\pgfpointxyz{0}{0}{1}}
% \pgfline{\pgfpointxyz{0}{1}{0}}{\pgfpointxyz{0}{1}{1}}
% \pgfline{\pgfpointxyz{1}{0}{0}}{\pgfpointxyz{1}{0}{1}}
% \pgfline{\pgfpointxyz{1}{1}{0}}{\pgfpointxyz{1}{1}{1}}
% \pgfline{\pgfpointxyz{0}{0}{0}}{\pgfpointxyz{0}{1}{0}}
% \pgfline{\pgfpointxyz{0}{0}{1}}{\pgfpointxyz{0}{1}{1}}
% \pgfline{\pgfpointxyz{1}{0}{0}}{\pgfpointxyz{1}{1}{0}}
% \pgfline{\pgfpointxyz{1}{0}{1}}{\pgfpointxyz{1}{1}{1}}
% \pgfline{\pgfpointxyz{0}{0}{0}}{\pgfpointxyz{1}{0}{0}}
% \pgfline{\pgfpointxyz{0}{0}{1}}{\pgfpointxyz{1}{0}{1}}
% \pgfline{\pgfpointxyz{0}{1}{0}}{\pgfpointxyz{1}{1}{0}}
% \pgfline{\pgfpointxyz{0}{1}{1}}{\pgfpointxyz{1}{1}{1}}

\def\pgfpointxyz#1#2#3{%
  \pgfmathparse{#1}%
  \let\pgftemp@x=\pgfmathresult%
  \pgfmathparse{#2}%
  \let\pgftemp@y=\pgfmathresult%
  \pgfmathparse{#3}%
  \let\pgftemp@z=\pgfmathresult%
  \pgf@x=\pgftemp@x\pgf@xx%
  \advance\pgf@x by \pgftemp@y\pgf@yx%
  \advance\pgf@x by \pgftemp@z\pgf@zx%
  \pgf@y=\pgftemp@x\pgf@xy%
  \advance\pgf@y by \pgftemp@y\pgf@yy%
  \advance\pgf@y by \pgftemp@z\pgf@zy}




% Set the x-vector
%
% #1 = a point the is the new x-vector
%
% Example:
%
% \pgfsetxvec{\pgfpoint{1cm}{0cm}}

\def\pgfsetxvec#1{%
  \pgf@process{#1}%
  \pgf@xx=\pgf@x%
  \pgf@xy=\pgf@y%
  \ignorespaces}


% Set the y-vector
%
% #1 = a point the is the new y-vector
%
% Example:
%
% \pgfsetyvec{\pgfpoint{0cm}{1cm}}

\def\pgfsetyvec#1{%
  \pgf@process{#1}%
  \pgf@yx=\pgf@x%
  \pgf@yy=\pgf@y%
  \ignorespaces}


% Set the z-vector
%
% #1 = a point the is the new z-vector
%
% Example:
%
% \pgfsetzvec{\pgfpoint{-0.385cm}{-0.385cm}}

\def\pgfsetzvec#1{%
  \pgf@process{#1}%
  \pgf@zx=\pgf@x%
  \pgf@zy=\pgf@y%
  \ignorespaces}



% Default values
\pgfsetxvec{\pgfpoint{1cm}{0cm}}
\pgfsetyvec{\pgfpoint{0cm}{1cm}}
\pgfsetzvec{\pgfpoint{-0.385cm}{-0.385cm}}




% Normalise a point.
%
% #1 = point with coordinates (a,b)
%
% x  = a/\sqrt(a*a+b*b)
% y  = b/\sqrt(a*a+b*b)
%
% Example:
%
% \pgfpointnormalised{\pgfpointxy{2}{1}}

\def\pgfpointnormalised#1{%
  \pgf@process{#1}%
  \pgf@xa=\pgf@x%
  \pgf@ya=\pgf@y%
  \ifdim\pgf@x<0pt\relax% move into first quadrant
    \pgf@x=-\pgf@x%
  \fi%
  \ifdim\pgf@y<0pt\relax%
    \pgf@y=-\pgf@y%
  \fi%
  \ifdim\pgf@x>\pgf@y% x > y
    % make point small
    \c@pgf@counta=\pgf@x%
    \divide\c@pgf@counta by 65536\relax%
    \ifnum\c@pgf@counta=0\relax%
      \c@pgf@counta=1\relax%
    \fi%
    \divide\pgf@x by\c@pgf@counta%
    \divide\pgf@y by\c@pgf@counta%
    \divide\pgf@xa by\c@pgf@counta%
    \divide\pgf@ya by\c@pgf@counta%
    % ok.
    \pgf@x=.125\pgf@x%
    \pgf@y=.125\pgf@y%
    \c@pgf@counta=\pgf@x%
    \c@pgf@countb=\pgf@y%
    \multiply\c@pgf@countb by 100%
    \ifnum\c@pgf@counta<64\relax%
      \pgf@x=1pt\relax%
      \pgf@y=0pt\relax%
    \else%
      \divide\c@pgf@countb by \c@pgf@counta%
      \pgf@x=\csname pgf@cosfrac\the\c@pgf@countb\endcsname pt%
      \pgf@xc=8192pt%
      \divide\pgf@xc by\c@pgf@counta%
      \pgf@y=\pgf@sys@tonumber{\pgf@xc}\pgf@ya%
      \pgf@y=\pgf@sys@tonumber{\pgf@x}\pgf@y%
    \fi%
    \ifdim\pgf@xa<0pt%
      \pgf@x=-\pgf@x%
    \fi%
  \else% x <= y
    % make point small
    \c@pgf@counta=\pgf@y%
    \divide\c@pgf@counta by 65536\relax%
    \ifnum\c@pgf@counta=0\relax%
      \c@pgf@counta=1\relax%
    \fi%
    \divide\pgf@x by\c@pgf@counta%
    \divide\pgf@y by\c@pgf@counta%
    \divide\pgf@xa by\c@pgf@counta%
    \divide\pgf@ya by\c@pgf@counta%
    % ok.
    \pgf@x=.125\pgf@x%
    \pgf@y=.125\pgf@y%
    \c@pgf@counta=\pgf@y%
    \c@pgf@countb=\pgf@x%
    \multiply\c@pgf@countb by 100%
    \ifnum\c@pgf@counta<64\relax%
      \pgf@y=1pt\relax%
      \pgf@x=0pt\relax%
    \else%
      \divide\c@pgf@countb by \c@pgf@counta%
      \pgf@y=\csname pgf@cosfrac\the\c@pgf@countb\endcsname pt%
      \pgf@xc=8192pt%
      \divide\pgf@xc by\c@pgf@counta%
      \pgf@x=\pgf@sys@tonumber{\pgf@xc}\pgf@xa%
      \pgf@x=\pgf@sys@tonumber{\pgf@y}\pgf@x%
    \fi%
    \ifdim\pgf@ya<0pt%
      \pgf@y=-\pgf@y%
    \fi%
  \fi\ignorespaces%
}





% A point on a rectangle in a certain direction.
%
% #1 = a point pointing in some direction (length should be about 1pt,
%      but need not be exact)
% #2 = upper right corner of a rectangle centered at the origin
%
% Returns the intersection of a line starting at the origin going in
% the given direction and the rectangle's border.
%
% Example:
%
% \pgfpointborderrectangle{\pgfpointnormalised{\pgfpointxy{2}{1}}
%   {\pgfpoint{1cm}{2cm}}

\def\pgfpointborderrectangle#1#2{%
  \pgf@process{#2}%
  \pgf@xb=\pgf@x%
  \pgf@yb=\pgf@y%
  \pgf@process{#1}%
  % Ok, let's find out about the direction:
  \pgf@xa=\pgf@x%
  \pgf@ya=\pgf@y%
  \ifnum\pgf@xa<0\relax% move into first quadrant
    \pgf@x=-\pgf@x%
  \fi%
  \ifnum\pgf@ya<0\relax%
    \pgf@y=-\pgf@y%
  \fi%
  \pgf@xc=.125\pgf@x%
  \pgf@yc=.125\pgf@y%
  \c@pgf@counta=\pgf@xc%
  \c@pgf@countb=\pgf@yc%
  \ifnum\c@pgf@countb<\c@pgf@counta%
    \ifnum\c@pgf@counta<255\relax%
      \pgf@y=\pgf@yb\relax%
      \pgf@x=0pt\relax%
    \else%
      \pgf@xc=8192pt%
      \divide\pgf@xc by\c@pgf@counta% \pgf@xc = 1/\pgf@x
      \pgf@y=\pgf@sys@tonumber{\pgf@xc}\pgf@y%
      \pgf@y=\pgf@sys@tonumber{\pgf@xb}\pgf@y%
      \ifnum\pgf@y<\pgf@yb%
        \pgf@x=\pgf@xb%
      \else% rats, calculate intersection on upper side
        \ifnum\c@pgf@countb<255\relax%
          \pgf@x=\pgf@xb\relax%
          \pgf@y=0pt\relax%
        \else%
          \pgf@yc=8192pt%
          \divide\pgf@yc by\c@pgf@countb% \pgf@xc = 1/\pgf@x
          \pgf@x=\pgf@sys@tonumber{\pgf@yc}\pgf@x%
          \pgf@x=\pgf@sys@tonumber{\pgf@yb}\pgf@x%
          \pgf@y=\pgf@yb%
        \fi%
      \fi%  
    \fi%
  \else%
    \ifnum\c@pgf@countb<255\relax%
      \pgf@x=\pgf@xb\relax%
      \pgf@y=0pt\relax%
    \else%
      \pgf@yc=8192pt%
      \divide\pgf@yc by\c@pgf@countb% \pgf@xc = 1/\pgf@x
      \pgf@x=\pgf@sys@tonumber{\pgf@yc}\pgf@x%
      \pgf@x=\pgf@sys@tonumber{\pgf@yb}\pgf@x%
      \ifnum\pgf@x<\pgf@xb%
        \pgf@y=\pgf@yb%
      \else%
        \ifnum\c@pgf@counta<255\relax%
          \pgf@y=\pgf@yb\relax%
          \pgf@x=0pt\relax%
        \else%
          \pgf@xc=8192pt%
          \divide\pgf@xc by\c@pgf@counta% \pgf@xc = 1/\pgf@x
          \pgf@y=\pgf@sys@tonumber{\pgf@xc}\pgf@y%
          \pgf@y=\pgf@sys@tonumber{\pgf@xb}\pgf@y%
          \pgf@x=\pgf@xb%
        \fi%
      \fi%  
    \fi%
  \fi%  
  \ifnum\pgf@xa<0\relax\pgf@x=-\pgf@x\fi%
  \ifnum\pgf@ya<0\relax\pgf@y=-\pgf@y\fi%    
}




% An approximation to a point on an ellipse in a certain
% direction. Will be exact only if the ellipse is a circle. 
%
% #1 = a point pointing in some direction
% #2 = upper right corner of a bounding box for the ellipse
%
% Returns the intersection of a line starting at the origin going in
% the given direction and the ellipses border.
%
% Example:
%
% \pgfpointborderellipse{\pgfpointnormalised{\pgfpointxy{2}{1}}
%   {\pgfpoint{1cm}{2cm}}

\def\pgfpointborderellipse#1#2{%
  \pgf@process{#2}%
  \pgf@xa=\pgf@x%
  \pgf@ya=\pgf@y%
  \ifdim\pgf@xa=\pgf@ya% circle. that's easy!
    \pgf@process{\pgfpointnormalised{#1}}%
    \pgf@x=\pgf@sys@tonumber{\pgf@xa}\pgf@x%
    \pgf@y=\pgf@sys@tonumber{\pgf@xa}\pgf@y%
  \else%
    \ifdim\pgf@xa<\pgf@ya%
      % Ok, first, let's compute x/y:
      \c@pgf@countb=\pgf@ya%
      \divide\c@pgf@countb by65536\relax%
      \divide\pgf@x by\c@pgf@countb%
      \divide\pgf@y by\c@pgf@countb%
      \pgf@xc=\pgf@x%
      \pgf@yc=8192pt%
      \pgf@y=.125\pgf@y%
      \c@pgf@countb=\pgf@y%
      \divide\pgf@yc by\c@pgf@countb%
      \pgf@process{#1}%
      \pgf@y=\pgf@sys@tonumber{\pgf@yc}\pgf@y%
      \pgf@y=\pgf@sys@tonumber{\pgf@xc}\pgf@y%
      \pgf@process{\pgfpointnormalised{}}%
      \pgf@x=\pgf@sys@tonumber{\pgf@xa}\pgf@x%
      \pgf@y=\pgf@sys@tonumber{\pgf@ya}\pgf@y%
    \else%
      % Ok, now let's compute y/x:
      \c@pgf@countb=\pgf@xa%
      \divide\c@pgf@countb by65536\relax%
      \divide\pgf@x by\c@pgf@countb%
      \divide\pgf@y by\c@pgf@countb%
      \pgf@yc=\pgf@y%
      \pgf@xc=8192pt%
      \pgf@x=.125\pgf@x%
      \c@pgf@countb=\pgf@x%
      \divide\pgf@xc by\c@pgf@countb%
      \pgf@process{#1}%
      \pgf@x=\pgf@sys@tonumber{\pgf@yc}\pgf@x%
      \pgf@x=\pgf@sys@tonumber{\pgf@xc}\pgf@x%
      \pgf@process{\pgfpointnormalised{}}%
      \pgf@x=\pgf@sys@tonumber{\pgf@xa}\pgf@x%
      \pgf@y=\pgf@sys@tonumber{\pgf@ya}\pgf@y%
    \fi%  
  \fi%
}





% Extract the x-coordinate of a point to a dimensions
%  
% #1 = a TeX dimension
% #2 = a point
%
% Example:
%
% \newdimen\mydim
% \pgfextractx{\mydim}{\pgfpoint{2cm}{4pt}}
% % \mydim is now 2cm

\def\pgfextractx#1#2{%
  \pgf@process{#2}%
  #1=\pgf@x\relax}


% Extract the y-coordinate of a point to a dimensions
%  
% #1 = a TeX dimension
% #2 = a point
%
% Example:
%
% \newdimen\mydim
% \pgfextracty{\mydim}{\pgfpoint{2cm}{4pt}}
% % \mydim is now 4pt

\def\pgfextracty#1#2{%
  \pgf@process{#2}%
  #1=\pgf@y\relax}



\def\pgf@def#1#2#3{\expandafter\def\csname pgf@#1#2\endcsname{#3}}
\pgf@def{cosfrac}{0}{1}
\pgf@def{cosfrac}{1}{0.99995}    \pgf@def{cosfrac}{2}{0.9998}
\pgf@def{cosfrac}{3}{0.99955}    \pgf@def{cosfrac}{4}{0.999201}
\pgf@def{cosfrac}{5}{0.998752}   \pgf@def{cosfrac}{6}{0.998205}
\pgf@def{cosfrac}{7}{0.997559}   \pgf@def{cosfrac}{8}{0.996815}
\pgf@def{cosfrac}{9}{0.995974}   \pgf@def{cosfrac}{10}{0.995037}
\pgf@def{cosfrac}{11}{0.994004}  \pgf@def{cosfrac}{12}{0.992877}
\pgf@def{cosfrac}{13}{0.991656}  \pgf@def{cosfrac}{14}{0.990342}
\pgf@def{cosfrac}{15}{0.988936}  \pgf@def{cosfrac}{16}{0.987441}
\pgf@def{cosfrac}{17}{0.985856}  \pgf@def{cosfrac}{18}{0.984183}
\pgf@def{cosfrac}{19}{0.982424}  \pgf@def{cosfrac}{20}{0.980581}
\pgf@def{cosfrac}{21}{0.978653}  \pgf@def{cosfrac}{22}{0.976644}
\pgf@def{cosfrac}{23}{0.974555}  \pgf@def{cosfrac}{24}{0.972387}
\pgf@def{cosfrac}{25}{0.970143}  \pgf@def{cosfrac}{26}{0.967823}
\pgf@def{cosfrac}{27}{0.965429}  \pgf@def{cosfrac}{28}{0.962964}
\pgf@def{cosfrac}{29}{0.960429}  \pgf@def{cosfrac}{30}{0.957826}
\pgf@def{cosfrac}{31}{0.955157}  \pgf@def{cosfrac}{32}{0.952424}
\pgf@def{cosfrac}{33}{0.949629}  \pgf@def{cosfrac}{34}{0.946773}
\pgf@def{cosfrac}{35}{0.943858}  \pgf@def{cosfrac}{36}{0.940887}
\pgf@def{cosfrac}{37}{0.937862}  \pgf@def{cosfrac}{38}{0.934784}
\pgf@def{cosfrac}{39}{0.931655}  \pgf@def{cosfrac}{40}{0.928477}
\pgf@def{cosfrac}{41}{0.925252}  \pgf@def{cosfrac}{42}{0.921982}
\pgf@def{cosfrac}{43}{0.918669}  \pgf@def{cosfrac}{44}{0.915315}
\pgf@def{cosfrac}{45}{0.911922}  \pgf@def{cosfrac}{46}{0.90849}
\pgf@def{cosfrac}{47}{0.905024}  \pgf@def{cosfrac}{48}{0.901523}
\pgf@def{cosfrac}{49}{0.89799}   \pgf@def{cosfrac}{50}{0.894427}
\pgf@def{cosfrac}{51}{0.890835}  \pgf@def{cosfrac}{52}{0.887217}
\pgf@def{cosfrac}{53}{0.883573}  \pgf@def{cosfrac}{54}{0.879905}
\pgf@def{cosfrac}{55}{0.876216}  \pgf@def{cosfrac}{56}{0.872506}
\pgf@def{cosfrac}{57}{0.868777}  \pgf@def{cosfrac}{58}{0.865031}
\pgf@def{cosfrac}{59}{0.861269}  \pgf@def{cosfrac}{60}{0.857493}
\pgf@def{cosfrac}{61}{0.853704}  \pgf@def{cosfrac}{62}{0.849903}
\pgf@def{cosfrac}{63}{0.846092}  \pgf@def{cosfrac}{64}{0.842271}
\pgf@def{cosfrac}{65}{0.838444}  \pgf@def{cosfrac}{66}{0.834609}
\pgf@def{cosfrac}{67}{0.83077}   \pgf@def{cosfrac}{68}{0.826927}
\pgf@def{cosfrac}{69}{0.82308}   \pgf@def{cosfrac}{70}{0.819232}
\pgf@def{cosfrac}{71}{0.815383}  \pgf@def{cosfrac}{72}{0.811534}
\pgf@def{cosfrac}{73}{0.807687}  \pgf@def{cosfrac}{74}{0.803842}
\pgf@def{cosfrac}{75}{0.8}       \pgf@def{cosfrac}{76}{0.796162}
\pgf@def{cosfrac}{77}{0.792329}  \pgf@def{cosfrac}{78}{0.788502}
\pgf@def{cosfrac}{79}{0.784682}  \pgf@def{cosfrac}{80}{0.780869}
\pgf@def{cosfrac}{81}{0.777064}  \pgf@def{cosfrac}{82}{0.773268}
\pgf@def{cosfrac}{83}{0.769481}  \pgf@def{cosfrac}{84}{0.765705}
\pgf@def{cosfrac}{85}{0.761939}  \pgf@def{cosfrac}{86}{0.758185}
\pgf@def{cosfrac}{87}{0.754443}  \pgf@def{cosfrac}{88}{0.750714}
\pgf@def{cosfrac}{89}{0.746997}  \pgf@def{cosfrac}{90}{0.743294}
\pgf@def{cosfrac}{91}{0.739605}  \pgf@def{cosfrac}{92}{0.735931}
\pgf@def{cosfrac}{93}{0.732272}  \pgf@def{cosfrac}{94}{0.728628}
\pgf@def{cosfrac}{95}{0.724999}  \pgf@def{cosfrac}{96}{0.721387}
\pgf@def{cosfrac}{97}{0.717792}  \pgf@def{cosfrac}{98}{0.714213}
\pgf@def{cosfrac}{99}{0.710651}  \pgf@def{cosfrac}{100}{0.707107}



\endinput
