% Copyright 2006 by Till Tantau
%
% This file may be distributed and/or modified
%
% 1. under the LaTeX Project Public License and/or
% 2. under the GNU Public License.
%
% See the file doc/generic/pgf/licenses/LICENSE for more details.

\ProvidesFileRCS $Header: /cvsroot/pgf/pgf/generic/pgf/basiclayer/pgfcorepatterns.code.tex,v 1.2 2008/03/19 20:24:49 vibrovski Exp $

% Creates a new uncolored pattern
%
% [#1] = optional list of variables.
% #2   = pattern name
% #3   = lower left of bounding box
% #4   = upper right of bounding box
% #5   = step vector
% #6   = pattern code
%
% Description:
%
% Declares a new uncolored pattern. An uncolored pattern is a pattern
% that has no inherent color. When the pattern is used, a color for
% the pattern must be supplied.
%
% See the pdf-manual for more details on uncolored patterns.
%
% Specifying a variable list makes the pattern ``mutable''.
%
% The bounding box must be a bounding box of the pattern cell.
%
% The step vector must provide the x- and y-step by which the cells
% are spaced.
%
% The pattern code should be pgf code that can be protocolled.
%
% The transformation matrix will be reset prior to invoking the
% pattern code.
%
% Example:
%
% \pgfdeclarepatternformonly{stars}{\pgfpointorigin}{\pgfpoint{1cm}{1cm}}
% {\pgfpoint{1cm}{1cm}}{
%   \pgftransformshift{\pgfpoint{.5cm}{.5cm}}
%   \pgfpathmoveto{\pgfpointpolar{0}{4mm}}
%   \pgfpathlineto{\pgfpointpolar{144}{4mm}}
%   \pgfpathlineto{\pgfpointpolar{288}{4mm}}
%   \pgfpathlineto{\pgfpointpolar{72}{4mm}}
%   \pgfpathlineto{\pgfpointpolar{216}{4mm}}
%   \pgfpathclose%
%   \pgfusepath{stroke}
% }
%
\def\pgfdeclarepatternformonly{%
	\def\pgf@pattern@type{formonly}%
	\pgfutil@ifnextchar[{\pgf@declarepattern@checkarg}{\pgf@declarepattern@checkarg[]}%
}


% Creates a new colored pattern
%
% [#1] = optional list of variables.
% #2   = pattern name
% #3   = lower left of bounding box
% #4   = upper right of bounding box
% #5   = step vector
% #6   = pattern code
%
% Description:
%
% Declares a new colored pattern. Such patterns have a one or more
% fixed inherent colors. See the pdf-manual for more details on
% uncolored patterns.
%
% The parameters have the same effect as for uncolored patterns.
%
% Example:
%
% \pgfdeclarepatterninherentlycolored{red stars}{\pgfpointorigin}{\pgfpoint{1cm}{1cm}}
% {\pgfpoint{1cm}{1cm}}{
%   \pgfsetfillcolor{red}
%   \pgfsetstrokecolor{black}
%   \pgftransformshift{\pgfpoint{.5cm}{.5cm}}
%   \pgfpathmoveto{\pgfpointpolar{0}{4mm}}
%   \pgfpathlineto{\pgfpointpolar{144}{4mm}}
%   \pgfpathlineto{\pgfpointpolar{288}{4mm}}
%   \pgfpathlineto{\pgfpointpolar{72}{4mm}}
%   \pgfpathlineto{\pgfpointpolar{216}{4mm}}
%   \pgfpathclose%
%   \pgfusepath{stroke,fill}
% }
%
\def\pgfdeclarepatterninherentlycolored{%
	\def\pgf@pattern@type{inherentlycolored}%
	\pgfutil@ifnextchar[{\pgf@declarepattern@checkarg}{\pgf@declarepattern@checkarg[]}%
}





\def\pgf@declarepattern@checkarg[#1]{%
	\def\pgf@pattern@tempvars{#1}%
	\ifx\pgf@pattern@tempvars\pgfutil@empty%
		\expandafter\let\expandafter\pgf@next\csname pgf@declarepattern\pgf@pattern@type\endcsname%
	\else%
		\expandafter\let\expandafter\pgf@next\csname pgf@declarepattern\pgf@pattern@type @mutable\endcsname%
	\fi%
	\pgf@next}

\long\def\pgf@declarepatternformonly#1#2#3#4#5{\pgf@declarepattern{#1}{#2}{#3}{#4}{#5}{0}}
	
\long\def\pgf@declarepatterninherentlycolored#1#2#3#4#5{\pgf@declarepattern{#1}{#2}{#3}{#4}{#5}{1}}
	
\long\def\pgf@declarepatternformonly@mutable#1#2#3#4#5{%
	\def\pgf@marshal{\pgf@declarepatternmutable{#1}}%	
	\expandafter\pgf@marshal\expandafter{\pgf@pattern@tempvars}{#2}{#3}{#4}{#5}{2}%
}

\long\def\pgf@declarepatterninherently@mutable#1#2#3#4#5#6{%
	\def\pgf@marshal{\pgf@declarepatternmutable{#1}}%	
	\expandafter\pgf@marshal\expandafter{\pgf@pattern@tempvars}{#2}{#3}{#4}{#5}{3}%
}

\def\pgf@declarepattern#1#2#3#4#5#6{%
  \pgfutil@ifundefined{pgf@pattern@name@#1}{%
  \pgfsysprotocol@getcurrentprotocol\pgf@pattern@temp%
  {%
    \pgfinterruptpath%
      \pgfpicturetrue%
      \pgf@relevantforpicturesizefalse%
      \pgftransformreset%
      \pgfsysprotocol@setcurrentprotocol\pgfutil@empty%
      \pgfsysprotocol@bufferedtrue%
      \pgfsys@beginscope%
      #5%
      \pgfsys@endscope%
      \pgfsysprotocol@getcurrentprotocol\pgf@pattern@code%
      \global\let\pgf@pattern@code=\pgf@pattern@code%
    \endpgfinterruptpath%
    \pgf@process{#2}%
    \pgf@xa=\pgf@x%
    \pgf@ya=\pgf@y%
    \pgf@process{#3}%
    \pgf@xb=\pgf@x%
    \pgf@yb=\pgf@y%
    \pgf@process{#4}%
    \pgf@xc=\pgf@x%
    \pgf@yc=\pgf@y%
    % Now, build a name for the pattern
    \pgfutil@tempcnta=\pgf@pattern@number%
    \advance\pgfutil@tempcnta by1\relax%
    \xdef\pgf@pattern@number{\the\pgfutil@tempcnta}%
    \expandafter\xdef\csname pgf@pattern@name@#1\endcsname{\the\pgfutil@tempcnta}%
    \expandafter\gdef\csname pgf@pattern@type@#1\endcsname{#6}%
    \xdef\pgf@marshal{\noexpand\pgfsys@declarepattern
      {\csname pgf@pattern@name@#1\endcsname}
      {\the\pgf@xa}{\the\pgf@ya}{\the\pgf@xb}{\the\pgf@yb}{\the\pgf@xc}{\the\pgf@yc}{\pgf@pattern@code}{#6}}%
  }%
  \pgfsysprotocol@setcurrentprotocol\pgf@pattern@temp%
  \expandafter\global\expandafter\let\csname pgf@pattern@instantiate@#1\endcsname=\pgf@marshal%
  }
  {%
    \PackageError{pgfcorepatterns}{Pattern `#1' already defined}{}%
  }%
}

\def\pgf@pattern@number{0}%

\def\pgf@declarepatternmutable#1#2#3#4#5#6#7{%
	\pgfutil@ifundefined{pgf@pattern@name@#1}{%
		\expandafter\gdef\csname pgf@pattern@name@#1\endcsname{#1}%
		\expandafter\gdef\csname pgf@pattern@variables@#1\endcsname{#2}%
		\expandafter\gdef\csname pgf@pattern@lowerleft@#1\endcsname{#3}%
		\expandafter\gdef\csname pgf@pattern@upperright@#1\endcsname{#4}%
		\expandafter\gdef\csname pgf@pattern@tilesize@#1\endcsname{#5}%
		\expandafter\long\expandafter\gdef\csname pgf@pattern@code@#1\endcsname{#6}%
		\expandafter\gdef\csname pgf@pattern@type@#1\endcsname{7}%    
	}{\PackageError{pgfcorepatterns}{The pattern `#1' is already defined}{}}%
}

\def\pgf@ifpatternisinherentlycolored#1#2#3{%
  \expandafter\ifodd\csname pgf@pattern@type@#1\endcsname\relax#2\else#3\fi%
}

\def\pgf@ifpatternismutable#1#2#3{%
  \expandafter\ifnum\csname pgf@pattern@type@#1\endcsname>1\relax#2\else#3\fi%
}

\def\pgfpatternreleasename#1{\expandafter\let\csname pgf@pattern@name@#1\endcsname\relax}






% Set an pattern as fill color
%
% #1 = Name of the pattern
% #2 = Color to be used for the pattern. If the pattern is already
%      colored, this parameter is ignored.
%
% Description:
%
% Sets the pattern for subsequent filling operations.
%
% Example:
%
% \pgfsetfillpattern{stars}{red}
%
\def\pgfsetfillpattern#1#2{%
	\pgfutil@ifundefined{pgf@pattern@name@#1}%
	{%
		\PackageError{pgfcorepatterns}{Undefined pattern `#1'}{}%
	}%
	{%
		\pgf@ifpatternismutable{#1}%
		{%
			% So, a mutable pattern. Check to see if the pattern 
			% has been used with the current variable values...
			\let\pgf@pattern@tempvars\pgfutil@empty%
			\expandafter\expandafter\expandafter\pgf@pattern@check@vars%
				\csname pgf@pattern@variables@#1\endcsname,\pgf@stop,%
			\edef\pgf@pattern@nametemp{pgf@pattern@#1@\pgf@pattern@tempvars}%
			\expandafter\pgfutil@ifundefined\expandafter{\pgf@pattern@nametemp}%
			{%
				% ...Nope. So declare a new instance of the mutable pattern.
				\pgf@ifpatternisinherentlycolored{#1}%
					{\let\pgf@set@fillpattern@declare\pgfdeclarepatterninherentlycolored}%
					{\let\pgf@set@fillpattern@declare\pgfdeclarepatternformonly}%
				\pgf@set@fillpattern@declare%
					{\pgf@pattern@nametemp}%
					{\csname pgf@pattern@lowerleft@#1\endcsname}%
					{\csname pgf@pattern@upperright@#1\endcsname}%
					{\csname pgf@pattern@tilesize@#1\endcsname}%
					{\csname pgf@pattern@code@#1\endcsname}%
			}%
			{%
				% ...Yep. So do nothing.
			}%
			\expandafter\pgf@set@fillpattern\expandafter{\pgf@pattern@nametemp}{#2}%
			\expandafter\gdef\csname\pgf@pattern@nametemp\endcsname{#1}%
		}%
		{%
			% A non-mutable pattern.
			\pgf@set@fillpattern{#1}{#2}%
		}%
	}%
}


\def\pgf@set@fillpattern#1#2{%
  \pgfutil@ifundefined{pgf@pattern@name@#1}{%
    \PackageError{pgfcorepatterns}{Undefined pattern `#1'}{}}
  {%
    \csname pgf@pattern@instantiate@#1\endcsname%
    \expandafter\global\expandafter\let\csname pgf@pattern@instantiate@#1\endcsname=\relax%
    \pgf@ifpatternisinherentlycolored{#1}{%
      \pgfsys@setpatterncolored{\csname pgf@pattern@name@#1\endcsname}%
    }{%
      \pgfutil@colorlet{pgf@tempcolor}{#2}%
      \pgfutil@ifundefined{applycolormixins}{}{\applycolormixins{pgf@tempcolor}}%
      \pgfutil@extractcolorspec{pgf@tempcolor}{\pgf@tempcolor}%
      \expandafter\pgfutil@convertcolorspec\pgf@tempcolor{rgb}{\pgf@rgbcolor}%
      \expandafter\pgf@set@fill@patternuncolored\pgf@rgbcolor\relax{#1}%
    }%
  }%
}

\def\pgf@set@fill@patternuncolored#1,#2,#3\relax#4{%
  \pgfsys@setpatternuncolored{\csname pgf@pattern@name@#4\endcsname}{#1}{#2}{#3}%
}

\def\pgf@pattern@check@vars#1,{%
	\ifx#1\pgf@stop%
		\let\pgf@next\relax%
	\else%
		\afterassignment\pgfmath@gobbletilpgfmath@%
		\expandafter\let\expandafter\pgf@pattern@char#1#1\pgfmath@%
		\ifx\pgf@pattern@char/%
			\edef\pgf@pattern@tempvar{(\pgfkeysvalueof{#1})}%
		\else%
			\expandafter\ifcat\pgf@pattern@char\relax%
				\edef\pgf@pattern@tempvar{(\the#1)}%
			\else%
				\edef\pgf@pattern@tempvar{(#1)}%
			\fi%
		\fi%
		\edef\pgf@pattern@tempvars{\pgf@pattern@tempvars\pgf@pattern@tempvar}%
		\let\pgf@next\pgf@pattern@check@vars%
	\fi%
	\pgf@next%
}

\endinput
