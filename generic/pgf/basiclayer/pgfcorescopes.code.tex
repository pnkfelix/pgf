\ProvidesFileRCS $Header: /cvsroot/pgf/pgf/generic/pgf/basiclayer/pgfcorescopes.code.tex,v 1.10 2005/10/25 08:18:14 tantau Exp $

% Copyright 2005 by Till Tantau <tantau@cs.tu-berlin.de>.
%
% This program can be redistributed and/or modified under the terms
% of the GNU Public License, version 2.

% Globals

\newbox\pgfpic
\newbox\pgf@hbox

\newbox\pgf@layerbox@main


% Scopes


% Pgf scope environment. All changes of the graphic state are local to
% the scope.
%  
% Example:
%
% \begin{pgfscope}
%    \pgfsetlinewidth{3pt}
%    \pgfline{\pgfxy(0,0)}{\pgfxy(3,3)}
% \end{pgfscope}

\def\pgfscope{%
  \pgfsyssoftpath@setcurrentpath\@empty%
  \pgfsys@beginscope%
    \pgf@resetpathsizes%
    \edef\pgfscope@linewidth{\the\pgflinewidth}%
    \begingroup}
\def\endpgfscope{%
    \endgroup%
    \global\pgflinewidth=\pgfscope@linewidth%
  \pgfsys@endscope}




% Quickly insert a box can contain normal TeX text at the origin.
% 
% #1 = box of width/height and depth 0pt
% 
% Example:
%
% \pgfqbox{\mybox}

\def\pgfqbox#1{%
  \pgfsys@hbox#1%
} 


% Insert a box that can contain normal TeX text at the origin, but
% with the current coordinate transformation matrix synced with the
% low-level transformation matrix.
% 
% #1 = box of width/height and depth 0pt
%
% In essence, this command does the same as if you first said
% \pgflowlevelsynccm and then \pgfqbox. However, pgf will use a
% ``TeX-translation'' for the translation part of the transformation
% cm. This will ensure that hyperlinks ``survive'' at least
% translations. 
% 
% Example:
%
% \pgfqboxsynced{\mybox}

\def\pgfqboxsynced#1{%
  \pgfsys@hboxsynced#1%
}


% Puts some text in a box and inserts it with the current
% transformations applied.
% 
% #1 = List of optional positioning. Possible values are ``left'', ``right'',
%      ``top'', ``bottom'' and ``base''.
% #2 = TeX text. May contain verbatims.
% 
% Example:
%
% \pgftransformshift{\pgfpoint{1cm}{0cm}}
% \pgftext{Hello World!}

\def\pgftext{\@ifnextchar[\pgf@text{\pgf@text[]}}%
\def\pgf@text[#1]{%
  \def\pgf@text@options{#1}%
  \pgf@maketext\pgf@after@text}
\def\pgf@after@text{%
  {
  \def\pgf@text@hshift{center}%
  \def\pgf@text@vshift{center}%
  \def\pgf@marshal{\setkeys{pgfbox}}%
  \expandafter\pgf@marshal\expandafter{\pgf@text@options}%
  \csname pgf@halign\pgf@text@hshift\endcsname%
  \csname pgf@valign\pgf@text@vshift\endcsname%
  % Protocol sizes:
  \pgf@process{\pgfpointtransformed{\pgfpoint{0pt}{\dp\pgf@hbox}}}%
  \pgf@protocolsizes{\pgf@x}{\pgf@y}%
  \pgf@process{\pgfpointtransformed{\pgfpoint{\wd\pgf@hbox}{\dp\pgf@hbox}}}%
  \pgf@protocolsizes{\pgf@x}{\pgf@y}%
  \pgf@process{\pgfpointtransformed{\pgfpoint{0pt}{\ht\pgf@hbox}}}%
  \pgf@protocolsizes{\pgf@x}{\pgf@y}%
  \pgf@process{\pgfpointtransformed{\pgfpoint{\wd\pgf@hbox}{\ht\pgf@hbox}}}%
  \pgf@protocolsizes{\pgf@x}{\pgf@y}%
  \pgfqboxsynced{\pgf@hbox}%
  }%
}

\define@key{pgfbox}{left}[]{\def\pgf@text@hshift{left}}
\define@key{pgfbox}{center}[]{}
\define@key{pgfbox}{right}[]{\def\pgf@text@hshift{right}}
\define@key{pgfbox}{top}[]{\def\pgf@text@vshift{top}}
\define@key{pgfbox}{bottom}[]{\def\pgf@text@vshift{bottom}}
\define@key{pgfbox}{base}[]{\def\pgf@text@vshift{base}}
\define@key{pgfbox}{at}{\pgftransformshift{#1}}
\define@key{pgfbox}{x}{\pgftransformxshift{#1}}
\define@key{pgfbox}{y}{\pgftransformyshift{#1}}
\define@key{pgfbox}{rotate}{\pgftransformrotate{#1}}

\def\pgf@halignleft{}% do nothing
\def\pgf@haligncenter{\pgftransformxshift{-.5\wd\pgf@hbox}}
\def\pgf@halignright{\pgftransformxshift{-\wd\pgf@hbox}}%
\def\pgf@valignbase{}% do nothing
\def\pgf@valignbottom{\pgftransformyshift{\dp\pgf@hbox}}%
\def\pgf@valigncenter{\pgftransformyshift{.5\dp\pgf@hbox}\pgftransformyshift{-.5\ht\pgf@hbox}}%
\def\pgf@valigntop{\pgftransformyshift{-\ht\pgf@hbox}}%


% Internal function for creating a hbox.
\def\pgf@maketext#1{%
  \def\pgf@@maketextafter{#1}%
  \setbox\pgf@hbox=\hbox\bgroup%
    \pgfinterruptpicture%
      \bgroup%
        \aftergroup\pgf@collectresetcolor%
        \let\next=%
}
\def\pgf@collectresetcolor{%
  \@ifnextchar\reset@color%
  {\reset@color\afterassignment\pgf@collectresetcolor\let\pgf@temp=}%
  {\pgf@textdone}%
}
\def\pgf@textdone{%
    \endpgfinterruptpicture%
  \egroup%
  \pgf@@maketextafter%  
}

\long\def\pgf@makehbox#1{%
  \setbox\pgf@hbox=\hbox{{%
    \pgfinterruptpicture%
      #1%
    \endpgfinterruptpicture%
    }}}
 



% Picture environment
%
% Example:
%
% \begin{pgfpicture}
%   \pgfsetendarrow{\pgfarrowto}
%   \pgfpathmoveto{\pgfpointxy{-0.9}{0.2}}
%   \pgfpathlineto{\pgfpointxy{0.9}{0.4}}
%   \pgfusepath{stroke}
% \end{pgfpicture}

\newif\ifpgfpicture

\def\pgfpicture{%
  \begingroup%
  \pgfpicturetrue%
  \pgf@picmaxx=-16000pt\relax%
  \pgf@picminx=16000pt\relax%
  \pgf@picmaxy=-16000pt\relax%
  \pgf@picminy=16000pt\relax%
  \pgf@relevantforpicturesizetrue%
  \pgf@resetpathsizes%
  \@ifnextchar\bgroup\pgf@oldpicture\pgf@picture}
\def\pgf@oldpicture#1#2#3#4{%
  \setlength\pgf@picminx{#1}%
  \setlength\pgf@picminy{#2}%
  \setlength\pgf@picmaxx{#3}%
  \setlength\pgf@picmaxy{#4}%
  \pgf@relevantforpicturesizefalse%
  \pgf@picture}

\def\pgf@picture{%
  \setbox\pgfpic\hbox to0pt\bgroup%
    \begingroup%
    \color{.}%
    \pgfsys@beginpicture%
      \pgfsys@beginscope%
        \pgfsetlinewidth{0.4pt}%
        \pgftransformreset%
        \pgfsyssoftpath@setcurrentpath\@empty%
        \begingroup%
          \let\pgf@setlengthorig=\setlength%
          \let\pgf@addtolengthorig=\addtolength%
          \let\pgf@selectfontorig=\selectfont%
          \let\setlength=\pgf@setlength%
          \let\addtolength=\pgf@addtolength%
          \let\selectfont=\pgf@selectfont%
          \nullfont\spaceskip\z@\xspaceskip\z@%
          \setbox\pgf@layerbox@main\hbox to0pt\bgroup%
            \begingroup%
  }
\def\endpgfpicture{%
              % Shift baseline outside:
              \global\let\pgf@shift@baseline=\pgf@baseline%
            \endgroup%
            \hss%
          \egroup%
          \pgf@insertlayers%
        \endgroup%    
      \pgfsys@discardpath\pgfsys@endscope%
    \pgfsys@endpicture%
    \endgroup%
    \hss
  \egroup%
  % ok, now let's position the box
  \ifdim\pgf@picmaxx=-16000pt\relax%
    % empty picture. make size 0.  
    \pgf@picmaxx=0pt\relax%
    \pgf@picminx=0pt\relax%
    \pgf@picmaxy=0pt\relax%
    \pgf@picminy=0pt\relax%
  \fi%
  \pgfsys@typesetpicturebox\pgfpic%
  \endgroup%
}

\def\pgf@insertlayers{%
  \box\pgf@layerbox@main%
}


% Sets the baseline
%
% #1 = baseline
%
% Sets the baseline of the picture. Default is the lower border, which
% is the same as \pgf@picminy
%
% Example:
%
% \pgfsetbaseline{1cm+2pt}

\def\pgfsetbaseline#1{\def\pgf@baseline{#1}}
\pgfsetbaseline{\pgf@picminy}





% Interrupt path
%
% Description:
%
% The environment can be used to insert some drawing commands while
% constructing a path. The drawing commands inside the environment
% will not interfere with the path being constructed ``outside.''
% However, you must ward against graphic state changes using a scope. 
%
% Example: Draw two parallel lines
%
% \pgfmoveto{\pgfpoint{0cm}{0cm}}
% \begin{pgfinterruptpath}
%   \pgfmoveto{\pgfpoint{1cm}{0cm}}
%   \pgfmoveto{\pgfpoint{1cm}{1cm}}
%   \pgfusepath{stroke}
% \end{pgfinterruptpath}
% \pgflineto{\pgfpoint{0cm}{1cm}}
% \pgfusepath{stroke}

\def\pgfinterruptpath
{%
  \begingroup%
  % save all sorts of things...
  \edef\pgf@interrupt@savex{\the\pgf@path@lastx}%
  \edef\pgf@interrupt@savey{\the\pgf@path@lasty}%
  \pgf@getpathsizes\pgf@interrupt@pathsizes%
  \pgfsyssoftpath@getcurrentpath\pgf@interrupt@path%
  \pgfsyssoftpath@setcurrentpath\@empty%
  \edef\pgfscope@linewidth{\the\pgflinewidth}%
  \begingroup%
}
\def\endpgfinterruptpath
{%
  \endgroup%
  \global\pgflinewidth=\pgfscope@linewidth%
  \pgfsyssoftpath@setcurrentpath\pgf@interrupt@path%
  \pgf@setpathsizes\pgf@interrupt@pathsizes%
  \global\pgf@path@lastx=\pgf@interrupt@savex%
  \global\pgf@path@lasty=\pgf@interrupt@savey%
  \endgroup%
}




% Interrupts a picture
%
% Description:
%
% This environment interrupts a picture and temporarily returns to
% normal TeX mode. All sorts of things are saved and restored by this
% environment.
%
% WARNING: Using this environment in conjuction with low level
% transformations can *strongly* upset the typesetting. Typically, the
% contents of this environment should have size/height/depth 0pt in
% the end.
%
% WARNING: This environment should only be used inside typesetting a
% box and this box must in turn be inserted using \pgfqbox.
%
% Example: Draw two parallel lines
%
% \pgfmoveto{\pgfpoint{0cm}{0cm}}
% \setbox\mybox=\hbox{
%    \begin{pgfinterruptpicture}
%      This is normal text.
%      \begin{pgfpicture} % a subpicture
%        \pgfmoveto{\pgfpoint{1cm}{0cm}}
%        \pgfmoveto{\pgfpoint{1cm}{1cm}}
%        \pgfusepath{stroke}
%      \end{pgfpicture}
%      More text.
%    \end{pgfinterruptpicture}
%  }
% \ht\mybox=0pt
% \wd\mybox=0pt
% \dp\mybox=0pt
% \pgfqbox{\mybox}%
% \pgfpathlineto{\pgfpoint{0cm}{1cm}}
% \pgfusepath{stroke}

\def\pgfinterruptpicture
{%
  \begingroup%
  \edef\pgf@interrupt@savemaxx{\the\pgf@picmaxx}%
  \edef\pgf@interrupt@saveminx{\the\pgf@picminx}%
  \edef\pgf@interrupt@savemaxy{\the\pgf@picmaxy}%
  \edef\pgf@interrupt@saveminy{\the\pgf@picminy}%
  \pgftransformreset%
  \pgfinterruptpath%
  \ifx\pgf@setlengthorig\@undefined%
  \else%
    \let\setlength\pgf@setlengthorig%
    \let\addtolength\pgf@addtolengthorig%
    \let\selectfont\pgf@selectfontorig%
  \fi%
  \selectfont%
  \pgfpicturefalse%
  \pgf@savelayers%
}
\def\endpgfinterruptpicture
{%
  \pgf@restorelayers%
  \endpgfinterruptpath%
  \global\pgf@picmaxx=\pgf@interrupt@savemaxx%
  \global\pgf@picmaxy=\pgf@interrupt@savemaxy%
  \global\pgf@picminx=\pgf@interrupt@saveminx%
  \global\pgf@picminy=\pgf@interrupt@saveminy%
  \endgroup%
}

\let\pgf@savelayers=\relax
\let\pgf@restorelayers=\relax


\endinput
