\ProvidesPackageRCS[v\pgfversion] $Header: /cvsroot/pgf/pgf/generic/pgf/basiclayer/Attic/pgfbaseimage.code.tex,v 1.2 2005/07/06 15:58:57 tantau Exp $

% Copyright 2005 by Till Tantau <tantau@cs.tu-berlin.de>.
%
% This program can be redistributed and/or modified under the terms
% of the GNU Public License, version 2.




% Declare an image
%
% #1 = optional argument: width, height, page specification
% #2 = name of the image for later use
% #4 = filename without extension, automatic extensions are .pdf,
%      .jpg, and .png for PDF and .ps, .eps, .epsi for postscript.
%
% Description:
%
% This command declares an image file for later use. Even if the image
% is used several times in the document, in PDF it will only be stored
% once. 
%
% Example:
%
% \pgfdeclareimage[width=2cm,page=2]{myimage}{imagefilename}

\def\pgfdeclareimage{\@ifnextchar[{\pgf@declareimage}{\pgf@declareimage[]}}
\def\pgf@declareimage[#1]#2#3{%
  \setkeys{pgfimage}{height=,width=,page=,interpolate=false,mask=}% defaults
  \setkeys{pgfimage}{#1}%
  % Try to find the file
  \gdef\pgf@filename{}%
  % If page= parameter is not empty, try that file first:
  \ifx\pgf@imagepage\@empty%
  \else%
  \expandafter\pgf@findfile\pgfsys@imagesuffixlist:+{#3.page\pgf@imagepage}%
  \fi%
  \ifx\pgf@filename\@empty%
    \expandafter\pgf@findfile\pgfsys@imagesuffixlist:+{#3}%
  \else%
    \setkeys{pgfimage}{page=}% make page empty     
  \fi%
  \ifx\pgf@filename\@empty%
    \PackageWarning{pgf}%
    {File "#3" not found when defining image "#2".\MessageBreak
      Tried all extensions in "\pgfsys@imagesuffixlist"}%
    \pgf@declaredraftimage{#2}%
  \else%
    \ifpgf@draftmode%
      \pgf@declaredraftimage{#2}%
    \else%
      \pgfsys@defineimage%
    \fi%
  \fi%
  \expandafter\global\expandafter\let\csname pgf@image@#2!\endcsname=\pgf@image%
}

\define@key{pgfimage}{width}{\edef\pgf@imagewidth{#1}}
\define@key{pgfimage}{height}{\edef\pgf@imageheight{#1}}
\define@key{pgfimage}{page}{\edef\pgf@imagepage{#1}}
\define@key{pgfimage}{interpolate}[true]{\edef\pgf@imageinterpolate{/Interpolate #1}}
\define@key{pgfimage}{mask}{%
  \edef\pgf@imagemask{#1}%
  \ifx\pgf@imagemask\@empty%
  \else%
    \@ifundefined{pgf@mask@#1}%
    {\PackageError{pgf}{Undefined mask "#1".}{}%
      \edef\pgf@imagemask{}}%
    {\edef\pgf@imagemask{\csname pgf@mask@#1\endcsname}%
      \ifx\pgf@imagemask\@empty%
      \else
      \edef\pgf@imagemask{/SMask \pgf@imagemask\space 0 R}%
      \fi}%
    \fi}

\def\pgf@findfile#1:#2+#3{%
  \IfFileExists{#3.#1}%
  {\xdef\pgf@filename{#3.#1}}%
  {\def\pgf@mightbeempty{#2}%
    \ifx\pgf@mightbeempty\@empty\else%
    \pgf@findfile#2+{#3}%
    \fi}}

\def\pgf@declaredraftimage#1{%
  \ifx\pgf@imagewidth\@empty%
    \PackageWarning{pgf}{Missing width for image "#1" in draft mode.\MessageBreak  Using 1cm instead}%
    \edef\pgf@imagewidth{1cm}%
  \fi%
  \ifx\pgf@imageheight\@empty%
    \PackageWarning{pgf}{Missing height for image "#1" in draft mode.\MessageBreak Using 1cm instead}%
    \edef\pgf@imageheight{1cm}%
  \fi%  
  \ifx\pgf@imagepage\@empty\else\edef\pgf@imagepagetext{ page \pgf@imagepage}\fi%
  \edef\pgf@image{%
    \hbox to \pgf@imagewidth{%
      \vrule\kern-0.4pt%
      \vbox to \pgf@imageheight{%
        \hrule\vfil%
        \hbox to \pgf@imagewidth{\hskip-10cm\hfil\noexpand\tiny#1\pgf@imagepage\hfil\hskip-10cm}%
        \vfil\hrule}%
      \kern-0.4pt\vrule}%
  }%
}



% Declare a soft mask
%
% #1 = optional argument: matte specification. default matte is
%      white. 
% #2 = name of the mask for later use
% #3 = filename without extension, automatic extensions are .pdf,
%      .jpg, and .png for PDF. Postscript is not supported.
%
% Description:
%
% This command declares a soft mask for later masking an image. The
% declared mask should be used together with an image of exactly the
% same height/width if matte is defined. Matte should be the
% preblended background color (see pdf spec for details).
%
% Example:
%
% \pgfdeclaremask[matte=white]{maskname}{maskfilename}
% \pgfdeclareimage[mask=maskname]{image}{imagefilename}

\def\pgfdeclaremask{\@ifnextchar[\pgf@declaremask{\pgf@declaremask[]}}
\def\pgf@declaremask[#1]#2#3{%
  \setkeys{pgfmask}{matte={1 1 1}}% defaults
  \setkeys{pgfmask}{#1}%
  \def\pgf@mask{}%
  % Try to find the file
  \gdef\pgf@filename{}%
  \expandafter\pgf@findfile\pgfsys@imagesuffixlist:+{#3}%
  \ifx\pgf@filename\@empty%
    \PackageWarning{pgf}%
    {File "#3" not found when defining mask "#2".\MessageBreak
      Tried all extensions in "\pgfsys@imagesuffixlist"}%
  \else%
    \pgfsys@definemask%
  \fi%
  \expandafter\global\expandafter\let\csname pgf@mask@#2\endcsname=\pgf@mask%
}

\define@key{pgfmask}{matte}{\edef\pgf@maskmatte{#1}}




% Create an alias name for an image
%
% #1 = name of the alias
% #2 = name of the original
%
% Example:
%
% \pgfdeclareimage{image}{1cm}{1cm}{filename
% \pgfaliasimage{alias}{image}
% \pgfuseimage{alias}

\def\pgfaliasimage#1#2{%
  \expandafter\global\expandafter\let\expandafter\pgf@temp\expandafter=\csname pgf@image@#2!\endcsname%
  \expandafter\global\expandafter\let\csname pgf@image@#1!\endcsname=\pgf@temp%
  }

  
% Use an image
%
% #1 = name of a previously declared image
%
% Example:
%
% \pgfputat{\pgforigin}{\pgfbox[left,base]{\pgfuseimage{myimage}}}

\def\pgfuseimage#1{%
  \def\pgf@imagename{pgf@image@#1}%
  \pgf@tryextensions{\pgf@imagename}{\pgfalternateextension}%
  \expandafter\@ifundefined\expandafter{\pgf@imagename}%
  {\PackageError{pgf}{Undefined image "#1"}{}}%
  {{\leavevmode\csname\pgf@imagename\endcsname}}}

\def\pgf@tryextensions#1#2{%
  \edef\pgf@args{[#2!]}\expandafter\pgf@@tryextensions\pgf@args{#1}}
\def\pgf@@tryextensions[#1!#2]#3{%
  \expandafter\@ifundefined\expandafter{#3.#1!#2}%
  {\def\pgf@temp{#2}%
    \ifx\pgf@temp\@empty%
    \edef#3{#3!}%
    \else%
    \pgf@@tryextensions[#2]{#3}%
    \fi}%
  {\edef#3{#3.#1!#2}}}


% Alternate image
%
% Description:
%
% When an image is used, pgf first attempts to use the image with the
% alternate extension added. If this fails, the original image is
% used. If the alternate extension contains ``!'', then the text up to
% the ! is successively removed and the remainder is tried as an
% alternate extension.
%
% Example:
%
% \def\pgfalternateextension{20!white}

\def\pgfalternateextension{}



% Directly insert an image
%
% #1 = optional argument: width, height, page specification
% #2 = file name
%
% Description:
%
% Directly inserts an image without declaration. You can, however,
% still access this image using the name pgflastimage. By using
% pgfaliasimage, you can also save this image for later.
%
% Example:
%
% \pgfimage[height=2cm]{filename}

\def\pgfimage{\@ifnextchar[\pgf@imagecom{\pgf@imagecom[]}}
\def\pgf@imagecom[#1]#2{%
  \pgfdeclareimage[#1]{pgflastimage}{#2}%
  \pgfuseimage{pgflastimage}}



\endinput
