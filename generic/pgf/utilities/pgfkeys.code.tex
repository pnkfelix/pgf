% Copyright 2007 by Till Tantau
%
% This file may be distributed and/or modified
%
% 1. under the LaTeX Project Public License and/or
% 2. under the GNU Public License.
%
% See the file doc/generic/pgf/licenses/LICENSE for more details.


% The purpose of this file is to provide a general settings engine that 
% works with all TeX formats and has no save-stack impact


% This is useful:

\def\pgfkeys@ifcsname#1\endcsname#2\else#3\fi{\expandafter\ifx\csname#1\endcsname\relax#3\else#2\fi}%
\ifx\eTeXrevision\undefined%
\else%
  \expandafter\let\expandafter\pgfkeys@ifcsname\csname ifcsname\endcsname%
\fi



% Set a key to a value
%
% #1 = key
% #2 = a value
%
% Description:
%
% This command sets the key to the given value. The value is stored as
% is and can even contain things like #9. 
%
% Keys are organized hierarchically using something similar to Unix
% paths. Thus, a typically key might be called "/tikz/length" or
% "/tikz/length dimension/.code". Some keys starting with a dot are
% special, so they should not be used as normal key names (they are
% similar to Unix files starting with a dot -- you can use them, but
% be careful).
%
% Keys are always local to the current TeX group.
%
% Example:
%
% \pgfkeysdefkey{/tikz/length/.value}{2cm-3cm}
% \pgfkeysdefkey{/algo/swap}{{#2}{#1}}

\def\pgfkeyssetkey#1#2{%
  \pgfkeys@temptoks{#2}\expandafter\edef\csname pgfk@#1\endcsname{\the\pgfkeys@temptoks}%
}



% Add text to a key at the end
%
% #1 = key
% #2 = a value to be added at the beginning
% #3 = a value to be added at the end
%
% Description:
%
% This command adds #2 to the definition of the key. The key should
% have been set previously using \pgfkeyssetkey.
%
% Example:
%
% \pgfkeysaddkey{/tikz/length/.value}{}{-3cm}

\def\pgfkeysaddkey#1#2#3{%
  {%
    \toks0{#1}%
    \pgfkeys@temptoks\expandafter\expandafter\expandafter{\csname pgfk@#1\endcsname}%
    \toks1{#2}%
    \expandafter
  }%
  \edef\pgfkeys@temp{\the\toks0 \the\pgfkeys@temptoks \the\toks1}% believe or don't: the spaces are important
  \pgfkeysletkey{#1}\pgfkeys@temp%
}



% Makes a key equal a given code
%
% #1 = key
% #2 = a code name
%
% Description:
%
% This command executes a \let command so that a key gets the same
% value as the parameter #2.
%
% Keys are always local to the current TeX group.
%
% Example:
%
% \pgfkeysletkey{/algo/swap}{\myswap}

\def\pgfkeysletkey#1#2{%
  \expandafter\let\csname pgfk@#1\endcsname#2%
}


% Retrieve the code stored in a key into a code
%
% #1 = key
% #2 = code
%
% Description:
%
% This command will set #2 to "point" to the value stored in the key.
%
% Example:
%
% \pgfkeysgetkey{/tikz/swap}{\myswap}

\def\pgfkeysgetkey#1#2{\expandafter\let\expandafter#2\csname pgfk@#1\endcsname}



% Retrieve the value stored in a key
%
% #1 = key
%
% Description:
%
% This command will expand to the value stored in the key. The key
% should previously have been set using \pgfsetkey or \pgfletkey. 
%
% Example:
%
% The length is \pgfkeysvalue{/tikz/length}.

\def\pgfkeysvalue#1{\csname pgfk@#1\endcsname}



% If for testing whether a key exists
%
% #1 = key
% #2 = if-case
% #3 = else-case
%
% Description:
%
% This if will be executed if the key exists. In eTeX mode this works
% like a normal if, in normal TeX mode you need to provide an \else.
%
% Example:
%
% \ifpgfkeydefined{/tikz/length}{key exists}{does not exist}

\def\ifpgfkeydefined#1#2#3{\pgfkeys@ifcsname pgfk@#1\endcsname#2\else#3\fi}




% Execute settings
%
% #1 = list of settings
%
% Description:
%
% The list of settings should contain comma-separated settings. Each
% setting has the following form:
%
% /path/key=value
%
% The parts "/path/" and "=value" are optional. When the path is not
% specified, the value of the token register "\pgfkeypath" is used. If
% "=value" is missing, the value of the setting "/path/key/.default" is used
% instead. If this key is set to "\pgfvaluerequired", the key
% "/errors/value required/.code" is executed. The code \pgfcurrentkey
% will store the name of the current key.
%
% Any spaces at the beginning and at the end and around the
% equals-sign are removed. The key with the complete path is set to
% the code \pgfcurrentkey.
%
% The setting is then processed according to the following rules:
%
% 1) If the key /path/key/.code" is present, its code is executed
%    with the value computed above, followed by \pgfeov (end of
%    value). So, to handle
%
%    "/stuff/height=  1.5 ,"
%
%    /stuff/height/.code should be set to some code, that can
%    handle the parameter
%
%    "1.5\pgfeov"
%
%    For instance, saying
%
%    \pgfsetkey{/stuff/height/.code}{#1\pgfeov}{\def\myheight{#1}}
%
%    will do nicely.
%
% 2) Otherwise, if the key /path/key/.value is present, this key is
%    set to the value computed above.
%
% 3) Otherwise, if the key /handlers/key/.code is present, it is executed
%    with the same parameters as in 1). Additionally, the
%    token register \pgfcurrentkeypath will be set to "/path/" and the
%    macor \pgfcurrentkeywithoutpath to "key". So, in the above
%    example if neither "/stuff/height/.code" nor
%    "/stuff/height/.value" is present, but "/handlers/height" is,
%    then "/handlers/height" is executed with the parameters:
%
%    "1.5\pgfeov"
%
%    and \pgfcurrentkey is set to "/stuff/height" and \pgfcurrentkeypath
%    is set to "/stuff/" and \pgfcurrentkeywithoutpath to "height".
%
% 4) Otherwise, if the key "/path/.unknown/.code" is present, its code is
%    executed with the same parameters as in 3).
%
% 5) Otherwise, the key "/handlers/.unknown/.code" is executed with the same
%    parameters as in 1).
%
% After all settings have been processed, the value of the token
% register \pgfdefaultkeypath is set to its original value. Thus, any local
% change of this token register has no effect outside the call.
%
% Example:
%
% \pgfset{/tikz/.set style={/path=/tikz/}}
% \pgfset{/path=/tikz/line width/,
%         .set code=\pgfsetlinewidth{##1},
%         .set default=.4pt}
% \pgfset{tikz,line width=1pt}

\newtoks\pgfkeysdefaultpath
\newtoks\pgfkeyscurrentpath
\newtoks\pgfkeys@temptoks
\pgfkeysdefaultpath{/}


\def\pgfset{% Save path
  \expandafter\pgfkeys@withpath\expandafter{\the\pgfkeysdefaultpath}%
}

\def\pgfkeys@withpath#1#2{%
  \pgfkeys@parse#2,\relax,%
  \pgfkeysdefaultpath{#1}% restore path
}

\def\pgfkeys@parse#1,{%
  \ifx\relax#1\pgfutil@empty%
  \else%
    \pgfkeys@unpack#1=\pgfkeys@novalue=\pgfkeys@stop%
    \expandafter\pgfkeys@parse%
  \fi%
}

\def\pgfkeys@novalue{\pgfkeys@novalue} % equals only itself
\def\pgfkeysvaluerequired{\pgfkeysvaluerequired} % equals only itself

\def\pgfkeys@unpack#1=#2=#3\pgfkeys@stop{%
  \pgfkeys@spdef\pgfkeyscurrent{#1}%
  \edef\pgfkeyscurrent{\pgfkeyscurrent}%
  \ifx\pgfkeyscurrent\pgfutil@empty%
    % Skip
  \else%
    \pgfkeys@add@path@as@needed%
    \pgfkeys@spdef\pgfkeys@val{#2}%
    \ifx\pgfkeys@val\pgfkeys@novalue% Hmm... no value
      \ifpgfkeydefined{\pgfkeyscurrent/.default}%
      {\pgfkeysgetkey{\pgfkeyscurrent/.default}{\pgfkeys@val}}
      {\let\pgfkeys@val\pgfutil@empty}% no default, so use empty string
    \fi%
    \ifx\pgfkeys@val\pgfkeysvaluerequired%
      \pgfkeysvalue{/errors/value required/.code}\pgfeov%
    \else%
      \pgfkeys@case@one%
    \fi%
  \fi}

\def\pgfkeys@case@one{%
  \ifpgfkeydefined{\pgfkeyscurrent/.code}%
  {\pgfkeysgetkey{\pgfkeyscurrent/.code}{\pgfkeys@code}%
   \expandafter\pgfkeys@code\pgfkeys@val\pgfeov}
  {\pgfkeys@case@two}%
}

\def\pgfkeys@case@two{%
  \ifpgfkeydefined{\pgfkeyscurrent/.value}%
  {\pgfletkey\pgfkeys@name\pgfkeys@val}
  {\pgfkeys@case@three}%
}

\def\pgfkeys@case@three{%
  \pgfkeys@split@path%
  \ifpgfkeydefined{/handlers/\pgfkeyscurrentwithoutpath/.code}%
  {\pgfkeysgetkey{/handlers/\pgfkeyscurrentwithoutpath/.code}{\pgfkeys@code}%
    \expandafter\pgfkeys@code\pgfkeys@val\pgfeov}
  {%
    \ifpgfkeydefined{\the\pgfkeyscurrentpath/.unknown/.code}%
    {%
      \pgfkeysgetkey{\the\pgfkeyscurrentpath/.unknown/.code}{\pgfkeys@code}%
      \expandafter\pgfkeys@code\pgfkeys@val\pgfeov}
    {%
      \pgfkeysgetkey{/handlers/.unknown/.code}{\pgfkeys@code}%
      \expandafter\pgfkeys@code\pgfkeys@val\pgfeov%
    }%
  }%
}


\def\pgfkey@argumentisspace#1{%
  \def\pgfkeys@spdef##1##2{%
    \futurelet\pgfkeys@possiblespace\pgfkeys@sp@a##2\pgfkeys@stop\pgfkeys@stop#1\pgfkeys@stop\relax##1}%
  \def\pgfkeys@sp@a{%
    \ifx\pgfkeys@possiblespace\pgfutil@sptoken%
      \expandafter\pgfkeys@sp@b%
    \else%
      \expandafter\pgfkeys@sp@b\expandafter#1%
    \fi}%
  \def\pgfkeys@sp@b#1##1 \pgfkeys@stop{\pgfkeys@sp@c##1}%
}
\pgfkey@argumentisspace{ }
\def\pgfkeys@sp@c#1\pgfkeys@stop#2\relax#3{\pgfkeys@temptoks{#1}\edef#3{\the\pgfkeys@temptoks}}





\def\pgfkeys@add@path@as@needed{% Should add the path if the
  % \pgfkeyscurrent does not start with /
  \expandafter\futurelet\expandafter\pgfkeys@possibleslash\expandafter\pgfkeys@check@slash\pgfkeyscurrent\relax%
}
\def\pgfkeys@check@slash{%
  \ifx\pgfkeys@possibleslash/%
    \expandafter\pgfkeys@nevermind%
  \else%
    \expandafter\pgfkeys@addpath%
  \fi%
}

\def\pgfkeys@nevermind#1\relax{}
\def\pgfkeys@addpath#1\relax{\edef\pgfkeyscurrent{\the\pgfkeysdefaultpath#1}}

\def\pgfkeys@split@path{% Should assign the two codes
                       % \pgfkeyscurrentwithoutpath and \pgfcurrentlkeypath
  \pgfkeyscurrentpath{}%
  \expandafter\pgfkeys@splitter\pgfkeyscurrent//%
}
\def\pgfkeys@splitter#1/#2/{%
  \def\pgfkeys@temp{#2}%
  \ifx\pgfkeys@temp\pgfutil@empty%
    % Ah. done
    \def\pgfkeyscurrentwithoutpath{#1}%
    \expandafter\pgfkeys@gobbletoslash%
  \else%
    \expandafter\pgfkeyscurrentpath\expandafter{\the\pgfkeyscurrentpath#1/}%
  \fi%
  \pgfkeys@splitter#2/%
}
\def\pgfkeys@gobbletoslash\pgfkeys@splitter/{\expandafter\pgfkeys@remove@slash\the\pgfkeyscurrentpath\relax}%
\def\pgfkeys@remove@slash#1/\relax{\pgfkeyscurrentpath{#1}}



% Sets keys while setting keys
%
% #1 = key-value pairs
%
% Desscription:
%
% This code may only be called inside the code that is executed for a
% key. The #1 should be a list of settings pairs. They will be executed
% as if they had been given as the argument to the \pgfset command.
%
% Example:
%
% \pgfset{tikz,myother length.set code=\def\myotherlength{#1}\pgfsetalso{length=#1}}

\def\pgfsetalso#1{\pgfkeys@parse#1,\relax,}




% Now setup the default handelers and keys:

% First, we setup a way of defining setting the .code:

\def\pgf@temp#1\pgfeov{
  \def\pgfkeys@temp##1\pgfeov{#1}%
  \pgfkeysletkey{\the\pgfkeyscurrentpath/.code}{\pgfkeys@temp}%
}
\pgfkeysletkey{/handlers/.set code/.code}{\pgf@temp}

\def\pgf@temp#1#2\pgfeov{%
  \def\pgfkeys@temp#1\pgfeov{#2}%
  \pgfkeysletkey{\the\pgfkeyscurrentpath/.code}{\pgfkeys@temp}%
  \pgfkeyssetkey{\the\pgfkeyscurrentpath/.args}{#1\pgfeov}%
  \pgfkeyssetkey{\the\pgfkeyscurrentpath/.body}{#2}%
}
\pgfkeysletkey{/handlers/.set code with args/.code}{\pgf@temp}

\pgfset{/handlers/.set code with no args/.set code=\pgfset{\the\pgfkeyscurrentpath/.set code with args={}{#1}}}
\pgfset{/handlers/.set code with 2 args/.set code=\pgfset{\the\pgfkeyscurrentpath/.set code with args={##1##2}{#1}}}

\pgfset{/handlers/.add code/.set code with 2 args=%
  % Find out, whether with args or not.
  \ifpgfkeydefined{\the\pgfkeyscurrentpath/.args}%
  {% Yes, so add to body and reuse args
    \pgfkeysaddkey{\the\pgfkeyscurrentpath/.body}{#1}{#2}%
    % Redefine code
    \pgfkeysgetkey{\the\pgfkeyscurrentpath/.args}{\pgfkeys@tempargs}
    \pgfkeysgetkey{\the\pgfkeyscurrentpath/.body}{\pgfkeys@tempbody}
    \expandafter\expandafter\expandafter\def\expandafter\pgfkeys@temp\expandafter\pgfkeys@tempargs\expandafter{\pgfkeys@tempbody}%
    \pgfkeysletkey{\the\pgfkeyscurrentpath/.code}{\pgfkeys@temp}%
  }%
  {%
    % No, so single argument. Redefine accordingly.
    {%
      \toks0{#1}%
      \pgfkeys@temptoks\expandafter\expandafter\expandafter{\csname pgfk@\the\pgfkeyscurrentpath/.code\endcsname{##1}}%
      \toks1{#2}%
      \expandafter
    }%
    \edef\pgfkeys@temp##1{\the\toks0 \the\pgfkeys@temptoks \the\toks1}%
    \pgfkeysletkey{#1}\pgfkeys@temp%
  }%
}
\pgfset{/handlers/.prepend code/.set code=\pgfset{\the\pgfkeyscurrentpath/.add code={#1}{}}}%
\pgfset{/handlers/.append code/.set code=\pgfset{\the\pgfkeyscurrentpath/.add code={}{#1}}}%

\pgfset{/handlers/.set key/.set code=\pgfkeyssetkey{\the\pgfkeyscurrentpath}{#1}}
\pgfset{/handlers/.get key/.set code=\pgfkeysgetkey{\the\pgfkeyscurrentpath}{#1}}
\pgfset{/handlers/.add key/.set code with 2 args=\pgfkeysaddkey{\the\pgfkeyscurrentpath}{#1}{#2}}%

\pgfset{/handlers/.set style/.set code=\pgfset{\the\pgfkeyscurrentpath/.set code=\pgfsetalso{#1}}}
\pgfset{/handlers/.set style with args/.set code with 2 args=\pgfset{\the\pgfkeyscurrentpath/.set code with args={#1}{\pgfsetalso{#2}}}}
\pgfset{/handlers/.set style with no args/.set code=\pgfset{\the\pgfkeyscurrentpath/.set code with no args=\pgfsetalso{#1}}}
\pgfset{/handlers/.set style with 2 args/.set code=\pgfset{\the\pgfkeyscurrentpath/.set code with 2 args=\pgfsetalso{#1}}}
\pgfset{/handlers/.add style/.set code with 2 args=\pgfset{\the\pgfkeyscurrentpath/.add code={\pgfsetalso{#1}}{\pgfsetalso{#2}}}}%
\pgfset{/handlers/.prepend style/.set code=\pgfset{\the\pgfkeyscurrentpath/.add code={\pgfsetalso{#1}}{}}}%
\pgfset{/handlers/.append style/.set code=\pgfset{\the\pgfkeyscurrentpath/.add code={}{\pgfsetalso{#1}}}}%

\pgfset{/handlers/.set value/.set code=\pgfkeyssetkey{\the\pgfkeyscurrentpath/.value}{#1}}
\pgfset{/handlers/.get value/.set code=\pgfkeysgetkey{\the\pgfkeyscurrentpath/.value}{#1}}
\pgfset{/handlers/.set default/.set code=\pgfkeyssetkey{\the\pgfkeyscurrentpath/.default}{#1}}
\pgfset{/handlers/.set value required/.set code=\pgfkeyssetkey{\the\pgfkeyscurrentpath/.default}{\pgfkeysvaluerequired}}

\pgfset{/handlers/.set store in code/.set code=\pgfsetkey{\the\pgfkeyscurrentpath/.set code=\def#1{##1}}}

\pgfset{/handlers/.set if/.set code=\pgfset{%
    \the\pgfkeyscurrentpath/.set code={%
      \pgfkeys@ifcsname#1##1\endcsname%
        \csname#1##1\endcsname%
      \else%
        \pgfset{/errors/boolean parameter=##1}%
      \fi},%
    \the\pgfkeyscurrentpath/.set default=true%
  }%
}

\pgfset{/handlers/.set choices/.set code=%
  \pgfset{%
    /path=\the\pgfkeyscurrentpath/,%
    .set code=\expandafter\pgfsetalso\expandafter{\pgfkeyscurrent/##1},
    .unknown/.set code=\PackageError{pgfkeys}{Choice '\pgfkeyscurrentwithoutpath' unknown in key '\the\pgfkeyscurrentpath'}{}}}


% Path handling

\pgfset{/path/.set code=\edef\pgfkeys@temp{#1}\pgfkeysdefaultpath\expandafter{\pgfkeys@temp}}
\pgfset{/path/.append/.set code=\edef\pgfkeys@temp{\the\pgfkeysdefaultpath#1}\pgfkeysdefaultpath\expandafter{\pgfkeys@temp}}


% Errors

\pgfset{/errors/boolean parameter/.set code=\PackageError{pgfkeys}{Boolean parameter must be true or false, not '#1'}{}}
\pgfset{/errors/value required/.set code=\PackageError{pgfkeys}{The key '\pgfkeyscurrent' requires a value}{}}

\pgfset{/handlers/.unknown/.set code=\PackageError{pgfkeys}{The key '\pgfkeyscurrent' is unknown}{}}