% Copyright 2007 by Till Tantau
%
% This file may be distributed and/or modified
%
% 1. under the LaTeX Project Public License and/or
% 2. under the GNU Public License.
%
% See the file doc/generic/pgf/licenses/LICENSE for more details.


% This file is perfectly self-contained, except that the catcode of @ should be made a letter. 


% Guard against reading twice
\ifx\pgfkeysloaded\undefined
  \let\pgfkeysloaded=\relax
\else
  \expandafter\endinput
\fi
  
% The purpose of this file is to provide a general settings engine that 
% works with all TeX formats and has no save-stack impact


% This is useful:

\def\pgfkeys@ifcsname#1\endcsname#2\else#3\fi{\expandafter\ifx\csname#1\endcsname\relax#3\else#2\fi}%
\ifx\eTeXrevision\undefined%
\else%
  \expandafter\let\expandafter\pgfkeys@ifcsname\csname ifcsname\endcsname%
\fi

\def\pgfkeys@empty{}


% Set a key to a value
%
% #1 = key
% #2 = a value
%
% Description:
%
% This command sets the key to the given value. The value is stored as
% is and can even contain things like #9. 
%
% Keys are organized hierarchically using something similar to Unix
% paths. Thus, a typically key might be called "/tikz/length" or
% "/tikz/length dimension/.@cmd". Some keys starting with a dot are
% special, so they should not be used as normal key names (they are
% similar to Unix files starting with a dot -- you can use them, but
% be careful).
%
% Keys are always local to the current TeX group.
%
% Example:
%
% \pgfkeysdefkey{/tikz/length/.@val}{2cm-3cm}
% \pgfkeysdefkey{/algo/swap}{{#2}{#1}}

\def\pgfkeyssetkey#1#2{%
  \pgfkeys@temptoks{#2}\expandafter\edef\csname pgfk@#1\endcsname{\the\pgfkeys@temptoks}%
}



% Add text to a key at the end
%
% #1 = key
% #2 = a value to be added at the beginning
% #3 = a value to be added at the end
%
% Description:
%
% This command adds #2 to the definition of the key. The key should
% have been set previously using \pgfkeyssetkey.
%
% Example:
%
% \pgfkeysaddkey{/tikz/length/.@val}{}{-3cm}

\def\pgfkeysaddkey#1#2#3{%
  {%
    \toks0{#1}%
    \ifpgfkeydefined{#1}
    {\pgfkeys@temptoks\expandafter\expandafter\expandafter{\csname pgfk@#1\endcsname}}%
    {\pgfkeys@temptoks{}}%
    \toks1{#2}%
    \xdef\pgfkeys@global@temp{\the\toks0 \the\pgfkeys@temptoks \the\toks1}% believe or don't: the spaces are important
  }%
  \pgfkeysletkey{#1}\pgfkeys@global@temp%
}



% Makes a key equal a given code
%
% #1 = key
% #2 = a code name
%
% Description:
%
% This command executes a \let command so that a key gets the same
% value as the parameter #2.
%
% Keys are always local to the current TeX group.
%
% Example:
%
% \pgfkeysletkey{/algo/swap}{\myswap}

\def\pgfkeysletkey#1#2{%
  \expandafter\let\csname pgfk@#1\endcsname#2%
}


% Retrieve the code stored in a key into a code
%
% #1 = key
% #2 = code
%
% Description:
%
% This command will set #2 to "point" to the value stored in the key.
%
% Example:
%
% \pgfkeysgetkey{/tikz/swap}{\myswap}

\def\pgfkeysgetkey#1#2{\expandafter\let\expandafter#2\csname pgfk@#1\endcsname}



% Retrieve the value stored in a key
%
% #1 = key
%
% Description:
%
% This command will expand to the value stored in the key. The key
% should previously have been set using \pgfsetkey or \pgfletkey. 
%
% Example:
%
% The length is \pgfkeysvalue{/tikz/length}.

\def\pgfkeysvalue#1{\csname pgfk@#1\endcsname}



% If for testing whether a key exists
%
% #1 = key
% #2 = if-case
% #3 = else-case
%
% Description:
%
% This if will be executed if the key exists. In eTeX mode this works
% like a normal if, in normal TeX mode you need to provide an \else.
%
% Example:
%
% \ifpgfkeydefined{/tikz/length}{key exists}{does not exist}

\def\ifpgfkeydefined#1#2#3{\pgfkeys@ifcsname pgfk@#1\endcsname#2\else#3\fi}




% Execute settings
%
% #1 = list of settings
%
% Description:
%
% The list of settings should contain comma-separated settings. Each
% setting has the following form:
%
% /path/key=value
%
% The parts "/path/" and "=value" are optional. When the path is not
% specified, the value of the token register "\pgfkeypath" is used. If
% "=value" is missing, the value of the setting "/path/key/.@def" is used
% instead. If this key is set to "\pgfvaluerequired", the key
% "/errors/value required/.@cmd" is executed. The code \pgfcurrentkey
% will store the name of the current key.
%
% Any spaces at the beginning and at the end and around the
% equals-sign are removed. The key with the complete path is set to
% the code \pgfcurrentkey.
%
% The setting is then processed according to the following rules:
%
% 1) If the key /path/key/.@cmd" is present, its code is executed
%    with the value computed above, followed by \pgfeov (end of
%    value). So, to handle
%
%    "/stuff/height=  1.5 ,"
%
%    /stuff/height/.@cmd should be set to some code, that can
%    handle the parameter
%
%    "1.5\pgfeov"
%
%    For instance, saying
%
%    \pgfsetkey{/stuff/height/.@cmd}{#1\pgfeov}{\def\myheight{#1}}
%
%    will do nicely.
%
% 2) Otherwise, if the key /path/key/.@val is present, this key is
%    set to the value computed above.
%
% 3) Otherwise, if the key /handlers/key/.@cmd is present, it is executed
%    with the same parameters as in 1). Additionally, the
%    token register \pgfcurrentkeypath will be set to "/path/" and the
%    macor \pgfcurrentkeywithoutpath to "key". So, in the above
%    example if neither "/stuff/height/.@cmd" nor
%    "/stuff/height/.@val" is present, but "/handlers/height" is,
%    then "/handlers/height" is executed with the parameters:
%
%    "1.5\pgfeov"
%
%    and \pgfcurrentkey is set to "/stuff/height" and \pgfcurrentkeypath
%    is set to "/stuff/" and \pgfcurrentkeywithoutpath to "height".
%
% 4) Otherwise, if the key "/path/.unknown/.@cmd" is present, its code is
%    executed with the same parameters as in 3).
%
% 5) Otherwise, the key "/handlers/.unknown/.@cmd" is executed with the same
%    parameters as in 1).
%
% After all settings have been processed, the value of the token
% register \pgfdefaultkeypath is set to its original value. Thus, any local
% change of this token register has no effect outside the call.
%
% Example:
%
% \pgfset{/tikz/.is family}
% \pgfset{/tikz/line width/.use as path,
%         .def=\pgfsetlinewidth{##1},
%         .set default=.4pt}
% \pgfset{tikz,line width=1pt}

\newtoks\pgfkeyscurrentpath
\newtoks\pgfkeys@temptoks

\def\pgfkeys@root{/}
\let\pgfkeysdefaultpath\pgfkeys@root

\def\pgfset{\expandafter\pgf@@set\expandafter{\pgfkeysdefaultpath}}%
\def\pgf@@set#1#2{\let\pgfkeysdefaultpath\pgfkeys@root\pgfkeys@parse#2,\pgfkeys@mainstop\def\pgfkeysdefaultpath{#1}}

\def\pgfkeys@parse{\futurelet\pgfkeys@possiblerelax\pgfkeys@parse@main}
\def\pgfkeys@parse@main{%
  \ifx\pgfkeys@possiblerelax\pgfkeys@mainstop%
    \expandafter\pgfkeys@cleanup%
  \else%
    \expandafter\pgfkeys@normal%
  \fi%
}
\def\pgfkeys@normal#1,{%
  \pgfkeys@unpack#1=\pgfkeysnovalue=\pgfkeys@stop%
  \pgfkeys@parse%
}
\def\pgfkeys@cleanup\pgfkeys@mainstop{}

\def\pgfkeys@mainstop{\pgfkeys@mainstop} % equals only itself
\def\pgfkeysnovalue{\pgfkeysnovalue} % equals only itself
\def\pgfkeysvaluerequired{\pgfkeysvaluerequired} % equals only itself

\def\pgfkeys@unpack#1=#2=#3\pgfkeys@stop{%
  \pgfkeys@spdef\pgfkeyscurrent{#1}%
  \edef\pgfkeyscurrent{\pgfkeyscurrent}%
  \ifx\pgfkeyscurrent\pgfkeys@empty%
    % Skip
  \else%
    \pgfkeys@add@path@as@needed%
    \pgfkeys@spdef\pgfkeys@val{#2}%
    \ifx\pgfkeys@val\pgfkeysnovalue% Hmm... no value
      \ifpgfkeydefined{\pgfkeyscurrent/.@def}%
      {\pgfkeysgetkey{\pgfkeyscurrent/.@def}{\pgfkeys@val}}
      {}% no default, so leave it
    \fi%
    \ifx\pgfkeys@val\pgfkeysvaluerequired%
      \pgfkeysvalue{/errors/value required/.@cmd}\pgfeov%
    \else%
      \pgfkeys@case@one%
    \fi%
  \fi}

\def\pgfkeys@case@one{%
  \ifpgfkeydefined{\pgfkeyscurrent/.@cmd}%
  {\pgfkeysgetkey{\pgfkeyscurrent/.@cmd}{\pgfkeys@code}%
   \expandafter\pgfkeys@code\pgfkeys@val\pgfeov}
  {\pgfkeys@case@two}%
}

\def\pgfkeys@case@two{%
  \ifpgfkeydefined{\pgfkeyscurrent/.@val}%
  {\pgfletkey\pgfkeys@name\pgfkeys@val}
  {\pgfkeys@case@three}%
}

\def\pgfkeys@case@three{%
  \pgfkeys@split@path%
  \ifpgfkeydefined{/handlers/\pgfkeyscurrentwithoutpath/.@cmd}%
  {\pgfkeysgetkey{/handlers/\pgfkeyscurrentwithoutpath/.@cmd}{\pgfkeys@code}%
    \expandafter\pgfkeys@code\pgfkeys@val\pgfeov}
  {%
    \ifpgfkeydefined{\the\pgfkeyscurrentpath/.unknown/.@cmd}%
    {%
      \pgfkeysgetkey{\the\pgfkeyscurrentpath/.unknown/.@cmd}{\pgfkeys@code}%
      \expandafter\pgfkeys@code\pgfkeys@val\pgfeov}
    {%
      \pgfkeysgetkey{/handlers/.unknown/.@cmd}{\pgfkeys@code}%
      \expandafter\pgfkeys@code\pgfkeys@val\pgfeov%
    }%
  }%
}


\def\pgfkey@argumentisspace#1{%
  \def\pgfkeys@spdef##1##2{%
    \futurelet\pgfkeys@possiblespace\pgfkeys@sp@a##2\pgfkeys@stop\pgfkeys@stop#1\pgfkeys@stop\relax##1}%
  \def\pgfkeys@sp@a{%
    \ifx\pgfkeys@possiblespace\pgfkeys@sptoken%
      \expandafter\pgfkeys@sp@b%
    \else%
      \expandafter\pgfkeys@sp@b\expandafter#1%
    \fi}%
  \def\pgfkeys@sp@b#1##1 \pgfkeys@stop{\pgfkeys@sp@c##1}%
}
\pgfkey@argumentisspace{ }
\def\pgfkeys@sp@c#1\pgfkeys@stop#2\relax#3{\pgfkeys@temptoks{#1}\edef#3{\the\pgfkeys@temptoks}}
{\def\:{\global\let\pgfkeys@sptoken= } \: }



\def\pgfkeys@add@path@as@needed{% Should add the path if the
  % \pgfkeyscurrent does not start with /
  \expandafter\futurelet\expandafter\pgfkeys@possibleslash\expandafter\pgfkeys@check@slash\pgfkeyscurrent\relax%
}
\def\pgfkeys@check@slash{%
  \ifx\pgfkeys@possibleslash/%
    \expandafter\pgfkeys@nevermind%
  \else%
    \expandafter\pgfkeys@addpath%
  \fi%
}

\def\pgfkeys@nevermind#1\relax{}
\def\pgfkeys@addpath#1\relax{\edef\pgfkeyscurrent{\pgfkeysdefaultpath#1}}

\def\pgfkeys@split@path{% Should assign the two codes
                       % \pgfkeyscurrentwithoutpath and \pgfcurrentlkeypath
  \pgfkeyscurrentpath{}%
  \expandafter\pgfkeys@splitter\pgfkeyscurrent//%
}
\def\pgfkeys@splitter#1/#2/{%
  \def\pgfkeys@temp{#2}%
  \ifx\pgfkeys@temp\pgfkeys@empty%
    % Ah. done
    \def\pgfkeyscurrentwithoutpath{#1}%
    \expandafter\pgfkeys@gobbletoslash%
  \else%
    \expandafter\pgfkeyscurrentpath\expandafter{\the\pgfkeyscurrentpath#1/}%
  \fi%
  \pgfkeys@splitter#2/%
}
\def\pgfkeys@gobbletoslash\pgfkeys@splitter/{\expandafter\pgfkeys@remove@slash\the\pgfkeyscurrentpath\relax}%
\def\pgfkeys@remove@slash#1/\relax{\pgfkeyscurrentpath{#1}}


% Sets keys while setting keys
%
% #1 = key-value pairs
%
% Desscription:
%
% This code may only be called inside the code that is executed for a
% key. The #1 should be a list of settings pairs. They will be executed
% as if they had been given as the argument to the \pgfset command.
%
% Example:
%
% \pgfset{tikz,myother length/.def=\def\myotherlength{#1}\pgfsetalso{length=#1}}

\def\pgfsetalso#1{\pgfkeys@parse#1,\pgfkeys@mainstop}



% Now setup the default handelers and keys:

% Define a key macro with one argument (\def or \edef)
%
% #1 = key
% #2 = code
%
% Description:
%
% This command will setup things so the key/.@cmd contains a macro
% that takes one parameter and has #2 as its code.
%
% Example:
%
% \pgfkeysdef{/my key}{\show#1}

\def\pgfkeysdef#1#2{%
  \def\pgfkeys@temp##1\pgfeov{#2}%
  \pgfkeysletkey{#1/.@cmd}{\pgfkeys@temp}%
}
\def\pgfkeysedef#1#2{%
  \edef\pgfkeys@temp##1\pgfeov{#2}%
  \pgfkeysletkey{#1/.@cmd}{\pgfkeys@temp}%
}


% Define a key macro with mutliple arguments (\def or \edef)
%
% #1 = key
% #2 = argument pattern
% #2 = code
%
% Description:
%
% This command will setup things so the key/.@cmd contains a macro
% that takes #2 as its parameter pattern and has #3 as its code.
%
% Example:
%
% \pgfkeysdefargs{/swap}{#1#2}{#2#1}

\def\pgfkeysdefargs#1#2#3{%
  \def\pgfkeys@temp#2\pgfeov{#3}%
  \pgfkeysletkey{#1/.@cmd}{\pgfkeys@temp}%
  \pgfkeyssetkey{#1/.@args}{#2\pgfeov}%
  \pgfkeyssetkey{#1/.@body}{#3}%
}
\def\pgfkeysedefargs#1#2#3{%
  \edef\pgfkeys@temp#2\pgfeov{#3}%
  \pgfkeysletkey{#1/.@cmd}{\pgfkeys@temp}%
  \pgfkeyssetkey{#1/.@args}{#2\pgfeov}%
  \pgfkeyssetkey{#1/.@body}{#3}%
}



% First, we setup a way of defining setting the .cmd:

\pgfkeysdef{/handlers/.def}{\pgfkeysdef{\the\pgfkeyscurrentpath}{#1}}
\pgfkeysdef{/handlers/.def 2 args}{\pgfkeysdefargs{\the\pgfkeyscurrentpath}{##1##2}{#1}}
\pgfkeysdef{/handlers/.edef}{\pgfkeysedef{\the\pgfkeyscurrentpath}{#1}}
\pgfkeysdef{/handlers/.edef 2 args}{\pgfkeysedefargs{\the\pgfkeyscurrentpath}{##1##2}{#1}}
\pgfkeysdefargs{/handlers/.def args}{#1#2}{\pgfkeysdefargs{\the\pgfkeyscurrentpath}{#1}{#2}}
\pgfkeysdefargs{/handlers/.edef args}{#1#2}{\pgfkeysedefargs{\the\pgfkeyscurrentpath}{#1}{#2}}

\pgfset{/handlers/.add def/.def 2 args=%
  % Find out, whether with args or not.
  \ifpgfkeydefined{\the\pgfkeyscurrentpath/.@args}%
  {% Yes, so add to body and reuse args
    \pgfkeysaddkey{\the\pgfkeyscurrentpath/.@body}{#1}{#2}%
    % Redefine code
    \pgfkeysgetkey{\the\pgfkeyscurrentpath/.@args}{\pgfkeys@tempargs}
    \pgfkeysgetkey{\the\pgfkeyscurrentpath/.@body}{\pgfkeys@tempbody}
    \expandafter\expandafter\expandafter\def\expandafter\pgfkeys@temp\expandafter\pgfkeys@tempargs\expandafter{\pgfkeys@tempbody}%
    \pgfkeysletkey{\the\pgfkeyscurrentpath/.@cmd}{\pgfkeys@temp}%
  }%
  {%
    % No, so single argument. Redefine accordingly.
    {%
      \toks0{#1}%
      \ifpgfkeydefined{\the\pgfkeyscurrentpath/.@cmd}%
      {\pgfkeys@temptoks\expandafter\expandafter\expandafter{\csname pgfk@\the\pgfkeyscurrentpath/.@cmd\endcsname##1\pgfeov}}%
      {\pgfkeys@temptoks{}}%
      \toks1{#2}%
      \xdef\pgfkeys@global@temp##1\pgfeov{\the\toks0 \the\pgfkeys@temptoks \the\toks1}%
    }%
    \pgfkeysletkey{\the\pgfkeyscurrentpath/.@cmd}\pgfkeys@global@temp%
  }%
}
\pgfset{/handlers/.prepend def/.def=\pgfset{\the\pgfkeyscurrentpath/.add def={#1}{}}}%
\pgfset{/handlers/.append def/.def=\pgfset{\the\pgfkeyscurrentpath/.add def={}{#1}}}%

\pgfset{/handlers/.key/.def=\pgfkeyssetkey{\the\pgfkeyscurrentpath}{#1}}
\pgfset{/handlers/.get key/.def=\pgfkeysgetkey{\the\pgfkeyscurrentpath}{#1}}
\pgfset{/handlers/.add key/.def 2 args=\pgfkeysaddkey{\the\pgfkeyscurrentpath}{#1}{#2}}%

\pgfset{/handlers/.style/.def=\pgfset{\the\pgfkeyscurrentpath/.def=\pgfsetalso{#1}}}
\pgfset{/handlers/.estyle/.def=\pgfset{\the\pgfkeyscurrentpath/.edef=\noexpand\pgfsetalso{#1}}}
\pgfset{/handlers/.style args/.def 2 args=\pgfset{\the\pgfkeyscurrentpath/.def args={#1}{\pgfsetalso{#2}}}}
\pgfset{/handlers/.estyle args/.def 2 args=\pgfset{\the\pgfkeyscurrentpath/.edef args={#1}{\noexpand\pgfsetalso{#2}}}}
\pgfset{/handlers/.style 2 args/.def=\pgfset{\the\pgfkeyscurrentpath/.def 2 args=\pgfsetalso{#1}}}
\pgfset{/handlers/.add style/.def 2 args=\pgfset{\the\pgfkeyscurrentpath/.add def={\pgfsetalso{#1}}{\pgfsetalso{#2}}}}%
\pgfset{/handlers/.prepend style/.def=\pgfset{\the\pgfkeyscurrentpath/.add def={\pgfsetalso{#1}}{}}}%
\pgfset{/handlers/.append style/.def=\pgfset{\the\pgfkeyscurrentpath/.add def={}{\pgfsetalso{#1}}}}%

\pgfset{/handlers/.value/.def=\pgfkeyssetkey{\the\pgfkeyscurrentpath/.@val}{#1}}
\pgfset{/handlers/.add value/.def 2 args=\pgfkeysaddkey{\the\pgfkeyscurrentpath/.@val}{#1}{#2}}
\pgfset{/handlers/.get value/.def=\pgfkeysgetkey{\the\pgfkeyscurrentpath/.@val}{#1}}

\pgfset{/handlers/.default/.def=\pgfkeyssetkey{\the\pgfkeyscurrentpath/.@def}{#1}}
\pgfset{/handlers/.value required/.def=\pgfkeyssetkey{\the\pgfkeyscurrentpath/.@def}{\pgfkeysvaluerequired}}

\pgfset{/handlers/.store in/.def=\pgfset{\the\pgfkeyscurrentpath/.def=\def#1{##1}}}

\pgfset{/handlers/.is if/.def=\pgfset{%
    \the\pgfkeyscurrentpath/.def={%
      \pgfkeys@ifcsname#1##1\endcsname%
        \csname#1##1\endcsname%
      \else%
        \pgfset{/errors/boolean parameter=##1}%
      \fi},%
    \the\pgfkeyscurrentpath/.default=true%
  }%
}

\pgfset{/handlers/.is choice/.def=%
  \pgfset{%
    \the\pgfkeyscurrentpath/.use as path,%
    .def=\expandafter\pgfsetalso\expandafter{\pgfkeyscurrent/##1},
    .unknown/.style=/errors/unknown choice value}}

% Inspection handlers

\pgfset{/handlers/.show key/.def=\pgfkeysgetkey{\the\pgfkeyscurrentpath}{\pgfkeysshower}\show\pgfkeysshower} % inspect the key itself
\pgfset{/handlers/.show value/.def=\pgfkeysgetkey{\the\pgfkeyscurrentpath/.@val}{\pgfkeysshower}\show\pgfkeysshower} % inspect the value itself
\pgfset{/handlers/.show def/.def=\pgfkeysgetkey{\the\pgfkeyscurrentpath/.@cmd}{\pgfkeysshower}\show\pgfkeysshower} % inspect the body of the command


% Path handling

\pgfset{/handlers/.is family/.def=\pgfset{\the\pgfkeyscurrentpath/.edef=\edef\noexpand\pgfkeysdefaultpath{\the\pgfkeyscurrentpath/}}}
\pgfset{/handlers/.use as path/.def=\edef\pgfkeysdefaultpath{\the\pgfkeyscurrentpath/}}


% Utilities

\pgfset{/utils/exec/.def=#1} % simply execute the given code directly.


% Errors

\pgfset{/errors/boolean parameter/.def=\PackageError{pgfkeys}{Boolean parameter must be true or false, not '#1'}{}}
\pgfset{/errors/value required/.def=\PackageError{pgfkeys}{The key '\pgfkeyscurrent' requires a value}{}}
\pgfset{/errors/unknown choice value/.def=\PackageError{pgfkeys}{Choice '\pgfkeyscurrentwithoutpath' unknown in key '\the\pgfkeyscurrentpath'}{}}

\pgfset{/handlers/.unknown/.def=\PackageError{pgfkeys}{The key '\pgfkeyscurrent' is unknown}{}}



\endinput
