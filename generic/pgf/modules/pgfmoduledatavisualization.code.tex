% Copyright 2006 by Till Tantau
%
% This file may be distributed and/or modified
%
% 1. under the LaTeX Project Public License and/or
% 2. under the GNU Public License.
%
% See the file doc/generic/pgf/licenses/LICENSE for more details.

\ProvidesFileRCS $Header: /cvsroot/pgf/pgf/generic/pgf/modules/pgfmoduledatavisualization.code.tex,v 1.2 2008/02/27 22:43:21 tantau Exp $




% When a set of datapoints is processed, one or more visualization
% automata are invoked.
%
% The first step is to define an automaton using
% \pgfdeclaredatavisualizationautomaton, which takes the description of
% the automaton. Second, to actually visualize something, you use a
% data visualization environment. Inside it, you setup a list of
% automata that you would like to use. Then comes some code that
% produces datapoints. For each datapoint all the automata are invoked
% in the order given to the data visualization. Then the next data
% point is  processed and so on. A "side effect" of the automata
% should be to create the graphic.



% Declare a data visualization automaton
%
% #1 = name of the automaton
% #2 = code setting up initial state attributes
%
% Description:
%
% Data visualization automata underlie the data visualization
% process. Conceptually, for each datapoint of the data set the
% visualization automaton can execute some code. This code can have
% side effects like a line being drawn. This code will be executed
% inside a local scope and it should not try to set global variables;
% the reason being that multiple instances of the automaton may
% actually be working on the datapoints at the same time. The
% automaton can, however, store information in its state, which is
% managed by the data visualization environment. For this, the
% commands \pgfsetstateattribute and \pgfstateattributevalueof can be
% used. You can also use the abbreviations \set and \valueof for these
% commands inside a data visualization environment.
% 

\def\pgfdeclaredatavisualizationautomaton#1#2{%
  \expandafter\def\csname pgf@dva@#1\endcsname{#2}%
}


% Sets a state attribute
%
% #1 = attribute name
% #2 = macro name or value
%
% Desccription:
%
% This command sets the attribute named #1 to #2 for the currently
% active automaton. If #2 does not start with a brace, a \let is used
% for the assignment, otherwise a \def is used.
%
% This command can only be called if there is an activate automaton,
% which is only the case inside a data visualization.

\def\pgfsetstateattribute#1{%
  \pgfutil@ifnextchar\bgroup{%
    \expandafter\gdef\csname pgf@dv@state@\the\pgf@dva@cur @#1\endcsname%
  }{%
    \expandafter\global\expandafter\let\csname pgf@dv@state@\the\pgf@dva@cur @#1\endcsname%
  }%
}


% Insert the value of a state attribute
%
% #1 = attribute name
%
% Desccription:
%
% This command inserts the value of the attribute named #1 for the
% currently active automaton.

\def\pgfstateattributevalueof#1{%
  \csname pgf@dv@state@\the\pgf@dva@cur @#1\endcsname%
}





% The main data visualization environment
%
% Description:
%
% The data visualization environment should contain calls of the
% following two macros:
%
% \pgfusedatavisualizationautomaton
% \pgfdatapoint
%
% Between these calls, aribtrary code may be executed.
%

\newcount\pgf@dva@num
\newcount\pgf@dva@cur

\def\pgfdatavisualization{%
  \begingroup%
    \pgf@dva@num=0\relax%  
    \pgf@dva@cur=0\relax%
    \let\pgforig@set=\set%
    \let\pgforig@valueof=\valueof%
    \let\set=\pgfsetstateattribute%
    \let\valueof=\pgfstateattributevalueof%
}
\def\endpgfdatavisualization{%
    % Create final data point
    \pgfdatapoint{final}%
  \endgroup%
}

% Restore the meaning of \set and \valueof
%
% Description:
%
% Inside a data visualization, the macro \set is set to mean the same
% as \pgfsetstateattribute and \valueof is set to mean the same as
% \pgfstateattributevalueof. If the original meaning of these macros
% are needed, the command \pgfrestoresetandvalueof can be used to
% locally restore their original meaning.
%
% Normally, this command should be used whenever user-defined text is
% to be typeset inside a data visualization.

\def\pgfrestoresetandvalueof{%
  \let\set=\pgforig@set%
  \let\valueof=\pgforig@valueof%
}


% Use a data visualization automaton inside the current data
% visualization environment.
%
% #1 = automaton path
% #2 = code for setting the initial state
%
% Description:
%
% This command is used ...

\def\pgfusedatavisualizationautomaton#1#2{%
  {%
    % Invoke startup code to set initial state
    \csname pgf@dva@#1\endcsname%
    #2%
    \csname pgf@dv@state@\the\pgf@dva@cur @init\endcsname%
  }%
  \advance\pgf@dva@num by1\relax%
  \advance\pgf@dva@cur by1\relax%
}



% Create a data point 
%
% #1 = actions to be taken for the data point
%
% Description:
%
% When this command is used, the current setting of the TeX state
% constitute a "data point". Typically, only the current values of the
% keys /data point/... are important, but this is just a convention.
%
% When a data point is created, the current automata of the data
% visualization are activated. There may be more than one automaton
% active and they are activated in the order they have been
% created. For each automaton, the code stored in the state attribute
% named "#1" is executed, in the order the automata were
% given.
%
% It is the job of the code stored in these attributes to "do
% something useful". Typically, the code will inspect some of the
% /data point/... keys and issue appropriate drawing commands. 

\def\pgfdatapoint#1{%
  % Invoke the visualization automata...
  {%
    \pgf@dva@cur=0\relax%
    \loop%
    \ifnum\pgf@dva@cur<\pgf@dva@num%
      \csname pgf@dv@state@\the\pgf@dva@cur @#1\endcsname%
      \advance\pgf@dva@cur by1\relax%
    \repeat%
  }%
}



\endinput
