% Copyright 2006 by Till Tantau
%
% This file may be distributed and/or modified
%
% 1. under the LaTeX Project Public License and/or
% 2. under the GNU Public License.
%
% See the file doc/generic/pgf/licenses/LICENSE for more details.

\ProvidesFileRCS $Header: /cvsroot/pgf/pgf/generic/pgf/modules/pgfmoduledatavisualization.code.tex,v 1.19 2008/11/04 22:43:25 tantau Exp $

\usepgfmodule{oo,shapes}


% This module defines the basic framework for data visualization. 

% In order to visualize data, you first need data. The format for this
% data is not specified, indeed, different formats are possible. A
% data point is created each time the command \pgfdatapoint is
% used. The "parameters" of the data point are just the current values
% of the keys or macros in the current scope. 
%
% A set of data points created using the \pgfdata command.
% When a data point is created, a number of signals are emitted, see
% the description of \pgfdatapoint. To actually visualize something,
% objects should be created that listen to these signals and that
% handle them. 
%
% The following class manages a data visualization

\pgfooclass{data visualization}
{
  % Class data visualization
  %
  % This class is used to "manage" a data visualization. It provides
  % methods for hooking into the data visualization process and its
  % constructor initializes the signals that are issued during a data
  % visualization. 
  %
  % When a data visualization object is created, a whole bunch of
  % signal objects. You should then create objects that connect to
  % these signals. They will be emitted when datapoints come
  % available.
  %
  % It is permissible to have several data visualization objects
  % active at the same time.
  %
  % To use a data visualization object, you should (possibly
  % repeatedly) call the method add data() or the macro \pgfdata. You should also
  % create transformation, mapping and visualization objects. Then,
  % you should, first, call the method survey, which will "survey" the
  % data, allowing the mapping and bounding objects to compute the
  % correct ranges. You may then create further objects based on this
  % data. Then, you should call the "visualize" method, which will
  % invoke the visualization signals for the data points.


  % These attribute store code that should be executed before or after
  % the survey phase.
  \attribute before survey;
  \attribute after survey;


  % As above
  \attribute before visualization;
  \attribute after visualization;

  
  % Stores the to-be-visualized data
  \attribute data;

  
  % Stores the signal objects
  \attribute prepare datapoint signal;
  \attribute map datapoint signal;
  \attribute transform datapoint signal;
  \attribute visualize datapoint signal;
  \attribute finish datapoint signal;
  \attribute survey datapoint signal;
  \attribute phase signal;

  
  % Constructor
  %
  % Inits the signals
  \method data visualization() {
    \pgfoonew \pgf@signalpreparedatapoint=new signal()%
    \pgfoolet{prepare datapoint signal}\pgf@signalpreparedatapoint
    \pgfoonew \pgf@signalmapdatapoint=new signal()%
    \pgfoolet{map datapoint signal}\pgf@signalmapdatapoint
    \pgfoonew \pgf@signaltransformdatapoint=new signal()%
    \pgfoolet{transform datapoint signal}\pgf@signaltransformdatapoint
    \pgfoonew \pgf@signalvisualizedatapoint=new signal()%
    \pgfoolet{visualize datapoint signal}\pgf@signalvisualizedatapoint
    \pgfoonew \pgf@signalfinishdatapoint=new signal()%
    \pgfoolet{finish datapoint signal}\pgf@signalfinishdatapoint
    \pgfoonew \pgf@signalsurveydatapoint=new signal()%
    \pgfoolet{survey datapoint signal}\pgf@signalsurveydatapoint
    \pgfoonew \pgf@signalphase=new signal()%
    \pgfoolet{phase signal}\pgf@signalphase
    %
    % Store this object in a key
    %
    \pgfoothis.get handle(\pgf@dv@me)
    \pgfkeyslet{/pgf/data visualization/obj}\pgf@dv@me
  }

  % The phase signal will emit the following constants:
  \def\pgfdvbeginsurvey{1}
  \def\pgfdvendsurvey{2}
  \def\pgfdvbeginvisualization{3}
  \def\pgfdvendvisualization{4}


  % Method
  %
  % Connect the object #1's slot "#2" to the signal named "#3"
  \method connect(#1,#2,#3) {
    \pgfoovalueof{#3}.connect(#1,#2)
  }
  
  % Method
  %
  % Add data that is to be visualized. The code #1 should call the
  % \pgfdatapoint macro for each data point it creates.
  \method add data(#1) {
    \pgfooappend{data}{#1}
  }
  

  % Method
  %
  % Add code to be added before the survey phase
  \method before survey(#1) {
    \pgfooappend{before survey}{#1}
  }

  % Method
  \method after survey(#1) {
    \pgfooprefix{after survey}{#1}
  }
  
  % Method
  \method before visualization(#1) {
    \pgfooappend{before visualization}{#1}
  }

  % Method
  \method after visualization(#1) {
    \pgfooprefix{after visualization}{#1}
  }


  % Method
  %
  % Copy the signals to macros. This is just for efficiency (ha!)
  \method prepare signal macros() {
    \pgfooget{prepare datapoint signal}\pgf@signalpreparedatapoint
    \pgfooget{map datapoint signal}\pgf@signalmapdatapoint
    \pgfooget{transform datapoint signal}\pgf@signaltransformdatapoint
    \pgfooget{visualize datapoint signal}\pgf@signalvisualizedatapoint
    \pgfooget{finish datapoint signal}\pgf@signalfinishdatapoint
    \pgfooget{survey datapoint signal}\pgf@signalsurveydatapoint
    \pgfooget{phase signal}\pgf@signalphase
  }

  
  % Survey method
  %
  % Call this method to "survey" the data. This should be done before
  % the "visualize" method is called.
  \method survey() {
    % Survey phase.
    \let\pgfdatapoint=\pgfdatapoint@surveyphase%
    \pgfoothis.prepare signal macros()
    \pgfoovalueof{before survey}%
    \pgf@signalphase.emit(\pgfdvbeginsurvey);
    \pgfoovalueof{data}%
    \pgf@signalphase.emit(\pgfdvendsurvey)%
    \pgfoovalueof{after survey}
  }

  
  % Visualize method
  %
  % This method will cause the actual visualization.
  \method visualize() {
    % Visualization phase.
    \let\pgfdatapoint=\pgfdatapoint@visualizationphase%
    \pgfoothis.prepare signal macros()
    \pgfoovalueof{before visualization}
    \pgf@signalphase.emit(\pgfdvbeginvisualization)%
    \pgfoovalueof{data}%
    \pgf@signalphase.emit(\pgfdvendvisualization)%
    \pgfoovalueof{after visualization}
  }
}  




% Create and handle a data point
%
% Description:
%
% This command is called by the survey and the visualize methods
% whenever a complete data point has been produced. Depending on the
% current circumstances, different signals will be emitted.
%
% The data that is represented by the data point is not given as a
% parameter. Rather, it is stored in macros and keys, that is, the
% data point is conceptually given by the settings of all the keys and
% macros in the local scope.
%
% There are two phases to data processing: In the survey phase data
% points are produced and handled in order to find out things like
% their number or the minimum and maximum values of attributes, so
% that axes and picture sizes can be prepared correctly. In the
% visualization phase, data point are actually shown. 
%
% During the survey phase, for each data point the signal
% "surveydatapoint" is emitted.
%
% During the visualization phase, more signals are emitted. A prepare
% signal is emitted first, giving all objects a 
% chance to "prepare" for the data point. Note that it is permissible
% for an object to manipulate the data point here (and also
% in later on).
%
% Next, the command \pgfcanvaspositionofdatapoint is called. Mainly,
% the effect of this command is to setup the keys /data point/canvas x
% and /data point/canvas y, see the description of this command for
% more details. 
%
% The next step consists of signaling "visualize data point". Objects
% listening to this will cause some form of visualization of the
% data point to occur. 
%
% Before the visualization is started, it is checked whether the key
% /data point/name is set (to a non-empty value). If so,
% a coordinate is created with the given canvas x and y values and
% this key's value as name.
%
% Finally, finish data point allows objects to do any final processing
% of the data point.

\def\pgfdatapoint@surveyphase{%
  \pgf@signalpreparedatapoint.emit()%
  \pgf@signalmapdatapoint.emit()%
  \pgf@signalsurveydatapoint.emit()%
}

\def\pgfdatapoint@visualizationphase{%
  \pgf@signalpreparedatapoint.emit()%
  \pgfcanvaspositionofdatapoint%
  \pgfkeysifdefined{/data point/name}
  {%
    \pgfcoordinate{\pgfkeysvalueof{/data point/name}}{\pgfpointcanvasposition}%
  }{}%
  \pgf@signalvisualizedatapoint.emit()%
  \pgf@signalfinishdatapoint.emit()%
}



% Compute a position of a data point
%
% Description:
%
% This command uses a special signal to compute the position where a
% data point should be visualized on the canvas. In detail, the
% following happens:
%
% A local scope is created and the
% transformation matrix is reset. Then, two signals are emitted: First,
% "map data points" and then "transform data point". The first
% lets listening objects "map" the object by setting up
% attributes of the data point. The second asks objects
% listening to this signal to transform  the current transformation
% matrix. After the signals, we compute where 
% the origin lies inside this transformed coordinate system. Then the
% two keys /data point/canvas x and /data point/canvas y are set to
% the values of this position. The local scope ends (but the settings
% of the keys persist by a bit of magic), thus restoring the
% transformation matrix to its original value.

\def\pgfcanvaspositionofdatapoint{%
  \pgf@canvaspositionofdatapoint{\pgfdatapointvirtualfalse}%
}
\def\pgf@canvaspositionofdatapoint#1{%
  {
    \pgftransformreset
    #1%
    \pgf@signalmapdatapoint.emit()%
    \pgf@signaltransformdatapoint.emit()%
    \pgfpointtransformed{\pgfpointorigin}
    % Smuggle outside group
    \expandafter
  }%
  \edef\pgf@marshal{%
    \noexpand\pgfkeyssetvalue{/data point/canvas x}{\the\pgf@x}
    \noexpand\pgfkeyssetvalue{/data point/canvas y}{\the\pgf@y}
  }%
  \pgf@marshal%
}

% Help functions for locating a canvas data point
%
% Description:
%
% The first function returns the data point computed by a
% canvasposition... call. The second function stores this position in
% a macro

\def\pgfpointcanvasposition{%
  \pgfqpoint{\pgfkeysvalueof{/data point/canvas x}}{\pgfkeysvalueof{/data point/canvas y}}%
}

\def\pgfsettocanvasposition#1{%
  \edef#1{\noexpand\pgfqpoint{\pgfkeysvalueof{/data point/canvas x}}{\pgfkeysvalueof{/data point/canvas y}}}%
}



% Compute a position of a virtual data point
%
% Description:
%
% This macros works like the previous macro, but the data point is
% considered to be "virtual". This happens when you wish to compute
% the position where a data point should be visualized even when no
% "real" data point is there. For virtual data points, bounders and
% coordinate trackers should not update internal data
% structures. They can detect whether a data point is virtual by
% testing \ifpgfdatapointvirtual

\newif\ifpgfdatapointvirtual
\def\pgfcanvaspositionofvirtualdatapoint{%
  \pgf@canvaspositionofdatapoint{\pgfdatapointvirtualtrue}%
}



%
% Special path constructions commands
%
%
% The following commands are used to construct paths based on
% datapoints.
%
% The moveto-command is easy: It simply causes a moveto to the
% position of the current datapoint (which is virtual). 




%
%
% Data parsing and formatting
%
%



% Run the rendering pipeline on a dataset.
% 
% #1 = options with path /pgf/dataset/
% #2 = optionally data given inline in curly braces.
%
% Description:
%
% This command is define a data sets. For a single data visualization,
% multiple data sets can be defined, they will accumulate. Data can be
% in different formats, as specified by the "format" key, and you can
% define new formats.  
%
% The setting of the following keys is important:
%
% /pgf/data visualization/obj contains a handle to the dv-object
% /pgf/data/format stores the format (see below)
% /pgf/data/source determines where the data is.
%
% If source is empty, the data is stored in the argument that
% follows. Otherwise, the file whose name is stored in source is read.  
% This data is stored in an internal variable, which is local to the
% current group.
%
% After the group, the key "/pgf/data/continue code" will be
% executed.
%
% When the data is actually used later on (during a survey or a
% visualization), independently of what source is used, a format
% handler is started for each data set. This works as follows: first,
% the handler's startup code is executed. Then for each line of the
% data file/the data given inline, the line handler function is
% called. Finally, the data format end handler is called. 
%
% The format handler's job is to call \pgfdatapoint each time a
% complete data point has been produced.
%
% Example:
%
% \pgfoonew \dv=new data visualization()
% \pgfoonew \obj=new attribute mapper(...)
% ...
%
% \pgfkeys{/pgf/dataset/.cd,
%          column 1=dax/low,
%          column 2=dax/high,
%          column 3=dax/entry,
%          column 4=dax/exit}
%
% \pgfdata[format=space separated columns]
% {
%   % today
%   2000 2300 2100 2200 
%   2000 2350 2200 2500
%   2200 2300 2250 2260 
%   1800 2260 2260 1900 
%   2000 2300 2100 2200
% }
%
% \pgfdata[format=comma separated columns]
% {
%   % yesterday
%   2000, 2350, 2200, 2250 
%   2200, 2300, 2250, 2260 
% }
%
% \pgfdata[source=data,format=comma separated columns]
%
% \dv.survey()
% \dv.visualize()

\def\pgfdata{\pgfutil@ifnextchar[{\pgf@dataset@data@opt}{\pgf@dataset@data@opt[]}}%}
\def\pgf@dataset@data@opt[#1]{%
  % Ok, add one data thing...
  \pgfkeysvalueof{/pgf/data visualization/obj}.add data(\pgf@do@data{#1})%
  \begingroup%
    \pgfkeys{/pgf/data/.cd,/pgf/every data/.try,#1}%
    \pgfkeysgetvalue{/pgf/data/continue code}\pgf@dv@cont@code%
    \global\let\pgf@dv@cont@code\pgf@dv@cont@code%
    \pgfkeysgetvalue{/pgf/data/format}\pgf@dv@format%
    \expandafter\let\expandafter\pgf@dv@format@catcodes\csname pgfdv@format@\pgf@dv@format @catcodes\endcsname%
    \ifx\pgf@dv@format@catcodes\relax
      \PackageError{pgf}{Unknown data format '\pgf@dv@format'}{}%
    \else%
      \pgfkeysgetvalue{/pgf/data/source}\pgf@dv@source%
      \ifx\pgf@dv@source\pgfutil@empty%
        \let\pgf@next\pgf@dataset@grab@inline%
      \else%
        \let\pgf@next\pgf@dataset@grab@external%
      \fi%
      \pgf@next%
}
\def\pgf@dataset@grab@inline{%
  \pgfutil@ifnextchar\bgroup{%
    \begingroup%
    \catcode`\^^M=\active%
    \pgf@dv@format@catcodes%
    \pgf@dataset@grab@@inline}%
  {\PackageError{pgf}{Opening brace expected}{}}%
}
\def\pgf@dataset@grab@external{%
    \fi%
  \endgroup%
  \pgfkeysvalueof{/pgf/data visualization/obj}.add data({{{\pgf@dataset@do@external}}})%
  \pgf@dv@cont@code%
}

{\catcode`\^^M=\active%
  \gdef\pgf@dataset@grab@@inline#1{%
    \endgroup%
    \pgfkeysvalueof{/pgf/data visualization/obj}.add data({{{\pgf@dataset@do@inline#1^^M\pgf@@eol}}})%
    \fi%
  \endgroup%
  \pgf@dv@cont@code%
  }%
}%
\def\pgf@@eol{\pgf@eol}

\def\pgf@do@data#1#2{%
  \begingroup%
    \pgfkeys{/pgf/data/.cd,/pgf/every data/.try,#1}%
    \pgfkeysgetvalue{/pgf/data/format}\pgf@dv@format%
    \expandafter\let\expandafter\pgf@dv@format@line\csname pgfdv@format@\pgf@dv@format @line\endcsname%
    \expandafter\let\expandafter\pgf@dv@format@emptyline\csname pgfdv@format@\pgf@dv@format @empty\endcsname%
    \csname pgfdv@format@\pgfkeysvalueof{/pgf/data/format}@startup\endcsname%
    #2%
    \csname pgfdv@format@\pgfkeysvalueof{/pgf/data/format}@end\endcsname%
  \endgroup%
}


%
% Read external file
%

\def\pgf@dataset@do@external{%
  \csname pgfdv@format@\pgfkeysvalueof{/pgf/data/format}@catcodes\endcsname%
  \immediate\openin1=\pgfkeysvalueof{/pgf/data/source} %
  \ifeof1\relax
     \PackageError{pgf}{Data source '\pgfkeysvalueof{/pgf/data/source}' not found}{}%
  \else
    \pgf@dataset@readline%
  \fi
  \immediate\closein1%
}

\def\pgf@partext{\par}%
\def\pgf@dataset@readline{%
  \immediate\read1 to \pgf@temp%
  \ifx\pgf@temp\pgf@partext%
    \pgf@dv@format@emptyline%
  \else%
    \ifx\pgf@temp\pgfutil@empty%
      \pgf@dv@format@emptyline%
    \else%
      \expandafter\pgf@dv@format@line\pgf@temp\pgfeol%
    \fi%
  \fi%
  \ifeof1\else\expandafter\pgf@dataset@readline\fi%
}


%
% Read inline data
%

\def\pgf@dataset@do@inline{%
  \pgf@dv@handle@line%
}

{\catcode`\^^M=\active%
\gdef\pgf@dv@handle@line{%
  \pgfutil@ifnextchar^^M{\pgf@dv@format@emptyline\expandafter\pgf@dv@handle@line\pgfutil@gobble}%
  {\pgfutil@ifnextchar\pgf@@eol{\pgfutil@gobble}{\pgf@dv@handle@nonemptyline}}%
}%
\gdef\pgf@dv@handle@nonemptyline#1^^M{%
  \pgf@dv@format@line#1\pgfeol%
  \pgf@dv@handle@line%
}%
}

\pgfkeys{
  /pgf/data/data visualization obj/.initial=\undefined,
  /pgf/data/format/.initial=,
  /pgf/data/source/.initial=,
  /pgf/data/continue code/.initial=
}






% Define a data format
%
% #1 = format name
% #2 = catcode code
% #3 = startup code
% #4 = line arguments
% #5 = line code
% #6 = empty line code
% #7 = end code
%
% Description:
%
% This command defines a new data format for data visualization. When
% a data set is visualized and the format is set to #1, this handler
% is used to parse the data.
%
% In detail, the \dataset command will select a source. Before this
% source is read, #2 will be executed to setup the
% catcodes. Additionally, each time the data is parsed, #3 will be
% called. Then, for each nonempty line of the source, the
% command #5 is executed, where the line will be matched against the
% argument pattern given in #4. For empty lines, #6 will be executed
% instead. At the end of the source, #7 will be executed.

\def\pgfdeclaredataformat#1#2#3#4#5#6#7{%
  \expandafter\def\csname pgfdv@format@#1@catcodes\endcsname{#2}%
  \expandafter\def\csname pgfdv@format@#1@startup\endcsname{#3}%
  \expandafter\def\csname pgfdv@format@#1@line\endcsname#4\pgfeol{#5}%
  \expandafter\def\csname pgfdv@format@#1@empty\endcsname{#6}%
  \expandafter\def\csname pgfdv@format@#1@end\endcsname{#7}%
}



%
% Predefined standard formats
%

% TeX code format
%
% Description:
%
% The lines of the data set are assumed to contains executable TeX
% code that will call \pgfdatapoint.
%
% Example:
%
% \pgfdatavisualizationrender[format=TeX code]
% \dataset{
%   \pgfkeyssetvalue{/data point/x}{5}
%   \pgfkeyssetvalue{/data point/y}{5}
%   \pgfdatapoint
%   \pgfkeyssetvalue{/data point/x}{6}
%   \pgfkeyssetvalue{/data point/y}{6}
%   \pgfdatapoint
% }

\pgfdeclaredataformat{TeX code}{}{}{#1}{#1 }{}{}



% Key-value lines format
%
% Description:
%
% The lines of the data set are passed to \pgfkeys with the path set
% to /data point.
%
% Example:
%
% \pgfdatavisualizationrender[format=key value pairs]
% \dataset{
%   x=5, y=6, hi=9
%   x=7, y=6, lo=10
% }

\pgfdeclaredataformat{key value pairs}{}{}{#1}{\pgfkeys{/data point/.cd,#1}\pgfdatapoint}{}{}



% Comma separated columns
%
% Description:
%
% Each input line for this format should contain values separated by
% commas. For each nonempty line, the first value will be stored in
% /data point/<col1>, where <col1> is the value stored in /pgf/data
% visualization/column 1, the second value is stored in /data
% point/<col2>, where <col2> is the value stored in /pgf/data
% visualization/column 2, and so on. If a <colX> is empty, the value
% is ignored.
%
% Example:
%
% \pgfdatavisualizationrender[format=space separated columns,
%             column 1=dax/low,
%             column 2=dax/high,
%             column 3=dax/entry,
%             column 4=dax/exit]
% \dataset{
%   % today
%   2000, 2300, 2100, 2200 
%   2000, 2350, 2200  
%   2200, 2300, 2250, 2260 
%   1800, 2260, 2260, 1900 
%   2000, 2300, 2100, 2200
%    
%   % yesterday
%   2000, 2350, 2200, 2250 
%   2200, 2300, 2250, 2260 
%   1800, 2260, 2260, 1900 
% }

\pgfkeys{/pgf/data/.cd,
  column 1/.initial=x,
  column 2/.initial=y,
  column 3/.initial=z,
  column 4/.initial=,
  column 5/.initial=,
  column 6/.initial=,
  column 7/.initial=,
  column 8/.initial=}

\pgfdeclaredataformat{comma separated columns}{}{\pgf@dv@separated@get}{#1}
{\pgf@dv@csc#1,,,,,,,,,\pgfeol}{}{}

\def\pgf@dv@csc#1,#2,#3,#4,#5,#6,#7,#8,#9\pgfeol{%
  \ifx\pgf@dv@cola\pgfutil@empty\else\pgfkeyssetvalue{/data point/\pgf@dv@cola}{#1}\fi%
  \ifx\pgf@dv@colb\pgfutil@empty\else\pgfkeyssetvalue{/data point/\pgf@dv@colb}{#2}\fi%
  \ifx\pgf@dv@colc\pgfutil@empty\else\pgfkeyssetvalue{/data point/\pgf@dv@colc}{#3}\fi%
  \ifx\pgf@dv@cold\pgfutil@empty\else\pgfkeyssetvalue{/data point/\pgf@dv@cold}{#4}\fi%
  \ifx\pgf@dv@cole\pgfutil@empty\else\pgfkeyssetvalue{/data point/\pgf@dv@cole}{#5}\fi%
  \ifx\pgf@dv@colf\pgfutil@empty\else\pgfkeyssetvalue{/data point/\pgf@dv@colf}{#6}\fi%
  \ifx\pgf@dv@colg\pgfutil@empty\else\pgfkeyssetvalue{/data point/\pgf@dv@colg}{#7}\fi%
  \ifx\pgf@dv@colh\pgfutil@empty\else\pgfkeyssetvalue{/data point/\pgf@dv@colh}{#8}\fi%
  \pgfdatapoint%
}

\def\pgf@dv@separated@get{%
  \pgfkeysgetvalue{/pgf/data/column 1}\pgf@dv@cola%
  \pgfkeysgetvalue{/pgf/data/column 2}\pgf@dv@colb%
  \pgfkeysgetvalue{/pgf/data/column 3}\pgf@dv@colc%
  \pgfkeysgetvalue{/pgf/data/column 4}\pgf@dv@cold%
  \pgfkeysgetvalue{/pgf/data/column 5}\pgf@dv@cole%
  \pgfkeysgetvalue{/pgf/data/column 6}\pgf@dv@colf%
  \pgfkeysgetvalue{/pgf/data/column 7}\pgf@dv@colg%
  \pgfkeysgetvalue{/pgf/data/column 8}\pgf@dv@colh%
}


% Semicolon separated columns
%
% Description:
%
% Like comma separated columns, only with a semicolon as separator

\pgfdeclaredataformat{semicolon separated columns}{}{\pgf@dv@separated@get}{#1}
{\pgf@dv@scsc#1;;;;;;;;;\pgfeol}{}{}

\def\pgf@dv@scsc#1;#2;#3;#4;#5;#6;#7;#8;#9\pgfeol{%
  \ifx\pgf@dv@cola\pgfutil@empty\else\pgfkeyssetvalue{/data point/\pgf@dv@cola}{#1}\fi%
  \ifx\pgf@dv@colb\pgfutil@empty\else\pgfkeyssetvalue{/data point/\pgf@dv@colb}{#2}\fi%
  \ifx\pgf@dv@colc\pgfutil@empty\else\pgfkeyssetvalue{/data point/\pgf@dv@colc}{#3}\fi%
  \ifx\pgf@dv@cold\pgfutil@empty\else\pgfkeyssetvalue{/data point/\pgf@dv@cold}{#4}\fi%
  \ifx\pgf@dv@cole\pgfutil@empty\else\pgfkeyssetvalue{/data point/\pgf@dv@cole}{#5}\fi%
  \ifx\pgf@dv@colf\pgfutil@empty\else\pgfkeyssetvalue{/data point/\pgf@dv@colf}{#6}\fi%
  \ifx\pgf@dv@colg\pgfutil@empty\else\pgfkeyssetvalue{/data point/\pgf@dv@colg}{#7}\fi%
  \ifx\pgf@dv@colh\pgfutil@empty\else\pgfkeyssetvalue{/data point/\pgf@dv@colh}{#8}\fi%
  \pgfdatapoint%
}


% Colon separated columns
%
% Description:
%
% Like comma separated columns, only with a colon as separator

\pgfdeclaredataformat{colon separated columns}{}{\pgf@dv@separated@get}{#1}
{\pgf@dv@scsc#1:::::::::\pgfeol}{}{}

\def\pgf@dv@scsc#1:#2:#3:#4:#5:#6:#7:#8:#9\pgfeol{%
  \ifx\pgf@dv@cola\pgfutil@empty\else\pgfkeyssetvalue{/data point/\pgf@dv@cola}{#1}\fi%
  \ifx\pgf@dv@colb\pgfutil@empty\else\pgfkeyssetvalue{/data point/\pgf@dv@colb}{#2}\fi%
  \ifx\pgf@dv@colc\pgfutil@empty\else\pgfkeyssetvalue{/data point/\pgf@dv@colc}{#3}\fi%
  \ifx\pgf@dv@cold\pgfutil@empty\else\pgfkeyssetvalue{/data point/\pgf@dv@cold}{#4}\fi%
  \ifx\pgf@dv@cole\pgfutil@empty\else\pgfkeyssetvalue{/data point/\pgf@dv@cole}{#5}\fi%
  \ifx\pgf@dv@colf\pgfutil@empty\else\pgfkeyssetvalue{/data point/\pgf@dv@colf}{#6}\fi%
  \ifx\pgf@dv@colg\pgfutil@empty\else\pgfkeyssetvalue{/data point/\pgf@dv@colg}{#7}\fi%
  \ifx\pgf@dv@colh\pgfutil@empty\else\pgfkeyssetvalue{/data point/\pgf@dv@colh}{#8}\fi%
  \pgfdatapoint%
}




% Space separated columns
%
% Description:
%
% Each input line for this format should contain values separated by
% spaces. Otherwise, the format works like comma separated columns.
%
% Example:
%
% \pgfdatavisualizationrender[format=space separated columns,
%             column 1=dax/low,
%             column 2=dax/high,
%             column 3=dax/entry,
%             column 4=dax/exit]
% {
%   % today
%   2000 2300 2100 2200 
%   2000 2350 2200  
%   2200 2300 2250 2260 
%   1800 2260 2260 1900 
%   2000 2300 2100 2200
%    
%   % yesterday
%   2000 2350 2200 2250 
%   2200 2300 2250 2260 
%   1800 2260 2260 1900 
% }


\pgfdeclaredataformat{space separated columns}{}{\pgf@dv@separated@get}{#1}
{\pgf@dv@ssc#1 @\pgf @ @\pgf @ @\pgf @ @\pgf @ @\pgf @ @\pgf @ @\pgf @ @\pgf @\pgfeol}{}{}

\def\pgf@dv@ssc#1 #2 #3 #4 #5 #6 #7 #8 #9\pgfeol{%
  \pgf@dv@css@handle\pgf@dv@cola{#1}%
  \pgf@dv@css@handle\pgf@dv@colb{#2}%
  \pgf@dv@css@handle\pgf@dv@colc{#3}%
  \pgf@dv@css@handle\pgf@dv@cold{#4}%
  \pgf@dv@css@handle\pgf@dv@cole{#5}%
  \pgf@dv@css@handle\pgf@dv@colf{#6}%
  \pgf@dv@css@handle\pgf@dv@colg{#7}%
  \pgf@dv@css@handle\pgf@dv@colh{#8}%
  \pgfdatapoint%
}
\def\pgf@dv@css@handle#1#2{%
  \ifx#1\pgfutil@empty%
  \else%
    \def\pgf@temp{#2}%
    \ifx\pgf@temp\pgf@atpgfattext%
      \let\pgf@temp\pgfutil@empty%
    \fi%
    \pgfkeyslet{/data point/#1}\pgf@temp%
  \fi%
}
\def\pgf@atpgfattext{@\pgf @}




%
%
% Standard objects for data visualization
% 
%


%
% Transformers
%

\pgfooclass{line transformer}
{
  % Class line transformer
  %
  % This class is a transformer class. It reacts to the transform
  % datapoint signal. When this signal is raised, it will shift the
  % coordinate system as follows: If the current value of /data
  % point/this.attribute  is 0, then the system is shifted to the
  % current value of this.origin. If the current value is 1, the
  % system is shifted to this.origin + this.unit vector. For other
  % values of /data point/this.attribuate, the shift is interpolated
  % between these two values. 

  
  \attribute attribute;
  % The attribute (/data point/this.attribute) by which
  % the unit vector is multiplied. If it is empty (which is
  % different from 0), no transformation is done at all.

  \attribute origin = \pgfpointorigin;
  % The shift in case the /data point/this.attribute is 0.

  \attribute unit vector;
  % The coordinate system is additionally shifted by this amount
  % times the current value of /data point/this.attribute.

  
  % Constructor
  %
  % #1 = attribute that is being transformed. Example: x
  % #2 = a vector corresponding to one unit of #1.
  %      Example: \pgfpoint{1cm}{0cm}
  \method line transformer(#1,#2) {
    \pgfooset{attribute}{#1}
    \pgf@process{#2}
    \edef\pgf@temp{\noexpand\pgfqpoint{\the\pgf@x}{\the\pgf@y}}
    \pgfoolet{unit vector}\pgf@temp%
  }

  % Method
  \method default connects() {
    \pgfoothis.get handle(\pgf@dv@me)
    \pgfkeysvalueof{/pgf/data visualization/obj}.connect(\pgf@dv@me,transform,transform datapoint signal)
  }
  
  % Setter
  \method set origin(#1) {
    \pgf@process{#1}
    \edef\pgf@temp{\noexpand\pgfqpoint{\the\pgf@x}{\the\pgf@y}}
    \pgfoolet{origin}\pgf@temp%
  }

  % Slot
  %
  % This slot should be connected to the transform datapoint
  % signal. When this signal is emitted, the coordinate system will be
  % shifted according to the current value of the attribute.
  \method transform() {
    \pgfkeysgetvalue{/data point/\pgfoovalueof{attribute}}\pgf@dv@val%
    \ifx\pgf@dv@val\pgfutil@empty%
    \else%
      \ifx\pgf@dv@val\relax%
      \else%
        \pgftransformshift{\pgfoovalueof{origin}}%
        \pgftransformshift{\pgfpointscale{\pgf@dv@val}{\pgfoovalueof{unit vector}}}%
      \fi%
    \fi%
  }

  % Shifts the coordinate system by the given amount
  %
  % This command will offset the coordinate system by the given
  % dimension in the direction of the unit vector.
  \method shift(#1) {
    \pgf@process{\pgfpointnormalised{\pgfoovalueof{unit vector}}}
    \pgfmathparse{#1}
    \pgftransformshift{\pgfqpoint{\pgfmathresult\pgf@x}{\pgfmathresult\pgf@y}}
  }
  
  % Shifts the coordinate system by the given amount orthogonally
  %
  % This command will offset the coordinate system by the given
  % dimension orthogonally to the direction of the unit vector.
  \method shift orthogonally(#1) {
    \pgf@process{\pgfpointnormalised{\pgfoovalueof{unit vector}}}
    \pgfmathparse{#1}
    \pgf@x=-\pgf@x
    \pgftransformshift{\pgfqpoint{\pgfmathresult\pgf@y}{\pgfmathresult\pgf@x}}
  }
  
}




%
% Mappers
%

\pgfooclass{mapper}
{
  % Class mapper
  %
  % This mapper reacts to the map datapoint signal. Its purpose is to
  % map one attribute to another attribute. For the first attribute,
  % two intervals are specified: The cut interval and the in
  % interval. Values outside the cut-interval are not mapped (they are
  % cut away). Note that if the min-part of an interval is missing,
  % this means that -infinity is the lower bound and if the max part
  % is missing, then the max-part is infinity.
  %
  % In addition to the in interval [a,b] there is also an outinterval
  % [c,d]. Values in the range [a,b] are linearly mapped to the
  % range [c,d] and the result is stored in the second attribute.
  %
  % For instance, if the first range is [10,20] and the second range
  % is [0,100], then 10 is mapped to 0, 11 is mapped to 10, 20 is
  % mapped to 100 and 30 is mapped to 200.
  %
  % It is permissible to specify an additional non-linear
  % transformation function f. In this case, for an input value x the
  % position of f(x) inside the interval [f(a),f(b)] is determined and
  % this position is linearly mapped to [c,d]. 
  %
  % For example, if f(x) = log_10 (x) and the first range [a,b] =
  % [10,1000] and the second range is [1,2], then 10 is mapped to 1,
  % 100 is mapped to 1.5 (since f(100) = 3 lies in the middle
  % between f(10) = 2 and f(1000) = 4) and 100000 is mapped to 3.
  

  \attribute in;
  % The name of the input attribute. If the value of this attribute
  % is empty or undefined, no mapping is done.

  \attribute cutter;
  % The code for cutting the interval. Will set \pgf@dvcuttrue, if a
  % cut is necessary.
  
  \attribute out;
  % The name of the output attribute.
  
  \attribute trans;
  % Stores (more or less) the transformation function.

  \attribute out min;
  % Start of the second range (the value of c).

  \attribute trans in min;
  % Transformed value of the start of the first range (the value
  % f(a)).

  \attribute scale;
  % The scaling factor, that is, the value of (f(d)-f(c))/(b-a).

  
  % Constructor
  %
  % #1 = input attribute. Example: velocity.
  % #2 = in interval object
  % #3 = output attribute. Example: x
  % #4 = out interval object
  % #5 = optional transformation function. The input value for this
  %      function is stored in the macro \pgfvalue, that is, when the
  %      function is called the macro will expand to something like
  %      "3.141". The output of the function should be stored in the
  %      macro \pgfmathresult.
  %
  % A cut interval is set using the method set cut interval.
  %
  \method mapper(#1,#2,#3,#4,#5) {
    {%
      % Let's start with the output, it's easier...
      \pgfooset{out}{#3}
      #4.get min and max()%
      \pgfmathparse{\pgfdvmin}
      \pgfooeset{out min}{\pgfmathresult}
      \pgfmathsubtract{\pgfdvmax}{\pgfdvmin}%
      \let\pgf@dv@out@diff=\pgfmathresult%
      % Now comes the transformation code:
      \def\pgf@temp{#5}%
      \ifx\pgf@temp\pgfutil@empty%
      \else%
        \pgfooset{trans}{%
          \let\pgfvalue\pgfmathresult%
          #5%
        }%
      \fi%
      % Ok, now the transformed input:
      \pgfooset{in}{#1}
      #2.get min and max()%
      \pgfmathparse{\pgfdvmin}
      \pgfoovalueof{trans}%
      \let\pgf@dv@min=\pgfmathresult
      \pgfooeset{trans in min}{\pgfmathresult}
      \pgfmathparse{\pgfdvmax}
      \pgfoovalueof{trans}%
      \pgfmathsubtract{\pgfmathresult}{\pgfoovalueof{trans in min}}%
      % 
      % Precompute the scaling
      %
      \let\pgf@dv@temp\pgfmathresult%
      \pgfmathdivide{\pgf@dv@out@diff}{\pgf@dv@temp}%
      \pgfooeset{scale}{\pgfmathresult}%
    }%
  } 

  % Method
  \method default connects() {
    \pgfoothis.get handle(\pgf@dv@me)
    \pgfkeysvalueof{/pgf/data visualization/obj}.connect(\pgf@dv@me,map,map datapoint signal)
  }

  % Method
  \method set cut interval(#1) {
    #1.get min and max()%
    \pgfooeset{cut}{%
      \noexpand\pgf@dvcutfalse%
      \ifx\pgfdvmin\pgfutil@empty%
      \else%
        \noexpand\ifdim\noexpand\pgfmathresult pt<\pgfdvmin pt\relax\noexpand\pgf@dvcuttrue\noexpand\fi%
      \fi%
      \ifx\pgfdvmax\pgfutil@empty%
      \else%
        \noexpand\ifdim\noexpand\pgfmathresult pt>\pgfdvmax pt\relax\noexpand\pgf@dvcuttrue\noexpand\fi%
      \fi%
    }
  }
  
  % Slot
  \method map() {
    \pgfkeysgetvalue{/data point/\pgfoovalueof{in}}\pgfmathresult%
    \ifx\pgfmathresult\pgfutil@empty%
    \else%
      \ifx\pgfmathresult\relax%
      \else%
        % Cut?
        \pgfoovalueof{cut}%
        \ifpgf@dvcut%
          \pgfkeyslet{/data point/\pgfoovalueof{out}}\pgfutil@empty%
        \else%
          \pgfoovalueof{trans}%
          \pgfmathsubtract{\pgfmathresult}{\pgfoovalueof{trans in min}}%
          \pgfmathmultiply{\pgfmathresult}{\pgfoovalueof{scale}}%
          \pgfmathadd{\pgfmathresult}{\pgfoovalueof{out min}}%
          \pgfkeyslet{/data point/\pgfoovalueof{out}}\pgfmathresult%
        \fi%
      \fi%
    \fi%
  }
}

\newif\ifpgf@dvcut
\newif\ifpgf@dvignore



\pgfooclass{polar mapper}
{
  % Class plor mapper
  %
  % A polar mapper is used to map attributes given as (2d) polar
  % coordinates to Cartesian coordinates. Note that no special ranges
  % can be specified and that the angle must be given in degrees. You
  % can, however, use a standard mapper to change these things.
  
  \attribute angle;
  % The attribute from which the angle is read.
  
  \attribute radius;
  % The attribute from which the radius is read.
  
  \attribute x;
  % The attribute to which the resulting x coordinate is written.
  
  \attribute y;
  % The attribute to which the resulting y coordinate is written. 

  
  % Constructor
  %
  % #1 = angle attribute. Example: rotation.
  % #2 = radius attribute. Example: height.
  % #3 = x attribute. Example: x.
  % #4 = y attribute. Example: y.
  %
  \method polar mapper(#1,#2,#3,#4) {
    \pgfooset{angle}{#1}
    \pgfooset{radius}{#2}
    \pgfooset{x}{#3}
    \pgfooset{y}{#4}
  } 

  % Method
  \method default connects() {
    \pgfoothis.get handle(\pgf@dv@me)
    \pgfkeysvalueof{/pgf/data visualization/obj}.connect(\pgf@dv@me,map,map datapoint signal)
  }

  % Slot
  %
  % Mapping is done only if both the angle and the radius attribute
  % have non-empty values.
  \method map() {
    \pgfkeysgetvalue{/data point/\pgfoovalueof{angle}}\pgf@dv@angle%
    \ifx\pgf@dv@angle\pgfutil@empty%
    \else%
      \ifx\pgf@dv@angle\relax%
      \else%
        \pgfkeysgetvalue{/data point/\pgfoovalueof{radius}}\pgf@dv@radius%
        \ifx\pgf@dv@radius\pgfutil@empty%
        \else%
          \ifx\pgf@dv@radius\relax%
          \else%
            \pgfmathsincos{\pgf@dv@angle}%
            \pgfmathmultiply{\pgfmathresultx}{\pgf@dv@radius}%
            \pgfkeyslet{/data point/\pgfoovalueof{x}}\pgfmathresult%
            \pgfmathmultiply{\pgfmathresulty}{\pgf@dv@radius}%
            \pgfkeyslet{/data point/\pgfoovalueof{y}}\pgfmathresult%
          \fi%
        \fi%
      \fi%
    \fi%
  }

  % Getter
  \method get angle and radius attributes() {
    \pgfooget{angle}\pgfdvangleattribute
    \pgfooget{radius}\pgfdvradiusattribute
  }
}




%
% Surveyors
%

\pgfooclass{range surveyor}
{
  % Class range surveyor
  %
  % A range surveyor is used only in the survey phase. Its job is to
  % determine the minimum and maximum values of an attribute that are
  % "seen" during the survey phase. Based on this value, the size of,
  % say, an axis can be detemined later on.

  \attribute attribute;
  % The to-be-surveyed attribute

  \attribute interval obj;
  % The interval object that protocols the minimum and maximum
  % values. 


  % Constructor
  %
  % #1 = the attribute
  % #2 = handle to the interval object
  %
  \method range surveyor(#1,#2) {
    \pgfooset{attribute}{#1}
    \pgfoolet{interval obj}#2
  }

  % Method
  \method default connects() {
    \pgfoothis.get handle(\pgf@dv@me)
    \pgfkeysvalueof{/pgf/data visualization/obj}.connect(\pgf@dv@me,survey,survey datapoint signal)
  }

  % Slot
  %
  % This slot should be connected to the survey datapoint signal. For
  % each datapoint, this method will call the adjust method of the
  % interval.
  %
  \method survey() {
    \pgfkeysgetvalue{/data point/\pgfoovalueof{attribute}}\pgf@dv@val%
    \pgfoovalueof{interval obj}.adjust(\pgf@dv@range@surv)
  }

  \def\pgf@dv@range@surv{%
    \ifpgfdatapointvirtual%
    \else%
      \ifx\pgf@dv@val\pgfutil@empty%
      \else%
        \ifx\pgf@dv@val\relax%
        \else%
          % Now, protocol value
          \ifx\pgfdvmin\pgfutil@empty%
            \let\pgfdvmin\pgf@dv@val%
          \else%
            \ifdim\pgf@dv@val pt<\pgfdvmin pt%
              \let\pgfdvmin\pgf@dv@val%
            \fi%
          \fi%
          \ifx\pgfdvmax\pgfutil@empty%
            \let\pgfdvmax\pgf@dv@val%
          \else%
            \ifdim\pgf@dv@val pt>\pgfdvmax pt%
              \let\pgfdvmax\pgf@dv@val%
            \fi%
          \fi%
        \fi%
      \fi%
    \fi
  }    
}




%
% Help class: Interval
%

\pgfooclass{interval}
{
  % Class interval
  %
  % Instances of this class store intervals. When either the min or
  % the max value is empty, this corresponds to "not yet set", (not to
  % "infinity). 

  \attribute min;
  % The minimum value of the interval
  
  \attribute max;
  % The maximum value of the interval

  % Constructor
  %
  % #1 = initial minimum value
  % #2 = initial maximum value
  %
  \method interval(#1,#2) {
    \pgfooset{min}{#1}
    \pgfooset{max}{#2}
  }

  % Method
  \method default connects() {
  }

  % Sets the minimum/maximum to a new value
  \method set min(#1) {
    \pgfooset{min}{#1}
  }  
  \method set max(#1) {
    \pgfooset{max}{#1}
  }

  % Returns the current values of the minimum and maximum in the
  % macros \pgfdvmin and \pgfdvmax.
  \method get min and max() {
    \pgfooget{min}\pgfdvmin
    \pgfooget{max}\pgfdvmax
  }

  % Adjusts the mimum and maximum values
  %
  % #1 = some code. When this code is executed, the values of
  %      \pgfdvmin and \pgfdvmax will be set to the current values of
  %      the minimum/maximum. The code may change these values. The
  %      changed values will then be stored.
  %
  % This method does nothing that cannot be achieved also by calling
  % set/get methods, but it is easier to use and faster.
  \method adjust(#1) {
    {%
      \pgfooget{min}\pgfdvmin
      \pgfooget{max}\pgfdvmax
      #1%
      \pgfoolet{min}\pgfdvmin
      \pgfoolet{max}\pgfdvmax
    }%
  }

  % Returns an interval that is "adjusted" according to given values
  %
  % Description:
  %
  % Returns the interval minimum and maximum, but adjusted in the
  % following ways: First, you can specify additional absolute
  % offsets, which will simply be added to the
  % minimum/maximum. Second, you can specify additional relative
  % offsets, which are expressed as multiples of the difference
  % between maximum and minimum. For instance, a relative maximum
  % offset of 0.1 means that the returned value will be the maximum
  % plus 10% of the difference between maximum and minimum.
  %
  % #1 = absolute minimum adjustment (typically negative, may be empty)
  % #2 = relative minimum adjustment (typically negative, may be empty)
  % #3 = absolute maximum adjustment (may be empty)
  % #4 = relative maximum adjustment (may be empty)
  %
  % The adjusted minimum and maximum valued are stored in \pgfdvmin
  % and \pgfdvmax.
  \method get adjusted interval(#1,#2,#3,#4) {
    \pgfooget{min}\pgfdvmin
    \pgfooget{max}\pgfdvmax
    \ifx\pgfdvmin\pgfutil@empty%
    \else%
      \ifx\pgfdvmax\pgfutil@empty%
      \else%
        \pgfmathsetmacro\pgf@dv@minpa{#1}% min adjust aboslute
        \pgfmathsetmacro\pgf@dv@minpr{#2}% min adjust relative
        \pgfmathsetmacro\pgf@dv@maxpa{#3}% max adjust aboslute
        \pgfmathsetmacro\pgf@dv@maxpr{#4}% max adjust relative
        \pgfmathsubtract{\pgfdvmax}{\pgfdvmin}%
        \let\pgf@dv@diff\pgfmathresult%
        % Start with minimum
        \let\pgfmathresult\pgfdvmin%
        \ifx\pgf@dv@minpa\pgfutil@empty%
        \else%
          \pgfmathadd{\pgfmathresult}{\pgf@dv@minpa}%
        \fi%
        \ifx\pgf@dv@minpr\pgfutil@empty%
        \else%
          \pgfmathadd{\pgfmathresult}{\pgf@dv@minpr*\pgf@dv@diff}%
        \fi%
        \let\pgfdvmin\pgfmathresult%  
        % Maximum next
        \let\pgfmathresult\pgfdvmax%
        \ifx\pgf@dv@maxpa\pgfutil@empty%
        \else%
          \pgfmathadd{\pgfmathresult}{\pgf@dv@maxpa}%
        \fi%
        \ifx\pgf@dv@maxpr\pgfutil@empty%
        \else%
          \pgfmathadd{\pgfmathresult}{\pgf@dv@maxpr*\pgf@dv@diff}%
        \fi%
        \let\pgfdvmax\pgfmathresult%  
      \fi%
    \fi%
  }
}



\pgfooclass{count}
{
  \attribute attribute;
  \attribute val=0;
  \attribute step=1;
  \attribute start val=0;
  \attribute zero action=;
  
  \method count(#1) {
    \pgfooset{attribute}{#1}
  }

  \method default connects() {
    \pgfoothis.get handle(\pgf@dv@me)
    \pgfkeysvalueof{/pgf/data visualization/obj}.connect(\pgf@dv@me,apply,prepare datapoint signal)
    \pgfkeysvalueof{/pgf/data visualization/obj}.connect(\pgf@dv@me,phase,phase signal)
  }

  \method set value(#1) {
    \pgfooset{val}{#1}
    \pgfkeyssetvalue{/data point/\pgfoovalueof{attribute}}{#1}%
  }

  \method set start value(#1) {
    \pgfooset{start val}{#1}
  }

  \method set step(#1) {
    \pgfooset{step}{#1}
  }

  \method set zero action(#1) {
    \pgfooset{zero action}{#1}
  }

  \method count down() {
    \pgfooset{step}{-1}
    \edef\pgf@temp{\noexpand\pgfooset{val}{\pgfoovalue{val}}}
    \pgfoolet{zero action}\pgf@temp
  }
  
  \method apply() {
    \pgfmathparse{\pgfoovalueof{val}+\pgfoovalueof{step}}%
    \pgfoolet{val}\pgfmathresult%
    \pgfkeyslet{/data point/\pgfoovalueof{attribute}}\pgfmathresult%
    \ifdim\pgfmathresult pt=0pt\relax%
      \pgfoovalueof{zero action}%
    \fi%
  }

  \method phase(#1) {
    \ifx#1\pgfdvbeginsurvey
      \pgfooeset{val}{\pgfoovalueof{start val}}
    \else \ifx#1\pgfdvbeginvisualization
      \pgfooeset{val}{\pgfoovalueof{start val}}
    \fi\fi
  }
}



\pgfooclass{sum}
{
  \attribute attribute;
  \attribute sum=0;
  
  \method sum(#1) {
    \pgfooset{attribute}{#1}
    \pgfoothis.reset()
  }

  \method default connects() {
    \pgfoothis.get handle(\pgf@dv@me)
    \pgfkeysvalueof{/pgf/data visualization/obj}.connect(\pgf@dv@me,apply,prepare datapoint signal)
    \pgfkeysvalueof{/pgf/data visualization/obj}.connect(\pgf@dv@me,phase,phase signal)
  }

  \method apply() {
    \pgfooget{sum}\pgf@temp
    \pgfkeyslet{/data point/\pgfoovalueof{attribute}/prev sum}\pgf@temp
    \pgfmathadd{\pgfkeysvalueof{/data point/\pgfoovalueof{attribute}}}{\pgfoovalueof{sum}}
    \pgfoolet{sum}\pgfmathresult
    \pgfkeyslet{/data point/\pgfoovalueof{attribute}/sum}\pgfmathresult
  }

  \method reset() {
    \pgfooset{sum}{0}
    \pgfkeyssetvalue{/data point/\pgfoovalueof{attribute}/prev sum}{0}
    \pgfkeyssetvalue{/data point/\pgfoovalueof{attribute}/sum}{0}
  }
  
  \method phase(#1) {
    \ifx#1\pgfdvbeginsurvey
      \pgfoothis.reset()
    \else \ifx#1\pgfdvbeginvisualization
      \pgfoothis.reset()
    \fi\fi
  }
}







% \pgfooclass{straight axis visualizer}
% {
%   \attribute attribute;
%   \attribute interval obj;
%   \attribute tick vec;
%   \attribute use path=\pgfusepath{stroke};
%   \attribute use tick path=\pgfusepath{stroke};
  
%   \method straight axis visualizer(#1,#2) {
%     \pgfooset{attribute}{#1}
%     \pgfoolet{interval obj}#2
%   }

%   \method default connects() {
%   }

%   \method set use path(#1) {
%     \pgfooset{use path}{#1}
%   }
  
%   \method set use tick path(#1) {
%     \pgfooset{use tick path}{#1}
%   }
  
%   \method visualize() {
%     {%
%       % we assume that the current data point settings determine where
%       % the axis should go
%       %
%       % We now compute the start and the end points
%       %
%       \pgfooget{interval obj}\pgf@dv@interval
%       \pgf@dv@interval.get adjusted interval()%
%       % Ok, got the values
%       % Save them.
%       \let\pgf@dv@sa@min=\pgfdvmin
%       \let\pgf@dv@sa@max=\pgfdvmax
%       \pgfkeyssetvalue{/data point/\pgfoovalueof{attribute}}{\pgf@dv@sa@min}%
%       \pgfcanvaspositionofvirtualdatapoint%
%       % Ok, that's where we start
%       \pgfpathmoveto{\pgfpointcanvasposition}%
%       % now the end:
%       \pgfkeyssetvalue{/data point/\pgfoovalueof{attribute}}{\pgf@dv@sa@max}%
%       \pgfcanvaspositionofvirtualdatapoint%
%       \pgfpathlineto{\pgfpointcanvasposition}%
%       % Finally, use the path:
%       \pgfoovalueof{use path}%
%     }%
%   }

%   \method visualize as tick() {
%     \pgfooget{tick vec}\pgf@dv@vec% cached tick vec
%     \ifx\pgf@dv@vec\pgfutil@empty%
%       {%
%         %
%         % This is only to compute the correct direction vector
%         %
%         \pgfooget{interval obj}\pgf@dv@interval
%         \pgf@dv@interval.get min and max()%
%         % Ok, got the values
%         % Save them.
%         \let\pgf@dv@sa@min=\pgfdvmin
%         \let\pgf@dv@sa@max=\pgfdvmax
%         \pgfkeyssetvalue{/data point/\pgfoovalueof{attribute}}{\pgf@dv@sa@min}%
%         \pgfcanvaspositionofvirtualdatapoint%
%         \pgf@process{\pgfpointcanvasposition}
%         \pgf@xc=\pgf@x%
%         \pgf@yc=\pgf@y%
%         % now the end:
%         \pgfkeyssetvalue{/data point/\pgfoovalueof{attribute}}{\pgf@dv@sa@max}%
%         \pgfcanvaspositionofvirtualdatapoint%
%         \pgf@process{\pgfpointnormalised{\pgfpointdiff{\pgfqpoint{\pgf@xc}{\pgf@yc}}{\pgfpointcanvasposition}}}
%       }
%       \edef\pgf@dv@vec{\noexpand\pgfqpoint{\the\pgf@x}{\the\pgf@y}}%
%       \pgfoolet{tick vec}\pgf@dv@vec%
%     \fi%
%     \pgfcanvaspositionofvirtualdatapoint%
%     {%
%       \pgftransformshift{\pgfpointcanvasposition}%
%       \pgfmathparse{\pgfkeysvalueof{/data point/tick lower length}}
%       \pgfpathmoveto{\pgfpointscale{-\pgfmathresult}{\pgf@dv@vec}}
%       \pgfmathparse{\pgfkeysvalueof{/data point/tick higher length}}
%       \pgfpathlineto{\pgfpointscale{\pgfmathresult}{\pgf@dv@vec}}
%       \pgfoovalueof{use tick path}%
%     }%
%   }
% }



% \pgfooclass{polar axis visualizer}
% {
%   \attribute mapper obj;
%   \attribute interval obj;
%   \attribute use patch=\pgfusepath{stroke};
  
%   \method polar axis visualizer(#1,#2) {
%     \pgfoolet{mapper obj}#1
%     \pgfoolet{interval obj}#2
%   }

%   \method default connects() {
%   }

%   \method set use path(#1) {
%     \pgfooset{use path}{#1}
%   }
  
%   \method visualize axis() {
%     {%
%       %
%       \pgfooget{mapper obj}\pgf@dv@trans%
%       \pgf@dv@trans.get angle and radius attributes()%
%       % we assume that the current data point settings determine where
%       % the axis should go
%       %
%       % we also assume that the axis will, indeed, be visualized as an
%       % arc. 
%       %
%       % Compute the origin:
%       {%
%         \pgfkeyssetvalue{/data point/\pgfdvradiusattribute}{0}%
%         \pgfcanvaspositionofvirtualdatapoint%
%         \pgfsettocanvasposition\pgf@temp
%         \global\let\pgf@dv@temp@orig=\pgf@temp%
%       }%
%       % Compute the first axis:
%       \pgfkeyssetvalue{/data point/\pgfdvangleattribute}{0}%
%       \pgfcanvaspositionofvirtualdatapoint%
%       \pgfsettocanvasposition\pgf@dv@veczero%
%       % Compute the second axis:
%       \pgfkeyssetvalue{/data point/\pgfdvangleattribute}{90}%
%       \pgfcanvaspositionofvirtualdatapoint%
%       \pgfsettocanvasposition\pgf@dv@vecninety%
%       %
%       % Start and end angle
%       %
%       \pgfooget{interval obj}\pgf@dv@interval
%       \pgf@dv@interval.get adjusted interval()%
%       % Ok, got the values
%       % Compute the first axis:
%       \pgfkeyssetvalue{/data point/\pgfdvangleattribute}{\pgfdvmin}%
%       \pgfcanvaspositionofvirtualdatapoint%
%       \pgfpathmoveto{\pgfpointcanvasposition}%
%       % Do arc:
%       \pgftransformtriangle{\pgf@dv@temp@orig}{\pgf@dv@veczero}{\pgf@dv@vecninety}%
%       \pgfpatharc{\pgfdvmin}{\pgfdvmax}{1pt}%
%       % Finally, use the path:
%       \pgfoovalueof{use path}%
%     }%
%   }
% }





%
% Visualizers
%

\usepgflibrary{plothandlers}


\pgfooclass{plot mark visualizer}
{
  \attribute mark;
  \attribute zero trigger=always 0;
  \attribute size=\pgfplotmarksize;
  
  \method plot mark visualizer(#1) {
    \pgfooset{mark}{#1}
  }

  \method default connects() {
    \pgfoothis.get handle(\pgf@dv@me)
    \pgfkeysvalueof{/pgf/data visualization/obj}.connect(\pgf@dv@me,visualize,visualize datapoint signal)
  }

  \method set size(#1) {
    \pgfooset{size}{#1}
  }

  \method set zero trigger(#1) {
    \pgfooset{zero trigger}{#1}
  }
  
  \method visualize() {
    {%
      \pgfkeysgetvalue{/data point/\pgfoovalueof{zero trigger}}\pgf@dv@trig%
      \ifx\pgf@dv@trig\pgf@dv@zero@text%
        \pgfplotmarksize\pgfoovalueof{size}
        \pgftransformshift{\pgfpointcanvasposition}
        \pgfuseplotmark{\pgfoovalueof{mark}}
      \fi%
    }%
  }

  \def\pgf@dv@zero@text{0}
}



\pgfooclass{plot handler visualizer}
{
  \attribute handler;
  \attribute use path=\pgfusepath{stroke};
  \attribute positions;
  
  \method plot handler visualizer(#1) {
    \pgfooset{handler}{#1}
  }

  \method default connects() {
    \pgfoothis.get handle(\pgf@dv@me)
    \pgfkeysvalueof{/pgf/data visualization/obj}.connect(\pgf@dv@me,protocol,visualize datapoint signal)
    \pgfkeysvalueof{/pgf/data visualization/obj}.connect(\pgf@dv@me,phase,phase signal)
  }

  \method set use path(#1) {
    \pgfooset{use path}{#1}
  }
  
  \method protocol() {
    % Just append it to positions
    \pgfooget{positions}\pgf@dv@temp
    \edef\pgf@dv@add{\noexpand\pgf@dv@ph{\pgfkeysvalueof{/data point/canvas x}}{\pgfkeysvalueof{/data point/canvas y}}}%
    \expandafter\expandafter\expandafter\def%
    \expandafter\expandafter\expandafter\pgf@dv@temp%
    \expandafter\expandafter\expandafter{\expandafter\pgf@dv@temp\pgf@dv@add}%
    \pgfoolet{positions}\pgf@dv@temp
  }

  \method phase(#1) {%
    \ifx#1\pgfdvendvisualization%
      \pgfoothis.render()%
    \fi%
  }
  
  \method render() {
    \pgfoovalueof{handler}
    \pgfplotstreamstart
    \pgfoovalueof{positions}
    \pgfplotstreamend
    \pgfoovalueof{use path}
    \pgfoolet{positions}\pgfutil@empty
  }

  \def\pgf@dv@ph#1#2{\pgfplotstreampoint{\pgfqpoint{#1}{#2}}}
}



%
%
% Help keys and, attributes
%
%

\pgfkeys{% do not even think of changing the values of the following:
  /data point/always true/.initial=true,
  /data point/always false/.initial=false,
  /data point/always 0/.initial=0,
  /data point/always 1/.initial=1,
  /data point/always empty/.initial=%
}

\pgfoonew \pgfdatavisualizationunitinterval=new interval(0,1)
\pgfoonew \pgfdatavisualizationemptyinterval=new interval(,)


\endinput
