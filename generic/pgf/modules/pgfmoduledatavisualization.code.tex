% Copyright 2006 by Till Tantau
%
% This file may be distributed and/or modified
%
% 1. under the LaTeX Project Public License and/or
% 2. under the GNU Public License.
%
% See the file doc/generic/pgf/licenses/LICENSE for more details.

\ProvidesFileRCS $Header: /cvsroot/pgf/pgf/generic/pgf/modules/pgfmoduledatavisualization.code.tex,v 1.1 2008/02/24 10:00:52 tantau Exp $




% Do something if a data point attribute is pgfnull.
%
% #1 = attribuate
% #2 = if-part
% #3 = else=part

\def\pgfifdatapointatributenull#1{%
  \expandafter\ifx\csname pgf@dv@@#1\endcsname\relax%
    \expandafter\pgfutil@firstoftwo%
  \else%
    \expandafter\pgfutil@secondoftwo%
  \fi%
}




%
% Part II: Visualization automata
%

% When a set of datapoints is processed, one or more visualization
% automata are invoked.
%
% The first step is to define an automaton using
% \pgfdeclaredatavisualizationautomaton, which takes the description of
% the automaton. Second, to actually visualize something, you use a
% data visualization environment. This takes a list of automata that
% you would like to use. Then comes some code that produces
% datapoints. For each datapoint all the automata are invoked in the
% order given to the data visualization. Then the next point is
% processed and so on. A "side effect" of the automata should be to
% create the graphic.



% Declare a data visualization automaton
%
% #1 = name of the automaton
% #2 = code setting up initial state attributes
%
% Description:
%
% Data visualization automata underlie the data visualization
% process. Conceptually, for each datapoint of the data set the
% visualization automaton can execute some code. This code can have
% side effects like a line being drawn. This code will be executed
% inside a local scope and it should not try to set global variables;
% the reason being that multiple instances of the automaton may
% actually be working on the datapoints at the same time. The
% automaton can, however, store information in its state, which is
% managed by the data visualization environment. For this, the two
% commands \pgfsetstateattribute and \pgfgetstateattribute can be
% used. 
% 

\def\pgfdeclaredatavisualizationautomaton#1#2{%
  \expandafter\def\csname pgf@dva@#1\endcsname{#2}%
}


% Sets a state attribute
%
% #1 = attribute name
% #2 = value
%
% Desccription:
%
% This command sets the attribute named #1 to value for the currently
% active automaton.
%
% This command can only be called if there is an activate automaton,
% which is only the case inside a data visualization.

\def\pgfsetstateattribute#1#2{%
  \expandafter\gdef\csname pgf@dv@state@\the\pgf@dva@cur @#1\endcsname{#2}%
}

% Get a state attribute
%
% #1 = attribute name
% #2 = macro
%
% Desccription:
%
% This command retrieves the attribute named #1 for the currently 
% active automaton.

\def\pgfgetstateattribute#1#2{%
  \expandafter\let\expandafter#2\csname pgf@dv@state@\the\pgf@dva@cur @#1\endcsname%
}

% Insert the value of a state attribute
%
% #1 = attribute name
%
% Desccription:
%
% This command inserts the value of the attribute named #1 for the
% currently active automaton.

\def\pgfstateattributevalueof#1{%
  \csname pgf@dv@state@\the\pgf@dva@cur @#1\endcsname%
}





% The main data visualization environment
%
% Description:
%
% The data visualization environment should contain calls of the
% following two macros:
%
% \pgfusedatavisualizationautomaton
% \pgfdatapoint
%
% Between these calls, aribtrary code may be executed.
%

\newcount\pgf@dva@num
\newcount\pgf@dva@cur

\def\pgfdatavisualization{%
  \begingroup%
    \pgf@dva@num=0\relax%  
    \pgf@dva@cur=0\relax%  
}

\def\endpgfdatavisualization{%
    % Invoke the final code for all automata
    \pgf@dv@foreachautomaton{\pgfstateattributevalueof{final preaction}}{\pgfstateattributevalueof{final postaction}}
  \endgroup%
}


% Use a data visualization automaton inside the current data
% visualization environment.
%
% #1 = automaton path
% #2 = code for setting the initial state
%
% Description:
%
% This command is used ...

\def\pgfusedatavisualizationautomaton#1#2{%
  {%
    % Invoke startup code to set initial state
    #2%
    \csname pgf@dva@#1\endcsname%
  }%
  \advance\pgf@dva@num by1\relax%
  \advance\pgf@dva@cur by1\relax%
}



% Create a datapoint 
%
% Description:
%
% \pgfdatapoint

\def\pgfdatapoint{%
  % Invoke the visualization automata...
  \pgf@dv@foreachautomaton{\pgfstateattributevalueof{preaction}}{\pgfstateattributevalueof{postaction}}%
}


\def\pgf@dv@foreachautomaton#1#2{%
  {%
    \pgf@dva@cur=0\relax%
    \loop%
    \ifnum\pgf@dva@cur<\pgf@dva@num%
      #1%
      \advance\pgf@dva@cur by1\relax%
    \repeat%
    \pgf@dva@cur=\pgf@dva@num\relax%
    \loop%
    \ifnum\pgf@dva@cur>0\relax%
      \advance\pgf@dva@cur by-11\relax%
      #2%
    \repeat%
  }%
}


\endinput
