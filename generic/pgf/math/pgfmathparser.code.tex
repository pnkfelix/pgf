% This file parses/evaluates a decimal expression.
%
% (c) 2007 Mark Wibrow
%
% but subject to the LaTeX Project Public License, 
% (http://www.latex-project.org/lppl.txt)
%
% and the GNU Public License,
% (http://www.gnu.org/licenses/gpl.txt)
%
% Version 0.0 08/03/2007


% \pgfmathparse
%
% Evaluates a decimal expression.
%
% #1 - the expression.
%
% returns
%
% x = the result as a dimension.
%
% E.g.
% \pgfmathparse{3pt*2cm+1.5}
% 
\def\pgfmathparse#1{%
  \begingroup%
    % Reinstall correct font, so that dimensions like 1em are correct
    \pgfutil@selectfont%
    % 
    \edef\pgfmath@temp{#1}%
    \pgfmath@resetparsingparameters%
    \global\pgfmath@unitsdeclaredfalse%
    \let\pgfmath@afterparsegroup\pgfmath@parseoperator%
    \expandafter\pgfmath@parse@\pgfmath@temp @@\pgfmath@empty}


% \pgfmath@resetparsingparameters
%
% Reset parsing parameters at begining of groups.
%
\def\pgfmath@resetparsingparameters{%
	\let\pgfmath@a\pgfmath@empty%
	\let\pgfmath@b\pgfmath@empty%
	\let\pgfmath@c\pgfmath@empty%
	\let\pgfmath@d\pgfmath@empty%
	\let\pgfmath@primaryoperation\pgfmath@empty%
	\let\pgfmath@secondaryoperation\pgfmath@empty%
	\let\pgfmath@tertiaryoperation\pgfmath@empty%
}

% \pgfmath@parse@
%
% Start parsing. Expect one of
% 1) the end of the parse
% 2) the start of a group
% 3) a (possible) operand.
%
\def\pgfmath@parse@#1{%
	\def\pgfmath@token{}%
	\ifx#1@%
		\let\pgfmath@parsenext\pgfmath@endparse%
	\else%
		\ifx#1(%
			\let\pgfmath@parsenext\pgfmath@startparsegroup%
		\else%
			\edef\pgfmath@token{#1}%
			\let\pgfmath@parsenext\pgfmath@parseoperand%
	\fi\fi%
	\pgfmath@parsenext%
}

% If no TeX units are declared *at any point* in the parse
% the result is scaled by \pgfmathresultunitscale.
\newif\ifpgfmath@unitsdeclared
\def\pgfmathsetresultunitscale#1{\def\pgfmathresultunitscale{#1}}
\def\pgfmathresultunitscale{1}

% \pgfmath@endparse
%
% Everything stops here.
%
\def\pgfmath@endparse#1\pgfmath@empty{%
        	\pgfmath@processalloperations%
		\ifpgfmath@unitsdeclared%
			\pgfmath@x1pt\relax%
		\else%
			\afterassignment\pgfmath@gobbletilpgfmath@
			\pgfmath@x\pgfmathresultunitscale pt\relax\pgfmath@%
		\fi%
		\pgfmath@x=\pgfmath@a\pgfmath@x%
		\pgfmath@returnone\pgfmath@x%
	\endgroup%
}

% \pgfmath@startparsegroup
%
% When opening ( is scanned start a new group.
%
\def\pgfmath@startparsegroup{%
	\begingroup%
		\pgfmath@resetparsingparameters
		\pgfmath@parse@}

% \pgfmath@startparsegroup
%
% When closing ) is scanned, processes all waiting
% operations (within the group) and close the group.
%
\def\pgfmath@endparsegroup{%
		\pgfmath@processalloperations%
		\pgfmath@x=\pgfmathresult pt\relax%
		\pgfmath@returnone\pgfmath@x%
	\endgroup%
	\pgfmath@afterparsegroup%
}

% \pgfmath@parseoperator
%
% An operator is expected here. 
% Or the end of the parse or parse group.
% 
\def\pgfmath@parseoperator#1{%
  \def\pgfmath@token{}%
  \let\pgfmath@pushback=\pgfmath@empty%
  \pgfmath@assignnumber%
  \ifx#1@%
    \let\pgfmath@parsenext\pgfmath@endparse%
  \else%
		\ifx#1+%
			\let\pgfmath@parsenext\pgfmath@parseadd%
		\else%
			\ifx#1-%
				\let\pgfmath@parsenext\pgfmath@parsesubtract%
			\else%
				\ifx#1*%
					\let\pgfmath@parsenext\pgfmath@parsemultiply%
				\else%
					\ifx#1/%
						\let\pgfmath@parsenext\pgfmath@parsedivide%
					\else%
						\ifx#1r%
							\let\pgfmath@parsenext\pgfmath@parseradians%
						\else%
							\ifx#1>%
								\let\pgfmath@parsenext\pgfmath@parsegreaterthan%
							\else%
								\ifx#1<%
									\let\pgfmath@parsenext\pgfmath@parselessthan%
								\else%
									\if#1=%
										\let\pgfmath@parsenext\pgfmath@parseequalto%
									\else%
										\if#1^%
											\let\pgfmath@parsenext\pgfmath@parsepower%
										\else%
											\if#1m%
												\let\pgfmath@parsenext\pgfmath@parsemod%
											\else%
												\if#1)%
													\let\pgfmath@parsenext\pgfmath@endparsegroup%
                                                                                                \else%
                                                                                                  % the following is for inputs like .5\pgflinewidth,
                                                                                                  % which should be read as .5*\pgflinewidth
                                                                                                  \ifcat#1\relax\relax%
                                                                                                    \let\pgfmath@parsenext\pgfmath@parsemultiply%
                                                                                                    \def\pgfmath@pushback{#1}%
                                                                                                  \else%
                                                                                                    \pgfmath@error{Unknown  operator `#1'}%
                                                                                                    \let\pgfmath@parsenext\relax%
	\fi\fi\fi\fi\fi\fi\fi\fi\fi\fi\fi\fi\fi%
	\expandafter\pgfmath@parsenext\pgfmath@pushback%
}

% \pgfmath@assignnumber
%
% Scanned or computed numbers are stored here.
%
\def\pgfmath@assignnumber{%
	\pgfmath@ifletto\pgfmath@a\pgfmath@empty{%
		\edef\pgfmath@a{\pgfmathresult}}{%
			\edef\pgfmath@b{\pgfmathresult}}%
}		

% \pgfmath@processalloperations
%
% If various operations are not empty, perform them.
% \pgfmath@primaryoperation      - used for any current operation.
% \pgfmath@secondaryoperation    - used to store + - if * / encountered
% \pgfmath@tertiaryoperation     - used to store = > <
%
\def\pgfmath@processalloperations{%
	\pgfmath@processprimaryoperation%
	\pgfmath@processsecondaryoperation%
	\pgfmath@processtertiaryoperation}

\def\pgfmath@processprimaryoperation{%
	\pgfmath@ifletto\pgfmath@primaryoperation\pgfmath@empty{}{%
		\pgfmath@primaryoperation{\pgfmath@a}{\pgfmath@b}%
		\edef\pgfmath@a{\pgfmathresult}%
	}%
	\let\pgfmath@primaryoperation\pgfmath@empty%
	\let\pgfmath@b\pgfmath@empty%
}

\def\pgfmath@processsecondaryoperation{%
	\pgfmath@ifletto\pgfmath@secondaryoperation\pgfmath@empty{}{%
		\pgfmath@secondaryoperation{\pgfmath@c}{\pgfmath@a}%
		\edef\pgfmath@a{\pgfmathresult}%
	}%
	\let\pgfmath@secondaryoperation\pgfmath@empty%
	\let\pgfmath@b\pgfmath@empty%
	\let\pgfmath@c\pgfmath@empty%
}
	
\def\pgfmath@processtertiaryoperation{%
	\pgfmath@ifletto\pgfmath@tertiaryoperation\pgfmath@empty{}{%
		\pgfmath@tertiaryoperation{\pgfmath@d}{\pgfmath@a}%
		\edef\pgfmath@a{\pgfmathresult}%
	}%
	\let\pgfmath@tertiaryoperation\pgfmath@empty%
	\let\pgfmath@b\pgfmath@empty%
	\let\pgfmath@c\pgfmath@empty%
	\let\pgfmath@d\pgfmath@empty%
}%

% OK. Now the operators are parsed.
% These correspond to the + - / * ^ < > = mod and r operators 
%
\def\pgfmath@parseadd{%
	% If no operator has been assigned (i.e. + is the first operator scanned),
	% do nothing, except make addition the primary operation.
	\pgfmath@ifletto\pgfmath@primaryoperation\pgfmath@empty{}{%
		% Otherwise process the primary and the secondary operations.
		%
		% Consider the epxression A-B*C+D.
		% At this point + has just been parsed, so the primary operation is *,
		% and the secondary operation is - (it was saved after * was parsed).
		\pgfmath@processprimaryoperation%
		\pgfmath@processsecondaryoperation}%
	\let\pgfmath@primaryoperation\pgfmathadd@%
	\pgfmath@parse@}

\def\pgfmath@parsesubtract{% Works like addition.
	\pgfmath@ifletto\pgfmath@primaryoperation\pgfmath@empty{}{%
		\pgfmath@processprimaryoperation%
		\pgfmath@processsecondaryoperation}%
	\let\pgfmath@primaryoperation\pgfmathsubtract@%
	\pgfmath@parse@}

\def\pgfmath@parsemultiply{%
	% If no operator has been assigned (i.e. * is the first operator scanned),
	% do nothing, except make multiplication the primary operation.
	\pgfmath@ifletto\pgfmath@primaryoperation\pgfmath@empty{}{%
		% Slighty tricker than addition/subtraction...
		\pgfmath@iflettoeither\pgfmath@primaryoperation\pgfmathmultiply@\pgfmathdivide@{%
			% Consder X\Y*Z, when * has been scanned. 
			% The primary operation is \, so this operation can be processed, 
			% and * made the new primary opertion.
			\pgfmath@processprimaryoperation}{%
				% Consder X+Y*Z, when * has been scanned. 
				% The addition must take place *after* the multiplication. 
				% So + (the current primary operation), is saved as the secondary  
				% operation, and X is saved. * becomes the new primary operation.
				\let\pgfmath@secondaryoperation\pgfmath@primaryoperation%
				\edef\pgfmath@c{\pgfmath@a}%
				\edef\pgfmath@a{\pgfmath@b}%
				\let\pgfmath@b\pgfmath@empty%		
	}}%
	\let\pgfmath@primaryoperation\pgfmathmultiply@%
	\pgfmath@parse@}

\def\pgfmath@parsedivide{% Works like multiplication.
	\pgfmath@ifletto\pgfmath@primaryoperation\pgfmath@empty{}{%
		\pgfmath@iflettoeither\pgfmath@primaryoperation\pgfmathmultiply@\pgfmathdivide@{%
			\pgfmath@processprimaryoperation}{%
				\let\pgfmath@secondaryoperation\pgfmath@primaryoperation%
				\edef\pgfmath@c{\pgfmath@a}%
				\edef\pgfmath@a{\pgfmath@b}%
				\let\pgfmath@b\pgfmath@empty%		
	}}%
	\let\pgfmath@primaryoperation\pgfmathdivide@%
	\pgfmath@parse@}

\def\pgfmath@parsegreaterthan{%
	% A Tricky one to explain precisely...
	%
	% On scanning an equality/inequality operator everything up to
	% (but not including) the operator is evaluated... 
	\pgfmath@ifletto\pgfmath@primaryoperation\pgfmath@empty{}{%
		\pgfmath@processalloperations}%
	% ...now the LHS of the inequality must be saved...
	\edef\pgfmath@temp{\pgfmath@a}%
	% ...all parsing parameters are reset...
	\pgfmath@resetparsingparameters%
	\edef\pgfmath@d{\pgfmath@temp}% ...but one parameter reinserted.
	% ....and the inequality defined as the tertiary opertation.
	\let\pgfmath@tertiaryoperation\pgfmathgreaterthan@%
	\pgfmath@parse@}
	
\def\pgfmath@parselessthan{% Similar to greater than.
	\pgfmath@ifletto\pgfmath@primaryoperation\pgfmath@empty{}{%
		\pgfmath@processalloperations}%
	\edef\pgfmath@temp{\pgfmath@a}%
	\pgfmath@resetparsingparameters%
	\edef\pgfmath@d{\pgfmath@temp}%
	\let\pgfmath@tertiaryoperation\pgfmathlessthan@%
	\pgfmath@parse@}
	
\def\pgfmath@parseequalto{% Similar to greater than.
	\pgfmath@ifletto\pgfmath@primaryoperation\pgfmath@empty{}{%
		\pgfmath@processalloperations}%
	\edef\pgfmath@temp{\pgfmath@a}%
	\pgfmath@resetparsingparameters%
	\edef\pgfmath@d{\pgfmath@temp}%
	\let\pgfmath@tertiaryoperation\pgfmathequalto@%
	\pgfmath@parse@}

\def\pgfmath@parsemod#1#2{%
	\pgfmath@ifin@{#1#2}{od}{% Just a little check.
		\pgfmath@ifletto\pgfmath@primaryoperation\pgfmath@empty{}{%
			\let\pgfmath@secondaryoperation\pgfmath@primaryoperation%
			\edef\pgfmath@c{\pgfmath@a}%
			\edef\pgfmath@a{\pgfmath@b}%
			\let\pgfmath@b\pgfmath@empty%		
		}%
		\let\pgfmath@primaryoperation\pgfmathmod@%
		\pgfmath@parse@}{%
			\edef\pgfmath@parsedfunctionname{m#1#2}%
			\pgfmath@reportunknownfunction}%
}
			
\def\pgfmath@parsepower{%
	\pgfmath@ifletto\pgfmath@primaryoperation\pgfmath@empty{}{%
		\let\pgfmath@secondaryoperation\pgfmath@primaryoperation%
		\edef\pgfmath@c{\pgfmath@a}%
		\edef\pgfmath@a{\pgfmath@b}%
		\let\pgfmath@b\pgfmath@empty%		
	}%
	\let\pgfmath@primaryoperation\pgfmathpower@%
	\pgfmath@parse@}
	
\def\pgfmath@parseradians{%
	\pgfmath@ifletto\pgfmath@primaryoperation\pgfmath@empty{%
		\pgfmathradians@{\pgfmath@a}%
		\let\pgfmath@a\pgfmath@empty}{%
			\pgfmath@iflettoeither\pgfmath@primaryoperation\pgfmathmultiply@\pgfmathdivide@{%
				\pgfmath@processprimaryoperation%
				\pgfmathradians@{\pgfmath@a}%
				\let\pgfmath@a\pgfmath@empty}{%
			\pgfmathradians@{\pgfmath@b}%	
	}}%	
	% Notice unlike the other operators, we go to \pgfmath@parseoperator,
	% This is because, r is really a post-fix function.
	\pgfmath@parseoperator}

% \pgfmath@parseoperand
%
% An operand can be an ASCII number (with or without dimensions, with
% or without a decimal point), a TeX register (count, dimen or skip,
% which may have expanded with \the), or a function e.g. sin(X).
%
% It is assumed that after \edef-ing, the only unexpanded tokens are registers. 
%
\def\pgfmath@parseoperand{%
	\expandafter\pgfmath@parseoperand@\pgfmath@token}
\def\pgfmath@parseoperand@#1{%
	\edef\pgfmath@temp{#1}%
	\expandafter\ifcat\pgfmath@temp\relax\relax%
		\begingroup%
			% If #1 is a count register the tokens pt will be
			% used on assignment. If #1 is dimension or skip
			% register then they will not.
			\afterassignment\pgfmath@parseregisterunit%
			\pgfmath@x=#1pt\relax\relax\pgfmath@%
		\endgroup%
	\fi%
	\def\pgfmath@sign{}%
	\expandafter\pgfmath@parsenumberorfunction\pgfmath@temp%
}%

\def\pgfmath@parseregisterunit#1#2\pgfmath@{%
	\ifx#1p%
		\xdef\pgfmath@temp{\pgfmath@tonumber{\pgfmath@x}pt}% 
	\else%
		\xdef\pgfmath@temp{\pgfmath@tonumber{\pgfmath@x}}%
	\fi} 
	
\def\pgfmath@parsenumberorfunction#1{%
	\pgfmath@ifin@#1{-+}{% 
		% I suppose there are (silly) people who might complain if
		% they can't say 2---5, or 3+-----7. Just for them...
		\edef\pgfmath@sign{\pgfmath@sign#1}%
			\pgfmath@parsenumberorfunction}{%
				\pgfmath@parsenumberorfunction@#1}}

\def\pgfmath@parsenumberorfunction@#1{%
  \expandafter\pgfmath@parsenumberorfunction@@%
  \ifcat#1\relax%
    % Ok, looks like a - followed by a register.
    % Handle this case by inserting a 1.
    1%
  \fi%
  #1%  
}

\def\pgfmath@parsenumberorfunction@@#1{%
  \pgfmath@ifin@{#1}{.0123456789}{\pgfmath@parsenumber#1}{%
    \pgfmath@parsefunction#1}}

\def\pgfmath@parsenumber{% 
	% Consider the number 3.14159
	% 3 is parsed by assignment to a, then '.' is absorbed and 14159
	% parsed by assignement to b (actually b=114159, see below). 
	\begingroup%
		\afterassignment\pgfmath@parsedecimalpoint%
			\c@pgfmath@counta0}

\def\pgfmath@parsedecimalpoint#1{%
		\ifx#1.% Is there a decimal point? If not, see if there are any units.
			\let\pgfmath@parsenext\pgfmath@parsemantissa%
		\else%
			\c@pgfmath@countb1\relax% The first digit of b is gobbled (see below). 
			\let\pgfmath@parsenext\pgfmath@parseunits%
		\fi%
		\pgfmath@parsenext#1}

\def\pgfmath@parsemantissa.{%
		\afterassignment\pgfmath@parseunits%
		% Consider the number: 2.005
		% 2 is assigned to a, but b will be assigned 5, which is *not right*. 
		% So using 1 here hereresults in  b=1005. The first digit is then  
		% gobbled later, when expanded with \the.
		\c@pgfmath@countb1}% 

% Unfortunately it is necessary (I can see no way out of it),
% to remove 2 tokens from the stream, and manually check for
% dimensions. If the tokens do not match a TeX dimension, they
% are reinserted into the stream, and a default unit of pt used.
%
\def\pgfmath@parseunits#1#2{%
		% Here the extra first digit in b is gobbled. 
		% Interestingly, TeX allows <dimen>=12.pt (which *I* think it shouldn't)
		\edef\pgfmath@temp{\pgfmath@sign\the\c@pgfmath@counta.\expandafter\pgfmath@gobbleone\the\c@pgfmath@countb}%
		% We scan for: bp dd cc cm em ex in mm pc pt sp 
		% Also we must scan for the 'pt' resulting from expansion using \the.
		\expandafter\pgfmath@in@\expandafter{\expandafter#1\expandafter#2\expandafter\pgfmath@\expandafter}%
			\expandafter{\pgfmath@pt\pgfmath@ bp\pgfmath@ dd\pgfmath@ cc\pgfmath@ cm\pgfmath@ em\pgfmath@ ex\pgfmath@ in\pgfmath@ mm\pgfmath@ pc\pgfmath@ pt\pgfmath@ sp\pgfmath@}%
		\ifpgfmath@in@%
			\global\pgfmath@unitsdeclaredtrue%
			\edef\pgfmath@token{}%
			\afterassignment\pgfmath@gobbletilpgfmath@%
			\pgfmath@x=\pgfmath@temp#1#2\relax\pgfmath@%
		\else%
			\afterassignment\pgfmath@gobbletilpgfmath@%
			\pgfmath@x=\pgfmath@temp pt\relax\pgfmath@%
			\edef\pgfmath@token{#1#2}%
		\fi%
		\pgfmath@smuggleone{\pgfmath@token}%
		\pgfmath@returnone\pgfmath@x%
	\endgroup%
	\expandafter\pgfmath@parseoperator\pgfmath@token%
}

% OK. Now the good stuff. We parse functions here. Note that 
% *no* calulations are done in the following macros. All the
% mathematics is done in pgfmathoperations.code and friends. 
%
% Functions dealt with here are:
%
% round(X)         'half-up' rounding.
% floor(X)         floor function.
% ciel(X)          ceiling function.
% abs(X)           absolute function.
%
% exp(X)           e^X (0 <= X <~= 9.7).
%
% sin(X)           sine function.
% cos(X)           cosine function.
% tan(X)           tan function.
% veclen(X)(Y)     the length Z where Z^2 = X^2 + Y^2
%
% rnd              generate pseudo-random number X (0 <= X <= 1).
% rand             generate pseudo-random number X (1 <= X <= -1).
% sqrt(X)          square root.
% 
% pi               the constant PI.

\def\pgfmath@parsefunction#1{%
	\def\pgfmath@parsedfunctionname{#1}%
	\ifx#1a%
		\let\pgfmath@parsefunctionnext\pgfmath@functionstartswith@a%
	\else%
		\ifx#1c%
			\let\pgfmath@parsefunctionnext\pgfmath@functionstartswith@c%
		\else%
			\ifx#1e%
				\let\pgfmath@parsefunctionnext\pgfmath@functionstartswith@e%
			\else%
				\ifx#1f%
					\let\pgfmath@parsefunctionnext\pgfmath@functionstartswith@f%
				\else%
					\ifx#1m%
						\let\pgfmath@parsefunctionnext\pgfmath@functionstartswith@m%
					\else%
						\ifx#1p%
							\let\pgfmath@parsefunctionnext\pgfmath@functionstartswith@p%
						\else%
							\ifx#1r%
								\let\pgfmath@parsefunctionnext\pgfmath@functionstartswith@r%
							\else%
								\ifx#1s%
									\let\pgfmath@parsefunctionnext\pgfmath@functionstartswith@s%
								\else%
									\ifx#1t%
										\let\pgfmath@parsefunctionnext\pgfmath@functionstartswith@t%
									\else%
										\ifx#1v%
											\let\pgfmath@parsefunctionnext\pgfmath@functionstartswith@v%
										\else
											\pgfmath@reportunknownfunction%
											\let\pgfmath@parsefunctionnext\relax%
	\fi\fi\fi\fi\fi\fi\fi\fi\fi\fi%
	\pgfmath@parsefunctionnext%
}

\def\pgfmath@reportunknownfunction{%
	\pgfmath@error{Unknown function `\pgfmath@parsedfunctionname'}{}%
}

% \pgfmath@postfunction
% 
% In scanning a function e.g. sin(40), we subvert the normal parsing 
% group mechanism by messing around with \pgfmath@afterparsegroup, so 
% that after scanning ), the parser doesn't scan for an operator, but 
% returns to the macros scanning the function. 
% Here the mechanism is restored, and the value of the function is 
% stored along with the approprate sign, which was saved earlier.
%
\def\pgfmath@postfunction{%
	\let\pgfmath@afterparsegroup\pgfmath@parseoperator%
	\edef\pgfmathresult{\pgfmath@sign\pgfmathresult}%
	\pgfmath@parseoperator}
	
\def\pgfmath@functionstartswith@a#1{%
	\edef\pgfmath@parsedfunctionname{\pgfmath@parsedfunctionname#1}%
	\ifx#1b%
		\let\pgfmath@parsefunctionnext\pgfmath@parseabs%
	\else%
		\ifx#1r%
			\let\pgfmath@parsefunctionnext\pgfmath@functionstartswith@ar%
		\else%
			\pgfmath@reportunknownfunction%
			\let\pgfmath@parsefunctionnext\relax%
	\fi\fi%
	\pgfmath@parsefunctionnext%
}

\def\pgfmath@functionstartswith@ar c#1{%
	\edef\pgfmath@parsedfunctionname{\pgfmath@parsedfunctionname#1}%
	\ifx#1c%
		\let\pgfmath@parsefunctionnext\pgfmath@parsearccos%
	\else%
		\ifx#1s%
			\let\pgfmath@parsefunctionnext\pgfmath@parsearcsin%
		\else%
			\ifx#1t%
				\let\pgfmath@parsefunctionnext\pgfmath@parsearctan% <- not implemented (yet).
			\else%
				\pgfmath@reportunknownfunction%
				\let\pgfmath@parsefunctionnext\relax%
	\fi\fi\fi%
	\pgfmath@parsefunctionnext%
}

\def\pgfmath@functionstartswith@c#1{%
	\edef\pgfmath@parsedfunctionname{\pgfmath@parsedfunctionname#1}%
	\ifx#1e%
		\let\pgfmath@parsefunctionnext\pgfmath@parseceil%
	\else%
		\ifx#1o%
			\let\pgfmath@parsefunctionnext\pgfmath@parsecos%
		\else%
			\pgfmath@reportunknownfunction%
			\let\pgfmath@parsefunctionnext\relax%
	\fi\fi%
	\pgfmath@parsefunctionnext%
}

\def\pgfmath@functionstartswith@e#1{%
	\edef\pgfmath@parsedfunctionname{\pgfmath@parsedfunctionname#1}%
	\ifx#1x%
		\let\pgfmath@parsefunctionnext\pgfmath@parseexp%
	\else%
		\pgfmath@reportunknownfunction%
		\let\pgfmath@parsefunctionnext\relax%
	\fi%
	\pgfmath@parsefunctionnext%
}

\def\pgfmath@functionstartswith@f#1{%
	\edef\pgfmath@parsedfunctionname{\pgfmath@parsedfunctionname#1}%
	\ifx#1l%
		\let\pgfmath@parsefunctionnext\pgfmath@parsefloor%
	\else%
		\pgfmath@reportunknownfunction%
		\let\pgfmath@parsefunctionnext\relax%
	\fi%
	\pgfmath@parsefunctionnext%
}

\def\pgfmath@functionstartswith@m#1{%
	\edef\pgfmath@parsedfunctionname{\pgfmath@parsedfunctionname#1}%
	\ifx#1o%
		\let\pgfmath@parsefunctionnext\pgfmath@parsemod%
	\else%
		\pgfmath@reportunknownfunction%
		\let\pgfmath@parsefunctionnext\relax%
	\fi%
	\pgfmath@parsefunctionnext%
}

\def\pgfmath@functionstartswith@p#1{%
	\edef\pgfmath@parsedfunctionname{\pgfmath@parsedfunctionname#1}%
	\ifx#1i%
		\let\pgfmath@parsefunctionnext\pgfmath@parsepi%
	\else%
		\pgfmath@reportunknownfunction%
		\let\pgfmath@parsefunctionnext\relax%
	\fi%
	\pgfmath@parsefunctionnext%
}

\def\pgfmath@functionstartswith@r#1{%
	\edef\pgfmath@parsedfunctionname{\pgfmath@parsedfunctionname#1}%
	\ifx#1a%
		\let\pgfmath@parsefunctionnext\pgfmath@parserand%
	\else%
		\ifx#1n%
			\let\pgfmath@parsefunctionnext\pgfmath@parsernd%
		\else%
			\ifx#1o%
				\let\pgfmath@parsefunctionnext\pgfmath@parseround%
			\else%
				\pgfmath@reportunknownfunction%
				\let\pgfmath@parsefunctionnext\relax%
	\fi\fi\fi%
	\pgfmath@parsefunctionnext%
}

\def\pgfmath@functionstartswith@s#1{%
	\edef\pgfmath@parsedfunctionname{\pgfmath@parsedfunctionname#1}%
	\ifx#1i%
		\let\pgfmath@parsefunctionnext\pgfmath@parsesin%
	\else%
		\ifx#1q%
			\let\pgfmath@parsefunctionnext\pgfmath@parsesqrt%
		\else%
			\pgfmath@reportunknownfunction%
			\let\pgfmath@parsefunctionnext\relax%
	\fi\fi%
	\pgfmath@parsefunctionnext%
}

\def\pgfmath@functionstartswith@t#1{%
	\edef\pgfmath@parsedfunctionname{\pgfmath@parsedfunctionname#1}%
	\ifx#1a%
		\let\pgfmath@parsefunctionnext\pgfmath@parsetan%
	\else%
			\pgfmath@reportunknownfunction%
			\let\pgfmath@parsefunctionnext\relax%
	\fi%
	\pgfmath@parsefunctionnext%
}

\def\pgfmath@functionstartswith@v#1{%
	\edef\pgfmath@parsedfunctionname{\pgfmath@parsedfunctionname#1}%
	\ifx#1e%
		\let\pgfmath@parsefunctionnext\pgfmath@parseveclen%
	\else%
			\pgfmath@reportunknownfunction%
			\let\pgfmath@parsefunctionnext\relax%
	\fi%
	\pgfmath@parsefunctionnext%
}

% \pgfmath@parseabs
%
\def\pgfmath@parseabs#1{%
	\pgfmath@ifin@{#1}{s}{%
		% The normal parsing mechansims is subverted here:
		\let\pgfmath@afterparsegroup\pgfmath@parseabs@%
		\pgfmath@parse@}{%
			\edef\pgfmath@parsedfunctionname{\pgfmath@parsedfunctionname#1}%
			\pgfmath@reportunknownfunction}}
\def\pgfmath@parseabs@{%	
	\expandafter\pgfmathabs@\expandafter{\pgfmathresult}%
	\pgfmath@postfunction%
}

% \pgfmath@parsesqrt
%
\def\pgfmath@parsesqrt #1#2{%
	\pgfmath@ifin@{#1#2}{rt}{%
		\let\pgfmath@afterparsegroup\pgfmath@parsesqrt@%
		\pgfmath@parse@}{%
			\edef\pgfmath@parsedfunctionname{\pgfmath@parsedfunctionname#1#2}%
			\pgfmath@reportunknownfunction}}%
\def\pgfmath@parsesqrt@{%	
	\expandafter\pgfmathsqrt@\expandafter{\pgfmathresult}%
	\pgfmath@postfunction%
}

% \pgfmath@parseround
%
\def\pgfmath@parseround #1#2#3{%
	\pgfmath@ifin@{#1#2#3}{und}{%
		\let\pgfmath@afterparsegroup\pgfmath@parseround@%
		\pgfmath@parse@}{%
			\edef\pgfmath@parsedfunctionname{\pgfmath@parsedfunctionname#1#2#3}%
			\pgfmath@reportunknownfunction}}%
\def\pgfmath@parseround@{%	
	\expandafter\pgfmathround@\expandafter{\pgfmathresult}%
	\pgfmath@postfunction%
}

% \pgfmath@parsefloor
%
\def\pgfmath@parsefloor #1#2#3{%
	\pgfmath@ifin@{#1#2#3}{oor}{%
		\let\pgfmath@afterparsegroup\pgfmath@parsefloor@%
		\pgfmath@parse@}{%
			\edef\pgfmath@parsedfunctionname{\pgfmath@parsedfunctionname#1#2#3}%
			\pgfmath@reportunknownfunction}}%
\def\pgfmath@parsefloor@{%	
	\expandafter\pgfmathfloor@\expandafter{\pgfmathresult}%
	\pgfmath@postfunction%
}

% \pgfmath@parseceil
%
\def\pgfmath@parseceil #1#2{%
	\pgfmath@ifin@{#1#2}{il}{%
		\let\pgfmath@afterparsegroup\pgfmath@parseceil@%
		\pgfmath@parse@}{%
			\edef\pgfmath@parsedfunctionname{\pgfmath@parsedfunctionname#1#2}%
			\pgfmath@reportunknownfunction}}%
\def\pgfmath@parseceil@{%	
	\expandafter\pgfmathceil@\expandafter{\pgfmathresult}%
	\pgfmath@postfunction%
}

% \pgfmath@parsesin
%
\def\pgfmath@parsesin#1{%
	\pgfmath@ifin@{#1}{n}{%
		\let\pgfmath@afterparsegroup\pgfmath@parsesin@%
		\pgfmath@parse@}{%
			\edef\pgfmath@parsedfunctionname{\pgfmath@parsedfunctionname#1}%
			\pgfmath@reportunknownfunction}}%
\def\pgfmath@parsesin@{%	
	\expandafter\pgfmathsin@\expandafter{\pgfmathresult}%
	\pgfmath@postfunction%
}

% \pgfmath@parsecos
%
\def\pgfmath@parsecos#1{%
	\pgfmath@ifin@{#1}{s}{%
		\let\pgfmath@afterparsegroup\pgfmath@parsecos@%
		\pgfmath@parse@}{%
			\edef\pgfmath@parsedfunctionname{\pgfmath@parsedfunctionname#1}%
			\pgfmath@reportunknownfunction}}%
\def\pgfmath@parsecos@{%	
	\expandafter\pgfmathcos@\expandafter{\pgfmathresult}%
	\pgfmath@postfunction%
}

% \pgfmath@parsearcsin
%
\def\pgfmath@parsearcsin#1#2{%
	\pgfmath@ifin@{#1#2}{in}{%
		\let\pgfmath@afterparsegroup\pgfmath@parsearcsin@%
		\pgfmath@parse@}{%
			\edef\pgfmath@parsedfunctionname{\pgfmath@parsedfunctionname#1#2}%
			\pgfmath@reportunknownfunction}}%
\def\pgfmath@parsearcsin@{%	
	\expandafter\pgfmatharcsin@\expandafter{\pgfmathresult}%
	\pgfmath@postfunction%
}

% \pgfmath@parsearccos
%
\def\pgfmath@parsearccos#1#2{%
	\pgfmath@ifin@{#1#2}{os}{%
		\let\pgfmath@afterparsegroup\pgfmath@parsearccos@%
		\pgfmath@parse@}{%
			\edef\pgfmath@parsedfunctionname{\pgfmath@parsedfunctionname#1#2}%
			\pgfmath@reportunknownfunction}}%
\def\pgfmath@parsearccos@{%	
	\expandafter\pgfmatharccos@\expandafter{\pgfmathresult}%
	\pgfmath@postfunction%
}

% \pgfmath@parsetan
%
\def\pgfmath@parsetan#1{%
	\pgfmath@ifin@{#1}{n}{%
		\let\pgfmath@afterparsegroup\pgfmath@parsetan@%
		\pgfmath@parse@}{%
			\edef\pgfmath@parsedfunctionname{\pgfmath@parsedfunctionname#1}%
			\pgfmath@reportunknownfunction}}%
\def\pgfmath@parsetan@{%	
	\expandafter\pgfmathtan@\expandafter{\pgfmathresult}%
	\pgfmath@postfunction%
}

% \pgfmath@parsernd
%
\def\pgfmath@parsernd#1{%
	\pgfmath@ifin@{#1}{d}{\pgfmathrnd}{%
		\edef\pgfmath@parsedfunctionname{\pgfmath@parsedfunctionname#1}%
			\pgfmath@reportunknownfunction}%
	\pgfmath@postfunction}%
			
% \pgfmath@parserand
%
\def\pgfmath@parserand#1#2{%
	\pgfmath@ifin@{#1}{nd}{\pgfmathrand}{%
		\edef\pgfmath@parsedfunctionname{\pgfmath@parsedfunctionname#1}%
			\pgfmath@reportunknownfunction}%
	\pgfmath@postfunction}%
			
% \pgfmath@parseexp
%
\def\pgfmath@parseexp#1{%
	\pgfmath@ifin@{#1}{p}{%
		\let\pgfmath@afterparsegroup\pgfmath@parseexp@%
		\pgfmath@parse@}{%
			\edef\pgfmath@parsedfunctionname{\pgfmath@parsedfunctionname#1}%
			\pgfmath@reportunknownfunction}}%
\def\pgfmath@parseexp@{%	
	\expandafter\pgfmathexp@\expandafter{\pgfmathresult}%
	\pgfmath@postfunction%
}

% \pgfmath@parsepi
%
\def\pgfmath@parsepi{%
	\pgfmathpi%
	\pgfmath@postfunction%
}

% \pgfmath@parseveclen
%
\def\pgfmath@parseveclen#1#2#3#4{%
	\pgfmath@ifin@{#1#2#3#4}{clen}{%
		\let\pgfmath@afterparsegroup\pgfmath@parseveclen@%
		\pgfmath@parse@}{%
			\edef\pgfmath@parsedfunctionname{\pgfmath@parsedfunctionname#1}%
			\pgfmath@reportunknownfunction}}%
\def\pgfmath@parseveclen@{%
	\edef\pgfmath@vecx{\pgfmathresult}%
	\let\pgfmath@afterparsegroup\pgfmath@parseveclen@@%
	\pgfmath@parse@}%
\def\pgfmath@parseveclen@@{%
	\edef\pgfmath@vecy{\pgfmathresult}%
	\pgfmathveclen@{\pgfmath@vecx}{\pgfmath@vecy}%
	\pgfmath@postfunction}