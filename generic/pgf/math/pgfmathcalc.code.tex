% Copyright 2007 by Mark Wibrow
%
% This file may be distributed and/or modified
%
% 1. under the LaTeX Project Public License and/or
% 2. under the GNU Public License.
%
% See the file doc/generic/pgf/licenses/LICENSE for more details.

% This file loads all the parsing, functions and operator stuff
%
% Version 0.0 08/03/2007

% This file defines utilities common to the \pgfmath files.
%
% (c) 2007 Mark Wibrow
%
% but subject to the LaTeX Project Public License 
% (http://www.latex-project.org/lppl.txt)
%
% and the GNU Public License 
% (http://www.gnu.org/licenses/gpl.txt)
%
% Version 0.0 08/03/2007

% Alias some pgf stuff, just in case it needs replacing later.
%
\let\pgfmath@x\pgf@x
\let\pgfmath@xa\pgf@xa
\let\pgfmath@xb\pgf@xb
\let\pgfmath@xc\pgf@xc

\let\pgfmath@y\pgf@y
\let\pgfmath@ya\pgf@ya
\let\pgfmath@yb\pgf@yb
\let\pgfmath@yc\pgf@yc

\let\c@pgfmath@counta\c@pgf@counta
\let\c@pgfmath@countb\c@pgf@countb
\let\c@pgfmath@countc\c@pgf@countc

\let\pgfmath@ifnextchar\pgfutil@ifnextchar

\let\pgfmath@tonumber\pgf@sys@tonumber

\let\pgfmath@ifundefined\pgfutil@ifundefined

\newif\ifpgfmath@in@
\def\pgfmath@in@#1#2{%
 \def\pgfmath@in@@##1#1##2##3\pgfmath@in@@{%
  \ifx\pgfmath@in@##2\pgfmath@in@false\else\pgfmath@in@true\fi}%
 \pgfmath@in@@#2#1\pgfmath@in@\pgfmath@in@@}
 
% Need to redefine to something more appropriate...?
%
\def\pgfmath@error#1#2{\PackageError{PGF Math}{#1}{#2}}
\def\pgfmath@warning#1#2{\PackageWarning{PGF Math}{#1}{#2}}

% \pgfmath@pt
%
% Needed to test for 'pt' resulting from 
% expansion using \the.
%
{\catcode`\p=12\catcode`\t=12\gdef\PgfmaTh@PT{pt}}		
\edef\pgfmath@pt{\PgfmaTh@PT}%

% \ifpgfmath@int
%
% Is #1 an integer?
% Actually we only test to see if 
% the decimal part of #1 is zero.
%
\def\pgfmath@ifinteger#1{%
	\expandafter\pgfmath@ifinteger@#1\pgfmath@%
}%
\def\pgfmath@ifinteger@#1.#2\pgfmath@{%
	\begingroup%
		\c@pgfmath@counta=#2\relax%
		\ifnum\c@pgfmath@counta=0\relax%
			\global\let\pgfmath@ifintegernext\pgfmath@firstoftwo%
		\else%
			\global\let\pgfmath@ifintegernext\pgfmath@secondoftwo%
		\fi%
	\endgroup%
	\pgfmath@ifintegernext}
	
% \pgfmath@ifin@
%
% Interface for \pgfutil@in@
%
\def\pgfmath@ifin@#1#2{%
	\pgfutil@in@{#1}{#2}%
	\ifpgfutil@in@%
		\expandafter\pgfmath@firstoftwo%
	\else%
		\expandafter\pgfmath@secondoftwo%
	\fi%
}

% \pgfmath@empty
%
% A handy macro.
%
\def\pgfmath@empty{}


% \ifpgfmath@empty
%
% Has #1 been let to \pgf@empty?
%
\def\ifpgfmath@empty#1{%
	\ifx#1\pgfmath@empty\relax
		\expandafter\pgfmath@firstoftwo%
	\else%
		\expandafter\pgfmath@secondoftwo%
	\fi%
}
\def\pgfmath@firstoftwo#1#2{#1}
\def\pgfmath@secondoftwo#1#2{#2}

% \ifpgfmath@ifletto
%
% Has #1 been let to #2?
%
\def\pgfmath@ifletto#1#2{%
	\ifx#1#2\relax%
		\expandafter\pgfmath@firstoftwo%
	\else%
		\expandafter\pgfmath@secondoftwo%
	\fi%
}

% \ifpgfmath@iflettoeither
%
% Has #1 been let to #2 or #3?
%
\def\pgfmath@iflettoeither#1#2#3{%
	\ifx#1#2\relax%
		\let\pgfmath@equaltoeithernext\pgfmath@firstoftwo%
	\else%
		\ifx#1#3\relax%
			\let\pgfmath@equaltoeithernext\pgfmath@firstoftwo%
		\else%
			\let\pgfmath@equaltoeithernext\pgfmath@secondoftwo%
		\fi%
	\fi%
	\pgfmath@equaltoeithernext%
}

% \pgfmath@gobbletilpgfmath@
%
% Gobble stream until \pgfmath@ {which is undefined}.
% 
\def\pgfmath@gobbletilpgfmath@#1\pgfmath@{}
\def\pgfmath@gobbleone#1{}%


% \pgfmathloop
%
% A version of the standard TeX and LaTeX
% \loop, with an additional macro \pgfmathcounter
% (which is *not* a TeX counter) which keeps
% track of the iterations.
%
\newif\ifpgfmathcontinueloop
\def\pgfmathloop#1\repeatpgfmathloop{%
	\def\pgfmathcounter{1}%
	\def\pgfmath@iterate{%
		#1\relax%
		{% Do this inside a group, just in case...
			\c@pgfmath@counta=\pgfmathcounter%
			\advance\c@pgfmath@counta by1\relax%
			\xdef\pgfmathloop@temp{\the\c@pgfmath@counta}%
		}%	
		\edef\pgfmathcounter{\pgfmathloop@temp}%
		\expandafter\pgfmath@iterate\fi}%
	\pgfmath@iterate\let\pgfmath@iterate=\relax}
\let\repeatpgfmathloop=\fi


% \pgfmath@returnone
%
% \edef \pfgmathresult to the value of #1 after the end
% of a group.
%
% #1 - a dimension/count/skip register or a macro
%      representing a number or dimension.
%
\def\pgfmath@returnone#1{%
	\afterassignment\pgfmath@gobbletilpgfmath@%
	\pgfmath@x#1pt\relax\pgfmath@%
	\xdef\pgfmath@resulttemp{\pgf@sys@tonumber{\pgfmath@x}}%
	\gdef\pgfmath@returnone@{%
		\edef\pgfmathresult{\pgfmath@resulttemp}%
	}%
	\aftergroup\pgfmath@returnone@}


% \pgfmath@returntwo
%
% \edef \pfgmathresultx to the value of #1, and
% \pgfmathresulty to the value of #2, after the end
% of a group.
%
% #1 - a dimension/count/skip register or a macro
%      representing a number or dimension.
%
% #2 - a dimension/count/skip register or a macro
%      representing a number or dimension.
%	
\def\pgfmath@returntwo#1#2{%
	\afterassignment\pgfmath@gobbletilpgfmath@%
	\pgfmath@x#1pt\relax\pgfmath@%
	\xdef\pgfmath@resulttempx{\pgf@sys@tonumber{\pgfmath@x}}%
	\afterassignment\pgfmath@gobbletilpgfmath@%
	\pgfmath@x#2pt\relax\pgfmath@%
	\xdef\pgfmath@resulttempx{\pgf@sys@tonumber{\pgfmath@x}}%
	\gdef\pgfmath@returntwo@{%
		\edef\pgfmathresultx{\pgfmath@resulttempx}%
		\edef\pgfmathresulty{\pgfmath@resulttempy}%
	}%
	\aftergroup\pgfmath@returntwo@}

% \pgfmath@smuggleone
%
% Summgle a macro outside a group.
%
\def\pgfmath@smuggleone#1{%
	\xdef\pgfmath@smuggleditem{#1}%
	\gdef\pgfmath@@smuggleone{\edef#1{\pgfmath@smuggleditem}}%
	\aftergroup\pgfmath@@smuggleone}
% Copyright 2007 Mark Wibrow
%
% This file may be distributed and/or modified
%
% 1. under the LaTeX Project Public License and/or
% 2. under the GNU Public License.
%
% See the file doc/generic/pgf/licenses/LICENSE for more details.

% This file parses/evaluates a decimal expression.
%
% Version 1.414213 29/9/2007

% \pgfmathparse, \pgfmathqparse
%
% Evaluates a decimal expression.
%
% #1 - the expression.
%
% returns
%
% x = the result as a dimension.
%
% E.g.
% \pgfmathparse{3pt*2cm+1.5}
% \pgfmathqparse{3pt*2cm+1.5pt}
%
% Every number in \pgfmathqparse *must*
% specify a unit.
%
\newif\ifpgfmath@quickparse

%\def\pgfmathparse{%
%	\pgfmath@quickparsefalse%
%	\pgfmathparse@}

\newif\ifpgfmathfloat

\def\pgfmathparse{%
	\pgfmath@quickparsefalse%
	\ifpgfmathfloat%
		\let\pgfmath@parse@next\pgfmathfloatparsenumber%
	\else%
		\let\pgfmath@parse@next\pgfmathparse@%
	\fi%
	\pgfmath@parse@next}

\def\pgfmathqparse{%
	\pgfmath@quickparsetrue%
	\pgfmathparse@}

% Stuff for compatability with the calc package.
%
\def\pgfmath@calc@real#1{#1}
\def\pgfmath@calc@minof#1#2{min(#1,#2)}
\def\pgfmath@calc@maxof#1#2{max(#1,#2)}
\def\pgfmath@calc@ratio#1#2{#1/#2}
\def\pgfmath@calc@widthof#1{widthof(#1)}
\def\pgfmath@calc@heightof#1{heightof(#1)}
\def\pgfmath@calc@depthof#1{depthof(#1)}

\def\pgfmathparse@#1{%
  \begingroup%    
    % Reinstall correct font, so that dimensions like 1em are correct
    \pgfmath@selectfont%
    \let\real=\pgfmath@calc@real%
    \let\minof=\pgfmath@calc@minof%
    \let\maxof=\pgfmath@calc@maxof%
    \let\ratio=\pgfmath@calc@ratio%
    \let\widthof=\pgfmath@calc@widthof%
    \let\heightof=\pgfmath@calc@heightof%
    \let\depthof=\pgfmath@calc@depthof%
    \edef\pgfmath@temp{#1}%
    \pgfmath@resetparsingparameters%
    \global\pgfmathunitsdeclaredfalse%
    \ifpgfmath@quickparse%
      \let\pgfmath@parse@operand=\pgfmath@qparse@operand%
    \else%
      \let\pgfmath@parse@operand=\pgfmath@parse@operand%
    \fi%
    \let\pgfmath@parse@postgroup=\pgfmath@parse@operator%
    \expandafter\pgfmath@parse@\pgfmath@temp @@@@@@@@@@@\pgfmath@empty}
    

% \pgfmath@resetparsingparameters
%
% Reset the stack at the begining of the parse/group.
%
\def\pgfmath@resetparsingparameters{%
	\pgfmath@stack{\pgfmath@empty\pgfmath@empty\pgfmath@empty\pgfmath@empty}%
	\def\pgfmath@stacknextoperator{\pgfmath@empty}% Will not work with \let
}

% Check to see if the parse starts with a TeX-group.
%
\def\pgfmath@parse@{\futurelet\pgfmath@token\pgfmath@parse@@}

\def\pgfmath@parse@@{%
	\ifx\pgfmath@token\bgroup% 
		\let\pgfmath@next=\pgfmath@parse@@@%
	\else%
		\let\pgfmath@next=\pgfmath@parse@@@@%
	\fi%
	\def\pgfmath@token{}%
	\pgfmath@next}%
	
% Remove any surrounding braces.
%
\def\pgfmath@parse@@@#1{\expandafter\pgfmath@parse@@@@#1}%

% Start parsing. Expect one of
%
% 1) the end of the parse.
% 2) the start of a group. 
% 3) a (possible) operand.
%
\def\pgfmath@parse@@@@#1{%
	\def\pgfmath@token{}%
	\ifx#1@%
		\let\pgfmath@parse@next=\pgfmath@parse@end%
	\else%
		\ifx#1(%
			\let\pgfmath@parse@next=\pgfmath@parse@startgroup%
		\else%
			\edef\pgfmath@token{#1}%
			\let\pgfmath@parse@next=\pgfmath@parse@operand%
	\fi\fi%
	\pgfmath@parse@next%
}

% If no TeX units are declared *at any point* in the parse
% the result is scaled by \pgfmathresultunitscale.
\newif\ifpgfmathunitsdeclared
\newif\ifpgfmathignoreunitscale
\def\pgfmathsetresultunitscale#1{\def\pgfmathresultunitscale{#1}}
\def\pgfmathresultunitscale{1}

% \pgfmath@parse@end
%
% Everything stops here.
%
\def\pgfmath@parse@end#1\pgfmath@empty{%
    \pgfmath@processalloperations%
    \pgfmath@stackpop{\pgfmathresult}% 
    \pgfmathpostparse%
    \pgfmath@smuggleone{\pgfmathresult}%
  \endgroup%
  \ignorespaces%
}


\def\pgfmathscaleresult{%
	\ifpgfmathunitsdeclared%
  \else%
  	\ifpgfmathignoreunitscale%
  	\else%
   	  \afterassignment\pgfmath@gobbletilpgfmath@%
      \pgfmath@x\pgfmathresultunitscale pt\relax\pgfmath@%
      \expandafter\pgfmath@x\pgfmathresult\pgfmath@x%
      \edef\pgfmathresult{\pgfmath@tonumber{\pgfmath@x}}%
    \fi%
  \fi%
}

\let\pgfmathpostparse=\pgfmathscaleresult
 
% \pgfmath@parse@startgroup
%
% When opening ( is scanned start a new group.
%
\def\pgfmath@parse@startgroup{%
	\begingroup%
		\let\pgfmath@parse@postgroup=\pgfmath@parse@operator%
		\pgfmath@resetparsingparameters%
		\pgfmath@parse@}

% \pgfmath@parse@endgroup
%
% When closing ) is scanned, processes all waiting
% operations (within the group) and close the group.
%
\def\pgfmath@parse@endgroup{%
		\pgfmath@processalloperations%
		\pgfmath@stackpop{\pgfmathresult}%
		\pgfmath@smuggleone{\pgfmathresult}%
	\endgroup%
	\pgfmath@parse@postgroup%
}

% \pgfmath@parse@operator
%
% An operator is expected here. 
% Or the end of the parse or parse group.
% 
\def\pgfmath@parse@operator#1{%
	\def\pgfmath@token{}%
	% Push the operand in \pgfmathresult on to the stack. 
	\expandafter\pgfmath@stackpushoperand\expandafter{\pgfmathresult}%
	\ifx#1@%
		\let\pgfmath@next=\pgfmath@parse@end%
	\else%
		\ifx#1+%
			\let\pgfmath@next=\pgfmath@parse@add%
		\else%
			\ifx#1-%
				\let\pgfmath@next=\pgfmath@parse@subtract%
			\else%
				\ifx#1*%
					\let\pgfmath@next=\pgfmath@parse@multiply%
				\else%
					\ifx#1/%
						\let\pgfmath@next=\pgfmath@parse@divide%
					\else
						\ifx#1)%
							\let\pgfmath@next=\pgfmath@parse@endgroup%
						\else%
							\ifx#1r%
								\let\pgfmath@next=\pgfmath@parse@radians%
							\else%
								\ifx#1>%
									\let\pgfmath@next=\pgfmath@parse@greaterthan%
								\else%
									\ifx#1<%
										\let\pgfmath@next=\pgfmath@parse@lessthan%
									\else%
										\if#1=%
											\let\pgfmath@next=\pgfmath@parse@equalto%
										\else%
											\if#1^%
												\let\pgfmath@next=\pgfmath@parse@power%
											\else%
												\pgfmath@error{Unknown operator `#1'}%
												\let\pgfmath@next\relax%
	\fi\fi\fi\fi\fi\fi\fi\fi\fi\fi\fi%
	\pgfmath@next%
}
	
% Use a \toks register as a stack.
\newtoks\pgfmath@stack

% \pgfmath@stackpushoperator
% 
% Push an operator (actually its macro e.g., \pgfmathadd@)
% on to the stack. And keep track of it using the macro
% \pgfmath@stacknextoperator.
%
\def\pgfmath@stackpushoperator#1{%
	\edef\pgfmath@temp{\noexpand#1\the\pgfmath@stack}%
	\expandafter\pgfmath@stack\expandafter{\pgfmath@temp}%
	\def\pgfmath@stacknextoperator{#1}}% <- Must \def. Cannot \let.

% \pgfmath@stackpushoperand
%
% Push an operand (i.e. a number) on the stack. It is
% put within a TeX group to make popping a lot simpler.
%
\def\pgfmath@stackpushoperand#1{%
	\def\pgfmath@temp{{#1}}%
	\expandafter\expandafter\expandafter\def%
		\expandafter\expandafter\expandafter\pgfmath@@temp%
			\expandafter\expandafter\expandafter{\expandafter\pgfmath@temp\the\pgfmath@stack}%
	\expandafter\pgfmath@stack\expandafter{\pgfmath@@temp}%
}

% \pgfmath@stackpeek
%
% Peek (i.e. without removal) at the top of the stack.
%
\def\pgfmath@stackpeek{\expandafter\pgfmath@stackpeek@\the\pgfmath@stack\pgfmath@}
\def\pgfmath@stackpeek@#1#2\pgfmath@{#1}%

% \pgfmath@stackpop
%
% Pop (i.e. remove) the top of the stack into #1.
%
\def\pgfmath@stackpop#1{\expandafter\pgfmath@stackpop@\expandafter#1\the\pgfmath@stack\pgfmath@}
\def\pgfmath@stackpop@#1#2#3\pgfmath@{\def#1{#2}\pgfmath@stack{#3}}%

% \pgfmath@stackpopoperation
%
% Remove and perform an operation from the stack.
%
\def\pgfmath@stackpopoperation{%
	\expandafter\pgfmath@stackpopoperation@\the\pgfmath@stack\pgfmath@%
}
\def\pgfmath@stackpopoperation@#1#2#3#4#5\pgfmath@{%
	\ifx\pgfmath@empty#1\relax%
			\pgfmath@stack{\pgfmath@empty\pgfmath@empty\pgfmath@empty\pgfmath@empty}%
	\else%
		\ifx\pgfmath@empty#2\relax%
			\pgfmath@stack{{#1}\pgfmath@empty\pgfmath@empty\pgfmath@empty\pgfmath@empty}%
		\else%
			#2{#3}{#1}%
			\pgfmath@stack{#4#5}%
			\expandafter\pgfmath@stackpushoperand\expandafter{\pgfmathresult}%
	\fi\fi%
	\def\pgfmath@stacknextoperator{#4}}

% \pgfmath@processalloperations
%
% Process all operation in the stack. The 
% overall result is at the top of the stack.
%
\def\pgfmath@processalloperations{%
	\expandafter\pgfmath@in@\pgfmath@stacknextoperator{\pgfmath@empty}%
	\ifpgfmath@in@%
		\let\pgfmath@next=\relax%		
	\else%
		\pgfmath@stackpopoperation%
		\let\pgfmath@next=\pgfmath@processalloperations%
	\fi%
	\pgfmath@next}

% \pgfmath@processoperationsuntil
%
% Process operations in the stack, until the specified
% operation/s is/are encountered. The overall result is 
% at the top of the stack.
%
\def\pgfmath@processoperationsuntil#1{%
	\expandafter\pgfmath@in@\pgfmath@stacknextoperator{#1\pgfmath@empty}%
	\ifpgfmath@in@%
		\let\pgfmath@next\pgfmath@processoperationsuntil@end%		
	\else%
		\pgfmath@stackpopoperation%
		\let\pgfmath@next\pgfmath@processoperationsuntil%
	\fi%
	\pgfmath@next{#1}}
\def\pgfmath@processoperationsuntil@end#1{}


% OK. Now the operators are parsed.
% These correspond to the + - / * ^ < > = operators and r (postfix) function.
%
\def\pgfmath@parse@add{%	
	% If no operator has been assigned (i.e. + is the first operator scanned),
	% do nothing, except add addition to the stack.
	\ifx\pgfmath@stacknextoperator\pgfmath@empty%
	\else%
		% Empty the process stack up to any inequalities.
		\pgfmath@processoperationsuntil{\pgfmathequalto@\pgfmathlessthan@\pgfmathgreaterthan@}%
	\fi%
	\pgfmath@stackpushoperator{\pgfmathadd@}%
	\pgfmath@parse@}
	
\def\pgfmath@parse@subtract{%	
	% If no operator has been assigned (i.e. - is the first operator scanned),
	% do nothing, except add subtract to the stack.
	\ifx\pgfmath@stacknextoperator\pgfmath@empty%
	\else%
		% Empty the process stack up to any inequalities.
		\pgfmath@processoperationsuntil{\pgfmathequalto@\pgfmathlessthan@\pgfmathgreaterthan@}%
	\fi%
	\pgfmath@stackpushoperator{\pgfmathsubtract@}%
	\pgfmath@parse@}
	
\def\pgfmath@parse@multiply{%
	% If no operator has been assigned (i.e. * is the first operator scanned),
	% do nothing, except push multiply onto the stack.
	\ifx\pgfmath@stacknextoperator\pgfmath@empty%
	\else%
		% Process all operations up to inequalites or addition/subtraction
		\pgfmath@processoperationsuntil{\pgfmathequalto@\pgfmathlessthan@\pgfmathgreaterthan@%
			\pgfmathadd@\pgfmathsubtract@}%
	\fi%
	\pgfmath@stackpushoperator{\pgfmathmultiply@}%
	\pgfmath@parse@}
	
\def\pgfmath@parse@divide{%
	% If no operator has been assigned (i.e. / is the first operator scanned),
	% do nothing, except push divide onto the stack.
	\ifx\pgfmath@stacknextoperator\pgfmath@empty%
	\else%
		% Process all operations up to inequalites or addition/subtraction
		\pgfmath@processoperationsuntil{\pgfmathequalto@\pgfmathlessthan@\pgfmathgreaterthan@%
			\pgfmathadd@\pgfmathsubtract@}%
	\fi%
	\pgfmath@stackpushoperator{\pgfmathdivide@}%
	\pgfmath@parse@}

\def\pgfmath@parse@greaterthan{%
	% On scanning an equality/inequality operator everything up to
	% (but not including) the operator is evaluated... 
	\pgfmath@processalloperations%
	% ...and the operation pushed onto the stack.
	\pgfmath@stackpushoperator{\pgfmathgreaterthan@}%
	\pgfmath@parse@}

\def\pgfmath@parse@lessthan{%
	\pgfmath@processalloperations%
	\pgfmath@stackpushoperator{\pgfmathlessthan@}%
	\pgfmath@parse@}

\def\pgfmath@parse@equalto={%
	\pgfmath@processalloperations%
	\pgfmath@stackpushoperator{\pgfmathequalto@}%
	\pgfmath@parse@}

\def\pgfmath@parse@power{%
	% Easy, just push power onto the stack.
	\pgfmath@stackpushoperator{\pgfmathpow@}%
	\pgfmath@parse@}

\def\pgfmath@parse@radians{%
	% r is a post-fix function...
	\ifx\pgfmath@primaryoperation\pgfmath@empty%
	\else%
		\pgfmath@processoperationsuntil{\pgfmathequalto@\pgfmathlessthan@\pgfmathgreaterthan@%
			\pgfmathadd@\pgfmathsubtract@}%
	\fi%	
	\pgfmath@stackpop{\pgfmath@temp}%
	\pgfmathdeg@{\pgfmath@temp}%
	% ...so processing returns to \pgfmath@parse@operator
	\pgfmath@parse@operator}
	
\newdimen\pgfmath@dimen
\newcount\c@pgfmath@count

% \pgfmath@qparse@operand
%
% An operand can *only* be a dimension.
%
\def\pgfmath@qparse@operand{%
	\afterassignment\pgfmath@qparse@operand@%
	\pgfmath@dimen\pgfmath@token}
\def\pgfmath@qparse@operand@{%
	\edef\pgfmathresult{\pgfmath@tonumber{\pgfmath@dimen}}%
	\pgfmath@parse@operator%
}

\def\pgfmath@parse@operand{%
	\let\pgfmathresult=\pgfmath@empty%
	\let\pgfmath@mantisse=\pgfmath@empty%
	\let\pgfmath@exponent=\pgfmath@empty%
	\let\pgfmath@number=\pgfmath@empty%
	\c@pgfmath@count=1\relax%
	\expandafter\pgfmath@parse@sign\pgfmath@token}
	
\def\pgfmath@parse@sign#1{%
	% Remove multiple signs.
	\ifx#1-%
		\c@pgfmath@count=-\c@pgfmath@count%
		\let\pgfmath@next=\pgfmath@parse@sign%
	\else%
		\ifx#1+%
			\let\pgfmath@next=\pgfmath@parse@sign%
		\else%
			\ifnum\c@pgfmath@count>0\relax%
				\let\pgfmath@sign=\pgfmath@empty%
			\else%
				\def\pgfmath@sign{-}%
			\fi%
			\def\pgfmath@insert@token{#1}%
			\let\pgfmath@insert@after=\pgfmath@parse@postsign%
			\let\pgfmath@next=\pgfmath@insert%
		\fi%
	\fi%
	\pgfmath@next%
}

\def\pgfmath@parse@postsign{%
	\if\pgfmath@insert@token(% Check for the start of a group.
		\let\pgfmath@next=\pgfmath@parse@startgroup%
	\else%
		\let\pgfmath@token=\pgfmath@insert@token%
		\let\pgfmath@next=\pgfmath@parse@number%
	\fi%
	\pgfmath@next}
	
\def\pgfmath@ifregister@dimen#1{%
	\let\pgfmath@register=#1%
	\edef\pgfmath@register@value{\the#1}%
	\afterassignment\pgfmath@@ifregister@dimen%
	#1=0pt\relax\pgfmath@}
	
\def\pgfmath@@ifregister@dimen#1#2\pgfmath@{%
	\ifx#1p%
		\let\pgfmath@next=\pgfmath@secondoftwo%
	\else%
		\let\pgfmath@next=\pgfmath@firstoftwo%
	\fi%
	\pgfmath@register=\pgfmath@register@value\relax% Restore value.
	\pgfmath@next%
}

\def\pgfmath@parse@number#1{%
	\ifx#1@% The end of the parse?
		\ifx\pgfmathresult\pgfmath@empty%
			\edef\pgfmathresult{\pgfmath@sign\pgfmath@number}%
		\fi%
		\let\pgfmath@next=\pgfmath@parse@operator%
	\else%
		\ifcat#1\relax% An unexpandable control sequence?
			\ifx#1\wd%
				\let\pgfmath@boxdimen=\wd%
				\let\pgfmath@next=\pgfmath@parse@box%
			\else%
				\ifx#1\ht%
					\let\pgfmath@boxdimen=\ht%
					\let\pgfmath@next=\pgfmath@parse@box%
				\else%
					\ifx#1\dp%
						\let\pgfmath@boxdimen=\dp%
						\let\pgfmath@next=\pgfmath@parse@box%
					\else%
						\pgfmath@ifregister@dimen#1{% A dimension or skip register?
							\ifx\pgfmath@number\pgfutil@empty%
								\edef\pgfmathresult{\pgfmath@sign\pgfmath@tonumber{#1}}%
							\else%
								\pgfmath@dimen=\pgfmath@number#1\relax%
								\edef\pgfmathresult{\pgfmath@sign\pgfmath@tonumber{\pgfmath@dimen}}%
							\fi%
							\global\pgfmathunitsdeclaredtrue% A dimension has units.
							\let\pgfmath@next=\pgfmath@parse@operator%
						}%
						{% So, a count register.
							\let\pgfmath@number=\pgfmath@register@value%
							\let\pgfmathresult=\pgfmath@register@value%
							\let\pgfmath@next=\pgfmath@parse@number%
						}%
					\fi%
				\fi%
			\fi%
		\else%
			\chardef\pgfmath@char=`#1\relax%
			\edef\pgfmath@char{\the\pgfmath@char}% Strange, but necessary.
			\ifnum\pgfmath@char=46\relax% A Period?
				\edef\pgfmath@number{\pgfmath@number.}%
				\let\pgfmath@next=\pgfmath@parse@number%
			\else%
				\let\pgfmath@next=\relax%
				\ifnum\pgfmath@char<58\relax% A digit?
					\ifnum\pgfmath@char>47\relax%
						\edef\pgfmath@number{\pgfmath@number#1}%
						\let\pgfmath@next=\pgfmath@parse@number%
					\fi%
				\fi%
				\ifx\pgfmath@next\relax%
					\def\pgfmath@insert@token{#1}%
					\ifnum\pgfmath@char>64\relax% From A-Z?
						\ifnum\pgfmath@char<91\relax%
							\let\pgfmath@next=\pgfmath@parse@alpha%
						\else%
							\ifnum\pgfmath@char>96\relax% From a-z?
								\ifnum\pgfmath@char<123\relax%
									\let\pgfmath@next=\pgfmath@parse@alpha%
								\fi%
							\fi%
						\fi%
					\fi%
					\ifx\pgfmath@next\relax% Must be an operator.
						\edef\pgfmathresult{\pgfmath@sign\pgfmath@number}%
						\let\pgfmath@insert@after=\pgfmath@parse@operator%
						\let\pgfmath@next=\pgfmath@insert%
					\fi%
				\fi%
			\fi%
		\fi%
	\fi%
	\pgfmath@next%
}

\def\pgfmath@parse@box#1{%
	\ifx\pgfmath@number\pgfutil@empty%
		\edef\pgfmathresult{\pgfmath@sign\the\pgfmath@boxdimen#1}%
	\else%
		\pgfmath@dimen=\pgfmath@sign\pgfmath@number\pgfmath@boxdimen#1\relax%
		\edef\pgfmathresult{\pgfmath@tonumber{\pgfmath@dimen}}%
	\fi%
	\global\pgfmathunitsdeclaredtrue% a box dimension has units.
	\pgfmath@parse@operator%
}

\def\pgfmath@parse@alpha#1{%
	\ifx\pgfmath@number\pgfmath@empty% A function.
		\edef\pgfmath@insert@token{\pgfmath@insert@token#1}%
		\let\pgfmath@insert@after=\pgfmath@parse@function%
		\let\pgfmath@next\pgfmath@insert%
	\else%
		\expandafter\if\pgfmath@insert@token e%
			\if#1x% Check for ex unit...
				\def\pgfmath@unit{ex}%
				\let\pgfmath@next=\pgfmath@parse@units%
			\else%
				\if#1m% ...check for em unit...
					\def\pgfmath@unit{em}%
					\let\pgfmath@next=\pgfmath@parse@units%
				\else% ...it's an exponent.
					\let\pgfmath@e@token\pgfmath@insert@token%
					\def\pgfmath@insert@token{#1}%
					\let\pgfmath@insert@after=\pgfmath@parse@exponent%
					\let\pgfmath@next=\pgfmath@insert%
				\fi%
			\fi%
		\else%
			\expandafter\if\pgfmath@insert@token E% The capital exponent.
				\let\pgfmath@e@token\pgfmath@insert@token%
				\def\pgfmath@insert@token{#1}%
				\let\pgfmath@insert@after=\pgfmath@parse@exponent%
				\let\pgfmath@next=\pgfmath@insert%
			\else%
				\expandafter\if\pgfmath@insert@token r\relax% The postfix r function?
					\edef\pgfmathresult{\pgfmath@sign\pgfmath@number}%
					\edef\pgfmath@insert@token{\pgfmath@insert@token#1}%
					\let\pgfmath@insert@after=\pgfmath@parse@operator%
					\let\pgfmath@next=\pgfmath@insert%
				\else%
					\edef\pgfmath@unit{\pgfmath@insert@token#1}%
					\edef\pgfmathresult{\pgfmath@sign\pgfmath@number}% Must be units.
					\let\pgfmath@next=\pgfmath@parse@units%
				\fi%
			\fi%
		\fi%
	\fi%
	\pgfmath@next%
}

\def\pgfmath@parse@exponent{%
	\let\pgfmath@mantisse=\pgfmath@number%
	\afterassignment\pgfmath@@parse@exponent% Parse exponent by assignment...
	\c@pgfmath@count=}

\def\pgfmath@@parse@exponent{% ...as it shouldn't be as big as 2^31-1
	\edef\pgfmath@exponent{\the\c@pgfmath@count}%
	\edef\pgfmathresult{\pgfmath@sign\pgfmath@number\pgfmath@e@token\the\c@pgfmath@count}%
	\pgfmath@parse@scientific}
	
\def\pgfmath@parse@units{%
	\edef\pgfmathresult{\pgfmath@sign\pgfmath@number}%
	\global\pgfmathunitsdeclaredtrue%
	\pgfmath@dimen\pgfmathresult\pgfmath@unit\relax%
	\edef\pgfmathresult{\pgfmath@tonumber{\pgfmath@dimen}}%
	\pgfmath@parse@operator%
}

\def\pgfmath@insert{\expandafter\pgfmath@insert@after\pgfmath@insert@token}%


% Process 'Scientific' notation in the form xEy
%
\def\pgfmath@parse@scientific{%
	\pgfmathscientific{\pgfmath@mantisse}{\pgfmath@exponent}% Comment this line out.
	% 
	% Here is an example of the creation of a result object. To test it,
	% comment out the first line in this macro to prevent evaluation.
	% 
	%	\let\pgfmathresultobject=\pgfmath@empty%
	%	\pgfmath@adddef@tomacro{\pgfmath@mantisse}{\pgfmathresultobject}%
	%	\pgfmath@adddef@tomacro{\pgfmath@exponent}{\pgfmathresultobject}%
	%	\pgfmath@adddef@tomacro{\pgfmathresult}{\pgfmathresultobject}%
	%	\let\pgfmathresult=\pgfmathresultobject%
	%
	\pgfmath@parse@operator}
	
\def\pgfmath@adddef@tomacro#1#2{%
	\edef\pgfmath@deftemp{\noexpand\def\noexpand#1{#1}}%
	\expandafter\expandafter\expandafter\def%
		\expandafter\expandafter\expandafter#2%
			\expandafter\expandafter\expandafter{\expandafter#2\pgfmath@deftemp}%
}



% Functions parsed (not calculated) here are:
%
% round(X)         'half-up' rounding.
% floor(X)         floor function.
% ciel(X)          ceiling function.
% abs(X)           absolute function.
%
% exp(X)           e^X (0 <= X <~= 9.7).
% ln(X)            logarithm of X.
% pow(X,Y)         X^Y,
%
% sin(X)           sine function.
% cos(X)           cosine function.
% tan(X)           tan function.
% asin(X)          arcsine of X (in radians)    -1 <= X <= 1
% acos(X)          arccosine of X (in radians)  -1 <= X <= 1
% atan(X)          arctangent of X (in radians) -1 <= X <= 1
% veclen(X,Y)      the length Z where Z^2 = X^2 + Y^2
% mod(X,Y)         X modulo Y
% max(X,Y)         the maximum of X or Y
% min(X,Y)         the minimum of X or Y
%
% NB veclen, mod, max, and min *cannot* be nested.
%
% deg(X)           converts X to degrees (X in radians)
% rad(X)           converts X to radians (X in degrees)
%
% rnd              generate pseudo-random number X (0 <= X <= 1).
% rand             generate pseudo-random number X (1 <= X <= -1).
% sqrt(X)          square root.
% 
% pi               the constant PI.
%


% Functions for compatability with calc.
%
\def\pgfmath@parse@function@widthof(#1){%
	\setbox\pgfmath@box=\hbox{{\let\nullfont=\relax\pgfmath@selectfont#1}}%
	\edef\pgfmathresult{\pgfmath@tonumber{\wd\pgfmath@box}}%
	\pgfmath@parse@operator}

\def\pgfmath@parse@function@heightof(#1){%
	\setbox\pgfmath@box=\hbox{{\let\nullfont=\relax\pgfmath@selectfont#1}}%
	\edef\pgfmathresult{\pgfmath@tonumber{\ht\pgfmath@box}}%
	\pgfmath@parse@operator}

\def\pgfmath@parse@function@depthof(#1){%
	\setbox\pgfmath@box=\hbox{{\let\nullfont=\relax\pgfmath@selectfont#1}}%
	\edef\pgfmathresult{\pgfmath@tonumber{\dp\pgfmath@box}}%
	\pgfmath@parse@operator}

\def\pgfmath@parse@function{%
	\let\pgfmath@parse@function@name=\pgfmath@empty%
	\pgfmath@parse@@function}

\def\pgfmath@parse@@function{\futurelet\pgfmath@let@token\pgfmath@parse@@@function}

\def\pgfmath@parse@@@function#1{%
	\def\pgfmath@parse@token{#1}%
	\ifx\pgfmath@let@token\pgfutil@sptoken%
		\let\pgfmath@next=\pgfmath@parse@@@@function%
	\else%
		\chardef\pgfmath@char=`#1\relax%
		\edef\pgfmath@char{\the\pgfmath@char}%
		\let\pgfmath@next=\pgfmath@parse@@@@function%
		\ifnum\pgfmath@char>96\relax% From a-z?
			\ifnum\pgfmath@char<123\relax%
				\expandafter\def\expandafter\pgfmath@parse@function@name\expandafter{%
					\pgfmath@parse@function@name#1}%
				\let\pgfmath@next=\pgfmath@parse@@function%
			\fi%
		\fi%
	\fi%
	\pgfmath@next}


\def\pgfmath@parse@@@@function{%
	\expandafter\ifx\csname pgfmath@parse@function@\pgfmath@parse@function@name\endcsname\relax%
		\pgfmath@parse@function@unknown%
		\let\pgfmath@next=\relax%
	\else%
		\expandafter\let\expandafter\pgfmath@next\expandafter=%
			\csname pgfmath@parse@function@\pgfmath@parse@function@name\endcsname%
	\fi%
	\expandafter\pgfmath@next\pgfmath@parse@token}

\def\pgfmath@parse@function@unknown{%
	\pgfmath@error{Unknown function `\pgfmath@parse@function@name'}{}%
}

% \pgfmath@postfunction
% 
% In scanning a function e.g. sin(40), we subvert the normal parsing 
% group mechanism by messing around with \pgfmath@parse@postgroup, so 
% that after scanning ), the parser doesn't scan for an operator, but 
% returns to the macros scanning the function. 
% Here the mechanism is restored, and the value of the function is 
% stored along with the approprate sign, which was saved earlier.
%
\def\pgfmath@postfunction{%
	\let\pgfmath@parse@postgroup=\pgfmath@parse@operator%
	\edef\pgfmathresult{\pgfmath@sign\pgfmathresult}%
	\pgfmath@parse@operator}
	

% \pgfmath@parse@function@abs
%
\def\pgfmath@parse@function@abs{%
	\let\pgfmath@parse@postgroup=\pgfmath@parse@function@abs@%
	\pgfmath@parse@}
\def\pgfmath@parse@function@abs@{%	
	\expandafter\pgfmathabs@\expandafter{\pgfmathresult}%
	\pgfmath@postfunction%
}

% \pgfmath@parse@function@sqrt
%
\def\pgfmath@parse@function@sqrt{%
	\let\pgfmath@parse@postgroup=\pgfmath@parse@function@sqrt@%
	\pgfmath@parse@}
\def\pgfmath@parse@function@sqrt@{%	
	\expandafter\pgfmathsqrt@\expandafter{\pgfmathresult}%
	\pgfmath@postfunction%
}

% \pgfmath@parse@function@round
%
\def\pgfmath@parse@function@round{%
	\let\pgfmath@parse@postgroup=\pgfmath@parse@function@round@%
	\pgfmath@parse@}
\def\pgfmath@parse@function@round@{%	
	\expandafter\pgfmathround@\expandafter{\pgfmathresult}%
	\pgfmath@postfunction%
}

% \pgfmath@parse@function@floor
%
\def\pgfmath@parse@function@floor{%
	\let\pgfmath@parse@postgroup=\pgfmath@parse@function@floor@%
	\pgfmath@parse@}
\def\pgfmath@parse@function@floor@{%	
	\expandafter\pgfmathfloor@\expandafter{\pgfmathresult}%
	\pgfmath@postfunction%
}

% \pgfmath@parse@function@ceil
%
\def\pgfmath@parse@function@ceil{%
	\let\pgfmath@parse@postgroup=\pgfmath@parse@function@ceil@%
	\pgfmath@parse@}
\def\pgfmath@parse@function@ceil@{%	
	\expandafter\pgfmathceil@\expandafter{\pgfmathresult}%
	\pgfmath@postfunction%
}

% \pgfmath@parse@function@sin
%
\def\pgfmath@parse@function@sin{%
	\let\pgfmath@parse@postgroup=\pgfmath@parse@function@sin@%
	\pgfmath@parse@}
\def\pgfmath@parse@function@sin@{%	
	\expandafter\pgfmathsin@\expandafter{\pgfmathresult}%
	\pgfmath@postfunction%
}

% \pgfmath@parse@function@cos
%
\def\pgfmath@parse@function@cos{%
	\let\pgfmath@parse@postgroup=\pgfmath@parse@function@cos@%
	\pgfmath@parse@}
\def\pgfmath@parse@function@cos@{%	
	\expandafter\pgfmathcos@\expandafter{\pgfmathresult}%
	\pgfmath@postfunction%
}

% \pgfmath@parse@function@asin
%
\def\pgfmath@parse@function@asin{%
	\let\pgfmath@parse@postgroup=\pgfmath@parse@function@asin@%
	\pgfmath@parse@}
\def\pgfmath@parse@function@asin@{%	
	\expandafter\pgfmathasin@\expandafter{\pgfmathresult}%
	\pgfmath@postfunction%
}

% \pgfmath@parse@function@acos
%
\def\pgfmath@parse@function@acos{%
	\let\pgfmath@parse@postgroup=\pgfmath@parse@function@acos@%
	\pgfmath@parse@}
\def\pgfmath@parse@function@acos@{%	
	\expandafter\pgfmathacos@\expandafter{\pgfmathresult}%
	\pgfmath@postfunction}

% \pgfmath@parse@function@atan
%
\def\pgfmath@parse@function@atan{%
	\let\pgfmath@parse@postgroup=\pgfmath@parse@function@atan@%
	\pgfmath@parse@}
\def\pgfmath@parse@function@atan@{%	
	\expandafter\pgfmathatan@\expandafter{\pgfmathresult}%
	\pgfmath@postfunction%
}

% \pgfmath@parse@function@tan
%
\def\pgfmath@parse@function@tan{%
	\let\pgfmath@parse@postgroup=\pgfmath@parse@function@tan@%
	\pgfmath@parse@}
\def\pgfmath@parse@function@tan@{%	
	\expandafter\pgfmathtan@\expandafter{\pgfmathresult}%
	\pgfmath@postfunction%
}

% \pgfmath@parse@function@cot
%
\def\pgfmath@parse@function@cot{%
	\let\pgfmath@parse@postgroup=\pgfmath@parse@function@cot@%
	\pgfmath@parse@}
\def\pgfmath@parse@function@cot@{%	
	\expandafter\pgfmathcot@\expandafter{\pgfmathresult}%
	\pgfmath@postfunction%
}

% \pgfmath@parse@function@sec
%
\def\pgfmath@parse@function@sec{%
	\let\pgfmath@parse@postgroup=\pgfmath@parse@function@sec@%
	\pgfmath@parse@}
\def\pgfmath@parse@function@sec@{%	
	\expandafter\pgfmathsec@\expandafter{\pgfmathresult}%
	\pgfmath@postfunction%
}

% \pgfmath@parse@function@cosec
%
\def\pgfmath@parse@function@cosec{%
	\let\pgfmath@parse@postgroup=\pgfmath@parse@function@cosec@%
	\pgfmath@parse@}
\def\pgfmath@parse@function@cosec@{%	
	\expandafter\pgfmathcosec@\expandafter{\pgfmathresult}%
	\pgfmath@postfunction%
}

% \pgfmath@parse@function@rad
%
\def\pgfmath@parse@function@rad{%
	\let\pgfmath@parse@postgroup=\pgfmath@parse@function@rad@%
	\pgfmath@parse@}
\def\pgfmath@parse@function@rad@{%
	\expandafter\pgfmathrad@\expandafter{\pgfmathresult}%
	\pgfmath@postfunction}%
	
% \pgfmath@parse@function@rad
%
\def\pgfmath@parse@function@deg{%
	\let\pgfmath@parse@postgroup=\pgfmath@parse@function@deg@%
	\pgfmath@parse@}
\def\pgfmath@parse@function@deg@{%
	\expandafter\pgfmathdeg@\expandafter{\pgfmathresult}%
	\pgfmath@postfunction}%
	
% \pgfmath@parse@function@rnd
%
\def\pgfmath@parse@function@rnd{%
	\pgfmathrnd%
	\pgfmath@postfunction}
			
% \pgfmath@parse@function@rand
%
\def\pgfmath@parse@function@rand{%
	\pgfmathrand%
	\pgfmath@postfunction}%
			
% \pgfmath@parse@function@exp
%
\def\pgfmath@parse@function@exp{%
	\let\pgfmath@parse@postgroup=\pgfmath@parse@function@exp@%
	\pgfmath@parse@}
\def\pgfmath@parse@function@exp@{%	
	\expandafter\pgfmathexp@\expandafter{\pgfmathresult}%
	\pgfmath@postfunction%
}

% \pgfmath@parse@function@ln
%
\def\pgfmath@parse@function@ln{%
	\let\pgfmath@parse@postgroup=\pgfmath@parse@function@ln@%
	\pgfmath@parse@}
\def\pgfmath@parse@function@ln@{%	
	\expandafter\pgfmathln@\expandafter{\pgfmathresult}%
	\pgfmath@postfunction%
}
	
% \pgfmath@parse@function@pi
%
\def\pgfmath@parse@function@pi{%
	\pgfmathpi%
	\pgfmath@postfunction%
}

% \pgfmath@parse@function@veclen
%
\def\pgfmath@parse@function@veclen(#1,{%
	\pgfmathparse@{#1}%
	\let\pgfmath@firstoperand=\pgfmathresult%
	\let\pgfmath@parse@postgroup=\pgfmath@parse@function@veclen@%
	\pgfmath@parse@startgroup}
\def\pgfmath@parse@function@veclen@{%
	\let\pgfmath@secondoperand=\pgfmathresult%
	\pgfmathveclen@{\pgfmath@firstoperand}{\pgfmath@secondoperand}%
	\pgfmath@postfunction}

% \pgfmath@parse@function@mod 
%
\def\pgfmath@parse@function@mod(#1,{%
	\pgfmathparse@{#1}%
	\let\pgfmath@firstoperand=\pgfmathresult%
	\let\pgfmath@parse@postgroup=\pgfmath@parse@function@mod@%
	\pgfmath@parse@startgroup}
\def\pgfmath@parse@function@mod@{%
	\let\pgfmath@secondoperand=\pgfmathresult%
	\pgfmathmod@{\pgfmath@firstoperand}{\pgfmath@secondoperand}%
	\pgfmath@postfunction}

% \pgfmath@parse@function@max
%
\def\pgfmath@parse@function@max(#1,{%
	\pgfmathparse@{#1}%
	\let\pgfmath@firstoperand=\pgfmathresult%
	\let\pgfmath@parse@postgroup=\pgfmath@parse@function@max@%
	\pgfmath@parse@startgroup}
\def\pgfmath@parse@function@max@{%
	\let\pgfmath@secondoperand=\pgfmathresult%
	\pgfmathmax@{\pgfmath@firstoperand}{\pgfmath@secondoperand}%
	\pgfmath@postfunction}	

% \pgfmath@parse@function@min 
%
\def\pgfmath@parse@function@min(#1,{%
	\pgfmathparse@{#1}%
	\let\pgfmath@firstoperand=\pgfmathresult%
	\let\pgfmath@parse@postgroup=\pgfmath@parse@function@min@@
	\pgfmath@parse@startgroup}
\def\pgfmath@parse@function@min@@{%
	\let\pgfmath@secondoperand=\pgfmathresult%
	\pgfmathmin@{\pgfmath@firstoperand}{\pgfmath@secondoperand}%
	\pgfmath@postfunction}
	
% \pgfmath@parse@function@pow 
%
\def\pgfmath@parse@function@pow(#1,{%
	\pgfmathparse@{#1}%
	\let\pgfmath@firstoperand=\pgfmathresult%
	\let\pgfmath@parse@postgroup=\pgfmath@parse@function@pow@@
	\pgfmath@parse@startgroup}
\def\pgfmath@parse@function@pow@@{%
	\let\pgfmath@secondoperand=\pgfmathresult%
	\pgfmathpow@{\pgfmath@firstoperand}{\pgfmath@secondoperand}%
	\pgfmath@postfunction}	
	
% Copyright 2007 by Mark Wibrow
%
% This file may be distributed and/or modified
%
% 1. under the LaTeX Project Public License and/or
% 2. under the GNU Public License.
%
% See the file doc/generic/pgf/licenses/LICENSE for more details.

% This file defines the mathematical functions and operators.
%
% Version 0.0 08/03/2007

% This file defines the mathematical functions and operators.
%
% Adding/redefining extra operators/functions:
%
% Each operator/function XXX has two forms:
%
%
% \pgfmathXXX#1...   a public version which evaluates any
%                    arguments passed to it and passes the
%                    results on to...
%
% \pgfmathXXX@#1...  a non-public version which performs 
%                    required calculation on arguments which
%                    must have already been evaluated (i.e.
%                    *without* dimensions).
% 
% If a function XXX is to be included in the parser, it is 
% recommended, for consistency, that where possible, the 
% pgfmathparser file should define the macro \pgfmath@parseXXX.
% The parser should (ideally) then call \pgfmathXXX@.
%
% It is recommend that the pgfmath versions of the pgf dimension
% and count registers be used, i.e., \pgfmath@x for \pgfmath@x, 
% \c@pgfmath@counta for c@pgfmath@counta, and so on. These are currently
% \let to their pgf equivalents, but it may be necessary to change 
% this.
%
% It is also recommened that all calculations (where necessary)
% take place within a TeX group. \pgfmath@returnone#1 makes and
% expanded version of #1 global and stores this in \pgfmathresult 
% after the group is ended.
%

\input pgfmathtrig.code.tex% Load the trig. stuff.
\input pgfmathrnd.code.tex%  Load the random stuff.


% \pgfmathadd
%
% Add #1 and #2.
%
\def\pgfmathadd#1#2{%
	\pgfmathparse{#1}\edef\pgfmath@adda{\pgfmathresult}%
	\pgfmathparse{#2}\edef\pgfmath@addb{\pgfmathresult}%
	\pgfmathadd@{\pgfmath@adda}{\pgfmath@addb}}
\def\pgfmathadd@#1#2{%
	\begingroup%
		\expandafter\pgfmath@x#1pt\relax%
		\expandafter\pgfmath@y#2pt\relax%
		\advance\pgfmath@x by\pgfmath@y%
		\pgfmath@returnone\pgfmath@x%
	\endgroup%
}

% \pgfmathsubtract
%
% Subtract #2 from #1.
%
\def\pgfmathsubtract#1#2{%
	\pgfmathparse{#1}\edef\pgfmath@subtracta{\pgfmathresult}%
	\pgfmathparse{#2}\edef\pgfmath@subtractb{\pgfmathresult}%
	\pgfmathsubtract@{\pgfmath@subtracta}{\pgfmath@subtractb}}

\def\pgfmathsubtract@#1#2{%
	\begingroup%
		\expandafter\pgfmath@x#1pt\relax%
		\expandafter\pgfmath@y#2pt\relax%
		\advance\pgfmath@x by-\pgfmath@y%
		\pgfmath@returnone\pgfmath@x%
	\endgroup%
}

% \pgfmathmultiply
%
% Multiply #1 by #2.
%
\def\pgfmathmultiply#1#2{%
	\pgfmathparse{#1}\edef\pgfmath@multiplya{\pgfmathresult}%
	\pgfmathparse{#2}\edef\pgfmath@multiplyb{\pgfmathresult}%
	\pgfmathmultiply@{\pgfmath@multiplya}{\pgfmath@multiplyb}}
\def\pgfmathmultiply@#1#2{%
	\begingroup%
		\expandafter\pgfmath@x#1pt\relax%
		\expandafter\pgfmath@x#2\pgfmath@x%
		\pgfmath@returnone\pgfmath@x%
	\endgroup%
}

% \pgfmathdivide
%
% Divide #1 by #2.
%
\def\pgfmathdivide#1#2{%
	\pgfmathparse{#1}\edef\pgfmath@dividea{\pgfmathresult}%
	\pgfmathparse{#2}\edef\pgfmath@divideb{\pgfmathresult}%
	\pgfmathdivide@{\pgfmath@dividea}{\pgfmath@divideb}}
\def\pgfmathdivide@#1#2{%
	\begingroup%
		\expandafter\pgfmath@x#1pt\relax%
		% If #2 is an integer use TeX arithmatic.
		\expandafter\pgfmath@xa#2pt\relax%
		\afterassignment\pgfmath@xa%
		\expandafter\c@pgfmath@counta\the\pgfmath@xa\relax%
		\ifdim\pgfmath@xa=0pt\relax%
			\divide\pgfmath@x\c@pgfmath@counta%
		\else%
			\pgfmathreciprocal@{#2}%
			\pgfmath@x=\pgfmathresult\pgfmath@x%
		\fi%
		\pgfmath@returnone\pgfmath@x%
	\endgroup%
}

% \pgfmathgreaterthan
%
% 1.0 if #1 > #2. Otherwise 0.0
%
\def\pgfmathgreaterthan#1#2{%
	\pgfmathparse{#1}\edef\pgfmath@greaterthana{\pgfmathresult}%
	\pgfmathparse{#2}\edef\pgfmath@greaterthanb{\pgfmathresult}%
	\pgfmathgreaterthan@{\pgfmath@greaterthana}{\pgfmath@greaterthanb}}
\def\pgfmathgreaterthan@#1#2{%
	\begingroup%
		\expandafter\pgfmath@x#1pt\relax%
		\expandafter\pgfmath@y#2pt\relax%
		\advance\pgfmath@x-\pgfmath@y%
		\ifdim\pgfmath@x>0pt\relax%
			\pgfmath@x1pt\relax%
		\else%
			\pgfmath@x0pt\relax%
		\fi%
		\pgfmath@returnone\pgfmath@x%
	\endgroup%
}

% \pgfmathlessthan
%
% 1.0 if #1< #2. Otherwise 0.0
%
\def\pgfmathlessthan#1#2{%
	\pgfmathparse{#1}\edef\pgfmath@lessthana{\pgfmathresult}%
	\pgfmathparse{#2}\edef\pgfmath@lessthanb{\pgfmathresult}%
	\pgfmathlessthan@{\pgfmath@lessthana}{\pgfmath@lessthanb}}
\def\pgfmathlessthan@#1#2{%
	\begingroup%
		\expandafter\pgfmath@x#1pt\relax%
		\expandafter\pgfmath@y#2pt\relax%
		\advance\pgfmath@x-\pgfmath@y\relax%
		\ifdim\pgfmath@x<0pt\relax%
			\pgfmath@x1pt\relax%
		\else%
			\pgfmath@x0pt\relax%
		\fi%
		\pgfmath@returnone\pgfmath@x%
	\endgroup%
}

% \pgfmathequalto
%
% 1.0 if #1 = #2. Otherwise 0.0
%
\def\pgfmathequalto#1#2{%
	\pgfmathparse{#1}\edef\pgfmath@equaltoa{\pgfmathresult}%
	\pgfmathparse{#2}\edef\pgfmath@equaltob{\pgfmathresult}%
	\pgfmathadd@{\pgfmath@equaltoa}{\pgfmath@equaltob}}
\def\pgfmathequalto@#1#2{%
	\begingroup%
		\expandafter\pgfmath@x#1pt\relax%
		\expandafter\pgfmath@y#2pt\relax%
		\advance\pgfmath@x-\pgfmath@y%
		\ifdim\pgfmath@x=0pt\relax%
			\pgfmath@x1pt\relax%
		\else%
			\pgfmath@x0pt\relax%
		\fi%
		\pgfmath@returnone\pgfmath@x%
	\endgroup%
}

% \pgfmathreciprocal
%
% 1 / #1
%
\def\pgfmathreciprocal#1{%
	\pgfmathparse{#1}%
	\pgfmathreciprocal@{\pgfmathresult}}
\def\pgfmathreciprocal@#1{%
	\begingroup%
		\expandafter\pgfmath@x#1pt\relax%
		\edef\pgfmath@reciprocaltemp{\pgfmath@tonumber{\pgfmath@x}}%
		\expandafter\pgfmathreciprocal@@\pgfmath@reciprocaltemp00000\pgfmath@}
\def\pgfmathreciprocal@@#1.#2#3#4#5#6#7\pgfmath@{%
		\c@pgfmath@counta#2#3#4#5#6\relax%
		% If the number is an integer, use TeX arithmatic.
		\ifnum\c@pgfmath@counta=0\relax%
			\pgfmath@x1pt\relax%
			\divide\pgfmath@x#1\relax%
		\else%
			\c@pgfmath@counta#1#2#3#4#5#6\relax%
			\c@pgfmath@countb1000000000\relax%
			\divide\c@pgfmath@countb\c@pgfmath@counta%
			\c@pgfmath@counta\c@pgfmath@countb%
			\divide\c@pgfmath@counta10000\relax%
			\pgfmath@x\c@pgfmath@counta pt\relax%
			\multiply\c@pgfmath@counta-10000\relax%
			\advance\c@pgfmath@countb\c@pgfmath@counta%
			\pgfmath@y\c@pgfmath@countb pt\relax%
			\pgfmath@y.1\pgfmath@y% Yes! This way is more accurate. Go figure...
			\pgfmath@y.1\pgfmath@y%	
			\pgfmath@y.1\pgfmath@y%	
			\pgfmath@y.1\pgfmath@y%			
			\advance\pgfmath@x\pgfmath@y%
		\fi%
		\pgfmath@returnone\pgfmath@x%
	\endgroup
}

	
% \pgfmathabs
%
% Calculate |#1|
%
\def\pgfmathabs#1{%
	\pgfmathparse{#1}%
	\pgfmathabsolute@{\pgfmathresult}}
\def\pgfmathabs@#1{%
	\begingroup%
		\expandafter\pgfmath@x#1pt\relax%
		\ifdim\pgfmath@x<0pt\relax%
			\pgfmath@x=-\pgfmath@x%
		\fi%
	\pgfmath@returnone\pgfmath@x%
	\endgroup%
}

% \pgfmathmod
%
% Calculate #1 mod #2.
%
\def\pgfmathmod#1#2{%
	\pgfmathparse{#1}\edef\pgfmath@moda{\pgfmathresult}%
	\pgfmathparse{#2}\edef\pgfmath@modb{\pgfmathresult}%
	\pgfmathmod@{\pgfmath@mod@a}{\pgfmath@modb}%
}
\def\pgfmathmod@#1#2{%
	\begingroup%
		\expandafter\pgfmath@x#1pt\relax%
		\pgfmath@xa\pgfmath@x%
		\expandafter\pgfmath@xb#2pt\relax%
		\c@pgfmath@counta=\pgfmath@xa%
		\c@pgfmath@countb=\pgfmath@xb%
		\divide\c@pgfmath@counta\c@pgfmath@countb%
		\multiply\pgfmath@xb\c@pgfmath@counta%
		\advance\pgfmath@x-\pgfmath@xb%
		\pgfmath@returnone\pgfmath@x%
	\endgroup%
}

% \pgfmathsqrt
%
% Square-root of #1.
%
%
\def\pgfmathsqrt#1{%
	\pgfmathparse{#1}%
	\pgfmathsqrt@{\pgfmathresult}}
\def\pgfmathsqrt@#1{%
	\begingroup%
		\expandafter\pgfmath@x#1pt\relax%
		\pgfmath@x.01\pgfmath@x%
		\pgfmath@xa\pgfmath@x%
		\pgfmath@xb\pgfmath@x%
		\pgfmathloop
			\pgfmath@xc\pgfmath@x%
			% If pgfmath@x >= 128pt, we get an Arithmetic overflow, so...
			% If x^2 >= 16384 then 16384/x < x
			\pgfmath@y=16383.99999pt\relax%
			\c@pgfmath@counta=\pgfmath@x%
			\divide\c@pgfmath@counta by655360\relax% Can't remember why we need the extra zero.
			\ifnum\c@pgfmath@counta=0\relax%
				\c@pgfmath@counta1\relax%
			\fi%
			\divide\pgfmath@y\c@pgfmath@counta%
			\ifdim\pgfmath@y<\pgfmath@x%
			\else%
				\pgfmath@x\pgfmath@tonumber{\pgfmath@x}\pgfmath@x%
				\advance\pgfmath@x-\pgfmath@xa\relax%
				\pgfmath@ya\pgfmath@x%
				\pgfmathreciprocal@{\pgfmath@tonumber{\pgfmath@xc}}%
				\pgfmath@x\pgfmathresult\pgfmath@ya%
				\pgfmath@x-.5\pgfmath@x%
				\advance\pgfmath@x\pgfmath@xb%
			\fi%
			% 10 iterations seems converge for most numbers.
			\ifnum\pgfmathcounter=10\relax%
			\else
				\pgfmath@xb\pgfmath@x%
		\repeatpgfmathloop%
		\pgfmath@x10.0\pgfmath@x%
		\pgfmath@returnone\pgfmath@x%
	\endgroup%
}


% \pgfmathpow
%
% Calculates #1 ^ #2
%
% #2 is expected to be an integer.
%
\def\pgfmathpow#1#2{%
	\pgfmathparse{#1}\edef\pgfmath@powera{\pgfmathresult}%
	\pgfmathparse{#2}\edef\pgfmath@powerb{\pgfmathresult}%
	\pgfmathpow@{\pgfmath@powera}{\pgfmath@powerb}}
\def\pgfmathpow@#1#2{%
	\begingroup%
		\expandafter\pgfmath@xa#1pt\relax%
		\afterassignment\pgfmath@gobbletilpgfmath@%
		\expandafter\c@pgfmath@counta#2\relax\pgfmath@
		% If #2 is negative, take the reciprocal of #1
		% and the absolute value of #2, and carry on.
		%
		\ifnum\c@pgfmath@counta<0\relax%
			\c@pgfmath@counta-\c@pgfmath@counta%
			\pgfmathreciprocal@{#1}%
			\pgfmath@xa\pgfmathresult pt\relax%
		\fi%
		\pgfmath@x=1pt\relax%
		\pgfmathloop%
			\ifnum\c@pgfmath@counta>0\relax%
				\ifodd\c@pgfmath@counta%
					\pgfmath@x\pgfmath@tonumber{\pgfmath@x}\pgfmath@xa%
				\fi
				\ifnum\c@pgfmath@counta>1\relax%
					\pgfmath@xa=\pgfmath@tonumber{\pgfmath@xa}\pgfmath@xa%
				\fi%
				\divide\c@pgfmath@counta by 2\relax%
		\repeatpgfmathloop%
		\pgfmath@returnone\pgfmath@x%
	\endgroup%
}	


% \pgfmathround
% 
% Half-up rounding.
%
\def\pgfmathround#1{%
	\pgfmathparse{#1}%
	\pgfmathround@{\pgfmathresult}}
\def\pgfmathround@#1{%
	\begingroup%
		\expandafter\pgfmath@x#1pt\relax%
		\afterassignment\pgfmath@xa%
		\expandafter\c@pgfmath@counta\the\pgfmath@x\relax%
		\pgfmath@xb\pgfmath@x%
		\ifdim\pgfmath@xb<0pt\relax%
			\ifdim\pgfmath@xa<0.5pt\relax%
			\else%
				\advance\c@pgfmath@counta-1\relax%
			\fi%
		\else%
			\ifdim\pgfmath@xa<0.5pt\relax%
			\else%
				\advance\c@pgfmath@counta1\relax%
			\fi%
		\fi%
		\pgfmath@returnone\c@pgfmath@counta%
	\endgroup%
}%

% \pgfmathfloor
% 
% Floor function.
%
\def\pgfmathfloor#1{%
	\pgfmathparse{#1}%
	\expandafter\pgfmathfloor@\expandafter{\pgfmathresult}}
\def\pgfmathfloor@#1{%
	\begingroup%
		\expandafter\pgfmath@x#1pt\relax%
		\afterassignment\pgfmath@gobbletilpgfmath@%
		\expandafter\c@pgfmath@counta\the\pgfmath@x\relax\pgfmath@%
		\pgfmath@x\c@pgfmath@counta pt\relax%
		\pgfmath@returnone\pgfmath@x%
	\endgroup
}%

% \pgfmathceil
% 
% Ceiling function.
%
\def\pgfmathceil#1{%
	\pgfmathparse{#1}%
	\expandafter\pgfmathceil@\expandafter{\pgfmathresult}}
\def\pgfmathceil@#1{%
	\begingroup%
		\expandafter\pgfmath@x#1pt\relax%
		\afterassignment\pgfmath@gobbletilpgfmath@%
		\expandafter\c@pgfmath@counta\the\pgfmath@x\relax\pgfmath@%
		\pgfmath@y\pgfmath@x%
		\advance\pgfmath@y-\c@pgfmath@counta pt\relax%
		\pgfmath@x\c@pgfmath@counta pt\relax%
		\ifdim\pgfmath@y>0pt\relax%
			\advance\pgfmath@x1pt\relax%
		\fi%
	\pgfmath@returnone\pgfmath@x%
	\endgroup%
}%

% \pgfmathexp
%
% A Maclaurens expansion for e^#1.
% 0 <= #1 < ln(16384).
%
\def\pgfmathexp#1{%
	\pgfmathparse{#1}%
	\expandafter\pgfmathexp@\expandafter{\pgfmathresult}}
\def\pgfmathexp@#1{%
	\begingroup%
		\pgfmath@x1pt\relax%
		\pgfmath@xa1pt\relax%
		\pgfmath@xb\pgfmath@x%
		\pgfmathloop%
			\pgfmath@xc\pgfmathcounter pt\relax%
			\c@pgfmath@counta\pgfmath@xc%
			\divide\c@pgfmath@counta65536\relax%
			\pgfmath@xc1pt\relax%
			\divide\pgfmath@xc\c@pgfmath@counta%
			\pgfmath@xa\pgfmath@tonumber{\pgfmath@xc}\pgfmath@xa%
			\expandafter\pgfmath@xa#1\pgfmath@xa%
			\advance\pgfmath@x\pgfmath@xa%
			\ifdim\pgfmath@x=\pgfmath@xb%
			\else%
				\pgfmath@xb\pgfmath@x%
		\repeatpgfmathloop%
	\pgfmath@returnone\pgfmath@x%
	\endgroup%
}



% \pgfmathvectorlength
%
% Calcluate the Eulidean length of a 2D vector.
%
% This based on polynomial approximation co-efficents
% contributed by Rouben Rostamian.
%
% #1 - the x component of the vector.
% #2 - the y component of the vector.
%
% P(x) = c0 + x^2 * (c1 + x^2 * (c2 + x^2 * ( c3 + c4 * x^2)))
\def\pgfmath@cE{-0.01019}
\def\pgfmath@cD{0.04453}
\def\pgfmath@cC{-0.11951}
\def\pgfmath@cB{0.49936}
\def\pgfmath@cA{1.00001}

\def\pgfmathveclen#1#2{%
	\pgfmathparse{#1}\edef\pgfmath@vecx{\pgfmathresult}%
	\pgfmathparse{#2}\edef\pgfmath@vecy{\pgfmathresult}%
	\pgfmathveclen@{\pgfmath@vecx}{\pgfmath@vecy}%
}
\def\pgfmathveclen@#1#2{%
	\begingroup%
		\expandafter\pgfmath@x#1pt\relax%
		\expandafter\pgfmath@y#2pt\relax%
		\pgfmath@xa\pgfmath@x%
		\ifdim\pgfmath@xa=0pt\relax%
			\pgfmath@xa\pgfmath@y%
		\fi%
		\ifdim\pgfmath@xa=0pt\relax%
		\else%
			\ifdim\pgfmath@x<0pt\relax\pgfmath@x-\pgfmath@x\fi%
			\ifdim\pgfmath@y<0pt\relax\pgfmath@y-\pgfmath@y\fi%
			\ifdim\pgfmath@x>\pgfmath@y%
				\pgfmath@xa\pgfmath@x%
				\pgfmath@x\pgfmath@y%
				\pgfmath@y\pgfmath@xa%
			\fi%
			% We use a scaling factor to reduce errors.
			\ifdim\pgfmath@y>10000pt\relax%
				\c@pgfmath@counta1500\relax%
			\else%
				\ifdim\pgfmath@y>1000pt\relax%
					\c@pgfmath@counta150\relax%
				\else%
					\ifdim\pgfmath@y>100pt\relax%
						\c@pgfmath@counta50\relax%
					\else%
						\c@pgfmath@counta1\relax%
					\fi%
				\fi%
			\fi%
			\divide\pgfmath@x by\c@pgfmath@counta\relax%
			\divide\pgfmath@y by\c@pgfmath@counta\relax%
			\pgfmathreciprocal@{\pgfmath@tonumber{\pgfmath@y}}%
			\pgfmath@x=\pgfmathresult\pgfmath@x%
			\pgfmath@xa=\pgfmath@tonumber{\pgfmath@x}\pgfmath@x%
			\edef\pgfmath@xsq{\pgfmath@tonumber{\pgfmath@xa}}%
			\pgfmath@x=\pgfmath@cE\pgfmath@xa%
			\advance\pgfmath@x by\pgfmath@cD pt\relax%
			\pgfmath@x=\pgfmath@xsq\pgfmath@x%
			\advance\pgfmath@x by\pgfmath@cC pt\relax%
			\pgfmath@x=\pgfmath@xsq\pgfmath@x%
			\advance\pgfmath@x by\pgfmath@cB pt\relax%
			\pgfmath@x=\pgfmath@xsq\pgfmath@x%
			\advance\pgfmath@x by\pgfmath@cA pt\relax%
			\ifdim\pgfmath@y<0pt\relax%
				\pgfmath@y=-\pgfmath@y%
			\fi%
			\pgfmath@x=\pgfmath@tonumber{\pgfmath@y}\pgfmath@x%
			% Invert the scaling factor.
			\multiply\pgfmath@x by\c@pgfmath@counta\relax%
		\fi%
		\pgfmath@returnone\pgfmath@x%
	\endgroup%
}

% \pgfmathmax
%
% Return the maximum of #1 or #2
%
\def\pgfmathmax#1#2{%
	\pgfmathparse@{#1}\edef\pgfmath@firstoperand{\pgfmathresult}%
	\pgfmathparse@{#2}\edef\pgfmath@secondoperand{\pgfmathresult}%
	\pgfmathmax@{\pgfmath@firstoperand}{\pgfmath@secondoperand}}
\def\pgfmathmax@#1#2{%
	\begingroup
		\expandafter\pgfmath@x#1pt\relax%
		\expandafter\pgfmath@y#2pt\relax%
		\ifdim\pgfmath@x>\pgfmath@y%
			\pgfmath@returnone\pgfmath@x%
		\else%
			\pgfmath@returnone\pgfmath@y%
		\fi%
	\endgroup}

% \pgfmathmax
%
% Return the minimim of #1 or #2
%
\def\pgfmathmin#1#2{%
	\pgfmathparse@{#1}\edef\pgfmath@firstoperand{\pgfmathresult}%
	\pgfmathparse@{#2}\edef\pgfmath@secondoperand{\pgfmathresult}%
	\pgfmathmin@{\pgfmath@firstoperand}{\pgfmath@secondoperand}}
\def\pgfmathmin@#1#2{%
	\begingroup
		\expandafter\pgfmath@x#1pt\relax%
		\expandafter\pgfmath@y#2pt\relax%
		\ifdim\pgfmath@x<\pgfmath@y%
			\pgfmath@returnone\pgfmath@x%
		\else%
			\pgfmath@returnone\pgfmath@y%
		\fi%
	\endgroup%
}

% \pgfmathscientific
%
% Return the value of #1e#2
%
% e.g. \pgfmathscientific{1.23456789123}{4}
%
% defines \pgfmathresult as 12345.67891
%
% NB This arguments *are not parsed*, as the long mantissa would be lost.
%
\def\pgfmathscientific#1#2{%
	\begingroup%
		\edef\pgfmath@sci@exponent{#2}%
		\expandafter\pgfmath@scientific@@#100000000000\pgfmath@}

\def\pgfmath@scientific@@#1.#2#3#4#5#6{%
		\edef\pgfmath@sci@int{#1}%
		\edef\pgfmath@sci@mantissaA{#2#3#4#5#6}%
		\pgfmath@scientific@@@}
	
\def\pgfmath@scientific@@@#1#2#3#4#5#6\pgfmath@{%
		\edef\pgfmath@sci@mantissaB{#1#2#3#4#5}%
		\c@pgfmath@counta\pgfmath@sci@exponent\relax%
		\c@pgfmath@countb\c@pgfmath@counta%
		\ifnum\c@pgfmath@counta<0\relax%
			\c@pgfmath@counta-\c@pgfmath@counta%
		\fi%
		\pgfmathpow@{10}{\the\c@pgfmath@counta}%
		\afterassignment\pgfmath@gobbletilpgfmath@
		\c@pgfmath@countc\pgfmathresult\relax\pgfmath@
		\edef\pgfmath@sci@factor{\the\c@pgfmath@countc}%
		\ifnum\c@pgfmath@countb<0\relax%
			% xE-y: easy...
			\pgfmath@x\pgfmath@sci@int.\pgfmath@sci@mantissaA pt\relax%
			\divide\pgfmath@x\pgfmath@sci@factor\relax%
		\else%
			% xE+y: 
			% Must do this way so as not lose digits in a long mantissa. Sigh...
			\c@pgfmath@counta\pgfmath@sci@int%
			\c@pgfmath@countb\pgfmath@sci@mantissaA%
			\multiply\c@pgfmath@counta\pgfmath@sci@factor\relax%
			\multiply\c@pgfmath@countb\pgfmath@sci@factor\relax%
			\c@pgfmath@countc\c@pgfmath@countb%
			\divide\c@pgfmath@countb100000\relax%
			\advance\c@pgfmath@counta\c@pgfmath@countb%
			\multiply\c@pgfmath@countb100000\relax%
			\advance\c@pgfmath@countc-\c@pgfmath@countb%
			\c@pgfmath@countb\pgfmath@sci@mantissaB\relax%
			\multiply\c@pgfmath@countb\pgfmath@sci@factor\relax%
			\divide\c@pgfmath@countb100000\relax%
			\advance\c@pgfmath@countc\c@pgfmath@countb%
			\edef\pgfmath@sci@result{\the\c@pgfmath@counta.\the\c@pgfmath@countc pt}%
			\pgfmath@x\pgfmath@sci@result\relax%
		\fi%
		\pgfmath@returnone\pgfmath@x%
	\endgroup}


% \pgfmathsetlength, \pgfmathaddtolength
%
% #1 = dimension register
% #2 = expression
%
% Description:
%
% These functions work similar to \setlength and \addtolength. Only,
% they allow #2 to contain an expression, which is evaluated before
% assignment. Furthermore, the font is setup before the assignment is
% done, so that dimensions like 1em are evaluated correctly.
%
% If #2 starts with "+", then a simple assignment is done (but the
% font is still setup). This is orders of magnitude faster than a
% parsed assignment.

\newdimen\mydim
\def\pgfmathsetlength#1#2{%
  \expandafter\pgfmath@onquick#2@@\pgfmath@%
  {%
    % Ok, quick version:
    #1#2\relax%
  }%
  {\pgfmathparse{#2}#1\pgfmathresult pt\relax}%
}
\def\pgfmathaddtolength#1#2{%
  \expandafter\pgfmath@onquick#2@@\pgfmath@%
  {%
    % Ok, quick version:
    \advance#1by#2\relax%
  }%
  {\pgfmathparse{#2}\advance#1\pgfmathresult pt\relax}%
}

% \pgfmathsetcounter, \pgfmathaddtocounter
%
% Results of parsing are truncated.
%
\def\pgfmathsetcounter#1#2{%
  \expandafter\pgfmath@onquick#2@@\pgfmath@%
  {%
    \csname c@#1\endcsname=#2\relax%
  }%
  {%
    \pgfmath@ifundefined{c@#1}{\pgfmath@error{No counter named '#1' is known}{}}{%
      \pgfmathparse{#2}%
      \afterassignment\pgfmath@gobbletilpgfmath@%
      \csname c@#1\endcsname\pgfmathresult\relax\pgfmath@%
    }%
  }%
}

\def\pgfmathaddtocounter#1#2{%
  \expandafter\pgfmath@onquick#2@@\pgfmath@%
  {%
    \advance\csname c@#1\endcsname by#2\relax%
  }%
  {%
    \pgfmath@ifundefined{c@#1}{\pgfmath@error{No counter named '#1' is known}{}}{%
      \edef\pgfmath@addtocountertemp{\expandafter\the\csname c@#1\endcsname}%
      \pgfmathparse{#2}%
      \afterassignment\pgfmath@gobbletilpgfmath@%
      \csname c@#1\endcsname\pgfmathresult\relax\pgfmath@%
      \expandafter\advance\csname c@#1\endcsname\pgfmath@addtocountertemp%
    }%
  }%
}


% Check whether a given parameter starts with quick.
%
% The command should be followed by nonempty text, ending with
% \pgfmath@ as a stop-token. Then should follow
%
% #1 = code to execute if text starts with +
% #2 = code to execute if text does not
%
% Example:
%
% \pgfmath@onquick+0pt\pgfmath@{is quick}{is slow}

\def\pgfmath@onquick{%
  \afterassignment\pgfmath@afterquick%
  \let\pgfmath@next=%
}

\def\pgfmath@afterquick#1\pgfmath@{%
  \ifx\pgfmath@next+%
    \expandafter\pgfmath@firstoftwo%
  \else%
    \expandafter\pgfmath@secondoftwo%
  \fi%
}




% \pgfmathdeclarefunction
%
% Declare a function to be used with \pgfmathusefunction
%
% #1 -> the name of the function.
% #2 -> a list of variables used by the function. When the
%       function is called with a list of arguments, these
%       variables will be instantiated to the arguments.
% #3 -> the function definition.
%
% E.g.
%
% \pgfmathdeclarefunction{my function}{\t}{sin{\t r}*60}
%
\def\pgfmathdeclarefunction#1#2#3{%	
	\expandafter\def\csname pgfmath@function@#1\endcsname{#3}%
	\expandafter\def\csname pgfmath@function@#1@arguments@\endcsname{#2}%
}%

% \pgfmathusefunction
%
% Call a declared function with appropriate arguments to
% calculate the value of the function. 
%
% Assume the function f has been declared using
%
% \pgfmathdeclarefunction{f}{\t}{3*\t+4}
%
% We say:
%
% \pgfmathusefunction{\y}{f}{\x}
%
% where \x is a macro representing a number, which forms
% the argument to the function. This will calculate f(x)
% and place the value (*without* dimension) in the macro 
% \y. The \t in the function definition is equated with
% the value of the argument \x.
%
% #1 -> the macro to store the evaluated function.
% #1 -> the function name.
% #2 -> a list of arguments.
%
\def\pgfmathusefunction#1#2#3{%
	\def\pgfmath@functionname{#2}%
	\pgfmath@ifundefined{pgfmath@function@#2}{%
		\pgfmath@error{Unknown function `#2'.}}{%
			\begingroup%
				\expandafter\ifx\csname pgfmath@function@#2@arguments@\endcsname\pgfmath@empty
				\else%
					\expandafter%
					\pgfmath@equatevariables\expandafter{\csname pgfmath@function@#2@arguments@\endcsname}{#3}%
				\fi%
				\edef\pgfmath@temp{\csname pgfmath@function@#2\endcsname}%
				\expandafter\pgfmathparse\expandafter{\pgfmath@temp}%
				\pgfmath@x=\pgfmathresult pt\relax%
				\pgfmath@returnone\pgfmath@x
			\endgroup%
			\edef#1{\pgfmathresult}%
		}%
}

% \pgfmath@equatevariables
%
% Internal macro for assigning a list of values to
% a list of variables. 
%
% \pgfmath@equatevariables{\a,\b,\c}{1.2,4,6.73}
%
% will resulit in \a <- 1.2, \b <- 4 and \c <- 6.73
%
% NB Assumes \pgfmath@functionname is defined.
%
% #1 -> list of variables.
% #2 -> list of arguments.
%
\def\pgfmath@equatevariables#1#2{%
		\expandafter\pgfmath@equatevariables@\expandafter{#2}{#1}}%
\def\pgfmath@equatevariables@#1#2{%
	\expandafter\pgfmath@equatevariables@@#2,\pgfmath@empty,\pgfmath@empty,\pgfmath@%
		#1,\pgfmath@empty,\pgfmath@empty,\pgfmath@}%
\def\pgfmath@equatevariables@@#1,#2,\pgfmath@#3,#4,\pgfmath@{%
	\ifx\pgfmath@empty#1\relax%
		\ifx\pgfmath@empty#3\relax%
		\else%
			\pgfmath@reporterror{Function `\pgfmath@functionname' called with too many variables}{}%
		\fi%
		\let\pgfmath@next\pgfmath@endequatevariables%
	\else%
		\ifx\pgfmath@empty#3\relax%
			\pgfmath@reporterror{Function `\pgfmath@functionname' called with too few variables}{}%
			\let\pgfmath@next\pgfmath@endequatevariables%
		\else%
			\expandafter\pgfmathparse\expandafter{#3}%
			\edef#1{\pgfmathresult}%
			\let\pgfmath@next\pgfmath@equatevariables@@%
		\fi%
	\fi%
	\pgfmath@next#2,\pgfmath@#4,\pgfmath@}

\def\pgfmath@endequatevariables#1\pgfmath@#2\pgfmath@{}


% \pgfmathlinearspace
%
% Partition a linear space.
%
% #1 -> the macro to store the linear space.
% #2 -> the first term.
% #3 -> the last term.
% #4 -> the number of partitions.
%
% e.g.
% \pgfmathlinearspace{\x}{0}{20}{4}
% 
% Results in \x -> 0.0,5.0,10.0,15.0,20.0
%
\def\pgfmathlinearspace#1#2#3#4{%
	\begingroup%
		\def\pgfmath@temppartitions{#4}%
		\ifx\pgfmath@temppartitions\pgfmath@empty%
			\def\pgfmath@temppartitions{100}%
		\fi%
		\pgfmathsetlength{\pgfmath@xa}{#2}%
		\pgfmathsetlength{\pgfmath@xb}{#3}%
		\pgfmath@xc=\pgfmath@xb%
		\advance\pgfmath@xb by-\pgfmath@xa%
		\pgfmath@y=\pgfmath@xb%
		\divide\pgfmath@y by\pgfmath@temppartitions%
		\xdef\pgfmathlinearspace@tempa{\pgfmath@tonumber{\pgfmath@y}}%
		\xdef\pgfmathlinearspace@tempb{\pgfmath@tonumber{\pgfmath@xa}}%
		\pgfmathloop%
			\ifnum\pgfmathcounter<#4\relax%
				\pgfmath@x=\pgfmath@xb%
				\divide\pgfmath@x by#4\relax%
				\multiply\pgfmath@x by\pgfmathcounter\relax%
				\advance\pgfmath@x by\pgfmath@xa%
				\xdef\pgfmathlinearspace@tempb{\pgfmathlinearspace@tempb,\pgfmath@tonumber{\pgfmath@x}}%
		\repeatpgfmathloop%
		\xdef\pgfmathlinearspace@tempb{\pgfmathlinearspace@tempb,\pgfmath@tonumber{\pgfmath@xc}}%
	\endgroup%
	\edef\pgfmathlinearspacepartitionwidth{\pgfmathlinearspace@tempa}%
	\edef#1{\pgfmathlinearspace@tempb}%	
}

% \pgfmathlinearseries
%
% Create a linear series.
%
% #1 -> the macro to store the linear space.
% #2 -> the first term.
% #3 -> the common difference.
% #4 -> the number of terms.
%
% e.g.
% \pgfmathlinearseries{\x}{0}{3.5}{5}
% 
% Results in \x -> 0.0,3.5,7.0,10.5,14.0
%
\def\pgfmathlinearseries#1#2#3#4{%
	\begingroup
		\pgfmathsetlength{\pgfmath@x}{#2}%
		\pgfmathsetlength{\pgfmath@xa}{#3}%
		\pgfmathsetcounter{pgfmath@counta}{#4}%
		\edef\pgfmathlinearseries@temp{\pgfmath@tonumber{\pgfmath@x}}%
		\pgfmathloop
			\ifnum\pgfmathcounter<\c@pgfmath@counta
				\advance\pgfmath@x by\pgfmath@xa%
				\xdef\pgfmathlinearseries@temp{\pgfmathlinearseries@temp,\pgfmath@tonumber{\pgfmath@x}}%
		\repeatpgfmathloop
	\endgroup
	\edef#1{\pgfmathlinearseries@temp}%	
}

% \foreach
%
% Extended to support linear series and linear spaces.
% Now it is possible to say:
%
% \foreach \x in linear space [from=0, to=2*pi, partitions=100]...
% or...
% \foreach \x in linear space [0:2*pi:100]...
%
% and...
%
% \foreach \x in linear series [first term=4, common difference=2, terms=10]...
% or...
% \foreach \x in linear series [4:2:10]...
%
%
\def\foreach#1in{%
	\def\pgffor@var{#1}%
	\pgfmath@ifnextchar l{%
		\pgfmath@collectlinearspaceorseries}{%
			\pgfmath@continueforeach}}

\def\pgfmath@collectlinearspaceorseries linear s{%
	\pgfmath@ifnextchar p{%
		\pgfmath@collectlinearspace}{%
			\pgfmath@ifnextchar e{%
				\pgfmath@collectlinearseries}{%
  					\pgfmath@reporterror{Unknown 'linear' option in \foreach.}{}%
  	}}%
}%

%
% Removed [PgfMath] since xkeyval currently is not necessarily available.
% 


\define@key{pgfmath linear space}{from}[0.0]{%
	\pgfmathparse{#1}%
	\edef\pgfmath@linearspace@from{\pgfmathresult}}%
\define@key{pgfmath linear space}{to}[100]{%
	\pgfmathparse{#1}%
	\edef\pgfmath@linearspace@to{\pgfmathresult}}%
\define@key{pgfmath linear space}{partitions}[10]{%
	\pgfmathparse{#1}%
	\edef\pgfmath@linearspace@partitions{\pgfmathresult}}%
\def\pgfmath@collectlinearspace pace#1[#2]{% #1 is a dummy.
	\pgfmath@ifin@{:}{#2}{\pgfmath@collectlinearspace@#2\pgfmath@}{%
		\setkeys{pgfmath linear space}{#2}%
		\pgfmathlinearspace{\pgfmathforeach@temp}%
			{\pgfmath@linearspace@from}%
			{\pgfmath@linearspace@to}%
			{\pgfmath@linearspace@partitions}%
		\expandafter\pgfmath@continueforeach\expandafter{\pgfmathforeach@temp}%
	}%
}
\def\pgfmath@collectlinearspace@#1:#2:#3\pgfmath@{%
	\pgfmathlinearspace{\pgfmathforeach@temp}{#1}{#2}{#3}%
	\expandafter\pgfmath@continueforeach\expandafter{\pgfmathforeach@temp}%
}

\define@key{pgfmath linear series}{first term}[0.0]{%
	\pgfmathparse{#1}%
	\edef\pgfmath@linearseries@from{\pgfmathresult}}%
\define@key{pgfmath linear series}{common difference}[100.0]{%
	\pgfmathparse{#1}%
	\edef\pgfmath@linearseries@difference{\pgfmathresult}}%
\define@key{pgfmath linear series}{terms}[10]{%
	\pgfmathparse{#1}%
	\edef\pgfmath@linearseries@terms{\pgfmathresult}}%
\def\pgfmath@collectlinearseries eries#1[#2]{% #1 is a dummy.
	\pgfmath@ifin@{:}{#2}{\pgfmath@collectlinearseries@#2\pgfmath@}{%
		\setkeys{pgfmath linear series}{#2}%
		\pgfmathlinearseries{\pgfmathforeach@temp}%
			{\pgfmath@linearseries@from}%
			{\pgfmath@linearseries@difference}%
			{\pgfmath@linearseries@terms}%
		\expandafter\pgfmath@continueforeach\expandafter{\pgfmathforeach@temp}%
	}%
}
\def\pgfmath@collectlinearseries@#1:#2:#3\pgfmath@{%
	\pgfmathlinearseries{\pgfmathforeach@temp}{#1}{#2}{#3}%
	\expandafter\pgfmath@continueforeach\expandafter{\pgfmathforeach@temp}%
}

\def\pgfmath@continueforeach#1{%
	\def\pgffor@values{#1, \pgffor@stop,}%
	\ifx\pgffor@values\pgffor@emptyvalues%
		\def\pgffor@values{\pgffor@stop,}%
	\fi%
	\let\pgffor@body\pgfutil@empty%
	\global\pgffor@continuetrue%
	\pgffor@collectbody}


\def\pgfpathfunctionto#1#2#3#4{%
	\pgfmathparse{#2}\edef\pgfmath@functionstart{\pgfmathresult}%
	\pgfmathparse{#3}\edef\pgfmath@functionend{\pgfmathresult}%
	\pgfmathparse{#2+#4}\edef\pgfmath@functionfirststep{\pgfmathresult}%
	\foreach \pgfmath@functiontemp in{\pgfmath@functionstart, \pgfmath@functionfirststep,...,\pgfmath@functionend}{%
		\pgfmathusefunction{\pgfmath@functiony}{#1}{\pgfmath@functiontemp}%
		\pgfpathlineto{\pgfpoint{\pgfmath@functiontemp pt}{\pgfmath@functiony pt}}%
	}%
}%

\def\pgfpathbetweenfunctionandxaxis#1#2#3#4{%
	\pgfmathparse{#2}\edef\pgfmath@functionstart{\pgfmathresult}%
	\pgfmathparse{#3+#4}\edef\pgfmath@functionend{\pgfmathresult}%
	\pgfmathparse{#2+#4}\edef\pgfmath@functionfirststep{\pgfmathresult}%
	\pgfpathmoveto{\pgfpoint{#2}{0}}%
	\foreach\pgfmath@functiontemp in{\pgfmath@functionstart,\pgfmath@functionfirststep,...,\pgfmath@functionend}{%
		\pgfmathusefunction{\pgfmath@functiony}{#1}{\pgfmath@functiontemp}%
		\pgfpathlineto{\pgfpoint{\pgfmath@functiontemp}{\pgfmath@functiony}}%
	}%
	\pgfmathparse{#3}%
	\pgfpathlineto{\pgfpoint{\pgfmathresult}{0}}%
	\pgfpathclose%
}%

\def\pgfpathbetweenfunctionandyaxis#1#2#3#4{%
	\pgfmathparse{#2}\edef\pgfmath@functionstart{\pgfmathresult}%
	\pgfmathparse{#3+#4}\edef\pgfmath@functionend{\pgfmathresult}%
	\pgfmathparse{#2+#4}\edef\pgfmath@functionfirststep{\pgfmathresult}%
	\pgfmathusefunction{\pgfmath@functiony}{#1}{#2}%
	\pgfpathmoveto{\pgfpoint{0pt}{\pgfmath@functiony pt}}%
	\foreach\pgfmath@functiontemp in{\pgfmath@functionstart,\pgfmath@functionfirststep,...,\pgfmath@functionend}{%
		\pgfmathusefunction{\pgfmath@functiony}{#1}{\pgfmath@functiontemp}%
		\pgfpathlineto{\pgfpoint{\pgfmath@functiontemp pt}{\pgfmath@functiony pt}}%
	}%
	\pgfmathusefunction{\pgfmath@functiony}{#1}{#3}%
	\pgfpathlineto{\pgfpoint{0pt}{\pgfmath@functiony pt}}%
	\pgfpathclose%
}%
