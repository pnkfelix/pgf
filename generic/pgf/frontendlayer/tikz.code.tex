\ProvidesPackageRCS[v\pgfversion] $Header: /cvsroot/pgf/pgf/generic/pgf/frontendlayer/Attic/tikz.code.tex,v 1.38 2005/12/12 17:50:54 tantau Exp $

% Copyright 2005 by Till Tantau <tantau@cs.tu-berlin.de>.
%
% This program can be redistributed and/or modified under the terms
% of the GNU Public License, version 2.


\newdimen\tikz@lastx
\newdimen\tikz@lasty
\newdimen\tikz@lastxsaved
\newdimen\tikz@lastysaved

\newdimen\tikzleveldistance
\newdimen\tikzsiblingdistance

\newbox\tikz@figbox
\newbox\tikz@tempbox

\newcount\tikztreelevel
\newcount\tikznumberofchildren
\newcount\tikznumberofcurrentchild

\newcount\tikz@fig@count

\newif\iftikz@node@is@a@label
\newif\iftikz@snaked

\let\tikz@options=\@empty
\let\tikz@transform=\relax
\def\tikz@addoption#1{\expandafter\def\expandafter\tikz@options\expandafter{\tikz@options#1}}
\def\tikz@addmode#1{\expandafter\def\expandafter\tikz@mode\expandafter{\tikz@mode#1}}
\def\tikz@addtransform#1{%
  \ifx\tikz@transform\relax%
    #1%
  \else%
    \expandafter\def\expandafter\tikz@transform\expandafter{\tikz@transform#1}%
  \fi%
}



% Baseline options
\define@key{tikz}{baseline}[0pt]{\pgfsetbaseline{#1}}%

% Draw options
\define@key{tikz}{line width}{\tikz@semiaddlinewidth{#1}}%

\def\tikz@semiaddlinewidth#1{\tikz@addoption{\pgfsetlinewidth{#1}}\setlength\pgflinewidth{#1}}

\define@key{tikz}{cap}{\tikz@addoption{\csname pgfset#1cap\endcsname}}
\define@key{tikz}{join}{\tikz@addoption{\csname pgfset#1join\endcsname}}
\define@key{tikz}{miter limit}{\tikz@addoption{\pgfsetmiterlimit{#1}}}

\define@key{tikz}{dash pattern}{% syntax: on 2pt off 3pt on 4pt ...
  \def\tikz@temp{#1}%
  \ifx\tikz@temp\@empty%
    \def\tikz@dashpattern{}%
    \tikz@addoption{\pgfsetdash{}{0pt}}%
  \else%
    \def\tikz@dashpattern{}%
    \expandafter\tikz@scandashon\@gobble#1o\@nil%
    \edef\tikz@temp{{\tikz@dashpattern}{\noexpand\tikz@dashphase}}%
    \expandafter\tikz@addoption\expandafter{\expandafter\pgfsetdash\tikz@temp}%
  \fi}
\define@key{tikz}{dash phase}{%
  \def\tikz@dashphase{#1}%
  \edef\tikz@temp{{\tikz@dashpattern}{\noexpand\tikz@dashphase}}%
  \expandafter\tikz@addoption\expandafter{\expandafter\pgfsetdash\tikz@temp}%
}%
\def\tikz@dashphase{0pt}

\def\tikz@scandashon n#1o{%
  \expandafter\def\expandafter\tikz@dashpattern\expandafter{\tikz@dashpattern{#1}}%
  \@ifnextchar\@nil{\@gobble}{\tikz@scandashoff}}
\def\tikz@scandashoff ff#1o{%
  \expandafter\def\expandafter\tikz@dashpattern\expandafter{\tikz@dashpattern{#1}}%
  \@ifnextchar\@nil{\@gobble}{\tikz@scandashon}}

\define@key{tikz}{draw opacity}{\tikz@addoption{\pgfsetstrokeopacity{#1}}}

% Double draw options
\define@key{tikz}{double}[]{%
  \def\tikz@temp{#1}%
  \ifx\tikz@temp\@empty%
  \else%
    \def\tikz@double@color{#1}%
  \fi%
  \tikz@addmode{\tikz@mode@doubletrue}}
\define@key{tikz}{double distance}{%
  \setlength{\pgf@x}{#1}%
  \edef\tikz@double@width@distance{\the\pgf@x}%
  \tikz@addmode{\tikz@mode@doubletrue}}

\def\tikz@double@width@distance{0.6pt}
\def\tikz@double@color{white}

% Fill options

\define@key{tikz}{even odd rule}[]{\tikz@addoption{\pgfseteorule}}
\define@key{tikz}{nonzero rule}[]{\tikz@addoption{\pgfsetnonzerorule}}

\define@key{tikz}{fill opacity}{\tikz@addoption{\pgfsetfillopacity{#1}}}


% Joined fill/draw options

\define@key{tikz}{opacity}{\tikz@addoption{\pgfsetstrokeopacity{#1}\pgfsetfillopacity{#1}}}


% Main color options
\define@key{tikz}{color}{%
  \tikz@addoption{%
    \ifx\tikz@fillcolor\@empty%
      \ifx\tikz@strokecolor\@empty%
      \else%
        \pgfsys@color@reset@inorderfalse%
        \let\tikz@strokecolor\@empty%
        \let\tikz@fillcolor\@empty%
      \fi%
    \else%
      \pgfsys@color@reset@inorderfalse%
      \let\tikz@strokecolor\@empty%
      \let\tikz@fillcolor\@empty%
    \fi%
    \colorlet{tikz@color}{#1}%
    \colorlet{.}{tikz@color}%
    \pgfsetcolor{.}%
    \pgfsys@color@reset@inordertrue%
  }%
  \def\tikz@textcolor{#1}}


% Rounding options
\define@key{tikz}{rounded corners}[4pt]{\pgfsetcornersarced{\pgfpoint{#1}{#1}}}
\define@key{tikz}{sharp corners}[]{\pgfsetcornersarced{\pgfpointorigin}}



% Coordinate options
\define@key{tikz}{x}{\tikz@handle@vec{\pgfsetxvec}{\tikz@handle@x}#1\relax}
\define@key{tikz}{y}{\tikz@handle@vec{\pgfsetyvec}{\tikz@handle@y}#1\relax}
\define@key{tikz}{z}{\tikz@handle@vec{\pgfsetzvec}{\tikz@handle@z}#1\relax}

\def\tikz@handle@vec#1#2{\@ifnextchar({\tikz@handle@coordinate#1}{\tikz@handle@single#2}}
\def\tikz@handle@coordinate#1{\tikz@scan@one@point#1}
\def\tikz@handle@single#1#2\relax{#1{#2}}
\def\tikz@handle@x#1{\pgfsetxvec{\pgfpoint{#1}{0pt}}}
\def\tikz@handle@y#1{\pgfsetyvec{\pgfpoint{0pt}{#1}}}
\def\tikz@handle@z#1{\pgfsetzvec{\pgfpoint{#1}{#1}}}


% Transformation options
\define@key{tikz}{scale}{\tikz@addtransform{\pgftransformscale{#1}}}
\define@key{tikz}{xscale}{\tikz@addtransform{\pgftransformxscale{#1}}}
\define@key{tikz}{xslant}{\tikz@addtransform{\pgftransformxslant{#1}}}
\define@key{tikz}{yscale}{\tikz@addtransform{\pgftransformyscale{#1}}}
\define@key{tikz}{yslant}{\tikz@addtransform{\pgftransformyslant{#1}}}
\define@key{tikz}{rotate}{\tikz@addtransform{\pgftransformrotate{#1}}}
\define@key{tikz}{rotate around}{\tikz@addtransform{\tikz@rotatearound{#1}}}
\def\tikz@rotatearound#1{%
  \edef\tikz@temp{#1}% get rid of active stuff
  \expandafter\tikz@rotateparseA\tikz@temp%
}%
\def\tikz@rotateparseA#1:{%
  \def\tikz@temp@rot{#1}%
  \tikz@scan@one@point\tikz@rotateparseB%
}
\def\tikz@rotateparseB#1{%
  \pgf@process{#1}%
  \pgf@xc=\pgf@x%
  \pgf@yc=\pgf@y%
  \pgftransformshift{\pgfpoint{\pgf@xc}{\pgf@yc}}%
  \pgftransformrotate{\tikz@temp@rot}%
  \pgftransformshift{\pgfpoint{-\pgf@xc}{-\pgf@yc}}%
}

\define@key{tikz}{shift}{\tikz@addtransform{\tikz@scan@one@point\pgftransformshift#1\relax}}
\define@key{tikz}{xshift}{\tikz@addtransform{\pgftransformxshift{#1}}}
\define@key{tikz}{yshift}{\tikz@addtransform{\pgftransformyshift{#1}}}
\define@key{tikz}{cm}{\tikz@addtransform{\tikz@parse@cm#1\relax}}
\define@key{tikz}{reset cm}{\tikz@addtransform{\pgftransformreset}}

\def\tikz@parse@cm#1,#2,#3,#4,{%
  \def\tikz@p@cm{{#1}{#2}{#3}{#4}}%
  \tikz@scan@one@point\tikz@parse@cmA}
\def\tikz@parse@cmA#1{%
  \expandafter\pgftransformcm\tikz@p@cm{#1}%
}



% Grid options
\define@key{tikz}{xstep}{\def\tikz@grid@x{#1}}
\define@key{tikz}{ystep}{\def\tikz@grid@y{#1}}
\define@key{tikz}{step}{\def\tikz@grid@x{#1}\def\tikz@grid@y{#1}}

\def\tikz@grid@x{1cm}
\def\tikz@grid@y{1cm}


% Path usage options
\newif\iftikz@mode@double
\newif\iftikz@mode@fill
\newif\iftikz@mode@draw
\newif\iftikz@mode@clip
\newif\iftikz@mode@boundary
\newif\iftikz@mode@shade
\let\tikz@mode=\@empty

\def\tikz@nonetext{none}

\define@key{tikz}{path only}[]{\let\tikz@mode=\@empty}
\define@key{tikz}{shade}[]{\tikz@addmode{\tikz@mode@shadetrue}}
\define@key{tikz}{fill}[]{%
  \def\tikz@temp{#1}%
  \ifx\tikz@temp\tikz@nonetext%
    \tikz@addmode{\tikz@mode@fillfalse}%
  \else%
    \ifx\tikz@temp\@empty%
    \else%
      \tikz@addoption{\pgfsetfillcolor{#1}}%
      \def\tikz@fillcolor{#1}%
    \fi%
    \tikz@addmode{\tikz@mode@filltrue}%
  \fi%
}
\define@key{tikz}{draw}[]{%
  \def\tikz@temp{#1}%
  \ifx\tikz@temp\tikz@nonetext%
    \tikz@addmode{\tikz@mode@drawfalse}%
  \else%
    \ifx\tikz@temp\@empty%
    \else%
      \tikz@addoption{\pgfsetstrokecolor{#1}}%
      \def\tikz@strokecolor{#1}%
    \fi%
    \tikz@addmode{\tikz@mode@drawtrue}%
  \fi%
}
\define@key{tikz}{clip}[]{\tikz@addmode{\tikz@mode@cliptrue}}
\define@key{tikz}{use as bounding box}[]{\tikz@addmode{\tikz@mode@boundarytrue}}

\define@key{tikz}{save path}{\tikz@addmode{\pgfsyssoftpath@getcurrentpath#1\global\let#1=#1}}

\let\tikz@fillcolor=\@empty
\let\tikz@strokecolor=\@empty


% Shading options
\define@key{tikz}{shading}{\def\tikz@shading{#1}\tikz@addmode{\tikz@mode@shadetrue}}
\define@key{tikz}{shading angle}{\def\tikz@shade@angle{#1}\tikz@addmode{\tikz@mode@shadetrue}}
\define@key{tikz}{top color}{%
  \colorlet{tikz@axis@top}{#1}%
  \colorlet{tikz@axis@middle}{tikz@axis@top!50!tikz@axis@bottom}%
  \def\tikz@shading{axis}\def\tikz@shade@angle{0}\tikz@addmode{\tikz@mode@shadetrue}}
\define@key{tikz}{bottom color}{%
  \colorlet{tikz@axis@bottom}{#1}%
  \colorlet{tikz@axis@middle}{tikz@axis@top!50!tikz@axis@bottom}%
  \def\tikz@shading{axis}\def\tikz@shade@angle{0}\tikz@addmode{\tikz@mode@shadetrue}}
\define@key{tikz}{middle color}{%
  \colorlet{tikz@axis@middle}{#1}%
  \def\tikz@shading{axis}\tikz@addmode{\tikz@mode@shadetrue}}
\define@key{tikz}{left color}{%
  \colorlet{tikz@axis@top}{#1}%
  \colorlet{tikz@axis@middle}{tikz@axis@top!50!tikz@axis@bottom}%
  \def\tikz@shading{axis}\def\tikz@shade@angle{90}\tikz@addmode{\tikz@mode@shadetrue}}
\define@key{tikz}{right color}{%
  \colorlet{tikz@axis@bottom}{#1}%
  \colorlet{tikz@axis@middle}{tikz@axis@top!50!tikz@axis@bottom}%
  \def\tikz@shading{axis}\def\tikz@shade@angle{90}\tikz@addmode{\tikz@mode@shadetrue}}
\define@key{tikz}{ball color}{\colorlet{tikz@ball}{#1}\def\tikz@shading{ball}\tikz@addmode{\tikz@mode@shadetrue}}
\define@key{tikz}{inner color}{\colorlet{tikz@radial@inner}{#1}\def\tikz@shading{radial}\tikz@addmode{\tikz@mode@shadetrue}}
\define@key{tikz}{outer color}{\colorlet{tikz@radial@outer}{#1}\def\tikz@shading{radial}\tikz@addmode{\tikz@mode@shadetrue}}

\def\tikz@shading{axis}
\def\tikz@shade@angle{0}

\pgfdeclareverticalshading[tikz@axis@top,tikz@axis@middle,tikz@axis@bottom]{axis}{100bp}{%
  color(0bp)=(tikz@axis@bottom);
  color(25bp)=(tikz@axis@bottom);
  color(50bp)=(tikz@axis@middle);
  color(75bp)=(tikz@axis@top);
  color(100bp)=(tikz@axis@top)}

\colorlet{tikz@axis@top}{gray}
\colorlet{tikz@axis@middle}{gray!50!white}
\colorlet{tikz@axis@bottom}{white}

\pgfdeclareradialshading[tikz@ball]{ball}{\pgfpoint{-10bp}{10bp}}{%
 color(0bp)=(tikz@ball!15!white);
 color(9bp)=(tikz@ball!75!white);
 color(18bp)=(tikz@ball!70!black);
 color(25bp)=(tikz@ball!50!black);
 color(50bp)=(black)}

\colorlet{tikz@ball}{blue}

\pgfdeclareradialshading[tikz@radial@inner,tikz@radial@outer]{radial}{\pgfpointorigin}{%
 color(0bp)=(tikz@radial@inner);
 color(25bp)=(tikz@radial@outer);
 color(50bp)=(tikz@radial@outer)}

\colorlet{tikz@radial@inner}{gray}
\colorlet{tikz@radial@outer}{white}





% General shape options
\define@key{tikz}{name}{\edef\tikz@fig@name{#1}}

\define@key{tikz}{shape}{\edef\tikz@shape{#1}}

\define@key{tikz}{inner sep}{\def\pgfshapeinnerxsep{#1}\def\pgfshapeinnerysep{#1}}
\define@key{tikz}{inner xsep}{\def\pgfshapeinnerxsep{#1}}
\define@key{tikz}{inner ysep}{\def\pgfshapeinnerysep{#1}}

\define@key{tikz}{outer sep}{\def\pgfshapeouterxsep{#1}\def\pgfshapeouterysep{#1}}
\define@key{tikz}{outer xsep}{\def\pgfshapeouterxsep{#1}}
\define@key{tikz}{outer ysep}{\def\pgfshapeouterysep{#1}}

\define@key{tikz}{minimum width}{\def\pgfshapeminwidth{#1}}
\define@key{tikz}{minimum height}{\def\pgfshapeminheight{#1}}
\define@key{tikz}{minimum size}{\def\pgfshapeminwidth{#1}\def\pgfshapeminheight{#1}}

\define@key{tikz}{anchor}{\def\tikz@anchor{#1}}

\define@key{tikz}{left}[]{\def\tikz@anchor{east}\tikz@possibly@transform{x}{-}{#1}}
\define@key{tikz}{right}[]{\def\tikz@anchor{west}\tikz@possibly@transform{x}{}{#1}}
\define@key{tikz}{above}[]{\def\tikz@anchor{south}\tikz@possibly@transform{y}{}{#1}}
\define@key{tikz}{below}[]{\def\tikz@anchor{north}\tikz@possibly@transform{y}{-}{#1}}
\define@key{tikz}{above left}[]%
  {\def\tikz@anchor{south east}%
    \tikz@possibly@transform{x}{-}{#1}\tikz@possibly@transform{y}{}{#1}}
\define@key{tikz}{above right}[]%
  {\def\tikz@anchor{south west}%
    \tikz@possibly@transform{x}{}{#1}\tikz@possibly@transform{y}{}{#1}}
\define@key{tikz}{below left}[]%
  {\def\tikz@anchor{north east}%
    \tikz@possibly@transform{x}{-}{#1}\tikz@possibly@transform{y}{-}{#1}}
\define@key{tikz}{below right}[]%
  {\def\tikz@anchor{north west}%
    \tikz@possibly@transform{x}{}{#1}\tikz@possibly@transform{y}{-}{#1}}

\define@key{tikz}{transform shape}[true]{%
  \csname tikz@fullytransformed#1\endcsname%
  \iftikz@fullytransformed%
    \pgfresetnontranslationattimefalse%
  \else%
    \pgfresetnontranslationattimetrue%
  \fi%
}

\newif\iftikz@fullytransformed
\pgfresetnontranslationattimetrue%

\def\tikz@anchor{center}%
\def\tikz@shape{rectangle}%

\def\tikz@possibly@transform#1#2#3{%
  \def\tikz@test{#3}%
  \ifx\tikz@test\@empty%
  \else%
    \setlength{\pgf@x}{#3}%
    \pgf@x=#2\pgf@x\relax%
    \edef\tikz@marshal{\noexpand\tikz@addtransform{%
        \expandafter\noexpand\csname  pgftransform#1shift\endcsname{\the\pgf@x}}}% 
    \tikz@marshal%
  \fi%
}


% Line/curve label placement options
\define@key{tikz}{sloped}[true]{\csname pgfslopedattime#1\endcsname}

\define@key{tikz}{pos}{\edef\tikz@time{#1}}

\def\tikz@time{.5}



% Text options
\define@key{tikz}{text}{\def\tikz@textcolor{#1}}
\define@key{tikz}{font}{\def\tikz@textfont{#1}}
\define@key{tikz}{text opacity}{\def\tikz@textopacity{#1}}
\define@key{tikz}{text width}{\def\tikz@text@width{#1}}
\define@key{tikz}{text ragged}[]%
{\def\tikz@text@action{\raggedright\rightskip\z@ plus2em \spaceskip.3333em \xspaceskip.5em\relax}}
\define@key{tikz}{text badly ragged}[]{\def\tikz@text@action{\raggedright\relax}}
\define@key{tikz}{text ragged left}[]%
{\def\tikz@text@action{\raggedleft\leftskip\z@ plus2em \spaceskip.3333em \xspaceskip.5em\relax}}
\define@key{tikz}{text badly ragged left}[]{\def\tikz@text@action{\raggedleft\relax}}
\define@key{tikz}{text justified}[]{\def\tikz@text@action{\leftskip\z@\rightskip\z@\relax}}
\define@key{tikz}{text centered}[]{\def\tikz@text@action{%
  \leftskip\z@ plus2em%
  \rightskip\z@ plus2em%
  \spaceskip.3333em \xspaceskip.5em%
  \parfillskip=0pt%
  \let\\=\@centercr% for latex
  \relax}}
\define@key{tikz}{text badly centered}[]%
{\def\tikz@text@action{%
  \let\\=\@centercr% for latex
  \parfillskip=0pt%
  \rightskip\@flushglue%
  \leftskip\@flushglue\relax}}

\let\tikz@text@width=\@empty
\let\tikz@textcolor=\@empty
\let\tikz@textfont=\@empty
\let\tikz@textopacity=\@empty

\def\tikz@text@action{\raggedright\rightskip\z@ plus2em \spaceskip.3333em \xspaceskip.5em\relax}





% Arrow options
\define@key{tikz}{arrows}{\tikz@processarrows{#1}}

\define@key{tikz}{>}{%
  \tikz@set@pointed{\csname pgf@arrows@invert#1\endcsname}{#1}%
  \expandafter\tikz@processarrows\expandafter{\tikz@current@arrows}%
}

\define@key{tikz}{shorten <}{\pgfsetshortenstart{#1}}
\define@key{tikz}{shorten >}{\pgfsetshortenend{#1}}

\def\tikz@set@pointed#1#2{%
  \@ifundefined{pgf@arrow@code@tikze@>@#2}
  {%
    \pgfarrowsdeclarealias{tikzs@<@#2}{tikze@>@#2}{#1}{#2}%
    \pgfarrowsdeclarereversed{tikzs@>@#2}{tikze@<@#2}{#1}{#2}%
    \pgfarrowsdeclarecombine*{tikz@|<@#2}{tikz@>|@#2}{#1}{#2}{|}{|}%
    \pgfarrowsdeclaredouble[\pgflinewidth]{tikzs@<<@#2}{tikze@>>@#2}{#1}{#2}%
    \pgfarrowsdeclarereversed{tikzs@>>@#2}{tikze@<<@#2}{tikzs@<<@#2}{tikze@>>@#2}%
  }{}%
  \@namedef{tikz@special@arrow@start<}{tikzs@<@#2}%
  \@namedef{tikz@special@arrow@end>}{tikze@>@#2}%
  \@namedef{tikz@special@arrow@start>}{tikzs@>@#2}%
  \@namedef{tikz@special@arrow@end<}{tikze@<@#2}%
  \@namedef{tikz@special@arrow@start|<}{tikz@|<@#2}%
  \@namedef{tikz@special@arrow@end>|}{tikz@>|@#2}%
  \@namedef{tikz@special@arrow@start<<}{tikzs@<<@#2}%
  \@namedef{tikz@special@arrow@end>>}{tikze@>>@#2}%
  \@namedef{tikz@special@arrow@start>>}{tikzs@<<@#2}%
  \@namedef{tikz@special@arrow@end<<}{tikze@>>@#2}%
}

\def\tikz@processarrows#1{%
  \def\tikz@current@arrows{#1}%
  \def\tikz@temp{#1}%
  \ifx\tikz@temp\@empty%
  \else%
    \tikz@@processarrows#1\@nil
  \fi%
}
\def\tikz@@processarrows#1-#2\@nil{%
  \expandafter\ifx\csname tikz@special@arrow@start#1\endcsname\relax%
    \pgfsetarrowsstart{#1}
  \else%
    \pgfsetarrowsstart{\csname tikz@special@arrow@start#1\endcsname}%
  \fi%
  \expandafter\ifx\csname tikz@special@arrow@end#2\endcsname\relax%
    \pgfsetarrowsend{#2}
  \else%
    \pgfsetarrowsend{\csname tikz@special@arrow@end#2\endcsname}%
  \fi%
}

\tikz@set@pointed{\pgf@arrows@invertto}{to}
\def\tikz@current@arrows{-}

% Parabola options
\define@key{tikz}{bend}{\tikz@scan@one@point\tikz@set@parabola@bend#1\relax}%
\define@key{tikz}{bend pos}{\def\tikz@parabola@bend@factor{#1}}
\define@key{tikz}{parabola height}{%
  \def\tikz@parabola@bend@factor{.5}%
  \def\tikz@parabola@bend{\pgfpointadd{\pgfpoint{0pt}{#1}}{\tikz@last@position@saved}}}

\def\tikz@parabola@bend{\tikz@last@position@saved}
\def\tikz@parabola@bend@factor{0}

\def\tikz@set@parabola@bend#1{\def\tikz@parabola@bend{#1}}

% Axis options
\define@key{tikz}{domain}{\def\tikz@plot@domain{#1}}
\define@key{tikz}{range}{\def\tikz@plot@range{#1}}

% Plot options
\define@key{tikz}{smooth}[]{\let\tikz@plot@handler=\pgfplothandlercurveto}
\define@key{tikz}{smooth cycle}[]{\let\tikz@plot@handler=\pgfplothandlerclosedcurve}
\define@key{tikz}{sharp plot}[]{\let\tikz@plot@handler\pgfplothandlerlineto}

\define@key{tikz}{tension}{\pgfsetplottension{#1}}

\define@key{tikz}{xcomb}[]{\let\tikz@plot@handler=\pgfplothandlerxcomb}
\define@key{tikz}{ycomb}[]{\let\tikz@plot@handler=\pgfplothandlerycomb}
\define@key{tikz}{polar comb}[]{\let\tikz@plot@handler=\pgfplothandlerpolarcomb}

\define@key{tikz}{raw gnuplot}[true]{\csname tikz@plot@raw@gnuplot#1\endcsname}
\define@key{tikz}{prefix}{\def\tikz@plot@prefix{#1}}
\define@key{tikz}{id}{\def\tikz@plot@id{#1}}

\define@key{tikz}{samples}{\def\tikz@plot@sampels{#1}}
\define@key{tikz}{parametric}[true]{\csname tikz@plot@parametric#1\endcsname}

\define@key{tikz}{only marks}[]{\let\tikz@plot@handler\pgfplothandlerdiscard}

\define@key{tikz}{mark}{\def\tikz@plot@mark{#1}}
\define@key{tikz}{mark options}{\def\tikz@plot@mark@options{#1}}
\define@key{tikz}{mark size}{\pgfsetplotmarksize{#1}}

\let\tikz@plot@mark@options=\@empty

\let\tikz@plot@handler=\pgfplothandlerlineto
\let\tikz@plot@mark=\@empty

\def\tikz@plot@sampels{25}
\def\tikz@plot@domain{-5:5}

\def\tikz@plot@prefix{\jobname.}
\def\tikz@plot@id{pgf-plot}

\newif\iftikz@plot@parametric
\newif\iftikz@plot@raw@gnuplot


% Tree options
\define@key{tikz}{edge from parent path}{\def\tikz@edge@to@parent@path{#1}}

\define@key{tikz}{parent anchor}{\def\tikzparentanchor{.#1}\ifx\tikzparentanchor\tikz@border@text\let\tikzparentanchor\@empty\fi}
\define@key{tikz}{child anchor}{\def\tikzchildanchor{.#1}\ifx\tikzchildanchor\tikz@border@text\let\tikzchildanchor\@empty\fi}

\define@key{tikz}{level distance}{\setlength\tikzleveldistance{#1}}
\define@key{tikz}{sibling distance}{\setlength\tikzsiblingdistance{#1}}

\define@key{tikz}{growth function}{\let\tikz@grow=#1}
\define@key{tikz}{grow}{\tikz@set@growth{#1}\edef\tikz@special@level{\the\tikztreelevel}}%
\define@key{tikz}{grow'}{\tikz@set@growth{#1}\tikz@swap@growth\edef\tikz@special@level{\the\tikztreelevel}}%


\def\tikz@special@level{-1}% never

\def\tikz@swap@growth{%
  % Swap left and right
  \let\tikz@temp=\tikz@angle@grow@right%
  \let\tikz@angle@grow@right=\tikz@angle@grow@left%
  \let\tikz@angle@grow@left=\tikz@temp%
}%

\def\tikz@set@growth#1{%
  \let\tikz@grow=\tikz@grow@direction%
  \expandafter\ifx\csname tikz@grow@direction@#1\endcsname\relax%
    \c@pgf@counta=#1\relax%
  \else%
    \c@pgf@counta=\csname tikz@grow@direction@#1\endcsname%
  \fi%
  \edef\tikz@angle@grow{\the\c@pgf@counta}%
  \advance\c@pgf@counta by-90\relax%
  \edef\tikz@angle@grow@left{\the\c@pgf@counta}%
  \advance\c@pgf@counta by180\relax%
  \edef\tikz@angle@grow@right{\the\c@pgf@counta}%
}

\def\tikz@border@text{.border}
\let\tikzparentanchor=\@empty
\let\tikzchildanchor=\@empty
\def\tikz@edge@to@parent@path{(\tikzparentnode\tikzparentanchor) -- (\tikzchildnode\tikzchildanchor)}

\tikzleveldistance=15mm
\tikzsiblingdistance=15mm

\def\tikz@grow@direction@down{-90}
\def\tikz@grow@direction@up{90}
\def\tikz@grow@direction@left{180}
\def\tikz@grow@direction@right{0}

\def\tikz@grow@direction@south{-90}
\def\tikz@grow@direction@north{90}
\def\tikz@grow@direction@west{180}
\def\tikz@grow@direction@east{0}

\expandafter\def\csname tikz@grow@direction@north east\endcsname{45}
\expandafter\def\csname tikz@grow@direction@north west\endcsname{135}
\expandafter\def\csname tikz@grow@direction@south east\endcsname{-45}
\expandafter\def\csname tikz@grow@direction@south west\endcsname{-135}

\def\tikz@grow@direction{%
  \pgftransformshift{\pgfpointpolar{\tikz@angle@grow}{\tikzleveldistance}}%
  \ifnum\tikztreelevel=\tikz@special@level%
  \else%
    \pgf@xc=.5\tikzsiblingdistance%
    \c@pgf@counta=\tikznumberofchildren%
    \advance\c@pgf@counta by1\relax%
    \@tempdima=\c@pgf@counta\pgf@xc%
    \pgftransformshift{\pgfpointpolar{\tikz@angle@grow@left}{\@tempdima}}%
    \pgftransformshift{\pgfpointpolar{\tikz@angle@grow@right}{\tikznumberofcurrentchild\tikzsiblingdistance}}%
  \fi%
}

\setkeys{tikz}{grow=down}




% Snake options
\define@key{tikz}{snake}[]{%
  \def\tikz@@snake{#1}%
  \ifx\tikz@@snake\@empty%
    \tikz@snakedtrue%
  \else%
    \ifx\tikz@@snake\tikz@nonetext%
      \tikz@snakedfalse%
    \else%
      \tikz@snakedtrue%
      \let\tikz@snake=\tikz@@snake%
    \fi%
  \fi}

\define@key{tikz}{segment amplitude}{\setlength{\pgfsnakesegmentamplitude}{#1}}
\define@key{tikz}{segment length}{\setlength{\pgfsnakesegmentlength}{#1}}
\define@key{tikz}{segment angle}{\def\pgfsnakesegmentangle{#1}}
\define@key{tikz}{segment aspect}{\def\pgfsnakesegmentaspect{#1}}

\define@key{tikz}{segment object length}{\def\pgfsnakesegmentobjectlength{#1}}

\define@key{tikz}{raise snake}{\def\pgf@snake@raise{\pgftransformyshift{#1}}}
\define@key{tikz}{mirror snake}[true]{%
  \csname if#1\endcsname
    \def\pgf@snake@mirror{\pgftransformyscale{-1}}%
  \else%
    \let\pgf@snake@mirror=\@empty%
  \fi
}

\define@key{tikz}{gap before snake}{\def\tikz@presnake{{moveto}{#1}}}
\define@key{tikz}{line before snake}{\def\tikz@presnake{{lineto}{#1}}}

\define@key{tikz}{gap after snake}{\def\tikz@postsnake{{moveto}{#1}}\def\tikz@mainsnakelength{\pgfsnakeremainingdistance-#1}}
\define@key{tikz}{line after snake}{\def\tikz@postsnake{{lineto}{#1}}\def\tikz@mainsnakelength{\pgfsnakeremainingdistance-#1}}

\define@key{tikz}{gap around snake}{%
  \def\tikz@presnake{{moveto}{#1}}%
  \def\tikz@postsnake{{moveto}{#1}}%
  \def\tikz@mainsnakelength{\pgfsnakeremainingdistance-#1}%
}
\define@key{tikz}{line around snake}{%
  \def\tikz@presnake{{lineto}{#1}}%
  \def\tikz@postsnake{{lineto}{#1}}%
  \def\tikz@mainsnakelength{\pgfsnakeremainingdistance-#1}%
}
\let\pgf@snake@mirror=\@empty
\let\pgf@snake@raise=\@empty

\pgfsetsnakesegmenttransformation{\pgf@snake@mirror\pgf@snake@raise}

\def\tikz@snake{zigzag}

\let\tikz@presnake=\@empty
\let\tikz@postsnake=\@empty
\def\tikz@mainsnakelength{\pgfsnakeremainingdistance}



% Execute option

\define@key{tikz}{execute at begin picture}{\expandafter\def\expandafter\tikz@atbegin@picture\expandafter{\tikz@atbegin@picture#1}}
\define@key{tikz}{execute at end picture}{\expandafter\def\expandafter\tikz@atend@picture\expandafter{\tikz@atend@picture#1}}
\define@key{tikz}{execute at begin scope}{\expandafter\def\expandafter\tikz@atbegin@scope\expandafter{\tikz@atbegin@scope#1}}
\define@key{tikz}{execute at end scope}{\expandafter\def\expandafter\tikz@atend@scope\expandafter{\tikz@atend@scope#1}}

\let\tikz@atbegin@picture=\@empty
\let\tikz@atend@picture=\@empty
\let\tikz@atbegin@scope=\@empty
\let\tikz@atend@scope=\@empty




% Styles
\define@key{tikz}{set style}{\tikzstyle#1}

% Handled in a special way.
\def\tikzstyle#1=#2[#3]{% #2 is dummy
  \in@+{#1}%
  \ifin@%
    \tikz@style#1{#3}%
  \else%
    \expandafter\def\csname tikz@st@#1\endcsname{#3}%
  \fi}
\def\tikz@style#1+#2{%
  \edef\tikz@marshal{\def\expandafter\noexpand\csname tikz@st@#1\endcsname}%
  \expandafter\expandafter\expandafter\tikz@marshal\expandafter\expandafter\expandafter{\csname tikz@st@#1\endcsname,#2}}

\def\iftikzstyleempty#1#2#3{%
  \expandafter\ifx\csname tikz@st@#1\endcsname\@empty%
    \let\pgf@next=\@firstoftwo%
  \else%
    \expandafter\ifx\csname tikz@st@#1\endcsname\relax%
      \let\pgf@next=\@firstoftwo%
    \else
      \let\pgf@next=\@secondoftwo%
    \fi%
  \fi%
  \pgf@next{#2}{#3}}



%
%
% Predefined styles
%
%

\tikzstyle{help lines}=              [color=gray,line width=0.2pt]

\tikzstyle{every picture}=           []
\tikzstyle{every path}=              []
\tikzstyle{every scope}=             []
\tikzstyle{every plot}=              []
\tikzstyle{every node}=              []
\tikzstyle{every child}=             []
\tikzstyle{every child node}=        []

\tikzstyle{ultra thin}=              [line width=0.1pt]
\tikzstyle{very thin}=               [line width=0.2pt]
\tikzstyle{thin}=                    [line width=0.4pt]
\tikzstyle{semithick}=               [line width=0.6pt]
\tikzstyle{thick}=                   [line width=0.8pt]
\tikzstyle{very thick}=              [line width=1.2pt]
\tikzstyle{ultra thick}=             [line width=1.6pt]

\tikzstyle{solid}=                   [dash pattern=]
\tikzstyle{dotted}=                  [dash pattern=on \pgflinewidth off 2pt]
\tikzstyle{densely dotted}=          [dash pattern=on \pgflinewidth off 1pt]
\tikzstyle{loosely dotted}=          [dash pattern=on \pgflinewidth off 4pt]
\tikzstyle{dashed}=                  [dash pattern=on 3pt off 3pt]
\tikzstyle{densely dashed}=          [dash pattern=on 3pt off 2pt]
\tikzstyle{loosely dashed}=          [dash pattern=on 3pt off 6pt]

\tikzstyle{transparent}=             [opacity=0]
\tikzstyle{ultra nearly transparent}=[opacity=0.05]
\tikzstyle{very nearly transparent}= [opacity=0.1]
\tikzstyle{nearly transparent}=      [opacity=0.25]
\tikzstyle{semitransparent}=         [opacity=0.5]
\tikzstyle{nearly opaque}=           [opacity=0.75]
\tikzstyle{very nearly opaque}=      [opacity=0.9]
\tikzstyle{ultra nearly opaque}=     [opacity=0.95]
\tikzstyle{opaque}=                  [opacity=1]

\tikzstyle{at start}=                [pos=0]
\tikzstyle{very near start}=         [pos=0.125]
\tikzstyle{near start}=              [pos=0.25]
\tikzstyle{midway}=                  [pos=0.5]
\tikzstyle{near end}=                [pos=0.75]
\tikzstyle{very near end}=           [pos=0.875]
\tikzstyle{at end}=                  [pos=1]

\tikzstyle{bend at start}=           [bend pos=0,bend={+(0,0)}]
\tikzstyle{bend at end}=             [bend pos=1,bend={+(0,0)}]

\tikzstyle{edge from parent}=        [draw]

\tikzstyle{snake triangles 45}=      [snake=triangles,segment object length=2.41421356\pgfsnakesegmentamplitude]
\tikzstyle{snake triangles 60}=      [snake=triangles,segment object length=1.73205081\pgfsnakesegmentamplitude]
\tikzstyle{snake triangles 90}=      [snake=triangles,segment object length=\pgfsnakesegmentamplitude]


%
% Setting keys
%

\let\tikz@late@keys=\@empty%

\def\tikz@set@one@key#1{%
  \setkeys*{tikz}{#1}%
  \ifx\XKV@rm\@empty%
    % fine
  \else%
    \expandafter\in@\expandafter!\expandafter{\XKV@rm}%
    \ifin@%
      % this is a color!
      \expandafter\tikz@addoption\expandafter{\expandafter\color\expandafter{\XKV@rm}}%
      \edef\tikz@textcolor{\XKV@rm}%
    \else%
      \expandafter\ifx\csname\string\color@\XKV@rm\endcsname\relax%
        % Ok, second chance: This might be an arrow specification:
        \expandafter\in@\expandafter-\expandafter{\XKV@rm}
        \ifin@%
          % Ah, an arrow spec!
          \expandafter\tikz@processarrows\expandafter{\XKV@rm}%
        \else%
          % Ok, third chance: A shape!
          \expandafter\ifx\csname pgf@sh@s@\XKV@rm\endcsname\relax%
            \PackageError{tikz}{I do not know what to do with the option ``\XKV@rm''}{}
          \else%
            \edef\tikz@shape{\XKV@rm}%
          \fi%
        \fi%
      \else%
        \expandafter\tikz@addoption\expandafter{\expandafter\color\expandafter{\XKV@rm}}%
        \edef\tikz@textcolor{\XKV@rm}%
      \fi%
    \fi%
  \fi%  
}

\def\tikz@setkeys#1{\tikz@@setkeys#1,\pgf@stop}

\def\tikz@@setkeys#1,#2\pgf@stop{%
  \def\tikz@key@test{#1}%
  \def\tikz@key@rest{#2}%
  \ifx\tikz@key@test\@empty%
  \else%
    \@ifnextchar s{\tikz@parse@key}{\tikz@parse@key}#1==\pgf@stop%
  \fi%
  \ifx\tikz@key@rest\@empty%
  \else%
    \expandafter\expandafter\expandafter\tikz@@setkeys\expandafter\tikz@key@rest\expandafter\pgf@stop%
  \fi%
}

\def\tikz@style@text{style}

\def\tikz@parse@key#1=#2=#3\pgf@stop{
  \def\tikz@key@test{#1}%
  \ifx\tikz@key@test\tikz@style@text%
    % Ok, style!
    \@ifundefined{tikz@st@#2}%
    {\PackageError{tikz}{Unknown style ``#2}''{}}
    {%
      \expandafter\let\expandafter\tikz@temp\expandafter=\csname tikz@st@#2\endcsname%
      \expandafter\expandafter\expandafter\def
      \expandafter\expandafter\expandafter\tikz@key@rest
      \expandafter\expandafter\expandafter{\expandafter\tikz@temp\expandafter,\tikz@key@rest}%
    }%
  \else%
    \expandafter\ifx\csname tikz@st@#1\endcsname\relax%
      % Ok, normal!
      \def\tikz@test{#3}%
      \ifx\tikz@test\@empty%
        \tikz@set@one@key{#1}%
      \else%
        \tikz@set@one@key{#1={#2}}%
      \fi%
    \else%
      % Ok, style!
      \expandafter\let\expandafter\tikz@temp\expandafter=\csname tikz@st@#1\endcsname%
      \expandafter\expandafter\expandafter\def
      \expandafter\expandafter\expandafter\tikz@key@rest
      \expandafter\expandafter\expandafter{\expandafter\tikz@temp\expandafter,\tikz@key@rest}%
    \fi%
  \fi%  
}

\def\tikz@every@style#1{%
  \expandafter\ifx\csname tikz@st@#1\endcsname\relax%
  \else%
    \expandafter\expandafter\expandafter\tikz@setkeys
    \expandafter\expandafter\expandafter{\csname tikz@st@#1\endcsname}%
  \fi%
}





%
% Main TikZ Environment
%

\def\tikzpicture{\@ifnextchar[\tikz@picture{\tikz@picture[]}}%}
\def\tikz@picture[#1]{%
  \pgfpicture%
  \let\tikz@atbegin@picture=\@empty%
  \let\tikz@atend@picture=\@empty%
  \tikz@installcommands\scope[style=every picture,#1]%
  \tikz@atbegin@picture%
}
\def\endtikzpicture{%
    \tikz@atend@picture%
    \global\let\pgf@shift@baseline=\pgf@baseline%
    \endscope%
    \let\pgf@baseline=\pgf@shift@baseline%
  \endpgfpicture}

  

% Inlined picture
%
% #1 - some code to be put in a tikzpicture environment.
%
% If the command is not followed by braces, everything up to the next
% semicolon is used as argument.
%
% Example:
%
% The rectangle \tikz{\draw (0,0) rectangle (1em,1ex)} has width 1em and
% height 1ex.

\def\tikz{\@ifnextchar[{\tikz@opt}{\tikz@opt[]}}
\def\tikz@opt[#1]{\@ifnextchar\bgroup{\tikz@[#1]}{\tikz@@[#1]}}
\def\tikz@[#1]#2{\tikzpicture[#1]#2\endtikzpicture}
\def\tikz@@{%
  \let\tikz@next=\tikz@collectnormalsemicolon%
  \ifnum\the\catcode`\;=\active\relax%
    \let\tikz@next=\tikz@collectactivesemicolon%
  \fi%
  \tikz@next}
\def\tikz@collectnormalsemicolon[#1]#2;{\tikzpicture[#1]#2;\endtikzpicture}
{
  \catcode`\;=\active
  \gdef\tikz@collectactivesemicolon[#1]#2;{\tikzpicture[#1]#2;\endtikzpicture}
}



%
% Environment for scoping graphic state settings
%
\def\tikz@scope@env{\@ifnextchar[\tikz@@scope@env{\tikz@@scope@env[]}}
\def\tikz@@scope@env[#1]{%
  \pgfscope%
  \begingroup%
  \let\tikz@atbegin@scope=\@empty%
  \let\tikz@atend@scope=\@empty%
  \let\tikz@options=\@empty%
  \let\tikz@mode=\@empty%
  \tikz@every@style{every scope}%
  \tikz@setkeys{#1}%
  \tikz@options%
  \tikz@atbegin@scope%
}
\def\endtikz@scope@env{%
  \tikz@atend@scope%
  \endgroup%
  \endpgfscope%
}


%
% Install the abbreviated commands
%
\def\tikz@installcommands{%
  \ifnum\the\catcode`\;=\active\relax\expandafter\let\expandafter\tikz@origsemi\expandafter=\tikz@activesemicolon\fi%
  \ifnum\the\catcode`\:=\active\relax\expandafter\let\expandafter\tikz@origcolon\expandafter=\tikz@activecolon\fi%
  \ifnum\the\catcode`\|=\active\relax\expandafter\let\expandafter\tikz@origbar\expandafter=\tikz@activebar\fi%
  \let\tikz@origscope=\scope%
  \let\tikz@origendscope=\endscope%
  \let\tikz@origstartscope=\startscope%
  \let\tikz@origstopscope=\stopscope%
  \let\tikz@origpath=\path%
  \let\tikz@origagainpath=\againpath%
  \let\tikz@origdraw=\draw%
  \let\tikz@origfill=\fill%
  \let\tikz@origfilldraw=\filldraw%
  \let\tikz@origshade=\shade%
  \let\tikz@origshadedraw=\shadedraw%
  \let\tikz@origclip=\clip%
  \let\tikz@origuseasboundingbox=\useasboundingbox%
  \let\tikz@orignode=\node%
  \let\tikz@orignode=\coordiante%
  %
  \tikz@deactivatthings%
  %
  \let\scope=\tikz@scope@env%
  \let\endscope=\endtikz@scope@env%
  \let\startscope=\scope%
  \let\stopscope=\endscope%
  \let\path=\tikz@command@path%
  \let\againpath=\tikz@command@againpath%
  %
  \def\draw{\path[draw]}
  \def\fill{\path[fill]}
  \def\filldraw{\path[fill,draw]}
  \def\shade{\path[shade]}
  \def\shadedraw{\path[shade,draw]}
  \def\clip{\path[clip]}
  \def\useasboundingbox{\path[use as bounding box]}
  \def\node{\path node}
  \def\coordinate{\path coordinate}
}

\def\tikz@uninstallcommands{%
  \ifnum\the\catcode`\;=\active\relax\expandafter\let\tikz@activesemicolon=\tikz@origsemi\fi%
  \ifnum\the\catcode`\:=\active\relax\expandafter\let\tikz@activecolon=\tikz@origcolon\fi%
  \ifnum\the\catcode`\|=\active\relax\expandafter\let\tikz@activebar=\tikz@origbar\fi%
  \let\scope=\tikz@origscope%
  \let\endscope=\tikz@origendscope%
  \let\startscope=\tikz@origstartscope%
  \let\stopscope=\tikz@origstopscope%
  \let\path=\tikz@origpath%
  \let\againpath=\tikz@origagainpath%
  \let\draw=\tikz@origdraw%
  \let\fill=\tikz@origfill%
  \let\filldraw=\tikz@origfilldraw%
  \let\shade=\tikz@origshade%
  \let\shadedraw=\tikz@origshadedraw%
  \let\clip=\tikz@origclip%
  \let\useasboundingbox=\tikz@origuseasboundingbox%
  \let\node=\tikz@orignode%
}


{
  \catcode`\;=12
  \gdef\tikz@nonactivesemicolon{;}
  \catcode`\:=12
  \gdef\tikz@nonactivecolon{:}
  \catcode`\|=12
  \gdef\tikz@nonactivebar{|}
  \catcode`\;=\active
  \catcode`\:=\active
  \catcode`\|=\active
  \catcode`\"=\active
  \gdef\tikz@activesemicolon{;}%
  \gdef\tikz@activecolon{:}%
  \gdef\tikz@activebar{|}%
  \gdef\tikz@activequotes{"}%
  \gdef\tikz@deactivatthings{%
    \def;{\tikz@nonactivesemicolon}
    \def:{\tikz@nonactivecolon}
    \def|{\tikz@nonactivebar}
  }
}





% Constructs a path and draws/fills them according to the current
% settings.  

\def\tikz@command@path{%
  \@ifnextchar[{\tikz@check@earg}%]
  {\@ifnextchar<{\tikz@doopt}{\tikz@@command@path}}}
\def\tikz@check@earg[#1]{%
  \@ifnextchar<{\tikz@swap@args[#1]}{\tikz@@command@path[#1]}}
\def\tikz@swap@args[#1]<#2>{\tikz@command@path<#2>[#1]}

\def\tikz@doopt{%
  \let\tikz@next=\tikz@eargnormalsemicolon%
  \ifnum\the\catcode`\;=\active\relax%
    \let\tikz@next=\tikz@eargactivesemicolon%
  \fi%
  \tikz@next}
\long\def\tikz@eargnormalsemicolon<#1>#2;{\only<#1>{\tikz@@command@path#2;}}
{
  \catcode`\;=\active
  \long\global\def\tikz@eargactivesemicolon<#1>#2;{\only<#1>{\tikz@@command@path#2;}}
}

\def\tikz@@command@path{%
  \edef\tikzscope@linewidth{\the\pgflinewidth}%
  \begingroup%
    \let\tikz@options=\@empty%
    \let\tikz@mode=\@empty%
    \let\tikz@moveto@waiting=\relax%
    \let\tikz@timer=\relax%
    \let\tikz@collected@onpath=\@empty%
    \tikz@snakedfalse%
    \tikz@node@is@a@labelfalse%
    \tikz@expandcount=1000\relax%
    \tikz@lastx=0pt%
    \tikz@lasty=0pt%
    \tikz@lastxsaved=0pt%
    \tikz@lastysaved=0pt%
    \tikz@every@style{every path}%
    \tikz@scan@next@command%
}
\def\tikz@scan@next@command{%
  \ifx\tikz@collected@onpath\@empty%
  \else%
    \tikz@invoke@collected@onpath%
  \fi%
  \afterassignment\tikz@handle\let\@let@token=%
}
\newcount\tikz@expandcount

% Central dispatcher for commands
\def\tikz@handle{%
  \let\@next=\tikz@expand%
    \ifx\@let@token(%)
      \let\@next=\tikz@movetoabs%
    \else%
      \ifx\@let@token+%
        \let\@next=\tikz@movetorel%
      \else%
        \ifx\@let@token-%
          \let\@next=\tikz@lineto%
        \else%
          \ifx\@let@token.%
            \let\@next=\tikz@dot%
          \else%
            \ifx\@let@token r%
              \let\@next=\tikz@rect%
            \else%
              \ifx\@let@token a%
                \let\@next=\tikz@arcA%
              \else%
                \ifx\@let@token[%]
                  \let\@next=\tikz@parse@options%
                \else%
                  \ifx\@let@token n%
                    \let\@next=\tikz@fig%
                  \else%
                    \ifx\@let@token\bgroup%
                      \let\@next=\tikz@beginscope%
                    \else%
                      \ifx\@let@token\egroup%
                        \let\@next=\tikz@endscope%
                      \else%
                        \ifx\@let@token;%
                          \let\@next=\tikz@finish%
                        \else%
                          \ifx\@let@token c%
                            \let\@next=\tikz@cchar%
                          \else%
                            \ifx\@let@token e%
                              \let\@next=\tikz@e@char%
                            \else%
                              \ifx\@let@token g%
                                \let\@next=\tikz@grid%
                              \else%
                                \ifx\@let@token s%
                                   \let\@next=\tikz@sine%
                                \else%
                                  \ifx\@let@token |%
                                     \let\@next=\tikz@vh@lineto%
                                  \else%
                                    \ifx\@let@token p%
                                      \let\@next=\tikz@pchar%
                                      \pgfsetmovetofirstplotpoint%
                                    \else%
                                      \ifx\@let@token\pgfextra%
                                        \let\@next=\tikz@extra%
                                      \else%
                                        \ifx\@let@token\foreach%
                                          \let\@next=\tikz@foreach%
                                        \else%
                                          \ifx\@let@token\pgf@stop%
                                            \let\@next=\relax%
                                          \else%
                                            \ifx\@let@token\par%
                                              \let\@next=\tikz@scan@next@command%
                                            \fi%      
                                          \fi%      
                                        \fi%      
                                      \fi%      
                                    \fi%  
                                  \fi%  
                                \fi%
                              \fi%
                            \fi%
                          \fi%
                        \fi%
                      \fi%  
                    \fi%  
                  \fi%  
                \fi%  
              \fi%
            \fi%
          \fi%
        \fi%
      \fi%
    \fi%
  \@next%
}

\def\tikz@pchar{\@ifnextchar l{\tikz@plot}{\tikz@parabola}}
\def\tikz@cchar{%
  \@ifnextchar i{\tikz@circle}%
  {\@ifnextchar h{\tikz@children}{\tikz@cochar}}}%
\def\tikz@cochar o{%
  \@ifnextchar o{\tikz@coordinate}{\tikz@cosine}}
\def\tikz@e@char{%
  \@ifnextchar l{\tikz@ellipse}{\tikz@edgetoparent}}


\def\tikz@finish{%
  \tikz@mode@fillfalse%
  \tikz@mode@drawfalse%
  \tikz@mode@doublefalse%
  \tikz@mode@clipfalse%
  \tikz@mode@boundaryfalse%
  \edef\tikz@pathextend{%
    {\noexpand\pgfpoint{\the\pgf@pathminx}{\the\pgf@pathminy}}%
    {\noexpand\pgfpoint{\the\pgf@pathmaxx}{\the\pgf@pathmaxy}}%
  }%
  \tikz@mode% installs the mode settings
  % Rendering pipeline:  
  % 
  % Step 1: Setup options
  % 
  \ifx\tikz@options\@empty%
  \else%
    \pgfsys@beginscope%
      \begingroup%
      \tikz@options%
  \fi%
  % 
  % Step 2: Do a fill if shade follows.
  %
  \iftikz@mode@fill%
    \iftikz@mode@shade%
      \pgfprocessround{\pgfsyssoftpath@currentpath}{\pgfsyssoftpath@currentpath}% change the current path
      \pgfsyssoftpath@invokecurrentpath%
      \pgfsys@fill%
      \tikz@mode@fillfalse% no more filling...
    \fi%
  \fi%
  % 
  % Step 3: Do a shade if necessary.
  %
  \iftikz@mode@shade%
    \pgfprocessround{\pgfsyssoftpath@currentpath}{\pgfsyssoftpath@currentpath}% change the current path
    \pgfshadepath{\tikz@shading}{\tikz@shade@angle}%
    \tikz@mode@shadefalse% no more shading...
  \fi%
  % 
  % Step 4: Double stroke, if necessary
  %
  \iftikz@mode@draw%
    \iftikz@mode@double%
      % Change line width
      \begingroup%
        \pgfsys@beginscope%
          \pgf@x=2\pgflinewidth%
          \advance\pgf@x by\tikz@double@width@distance%
          \pgflinewidth=\pgf@x%
          \pgfsetlinewidth{\the\pgflinewidth}%
    \fi%
  \fi%
  % 
  % Step 5: Do stroke/fill/clip as needed
  %
  \edef\tikz@temp{\noexpand\pgfusepath{%
    \iftikz@mode@fill fill,\fi%
    \iftikz@mode@draw draw,\fi%
    \iftikz@mode@clip clip,\fi%
    }}%
  \tikz@temp%
  \tikz@mode@fillfalse% no more filling
  % 
  % Step 6: Double stroke, if necessary
  %
  \iftikz@mode@draw%
    \iftikz@mode@double%
          \pgfsyssoftpath@setcurrentpath\pgf@last@used@path% reinstall
          \pgf@x=\tikz@double@width@distance%
          \pgfsetlinewidth{\the\pgf@x}%
          \pgfsetstrokecolor{\tikz@double@color}%
          \pgfsyssoftpath@flushcurrentpath%
          \pgfsys@stroke%
        \pgfsys@endscope%
        \pgf@add@arrows@as@needed
      \endgroup%
    \fi%
  \fi%
  \tikz@mode@drawfalse% no more stroking
  % 
  % Step 7: Add labels and nodes
  %
  \copy\tikz@figbox%
  \setbox\tikz@figbox=\box\voidb@x%
  %
  % Step 8: Close option brace
  %
  \ifx\tikz@options\@empty%
  \else%
      \endgroup%
    \pgfsys@endscope%
    \iftikz@mode@clip%
      \PackageError{tikz}{Extra options not allowed for clipping path command.}{}%
    \fi%
  \fi%
  \iftikz@mode@clip%
    \aftergroup\pgf@relevantforpicturesizefalse%
  \fi%
  \iftikz@mode@boundary%
    \aftergroup\pgf@relevantforpicturesizefalse%
  \fi%
  \endgroup%
  \global\pgflinewidth=\tikzscope@linewidth%
}




\def\tikz@skip#1{\tikz@scan@next@command#1}
\def\tikz@expand{%
  \advance\tikz@expandcount by -1%
  \ifnum\tikz@expandcount<0\relax%
    \PackageError{tikz}{Giving up on this path}{}%
    \let\@next=\tikz@finish%
  \else%
    \let\@next=\tikz@@expand
  \fi%
  \@next}

\def\tikz@@expand{%
  \expandafter\tikz@scan@next@command\@let@token}



% Syntax for scopes: 
% {scoped path commands}

\def\tikz@beginscope{\begingroup\tikz@scan@next@command}
\def\tikz@endscope{%
  \global\setbox\tikz@tempbox=\copy\tikz@figbox%
  \endgroup%
  \setbox\tikz@figbox=\box\tikz@tempbox%
  \tikz@scan@next@command}


% Syntax for pgfextra: 
% \pgfextra {normal tex text}

\long\def\tikz@extra#1{#1\tikz@scan@next@command}
\def\pgfextra{pgfextra}


% Syntax for \foreach: 
% \foreach \var in {list} {path text}
%
% Example:
%
% \draw (0,0) \foreach \x in {1,2,3} {-- (\x,0) circle (1cm)} -- (5,5);

\def\tikz@foreach{%
  \def\pgffor@beginhook{\setbox\tikz@figbox=\box\tikz@tempbox\expandafter\tikz@scan@next@command\@firstofone}%
  \def\pgffor@endhook{\pgfextra{\global\setbox\tikz@tempbox=\copy\tikz@figbox\@gobble}}%
  \def\pgffor@afterhook{\setbox\tikz@figbox=\box\tikz@tempbox\tikz@scan@next@command}%
  \global\setbox\tikz@tempbox=\copy\tikz@figbox%
  \foreach}

% Syntax for againpath: 
% \againpath \somepathname

\def\tikz@command@againpath#1{%
  \pgfextra{%
    \pgfsyssoftpath@getcurrentpath\tikz@temp%
    \expandafter\g@addto@macro\expandafter\tikz@temp\expandafter{#1}%
    \pgfsyssoftpath@setcurrentpath\tikz@temp%
  }
}




%
% When this if is set, a just-scanned point is a shape and its border
% position still needs to be determined, depending on subsequent
% commands. 
%

\newif\iftikz@shapeborder


% Syntax for moveto: 
% <point>
\def\tikz@movetoabs{\tikz@moveto(}
\def\tikz@movetorel{\tikz@moveto+}
\def\tikz@moveto{%
  \tikz@scan@one@point{\tikz@@moveto}}
\def\tikz@@moveto#1{%
  \tikz@make@last@position{#1}%
  \iftikz@shapeborder%
    % ok, the moveto will have to wait. flag that we have a moveto in
    % wainting:
    \edef\tikz@moveto@waiting{\tikz@shapeborder@name}%
  \else%
    \pgfpathmoveto{\tikz@last@position}%
    \let\tikz@moveto@waiting=\relax%
  \fi%
  \tikz@scan@next@command%
}

\let\tikz@moveto@waiting=\relax % normally, nothing is waiting...

\def\tikz@flush@moveto{%
  \ifx\tikz@moveto@waiting\relax%
  \else%
    \pgfpathmoveto{\tikz@last@position}%
  \fi%
  \let\tikz@moveto@waiting=\relax%
}


\def\tikz@flush@moveto@toward#1#2#3{%
  % #1 = a point towards which the last moveto should be corrected
  % #2 = a dimension to which the corrected x-coordinate should be stored
  % #3 = a dimension for the corrected y-coordinate
  \ifx\tikz@moveto@waiting\relax%
    % do nothing
  \else%
    \pgf@process{\pgfpointshapeborder{\tikz@moveto@waiting}{#1}}%
    #2=\pgf@x%
    #3=\pgf@y%
    \edef\tikz@timer@start{\noexpand\pgfpoint{\the\pgf@x}{\the\pgf@y}}%
    \pgfpathmoveto{\pgfpoint{\pgf@x}{\pgf@y}}%
  \fi%
  \let\tikz@moveto@waiting=\relax%
}


%
% Collecting labels on the path 
%

\def\tikz@collect@coordinate@onpath#1coordinate
\def\tikz@@collect@coordinate@opt#1[#2]{%
  \@ifnextchar({\tikz@@collect@coordinate#1[#2]}
\def\tikz@@collect@coordinate#1[#2](#3){%
  \tikz@collect@label@onpath#1node[shape=coordinate,#2](#3){}}

\def\tikz@collect@label@onpath#1node{%
  \expandafter\def\expandafter\tikz@collected@onpath\expandafter{\tikz@collected@onpath node}%
  \tikz@collect@label@scan#1}

\def\tikz@collect@label@scan#1{%  
  \@ifnextchar({\tikz@collect@paran#1}%
  {\@ifnextchar[{\tikz@collect@options#1}%
    {\@ifnextchar\bgroup{\tikz@collect@arg#1}%
      {#1}}}%
}%}}

\def\tikz@collect@paran#1(#2){%
  \expandafter\def\expandafter\tikz@collected@onpath\expandafter{\tikz@collected@onpath(#2)}%
  \tikz@collect@label@scan#1%
}
\def\tikz@collect@options#1[#2]{%
  \expandafter\def\expandafter\tikz@collected@onpath\expandafter{\tikz@collected@onpath[#2]}%
  \tikz@collect@label@scan#1%
}
\def\tikz@collect@arg#1#2{%
  \expandafter\def\expandafter\tikz@collected@onpath\expandafter{\tikz@collected@onpath{#2}}%
  #1%
}


\def\tikz@invoke@collected@onpath{%
  \tikz@node@is@a@labeltrue%
  \let\tikz@temp=\tikz@collected@onpath%
  \let\tikz@collected@onpath=\@empty%
  \expandafter\tikz@scan@next@command\tikz@temp\pgf@stop%
  \tikz@node@is@a@labelfalse%
}




% Syntax for lineto: 
% -- <point>

\def\tikz@lineto{%
  \@ifnextchar |%
  {\expandafter\tikz@hv@lineto\@gobble}%
  {\expandafter\@ifnextchar\tikz@activebar{\expandafter\tikz@hv@lineto\@gobble}%
    {\expandafter\tikz@lineto@mid\@gobble}}}
\def\tikz@lineto@mid{%
  \@ifnextchar n{\tikz@collect@label@onpath\tikz@lineto@mid}%
  {%
    \@ifnextchar c{\tikz@close}{%
      \@ifnextchar p{\pgfsetlinetofirstplotpoint\expandafter\tikz@plot\@gobble}%
        {\tikz@scan@one@point{\tikz@@lineto}}}}}
\def\tikz@@lineto#1{%
  % Record the starting point for later labels on the path:
  \edef\tikz@timer@start{\noexpand\pgfpoint{\the\tikz@lastx}{\the\tikz@lasty}}
  \iftikz@shapeborder%
    % ok, target is a shape. recalculate end
    \pgf@process{\pgfpointshapeborder{\tikz@shapeborder@name}{\tikz@last@position}}%
    \tikz@make@last@position{\pgfpoint{\pgf@x}{\pgf@y}}%
    \tikz@flush@moveto@toward{\tikz@last@position}\pgf@x\pgf@y%
    \tikz@path@lineto{\tikz@last@position}%
    \edef\tikz@timer@end{\noexpand\pgfpoint{\the\tikz@lastx}{\the\tikz@lasty}}%
    \tikz@make@last@position{#1}%
    \edef\tikz@moveto@waiting{\tikz@shapeborder@name}%    
  \else%
    % target is a reasonable point...
    % Record the starting point for later labels on the path:
    \tikz@make@last@position{#1}%
    \tikz@flush@moveto@toward{\tikz@last@position}\pgf@x\pgf@y%
    \tikz@path@lineto{\tikz@last@position}%
    \edef\tikz@timer@end{\noexpand\pgfpoint{\the\tikz@lastx}{\the\tikz@lasty}}%
  \fi%
  \let\tikz@timer=\tikz@timer@line%
  \tikz@scan@next@command%
}

% snake or lineto?
\def\tikz@path@lineto#1{%
  \iftikz@snaked%
    \pgfpathsnakesto{\tikz@presnake,{\tikz@snake}{\tikz@mainsnakelength},\tikz@postsnake}{#1}%
  \else%
    \pgfpathlineto{#1}%
  \fi%
}

% snake or lineto?
\def\tikz@path@close#1{%
  \iftikz@snaked%
    \pgfpathsnakesto{\tikz@presnake,{\tikz@snake}{\tikz@mainsnakelength},\tikz@postsnake}{#1}%
    \pgfpathclose%
  \else%
    \pgfpathclose%
  \fi%
}


% Syntax for lineto horizontal/vertical: 
% -| <point>

\def\tikz@hv@lineto{%
  \@ifnextchar n
  {\tikz@collect@label@onpath\tikz@hv@lineto}
  {\@ifnextchar c{\tikz@collect@coordinate@onpath\tikz@hv@lineto}%
    {\tikz@scan@one@point{\tikz@@hv@lineto}}}}
\def\tikz@@hv@lineto#1{%
  \edef\tikz@timer@start{\noexpand\pgfpoint{\the\tikz@lastx}{\the\tikz@lasty}}%
  \pgf@yc=\tikz@lasty%
  \tikz@make@last@position{#1}%
  \tikz@flush@moveto@toward{\pgfpoint{\tikz@lastx}{\pgf@yc}}\pgf@x\pgf@yc%
  \iftikz@shapeborder%
    % ok, target is a shape. have to work now:
    {%
      \pgf@process{\pgfpointshapeborder{\tikz@shapeborder@name}{\pgfpoint{\tikz@lastx}{\pgf@yc}}}%
      \tikz@make@last@position{\pgfpoint{\pgf@x}{\pgf@y}}%
      \tikz@path@lineto{\pgfpoint{\tikz@lastx}{\pgf@yc}}%
      \tikz@path@lineto{\tikz@last@position}%
      \xdef\tikz@timer@end@temp{\noexpand\pgfpoint{\the\tikz@lastx}{\the\tikz@lasty}}% move out of group
    }%
    \let\tikz@timer@end=\tikz@timer@end@temp%
    \edef\tikz@moveto@waiting{\tikz@shapeborder@name}%    
  \else%
    \tikz@path@lineto{\pgfpoint{\tikz@lastx}{\pgf@yc}}%
    \tikz@path@lineto{\tikz@last@position}%
    \edef\tikz@timer@end{\noexpand\pgfpoint{\the\tikz@lastx}{\the\tikz@lasty}}% move out of group
  \fi%
  \let\tikz@timer=\tikz@timer@hvline%
  \tikz@scan@next@command%
}

% Syntax for lineto vertical/horizontal: 
% |- <point>

\def\tikz@vh@lineto-{\tikz@vh@lineto@next}
\def\tikz@vh@lineto@next{%
  \@ifnextchar n
  {\tikz@collect@label@onpath\tikz@vh@lineto@next}
  {\@ifnextchar c{\tikz@collect@coordinate@onpath\tikz@vh@lineto@next}%
    {\tikz@scan@one@point\tikz@@vh@lineto}}}
\def\tikz@@vh@lineto#1{%
  \edef\tikz@timer@start{\noexpand\pgfpoint{\the\tikz@lastx}{\the\tikz@lasty}}%
  \pgf@xc=\tikz@lastx%
  \tikz@make@last@position{#1}%
  \tikz@flush@moveto@toward{\pgfpoint{\pgf@xc}{\tikz@lasty}}\pgf@xc\pgf@y%
  \iftikz@shapeborder%
    % ok, target is a shape. have to work now:
    {%
      \pgf@process{\pgfpointshapeborder{\tikz@shapeborder@name}{\pgfpoint{\pgf@xc}{\tikz@lasty}}}%
      \tikz@make@last@position{\pgfpoint{\pgf@x}{\pgf@y}}%
      \tikz@path@lineto{\pgfpoint{\pgf@xc}{\tikz@lasty}}%
      \tikz@path@lineto{\tikz@last@position}%
      \xdef\tikz@timer@end@temp{\noexpand\pgfpoint{\the\tikz@lastx}{\the\tikz@lasty}}% move out of group
    }%
    \let\tikz@timer@end=\tikz@timer@end@temp%
    \edef\tikz@moveto@waiting{\tikz@shapeborder@name}%    
  \else%
    \tikz@path@lineto{\pgfpoint{\pgf@xc}{\tikz@lasty}}%
    \tikz@path@lineto{\tikz@last@position}%
    \edef\tikz@timer@end{\noexpand\pgfpoint{\the\tikz@lastx}{\the\tikz@lasty}}%
  \fi%
  \let\tikz@timer=\tikz@timer@vhline%
  \tikz@scan@next@command%
}

% Syntax for cycle: 
% -- cycle
\def\tikz@close c{%
  \@ifnextchar o{\tikz@collect@coordinate@onpath\tikz@lineto@mid c}% oops, a coordinate
  {\tikz@@close c}}%
\def\tikz@@close cycle{%
  \tikz@flush@moveto%
  \tikz@path@close{\expandafter\pgfpoint\pgfsyssoftpath@lastmoveto}%
  \def\pgfstrokehook{}%
  \let\tikz@timer=\@undefined%
  \tikz@scan@next@command%
}


% Syntax for options: 
% [options]
\def\tikz@parse@options#1]{%
  \tikz@setkeys{#1}%
  \tikz@scan@next@command%
}



% Syntax for edge from parent: 
% edge from parent [options]
\def\tikz@edgetoparent dge from parent{\@ifnextchar[\tikz@@edgetoparent{\tikz@@edgetoparent[]}}%}
\def\tikz@@edgetoparent[#1]{%
  \let\tikz@edge@to@parent@needed=\@empty%
  \tikz@node@is@a@labeltrue%
  \tikz@scan@next@command [style=edge from parent,#1] \tikz@edge@to@parent@path%
}


% Syntax for bezier curves
% .. controls(point) and (point) .. (target)
% .. controls(point) .. (target) 
% .. (target) % currently not supported

\def\tikz@dot.{\tikz@@dot}%
\def\tikz@@dot{%
  \@ifnextchar n%
  {\tikz@collect@label@onpath\tikz@@dot}%
  {\@ifnextchar c{\tikz@curveto@double}{\tikz@curveto@auto}}}

\def\tikz@curveto@double co{%
  \@ifnextchar o{\tikz@collect@coordinate@onpath\tikz@@dot co}
  {\tikz@cureveto@@double}}
\def\tikz@cureveto@@double ntrols#1{%
  \tikz@scan@one@point\tikz@curveA#1%
}
\def\tikz@curveA#1{%
  \edef\tikz@timer@start{\noexpand\pgfpoint{\the\tikz@lastx}{\the\tikz@lasty}}%
  {%
    \tikz@make@last@position{#1}%
    \xdef\tikz@curve@first{\noexpand\pgfpoint{\the\tikz@lastx}{\the\tikz@lasty}}%
  }%
  \@ifnextchar a
  {\tikz@curveBand}%
  {\let\tikz@curve@second\tikz@curve@first\tikz@curveCdots}%
}
\def\tikz@curveBand and{%
  \tikz@scan@one@point\tikz@curveB%
}
\def\tikz@curveB#1{%
  \def\tikz@curve@second{#1}%
  \tikz@curveCdots}
\def\tikz@curveCdots{%
  \afterassignment\tikz@curveCdot\let\@next=}
\def\tikz@curveCdot.{%
  \ifx\@next.%
  \else%
    \PackageError{tikz}{Dot expected}{}%
  \fi%
  \tikz@updatecurrenttrue%
  \tikz@curveCcheck%
}
\def\tikz@curveCcheck{%
  \@ifnextchar n%
  {\tikz@collect@label@onpath\tikz@curveCcheck}
  {\@ifnextchar c{\tikz@collect@coordinate@onpath\tikz@curveCcheck}
    {\tikz@scan@one@point\tikz@curveC}}%
}
\def\tikz@curveC#1{%
  \tikz@make@last@position{#1}%
  \edef\tikz@curve@third{\noexpand\pgfpoint{\the\tikz@lastx}{\the\tikz@lasty}}%
  {%
    \tikz@lastxsaved=\tikz@lastx%
    \tikz@lastysaved=\tikz@lasty%
    \tikz@make@last@position{\tikz@curve@second}%
    \xdef\tikz@curve@second{\noexpand\pgfpoint{\the\tikz@lastx}{\the\tikz@lasty}}%
  }%
  %
  % Start recalculating things in case start and end are shapes.
  %
  % First, the start:
  \ifx\tikz@moveto@waiting\relax%
  \else%
    \pgf@process{\pgfpointshapeborder{\tikz@moveto@waiting}{\tikz@curve@first}}%
    \edef\tikz@timer@start{\noexpand\pgfpoint{\the\pgf@x}{\the\pgf@y}}%
    \pgfpathmoveto{\pgfpoint{\pgf@x}{\pgf@y}}%
  \fi%
  \let\tikz@timer@cont@one=\tikz@curve@first%
  \let\tikz@timer@cont@two=\tikz@curve@second%    
  % Second, the end:
  \iftikz@shapeborder%
    % ok, target is a shape. recalculate third
    {%
      \pgf@process{\pgfpointshapeborder{\tikz@shapeborder@name}{\tikz@curve@second}}%
      \tikz@make@last@position{\pgfpoint{\pgf@x}{\pgf@y}}%
      \edef\tikz@curve@third{\noexpand\pgfpoint{\the\tikz@lastx}{\the\tikz@lasty}}%
      \pgfpathcurveto{\tikz@curve@first}{\tikz@curve@second}{\tikz@curve@third}%
      \global\let\tikz@timer@end@temp=\tikz@curve@third% move out of group
    }%
    \let\tikz@timer@end=\tikz@timer@end@temp%
    \edef\tikz@moveto@waiting{\tikz@shapeborder@name}%    
  \else%
    \pgfpathcurveto{\tikz@curve@first}{\tikz@curve@second}{\tikz@curve@third}%
    \let\tikz@timer@end=\tikz@curve@third
    \let\tikz@moveto@waiting=\relax%
  \fi%
  \let\tikz@timer=\tikz@timer@curve%  
  \tikz@scan@next@command%
}


% Syntax for rectangles: 
% rectangle <corner point> 
\def\tikz@rect ectangle{%
  \tikz@flush@moveto%
  \edef\tikz@timer@start{\noexpand\pgfpoint{\the\tikz@lastx}{\the\tikz@lasty}}%
  \tikz@@rect}%
\def\tikz@@rect{%
  \@ifnextchar n
  {\tikz@collect@label@onpath\tikz@@rect}
  {\@ifnextchar c{\tikz@collect@coordinate@onpath\tikz@@rect}%
    {
      \pgf@xa=\tikz@lastx\relax%
      \pgf@ya=\tikz@lasty\relax%
      \tikz@scan@one@point\tikz@rectB}}}
\def\tikz@rectB#1{%
  \tikz@make@last@position{#1}%
  \edef\tikz@timer@end{\noexpand\pgfpoint{\the\tikz@lastx}{\the\tikz@lasty}}%
  \let\tikz@timer=\tikz@timer@line%  
  \pgfpathmoveto{\pgfpoint{\pgf@xa}{\pgf@ya}}%
  \tikz@path@lineto{\pgfpoint{\pgf@xa}{\tikz@lasty}}%
  \tikz@path@lineto{\pgfpoint{\tikz@lastx}{\tikz@lasty}}%
  \tikz@path@lineto{\pgfpoint{\tikz@lastx}{\pgf@ya}}%
  \tikz@path@close{\pgfpoint{\pgf@xa}{\pgf@ya}}%
  \pgfpathmoveto{\pgfpoint{\tikz@lastx}{\tikz@lasty}}%
  \def\pgfstrokehook{}%
  \tikz@scan@next@command%
}



% Syntax for grids: 
% grid <corner point> 
\def\tikz@grid rid{%
  \tikz@flush@moveto%
  \pgf@xa=\tikz@lastx\relax%
  \pgf@ya=\tikz@lasty\relax%
  \tikz@scan@one@point\tikz@gridB}%
\def\tikz@gridB#1{%
  \tikz@make@last@position{#1}%
  \pgfpathgrid[stepx=\tikz@grid@x,stepy=\tikz@grid@y]%
    {\pgfpoint{\pgf@xa}{\pgf@ya}}{\pgfpoint{\tikz@lastx}{\tikz@lasty}}%
  \tikz@scan@next@command%
}



% Syntax for plot: 
% plot [local options] ...    % starts with a moveto
% -- plot [local options] ... % starts with a lineto
\def\tikz@plot lot{%
  \tikz@flush@moveto%
  \@ifnextchar[{\tikz@@plot}{\tikz@@plot[]}}%}
\def\tikz@@plot[#1]{%
  \begingroup%
    \let\tikz@options=\@empty%
    \tikz@every@style{every plot}%
    \tikz@setkeys{#1}%
    \@ifnextchar f{\tikz@plot@f}%
    {\@ifnextchar c{\tikz@plot@scan@points}%
      {\PackageError{tikz}{Cannot parse this plotting data}{}%
       \endgroup}}}
\def\tikz@plot@f f{\@ifnextchar i{\tikz@plot@file}{\tikz@plot@function}}

\def\tikz@plot@file ile#1{\def\tikz@plot@data{\pgfplotxyfile{#1}}\tikz@@@plot}%
\def\tikz@plot@scan@points coordinates#1{%
  \pgfplothandlerrecord\tikz@plot@data%
  \pgfplotstreamstart%
  \@ifnextchar\pgf@stop{\pgfplotstreamend\expandafter\tikz@@@plot\@gobble}
  {\tikz@scan@one@point\tikz@plot@next@point}%
  #1\pgf@stop%
}
\def\tikz@plot@next@point#1{%
  \pgfplotstreampoint{#1}%
  \@ifnextchar\pgf@stop{\pgfplotstreamend\expandafter\tikz@@@plot\@gobble}%
  {\tikz@scan@one@point\tikz@plot@next@point}%
}  
\def\tikz@plot@function unction#1{%
  \def\tikz@plot@filename{\tikz@plot@prefix\tikz@plot@id}%  
  \iftikz@plot@raw@gnuplot%
    \def\tikz@plot@data{\pgfplotgnuplot[\tikz@plot@filename]{#1}}%
  \else%
    \iftikz@plot@parametric%   
      \def\tikz@plot@data{\pgfplotgnuplot[\tikz@plot@filename]{%
          set samples \tikz@plot@sampels;
          set parametric;
          plot [t=\tikz@plot@domain] #1}}%
    \else%
      \def\tikz@plot@data{\pgfplotgnuplot[\tikz@plot@filename]{%
          set samples \tikz@plot@sampels;
          plot [x=\tikz@plot@domain] #1}}%
    \fi%
  \fi%
  \tikz@@@plot%
}

\def\tikz@plot@no@resample{%
  \IfFileExists{\tikz@plot@filename.table}%
  {\def\tikz@plot@data{\pgfplotxyfile{\tikz@plot@filename.table}}}%
  {}%
}


\def\tikz@@@plot{%
    \def\pgfplotlastpoint{\pgfpointorigin}%
    \tikz@plot@handler%
    \tikz@plot@data%
    \global\let\tikz@@@temp=\pgfplotlastpoint%
    \ifx\tikz@plot@mark\@empty%
    \else%
      % Marks are drawn after the path.
      \setbox\tikz@figbox=\hbox{%
        \unhbox\tikz@figbox%
        \hbox{{%
          \pgfinterruptpath%
            \pgfscope%
              \let\tikz@options=\@empty%
              \let\tikz@transform=\@empty%
              \expandafter\tikz@setkeys\expandafter{\tikz@plot@mark@options}%
              \tikz@options%
              \pgfplothandlermark{\tikz@transform\pgfuseplotmark{\tikz@plot@mark}}%
              \tikz@plot@data%
            \endpgfscope
          \endpgfinterruptpath%
        }}%
      }%
    \fi%
    \global\setbox\tikz@tempbox=\copy\tikz@figbox%
  \endgroup%
  \setbox\tikz@figbox=\box\tikz@tempbox%  
  \tikz@make@last@position{\tikz@@@temp}%  
  \tikz@scan@next@command%
}


\pgfdeclareplotmark{ball}
{%
  \def\tikz@shading{ball}%
  \shade (0,0) circle (\pgfplotmarksize);%
}




% Syntax for cosine curves:
% cos <end of quarter-period>
\def\tikz@cosine s{\tikz@scan@one@point\tikz@@cosine}
\def\tikz@@cosine#1{%
  \tikz@flush@moveto%
  \pgf@process{#1}%
  \pgf@xc=\pgf@x%
  \pgf@yc=\pgf@y%
  \advance\pgf@xc by-\tikz@lastx%
  \advance\pgf@yc by-\tikz@lasty%
  \advance\tikz@lastx by\pgf@xc%
  \advance\tikz@lasty by\pgf@yc%
  \tikz@lastxsaved=\tikz@lastx%
  \tikz@lastysaved=\tikz@lasty%
  \tikz@updatecurrenttrue%
  \pgfpathcosine{\pgfpoint{\pgf@xc}{\pgf@yc}}%
  \tikz@scan@next@command%
}

% Syntax for sine curves:
% sin <end of quarter-period>
\def\tikz@sine in{\tikz@scan@one@point\tikz@@sine}
\def\tikz@@sine#1{%
  \tikz@flush@moveto%
  \pgf@process{#1}%
  \pgf@xc=\pgf@x%
  \pgf@yc=\pgf@y%
  \advance\pgf@xc by-\tikz@lastx%
  \advance\pgf@yc by-\tikz@lasty%
  \advance\tikz@lastx by\pgf@xc%
  \advance\tikz@lasty by\pgf@yc%
  \tikz@lastxsaved=\tikz@lastx%
  \tikz@lastysaved=\tikz@lasty%
  \tikz@updatecurrenttrue%
  \pgfpathsine{\pgfpoint{\pgf@xc}{\pgf@yc}}%
  \tikz@scan@next@command%
}

% Syntax for parabolas: 
% parabola[options] bend <coordinate> <coordinate>
\def\tikz@parabola arabola

\def\tikz@parabola@options[#1]{%
  \def\tikz@parabola@option{#1}%
  \@ifnextchar b{\tikz@parabola@scan@bend}{\tikz@scan@one@point\tikz@parabola@semifinal}}
\def\tikz@parabola@scan@bend bend{\tikz@scan@one@point\tikz@parabola@scan@bendB}
\def\tikz@parabola@scan@bendB#1{%
  \def\tikz@parabola@bend{#1}%
  \tikz@scan@one@point\tikz@parabola@semifinal%
}
\def\tikz@parabola@semifinal#1{%
  % Save original start:
  \pgf@xb=\tikz@lastx%
  \pgf@yb=\tikz@lasty%
  \tikz@make@last@position{#1}%
  \pgf@xc=\tikz@lastx%
  \pgf@yc=\tikz@lasty%
  \begingroup% now calculate bend:
    \expandafter\tikz@setkeys\expandafter{\tikz@parabola@option}%
    \tikz@lastxsaved=\tikz@parabola@bend@factor\tikz@lastx%
    \tikz@lastysaved=\tikz@parabola@bend@factor\tikz@lasty%
    \advance\tikz@lastxsaved by\pgf@xb%
    \advance\tikz@lastysaved by\pgf@yb%
    \advance\tikz@lastxsaved by-\tikz@parabola@bend@factor\pgf@xb%
    \advance\tikz@lastysaved by-\tikz@parabola@bend@factor\pgf@yb%
    \expandafter\tikz@make@last@position\expandafter{\tikz@parabola@bend}%
    % Calculate delta from bend
    \advance\pgf@xc by-\tikz@lastx%
    \advance\pgf@yc by-\tikz@lasty%
    % Ok, now calculate delta to bend
    \advance\tikz@lastx by-\pgf@xb%
    \advance\tikz@lasty by-\pgf@yb%
    \xdef\tikz@parabola@b{{\noexpand\pgfpoint{\the\tikz@lastx}{\the\tikz@lasty}}{\noexpand\pgfpoint{\the\pgf@xc}{\the\pgf@yc}}}%
  \endgroup%
  \expandafter\pgfpathparabola\tikz@parabola@b%
  \tikz@scan@next@command%
}


% Syntax for circles:
% circle (radius)
%
% Syntax for ellipses:
% ellipse (x-radius/y-radius)
\def\tikz@circle ircle{\tikz@@circle}
\def\tikz@ellipse llipse{\tikz@@circle}
\def\tikz@@circle{%
  \tikz@flush@moveto%
  \@ifnextchar(\tikz@@@circle{%)
    \advance\tikz@expandcount by -1%
    \ifnum\tikz@expandcount<0\relax%
      \let\@next=\tikz@@circle@scangiveup%
    \else%
      \let\@next=\tikz@@circle@scanexpand%
    \fi%
    \@next%
  }%
}
\def\tikz@@circle@scanexpand{\expandafter\tikz@@circle}
\def\tikz@@circle@scangiveup#1{\PackageError{tikz}{Cannot parse this radius}{}#1{\tikz@scan@next@command}}
\def\tikz@@@circle(#1){%
  \in@{ and }{#1}%
  \ifin@%
    \tikz@@ellipseB(#1)%
  \else%
    \tikz@@ellipseB(#1 and #1)%
  \fi%
  \tikz@scan@next@command%
}
\def\tikz@@ellipseB(#1 and #2){%
  \pgfpathellipse{\tikz@last@position}{\pgfpoint{#1}{0pt}}{\pgfpoint{0pt}{#2}}%
}

% Syntax 1 for arcs:
% arc (start angle:end angle:radius)
%
% Syntax 2 for arcs:
% arc (start angle:end angle:x-radius/y-radius)
\def\tikz@arcA rc{%
  \tikz@flush@moveto%
  \@ifnextchar({\tikz@@arcto}{\expandafter\tikz@arcA\expandafter r\expandafter c}}

\def\tikz@@arcto(#1){%
  \edef\tikz@temp{(#1)}%
   \expandafter\tikz@@@arcto@and\tikz@temp%
}

\def\tikz@@@arcto@and(#1:#2:#3){%
  \in@{ and }{#3}%
  \ifin@% 
    \tikz@parse@arc@and(#1:#2:#3)%
  \else%
    \tikz@@@arcto(#1:#2:#3)%
  \fi%
}

\def\tikz@parse@arc@and(#1:#2:#3 and #4){%
  \tikz@@@arcto(#1:#2:#3/#4)%
}

\def\tikz@@@arcto(#1:#2:#3){%
  \pgfpatharc{#1}{#2}{#3}%
  \pgf@process{\pgfpointpolar{#1}{#3}}%
  \advance\tikz@lastx by-\pgf@x%
  \advance\tikz@lasty by-\pgf@y%
  \pgf@process{\pgfpointpolar{#2}{#3}}%
  \advance\tikz@lastx by\pgf@x%
  \advance\tikz@lasty by\pgf@y%
  \tikz@lastxsaved=\tikz@lastx%
  \tikz@lastysaved=\tikz@lasty%
  \tikz@scan@next@command%
}


% Syntax for coordinates:
% coordinate[options] (coordinate name) at (point)
% where ``at (point)'' is optional
\def\tikz@coordinate ordinate{%
  \@ifnextchar[{\tikz@@coordinate@opt}{\tikz@@coordinate@opt[]}}
\def\tikz@@coordinate@opt[#1]
\def\tikz@@coordinate[#1](#2){%
  \@ifnextchar a{\tikz@@coordinate@at[#1](#2)}
  {\tikz@fig ode[shape=coordinate,#1](#2){}}}
\def\tikz@@coordinate@at[#1](#2)at#3(#4){%
  \tikz@fig ode[shape=coordinate,#1](#2)at(#4){}}
  


% Syntax for nodes:
% node[options] (node name) {label text}
%
% all of [options], (node name) and {label text} are optional. There
% can be multiple options before the label text as in
% node[draw] (a) [rotate=10] {text}
%
% A label text always ``ends'' the node.
\def\tikz@fig ode{%
  \begingroup%
  \let\tikz@fig@name=\@empty%
    \begingroup%
      \let\nodepart=\tikz@nodepart%
      \let\tikz@options=\@empty%
      \let\tikz@transform=\@empty%
      \let\tikz@mode=\@empty%
      \def\tikz@node@at{\pgfpoint{\the\tikz@lastx}{\the\tikz@lasty}}%
      \iftikz@node@is@a@label%
      \else%
        \let\tikz@time=\@empty%
      \fi%
      \tikz@every@style{every node}%
      \tikz@@scan@fig}%
\def\tikz@@scan@fig{%
  \@ifnextchar a{\tikz@fig@scan@at}
  {\@ifnextchar({\tikz@fig@scan@name}
    {\@ifnextchar[{\tikz@fig@scan@options}%
      {\@ifnextchar\bgroup{\tikz@fig@main}%
      {\PackageError{tikz}{A node must have a (possibly empty) label text}{}%
       \tikz@fig@main{}}}}}}%}}
\def\tikz@fig@scan@at at{%
  \tikz@scan@one@point\tikz@@fig@scan@at}
\def\tikz@@fig@scan@at#1{%
  \def\tikz@node@at{#1}\tikz@@scan@fig}%
\def\tikz@fig@scan@name(#1){\edef\tikz@fig@name{#1}\tikz@@scan@fig}%
\def\tikz@fig@scan@options[#1]{\tikz@setkeys{#1}\tikz@@scan@fig}%
\def\tikz@fig@main{\afterassignment\tikz@@fig@main\let\next=}
\def\tikz@@fig@main{%
    \@ifundefined{pgf@sh@s@\tikz@shape}%
    {\PackageError{tikz}%
      {Unknown shape ``\tikz@shape.'' Using ``rectangle'' instead}{}%
      \def\tikz@shape{rectangle}}%
    {}%
    \tikz@every@style{every \tikz@shape\space node}%
    \setbox\pgfnodeparttextbox=\hbox%
      \bgroup%
        \tikz@every@style{every text node part}%
        \ifx\tikz@textopacity\@empty%
        \else%
          \pgfsetfillopacity{\tikz@textopacity}%
          \pgfsetstrokeopacity{\tikz@textopacity}%
        \fi%
        \pgfinterruptpicture%
          \ifx\tikz@text@width\@empty%
          \else%
            \begingroup%
              \minipage[t]{\tikz@text@width}%
                \tikz@text@action%
          \fi%
          \bgroup%
            \aftergroup\unskip%
            \ifx\tikz@textcolor\@empty%
            \else%
              \colorlet{.}{\tikz@textcolor}%
            \fi%
            \pgfsetcolor{.}%
            \tikz@textfont%  
            \setbox\tikz@figbox=\box\voidb@x%
            \tikz@uninstallcommands%
            \aftergroup\tikz@fig@collectresetcolor%
            \ignorespaces%
}
\def\tikz@fig@collectresetcolor{%
  \@ifnextchar\reset@color%
  {\reset@color\afterassignment\tikz@fig@collectresetcolor\let\tikz@temp=}%
  {\tikz@fig@boxdone}%
}
\def\tikz@fig@boxdone{%
          \ifx\tikz@text@width\@empty%
          \else%
              \endminipage%
            \endgroup%
          \fi%
        \endpgfinterruptpicture%
      \egroup%
    \@ifnextchar c{\tikz@fig@withchildren}{\@ifnextchar[{\tikz@fig@withchildren}{\tikz@fig@continue}}}%}
\def\tikz@fig@withchildren{%
  \ifx\tikz@fig@name\@empty%
    % Assign a dummy name
    \global\advance\tikz@fig@count by1\relax
    \edef\tikz@fig@name{tikz@f@\the\tikz@fig@count}%
  \fi%
  \tikz@fig@continue%
}
\def\tikz@fig@continue{%
    \ifx\tikz@text@width\@empty%
    \else%
      \setlength{\pgf@x}{\tikz@text@width}%
      \wd\pgfnodeparttextbox=\pgf@x%
    \fi%
    % Possibly, we are ``online''
    \ifx\tikz@time\@empty%
      \pgftransformshift{\tikz@node@at}%
      \iftikz@fullytransformed%
      \else%
        \pgftransformresetnontranslations
      \fi%
    \else%
      \tikz@timer%
    \fi%
    % Invoke local transformations
    \tikz@transform%  
    \setbox\tikz@figbox=\hbox{%
      \setbox\@tempboxa=\copy\tikz@figbox%
      \unhbox\@tempboxa%
      \hbox{{%
          \pgfinterruptpath%
            \pgfscope%
              \tikz@options%
              \setbox\tikz@figbox=\box\voidb@x%
              \pgfmultipartnode{\tikz@shape}{\tikz@anchor}{\tikz@fig@name}{%
                \@tempdima=\pgflinewidth%
                {\begingroup\tikz@finish}%
                \global\pgflinewidth=\@tempdima%
              }%
            \endpgfscope
          \endpgfinterruptpath%
      }}%
    }%
    % shift box outside group
    \global\setbox\tikz@tempbox=\copy\tikz@figbox%
    \global\let\tikz@last@fig@name=\tikz@fig@name%
  \endgroup\endgroup%
  \setbox\tikz@figbox=\box\tikz@tempbox%  
  \tikz@scan@next@command%
}
\let\tikz@fig@continue@orig=\tikz@fig@continue


% Syntax for parts of  nodes:
% node ... {... \nodepart{name} ... \nodepart{name} ...}

\def\tikz@nodepart#1{%
  \unskip%
  \gdef\tikz@nodepart@name{#1}%
  \global\let\tikz@fig@continue=\tikz@nodepart@continue%
  \@ifnextchar x{\egroup\relax}{\egroup\relax}% gobble spaces
}
\def\tikz@nodepart@continue{%
  \global\let\tikz@fig@continue=\tikz@fig@continue@orig%
  % Now start new box:
   \expandafter\setbox\csname pgfnodepart\tikz@nodepart@name box\endcsname=\hbox%
      \bgroup%
        \tikz@every@style{every \tikz@nodepart@name\space node part}%
        \pgfinterruptpicture%
          \ifx\tikz@text@width\@empty%
          \else%
            \begingroup%
              \minipage[t]{\tikz@text@width}%
                \tikz@text@action%
          \fi%
          \bgroup%
            \aftergroup\unskip%
            \ifx\tikz@textcolor\@empty%
            \else%
              \colorlet{.}{\tikz@textcolor}%
            \fi%
            \pgfsetcolor{.}%
            \tikz@textfont%  
            \setbox\tikz@figbox=\box\voidb@x%
            \tikz@uninstallcommands%
            \aftergroup\tikz@fig@collectresetcolor%
            \ignorespaces%
}



% Syntax for trees:
% node {...} child [options] {...} child [options] {...} ...
% node {...} children [options] \var in {list} [options] {...} ...

\def\tikz@children{%
  % Start collecting the children:
  \let\tikz@children@list=\@empty%
  \tikznumberofchildren=0\relax%
  \tikz@collect@children c}

\def\tikz@collect@children{\@ifnextchar c{\tikz@collect@children@cchar}{\tikz@children@collected}}
\def\tikz@collect@children@cchar c{\@ifnextchar h{\tikz@collect@child}{\tikz@children@collected c}}
\def\tikz@collect@child hild{\@ifnextchar[{\tikz@collect@childA}{\tikz@collect@childA[]}}%}
\def\tikz@collect@childA[#1]{\@ifnextchar f{\tikz@collect@children@foreach[#1]}{\tikz@collect@childB[#1]}}
\def\tikz@collect@childB[#1]{%
  \advance\tikznumberofchildren by1\relax
  \expandafter\def\expandafter\tikz@children@list\expandafter{\tikz@children@list \tikz@childnode[#1]}%
  \@ifnextchar\bgroup{\tikz@collect@child@code}{\tikz@collect@child@code{}}}
\def\tikz@collect@child@code#1{%
  \expandafter\def\expandafter\tikz@children@list\expandafter{\tikz@children@list{#1}}%
  \tikz@collect@children%
}
\def\tikz@collect@children@foreach[#1]foreach#2in#3{%
  \@ifnextchar\bgroup{\tikz@collect@children@foreachA{#1}{#2}{#3}}{\tikz@collect@children@foreachA{#1}{#2}{#3}{}}}
\def\tikz@collect@children@foreachA#1#2#3#4{%
  \expandafter\def\expandafter\tikz@children@list\expandafter
    {\tikz@children@list\tikz@childrennodes[#1]{#2}{#3}{#4}}%
  \c@pgf@counta=\tikznumberofchildren%
  \foreach#2in{#3}%
  {%
    \global\advance\c@pgf@counta by1\relax%
  }%
  \tikznumberofchildren=\c@pgf@counta%
  \tikz@collect@children%
}
\long\def\tikz@children@collected{%
  \begingroup%
    \advance\tikztreelevel by 1\relax%
    \let\tikz@options=\@empty%
    \let\tikz@transform=\@empty%
    \tikz@every@style{level \the\tikztreelevel}%
    \tikz@transform%            
    \let\tikzparentnode=\tikz@last@fig@name%
    % Transform to center of node
    \pgftransformshift{\pgfpointanchor{\tikzparentnode}{center}}%
    \tikznumberofcurrentchild=0\relax%
    \tikz@children@list%
    \global\setbox\tikz@tempbox=\copy\tikz@figbox%
  \endgroup%
  \setbox\tikz@figbox=\box\tikz@tempbox%  
  \tikz@scan@next@command%
}


% Syntax for children:
%
% children [all children options] \var in {values} [child options] {...}
\def\tikz@childrennodes[#1]#2#3#4{%
  \c@pgf@counta=\tikznumberofcurrentchild\relax%
  \setbox\tikz@tempbox=\box\tikz@figbox%
  \foreach#2in{#3}{%
    \tikznumberofcurrentchild=\c@pgf@counta\relax%
    \setbox\tikz@figbox=\box\tikz@tempbox%
    \tikz@childnode[#1]{#4}%
    % we must now make the current child number and the figbox survive
    % the group
    \global\c@pgf@counta=\tikznumberofcurrentchild\relax%
    \global\setbox\tikz@tempbox=\box\tikz@figbox%
  }%
  \tikznumberofcurrentchild=\c@pgf@counta\relax%
  \setbox\tikz@figbox=\box\tikz@tempbox%
}


% Syntax for child:
%
% child
%
% child[options]
%
% child[options] {node (name) {child node text} ...
%   edge from parent[options] node {label text} node {label text}}

\def\tikz@childnode[#1]#2{%
  \advance\tikznumberofcurrentchild by1\relax%
  \setbox\tikz@figbox=\hbox\bgroup%
    \unhbox\tikz@figbox%
    \hbox\bgroup\bgroup%
        \pgfinterruptpath%
          \pgfscope%
            \let\tikz@transform=\@empty%
            \tikz@every@style{every child}%
            \tikz@setkeys{#1}%
            \tikz@options%
            \tikz@transform%            
            \tikz@grow%
            % Typeset node:
            \edef\tikz@parent@node@name{[name=\tikzparentnode-\the\tikznumberofcurrentchild,style=every child node]}%
            \def\tikz@child@node@text{[shape=coordinate]{}}
            \tikz@parse@child@node#2\pgf@stop%
            \expandafter\expandafter\expandafter\node
            \expandafter\tikz@parent@node@name
              \tikz@child@node@text
              \pgfextra{\global\let\tikz@childnode@name=\tikz@last@fig@name};%
            \let\tikzchildnode=\tikz@childnode@name%
            {%
              \def\tikz@edge@to@parent@needed{edge from parent}
              \ifx\tikz@child@node@rest\@empty%
                \path edge from parent;%
              \else%
                \path (0,0) \tikz@child@node@rest \tikz@edge@to@parent@needed;%
              \fi%
            }%
        \endpgfscope%
      \endpgfinterruptpath%
    \egroup\egroup%
  \egroup%
}

\def\tikz@parse@child@node{%
  \@ifnextchar n{\tikz@parse@child@node@n}%
  {\@ifnextchar c{\tikz@parse@child@node@c}%
    {\tikz@parse@child@node@rest}}}
\def\tikz@parse@child@node@rest#1\pgf@stop{\def\tikz@child@node@rest{#1}}
\def\tikz@parse@child@node@c c{\@ifnextchar o{\tikz@parse@child@node@co}{\tikz@parse@child@node@rest c}}
\def\tikz@parse@child@node@co o{\@ifnextchar o{\tikz@parse@child@node@coordinate}{\tikz@parse@child@node@rest co}}
\def\tikz@parse@child@node@coordinate ordinate{%
  \@ifnextchar ({\tikz@@parse@child@node@coordinate}{%
    \def\tikz@child@node@text{[shape=coordinate]{}}%
    \tikz@parse@child@node@rest}}%}
\def\tikz@@parse@child@node@coordinate(#1){%
  \@ifnextchar a{\tikz@p@c@n@c@at(#1)}{%
    \def\tikz@child@node@text{[shape=coordinate,name=#1]{}}%
    \tikz@parse@child@node@rest}}
\def\tikz@p@c@n@c@at(#1)at#2(#3){%
  \def\tikz@child@node@text{[shape=coordinate,name=#1]at(#3){}}%
  \tikz@parse@child@node@rest}%
\def\tikz@parse@child@node@n node{%
  \let\tikz@child@node@text=\@empty%
  \tikz@p@c@s}%
\def\tikz@p@c@s}
\def\tikz@p@c@s@at at#1(#2){%
  \expandafter\def\expandafter\tikz@child@node@text\expandafter{\tikz@child@node@text at(#2)}
  \tikz@p@c@s}
\def\tikz@p@c@s@paran(#1){%
  \expandafter\def\expandafter\tikz@child@node@text\expandafter{\tikz@child@node@text(#1)}
  \tikz@p@c@s}
\def\tikz@p@c@s@bra[#1]{%
  \expandafter\def\expandafter\tikz@child@node@text\expandafter{\tikz@child@node@text[#1]}
  \tikz@p@c@s}
\def\tikz@p@c@s@group#1{%
  \expandafter\def\expandafter\tikz@child@node@text\expandafter{\tikz@child@node@text{#1}}
  \tikz@parse@child@node@rest}


%
% Timers
% 

\def\tikz@timer@line{%
  \pgftransformlineattime{\tikz@time}{\tikz@timer@start}{\tikz@timer@end}%
}

\def\tikz@timer@vhline{%
  \ifdim\tikz@time pt<0.5pt% first half
    \pgf@process{\tikz@timer@start}%
    \pgf@xa=\pgf@x%
    \pgf@ya=\pgf@y%
    \pgf@process{\tikz@timer@end}%
    \pgf@xb=\tikz@time pt%
    \pgf@xb=2\pgf@xb%    
    \edef\tikz@marshal{\noexpand\pgftransformlineattime{\pgf@sys@tonumber{\pgf@xb}}{\noexpand\tikz@timer@start}{%
        \noexpand\pgfpoint{\the\pgf@xa}{\the\pgf@y}}}%
    \tikz@marshal%
  \else% second half
    \pgf@process{\tikz@timer@start}%
    \pgf@xa=\pgf@x%
    \pgf@ya=\pgf@y%
    \pgf@process{\tikz@timer@end}%
    \pgf@xb=\tikz@time pt%
    \pgf@xb=2\pgf@xb%
    \advance\pgf@xb by-1pt%
    \edef\tikz@marshal{\noexpand\pgftransformlineattime{\pgf@sys@tonumber{\pgf@xb}}%
      {\noexpand\pgfpoint{\the\pgf@xa}{\the\pgf@y}}{\noexpand\tikz@timer@end}}%
    \tikz@marshal%
  \fi%
}

\def\tikz@timer@hvline{%
  \ifdim\tikz@time pt<0.5pt% first half
    \pgf@process{\tikz@timer@start}%
    \pgf@xa=\pgf@x%
    \pgf@ya=\pgf@y%
    \pgf@process{\tikz@timer@end}%
    \pgf@xb=\tikz@time pt%
    \pgf@xb=2\pgf@xb%    
    \edef\tikz@marshal{\noexpand\pgftransformlineattime{\pgf@sys@tonumber{\pgf@xb}}{\noexpand\tikz@timer@start}{%
        \noexpand\pgfpoint{\the\pgf@x}{\the\pgf@ya}}}%
    \tikz@marshal%
  \else% second half
    \pgf@process{\tikz@timer@start}%
    \pgf@xa=\pgf@x%
    \pgf@ya=\pgf@y%
    \pgf@process{\tikz@timer@end}%
    \pgf@xb=\tikz@time pt%
    \pgf@xb=2\pgf@xb%
    \advance\pgf@xb by-1pt%
    \edef\tikz@marshal{\noexpand\pgftransformlineattime{\pgf@sys@tonumber{\pgf@xb}}%
      {\noexpand\pgfpoint{\the\pgf@x}{\the\pgf@ya}}{\noexpand\tikz@timer@end}}%
    \tikz@marshal%
  \fi%
}

\def\tikz@timer@curve{%
  \pgftransformcurveattime{\tikz@time}{\tikz@timer@start}{\tikz@timer@cont@one}{\tikz@timer@cont@two}{\tikz@timer@end}%
}




%
% Coordinate management
%


% Last position visited
\def\tikz@last@position{\pgfpoint{\tikz@lastx}{\tikz@lasty}}
\def\tikz@last@position@saved{\pgfpoint{\tikz@lastxsaved}{\tikz@lastysaved}}

% Make given point the last position visited
\def\tikz@make@last@position#1{%
  \pgf@process{#1}%
  \tikz@lastx=\pgf@x\relax%
  \tikz@lasty=\pgf@y\relax%
  \iftikz@updatecurrent%
    \tikz@lastxsaved=\pgf@x\relax%
    \tikz@lastysaved=\pgf@y\relax%
  \fi%
  \tikz@updatecurrenttrue%
}

\newif\iftikz@updatecurrent
\tikz@updatecurrenttrue



% Scanner: Scans a point or a relative point. 
% It then calls the first parameter with the argument set to an
% appropriate pgf command representing that point.

\def\tikz@scan@one@point#1{%
  \tikz@shapeborderfalse%
  \@ifnextchar+{\tikz@scan@relative#1}{\tikz@scan@absolute#1}}
\def\tikz@scan@absolute#1{%
  \@ifnextchar({\tikz@scan@@absolute#1}%)
  {%
    \advance\tikz@expandcount by -1%
    \ifnum\tikz@expandcount<0\relax%
      \let\@next=\tikz@@scangiveup%
    \else%
      \let\@next=\tikz@@scanexpand
    \fi%
    \@next{#1}%
  }%
}
\def\tikz@@scanexpand#1{\expandafter\tikz@scan@one@point\expandafter#1}
\def\tikz@@scangiveup#1{\PackageError{tikz}{Cannot parse this coordinate}{}#1{\pgfpointorigin}}
\def\tikz@scan@@absolute#1(#2){%
  \edef\tikz@temp{(#2)}%
  \expandafter\tikz@@scan@@absolute\expandafter#1\tikz@temp%
}
\def\tikz@@scan@@absolute#1({%
  \@ifnextchar[% uhoh... options!
  {\def\tikz@scan@point@recall{#1}\tikz@scan@options}%
  {\tikz@@@scan@@absolute#1(}%
}

\def\tikz@scan@options[#1]#2{%
  \def\tikz@scan@point@options{#1}%
  \tikz@@@scan@@absolute\tikz@scan@handle@options(#2%
}

\def\tikz@scan@handle@options#1{%
  {%
    % Ok, compute point with options set and zero transformation
    % matrix:
    \pgftransformreset%
    \let\tikz@transform=\@empty%
    \expandafter\tikz@setkeys\expandafter{\tikz@scan@point@options}%
    \tikz@transform%
    \pgf@process{\pgfpointtransformed{#1}}%
    \xdef\tikz@marshal{\expandafter\noexpand\tikz@scan@point@recall{\noexpand\pgfpoint{\the\pgf@x}{\the\pgf@y}}}%
  }%
  \tikz@marshal%  
}

\def\tikz@@@scan@@absolute#1(#2){%
  \in@{intersection of}{#2}%
  \ifin@%
    \let\@next\tikz@parse@intersection%
  \else%
    \in@|{#2}%
    \ifin@
      \in@{-|}{#2}%
      \ifin@
        \let\@next\tikz@parse@hv%
      \else%
        \let\@next\tikz@parse@vh%
      \fi%
    \else%
      \in@:{#2}%
      \ifin@
        \let\@next\tikz@parse@polar%
      \else%
        \in@,{#2}%
        \ifin@%      
          \let\@next\tikz@parse@regular%
        \else%
          \let\@next\tikz@parse@node%
        \fi%
      \fi%
    \fi%
  \fi%
  \@next#1(#2)%
}

\newif\iftikz@isdimension
\def\tikz@checkunit#1{%
  \@tempdima\z@%
  \afterassignment\tikz@@checkunit%
  \@tempdima#1\@tempdima\tikz@unique%
}
\def\tikz@@checkunit{\@ifnextchar\tikz@unique{\tikz@checkunit@number}{\tikz@checkunit@dimension}}
\def\tikz@checkunit@number\tikz@unique{\tikz@isdimensionfalse}
\def\tikz@checkunit@dimension#1\tikz@unique{\tikz@isdimensiontrue}

\def\tikz@parse@polar#1(#2:#3){%
  \@ifundefined{tikz@polar@dir@#2}
  {#1{\pgfpointpolar{#2}{#3}}}
  {%
    \edef\tikz@marshal{\noexpand#1{\noexpand\pgfpointpolar{\csname tikz@polar@dir@#2\endcsname}{#3}}}%
    \tikz@marshal%
  }%
}
\def\tikz@polar@dir@up{90}
\def\tikz@polar@dir@down{-90}
\def\tikz@polar@dir@left{180}
\def\tikz@polar@dir@right{0}
\def\tikz@polar@dir@north{90}
\def\tikz@polar@dir@south{-90}
\def\tikz@polar@dir@east{0}
\def\tikz@polar@dir@west{180}
\expandafter\def\csname tikz@polar@dir@north east\endcsname{45}
\expandafter\def\csname tikz@polar@dir@north west\endcsname{135}
\expandafter\def\csname tikz@polar@dir@south east\endcsname{-45}
\expandafter\def\csname tikz@polar@dir@south west\endcsname{-135}

\def\tikz@parse@regular#1(#2,#3){%
  \in@,{#3}%
  \ifin@%  
    \tikz@parse@splitxyz{#1}{#2}#3,%
  \else%
    \tikz@checkunit{#2}%
    \iftikz@isdimension%
      \def\@next{#1{\pgfpoint{#2}{#3}}}%
    \else%
      \def\@next{#1{\pgfpointxy{#2}{#3}}}%
    \fi%
  \fi%
  \@next%
}

\def\tikz@parse@splitxyz#1#2#3,#4,{%
  \def\@next{#1{\pgfpointxyz{#2}{#3}{#4}}}%
}

\def\tikz@coordinate@text{coordinate}

\def\tikz@parse@node#1(#2){%
  \in@.{#2}% Ok, flag this
  \ifin@
    \tikz@calc@anchor#2\tikz@stop%
  \else%
    \tikz@calc@anchor#2.center\tikz@stop% to be on the save side, in
                                % case iftikz@shapeborder is ignored...
    \expandafter\ifx\csname pgf@sh@ns@#2\endcsname\tikz@coordinate@text%
    \else
      \tikz@shapebordertrue%
      \def\tikz@shapeborder@name{#2}%
    \fi%
  \fi%
  \edef\tikz@marshal{\noexpand#1{\noexpand\pgfpoint{\the\pgf@x}{\the\pgf@y}}}%
  \tikz@marshal%
}

\def\tikz@calc@anchor#1.#2\tikz@stop{%
  \pgfpointanchor{#1}{#2}%
}


\def\tikz@parse@hv#1(#2){%
  \in@{ -| }{#2}%
  \ifin@%
    \let\tikz@next=\tikz@parse@hvboth%
  \else%
    \in@{ -|}{#2}%
    \ifin@%
      \let\tikz@next=\tikz@parse@hvleft%
    \else%
      \in@{-| }{#2}%
      \ifin@%
        \let\tikz@next=\tikz@parse@hvright%
      \else%
        \let\tikz@next=\tikz@parse@hvdone%
      \fi%
    \fi%
  \fi%
  \tikz@next#1(#2)}
\def\tikz@parse@hvboth#1(#2 -| #3){\tikz@parse@vhdone#1(#3|-#2)}
\def\tikz@parse@hvleft#1(#2 -|#3){\tikz@parse@vhdone#1(#3|-#2)}
\def\tikz@parse@hvright#1(#2-| #3){\tikz@parse@vhdone#1(#3|-#2)}
\def\tikz@parse@hvdone#1(#2-|#3){\tikz@parse@vhdone#1(#3|-#2)}

\def\tikz@parse@vh#1(#2){%
  \in@{ |- }{#2}%
  \ifin@%
    \let\tikz@next=\tikz@parse@vhboth%
  \else%
    \in@{ |-}{#2}%
    \ifin@%
      \let\tikz@next=\tikz@parse@vhleft%
    \else%
      \in@{|- }{#2}%
      \ifin@%
        \let\tikz@next=\tikz@parse@vhright%
      \else%
        \let\tikz@next=\tikz@parse@vhdone%
      \fi%
    \fi%
  \fi%
  \tikz@next#1(#2)}
\def\tikz@parse@vhboth#1(#2 |- #3){\tikz@parse@vhdone#1(#2|-#3)}
\def\tikz@parse@vhleft#1(#2 |-#3){\tikz@parse@vhdone#1(#2|-#3)}
\def\tikz@parse@vhright#1(#2|- #3){\tikz@parse@vhdone#1(#2|-#3)}
\def\tikz@parse@vhdone#1(#2|-#3){%
  {%
    \tikz@@@scan@@absolute\tikz@parse@vh@mid(#2)%
    \tikz@@@scan@@absolute\tikz@parse@vh@end(#3)%
    \xdef\tikz@marshal{\noexpand#1{\noexpand\pgfpoint{\the\pgf@xa}{\the\pgf@ya}}}%
  }%
  \tikz@shapeborderfalse%
  \tikz@marshal%
}
\def\tikz@parse@vh@mid#1{\pgf@process{#1}\pgf@xa=\pgf@x}
\def\tikz@parse@vh@end#1{\pgf@process{#1}\pgf@ya=\pgf@y}

\def\tikz@parse@intersection#1(intersection of #2--#3 and #4--#5){%
  {%
    \tikz@@@scan@@absolute\tikz@parse@intersection@a(#2)%
    \tikz@@@scan@@absolute\tikz@parse@intersection@b(#3)%
    \tikz@@@scan@@absolute\tikz@parse@intersection@c(#4)%
    \tikz@@@scan@@absolute\tikz@parse@intersection@d(#5)%
    \xdef\tikz@marshal{\noexpand#1{\noexpand\pgfpointintersectionoflines%
        {\noexpand\pgfpoint{\the\pgf@xa}{\the\pgf@ya}}%
        {\noexpand\pgfpoint{\the\pgf@xb}{\the\pgf@yb}}%
        {\noexpand\pgfpoint{\the\pgf@xc}{\the\pgf@yc}}%
        {\noexpand\pgfpoint{\the\pgf@x}{\the\pgf@y}}}}%
  }%
  \tikz@shapeborderfalse%
  \tikz@marshal%  
}

\def\tikz@parse@intersection@a#1{\pgf@process{#1}\pgf@xa=\pgf@x\pgf@ya=\pgf@y}
\def\tikz@parse@intersection@b#1{\pgf@process{#1}\pgf@xb=\pgf@x\pgf@yb=\pgf@y}
\def\tikz@parse@intersection@c#1{\pgf@process{#1}\pgf@xc=\pgf@x\pgf@yc=\pgf@y}
\def\tikz@parse@intersection@d#1{\pgf@process{#1}}

\def\tikz@scan@relative#1+{%
  \@ifnextchar+{\tikz@scan@plusplus#1}{\tikz@scan@oneplus#1}}

\def\tikz@scan@plusplus#1+{%
  \def\tikz@doafter{#1}%
  \tikz@scan@absolute\tikz@add%
}
\def\tikz@add#1{%
  \tikz@doafter{\pgfpointadd{#1}{\tikz@last@position@saved}}%
}
\def\tikz@scan@oneplus#1{%
  \def\tikz@doafter{#1}%
  \tikz@updatecurrentfalse%
  \tikz@scan@absolute\tikz@add%
} 


\endinput
