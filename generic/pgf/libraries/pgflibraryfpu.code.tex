% Copyright 2008/2009 by Christian Feuersaenger
%
% This file may be distributed and/or modified
%
% 1. under the LaTeX Project Public License and/or
% 2. under the GNU General Public License.
%
% See the file doc/generic/pgf/licenses/LICENSE for more details.

\newif\ifpgfmathfloatparseactive
\newif\ifpgfmathfloat@scaleactive

\pgfqkeys{/pgf}{
	% enable the FPU parser if it is not yet active
	%
	% It will be deactivated after the current TeX group
	fpu/.code={%
		\ifpgfmathfloatparseactive
		\else
			\pgfmathfloat@plots@install%
			\pgfutil@ifundefined{pgfmathdeclarefunction}{%
				% Ohoh - we are running on a TeX distribution with 
				% PGF 2.00 which doesn't have the new math engine.
				% I can provide special treatment here, provided that
				% all float commands are still able to run (that means
				% more information needs to be copied from the pgf cvs
				% to pgf 2.00 - for example pgfmathfloat.code.tex).
				% 
				% I employ this to work with pgfplots and pgf 2.00
				% using all new features.
				\pgfmathfloat@parser@install@pgf@two@null@null%
			}{%
				\pgfmathfloat@parser@install%
			}%
			\pgfqkeys{/pgf/fpu}{#1}%
			\pgfmathfloatparseactivetrue
		\fi
	},%
	% Use this to introduce a result scaling.
	% Every expression in which the prefix '*' occurs
	% will be multiplied with the result and converted to fixed point
	% representation.
	fpu/scale results/.code={%
		\ifpgfmathfloatparseactive
			\pgfmathparse{#1}%
		\else
			\pgfmathfloatparsenumber{#1}%
		\fi
		\let\pgfmathfloatscale=\pgfmathresult%
	}%		
}

\def\pgfmathfloatscale{1}

% This is the replacement parser invocation.
% It does two things which are different to \pgfmathparse:
% 1. it disables any dimension dependand scalings,
% 2. it implements the 'scale results' feature.
\def\pgfmathfloatparse{%
	\begingroup%
		% disable any dimension-dependant scalings:
		\let\pgfmathpostparse=\relax%
		\pgfmath@catcodes%
		\pgfmath@quickparsefalse%
		\pgfmathfloatparse@}

% for pgf 2.00 :
\def\pgfmathfloatparse@pgf@two@null@null{%
	\pgfmath@quickparsefalse%
	\pgfmathfloatparse@}

\def\pgfmathfloatparse@#1{%
	\edef\pgfmathfloat@expression{#1}%
	\expandafter\pgfmathfloatparse@@\pgfmathfloat@expression\pgfmathfloat@
	\ifpgfmathfloat@scaleactive
		\expandafter\pgfmathfloatmultiply@\expandafter{\pgfmathresult}{\pgfmathfloatscale}
		\pgfmathfloattofixed{\pgfmathresult}%
	\fi
}

\def\pgfmathfloat@char@asterisk{*}
\def\pgfmathfloatparse@@#1#2\pgfmathfloat@{%
	\def\pgfmathfloat@test{#1}%
	\ifx\pgfmathfloat@test\pgfmathfloat@char@asterisk%
		\def\pgfmathfloat@expression{#2}%
		\pgfmathfloat@scaleactivetrue
	\fi%
	\expandafter\pgfmathparse@\expandafter{\pgfmathfloat@expression}%
	% \endgroup provided by \pgfpathmarse@end
}

% Crude handling of file plots
%
\pgfkeys{/pgf/fixed point/.cd,
  scale file plot x/.code=\pgfmathfloatparse{#1}\edef\pgfmathfloatplotscalex{\pgfmathresult*},
  scale file plot y/.code=\pgfmathfloatparse{#1}\edef\pgfmathfloatplotscaley{\pgfmathresult*},
  scale file plot z/.code=\pgfmathfloatparse{#1}\edef\pgfmathfloatplotscalez{\pgfmathresult*}
}
\def\pgfmathfloat@plots@install{%
	\let\pgfmathfloatplotscalex=\pgfutil@empty
	\let\pgfmathfloatplotscaley=\pgfutil@empty
	\let\pgfmathfloatplotscalez=\pgfutil@empty
	\let\pgf@parsexyline=\pgfmathfloat@parsexyline%
	\let\pgf@parsexyzline=\pgfmathfloat@parsexyzline%
}
\def\pgfmathfloat@parsexyline#1 #2 #3\pgf@stop{%
	\edef\pgfmathfloat@marshal{%
		\noexpand\pgfplotstreampoint{\noexpand\pgfpointxy{\pgfmathfloatplotscalex#1}{\pgfmathfloatplotscaley#2}}%
	}%
	\pgfmathfloat@marshal%
}
\def\pgfmathfloat@parsexyzline#1 #2 #3 #4\pgf@stop{%
  \edef\pgfmathfloat@marshal{%
		\noexpand\pgfplotstreampoint{%
			\noexpand\pgfpointxyz{\pgfmathfloatplotscalex#1}{\pgfmathfloatplotscaley#2}{\pgfmathfloatplotscalez#3}%
		}%
	}%
	\pgfmathfloat@marshal%
}

% 
\def\pgfmathfloat@parser@install@functions{%
	% Install float commands...
	%
	\let\pgfmathadd@=\pgfmathfloatadd@%
	\let\pgfmathsubtract@=\pgfmathfloatsubtract@%
	\let\pgfmathneg@=\pgfmathfloatneg@%
	\let\pgfmathmultiply@=\pgfmathfloatmultiply@%
	\let\pgfmathdivide@=\pgfmathfloatdivide@%
	\let\pgfmathabs@=\pgfmathfloatabs@%
	\let\pgfmathround@=\pgfmathfloatround@%
	\let\pgfmathfloor@=\pgfmathfloatfloor@%
	\let\pgfmathmod@=\pgfmathfloatmod@%
	\let\pgfmathmax@=\pgfmathfloatmax@%
	\let\pgfmathmin@=\pgfmathfloatmin@%
	\let\pgfmathsin@=\pgfmathfloatsin@%
	\let\pgfmathcos@=\pgfmathfloatcos@%
	\let\pgfmathtan@=\pgfmathfloattan@%
	\let\pgfmathdeg@=\pgfmathfloatdeg@%
	\let\pgfmathrad@=\pgfmathfloatrad@%
	\let\pgfmathatan@=\pgfmathfloatatan@%
	\let\pgfmathasin@=\pgfmathfloatasin@%
	\let\pgfmathacos@=\pgfmathfloatacos@%
	\let\pgfmathcot@=\pgfmathfloatcot@%
	\let\pgfmathsec@=\pgfmathfloatsec@%
	\let\pgfmathcosec@=\pgfmathfloatcosec@%
	\let\pgfmathexp@=\pgfmathfloatexp@%
	\let\pgfmathln@=\pgfmathfloatln@%
	\let\pgfmathsqrt@=\pgfmathfloatsqrt@%
	\let\pgfmath@pi=\pgfmathfloatpi@%
	\let\pgfmathpi=\pgfmathfloatpi@%
	\let\pgfmathe@=\pgfmathfloate@%
	\let\pgfmathlessthan@=\pgfmathfloatlessthan@%
	\let\pgfmathpow@=\pgfmathfloatpow@
	\pgfmathfloat@install@unimplemented{ceil}%
	%\pgfmathfloat@install@unimplemented{pow}%
	\pgfmathfloat@install@unimplemented{factorial}%
	\pgfmathfloat@install@unimplemented{oct}%
	\pgfmathfloat@install@unimplemented{frac}%
	\pgfmathfloat@install@unimplemented{log2}%
	\pgfmathfloat@install@unimplemented{log10}%
	\pgfmathfloat@install@unimplemented{veclen}%
	\pgfmathfloat@install@unimplemented{greaterthan}%
	\pgfmathfloat@install@unimplemented{equalto}%
	\pgfmathfloat@install@unimplemented{rnd}%
	\pgfmathfloat@install@unimplemented{rand}%
	\pgfmathfloat@install@unimplemented{random}%
	\pgfmathfloat@install@unimplemented{setseed}%
	\pgfmathfloat@install@unimplemented{or}%
	\pgfmathfloat@install@unimplemented{hex}%
	\pgfmathfloat@install@unimplemented{Mod}%
	\pgfmathfloat@install@unimplemented{Hex}%
	\pgfmathfloat@install@unimplemented{true}%
	\pgfmathfloat@install@unimplemented{false}%
	\pgfmathfloat@install@unimplemented{not}%
	\pgfmathfloat@install@unimplemented{real}%
	\pgfmathfloat@install@unimplemented{bin}%
	\pgfmathfloat@install@unimplemented{notequal}%
%	\pgfmathfloat@install@unimplemented{atan2}%
%	\pgfmathfloat@install@unimplemented{height}%
	%
	%
	\let\pgfmathscientific=\pgfmathfloatscientific%
}
\def\pgfmathfloat@parser@install{%
	\pgfmathfloat@parser@install@functions
	%
	%
	%
	% The following methods actually enable the parser to work with
	% the internal floating point number representation.
	%
	% The idea is as follows:
	% 1. Every operand must be given in internal float representation.
	% 2. The internal float repr can be distinguished by a normal
	% number. This is accomplished by introducing a new "exponent"
	% token.
	% 3. The stack-push-operation checks whether the argument is a
	% float. If not, it is parsed properly before pushing it.
	\pgfmath@tokens@make{exponent}{Y}%
	\let\pgfmath@basic@parse@exponent=\pgfmath@parse@exponent%
	\let\pgfmath@basic@stack@push@operand=\pgfmath@stack@push@operand
	\let\pgfmath@stack@push@operand=\pgfmathfloat@stack@push@operand
	\let\pgfmath@parse@exponent=\pgfmathfloat@parse@float@or@exponent
	%
	\let\pgfmathparse=\pgfmathfloatparse%
}%

\def\pgfmathfloat@parser@install@pgf@two@null@null{%
	\pgfmathfloat@parser@install@functions
	%
	% The following methods actually enable the parser to work with
	% the internal floating point number representation.
	%
	% The idea is as follows:
	% 1. Every operand must be given in internal float representation.
	% 2. The internal float repr can be distinguished by a normal
	% number. This is accomplished by introducing a new "exponent"
	% token.
	% 3. The stack-push-operation checks whether the argument is a
	% float. If not, it is parsed properly before pushing it.
	\let\pgfmath@basic@parseunits@=\pgfmath@parseunits@%
	\let\pgfmath@basic@stack@push@operand=\pgfmath@stackpushoperand
	\let\pgfmath@stackpushoperand=\pgfmathfloat@stack@push@operand
	\let\pgfmath@parseunits@=\pgfmathfloat@parseunits@pgf@two@null@null
	\let\pgfmath@endparse=\pgfmathfloat@endparse@pgf@two@null@null
	\let\pgfmath@endparsegroup=\pgfmathfloat@endparsegroup@pgf@two@null@null
	%
	\let\pgfmathparse=\pgfmathfloatparse@pgf@two@null@null%
}%

%%%%%%%%%%%%%%%%%%%%%%%%%%%%%%%%%%%%%%%%%%%
%
% Hacks to the basic level pgf math engine:
%
% WARNING: These methods rely heavily on the internal float representation!
%%%%%%%%%%%%%%%%%%%%%%%%%%%%%%%%%%%%%%%%%%%

% for pgf2.00 :
\def\pgfmathfloat@parseunits@pgf@two@null@null#1#2{%
	\if#1Y% check whether it is a float
		\def\pgfmath@next{\pgfmathfloat@return@float@pgf@two@null@null#2}%
	\else
		\def\pgfmath@next{\pgfmath@basic@parseunits@#1#2}%
	\fi
	\pgfmath@next
}
% for pgf2.00:
\def\pgfmathfloat@return@float@pgf@two@null@null#1]{%
	\edef\pgfmathresult{\pgfmath@resulttemp Y#1]}%
	\let\pgfmath@resulttemp=\pgfmathresult
	\pgfmath@parseoperator%
}%
% for pgf2.00:
\def\pgfmathfloat@endparse@pgf@two@null@null#1\pgfmath@empty{%
    \pgfmath@processalloperations%
    \pgfmath@stackpop{\pgfmathresult}%
	% delete the final unit scalings
    \pgfmath@smuggleone{\pgfmathresult}%
  \endgroup%
  \ignorespaces%
}
% for pgf2.00:
\def\pgfmathfloat@endparsegroup@pgf@two@null@null{%
		\pgfmath@processalloperations%
		\pgfmath@stackpop{\pgfmathresult}%
		% eliminated register usage here...
		\pgfmath@smuggleone{\pgfmathresult}%
	\endgroup%
	\pgfmath@parsepostgroup%
}



% PRECONDITION:
% either
% 	<number>e
% 	        ^
% 	-> read the exponent.
% or
%   <sign>Y
%         ^
%   -> we have a parsed floating point number -> read it.
\def\pgfmathfloat@parse@float@or@exponent{%
	\if\pgfmath@token Y%
		% Ok, we actually HAVE a pre-parsed floating point number!
		% Return it.
		\expandafter\pgfmathfloat@return@float
	\else
		% We have a standard number in scientific format. Parse it.
		\expandafter\pgfmath@basic@parse@exponent
	\fi
}%
\def\pgfmathfloat@return@float#1]{%
	\edef\pgfmathresult{\pgfmath@number Y#1]}%
	\expandafter\pgfmath@basic@stack@push@operand\expandafter{\pgfmathresult}%
	\pgfmath@parse@@operator%
}%

% This extends the functionality of the basic level operand stack: it
% assures every element on the stack is a float.
\def\pgfmathfloat@stack@push@operand#1{%
	\pgfutil@in@ Y{#1}%
	\ifpgfutil@in@
		\pgfmath@basic@stack@push@operand{#1}%
	\else
		\pgfmathfloatparsenumber{#1}%
		\expandafter\pgfmath@basic@stack@push@operand\expandafter{\pgfmathresult}%
	\fi
}%
%%%%%%%%%%%%%%%%%%%%%%%%%%%%%%%%%%%%%%%%%%%
%
% --- END --- Hacks to the basic level pgf math engine
%
%%%%%%%%%%%%%%%%%%%%%%%%%%%%%%%%%%%%%%%%%%%



%%%%%%%%%%%%%%%%%%%%%%%%%%%%%%%%%%%%%%%%%%%%%%%%%%%%%%%%
%
% Here starts the implementation of the floating point
% routines.
%
% They can be used even if the FPU parser is not active.
%
%%%%%%%%%%%%%%%%%%%%%%%%%%%%%%%%%%%%%%%%%%%%%%%%%%%%%%%%

% Remember the basic math commands. They will be invoked as subroutines in floating point routines.
\let\pgfmath@basic@add@=\pgfmathadd@
\let\pgfmath@basic@subtract@=\pgfmathsubtract@
\let\pgfmath@basic@multiply@=\pgfmathmultiply@
\let\pgfmath@basic@divide@=\pgfmathdivide@
\let\pgfmath@basic@abs@=\pgfmathabs@
\let\pgfmath@basic@round@=\pgfmathround@
\let\pgfmath@basic@floor@=\pgfmathfloor@
\let\pgfmath@basic@ceil@=\pgfmathceil@
\let\pgfmath@basic@mod@=\pgfmathmod@
\let\pgfmath@basic@max@=\pgfmathmax@
\let\pgfmath@basic@min@=\pgfmathmin@
\let\pgfmath@basic@sin@=\pgfmathsin@
\let\pgfmath@basic@cos@=\pgfmathcos@
\let\pgfmath@basic@tan@=\pgfmathtan@
\let\pgfmath@basic@deg@=\pgfmathdeg@
\let\pgfmath@basic@rad@=\pgfmathrad@
\let\pgfmath@basic@atan@=\pgfmathatan@
\let\pgfmath@basic@asin@=\pgfmathasin@
\let\pgfmath@basic@acos@=\pgfmathacos@
\let\pgfmath@basic@cot@=\pgfmathcot@
\let\pgfmath@basic@sec@=\pgfmathsec@
\let\pgfmath@basic@cosec@=\pgfmathcosec@
\let\pgfmath@basic@pow@=\pgfmathpow@
\let\pgfmath@basic@exp@=\pgfmathexp@
\let\pgfmath@basic@ln@=\pgfmathln@
\let\pgfmath@basic@sqrt@=\pgfmathsqrt@
\let\pgfmath@basic@@pi=\pgfmath@pi
\let\pgfmath@basic@veclen@=\pgfmathveclen@
\let\pgfmath@basic@e@=\pgfmathe@
\let\pgfmath@basic@lessthan@=\pgfmathlessthan@
\let\pgfmath@basic@greaterthan@=\pgfmathgreaterthan@
\let\pgfmath@basic@equalto@=\pgfmathequalto@

\def\pgfmathfloatscientific#1#2{%
	\edef\pgfmathresult{#1e#2}%
	\expandafter\pgfmathfloatparsenumber\expandafter{\pgfmathresult}%
}
% Compares #1 with #2 and sets \pgfmathresult either to 1.0 or 0.0.
% 
% It also sets the boolean \ifpgfmathfloatcomparison (globally.
\def\pgfmathfloatlessthan@#1#2{%
%\def\pgfmathfloatlessthan#1#2#3\and#4#5#6{%
	\global\pgfmathfloatcomparisonfalse
	\begingroup
	\expandafter\pgfmathfloat@decompose#1\relax\pgfmathfloat@a@S\pgfmathfloat@a@M\pgfmathfloat@a@E
	\expandafter\pgfmathfloat@decompose#2\relax\pgfmathfloat@b@S\pgfmathfloat@b@M\pgfmathfloat@b@E
	\ifcase\pgfmathfloat@a@S
		% x = 0 -> (x<y <=> y >0)
		\ifnum1=\pgfmathfloat@b@S
			\global\pgfmathfloatcomparisontrue
		\fi
	\or
		% x > 0 -> (x<y <=> ( y > 0 && |x| < |y|) )
		\ifnum1=\pgfmathfloat@b@S
			% y>0:
			\pgfmathfloatlessthan@positive
		\fi
	\or
		% x < 0 -> (x<y <=> (y >= 0 ||   |x| > |y|) )
		\ifnum2=\pgfmathfloat@b@S
			% 'y<0':
			\pgfmathfloatgreaterthan@positive
		\else
			\global\pgfmathfloatcomparisontrue
		\fi
	\else
		\pgfmath@error{Sorry, 'pgfmathfloatlessthan@{#1}{#2}' not yet supported.}{}%
	\fi
	\endgroup
	\ifpgfmathfloatcomparison
		\def\pgfmathresult{1.0}%
	\else
		\def\pgfmathresult{0.0}%
	\fi
}
\let\pgfmathfloatlessthan=\pgfmathfloatlessthan@

% compares \pgfmathfloat@a@[SME] < \pgfmathfloat@b@[SME]
\def\pgfmathfloatlessthan@positive{%
	\ifnum\pgfmathfloat@a@E<\pgfmathfloat@b@E
		\global\pgfmathfloatcomparisontrue
	\else
		\ifnum\pgfmathfloat@a@E=\pgfmathfloat@b@E
			\ifdim\pgfmathfloat@a@M<\pgfmathfloat@b@M
				\global\pgfmathfloatcomparisontrue
			\fi
		\fi
	\fi
}

% compares \pgfmathfloat@a@[SME] > \pgfmathfloat@b@[SME]
\def\pgfmathfloatgreaterthan@positive{%
	\ifnum\pgfmathfloat@a@E>\pgfmathfloat@b@E
		\global\pgfmathfloatcomparisontrue
	\else
		\ifnum\pgfmathfloat@a@E=\pgfmathfloat@b@E
			\ifdim\pgfmathfloat@a@M>\pgfmathfloat@b@M
				\global\pgfmathfloatcomparisontrue
			\fi
		\fi
	\fi
}


\def\pgfmathfloatmax@#1#2{%
	\pgfmathfloatlessthan{#1}{#2}%
	\ifpgfmathfloatcomparison
		\edef\pgfmathresult{#2}%
	\else
		\edef\pgfmathresult{#1}%
	\fi
}
\let\pgfmathfloatmax=\pgfmathfloatmax@

\def\pgfmathfloatmin@#1#2{%
	\pgfmathfloatlessthan{#1}{#2}%
	\ifpgfmathfloatcomparison
		\edef\pgfmathresult{#1}%
	\else
		\edef\pgfmathresult{#2}%
	\fi
}
\let\pgfmathfloatmin=\pgfmathfloatmin@

% Renormalizes #1 to extended precision mantisse, meaning
% 100 <= m < 1000
% instead of 1 <= m < 10.
% 
% The 'extended precision' means we have higher accuracy when we apply pgfmath operations to mantisses.
%
% The input argument is expected to be a normalized floating point number; the output argument is a non-normalized floating point number (well, normalized to extended precision).
%
% The operation is supposed to be very fast.
%
% @see \pgfmathfloatsetextprecision
%
% There is a routine for internal usage,
% \pgfmathfloattoextentedprecision@a. It also provides exponent and
% sign of #1 in output arguments and may be used to increase speed.
\def\pgfmathfloattoextentedprecision#1{%
	\begingroup
	\pgfmathfloattoextentedprecision@a{#1}%
	\pgfmathfloatcreate{\pgfmathfloat@a@S}{\pgfmathresult}{\pgfmathfloat@a@E}%
	\pgfmath@smuggleone\pgfmathresult
	\endgroup
}%

\def\pgfmathfloattoextentedprecision@@zero#1\pgfmathfloat@EOI{%
	\edef\pgfmathresult{#1}%
}%
\def\pgfmathfloattoextentedprecision@@one#1.#2#3\pgfmathfloat@EOI{%
	\edef\pgfmathresult{#1#2.#3}%
}%
\def\pgfmathfloattoextentedprecision@@two#1.#2#3#4\pgfmathfloat@EOI{%
	\edef\pgfmathresult{#1#2#3.#4}%
}%
\def\pgfmathfloattoextentedprecision@@three#1.#2#3#4#5\pgfmathfloat@EOI{%
	\edef\pgfmathresult{#1#2#3#4.#5}%
}%

% Sets extended precision to 10^#1.
%
% The different choices are
% 
% - 0:  normalization      0 <= m < 1 (disable extended precision)
% - 1:  normalization     10 <= m < 100
% - 2:  normalization    100 <= m < 1000  (default)
% - 3:  normalization   1000 <= m < 10000
%
% #1 is the exponent, #1 = 0,1,2 or 3.
%
% This setting applies to \pgfmathfloattoextentedprecision and friends.
\def\pgfmathfloatsetextprecision#1{%
	\ifcase#1\relax
		\let\pgfmathfloattoextentedprecision@@=\pgfmathfloattoextentedprecision@@zero
		\def\pgfmathfloatextprec@shift{0}%
	\or
		\let\pgfmathfloattoextentedprecision@@=\pgfmathfloattoextentedprecision@@one
		\def\pgfmathfloatextprec@shift{1}%
	\or
		\let\pgfmathfloattoextentedprecision@@=\pgfmathfloattoextentedprecision@@two
		\def\pgfmathfloatextprec@shift{2}%
	\else
		\let\pgfmathfloattoextentedprecision@@=\pgfmathfloattoextentedprecision@@three
		\def\pgfmathfloatextprec@shift{3}%
	\fi
}%
\pgfmathfloatsetextprecision{2}%

% Does the "hard" work for \pgfmathfloattoextentedprecision. It
% provides additional outputs.
%
% INPUT:
% #1  normalized floating point number. Maybe a macro (it will be expanded ONCE)
% 
% OUTPUT:
% - \pgfmathresult : the mantisse in extended precision
% - \pgfmathfloat@a@S : the sign of #1
% - \pgfmathfloat@a@E : the exponent of #1, adjusted for extended precision
% - \pgfmathfloat@a@Mtok : undefined (its contents will be destroyed.
%
\def\pgfmathfloattoextentedprecision@a#1{%
	\expandafter\pgfmathfloat@decompose@tok#1\relax\pgfmathfloat@a@S\pgfmathfloat@a@Mtok\pgfmathfloat@a@E
	\ifnum\pgfmathfloat@a@S<3
		\advance\pgfmathfloat@a@E by-\pgfmathfloatextprec@shift\relax% compensate for shift
		\expandafter\pgfmathfloattoextentedprecision@@\the\pgfmathfloat@a@Mtok 000\pgfmathfloat@EOI
	\else
		\edef\pgfmathresult{#1}%
	\fi
}%


% Similar to \pgfmathfloattoextentedprecision@a, this one here fills the '@b' registers.
\def\pgfmathfloattoextentedprecision@b#1{%
	\expandafter\pgfmathfloat@decompose@tok#1\relax\pgfmathfloat@b@S\pgfmathfloat@a@Mtok\pgfmathfloat@b@E
	\ifnum\pgfmathfloat@b@S<3
		\advance\pgfmathfloat@b@E by-\pgfmathfloatextprec@shift\relax
		\expandafter\pgfmathfloattoextentedprecision@@\the\pgfmathfloat@a@Mtok 00\pgfmathfloat@EOI
	\else
		\edef\pgfmathresult{#1}%
	\fi
}%

% Addition of two floating point numbers using 8 significant digits.
\def\pgfmathfloatadd@#1#2{%
	\begingroup
	%
	% renormalize argument to 100 <= m < 1000 for extended accuracy:
	\pgfmathfloattoextentedprecision@a{#1}%
	\let\pgfmathfloat@arga=\pgfmathresult
	%
	\pgfmathfloattoextentedprecision@b{#2}%
	\let\pgfmathfloat@argb=\pgfmathresult
	%
	\pgfmathfloatcomparisontrue% re-use this boolean here to handle special cases.
	\ifcase\pgfmathfloat@a@S
		\edef\pgfmathresult{#2}%
		\pgfmathfloatcomparisonfalse
	\or
	\or
		\edef\pgfmathfloat@arga{-\pgfmathfloat@arga}%
	\else
		\pgfmathfloatcomparisonfalse
		\pgfmathfloatcreate{\the\pgfmathfloat@a@S}{0.0}{0}%
	\fi
	\ifcase\pgfmathfloat@b@S
		\edef\pgfmathresult{#1}%
		\pgfmathfloatcomparisonfalse
	\or
	\or
		\edef\pgfmathfloat@argb{-\pgfmathfloat@argb}%
	\else
		\pgfmathfloatcomparisonfalse
		\pgfmathfloatcreate{\the\pgfmathfloat@b@S}{0.0}{0}%
	\fi
	\ifpgfmathfloatcomparison
		% Shift lesser mantisse to fit the larger one:
		\ifnum\pgfmathfloat@a@E<\pgfmathfloat@b@E
			\pgfmathfloatadd@shift{\pgfmathfloat@arga}{\pgfmathfloat@a@E}{\pgfmathfloat@b@E}%
		\else
			\pgfmathfloatadd@shift{\pgfmathfloat@argb}{\pgfmathfloat@b@E}{\pgfmathfloat@a@E}%
		\fi
		% add them!
		\pgfmath@basic@add@{\pgfmathfloat@arga}{\pgfmathfloat@argb}%
		% renormalize sum. This is the only part were an expensive routine comes into play:
		\edef\pgfmathresult{\pgfmathresult e\the\pgfmathfloat@a@E}%
		\expandafter\pgfmathfloatqparsenumber\expandafter{\pgfmathresult}%
	\fi
	\pgfmath@smuggleone\pgfmathresult
	\endgroup
}%

% #1= floating point number
% #2= TeX code to execute if #1 == 0
% #3= TeX code to execute if #1 != 0
\def\pgfmathfloatifzero#1#2#3{%
	\expandafter\pgfmathfloat@decompose@F#1\relax\pgfmathfloat@a@S
	\ifnum\pgfmathfloat@a@S=0 #2\else#3\fi
}%

% #1=mantisse which needs to be shifted (with smaller exponent)
% #2=smaller exponent
% #3=larger exponent
% 
% ATTENTION: this helper method DESTROYS contents of \pgfmathfloat@a@S.
\def\pgfmathfloatadd@shift#1#2#3{%
	\pgf@xa=#1 pt%
	\pgfmathfloat@a@S=#3\relax
	\advance\pgfmathfloat@a@S by-#2\relax
	\ifcase\pgfmathfloat@a@S
	\or
		\divide\pgf@xa by10\relax
	\or
		\divide\pgf@xa by100\relax
	\or
		\divide\pgf@xa by1000\relax
	\or
		\divide\pgf@xa by10000\relax
	\or
		\divide\pgf@xa by10000\relax
		\divide\pgf@xa by10\relax
	\or
		\divide\pgf@xa by10000\relax
		\divide\pgf@xa by100\relax
	\or
		\divide\pgf@xa by10000\relax
		\divide\pgf@xa by1000\relax
	\or
		\divide\pgf@xa by10000\relax
		\divide\pgf@xa by10000\relax
	\else
		\pgf@xa=0pt%
	\fi
	#2=#3\relax
	\edef#1{\pgf@sys@tonumber\pgf@xa}%
}

\let\pgfmathfloatadd=\pgfmathfloatadd@


% Subtracts two floating point numbers.
\def\pgfmathfloatsubtract@#1#2{%
	\begingroup
	\expandafter\pgfmathfloat@decompose@tok#2\relax\pgfmathfloat@b@S\pgfmathfloat@a@Mtok\pgfmathfloat@b@E
	\ifcase\pgfmathfloat@b@S
		\edef\pgfmathresult{#1}%
	\or
		\pgfmathfloatcreate{2}{\the\pgfmathfloat@a@Mtok}{\the\pgfmathfloat@b@E}%
		\let\pgfmathfloatsub@arg=\pgfmathresult
		\pgfmathfloatadd@{#1}{\pgfmathfloatsub@arg}%
	\or
		\pgfmathfloatcreate{1}{\the\pgfmathfloat@a@Mtok}{\the\pgfmathfloat@b@E}%
		\let\pgfmathfloatsub@arg=\pgfmathresult
		\pgfmathfloatadd@{#1}{\pgfmathfloatsub@arg}%
	\else
		\pgfmathfloatcreate{\the\pgfmathfloat@b@S}{0.0}{0}%
	\fi
	\pgfmath@smuggleone\pgfmathresult
	\endgroup
}%

\let\pgfmathfloatsubtract=\pgfmathfloatsubtract@

% Scales a floating point number #1 with a fixed point number #2 using pgfmathmultiply.
% 
% Use this method if #2 is small number.
\def\pgfmathfloatmultiplyfixed@#1#2{%
	\begingroup
	%
	% renormalize argument to 100 <= m < 1000 for extended accuracy:
	\pgfmathfloattoextentedprecision@a{#1}%
	\let\pgfmathfloat@arga=\pgfmathresult
	%
	\pgfmathfloatcomparisontrue% re-use this boolean here to handle special cases.
	\ifcase\pgfmathfloat@a@S
		\edef\pgfmathresult{#1}%
		\pgfmathfloatcomparisonfalse
	\or
	\or
		\edef\pgfmathfloat@arga{-\pgfmathfloat@arga}%
	\else
		\pgfmathfloatcomparisonfalse
		\pgfmathfloatcreate{\the\pgfmathfloat@a@S}{0.0}{0}%
	\fi
	\ifpgfmathfloatcomparison
		\pgfmath@basic@multiply@{\pgfmathfloat@arga}{#2}%
		% renormalize product. This is the only part were an expensive routine comes into play:
		\edef\pgfmathresult{\pgfmathresult e\the\pgfmathfloat@a@E}%
		\expandafter\pgfmathfloatqparsenumber\expandafter{\pgfmathresult}%
	\fi
	\pgfmath@smuggleone\pgfmathresult
	\endgroup
}%

\let\pgfmathfloatmultiplyfixed=\pgfmathfloatmultiplyfixed@


\def\pgfmathfloatmultiply@#1#2{%
	\begingroup
	\pgfmathfloatsetextprecision{1}%
	\edef\pgfmathfloat@arga{#1}%
	\pgfmathfloattoextentedprecision@a{\pgfmathfloat@arga}%
	\let\pgfmathfloat@arga=\pgfmathresult
	%
	\edef\pgfmathfloat@argb{#2}%
	\pgfmathfloattoextentedprecision@b{\pgfmathfloat@argb}%
	\let\pgfmathfloat@argb=\pgfmathresult
	%
	\pgfmathfloatcomparisontrue% re-use this boolean here to handle special cases.
	\ifcase\pgfmathfloat@a@S
	% 0
		\pgfmathfloatcreate{0}{0.0}{0}%
		\pgfmathfloatcomparisonfalse
	\or% +
		\ifcase\pgfmathfloat@b@S
			\pgfmathfloatcreate{0}{0.0}{0}%
			\pgfmathfloatcomparisonfalse
		\or
			\def\pgfmathresult@S{1}%
		\or
			\def\pgfmathresult@S{2}%
		\else
			\expandafter\pgfmathfloatcreate\the\pgfmathfloat@b@S{0.0}{0}%
			\pgfmathfloatcomparisonfalse
		\fi
	\or% -
		\ifcase\pgfmathfloat@b@S
			\pgfmathfloatcreate{0}{0.0}{0}%
			\pgfmathfloatcomparisonfalse
		\or
			\def\pgfmathresult@S{2}%
		\or
			\def\pgfmathresult@S{1}%
		\or
			\pgfmathfloatcreate{3}{0.0}{0}%
			\pgfmathfloatcomparisonfalse
		\or
			\pgfmathfloatcreate{5}{0.0}{0}%
			\pgfmathfloatcomparisonfalse
		\or
			\pgfmathfloatcreate{4}{0.0}{0}%
			\pgfmathfloatcomparisonfalse
		\fi
	\or% nan
		\pgfmathfloatcreate{3}{0.0}{0}%
		\pgfmathfloatcomparisonfalse
	\or% +infty
		\ifcase\pgfmathfloat@b@S
			\pgfmathfloatcreate{0}{0.0}{0}%
			\pgfmathfloatcomparisonfalse
		\or
			\pgfmathfloatcreate{4}{0.0}{0}%
			\pgfmathfloatcomparisonfalse
		\or
			\pgfmathfloatcreate{5}{0.0}{0}%
			\pgfmathfloatcomparisonfalse
		\or
			\pgfmathfloatcreate{3}{0.0}{0}%
			\pgfmathfloatcomparisonfalse
		\or
			\pgfmathfloatcreate{4}{0.0}{0}%
			\pgfmathfloatcomparisonfalse
		\or
			\pgfmathfloatcreate{5}{0.0}{0}%
			\pgfmathfloatcomparisonfalse
		\fi
	\or% -infty
		\ifcase\pgfmathfloat@b@S
			\pgfmathfloatcreate{0}{0.0}{0}%
			\pgfmathfloatcomparisonfalse
		\or
			\pgfmathfloatcreate{5}{0.0}{0}%
			\pgfmathfloatcomparisonfalse
		\or
			\pgfmathfloatcreate{4}{0.0}{0}%
			\pgfmathfloatcomparisonfalse
		\or
			\pgfmathfloatcreate{3}{0.0}{0}%
			\pgfmathfloatcomparisonfalse
		\or
			\pgfmathfloatcreate{5}{0.0}{0}%
			\pgfmathfloatcomparisonfalse
		\or
			\pgfmathfloatcreate{4}{0.0}{0}%
			\pgfmathfloatcomparisonfalse
		\fi
	\fi
	\ifpgfmathfloatcomparison
		\pgfmath@basic@multiply@{\pgfmathfloat@arga}{\pgfmathfloat@argb}%
		\advance\pgfmathfloat@a@E by\pgfmathfloat@b@E
		% renormalize sum. This is the only part were an expensive routine comes into play:
		\edef\pgfmathresult{\pgfmathresult e\the\pgfmathfloat@a@E}%
		\expandafter\pgfmathfloatqparsenumber\expandafter{\pgfmathresult}%
		\expandafter\pgfmathfloat@decompose@tok\pgfmathresult\relax\pgfmathfloat@a@S\pgfmathfloat@a@Mtok\pgfmathfloat@a@E
		\pgfmathfloatcreate{\pgfmathresult@S}{\the\pgfmathfloat@a@Mtok}{\the\pgfmathfloat@a@E}%
	\fi
	\pgfmath@smuggleone\pgfmathresult
	\endgroup
}%
\let\pgfmathfloatmultiply=\pgfmathfloatmultiply@

% Defines \pgfmathresult to be #1 / #2 for two floating point numbers.
%
% It employs the basic math engine internally to divide mantissas.
\def\pgfmathfloatdivide@#1#2{%
	\begingroup
	\pgfmathfloatsetextprecision{1}% is not too important, I think. After all, 0.1 <= #1/#2 < 10 or so due to normalization (no matter, which)
	\edef\pgfmathfloat@arga{#1}%
	\pgfmathfloattoextentedprecision@a{\pgfmathfloat@arga}%
	\let\pgfmathfloat@arga=\pgfmathresult
	%
	\edef\pgfmathfloat@argb{#2}%
	\pgfmathfloattoextentedprecision@b{\pgfmathfloat@argb}%
	\let\pgfmathfloat@argb=\pgfmathresult
	%
	\pgfmathfloatcomparisontrue% re-use this boolean here to handle special cases.
	\ifcase\pgfmathfloat@a@S
	% 0
		\pgfmathfloatcreate{0}{0.0}{0}%
		\pgfmathfloatcomparisonfalse
	\or% +
		\ifcase\pgfmathfloat@b@S
			\pgfmathfloatcreate{4}{0.0}{0}%
			\pgfmathfloatcomparisonfalse
		\or
			\def\pgfmathresult@S{1}%
		\or
			\def\pgfmathresult@S{2}%
		\or
			\pgfmathfloatcreate{3}{0.0}{0}%
			\pgfmathfloatcomparisonfalse
		\else
			\pgfmathfloatcreate{0}{0.0}{0}%
			\pgfmathfloatcomparisonfalse
		\fi
	\or% -
		\ifcase\pgfmathfloat@b@S
			\pgfmathfloatcreate{5}{0.0}{0}%
			\pgfmathfloatcomparisonfalse
		\or
			\def\pgfmathresult@S{2}%
		\or
			\def\pgfmathresult@S{1}%
		\or
			\pgfmathfloatcreate{3}{0.0}{0}%
			\pgfmathfloatcomparisonfalse
		\else
			\pgfmathfloatcreate{0}{0.0}{0}%
			\pgfmathfloatcomparisonfalse
		\fi
	\or% nan
		\pgfmathfloatcreate{3}{0.0}{0}%
		\pgfmathfloatcomparisonfalse
	\or% +infty
		\ifcase\pgfmathfloat@b@S
			\pgfmathfloatcreate{4}{0.0}{0}%
			\pgfmathfloatcomparisonfalse
		\or
			\pgfmathfloatcreate{4}{0.0}{0}%
			\pgfmathfloatcomparisonfalse
		\or
			\pgfmathfloatcreate{5}{0.0}{0}%
			\pgfmathfloatcomparisonfalse
		\or
			\pgfmathfloatcreate{3}{0.0}{0}%
			\pgfmathfloatcomparisonfalse
		\or
			\pgfmathfloatcreate{4}{0.0}{0}% what is inf/inf ?
			\pgfmathfloatcomparisonfalse
		\or
			\pgfmathfloatcreate{5}{0.0}{0}% or inf/-inf ?
			\pgfmathfloatcomparisonfalse
		\fi
	\or% -infty
		\ifcase\pgfmathfloat@b@S
			\pgfmathfloatcreate{5}{0.0}{0}%
			\pgfmathfloatcomparisonfalse
		\or
			\pgfmathfloatcreate{5}{0.0}{0}%
			\pgfmathfloatcomparisonfalse
		\or
			\pgfmathfloatcreate{4}{0.0}{0}%
			\pgfmathfloatcomparisonfalse
		\or
			\pgfmathfloatcreate{3}{0.0}{0}%
			\pgfmathfloatcomparisonfalse
		\or
			\pgfmathfloatcreate{5}{0.0}{0}%
			\pgfmathfloatcomparisonfalse
		\or
			\pgfmathfloatcreate{4}{0.0}{0}%
			\pgfmathfloatcomparisonfalse
		\fi
	\fi
	\ifpgfmathfloatcomparison
		\pgfmath@basic@divide@{\pgfmathfloat@arga}{\pgfmathfloat@argb}%
		\advance\pgfmathfloat@a@E by-\pgfmathfloat@b@E
		% renormalize. This is the only part were an expensive float routine comes into play:
		\edef\pgfmathresult{\pgfmathresult e\the\pgfmathfloat@a@E}%
		\expandafter\pgfmathfloatqparsenumber\expandafter{\pgfmathresult}%
		% And re-insert the proper sign:
		\expandafter\pgfmathfloat@decompose@tok\pgfmathresult\relax\pgfmathfloat@a@S\pgfmathfloat@a@Mtok\pgfmathfloat@a@E
		\pgfmathfloatcreate{\pgfmathresult@S}{\the\pgfmathfloat@a@Mtok}{\the\pgfmathfloat@a@E}%
	\fi
	\pgfmath@smuggleone\pgfmathresult
	\endgroup
}%
\let\pgfmathfloatdivide=\pgfmathfloatdivide@

\def\pgfmathfloatreciprocal@#1{%
	\begingroup
	% FIXME optimize
	\edef\pgfmathfloat@loc@TMPa{#1}%
	\pgfmathfloatcreate{1}{1.0}{0}%
	\pgfmathfloatdivide@{\pgfmathresult}{\pgfmathfloat@loc@TMPa}%
	\pgfmath@smuggleone\pgfmathresult
	\endgroup
}%

% Computes sqrt(#1) in floating point arithmetics.
%
% It employs sqrt( m * 10^e ) = sqrt(m) * sqrt(10^e).
\def\pgfmathfloatsqrt@#1{%
	\begingroup
	\pgfmathfloatsetextprecision{3}%
	\edef\pgfmathfloat@arga{#1}%
	\pgfmathfloattoextentedprecision@a{\pgfmathfloat@arga}%
	\let\pgfmathfloat@arga=\pgfmathresult
	%
	\ifcase\pgfmathfloat@a@S
	% 0
		\pgfmathfloatcreate{0}{0.0}{0}%
	\or% +
		\pgfmath@basic@sqrt@{\pgfmathfloat@arga}%
		\ifodd\pgfmathfloat@a@E
			\ifnum\pgfmathfloat@a@E>0
				\expandafter\pgfmathmultiply@\expandafter{\pgfmathresult}{3.16227766}% * sqrt(10)
			\else
				\expandafter\pgfmathmultiply@\expandafter{\pgfmathresult}{0.316227766}% * sqrt(0.1)
			\fi
		\fi
		\divide\pgfmathfloat@a@E by2 % sqrt(10^e) = 10^{e/2} (see above for odd e)
		% renormalize sum. This is the only part were an expensive routine comes into play:
		\edef\pgfmathfloat@arga{\pgfmathresult e\the\pgfmathfloat@a@E}%
		\pgfmathfloatqparsenumber{\pgfmathfloat@arga}%
	\or% -
		\pgfmathfloatcreate{3}{0.0}{0}%
	\or% nan
		\pgfmathfloatcreate{3}{0.0}{0}%
	\or% +infty
		\pgfmathfloatcreate{4}{0.0}{0}%
	\or% -infty
		\pgfmathfloatcreate{3}{0.0}{0}%
	\fi
	\pgfmath@smuggleone\pgfmathresult
	\endgroup
}%
\let\pgfmathfloatsqrt=\pgfmathfloatsqrt@

% Returns the integer part of the floating point number #1.
%
% The result is returned as floating point as well. 
%
% This operation is not limited to TeX's range of count registers (it
% works symbolly)
\def\pgfmathfloatint@#1{%
	\begingroup
	\edef\pgfmathresult{#1}%
	\expandafter\pgfmathfloat@decompose@tok\pgfmathresult\relax\pgfmathfloat@a@S\pgfmathfloat@a@Mtok\pgfmathfloat@a@E
	\ifcase\pgfmathfloat@a@S
		% 0: nothing to do.
	\or% +
		\expandafter\pgfmathfloatint@@\the\pgfmathfloat@a@Mtok\pgfmathfloat@EOI
		\pgfmathfloatcreate{\the\pgfmathfloat@a@S}{\the\pgfmathfloat@a@Mtok}{\the\pgfmathfloat@a@E}%
	\or% -
		\expandafter\pgfmathfloatint@@\the\pgfmathfloat@a@Mtok\pgfmathfloat@EOI
		\pgfmathfloatcreate{\the\pgfmathfloat@a@S}{\the\pgfmathfloat@a@Mtok}{\the\pgfmathfloat@a@E}%
	\else
		% nothing to do
	\fi
	\pgfmath@smuggleone\pgfmathresult
	\endgroup
}%
\def\pgfmathfloatint@@#1.{%
	\ifnum\pgfmathfloat@a@E<0
		\pgfmathfloat@a@S=0
		\pgfmathfloat@a@Mtok={0.0}%
		\pgfmathfloat@a@E=0
		\expandafter\pgfmathfloatint@@loop@gobble
	\else
		\pgfmathfloat@a@Mtok={#1.}%
		\pgfmathfloat@b@E=\pgfmathfloat@a@E
		\expandafter\pgfmathfloatint@@loop
	\fi
}%
\def\pgfmathfloatint@@loop#1{%
	\def\pgfmathfloatint@@loop@{#1}%
	\ifx\pgfmathfloatint@@loop@\pgfmathfloat@EOI
		\let\pgfmathfloatint@@loop@next=\relax
	\else
		\ifnum\pgfmathfloat@b@E=0
			\let\pgfmathfloatint@@loop@next=\pgfmathfloatint@@loop@gobble
		\else
			\pgfmathfloat@a@Mtok=\expandafter{\the\pgfmathfloat@a@Mtok#1}%
			\advance\pgfmathfloat@b@E by-1
			\let\pgfmathfloatint@@loop@next=\pgfmathfloatint@@loop
		\fi
	\fi
	\pgfmathfloatint@@loop@next
}%
\def\pgfmathfloatint@@loop@gobble#1\pgfmathfloat@EOI{}%
\let\pgfmathfloatint=\pgfmathfloatint@
\let\pgfmathfloatfloor=\pgfmathfloatint
\let\pgfmathfloatfloor@=\pgfmathfloatint@

\def\pgfmathfloat@install@unimplemented#1{%
	\expandafter\def\csname pgfmath#1@\endcsname##1{\pgfmathfloat@notimplemented{#1}}%
}
\def\pgfmathfloat@notimplemented#1{%
	\pgfmath@error{Sorry, the operation '#1' has not yet been implemented in the floating point unit :-(}{}%
	\pgfmathfloatcreate{0}{0.0}{0}%
}%

% Divides or multiplies the input number by 10^#4 using an arithmetic
% left/right shift.
%
% Input:
% #1 a normalised floating point number.
% #2 a positive or negative integer number denoting the shift.
%
% Example:
% \pgfmathfloatshift{11e3}{4}%
% -> pgfmathresult = 11e7
\def\pgfmathfloatshift@#1#2{%
	\begingroup
	\expandafter\pgfmathfloat@decompose@tok#1\relax\pgfmathfloat@a@S\pgfmathfloat@a@Mtok\pgfmathfloat@a@E
	\advance\pgfmathfloat@a@E by#2\relax
	\pgfmathfloatcreate{\the\pgfmathfloat@a@S}{\the\pgfmathfloat@a@Mtok}{\the\pgfmathfloat@a@E}%
	\pgfmath@smuggleone\pgfmathresult
	\endgroup
}
\let\pgfmathfloatshift=\pgfmathfloatshift@

% Defines \pgfmathresult to be |#1|, the absolute value of the
% normalized floating point number #1.
\def\pgfmathfloatabs@#1{%
	\begingroup
	\expandafter\pgfmathfloat@decompose@tok#1\relax\pgfmathfloat@a@S\pgfmathfloat@a@Mtok\pgfmathfloat@a@E
	\ifcase\pgfmathfloat@a@S
		% 0: do nothing.
	\or
		% +: ok, is positive.
	\or
		% -: multiply with -1:
		\pgfmathfloat@a@S=1
	\or
		% nan: do nothing.
	\or
		% +infty: ok.
	\or
		% -infty: multiply with -1:
		\pgfmathfloat@a@S=4
	\fi
	\pgfmathfloatcreate{\the\pgfmathfloat@a@S}{\the\pgfmathfloat@a@Mtok}{\the\pgfmathfloat@a@E}%
	\pgfmath@smuggleone\pgfmathresult
	\endgroup
}%
\let\pgfmathfloatabs=\pgfmathfloatabs@

% Computes the absolute error |#1 - #2| into \pgfmathresult.
\def\pgfmathfloatabserror@#1#2{%
	\pgfmathfloatsubtract@{#1}{#2}%
	\pgfmathfloatabs@{\pgfmathresult}%
}%
\let\pgfmathfloatabserror=\pgfmathfloatabserror@

% Computes the relative error |#1 - #2|/|#2| into \pgfmathresult,
% assuming #2 != 0.
\def\pgfmathfloatrelerror@#1#2{%
	\pgfmathfloatsubtract@{#1}{#2}%
	\pgfmathfloatdivide@{\pgfmathresult}{#2}%
	\pgfmathfloatabs@{\pgfmathresult}%
}%
\let\pgfmathfloatrelerror=\pgfmathfloatrelerror@

% Computes \pgfmathresult = #1 mod #2 using truncated division.
%
\def\pgfmathfloatmod@#1#2{%
	\begingroup
	\pgfmathfloatdivide@{#1}{#2}%
	\pgfmathfloatint@{\pgfmathresult}%
	\let\pgfmathfloat@loc@TMPa=\pgfmathresult
	\pgfmathfloatmultiply@{\pgfmathfloat@loc@TMPa}{#2}%
	\let\pgfmathfloat@loc@TMPb=\pgfmathresult
	\pgfmathfloatsubtract@{#1}{\pgfmathfloat@loc@TMPb}%
	\pgfmath@smuggleone\pgfmathresult
	\endgroup
}
\let\pgfmathfloatmod=\pgfmathfloatmod@

% A modification of \pgfmathfloatmod@ where #3 = 1/#2 is already
% known. This may be faster.
\def\pgfmathfloatmodknowsinverse@#1#2#3{%
	\begingroup
	\pgfmathfloatmultiply@{#1}{#3}%
	\pgfmathfloatint@{\pgfmathresult}%
	\let\pgfmathfloat@loc@TMPa=\pgfmathresult
	\pgfmathfloatmultiply@{\pgfmathfloat@loc@TMPa}{#2}%
	\let\pgfmathfloat@loc@TMPb=\pgfmathresult
	\pgfmathfloatsubtract@{#1}{\pgfmathfloat@loc@TMPb}%
	\pgfmath@smuggleone\pgfmathresult
	\endgroup
}
\let\pgfmathfloatmodknowsinverse=\pgfmathfloatmodknowsinverse@

\def\pgfmathfloatpi@{%
	\pgfmathfloatcreate{1}{3.14159265358979}{0}%
}%
\let\pgfmathfloatpi=\pgfmathfloatpi@

\def\pgfmathfloate@{%
	\pgfmathfloatcreate{1}{2.71828182845905}{0}%
}
\let\pgfmathfloate=\pgfmathfloate@

% Converts #1 from radians to degrees.
\def\pgfmathfloatdeg@#1{%
	\expandafter\ifx\csname pgfmfltdeg@factor\endcsname\relax
		% Lazy evaluation:
		\pgfmathfloatcreate{1}{5.72957795130823}{1}%
		\global\let\pgfmfltdeg@factor=\pgfmathresult
	\fi
	\pgfmathfloatmultiply@{#1}\pgfmfltdeg@factor%
}
\let\pgfmathfloatdeg=\pgfmathfloatdeg@

% Converts #1 from degree to radians.
\def\pgfmathfloatrad@#1{%
	\expandafter\ifx\csname pgfmfltrad@factor\endcsname\relax
		% Lazy evaluation:
		\pgfmathfloatcreate{1}{1.74532925199433}{-2}%
		\global\let\pgfmfltrad@factor=\pgfmathresult
	\fi
	\pgfmathfloatmultiply@{#1}\pgfmfltrad@factor%
}
\let\pgfmathfloatrad=\pgfmathfloatrad@

% Computes #1(#2) where #1 is a trigonometric function, i.e.
% #1(#2) = #1( #2 + r*360 )
%
% #1 is a one-argument macro which assigns \pgfmathresult.
\def\pgfmathfloatTRIG@#1#2{%
	\expandafter\ifx\csname pgfmathfloatTRIG@NUM\endcsname\relax% why not directly!??? Did not work!??
		% Lazy evaluation:
		\pgfmathfloatcreate{1}{3.6}{2}%
		\global\let\pgfmathfloatTRIG@NUM=\pgfmathresult
		\pgfmathfloatcreate{1}{2.77777777777778}{-3}%
		\global\let\pgfmathfloatTRIG@NUM@INV=\pgfmathresult
	\fi
	\pgfmathfloatmodknowsinverse@{#2}{\pgfmathfloatTRIG@NUM}{\pgfmathfloatTRIG@NUM@INV}%
	\pgfmathfloattofixed@{\pgfmathresult}%
	\expandafter#1\expandafter{\pgfmathresult}%
	\pgfmathfloatparsenumber{\pgfmathresult}%
}%

\def\pgfmathfloatsin@#1{\pgfmathfloatTRIG@\pgfmath@basic@sin@{#1}}
\let\pgfmathfloatsin=\pgfmathfloatsin@
\def\pgfmathfloatcos@#1{\pgfmathfloatTRIG@\pgfmath@basic@cos@{#1}}
\let\pgfmathfloatcos=\pgfmathfloatcos@
\def\pgfmathfloattan@#1{\pgfmathfloatTRIG@\pgfmath@basic@tan@{#1}}
\let\pgfmathfloattan=\pgfmathfloattan@
\def\pgfmathfloatsec@#1{\pgfmathfloatTRIG@\pgfmath@basic@sec@{#1}}
\let\pgfmathfloatsec=\pgfmathfloatsec@
\def\pgfmathfloatcosec@#1{\pgfmathfloatTRIG@\pgfmath@basic@cosec@{#1}}
\let\pgfmathfloatcosec=\pgfmathfloatcosec@
\def\pgfmathfloatcot@#1{\pgfmathfloatTRIG@\pgfmath@basic@cot@{#1}}
\let\pgfmathfloatcot=\pgfmathfloatcot@
\def\pgfmathfloatasin@#1{\pgfmathfloatTRIG@\pgfmath@basic@asin@{#1}}
\let\pgfmathfloatasin=\pgfmathfloatasin@
\def\pgfmathfloatacos@#1{\pgfmathfloatTRIG@\pgfmath@basic@acos@{#1}}
\let\pgfmathfloatacos=\pgfmathfloatacos@
\def\pgfmathfloatatan@#1{\pgfmathfloatTRIG@\pgfmath@basic@atan@{#1}}
\let\pgfmathfloatatan=\pgfmathfloatatan@

% Expands #2 using \edef and invokes #1 with the resulting string.
%
% DEPRECATED
% Example:
%   \pgfmath@y=7.9pt
%   \pgfmathlog@invoke@expanded\pgfmathexp@{{\pgf@sys@tonumber{\pgfmath@y}}}%
% will invoke
%   \pgfmathexp@{7.9}
\def\pgfmathlog@invoke@expanded#1#2{%
	\edef\pgfmath@resulttemp{#2}%
	\expandafter#1\pgfmath@resulttemp
}

\def\pgfmathfloatln@#1{%
	\pgfmathlog@float{#1}%
	\pgfmathfloatparsenumber{\pgfmathresult}%
}
\let\pgfmathfloatln=\pgfmathfloatln@

% Computes log(x) into \pgfmathresult.
%
% This allows numbers such at 10000000 or 5.23e-10 to be represented
% properly, although TeX-registers would produce overflow/underflow
% errors in these cases. 
%
% The natural logarithm is computed using log(X*10^Y) = log(X) + log(10)*Y
%
% FIXME This routine is only kept for backwards compatibility!
% It does not work as expected because
% 1. it calls \pgfmathfloatparsenumber
% 2. it returns the result as fixed point number
% Use \pgfmathln@ instead!
\def\pgfmathlog@#1{%
	\pgfmathfloatparsenumber{#1}%
	\pgfmathlog@float{\pgfmathresult}%
}
\let\pgfmathlog=\pgfmathlog@
\def\pgfmathlog@float#1{%
	\begingroup%
		% compute #1 = M*10^E with normalised mantisse M = [+-]*[1-9].XXXXX
		\expandafter\pgfmathfloat@decompose@tok#1\relax\pgfmathfloat@a@S\pgfmathfloat@a@Mtok\pgfmathfloat@a@E
		\ifnum\pgfmathfloat@a@S=1
			% Now, compute log(#1) = log(M) + E*log(10)
			\expandafter\pgfmath@basic@ln@\expandafter{\the\pgfmathfloat@a@Mtok}%
			\pgfmathfloat@b@M=\pgfmathresult pt%
			\pgfmathfloat@a@M=2.302585pt% = log(10)
			\multiply\pgfmathfloat@a@M by\pgfmathfloat@a@E\relax
			\advance\pgfmathfloat@b@M by\pgfmathfloat@a@M
			\edef\pgfmathresult{\pgf@sys@tonumber{\pgfmathfloat@b@M}}%
		\else
			\let\pgfmathresult=\pgfutil@empty%
		\fi
	\pgfmath@smuggleone\pgfmathresult
	\endgroup%
}


% Computes exp(#1) in floating point.
%
% The algorithm employs the identity 
%  exp(x) = exp(x - log(10^k) + log(10^k)
%         = 10^k exp( x - k*log 10 )
% with k choosen such that exp( x - k*log10) can be computed with the
% basic level math engine.
%
% The precision (relative error) is between 10^{-4} and 10^{-6}. For
% #1 = 700, it is even 10^{-3}. I will need to improve that someday.
\def\pgfmathfloatexp@#1{%
	\begingroup
		\expandafter\pgfmathfloat@decompose@tok#1\relax\pgfmathfloat@a@S\pgfmathfloat@a@Mtok\pgfmathfloat@a@E
		\ifcase\pgfmathfloat@a@S
		% #1 = 0:
			\pgfmathfloatcreate{1}{1.0}{0}%
		\or% #1 > 0
			\pgfmathfloatexp@@{#1}%
		\or% #1 < 0
			\pgfmathfloatexp@@{#1}%
		\else
			\edef\pgfmathresult{#1}%
		\fi
		\pgfmath@smuggleone\pgfmathresult
	\endgroup
}%
\def\pgfmathfloatexp@@#1{%
	% Employ the identity
	% exp(x) = exp(x - log(10^k) + log(10^k)) = 10^k exp( x - k *log(10))
	%
	% I'd like to have x - k*log(10) <= 1
	% => compute k := int( (x - 1) * 1/log(10) )
	% that should suffice since \pgfmathexp@ should be
	% accurate enough for those numbers.
	%
	% please note that we can do all this in TeX registers.
	% exp(700) is almost the maximum of double precision
	% anyway, and exp(16000) is certainly the largest we will
	% ever need.
	\pgfmathfloattofixed@{#1}%
	\pgf@xa=\pgfmathresult pt
	\pgf@xa=0.434294481\pgf@xa\relax
	\edef\pgfmathfloat@loc@TMPa{\pgf@sys@tonumber{\pgf@xa}}%
	\expandafter\pgfmathfloatexp@@toint\pgfmathfloat@loc@TMPa\relax
	\pgf@xa=2.302585092pt
	\multiply\pgf@xa by-\pgfmathfloat@k\relax
	\advance\pgf@xa by\pgfmathresult pt
	\edef\pgfmathfloat@loc@TMPa{\pgf@sys@tonumber{\pgf@xa}}%
%\message{computing exp(\pgfmathresult) = 10^\pgfmathfloat@k * exp(\pgfmathfloat@loc@TMPa)...}%
	\pgfmath@basic@exp@{\pgfmathfloat@loc@TMPa}%
	\let\pgfmathfloat@loc@TMPa=\pgfmathresult
	\pgfmathfloatparsenumber{\pgfmathfloat@loc@TMPa}%
	\let\pgfmathfloat@loc@TMPa=\pgfmathresult
	\pgfmathfloatshift@{\pgfmathfloat@loc@TMPa}{\pgfmathfloat@k}%
}
% determine 'k'. This is a heuristics. The exponential series
% converges best for |x| <= 1. However, the fixed point arithmetics
% for tex results in best results for large |x|. Well, I'll need to
% tune this here.
\def\pgfmathfloatexp@@toint#1.#2\relax{%
	\c@pgf@counta=#1\relax
	\ifnum\c@pgf@counta<0
\advance\c@pgf@counta by-1 % FIXME . this is a test for optimizations.
		\c@pgf@countb=#2\relax
		\ifnum\c@pgf@countb>0
			\advance\c@pgf@counta by-1
		\fi
	\fi
	\edef\pgfmathfloat@k{\the\c@pgf@counta}%
}%
\let\pgfmathfloatexp=\pgfmathfloatexp@

\def\pgfmathfloatround@#1{%
	\begingroup
	\pgfkeysvalueof{/pgf/number format/precision/.@cmd}0\pgfeov
	\pgfmathfloatround{#1}%
	\ifpgfmathfloatroundhasperiod
	\else
		\expandafter\pgfmathfloat@decompose@tok\pgfmathresult\relax\pgfmathfloat@a@S\pgfmathfloat@a@Mtok\pgfmathfloat@a@E
		\pgfmathfloatcreate{\the\pgfmathfloat@a@S}{\the\pgfmathfloat@a@Mtok.0}{\the\pgfmathfloat@a@E}%
	\fi
	\pgfmath@smuggleone\pgfmathresult
	\endgroup
}%


\def\pgfmathfloatneg@#1{%
	\begingroup
	\expandafter\pgfmathfloat@decompose@tok#1\relax\pgfmathfloat@a@S\pgfmathfloat@a@Mtok\pgfmathfloat@a@E
	\ifcase\pgfmathfloat@a@S\relax
		% 0:
		\edef\pgfmathresult{#1}%
	\or
		% +:
		\pgfmathfloatcreate{2}{\the\pgfmathfloat@a@Mtok}{\the\pgfmathfloat@a@E}%
	\or
		% -:
		\pgfmathfloatcreate{1}{\the\pgfmathfloat@a@Mtok}{\the\pgfmathfloat@a@E}%
	\or
		% nan:
		\edef\pgfmathresult{#1}%
	\or
		% +infty:
		\pgfmathfloatcreate{5}{\the\pgfmathfloat@a@Mtok}{\the\pgfmathfloat@a@E}%
	\or
		% -infty:
		\pgfmathfloatcreate{4}{\the\pgfmathfloat@a@Mtok}{\the\pgfmathfloat@a@E}%
	\fi
	\pgfmath@smuggleone\pgfmathresult
	\endgroup
}%


\def\pgfmathfloatpow@#1#2{%
	\begingroup%
	\expandafter\pgfmathfloat@decompose@tok#2\relax\pgfmathfloat@a@S\pgfmathfloat@a@Mtok\pgfmathfloat@a@E
	\ifcase\pgfmathfloat@a@S\relax
		% #1 ^ 0 = 1
		\pgfmathfloatcreate{1}{1.0}{0}%
	\or
		% #2 > 0
		\pgfmathfloatpow@@{#1}{#2}%
	\or
		% #2 < 0
		\pgfmathfloatpow@@{#1}{#2}%
	\or
		% #2 = nan
		\edef\pgfmathresult{#2}%
	\or
		% #2 = inf
		\edef\pgfmathresult{#2}%
	\or
		% #2 = -inf
		\pgfmathfloatcreate{0}{0.0}{0}%
	\fi
	\pgfmath@smuggleone\pgfmathresult
	\endgroup
}%

% computes #1^#2
% PRECONDITIONS
% - #2 is positive.
\def\pgfmathfloatpow@@#1#2{%
	\pgfmathfloattofixed@{#2}%
	\afterassignment\pgfmath@x%
	\expandafter\c@pgfmath@counta\pgfmathresult pt\relax%
	\ifdim\pgfmath@x=0pt %
		% loop "manually"; we have an integer exponent!
		\ifnum\c@pgfmath@counta<0
			\pgfmathfloatreciprocal@{#1}%
			\let\pgfmathfloat@loc@TMPa=\pgfmathresult
			\c@pgfmath@counta=-\c@pgfmath@counta
		\else
			\edef\pgfmathfloat@loc@TMPa{#1}%
		\fi
		\pgfmathfloatcreate{1}{1.0}{0}%
		\let\pgfmathfloat@loc@TMPb=\pgfmathresult
		\pgfmathloop
			\ifnum\c@pgfmath@counta>0\relax%
				\ifodd\c@pgfmath@counta%
					\pgfmathfloatmultiply@{\pgfmathfloat@loc@TMPb}{\pgfmathfloat@loc@TMPa}%
					\let\pgfmathfloat@loc@TMPb=\pgfmathresult
				\fi
				\ifnum\c@pgfmath@counta>1\relax%
					\pgfmathfloatmultiply@{\pgfmathfloat@loc@TMPa}{\pgfmathfloat@loc@TMPa}%
					\let\pgfmathfloat@loc@TMPa=\pgfmathresult
				\fi%
				\divide\c@pgfmath@counta by2\relax%
		\repeatpgfmathloop%
	\else
		% employ #1^#2 = exp( #2 * ln(#1) )
		\pgfmathfloatln@{#1}%
		\let\pgfmathfloat@loc@TMPa=\pgfmathresult
		\edef\pgfmathfloat@loc@TMPb{#2}%
		\pgfmathfloatmultiply@{\pgfmathfloat@loc@TMPa}{\pgfmathfloat@loc@TMPb}%
		\pgfmathfloatexp@{\pgfmathresult}%
	\fi
}
\endinput
