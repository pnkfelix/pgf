\ProvidesPackageRCS[v\pgfversion] $Header: /cvsroot/pgf/pgf/generic/pgf/libraries/Attic/pgflibraryautomata.code.tex,v 1.1 2005/10/13 10:12:31 tantau Exp $

% Copyright 2005 by Till Tantau <tantau@cs.tu-berlin.de>.
%
% This program can be redistributed and/or modified under the terms
% of the GNU Public License, version 2.

\newbox\pgfnodepartoutputbox



%
% Node for states without output. This is just an alias for a circle
% node. 
%
\pgfdeclareshape{state}
{
  \inheritsavedanchors[from=circle] % this is a circle
  \inheritanchorborder[from=circle]
  \inheritanchor[from=circle]{north}
  \inheritanchor[from=circle]{north west}
  \inheritanchor[from=circle]{north east}
  \inheritanchor[from=circle]{center}
  \inheritanchor[from=circle]{west}
  \inheritanchor[from=circle]{east}
  \inheritanchor[from=circle]{mid}
  \inheritanchor[from=circle]{mid west}
  \inheritanchor[from=circle]{mid east}
  \inheritanchor[from=circle]{base}
  \inheritanchor[from=circle]{base west}
  \inheritanchor[from=circle]{base east}
  \inheritanchor[from=circle]{south}
  \inheritanchor[from=circle]{south west}
  \inheritanchor[from=circle]{south east}
  \inheritbackgroundpath[from=circle]
}


%
% Moore state node (a state with output).
%
% This node consists of two parts: The main part is the state
% name. The second part is the (optional) state output. This output is
% shown in the lower half of the node.
%
% Parts: text (=state name), output

\pgfdeclareshape{state with output}
{
  %
  % Node parts
  %
  \nodeparts{text,output}

  %
  % Anchors
  %
  \savedanchor\centerpoint{%
    \pgf@x=.5\wd\pgfnodeparttextbox%
    \setlength{\pgf@y}{\pgfshapeinnerysep}%
    \pgf@y=-\pgf@y%
    \advance\pgf@y by-\dp\pgfnodeparttextbox%
    \advance\pgf@y by-.5\pgflinewidth%
  }%
  \savedanchor\outputanchor{%
    \pgf@x=-.5\wd\pgfnodepartoutputbox%
    \advance\pgf@x by.5\wd\pgfnodeparttextbox%
    \setlength{\pgf@y}{\pgfshapeinnerysep}%
    \pgf@y=-2\pgf@y%
    \advance\pgf@y by-\ht\pgfnodepartoutputbox%
    \advance\pgf@y by-.5\pgflinewidth%
    \advance\pgf@y by-\dp\pgfnodeparttextbox%
    \advance\pgf@y by-.5\pgflinewidth%
  }
    
  \saveddimen\radius{%
    % 
    % Caculate ``height radius''
    % 
    \pgf@ya=.5\ht\pgfnodeparttextbox%
    \advance\pgf@ya by.5\dp\pgfnodeparttextbox%
    \advance\pgf@ya by.5\ht\pgfnodepartoutputbox%
    \advance\pgf@ya by.5\dp\pgfnodepartoutputbox%
    \advance\pgf@ya by.5\pgflinewidth%
    \setlength\pgf@yb{\pgfshapeinnerysep}%
    \advance\pgf@ya by2\pgf@yb%
    % 
    % Caculate ``width radius''
    % 
    \pgf@xa=.5\wd\pgfnodeparttextbox%
    \ifdim\pgf@xa<.5\wd\pgfnodepartoutputbox%
      \pgf@xa=.5\wd\pgfnodepartoutputbox%
    \fi%
    \setlength\pgf@xb{\pgfshapeinnerxsep}%
    \advance\pgf@xa by\pgf@xb%
    % 
    % Calculate length of radius vector:
    % 
    \pgf@process{\pgfpointnormalised{\pgfpoint{\pgf@xa}{\pgf@ya}}}%
    \ifdim\pgf@x>\pgf@y%
        \c@pgf@counta=\pgf@x%
        \ifnum\c@pgf@counta=0\relax%
        \else%
          \divide\c@pgf@counta by 255\relax%
          \pgf@xa=16\pgf@xa\relax%
          \divide\pgf@xa by\c@pgf@counta%
          \pgf@xa=16\pgf@xa\relax%
        \fi%
      \else%
        \c@pgf@counta=\pgf@y%
        \ifnum\c@pgf@counta=0\relax%
        \else%
          \divide\c@pgf@counta by 255\relax%
          \pgf@ya=16\pgf@ya\relax%
          \divide\pgf@ya by\c@pgf@counta%
          \pgf@xa=16\pgf@ya\relax%
        \fi%
    \fi%
    \pgf@x=\pgf@xa%
    % 
    % If necessary, adjust radius so that the size requirements are
    % met: 
    % 
    \setlength{\pgf@xb}{\pgfshapeminwidth}%  
    \setlength{\pgf@yb}{\pgfshapeminheight}%  
    \ifdim\pgf@x<.5\pgf@xb%
        \pgf@x=.5\pgf@xb%
    \fi%
    \ifdim\pgf@x<.5\pgf@yb%
        \pgf@x=.5\pgf@yb%
    \fi%
    % 
    % Now, add larger of outer sepearations.
    % 
    \setlength{\pgf@xb}{\pgfshapeouterxsep}%  
    \setlength{\pgf@yb}{\pgfshapeouterysep}%  
    \ifdim\pgf@xb<\pgf@yb%
      \advance\pgf@x by\pgf@yb%
    \else%
      \advance\pgf@x by\pgf@xb%
    \fi%
  }

  %
  % Anchors
  % 
  \inheritanchorborder[from=circle]
  \inheritanchor[from=circle]{north}
  \inheritanchor[from=circle]{north west}
  \inheritanchor[from=circle]{north east}
  \inheritanchor[from=circle]{center}
  \inheritanchor[from=circle]{west}
  \inheritanchor[from=circle]{east}
  \inheritanchor[from=circle]{mid}
  \inheritanchor[from=circle]{mid west}
  \inheritanchor[from=circle]{mid east}
  \inheritanchor[from=circle]{base}
  \inheritanchor[from=circle]{base west}
  \inheritanchor[from=circle]{base east}
  \inheritanchor[from=circle]{south}
  \inheritanchor[from=circle]{south west}
  \inheritanchor[from=circle]{south east}
  \anchor{output}{\outputanchor}

  %
  % Background path
  %
  \inheritbackgroundpath[from=circle]
  \beforebackgroundpath{
    \@tempdima=\radius%
    \setlength{\pgf@xb}{\pgfshapeouterxsep}%  
    \setlength{\pgf@yb}{\pgfshapeouterysep}%  
    \ifdim\pgf@xb<\pgf@yb%
      \advance\@tempdima by-\pgf@yb%
    \else%
      \advance\@tempdima by-\pgf@xb%
    \fi%
    \advance\@tempdima by-.5\pgflinewidth%  
    \pgfsetarrows{-}%  
    \pgfpathmoveto{\pgfpointadd{\centerpoint}{\pgfpoint{-\@tempdima}{0pt}}}%
    \pgfpathlineto{\pgfpointadd{\centerpoint}{\pgfpoint{\@tempdima}{0pt}}}%
    \pgfusepath{stroke}%
  }
}




\endinput
